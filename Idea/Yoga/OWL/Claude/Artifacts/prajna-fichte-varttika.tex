I.40
paramanu-parama-mahattva-anta asya vasikara

Fifth Lecture
Monday, April 23, 1804
Honored Guests:

It might be appropriate to cancel the lecture this Wednesday,
because of the general day of prayer.
I would have taken up preliminary matters again today
had I not also seen the necessity, because of this,
for sparing you the strongest nourishment.

Indeed, I have already adduced and shared with you
everything which is conducive for understanding these lectures
and which helps one enter their standpoint except for two things:
first, what really cannot be imparted,
namely the knack for grasping them;
and second, some observations which tend
not to be received well and which
I had hoped to be able to omit this time.

As far as concerns the first item,
the knack for grasping these lectures is
the knack of full, complete attention.
This should be acquired and exercised
before one enters on the study of the science of knowing.
For this reason, in the written prospectus for these lecture
(available at the place of subscription)
I have established as the sole, but serious precondition
for understanding this science the requirement
that participants should have experience with
fundamental scientific inquiry.
Not, of course, for the sake of
the specific information so gained,
since none of that is presupposed
or even accepted here without qualification;
rather, I did so because this study also
awakens and exercises full, complete attention.
Collaterally, one gains a knowledge of scientific terminology,
which we are using freely here.
Full, complete attention, I have said,
which throws itself into the present object
with all its spiritual power,
puts itself there and is completely absorbed in it,
so that no other thought or fancy can occur;
since there is no room for anything strange
in a spirit totally absorbed in its object:
full, complete attention as distinct
from that partial attention
which hears with half an ear
and thinks with half its thinking power,
interrupted and criss-crossed by
all kinds of fugitive thoughts and fantasies,
which eventually succeed in totally overwhelming
the mind so that the person gradually falls
into a dreamy fog with eyes wide open.
And if he should chance to come back to awareness,
he will wonder where he is and what he has heard.
This full, complete attention of which I speak,
and which only those who possess can recognize, has no degrees.
It is distinguished from that scattered attention,
which is capable of many degrees, not merely quantitatively;
but is totally different and even logically contradictory to it.
It fills the spirit completely,
while incomplete attention does not.

For understanding these lectures everything depends
on one's possessing this kind of attention;
everything which makes understanding difficult
or impossible follows solely from its absence;
if one is freed from this lack,
then all these things are ripped out by the roots.
So, for example, if this lack is removed,
the phenomenon of believing one cannot intuit
the particular theorems presented in the lecture
because one is too quickly confused will fall away.
As I like to repeat frequently so no one will lose heart,
in the nature of our science the same thing is
constantly repeated in the most various terms
and for the most diverse purposes,
so that an insight which is
missed on one occasion can be produced
or made good on another occasion.
But strictly speaking it should be, and is actually,
demanded of everyone that they see into each theorem
when it is initially presented.
So those for whom things don't happen as we expect
have not used these lectures as they should be used;
and if things don't flow smoothly,
they have only themselves to blame.
To give the most decisive proof that
what I demand is possible under conditions
of complete attentiveness;
any distinction between faster or slower mental capacity
has no place in the science of knowing,
and the presentation of this science aims
neither at good nor slow minds
but at minds as such,
if only they can pay attention.
For this is our procedure as it has gone up to now
and as it will remain:
first we are required to construct
a specific concept internally.
This is not difficult:
anyone just paying attention to the
description can do it;
and we construct it in front of him.
Next, hold together what has been constructed;
and then, without any assistance from us,
an insight will spring up by itself,
like a lightning flash.
The slowness or speed of one's mind has
nothing more to do in this final event,
because the mind in general has no role in it.
For we do not create the truth,
and things would be badly arranged
if we had to do so;
rather, truth creates itself by its own power,
and it does so wherever the conditions
of its creation are present,
in the same way and at the same rate.
And in case the ensuing manifestness did not arise
for someone who had really carried out
the construction which we postulated,
this would only mean that he did not sustain
the construction in all clarity and power,
but instead that it faded
because some distraction intervened.
That is, he did not place his total attention
on the present operation.

Or, if this lack is removed,
another equally common phenomenon
would be destroyed at its root:
namely, that an illusion which
we have already revealed as an illusion,
nevertheless can return and deceive us again,
as if it had truth and meaning,
or at least confuse us and make us uncertain
about insights that we had otherwise already achieved.
For example, if you have really seen that
in intuiting the one, eternally self-same knowing,
all differentiation into subjective and objective
completely disappears as arising only in what is changeable,
then how can you ever again allow yourself
to be deceived by the illusion
(which, to be sure, as an illusion, can always recur)
that you yourself are the very thing
that objectively posits this one knowing
(that you therefore are the subject with it as your object)?
Because you indeed have seen once and for all
that this disjunction is always and everywhere the same
illusion and never the truth,
no matter in what form or in what place it might be manifest.
If you have seen this, then you have attained this insight
and dissolved into it.
How then could you possibly cease being what you are, unless,
because you have not really entirely become it
but only entered half way,
you never threw yourself into,
and rooted yourself in,
this insight which now remains
wavering and deceptive for you.
In this case the old illusion returns
at the first opportunity.
But note well the sequence:
the insight does not leave you
because the illusion steps in,
rather the illusion enters
because the insight has left you!
So much as regards the talent of
total attentiveness as the sure and unerring
means of correctly grasping the science of knowing.
Second, I want to mention a few more things
that block apprehension of the science of knowing,
because they do not allow proper attention to arise.
I take these things up collectively in their oneness,
as is my custom (as will likely be the
custom of anyone who becomes familiar with the science).
They arise together from a lack in one's love of science,
which is either a simple lack:
a weak, powerless, and distracted love;
or a secret hatred of knowledge because of
some other love already present in the mind.

Let us first take up the last:
the other love which leads
to a secret hatred of knowledge is
the same one from which hatred
arises against every good,
namely, a perverted self-love
for the empirically arisen self
instead of for the self which is immersed
in the good, the true and the beautiful.
This love is either that of self-valuation,
which therefore becomes pride,
or that of self-enjoyment,
which therefore becomes spiritual lasciviousness.

The first of these is unwilling to admit
that anything could occur in the domain of knowing
that it had not itself discovered,
and long been aware of.
Whether it explicitly says so or not, to such a one,
the science of knowing's claim to absolute novelty
seems to be a statement of contempt for itself.
It would very much like to humiliate
this arrogance on the part of this science;
for this is how things must seem to it.
Therefore, instead of giving itself freely
and with complete attention from the start,
it focuses on whether it can possibly catch
this science in some failing;
and ambivalence of purpose distracts it
so that it misses the right idea,
does not enter into the true subject matter,
but rather finds just what it was looking for
in the confused concepts
which it obtains of the enterprise,
weaknesses in its “science of knowing.

The other mode of thought,
love for the empirical self's self-enjoyment,
loves the free play of its mental capacities
(which it partially becomes)
with the objects of knowing
(which it in the same manner partially becomes).
I think it can best be characterized in the following way:
it calls making something up “thinking”
and it names the invention of a truth
for oneself in one's own body
“thinking for oneself.
A science which brings all
thinking without exception
under the most stringent rules
and annuls all freedom of spirit
in the one, eternal, self-sustaining truth
can hardly please such a disposition,
and it must also incite this later
mode of thought to the same
secret polemic against itself,
producing the very consequences we have just described.
Moreover [just to take this opportunity
to make the point decisively]
I do not warn each of you against
this secret inner polemic for my own sake,
but rather for yours,
because one cannot achieve correct attention,
let alone understanding,
while doing it.
If one will only first understand
and master the science of knowing
and then feel a desire to argue against it,
I will have nothing more to say against this.

Or again it could be a ruling passion
for the merely empirical
and for the absolute impossibility of
feeling and enjoying one's spirit in any way
except as trained memory.
These personified memories are not capable of such secret hate;
but they necessarily become very ill-tempered in this setting.
They want what they call results:
namely what can be observed and
can be reproduced in similar circumstances;
“[that is] a sufficient statement and one that says something.
Now when they think they have grasped something of this kind,
the next lecture arrives further qualified,
differently arranged, symbols and expressions change,
so that not much remains from the hard-won treasure.
“What eccentricity! Why couldn't the man
just say what he meant from the start?”
For people like that the most extreme confusion and contradiction
must arise from what has the purest oneness and strongest coherence,
simply because it is the true inner coherence
and not the merely external schematic coherence,
which is all they really want.

Originally, I first mentioned cold, weak love of science
(which is not exactly hate) as an obstacle to attention.
Namely, whoever seeks, desires, or wishes in science
for something besides science itself does not love it
as it ought to be loved and will never enjoy
its complete love and favor in return.
Even the most beautiful of all purposes
(that of moral improvement)
is too lowly in this case;
what should I say of other, obviously inferior ones!
Love of the absolute (or God) is the rational spirit's true element,
in which alone it finds peace and blessedness;
but science is the absolute's sweet expression;
and, like the absolute, this can be loved only for its own sake.
It is self-evident that there is no room for anything
common or ignoble in a soul given over to this love,
and that its purification and healing are intrinsic to it.

This love, like every absolute,
recognizes only the one who has it.
To those who are not yet possessed by it,
it can give only the negative advice to remove
all false loves and subordinate purposes,
and to allow nothing of that kind to arise in them
so that the right will spontaneously manifest itself
without any assistance from them.
This much should be remembered
once and for all on this subject.

Now to the topic set for today.
When I presented it, I already suspected that
my last talk might seem too rigorous and
deep for a fourth lecture,
and it was made so in part to help me
discover what mode of presentation
I would need to follow with this new audience.
Now I will repeat it in a suitable form:

I.41
ksina-vrtti abhijatasya-iva maner
grahitr-grahana-grahyesu tat-stha-tad-anjanata samapatti

1. First, a remark that is valid
for all previous and subsequent lectures,
and that will be very useful in order
to reproduce and review them.
Our procedure is almost always this:

a. we perform something, undoubtedly led in this process
by a rule of reason which operates immediately in us.
What in this case we really are in our highest peak,
and that in which we culminate, is still only facticity.

b. we then search out and reveal the law
which guided us mechanically in the initial action.
Hence, we see mediately into what we previously
had seen into immediately,
on the basis of its principle
and the ground of its being as it is;
and we penetrate it in the origin
of its determinateness.
In this way we will ascend
from factical terms to genetic ones.
These genetic elements can themselves
become factical in another perspective,
in which case we would be compelled again
in connection with this new facticity
to ascend genetically,
until we arrive at the absolute source,
the source of the science of knowing.
This is now noted and can be clarified
in reference to the consequences:
x is nothing but the developmental link to y,
and y in turn to z.

Now, whoever either has not comprehended z from the start,
or has lost and forgotten this understanding in the process,
for that person neither x nor y exists and
the entire lecture has become an oration about nothing,
through no fault of the lecturer.
This, I say, has been and will continue to be
our procedure for some time.
It was so in the last lecture.
Whoever may have recognized this process
(it was obvious for everyone to see,
and the earlier distinction between
factical and genetic manifestness
should have led right to it)
could have reproduced the entire lecture
and made it intrinsically clear by simply asking:
Was any such factical term presented,
and which one was it?
Which could it be after the earlier ones?
Did the presentation succeed in
presenting the genetic term after the factical one?
Assume that I may have completely forgotten
this second step or perhaps never heard of it.
Then I will have to discover it for myself
just as it was discovered in the talk,
because the rule of reason is unitary, and
all reason which simply collects itself is self-same.

So what was the factical term?
It was not in A and not in the point,
but unconditionally in both.
We have now grasped this,
it has made itself evident,
and so it is.
Analyze it however you wish:
it contains A, the point, and,
in the background, a union of both.
With the first two terms denied as
the true point of oneness,
the other one is thereby posited,
and in this fashion you will not
arrive at any other term.
It is so, factically.
But now I ask on another level:
How have we brought it about that
this insight has arisen for us?
We did not reflect further on the content,
which we completely abandoned;
but focused instead on the procedure,
asking about the origin.
In this way, as I indicated earlier,
the initial, materially constituted,
immediacy becomes mediately visible:
once such an origination is posited,
this factical insight is posited,
but solely by means of
our establishment of the origin.

How did we do it?
Apparently we made a division in something which
on the other hand ought to be a oneness.
I say division and disjunction in a general way,
because one can ignore the fact that the terms separated
are “A” and the point, when it is a question of the act qua act.
This division shows itself to be invalid in an immediate insight.
We did not produce this insight because we wished to,
instead it produced itself absolutely
(not from any ground or premise)
in an absolutely self-generating
and self-presenting manifestness,
or pure light.
The distinction, in the sense that it should be valid by itself,
would therefore be annulled by the [one's] manifestness.
On the contrary, the same manifestness posits
a self-same, intrinsically valid oneness
which is incapable of any inner disjunction.
The principle of division equals the principle of construction,
and thus of the concept as well.
[Now, consider] this principle in its absoluteness
(and by that I mean, the principle as dividing
the wholly and intrinsically one,
which is seen into as one,
[working] wholly and absolutely by itself
without other ground and doing so rather
in contradiction to the truth)
this principle is negated in its absoluteness,
in its intrinsic validity.
It is seen unconditionally as negated,
and therefore it is negated
in and through the absolute light.
Thus, in this annulling of the absolute concept
in relation to intrinsic being,
this being is inconceivable.
Without this relation it is not even inconceivable
but rather is only absolute self-sufficiency.
But further even this predicate “is”
derives from manifestness.
Hence the sole remaining ground and midpoint is
the pure light, and so on.

This was by far the greatest part
of the earlier talk's content.
That this all has intrinsic clarity
and incomparable manifestness is
obvious to anyone who sees it at all.
I am convinced that it could not be presented
with greater order, distinctness, clarity,
and precision than in this case.
Whomever doesn't see it now must be lacking
in the undivided attention required here.

The part still to be added, which I will repeat now,
is another developmental analysis of the insight we have achieved:
I said that we saw into [the fact] that the light was the sole midpoint.
In this reflexion, the process we initially unfolded itself becomes factical.
Now, since in this case we have produced nothing,
and [since] rather the insight as insight has produced itself,
we cannot ask as we did before how we did this,
but we can rise to greater clarity.
It is clear; if only we see that it is the light,
then we are not immediately consumed in this light,
instead we have the light present
through its agent or representative,
that is, through an insight into the light,
into its originality or absolute quality.
We must disregard the fact that
we cannot now ask without contradiction
how the light itself is produced;
since it is recognized as the principle of absolute creation,
the question would deny the insight again.
We can certainly still ask how the insight into light
(which we called not light itself
but rather its agent and representative)
has been produced; that is a different question.
Therefore we only need to pay
attention to how the production of this insight has taken place.
1. We have put ourselves in the condition;
2. how could we do this?
Both are true:
not the light and not even the insight into light,
but the insight into the insight into the light
stands between both.
The emanence and the immanence:
these are matters with which we must concern ourselves.

Regarding the entire distinction
between the immanence and emanence
of the production of an insight into light,
one must not forget that the same thing
extends to the insight and to the light itself.
As before, the objective light qua objective
neither is, nor can become, the one true light.
Instead pure light enters insight under this aspect.
But here we have won this:
that the highest object is no longer
substance for itself, but light.
Substance is only the form of light as self-sufficient.
On the other hand, insight (subjectivity),
actually the inner expression and life of light,
disengages itself from the negation of
the concept, and of division.
Can you penetrate into the true midpoint
more deeply in any other way?
Into the entirely unique concept
that is nevertheless required here?
1. It is clear that its being is not grasped
except in immediate doing.
2. It can be made clear here that immediate doing is
a dissolving into immanence
(the initially uncovered making of his being,
as this sort of making).
First of all, doing deposes being
and being deposes doing
or stated otherwise:
here is the fundamental reversal;
and this must be understood:
doing replaces actual being;
being beyond all being
(not actual or material being)
deposes doing.
Now it is also very clear that
to posit it without any actuality
(just in barely hypothetical form and in a “should”)
and thereby to deduce and materialize doing itself,
thus also to intellectualize and idealize,
being negates itself in the other
as a result of its own doing.
Thus we once again come back to the previous point,
and we find the previous principle again in this self-negation.
Perhaps this is just the concept of being, dead in-itself:
clearly there is a division in it
between being (what endures) and doing:
and indeed as a division this is intrinsic
to constructing non-separation or oneness.
Thus this negation would be true in a certain respect.
It is right since primordially the division
into being and doing is nothing at all.

I.42
tatra sabda-artha-jnana-vikalpa sankirna savitarka samapatti

Sixth Lecture
Thursday, April 26, 1804
Honorable Guests

In today's and tomorrow's talks I will continue
with the further development of
what has already been presented.
In doing so I aim at an end
useful to both sorts of listeners.
That is, since, like all philosophy,
the science of knowing has the task of
tracing all multiplicity back to absolute oneness
(and, correlatively, to deduce all multiplicity from oneness),
it is clear that it itself stands
neither in oneness nor in multiplicity,
but rather stays persistently between both.
It never descends into absolute multiplicity,
which must after all exist and indeed does exist
(as mere empirical givenness),
but rather it maintains the perspective from above,
from the standpoint of its origin.
Therefore, in the science of knowing we will be
very busy with multiplicity and disjunctions.
Now, these disjunctions, or differences and distinctions,
which the science of knowing has to make
are new and previously unrecognized.
Therefore, in the usual modes of representation
and speech from which we begin,
these differences collapse unnoticed into oneness,
and when we are required to draw them,
they seem very minute.
It is hairsplitting, as the literary rabble has put it;
and it is necessary that it be so,
since if a science that is to trace
everything that is multiple, that is,
everything in which a distinction can be constructed,
back to oneness allows any distinction
that the science could possibly make to remain hidden,
then it has failed in its purpose.
Therefore one of the main problems for
the science of knowing consists in just this:
making its very precise distinctions visible and distinct;
so that when this problem is finally solved,
these distinctions will be fixed and established
in the mind of those studying it,
so that they will never again confuse them.
I think that both difficulties
will be significantly reduced
if I lay out for you in advance
(so far as this is possible)
the general schema and basic rule
in terms of which these divisions will come about
although in an empty and purely formal way.
And, so that this schema can be
correctly understood and noticed,
I will deduce it in its unity
and from its principles,
to the extent that this can be done
with what we know so far.

To begin I mention in general the following:

1. Since, according to the nature of our science,
we must stand neither in oneness nor in multiplicity
but instead between the two, it is clear
[and I focus on this because I believe
I have detected several of you making this error]
that no oneness at all
that appears to us as a simple oneness,
or that will appear to us as such in what follows,
can be the true oneness.
Rather, the true and proper oneness
can only be the principle simultaneously
of both the apparent oneness
and the apparent multiplicity.
And it cannot be this as something external,
such that it merely projects oneness
and the principle of multiplicity,
throwing off an objective appearance;
rather it must be so inwardly and organically,
so that it cannot be a principle of oneness
without at the same moment being
a principle of disjunction, and vice versa;
and it must be comprehended as such.
Oneness consists in just this absolute, inwardly living,
active and powerful, and utterly irrepressible essence.
To put it simply, oneness cannot in any way consist
in what we see or conceive as the science of knowing,
because that would be something objective;
rather it consists in what we are, and pursue, and live.
Let this be introduced once and for all
to characterize the oneness which we seek
and to eliminate all the errors about this central point
which, if they continued, would necessarily
be very confusing in what follows.
And be warned, not only so that you don't
content yourself merely with that sort of oneness,
taken just relatively and one-sidedly,
as if it was the absolute,
but also so that if I in this lecture,
or any other philosopher,
remain content with such a oneness,
you will know and state strongly that
this philosopher has stopped half way
and has not made things clear.

2. In consequence:
Since the true oneness is
the principle simultaneously of
the (apparent) oneness and of disjunction,
and not one without the other,
it therefore makes no difference
whether we regard what we will present
provisionally as our highest principle
at each juncture in the progress of our lectures
as a principle of oneness or of disjunction.
Both perspectives are one-sided,
merely our necessary point of view,
but not true in themselves.
Implicitly the principle is neither one nor the other;
rather it is both as an organic oneness
and is itself their organic oneness.

Therefore, so that I can say it even more clearly:
first of all only principles can enter the circle of our science.
Whatever is not in any possible respect a principle,
but is instead only a principled result and phenomenal,
falls to the empirical level,
which, of course, we understand
on the basis of its principle,
but which we never scientifically construct,
as this cannot ever be done.
Then, every principle that enters our science
(and indeed every principle qua principle) is
simultaneously a principle of oneness and multiplicity,
and it is truly understood only insofar
as it is conceived in those terms.
Our own scientific life and activity,
therefore, to the extent that it is a process of
penetrating and merging with the principle,
never enters into that oneness
which is opposed to multiplicity
nor into multiplicity;
rather it maintains itself undisturbed between both,
just like the principle.
Finally each principle in which we stand
(and we never stand anywhere but in a principle)
yields an absolutely self-differentiating oneness:
x {a = (a) — y} z
The only question is whether this oneness is the highest.
If not, and there are several such a's
(a 1, a 2, a 3, ...)
then not just in the former case
but in the latter as well,
“(a)” is still in this regard
a principle of disjunction for unities,
which to be sure would be unities in relation to
x y z
but in connection to one another,
they would by no means be so.
For these a's we need a new a,
until we have uncovered the highest oneness,
which would be the absolute disjunction,
just as we have described it in relation
to the absolute oneness.
This gives us the first general model
for the procedure of the science of knowing.
One comment here: the interchangeability of
direction from a to x, y, z
and vice versa is evident,
and this greatly aids their linkage.

3. Now the same point, from another side and deeper.
As regards the explanation we have been
pursuing up to now in this hour
(not about the principle of disjunction,
since strictly speaking there is no such thing,
but rather our view of the one implicit principle
as a principle of disjunction,
a one-sided view that we undoubtedly must start with
since the science of knowing finds us completely trapped
in this one-sidedness and starts from there),
we find ourselves trapped in the familiar,
frequently cited inexpressibility:
that the oneness is to separate itself
at one stroke into being and thinking
and into x, y, z, both equally immediately.
In expressing this verbally and in diagrams,
we were compelled to make one of the two the immediate term,
though our inner insight contradicted this,
negating the intrinsic adequacy of
the construction of our mode of expression.
Expressing this curious relation in
its logical form will help us to speak precisely:
in this actual disjunction there are
two distinguishing grounds
neither of which can occur without the other.
Therefore, expressing the matter just in the way
we have done is probably an empirically discovered turn of phrase,
since we found it on the occasion of explaining Kant's philosophy
and in adding a disjunction,
which has been demonstrated
neither by Kant nor yet by us,
not only between being and thinking,
but between sensible and supersensible being and thinking.
And the claim that both distinguishing grounds are
absolutely inseparable would therefore be grounded simply on this:
If what is evident in empirical self-observation is to be explained,
then we must assume that the distinguishing grounds are inseparable.
This “must” grounds itself directly
on a law of reason which operates in us mechanically
and without our awareness [of it].
Thus at bottom we had only an empirical basis
on top of which we postulated a supersensible one;
that is, we began a synthesis post factum.
This cannot be blamed on the science of knowing as
long as it is the science of knowing;
it is not permitted simply to report this
inseparability of the distinguishing grounds,
instead it must grasp this ground conceptually
in its principle and from its principle as necessary.
It must therefore see into it genetically and mediately.
“It grasps this ground conceptually”
means it sees the distinguishing grounds
(and by no means just the actual
factically evident distinctions;
whoever remains with these has simply not
finished the climb we have just completed)
as themselves disjunctive terms of a higher oneness,
in which they are one and inseparable
as they are when enacted, so that, as we have said,
it remains one and the same stroke.
But they are separable and conceptually distinguishable,
as we may provisionally think in order thereby
to have something to think.
“Separable” so that, for example, the ground for
distinguishing being and thinking can appear as
a further determination and modification of
the ground for distinguishing sensible and supersensible,
and so that likewise from another point of view
the latter can appear vice versa as
a further determination and modification of the former.
As has been said, when it is enacted,
this disjunction in the oneness of the mere concept
concresces into a factical oneness which is not further distinguishable,
and in this concrete union, every eye that remains factical is
entirely closed to the higher world of the conceptual beyond.

(And now a number of additional remarks.
I ask that you not allow yourself to be
distracted while I state them):

1. I have now specified the boundary point
between absolutely all factical insight
and truly philosophical and genetic insight
entirely and exactly,
and I have opened up the sources of
the entirely new world in concepts
which appear only in the science of knowing.
The creation and essence of this new world is found just here,
in the negation of the primordial disjunctive act as immediate
and in the insight into this primordial act's principle
materially, that it is thus, and formally, that it is at all.

2. I have here explained the essence of
the final scientific form of the science of knowing
from a single point more precisely
than I have been able to do previously.
The main point of this scientific form
resides in seeing into the oneness
of the distinguishing ground:
of being and thinking as one
and of sensible and supersensible
(as I will say in the meantime) as one.
Whoever has understood this
[as it is to be understood up to now,
namely as an empty form]
and holds on to it firmly,
can scarcely make any further errors
in the subsequent actual employment of this form.

3. In order to assist both your memories and repeated reproduction:
in the last hour I said that the path of our lectures was,
and would for a long time continue to be,
that we first present something
in factical manifestness
and then would ascend to
a genetic insight into this object
on the basis of its principles.
This is exactly what we have done
in the just-completed explanation.
Already since the second lecture,
we had developed the inseparability of
both recognized distinguishing grounds
historically out of Kant's own statement,
and we admitted the factical correctness
of this statement.
Now we raise ourselves
[to be sure not to a genetic insight
into the principle of this inseparability,
since we do not yet actually know
this inseparability itself, nor its terms,
but have only assumed all this temporarily
and for the time being]
but to the genetic insight,
which must be the form of this principle,
if such inseparability and
such a genetic principle are to exist).

Now back to our project.
It is also not at all our intention to see directly
into this inseparability and its principle,
since these do not allow themselves
to be “seen into” directly.
And in fact:

4. to take the process further
by our beginning we have already jumped past this principle,
which was discussed here in its form simply
for the intelligibility of what really concerns us,
in order to derive it deductively;
and indeed we have already uncovered
good preliminaries for this derivation.
Namely, you recall that we have already presented
a point of oneness and difference,
which covers the oneness of these distinguishing grounds:
the one between A and the point,
and, in connection with the deeper distinguishing grounds
which are materially different from oneness and difference,
we have said that this might be only a profounder view
of this same higher principle,
disregarding the fact that we could not
yet prove this contention.

[Let me] repeat a third time this oneness
which has already been constructed twice before our eyes.
To that end I recall only that
an absolute dividing principle was evident there;
not A and the mentioned disjunctive point,
since these are principled results of absolute division,
which disappear when one looks to the principle;
but rather the living absolute separation within us.
I stress again what I said before about this essential point,
which is designed to tear our eyes away from facticity
and to lead them into the world of the pure concept,
if I can succeed in making it clearer.
I hope nobody assumes here that the act of thinking
the distinction between A and . is actually grounded
in an original distinction in these things,
independent of our thinking.
Or, in case someone is led to this conclusion
by the previous factical ascent with which we had to begin,
he will recover from this idea if he considers
that in A and . he thinks only the oneness
which, according to him, should be unconditionally one
containing no distinction within itself;
that he himself thus makes clear that
the distinction is not based in the object itself,
since he could not think the object except
by virtue of this distinction;
that he thus expressly makes his own thinking as thinking
into the distinguishing principle.
But the validity and result of this product of thought
expressly surrenders and dies
in relation to the thing itself.
With it as the root, its products A and . are also
doubtlessly uprooted and destroyed as intrinsically valid.
Thus away with all words and signs!
Nothing remains except our living thinking and insight,
which can't be shown on a blackboard
nor be represented in any way
but can only be surrendered to nature.

We intuit, I say, that it rests neither in A nor in .
but rather in the absolute oneness of both;
we intuit it unconditionally without sources or premises.
Absolute insight therefore presents itself here.
Pure intuition, pure light,
from nothing out of nothing, going nowhere.
To be sure bringing oneness with it,
but in no way based on it.

Here everything depends on this:
that each person correctly identify
with this insight, in this pure light;
if each one does, then nothing will happen
to extinguish this light again
and to separate it from yourself.
Each will see that the light exists only
insofar as it intuits vitally in him,
even intuits what has been established.
The light exists only in living
self-presentation as absolute insight,
and whomever it does not thus grasp,
hold, and fix in the place where we now stand,
that one never arrives at the living light,
no matter what apparent substitute for it he may have.

5. Consideration of the light in its inner quality,
and what follows from it,
to which we will proceed after this step,
is entirely different from this
surrender and disappearance into the living light.
This consideration as such will inwardly
objectify and kill the light,
as we will soon see more precisely.
But first we said:
only the light remains as eternal and absolute;
and this [light], through its own inner immediate essence,
sets down what is self-subsistent,
and this latter loses its previously admitted
immediacy to the light, whose product it is.
But there is no life or expression of this light
except through negating the concept, and hence through positing it.
As we said, no expression or life of the light could arise
unless we first unconditionally posit and see a life as
a necessary determination of the light's being,
without which no being is ever reached, except,
that is, in the light itself;
its essence in itself and its being,
which can only be a living being.
Thereby, however, what matters to us,
since we add life to light, is
that we have nevertheless divided the two,
have therefore, as I said, actually killed
the light's inner liveliness
by our act of distinction;
that is, by the concept.
Now, to be sure, we contradict ourselves,
ipso facto denying that life can be distinguished from light,
the very thing we have just accomplished.
This is a contradiction which may well be essential and necessary,
since it may implicitly in itself be the negation of
the concept to which, according to the foregoing,
it must someday certainly come.
(What I'm saying now is added parenthetically for future use.
It is easy to remember;
since it connects with our reflections on
the objectifying consideration of the light,
and allows itself to be reproduced from it
for anyone who has paid even a little attention
to our proceedings, in case he has otherwise completely forgotten.)

To review: in this consideration of the light,
light shows itself through its mere positedness,
absolutely and without anything further,
as the ground for a self-subsisting being
and at the same time for the concept;
and, to be sure, for the concept in a twofold sense:
in part as negated, precisely in its intrinsic validity;
and in part as posited, posited as absolute
but not realized (though still actual);
that is, as appearance and as the light's vitality,
but in no way an appearance that conditions its inner essence.
By the concept's being posited,
A and . are also posited;
to be sure as appearance
and certainly not as primordial appearance,
but rather as conditioning appearance
and the inner life of the primordial appearance = b,
thus appearance's appearance.
[In its inner life,
appearing should occur again as
the unity of the above mentioned distinguishing grounds,
its life comes from the livingness of the concept,
this in turn stems from the light's livingness,
thus an appearance of appearance of appearance.
Everything is brought together again when enacted.
This would be the schema of
an established, rule-governed descent,
in no way like [the one outlined] yesterday,
one equally possible on all sides
and therefore very exposed to error.]

I.43
smrti-parisuddha svarupa-sunya-iva-artha-matra-nirbhasa nirvitarka

Seventh Lecture
Friday, April 27, 1804
Honorable Guests:

[Our] purpose [here is] to give a brief account of
the rules according to which the disjunction
we will have to make proceeds.

1. [It is a disjunction into] principles,
with each being equally a principle
of unity and of disjunction.

2. [It provides a deduction from this to a general]
schema of the total empirical domain
according to the form of its genetic principle.
This will be an entirely new explanation,
because I observed to my pleasure that
some of you had seen that there is
something else even more deeply hidden,
despite the fact that you could not
assist yourselves [to find it],
which, to be sure, was not even required.

3. With the remark that our investigation has
already gone beyond this principle to a higher one
and that it has already begun to deduce this principle itself,
[I repeat] this achieved insight.
Neither in A nor .;
for us the oneness beyond is nothing in itself,
although it is posited as in itself;
rather it exists only through the light
and in the light, and (is) its projection;
light itself; contemplation of light.
Now back to our former topic.

[There is] one further step which opens up
a whole new side of our investigation;
as I said: we have already previously begun to
derive the principle of oneness and disjunction of
materially different principles of division,
only without recognizing that this was what we were doing.

So (dropping what we have done so far until I take it up again)
recall with me and consider the following:
when we observe the light, the light is objectified,
alienated from us and killed as something primordial.
We have explained what is attributed materially to
the light in this observation of the light,
and we have connected this explanation to
the schema under consideration.
Now we will explain this observing itself
in its inner form, that is,
no longer asking what it contains and leads to,
but rather how it itself inwardly occurs,
while also rising to its principle
and viewing it genetically to some extent.
It is immediately clear that:
the light is in us
(that is, in what we ourselves are and do in observing it)
not immediately but rather through a representative or proxy,
which objectifies it as such, and so kills it.
So then, where does the highest oneness and
the true principle now rest?
No longer, as above, in the light itself,
since we, as living, dissolve in the light.
Neither [is it] in the representative and image
of the light which is to be identified now:
because it is clear that a representative
without the representation of what is represented
or an image without the imaging of what it images,
is nothing.
In short, an image as such,
according to its nature,
has no intrinsic self-sufficiency,
but rather points toward some external, primordial source.
Here, therefore, we have not only, as above,
factical manifestness, as with A and .
instead [we even have] conceptual manifestness:
oneness only with disjunction, and vice versa.
“Even conceptual manifestness,” I say:
something imaged (like the light, in this case)
is not thinkable without an image,
nor likewise an image, qua image,
without something imaged.
Notice this important fact,
which will take one deep into the subject matter,
if it is properly grasped here.
In this case you carry out an act of thinking,
which has essence, spirit, and meaning
and is fully and completely self-identical
and unchangeable in relation to this essence.
I cannot share this with you directly,
nor can you share it with me;
but we can construct it,
either from the concept of something imaged
which then posits an image,
or from the image which then posits something imaged.
I ask: apart from the arrangement of the terms,
which is irrelevant here,
have we then thought two different things
in the two concepts thus fulfilled,
or have we not rather thought exactly the same thing in both,
an issue that genuinely touches the inner content of thinking?
The listener must be able to elevate himself
to the required abstract level
out of the irrelevance of the arrangement
to the essential matter of the content,
of spirit, and of meaning, and
then the insight which is intended will
immediately manifest itself to him.
Should this indeed be the case,
then an absolute oneness of content is
manifest here that remains unaltered as oneness
but that splits itself only in
the vital fulfillment of
thinking into an inessential disjunction,
which neither spoils the content
in any way nor is grounded in it.
Either [there is] an objective disjunction
into something imaged and its image,
or, if you prefer, [there is]
a subjective-objective disjunction
into a conception of something imaged
on the basis of the directly posited image,
and a conception of the image on the basis of
the directly posited imaged something.
I do advise you to prefer the latter,
since in that case you have
the disjunction first hand.
And thus in this case our principle
in the genetically-oriented view of the light
would be the concealed oneness,
which cannot be described further,
but which is lived only immediately
in this act of seeing, and which,
as the primordial concept's content,
presents itself as absolute oneness,
and as absolute disjunction,
in its living fulfillment.
Now, the something imaged
in the concept's content
should be the light:
therefore our principle (we, ourselves)
rests no longer either in the light
or in the light's representative,
but rather in the oneness in and between the two,
a oneness realized in our act of thinking.
Therefore, I have called the concept situated here
the primordial concept.
[I call it this] because what up to now
we have taken as the source
of the absolutely self-sufficient,
and which therefore appeared as the original
and was original for us,
actually first arises in the way it appears,
in its objectivity,
from this concept
as one of its disjunctive terms.
Therefore, this concept is more original
than the light itself;
hence, so far as we have yet gone,
it is in this sense what is truly original.
Thus we have given a deeper genetic explanation of the hint,
given only as a fact in the lecture just eight days ago,
concerning the representatives of the original light,
although, to be sure, we have done so
for the particular end we intend here.

Thereby you will see that the concept is
determined further and grasped more deeply
than it has been heretofore.
Until now it was a dividing principle
which, as self-sufficient, expired in the light,
and which preserved only a bare
factical existence as an appearance,
qualifying the appearance of the original light.
Further, it had no contents and acquired no contents
except that which pure light added to it
in immediate intuition through a higher synthetic unity.
Now, however, the concept has its own implicit content,
which is self-subsistent, totally unchangeable and undeniable;
and the principle of division
(which arises again in this case,
and as before is negated as intrinsically valid)
is no longer essential to it,
but instead only conditions its life,
its appearance.
The concept's content, I say, is self-subsistent;
thus it is exactly the same substantial being
which was previously projected out of intuition
and which manifests here in the concept
as prior to all intuition
and as the principle of
the objective and objectifying intuition itself.
Previously, the concept qualified
both life and the appearance of light,
and these conversely qualified the concept's being.
Therefore, it was a reciprocal influence,
and every [act of] thinking the two terms
was qualified externally.
Now the same single concept
grounds its appearance
through its own essential being;
therefore, in this concept
the image and what it images
are posited absolutely,
things which are constructed
organically only through one another.
And, hence, its appearance announces,
and is the exponent of, its inner being,
as an organic unity of the through-one-another,
which must be presupposed.
Its being for itself,
permanent and unchanging,
and as an inner organization of
the through-one-another
(essential, but in no way externally constructed)
are completely one:
therefore, in this case,
absolute oneness is grounded
and explained through itself.

We will achieve a great deal if here and now
we see fundamentally into what I meant
by the inner organic oneness
of the primordial concept,
which I mentioned just now;
since this oneness is the very thing
we will need as we continue.
In this regard I ask:
Does the image, as image,
completely and unconditionally
posit something imaged?
And if you answer “Yes,”
does not the something imaged
likewise posit such an image?
Now without further ado I admit
that both can be seen (by you)
as posited immediately by the other,
but only if you posit one of the two as prior.
But I ask you for once to abstract from your own insight.
This is possible in the way
that I will preconstruct for you now,
and in ordinary life it happens
constantly when it shouldn't.
Further, one could not ever enter
the science of knowing without this abstraction.
That is, I ask about the truth in itself,
which we recognize as being and remaining true
even if no one saw it, and we ask:
Is it not true in itself that
the image entails something imaged
and vice versa?
And, in this case,
what exactly is true in itself?
Just reduce what remains as
a pure truth to the briefest expression.
Perhaps that a posits b and b, a?
Do we want to divide the true into two parts
and then link these parts
by the empty expletive “and,”
a word which we scarcely understand
and which is the least understandable word in all language,
a word which is unexplicated by any previous philosophy?
It is indeed the synthesis post factum.
How could we, since beyond this it is certainly clear
that the determination of the terms derives
solely from their place in the sequence:
for example, that image is the consequent
because something imaged is the antecedent,
and vice versa.
Further, if one enters more deeply into the
meaning and sense of both terms,
it is clear that their meaning simply changes itself
into the expression “antecedent” and “consequent,”
while something imaged is really antecedent and so forth:
thus all this dissolves into appearance.
So then, what common element remains behind
as the condition for the whole exchange?
Obviously only the through-one-another that
initially holds together every inference
however it might have been grasped,
and which, as through-one-another,
leaves the consequence relation exactly
as free in general as it has appeared to be.

[Let me say this in a preliminary way
(there is not time today to explain the
deeper view which is possible here and
which I will go into in the next hour).
The focal point = the concept of
a pure enduring through-one-another
in living appearance:

A | E /I B-A A-B \R a-B B-a

a—act and consequence either ideally, or really.
I say either/or, it always remains act,
the concept proceeds from it alive,
but not finished and complete;
whomever wants this must do it.

On the other side, the concept projects
the one eternally self-same light as intuition,
from which follows (and this is its absolute essence):
what stands beneath it are parts of its externalization
and thus are further modifications not of the light
but of its appearance in the living concept.

1. Only through life to the concept
and only through the concept to life
and the appearance of the light in itself:
but its first modification, never as pure
but in one or another variant.

2. Creation of the science of knowing
in its possible modifications.]

I.44
etaya-iva savicara nirvicara ca suksma-visaya vyakhyata

Eighth Lecture
Monday, April 30, 1804
Honorable Guests:

I believe we have arrived at a focal point in our lectures,
a point which more effectively facilitates
a clear insight and overview than does any other,
and which therefore will permit us greater brevity in what follows.
Therefore, let us not economize on time now,
and from here on out we will make ourselves more secure.
Today we will do this with the contents of the last lecture.

We have seen actually and in fact,
and not just provisionally,
that an absolute, self-grounded insight negates
an equally absolutely created division
(that is, one not grounded in things)
as invalid, and that this insight posits
in the background a self-subsistence,
which cannot itself be described more precisely.
Let this be today's first observation:
at this point the main thing is that
all of us assembled here together have
really and actually seen into this
just as it has been presented,
and that we will never again forget this
self-insight or allow it to fade;
but rather we will take root in it
and flow together with it into one.

Thus, what has been said is not just my report
or that of any other philosopher,
but rather it exists unconditionally,
and it remains always true,
before anyone actually sees into it,
and even if no one ever does.
We, in our own persons, have penetrated to the core
and have viewed the truth with our own eyes.
Likewise, as has been evident from the first,
what has been said is in no way proposed
as a hypothetical proposition,
which is shown to be true in itself
only by way of its usefulness in explaining phenomena,
as is the case in the Kantian, and every other philosophy.
Instead it is immediately true independently of
all phenomena and their explicability.
(A good reason for making this more precise!)
Therefore, what genuinely follows from this,
if only it is itself completely enough determined,
is also as unconditionally true as it [the original insight] is.
And everything which contradicts it,
or the least of its results, is unconditionally false
and should be abandoned as false and deceptive.
This categorical decisiveness between truth and error
is the condition for our, and every, science;
and it is presupposed.
It is far removed from that skeptical paralysis
which in our days parades as “wisdom,”
doubts what is unconditionally manifest,
and wants to make the latter clearer
and more manifest by the most derivative means.

And as particularly concerns the explanation
of phenomena from a principle that is manifest,
it is obvious right away that if the principle is sound,
and likewise the inferences,
then things will go well with the explanation;
we need only note that,
since the principle first grants us a true insight
into the phenomenon's essence as such,
it may well happen that in this proof process
many things do not even have the honor
of being genuine, orderly phenomena,
but rather dissolve into deceptions and phantasms,
although all ages have held them to be phenomena,
or, God forbid, even to be self-subsisting realities.
Therefore, it may happen that in this regard science,
far from acquiring some law or orientation
from the factical apprehension of appearances,
on the contrary rather legislates for them.
This situation can also be expressed as follows.
Only what can be derived from the principle
counts as a phenomenon;
what cannot be derived from it is an error
simply because of its non-derivation,
although it may perhaps incidentally also be immediate,
if one wants to boast of this direct proof.

This is the second observation;
already as a result of this just repeated insight,
a new world of light has opened for us,
which transcends our entire actual knowledge,
and a world of error,
in which nearly every mortal
without exception finds himself,
has perished,
especially if we appropriate
what follows and is recorded below.
If we take up this result directly here,
it will be invigorating for the attention
and throw a very beneficial light
on what comes next.

1. By annihilating the formal concept,
which is the condition for its own
real appearance and vivacity,
the light, as the one true self-sufficiency,
posits a self-sufficient being,
which is not further determinable and
which, as a result of
the insufficiency of the concept
that attempts to grasp it,
is inconceivable.
The light is simply one,
the concept that disappears in the light is one
(the division of what is one in-itself),
and being is one;
it can never be an issue of
anything other than these three.
The one existence arises in
the intuition of what is independent,
and in the concept's negation
(and it will turn out that whatever genuinely concerns
true existence will also rest in this).
If, as is customary, you want to call
the absolutely independent One,
the self consuming being, God,
then [you could say that] all
genuine existence is the intuition of God.
But at the same time note well
[and already a world of errors will be extinguished]
that this being, despite the fact that
the light posits it as absolutely independent
(because the light loses itself in its life),
is actually not so, just because it bears within itself
the predicate “is,” “persistence,” and therefore death.
Instead, what is truly absolutely
independent is just the light;
and thus divinity must be posited
in the living light and not in dead being.
And not, I certainly hope, in us, as
the science of knowing has
often been misunderstood to say;
for however one may try to understand that,
it is senseless.
This is the difficulty with every philosophy
that wants to avoid dualism
and is instead really serious about
the quest for oneness:
either we must perish, or God must.
We will not, and God ought not!
The first brave thinker
who saw the light about this
must have understood full well that
if the negation is to be carried out,
we must undergo it ourselves:
Spinoza was that thinker.
It is clear and undeniable in his system
that every separate existence vanishes
as [something] independently valid and self-subsistent.
But then he kills even this, his absolute or God.
Substance = being without life
[because he forgets his very own act of insight]
the life in which the science of knowing
as a transcendental philosophy makes its entrance.
(“Atheist or not atheist?”
Only those can accuse the science of knowing of atheism
[I am not concerned here with real events,
because regarding all those the science of knowing is not at issue,
since in fact no one knew anything about it]
who want a dead God, inwardly dead at the root,
notwithstanding that after this it is dressed up
with apparent life, temporal existence, will,
and even sometimes with blind caprice,
whereby neither its life nor ours becomes comprehensible;
and nothing is gained except that one more
number is added to the crowd of finite beings,
of whom there are more than enough in the apparent world:
one more that is just as constrained and finite as themselves
and that is in no way different from them in kind.
[I mention] this in passing to state clearly and in a timely way
a significant basic quality of the science of knowing.)

One term is being, the other [the negated concept] is
without doubt subjective thought, or consciousness.
Therefore we now have one of the two basic disjunctions,
that into B and T (being and thinking),
we have grasped this in its oneness, as we should,
and as proceeding completely and simply
from its oneness, (L = light);
and thereby, so that I can add this, too, parenthetically,
we would simultaneously have the schema for the negation
of the I in the pure light and even have it intuitively.
For if, as everyone could easily agree, one posits
that the principle of the negated concept is just the I
(since I indeed appear as freely constructing and sketching out
the concept in response to an invitation),
then its destruction in the face of what is valid in itself is
simultaneously my destruction in the same moment,
since I as its principle no longer exist.
My being grasped and torn apart by the manifestness
which I do not make, but which creates itself, is
the phenomenal image of my being negated and extinguished
in the pure light.

2.  This, I say, is a result of the light itself
and its inward living expression:
things must remain here as a consequence of
this insight and in case they simply follow it,
and we will never get beyond it.
But I claim that, if only we reflect correctly,
we are already beyond it:
we have certainly considered the light
and objectified it: the light therefore
(whomever has forgotten this circumstance
during the previous explanation should recall it now)
has a twofold expression and existence,
partly its inner expression and existence
(conditioned by the negation of the concept,
conditioning and positing absolute being),
partly an external and objective expression and existence,
in and for our insight.

As concerns the latter, that at first we speak of it alone;
we surely remember that we did not possess it,
and everything that lies within it, immediately;
instead we raised ourselves to it from
the beginning of our investigation,
initially by abstracting from the whole variety
of objective knowings [and rising] to
the absolute, self-presenting insight
that genuine knowing must always remain self-same in this variation;
and then [we continued] by means of
deeper, genetic examination of this insight itself.
So far this has been our procedure;
the new and unknown spiritual world
in which we pursue our path has arisen
by this procedure alone,
and without it we would be speaking of nothing.
It now further appears to us that
we could very properly have neglected this procedure,
just as we have undoubtedly neglected it every day
of our lives before we came to the science of knowing.
Taking up this appearance now
[while not making any further inquiry
into its validity or invalidity]
[we find that] it contains the following:
the light's external existence in an insight directed to it,
as the one absolute, eternally self-same
in its fundamental division into being and thinking;
[it] is conditioned by a series of abstractions and reflections
that we have conducted freely,
in short by the procedure that we state as
the free, artificially created science of knowing;
this external existence arises only in this way and for it,
and otherwise not at all.
This the first point here.

But, on the contrary, we assert,
as concerns the inner existence and
expression of the light
that if the light exists unconditionally
[and in particular whether we have insight into it or not,
and this is the very insight which depends on appearing freedom]
it is in and for itself the very same one,
eternally self-identical, and thoroughly necessary,
if only the light exists.
Therefore we assert a meaningful consequence,
something I bid you to mark well,
that there are two different modes
of light's life and existence:
the one mediately and externally in the concept,
the other simply immediate through itself,
even if no one realizes it.
Strictly speaking in actual fact
no one ever does realize it,
but instead this inner life of the light is
completely inconceivable.
This is the second point.

Light is originally divided into being and thinking.
That the light lives unconditionally therefore means
that it splits itself completely originally,
and also inexplicably, into being and concept,
which persists, even though it is negated qua concept.
To be sure, insight can follow this very split,
just as it is now on our side following reconstructively
the split into concept qua concept and being qua being;
but at the same time insight must leave the inner
split standing as impenetrable to it.
This yields, in addition to the previously
discovered and well-conceived form of inconceivability,
a material content of light that remains ever inconceivable.

(I have just expressed myself on a major point
in the science of knowing more clearly than
I have previously succeeded in doing.
We would accomplish a great deal if this
became clear to us right now on the spot.
That the light lives absolutely through itself must mean:
it splits itself absolutely into B (being) and T (thinking).
But “absolutely through itself” also means
“independent of any insight [into itself]
and absolutely negating the possibility of insight.”
Nevertheless for the last several lectures
we have seen, and had insight into,
the fact that light splits itself into B and T:
consequently this split as such
no longer resides in the light, as we had thought,
but in the insight into the light.
What then still remains?
The inward life of the light itself in pure identity,
from itself, out of itself, through itself
without any split;
a life which exists only in immediate living
and has itself and nothing else.
“It lives”;
and thus it will live and appear
and otherwise no path leads to it.
“Good, but can you not provide me
with a description of it?”
Very good, and I have given it to you;
it is precisely what cannot be realized,
what remains behind after
the completely fulfilled insight
which penetrates to the root,
and therefore what should exist through itself.
“How then do you arrive at these predicates
of what cannot be realized;
is not to be constructed from disjunctively related terms
as being is from thinking and vice versa
predicates such as that it is
“what remains behind after the insight,”
“that which ought to exist through itself,”
qualities which are the content or the reality
you have claimed to deduce fundamentally?
Manifestly only by negation of the insight:
hence all these predicates,
leading with the most powerful, absolute substance,
are only negative criteria, in themselves null and void.
“Then your system begins with negation and death?”
By no means;
rather it pursues death
all the way to its last resort
in order to arrive at life.
This lies in the light
that is one with reality,
and reality opens up in it.
And the whole of reality as such
according to its form is
nothing more than the graveyard of the concept,
which tries to find itself in the light.)

It is obvious that our entire enterprise
has achieved a new standpoint
and that we have penetrated more deeply into its core.
The light, which up to now has been understood only
in its form as self-creating manifestness,
and hence assigned only a merely formal being,
has transformed itself into one living being
without any disjunctive terms.
What we have so far assumed to be
the original light has now changed itself
into mere insight and representation of the light;
and we have not merely negated the concept
that has been recognized as a concept,
but even light and being as well.
Previously only the mere being
of the concept was to have been negated;
but how were we supposed to have arrived
at this being even though it is empty?
It was to be negated by something
that itself was nothing.
How could that be possible?
Now we have an absolute reality
in the light itself,
out of which perhaps the being that appears,
as well as its nonexistence in the face of the absolute,
might be made comprehensible.

I now explicitly mention in addition
what direct experience teaches in any case,
that this reality in the original light,
as it has been described,
is unconditionally and completely one and self same,
and that no insight is allowed into how it might arrive
inwardly at a division and at multiplicity.
Observe: the division into being and thinking
(as well as what might, following previously given hints, depend on this)
resides in the concept
which perishes in the presence of reality
and so has nothing at all to do with reality and the light.
Now, according to the testimony of appearance in life
to which our system has provisionally
granted phenomenological truth,
another disjunction ought to arise
which stands either higher,
or at least on the same level
with being and thinking,
since it ranges across both of the latter;
and this is taken for a disjunction in reality.
Since this last contradicts our previous
insight it is thus certainly false;
hence this new grounds for disjunction must
lie in a determination of the concept
that has not yet been recognized
or is not yet sufficiently explored.
The concept, as a concept, must itself be conceivable,
and so no new inconceivability can appear here.
But if this determination of
the concept is grasped conceptually,
then everything that it contains can be
derived conceptually from it.
Whatever range of differences may come forward
in appearing reality now and for all time,
yet it is once and for all clear
a priori that they are B—T + C + L;
one-and-the-same,
remaining eternally self-identical,
and only different in the concept.
Therefore, it is clear that,
since everything true must begin with it
and that falsity and illusion must be turned away,
reality [with which alone true philosophy can be concerned],
not only is generally completely deduced
and made comprehensible,
but also divided and analyzed a priori
into all its possible parts.
“Into its parts,” I say,
excluding from this L (= the light).
For in fact this is not a part,
but the one true essence.
It is hereby likewise clear how far
the deduction and reconstruction of
true knowing goes in the science of knowing:
insight can have insight into itself,
the concept can conceive itself;
as far as one reaches, the other reaches.
The concept finds its limits;
conceives itself as limited,
and its completed self-conceiving is
the conceiving of this limit.
The limit, which no one will transgress,
even without any request or command from us,
it recognizes exactly;
and beyond it lies the one, pure living light;
insight points therefore beyond itself
to life, or experience,
but not to that miserable assembly of
empty and null appearances in which
the honor of existing has no part:
but rather to that experience
which alone contains something new:
to a divine life.

I.45
suksma-visayatvam ca-alinga-paryavasanam

Ninth Lecture
Wednesday, May 2, 1804
Honored Guests:

In the next three lectures I am about to enter
into a deeper investigation than has been made so far.
This investigation will, as it happens,
set out to secure a stable focus
and, leading from it, a permanent guide for our science,
even before we possess this guide.
So, in order not to get confused, a lot depends
on our holding on to what we have laid down provisionally;
therefore:

1. Formally, in relation to the material
which we are investigating
and to the manner in which we take it,
we are already located beyond the prolegomenon
and actually inside the science of knowing;
because (the previous lecture began by recalling this)
we have already actually created insights in ourselves,
which have transposed us into an entirely new world
belonging to the science of knowing
and raised above all factical manifestness,
in whose realm the prolegomenon always remains.
We have passed unnoticed out of
the prolegomenon and into science;
and indeed the transition started as follows:
we had to elucidate the procedure of
the science of knowing by examples,
and, since I found that the state of
the audience made it possible,
we made use of the actual thing
as the original example.
Let us now drop this as a mere example
and take things up earnestly and for real;
thus we are inside the science.
Just as this has so far happened tacitly,
let us now proceed conscientiously and explicitly.

2. Here is how things stood in the hour before last:
I—L—B.a (a = our insight into the matter).
I (Image), positing something imaged in it,
= B (being) and vice versa;
united in the oneness of the light (L).
Thus, on the one side,
the connection of I—L—B,
the essential element of all light without exception:
on the other side, the modifications
without which it does not exist.
This effectively indicates the way in general,
but nothing is still specially known thereby.
It is only the prolegomenon to our investigation.

Additionally, this gives a good hint
concerning an important point
which is not to be handled
without difficulty as to its form.
Knowing should divide itself entirely at one stroke
according to two distinct principles of division:
B–T / oneness, and x, y, z / oneness.
Here we see that the light,
in itself eternally one and self-identical,
does not divide itself in itself,
but rather divides itself,
in its insight and as being seen,
into this multiplicity, whatever x, y, z may be;
the [same] light, which, in itself and in its eternal
identity independent from insight into it
(at least as we have posited this more deeply),
divides itself into being and thinking.
Therefore, if the light does not even exist
except in being the object of insight,
this again divided;
likewise the light does not exist in itself
without dividing itself into being and thinking,
so this disjunction is absolutely one and indivisible
according to both distinguishing grounds.
At present things must remain here,
and this proposition, together with all
further qualifications which it may yet receive,
true in itself and remaining true,
will never be permitted to fall.
(Just by virtue of the fact that
one has fixed termini in
the conduct of the investigation,
one is able to follow the investigation's
most divergent turns without confusion,
and to orient oneself in it
as long as the point at which
everything ties together remains;
while otherwise one would very quickly
be led into confusion.)

Now, in regard to the concept
[which lies neither in the light,
as what is imaged for the concept,
nor in the insight, as the image itself,
but rather between these two]
we realize that formally in itself
this concept is a mere through-one-another,
without any external consequences,
[without antecedent and without consequent,]
which two, and all their shifting relations,
arise only out of the living exhibition of this concept.
This insight, which, if I am not mistaken,
has been presented with the highest clarity,
is presupposed and here only recalled for you.
If I wished to add something here
to sharpen this insight,
then it could only be this:
since the concept, as absolute relation of
the imaged to the image and vice versa,
is only this relationship,
it makes no difference to it that
the thing imaged should be self-sufficient light
and that the image should be this image.
Something imaged and the image,
simply as such, are sufficient.
Further, the imaged thing and the image are
also of little concern to the concept's inner essence,
the latter presupposed as absolutely self-sufficient;
instead this inner essence is
evidently a mere through-one-another.
That this through-one-another, as simply existing,
manifests in the image and the thing imaged
has shown itself empirically.
But who then authorizes us to say,
on the one hand either that this
through-one-another must manifest itself or exist,
or, on the other hand in case the former should be true,
that it must construct itself directly
in the image and the thing imaged
rather than, say, construct itself
for another and under other conditions
in an endless variety of ways?
Through this consideration
we lose the subordinate terms,
and their distinguishing grounds,
for a system of genetic knowing.
On the other hand, in case someone is
willing to grant us this,
who would then authorize us
to assume that the thing imaged
could only be the light,
and that therefore necessarily
the image of the light,
which arises in the concept
as its imaged object and by means of it,
must thereby bring in the other distinguishing ground?
Thereby we also lose the second half in any system
which does not rest content with factical manifestness,
and rejects everything that is not
seen genetically as necessary.

Of course, this result comes out the only way it could,
as soon as we think seriously.
If we posit the concept, the absolute through-one-another,
as an independent self-subsisting being,
then everything external to it disappears
and no possibility of escaping it is to be found;
just as things happened previously
with the light when we likewise posited it.
That is obvious.
Any independent being annuls any other being external to it.
Whenever you might wish to posit a being of this sort
it will always similarly have this result,
which resides in its form.

This observation provides exactly the right task
for our further procedure;
and I wish that we could come to know
this procedure in its unity right now in advance,
so that we would not go astray
among the various forms and changes
which it may assume as we go along,
would easily recognize the same pathway
in every possible circumstance
only with this or that modification,
and would know which modification
it was and from whence it comes.
The genetic relation whose interruption
has come to light must be completed.
This cannot simply be done
by inserting new terms
and thereby filling the gap,
for where would we get them?
We are scarcely capable of adding something in
thought where nothing exists.
Therefore, the genetic relation which is currently absent
must be found in the terms already available;
we have not yet considered them correctly, completely genetically,
but so far still only considered them in part factically.
“In the terms already available,” I say;
thus, if the only important thing were
to arrive at our goal by this path,
it would not matter with which available term we began.
If we worked through only one of these
to its implicit, creative life,
then the flood of light
which simultaneously overcomes and connects
everything would of necessity dawn in us.
But beyond this we also have
the task of following the shortest way;
and so it is quite natural for us to hold on
to what has shown itself to us as the most immediate,
those terms in which we alternatingly have placed the absolute,
and in regard to which we now find ourselves in doubt
as to which is the true absolute,
namely light and concept.

If we work through both
so that each shows itself
as the principle of the other,
then it is clear that

a. in each we have grasped mediately
the distinguishing ground which is immediately
present in the other,
and that
b. beginning from both, we, in our scientific procedure,
have obtained a yet higher common principle
of distinction and oneness for both on essential grounds.

Therefore, both of these lose their absolute character
and retain only relative validity.
Thus, our knowledge of the emerging
science of knowing transcends
them as something absolutely presupposed;
and according to its external form
this is a synthesis post factum.
But since this transcendence is
itself genetic in its inner essence
(and it is not simply as Kant,
and indeed we ourselves speaking preliminarily, have said,
that “there must surely be some yet higher oneness,”
but rather this oneness in its inner essence is
actually constructed)
it is a genetic synthesis.
But again, the science of knowing,
which is genetic in its principles
and which permeates the higher oneness,
is permeated by it,
and is therefore itself identical with it,
steps down into multiplicity and is
simultaneously analytic and synthetic,
[truly, livingly genetic.]
Our task is discovering this
oneness of L [light] and C [concept],
and discovering it in this briefly
but precisely prescribed way;
this discovery is the common point to which the
whole of our next stage refers.
This procedure's modifications
and various turnings are grounded in
the necessity now of properly permeating C
by genetically permeating L,
and then again the other way.
Thus, [it grounds itself] on
constant shifts in standpoint
and being tossed from one to the other.
I will not conceal the fact that
this procedure is not without difficulty
and that it demands a particularly
high degree of attention;
instead I announce this explicitly.
But I am overwhelmingly convinced
that whoever has actually seen into
what has been presented so far,
and holds fast to the present schema
and the just asserted common point for our investigation,
orienting himself in terms of these from time to time,
will not be led astray.
On the other hand,
this is the only truly difficult part of our science.
The other part,
deducing the mediate and secondary disjunctions,
is a brief and easy affair for those
who have properly achieved the first,
no matter how monstrous and mad
it may appear to those who know nothing of the first.
This second part namely,
as is evident from the foregoing
and, which I mention here only redundantly,
has the task of deducing all possible modifications
of apparent reality.
The individual who has so far remained trapped
in factical manifestness wonders at this
because it is the only difficulty
which is accessible and apparent to him.
But until it has its own openly declared principle,
this deduction (of the manifold of apparent reality)
is nothing more than a clever discovery,
which has recourse to the reader's genius and sense of truth,
but can never justify itself before rigorous reason,
if it does not have, and declare, its own principle.
Now to discover and clarify this principle may
certainly be the right work:
for one who possesses it,
the application will thus surely be simple,
and (since the most complete clarity and distinctness is
to be found here)
it will be even simpler than the
application of principles in other cases.
Indeed, one could, if necessary,
simply rest satisfied to have shown
this application through a few examples.
Since I would gladly dispose of this once and for all,
let me take it down to specific cases:
the deduction of time and space
with which the Kantian philosophy exhausts itself
and in which a certain group of Kantians remain
imprisoned for life as if in genuine wisdom,
or of the material world in its various levels of organization,
or of the world of the understanding in universal concepts,
or of the realm of reason in moral or religious ideas,
or even the world of minds,
presents no difficulty and is certainly
not the masterpiece of philosophy.
Because all these things,
together with whatever one might wish to add to them,
are actually and in fact nonexistent;
instead, in case you have only just understood their nonexistence,
each one is the very easily grasped appearance
of the truly existing One.
To be sure, up until now,
some have freely believed in the existence of bodies
(truthfully, in the nothing which is presented as nothing)
and, at the most, in the existence of souls
(truthfully, in ghosts),
and perhaps they have even conducted
deep researches into the relation of body and soul
or into the soul's immortality.
Let me add that not for one moment do I support
skepticism about the latter or wish to wound faith.
The science of knowing legislates
nothing about the immortality of the soul,
since in its terms there is
neither soul, nor dying, nor mortality,
and hence there is likewise no immortality;
instead there is only life,
and this is eternal in itself.
Whatever exists, is in life,
and is as eternal as life is.
Thus, the science of knowing holds with Jesus that
“whoever believes in me shall never die,”
but it is given him to have life in himself.
But I say, picking up the thread,
whoever has believed something like this
and is used to philosophical questions of this kind
demands that a science, which says the things ours does,
address this point with him and free him from error,
if only by an induction on what he has so far taken as reality.
This is what Kant for instance, did;
but it did not help at all.
Nothing could help that did not address the problem at its roots.
The science of knowing does even better than they wish,
according to rigorous methods and in the shortest possible way.
It does not cut off errors individually,
since it is evident that in this work
as soon as error is removed on one side
it springs up on the other;
rather it insists on cutting off
the single root for all the various branches.
For now, the science of knowing asks for patience
and that one not sympathize with
the individual appearances of disease,
which [our science] has no wish to heal:
if only the inner man is first healed,
then these individual appearances
will take care of themselves.

What must be built up in us today is
this declaration of our proper standpoint,
and to consider the unity of our following proceedings,
its coherence as a first part,
with a later piece that can be seen as the second part;
and in relation with this you can view
everything previous as a condition of
clear insight into today's material.
Nothing is thereby gained for material insight
into our investigation's object;
there is indeed a very important point in this insight
which we found last time while doing something else,
dropped today as not relevant to our purpose,
and which we will investigate again tomorrow
for the purpose we have announced clearly today.
As concerns the form, however,
a general perspective and orientation has been achieved
which will guard us from any future confusions.
The schema serves as provisionally valid
and it will endure only those revisions
grounded in our growing insight and no capricious ones.

In conclusion, in order to point out to those
who have attended my previous course
where they find themselves in the process
and thereby to put them in a position
to view the science of knowing with
the complexity that repeatedly pursuing it allows:
what I am now calling concept was named
in the first series “inner essence of knowing,”
what is called here “light”
was there called “knowing's formal being,”
the former [was called] simply the intelligible,
the latter intuition.
For it is clear that the inner essence of knowing
can only be manifested in the concept,
and indeed in the original concept;
likewise, [it is clear that] that this concept,
as implicitly insight, must posit insight, or light.
Therefore, it is clear that the task here expressed as
“finding the oneness of C and L” is
the same one expressed there by the sentence:
“The essence of knowing [is] not without its being, and vice versa,
nor intellectual knowing without intuition and vice versa,
which are to be understood so that the disjunction
that lies within them must become one
in the oneness of the insight.”
Recall that we have concerned ourselves with this
insight for a long time,
and that it has returned under various guises
and in various relations, but always according to synthetic rules.
Certainly it could not happen any differently in this case,
and it was this something, surrendering itself even then,
which I meant when I spoke previously about the manifold shifts
and modifications of the one selfsame process.
The difference from before
[and it seems to me also an advantage of the present path]
is this:
that already from the start,
even before we plunge
into the labyrinth of appearance,
we can recognize our various future observations
in their spiritual oneness.
It is to be hoped
(and this hope does not really concern
my own knowledge and procedure in lecturing
but rather the capacity of this audience
to follow the presentation)
that an ordering principle for these various shifts
will soon be available,
by means of which the process will be further facilitated.
And so it will not be difficult for this part of the audience
to recognize in what is now expressed in a particular way
what was said in a different way before and vice versa.
In being liberated from my two different literal presentations
they may free themselves from any literal presentation,
which would mean nothing and would be better not existing
if it were possible to hold a lecture without one.
And in freeing themselves,
[they] build realization for themselves in their own spirit,
free from any formulas and with independent control in every direction.

Let me add the following while we still have time,
although it is not essential and has relevance
only to the smallest circle of those gathered here:
except for the fact that one possesses
as one's own genuine property only that which one possesses
independently of the form in which one received it,
one can intentionally present it afresh
and share it only under such conditions.
Only what is received living in the moment,
or not far removed from it, strikes living minds;
not those forms which have been deadened by
being passed from hand to hand or by a long interval.
If I had needed to hold these lectures
on the science of knowing immediately
after the previous series to the same audience,
who had all known the science for a long time
no need for further instruction in it,
and who simply wanted to prepare themselves
further for their own oral presentation of philosophy,
yet I still believe that I would have been required to
take almost as diverging a path as I have taken this time,
and I would have had to advise these future
teachers of philosophy about the utility of this divergence
in just about the same way I have advised you,
for whom it is relevant.

Tenth Lecture
Thursday, May 3, 1804
Honored Guests:

Our next task is now clearly determined:
to see into L (light)
as the genetic principle of C (concept)
and vice versa, and thus to find
the oneness and disjunction of the two.

(Let me add yet another parenthetical remark.
Who among you, prior to studying the science of knowing,
has known L or C, not in general and with confusion
[since any sort of philosophy
distinguishes an immediate presentation
of an actual object and a concept,
which is usually an abstract one]
but true L and C in the purity and simplicity
with which they have been presented here?
Our task concerns itself with doing this;
and, with the resolution of this task,
the science of knowing is completed
in its essentials.
Accordingly, the science answers
a question that it itself must pose,
and dissolves a doubt that it has first raised.
It should not seem strange to anyone that
there is no bridge to it from the usual point of view
and that one must first learn everything
about it from within it.)
Something has happened for the resolution
of this task on Monday,
which we will now briefly review
to confirm our grasp of it.

As factical manifestness makes clear,
light plainly arises in a dual relation:
in part as inwardly living
[and through this inner life of its own it must
divide itself into concept and being]
and in part in an external insight,
which is freely created
and which objectifies this light
along with its inner life.
Let us take up the first.
What makes this inward life inward?
Obviously that it is not external.
But it becomes external in [being seen by] the insight.
Thus, what follows immediately and is synonymous:
it is an [inward] life because in this regard it is
outside any insight, is inaccessible to it, and negates it.
Therefore light's absolute, inner life is posited;
it exists only in living itself and not otherwise;
therefore it can be encountered only immediately
in living and nowhere else.
I said that the genuinely, truly real in knowing rests here.
But we ourselves have just now spoken of this inner life
and therefore in someway conceived it.
Yes, but how? As absolutely inaccessible to insight;
so we have conceived and determined it only negatively.
It is not conceivable in any other way.
The concept of reality, of the inner material content of knowing, etc.,
which we have introduced is only the negation of insight
and arises only from it;
and this should not just be honestly admitted,
rather, a philosophy which truly understands its own advantage
should carefully enjoin this idea.
In truth it is no negation,
but rather the highest affirmation,
which indeed is once again a concept;
but in truth we in no way conceive it,
but rather we have it and are it.

Let this be completed and determined by us right now,
and in this act it will have its application uninterruptedly.
And don't let the truly crucial
point of the matter escape:
[there are] two ways
for the light to live absolutely:
internally and externally;
externally in the insight,
internally, therefore, absolutely not in the insight,
and not for it, but instead turning it away.
By this means, our system is protected against
the greatest offense with which
one can charge a philosophical system,
and without exception nearly always justly:
namely vacuity.
Reality, as genuine true reality, has been deduced.
No one will confuse this reality with being (objectivity);
the latter is subsistence-for-self and dependence-on-self
which is closed in on itself and therefore dead.
The former exists only in living,
and living exists only in it;
it can do nothing else than live.
Therefore, because our system has taken life
itself as its root,
it is secured against death,
which in the end grasps every [other]
system without exception somewhere in its root.
Finally, we have seen and enjoined that,
since light and life too are absolutely one,
this reality
[and the insight,
through the negation of which
it becomes reality,]
can altogether be only one
and eternally self-same.
Thereby our system has won enduring oneness
and has secured itself from the charge that
there may still be some duality in its root.

Insight, I assert, is completely negated in living light.
But then we see, and see into, the disjunction into C and B.
Therefore, this disjunction, which we previously ascribed
to the inner light itself,
should not be ascribed to this,
but instead to the insight that takes its place,
or to the original concept of light.
The concept reaches higher,
the true light withdraws itself.
The absolute negation of the concept may well
remain a nothing for the science of knowing,
which has its essence in concepts,
and only become an affirmation in living.
With this, two further comments which belong
to philosophy's art and method.

1. Here we retract an error in which we have so far hovered.
How did we arrive at this error,
or at the proposition that
we now take back as erroneous?
Let us recall the process.
Driven by a mechanically applied law of reason
(therefore, factically),
we realized immediately that it [the absolute]
cannot reside in A (the oneness of B and T)
nor in the disjunction point,
but rather in the oneness of both;
this was the first step.
Then, as the second step,
we raised ourselves to the apprehension
of the general law for this event,
which naturally we could apprehend only as follows:
in an immediately self-presenting insight,
a disjunction is negated as intrinsically valid
and a oneness, which cannot be any more exactly described,
is absolutely posited.
What then did we finally do?
In fact we did nothing new,
apart from the fact that we relinquished
the specificity of the disjunctive terms “A” and “.”
and likewise the specificity of their unity,
and posited disjunction,
and also self-sufficient oneness,
generally and unconditionally,
in which case the possibility of the procedure
could arouse wonder and give rise to a question.
Besides this, I say, we simply grasped
the rules of the event historically,
always led by the event,
and, if that were removed from us,
[we] lacked all support for our assertion.
Therefore, although this, our second insight,
appears to possess a certain genetic character
in the first mentioned part,
in the second part it is something merely factical;
and so what we advanced yesterday as the
ground for the uninterrupted connection
between the disjunctive terms is
confirmed here at a genuinely central point:
that our whole insight might not yet be purely genetic,
but is instead still partially factical.
To get back to the point;
this insight, arising in concrete cases
and led to its general rule in the second step,
we now named pure, absolute light,
simply in this respect,
that in terms of content it arises immediately,
without any premises or conditions.
But in its form it remains factical and is dependent
on the prior completion in concrete cases.
We might have inferred from the following
that it cannot possibly have its application here:
although dividing the concept
into “A” and “.” was given up as inadequate,
there yet remained a new disjunction
in what was taken as absolute,
since it was simultaneously negating and positing,
the former through its formal being,
and the latter through its essence.
But no disjunction can be absolute and merely factical,
rather, as surely as it is a disjunction,
it must become genetic,
since disjunction is genetic in its root.
(Remarks of the kind just made bring
no progress in the subject matter,
but they elevate the freedom of
self-possession and reproduction for everyone
and facilitate the comprehension of what follows.)
Result: since our initial supposition
grounded itself partially on factical insight,
we must give it up.

Further, how then have we arrived at
this insight of giving something up,
as well as at the higher [term]
for which we give it up?
If you recall, we did so by means
(i) of the distinction between two ways
the light exists and lives: inwardly and outwardly,
a distinction admittedly given only with factical manifestness;
(ii) by genetic insight into this distinction and by the question
how something absolutely inward might arise as inward;
and (iii) by elevating into a genetic perspective
something that had been thought previously
only in faded, factical terms.
Moreover, I admitted, as is indeed evident
and as everyone will remember,
that this entire disjunction between inner and outer
arises only in a factical point of view.
Here, too, this observation:
that in our present investigation as well,
the facticity, which is erased on one side,
pops up again on the other,
and that we will not be entirely and purely relieved of it
until the present task is completed.
(For returning listeners this as well:
the distinction drawn here
between the light's inner and outer life is
the same as the distinction
between immanent and emanent forms of existence
that was so important and meaningful in the previous series.)

Second remark: C and L are both only concepts:
the first is purely disjunction in general,
a disjunction which can give no further account of itself;
from our current point of view, it has simply two terms which are
not further distinguishable.
L, on the other hand, is not a disjunction in general,
but is rather the specific disjunction into being and concept.
The latter, as the principle of disjunction in general,
consequently has enduring inner content,
as does B [Being], the principle of oneness.
Therefore, the terms of this second disjunction are not
just two terms in general, they also have an internal difference.
From our standpoint, therefore, L is still by no means negated,
nor can it be from that perspective.
If nevertheless it must be negated,
as it evidently must be a priori,
since otherwise it could not come to the zero state
in which no further disjunction truly remains,
then completely different means will have to be employed
from the ones now available to us.

Now to characterize (in relation to our task)
the point I have just repeated,
a point which fits our system in every aspect,
of which I said last time that
it already belonged to our process,
and which we need to apply in solving
our next and primary task:
C and L are to be reduced to oneness,
just as they were before this point of ours;
and this will have to be done so that C is
so rigorously mastered that we see into it
as the genetic principle of light, and vice versa.
With which of the terms to begin is left either to our caprice or
our philosophical skill,
which is unable to give any account of its maxims before applying them.
In the previously discussed point,
L was taken up as a starting point,
as things stood then;
if so, the concept proceeds genetically from this L,
since L has transformed itself into the concept.
Or said more exactly:
our own observing
—which was not then visible, which we
lived, and into which we merged—
divided itself beyond the then regnant L,
and in this division negated the L (light) into 0/C;
thus creating both out of itself.
Now note well that this shift in viewpoint is
in no way merely a change of the word and the sign,
but rather that it truly is a real change;
because what stood here previously,
whether called L or C, light or concept,
was to be the absolute (which is a real predicate)
and should divide itself into C and B
(which is also a real predicate).
Both predicates combine to form
a synthetic sentence that determines the absolute.
In its essence
[entirely apart from the expressions and signs
in which one realizes and presents this essence]
this sentence is contradicted
by the really opposite sentence:
the principle of the disjunction
into being and thinking is not absolute
but something subordinated
(however one may more exactly name and signify
this subordinate something);
in the absolute the two are not distinguished.
From this, another correction must follow first and foremost,
not so much regarding our viewpoint as our manner of expression.
There were supposed to be two different distinguishing grounds,
which of course are to be reunited again,
but which would have been held sufficiently
far apart from one another by two basic principles
that were as distinct as until now
light in itself and its representative concept have been.
Now all disjunction collapses into one and the same concept,
and hence this latter could very easily provide
the one eternally self-same disjunctive factor,
which does not appear in the original appearance,
but rather appears as doubled in the secondary appearance,
in the appearance as appearance.

But let me go back.
As things stood previously,
the spirit of our task was to realize
L as the genetic principle of C and vice versa.
We have tried by beginning with L:
the attempt had its narrowly circumscribed result
and the matter does not stand as it did before
but as the schema instructs.
The spirit of the task remains the same
through all shifts in perspective,
just because it is spirit:
L through C and vice versa.
Our true L is now = 0,
and it is clear that
this cannot be approached more closely:
it negates all insight.
Therefore, this first path is
already completed on the initial attempt.
Nothing more remains besides
taking up C and testing whether
through it we can further determine
[not 0, since this remains purely
unchangeable and indeterminable]
but rather it as the truly highest term
that we now are and live.
Thus, a new classificatory division,
the determination on the basis of C,
becomes the second principal part
of our present work.

Let me now give some preliminary hints
and thereby prepare you for tomorrow's lecture,
setting out a rough outline for you.

The concept's inward and completely immutable essence has
already been acknowledged in an earlier lecture as a “through.”
Although in its content this insight is in no way factical
but is instead a purely intellectual object,
still it has an factical support:
the construction of the image and the thing imaged,
and the indifference of the inference between them.
However, we would be permitted to use this basic quality of the concept,
if only, in this application, we succeeded in negating its factical origin.
If one embraces a “through” just a little energetically,
it can readily be seen that the same principle is a disjunction.
Except the same question must always be repeated which
already arose previously on the same occasion:
how should a dead “through,” defined as we have defined it,
come to life
(despite all the capability with
which it is prepared to meet life,
especially by means of the “throughness”,
or the transition that it makes from one to another,
if only it is brought into play)
because it has no basis in itself
for coming to actualization?
How would it be if the internal life
of the absolute light (= 0) were its life,
and therefore the “through” was itself
first of all deducible from the light by this syllogism:

i. If there is to be an expression
[an outward existence of the immanent lifas such]
then this is possible only with an absolutely existent “through.”
ii. But there must be such an expression.
iii. Hence, the absolute “through” (the original concept, or reason)
exists absolutely as everyone can easily see for himself.

Further, how would it be if just this living “through”
(living to be sure by an alien life, but still living)
as the oneness of the “through,”
divided itself into thinking and being,
in itself, and in the origin of its life?
This division, as a division of the enduring “through” as such,
would be comprehensive for the same reason,
and inseparable from it and its life.
How would it be if it did not remain trapped in this,
its essence as “through,” but rather might itself
be objectifying and deducing the latter;
it would hence certainly have to be able
to do just as we ourselves have done
[the objectification and deduction
can themselves come about according to the law of the “through,”
since fundamentally and at base it is nothing but a “through”:
how would it be if in this objectification and deduction
it split itself again in the second way?
[Further, how would it be,
since a “through” clearly can exist
only by means of a “through”,
that is, its own being as a “through” can be only mediate—
and the first mode of being
would scarcely be possible
without at least a little of the second,
if the first division could not be
without some of the second, and vice versa?
Since this “through” is our own inner essence,
and a “through” dissolves completely into another “through,”
then absolutely everything based in these terms
must be completely conceivable and deducible.]
In all of these “how would it be if ____” clauses,
I have consistently regarded “0” as life;
but it is not merely this,
rather it is something indivisibly joined with life,
a thing we grasp by the purely negative concept of reality.
If it is indivisible from life,
if life lives in a “through,”
then it lives as absolute reality,
but since it is a “through,”
it lives it in a “through”
and as a “through.”
Now, consider what follows
if the one absolute reality,
which can only be lived immediately,
occurs in the form of an absolute “through.”
I should think this:
that it cannot be grasped anywhere,
unless an antecedent arises
for what has been grasped,
through which it is to be;
and, since it is grasped only as a “through,”
it must also have a consequence
that is to follow from it.
This must follow unavoidably
by absolutely every act
of apprehending reality.
In short, the infinite divisibility in absolute continuity;
in a word, what the science of knowing calls quantifiability
as the inseparable form of reality's appearance
arises as the basic phenomenon of all knowing.

In this last brief paragraph of my talk
I have pulled together the entire content
of the science of knowing.
Whoever has grasped it
and who can see it as necessary
[the premises and conditions for this
manifestness have already been completely laid out]
such a one can learn nothing more here,
and he can only clarify analytically
what has already been seen into.
Whoever has not yet seen into it has
at least been well prepared for what is to follow.
For the one as for the other,
we will move forward tomorrow.

Eleventh Lecture
Friday, May 4, 1804
Honored Guests:

Yesterday, I succeeded in presenting the essence
and entire contents of the science of knowing
in a few brief strokes.
Losing time at the right place means gaining it;
therefore, against my initial purpose,
I will apply today's hour to presenting
further observations about this brief sketch.
The more certain we are in advance about the form,
the easier the actual working out
of the contents within this form will be.

C = “through,” in which resides disjunction.
“If only this “through” could be brought to life,” I said;
it has nearly all the natural tendencies of life,
nevertheless in itself it is only death.
It would be useful to reflect further about this expression,
since the “through” can be more clearly understood in it,
than it has yet been understood:
this “through,” which according to the preceding
represents the central point in our entire investigation.
Indeed, it is immediately clear
what it means to say:
“a 'through' actually arises,”
“a 'through' has taken place,”
or “there is an existing 'through.'”
I further believe it will be clear
to anyone who considers the possibility of this existence that,
considered formally, something else belongs to it
besides the pure “through.”
In the “through” we find only
the bare formal duality of the terms;
if this is to find completion,
then it needs a transition from one to the other,
thus it needs a living oneness for the duality.
From this it is clear that life as life cannot lie in the “through,”
although the form which life assumes here,
as a transition from one to the other,
does lie in the “through”:
so life generally comes entirely from itself
and cannot be derived from death.

Result: the existence of a “through”
presupposes an original life,
grounded not in the “through”
but entirely in itself.

We see into this at once:
but what is contained in this insight?
Evidently the insight formed in positing
the “through's” existence
(and the question about the possibility of this existence)
brings life with it,
that is in the image and concept.
Therefore, in this insight life is grasped
in the form of a “through,” only mediately.
The explanation of the “through” is itself a “through.”
The first of these posits its terms at one stroke;
and, in the resulting insight,
it is itself posited as positing them in one stroke
by the explanatory “through” (horizontally arranged):
a
——
a × b
So too, in the same way, with regard to
the inner meaning, sense, or content,
the first “through” does not posit
its terms at one stroke;
rather, life should be the condition
and the “through's” existence should be
what is conditioned;
thus it [life] should be in the concept as a concept
[in truth and in itself] the antecedent,
and the latter should be the consequent.
[This is the] perpendicular arrangement.
Both obviously [exist] only in connection with
each other, and [are] only distinguishable in this context.

The concept remains the focus of everything.
(To reconstruct: here in a certain respect to preconstruct.)
[The concept] constructs a living “through,”
and, to be sure, does so hypothetically.
Should this latter be,
then the existence of living follows from it.
It is immediately clear that a hypothetical “should” is
not grounded on any existing thing,
but is rather purely in the concept,
and collapses if the concept collapses;
and that, therefore, the concept announces
itself in this “should” as pure,
as existing in itself,
and as creator and sustainer
from itself, of itself, and through itself.
The “should” is just the immediate
expression of its independence;
but if its inner form and essence are independent,
then so are its contents as well.
Hence, the existence of a “through”
announces itself here as completely absolute and a priori,
in no way grounded again on another real existence which precedes it.
Therefore, the concept is here the antecedent
and absolute prius to the hypothetical positing
of the “through's” existence:
the latter is only the concept's expression,
something which depends on it
and through which it, as concept, preserves itself
as an absolutely inward “through.”
Which was the first [point].

In this, its vivacity,
the concept changes itself into an insight,
which, unconditioned, produces itself
insight into a life in and of itself,
which must necessarily be presupposed.
Ascending, I can therefore say:
the absolute concept is the principle of the insight,
or intuition, into life in itself,
that is to say into [life] in intuition.
It seems quite possible, namely in a shallow and faded way,
to think the existence of a “through” without any insight
arising into the life that it absolutely presupposes.
For the latter to happen, this existence must
be conceived with full energy and vivacity:
Now, I say, (as is clear right away):
in this pale imitation the “through” is
not really thought as it must be thought:
that is, as a genetic principle.
For if it were thought in that way,
then what ought to be manifest would be manifest.
So the true focus, the genuine ideal prius is
no longer just the concept,
but rather the inward life
whose outcome (posterius) is the concept.
And the “should” is not, as I said before, the highest
exponent of the independence of reason,
rather the appearance of inner energy is this highest exponent.
(If I ask you to think energetically,
I am really demanding that you be fundamentally rational!)
The hypothetical “should” is again an exponent of this exponent,
and the concept is not, as I said initially,
the principle of intuition,
but rather the inner immediate life of reason,
merely existing and never appearing,
which appears as energy
(energy which obviously is again
the expression of a “through” immanent in itself).
This inner life, I say, is the principle
of concept and intuition at once
and in the same stroke:
thus it is the absolute principle of everything.
This, I say, would be the idealistic argument.

Once we have proceeded in this way,
let us climb higher in order to get
to know the real spirit and root
of this mode of argument.
Without further ado,
it is obvious that we could have expressed our entire procedure thus:
there may be the intuition of an original and absolute life,
but how and from what does it come to be?
Just construct this being as I have,
or grasp it in its becoming;
now this has happened actually and in fact,
and the inner life of reason as a living “through”
has been deduced as the genetic principle for this being.
Thus, the basic character of the ideal perspective is
that it originates from the presupposition of
a being which is only hypothetical
and therefore based wholly on itself;
and it is very natural that it finds just this same being,
which it presupposes as absolute,
to be absolute again in its genetic deduction,
since it certainly does not begin there
in order to negate itself,
but to produce itself genetically.
Thus, the maxim of the form of outward existence is
the principle and characteristic spirit
of the idealistic perspective.
By its means, reason, which we already know very well
as a living “through,” becomes the absolute;
it becomes this in the genetic process
because it already exists absolutely
as the constant presupposition.
Absolute reason, as absolute, is therefore a “through” =
the form of outward existence.
Prior and absolute being shows itself to be
inwardly static, motionless, and dead
at just this [point of going] “through,”
where it always remains;
the inward life of reason,
which we have already established,
shows itself in this being's hypothetical quality.
One need only add now what is implicitly clear:
that this idealistic perspective does not arise
purely in the genetic process,
since it assumes a being as given,
and that therefore it is not
the science of knowing's true standpoint.
This is also clear for another reason:
in the idealistic perspective, reason exists,
or lives, as absolute reason.
But it lives only as absolute
(in the image of this “as”);
hence it does not live absolutely;
its life or its absoluteness is itself mediated by
a higher “through,”
so that in this standpoint it is only derivative.
So much by  way of a sharp, penetrating
critique of the idealistic perspective.
This is especially important,
because beginners are easily tempted to remain
one-sidedly trapped in this point of view,
since it is the perspective in which
their speculative power first develops.

Now let us turn things around
and grasp them from the other side.
If the “through” is to come to existence,
then an absolute life, grounded in itself,
is likewise presupposed.
Therefore, this life is the true absolute
and all being originates in it.
With this, intuition itself is obviously negated,
though not indeed as empirically given,
since if we simply try to remain in an energy state
and to consider nothing further,
then we will always find that we still grasp
[this state] in intuition.
Let me mention in passing that
this is idealism's stubbornness:
not to let one go further,
once one has finally arrived at it.
Faced with this, since in any case
idealism is something absolute,
it does not allow itself to be explained away
by any machinations of reasoning,
but rather yields only to the arrival
of what is primordially absolute.
Among other things, this idealistic stubbornness too has been
attributed to the fantasy of the science of knowing
which circulates among the German public,
disregarding the fact that of course
people cannot speak clearly about this charge
because they do not know the genuine science;
e.g., Reinhold did this for his whole career.
It is like this:
the non-philosopher or half-philosopher forgets himself,
or the absolute intuition, either because he never knew it
or, if he knew it, he periodically forgets it again.
The one-sided idealist who knows it and holds it fast
does not let it develop, because he knows nothing else.
To return:
by recognizing the absolute immanent life
we negate intuition as something that is genetically explicable
and [that plays a part] in a system of purely genetic knowledge.
Because if the immanent life is self-enclosed,
and all reality whatsoever is encompassed in it;
then not only can one not see how to achieve
an objectifying and expressive intuition of it,
one can even see that such an intuition could never arise;
and, because of its facticity, just this last insight
cannot itself be comprehended anew,
but simply directly carried out;
it is the absolute, self-originating insight.
So, however stubbornly one might hold on
to his immediate consciousness of this intuition,
it does not help things;
no one challenges this intuition in its facticity.
What has been asserted and demonstrated is just this:
not merely that [this consciousness] is inconceivable,
but even that it can be conceived to be impossible.
Thus, the truth of what it asserts is denied,
but in no way its bare appearance.
Let me note in passing that the place
for denying ourselves at the root is here,
just in the intuition of the absolute,
which of course might very well be our root,
and which up to now has played that role.
Whoever perishes here will not expect
any restoration at some relative, finite, and limited place.
But we do not achieve this annulment
by an absence of thought and energy,
as happens in other cases.
Instead, [we do so] through the highest thinking,
the thought of the absolute immanent life,
and through devotion to the maxims of reason,
of genesis or of the absolute “through,”
which denies its applicability here
and thus denies itself through itself.
Everything has dissolved into the one = 0.

Reasoning which is conducted and characterized
in these terms is realistic.
There is no progression or multiplicity
in it except for pure oneness.
[Let me relocate you in the context.
The two highest disjunctive terms stand
here absolutely opposed to one another,
life's inner and outer life,
the forms of immanent and emanent existence as well,
separated by an impassable gulf
and by truly realized contradiction.
If one wishes to think of them as united,
then they are united exactly
by this gulf and this contradiction.]

As we did before with the idealistic perspective,
let us now discover the inner spirit and character
of the realistic perspective just laid out.
Obviously, this whole perspective takes
its departure from the maxim:
do not reflect on the factical self-givenness
of our thinking and insight,
or on how this occurs in mind,
rather reckon only the content of this insight as valid.
Thus, in other words:
do not pay attention to the external form
of thought's existence in ourselves,
but only the inner form of that thought.
We posit an absolute truth,
which manifests itself as the content of thinking,
and it alone can be true.
As before, it happens for us as we presupposed and wished;
since the inner content alone should be valid,
so in fact it alone really matters,
and it negates what it does not contain.
Made genetic by us, it was just so.
So much in general.

Now allow us to delineate realism's presupposition
of an inner absolute truth more precisely.
I believe that there is no way to give a better description,
such as is indeed needed, than this:
this implicit truth appears as an image,
living, completely determined, and immutable,
which holds and bears itself in this immutability.
Now this implicit truth reveals itself in absolute life,
and it is immediately evident that it can only
reveal itself in the latter.
Because life is just as truth is:
the self-grounded, held and sustained by itself.
Truth is, therefore, in and through itself only life's image,
and likewise only an image of life gives truth,
just as we have described it.
By means of the truth, as grounded by itself,
only the image is added.
So we stand at about the
same point as before,
between life itself and the image of life;
as regards this, we saw that they are
fully identical in terms of content,
which alone matters in realism,
and are different only in form,
which realism leaves to one side.

Living
—————————
Concept Image Being

Now it is noteworthy that the image
(which holds and sustains itself)
should exist only in the truth as truth,
when the former seems, according to its character,
to be exactly the same as thought
and the latter the same as being;
and that therefore in realism
(and working from it, if we simply compel it
to clarify its fundamental assumption)
we are led to a perspective,
which is so similar to idealism
that it could even be the same.

Without venturing further here into this last hint,
which meanwhile could be tossed in
to direct our attention to what follows,
I will simply conclude today's lecture
and pull it together into a whole
by means of the following observation.
Both idealism and realism grounded themselves on assumptions.
Both of these assumptions
[that is in their facticity and in the circumstance
that one actually arises here and the other there]
grounded themselves on an inner maxim of the thinking subject.
Hence, both rest on an empirical root.
This is less remarkable in the case of idealism,
which asserts facticity, than of realism,
which, in its effects and contents,
denies and contradicts what it itself fundamentally is.
As we have seen, both are equally possible,
and, if only one grants their premise,
they are equally consistent in their development.
Each contradicts the other in the same way:
absolute idealism denies the possibility of realism,
and realism denies the possibility of
being's conceivability and derivability.
It is clear that this conflict,
as a conflict in maxims, can be alleviated
only by setting out a law of maxims,
and that therefore we need to search out such a law.

We can get a rough idea in advance about
how a settlement to the conflict will work out.
All the expressions of the science of knowing so far
show a predilection for the realistic perspective.
The justifiability of this preference follows
from this, among other things:
that idealism renders impossible
even the being of its opposite,
and thus it is decidedly one-sided.
On the other hand, realism at least leaves
the being of its opposite undisputed.
It only makes it into an inconceivable being,
and thereby brings into the light of day
its inadequacy as the principle for a science
in which everything must be conceived genetically.
Perhaps a simple misunderstanding underlies the proof,
given earlier in the name of realism,
that an expressive intuition of
absolute life can in no wise arise.
In that case, what is proven
and needs to be asserted is only that
such an intuition,
as valid for itself and self-supporting,
can never arise.
This assertion very conveniently
leaves room for an interpolation:
this intuition might well arise,
and must arise under certain conditions,
simply as a phenomenon not grounded in itself.
Insight into this interpolation could thus provide
the standpoint for the science of knowing
and the true unification of idealism and realism;
so that the very intuition, purely as such,
which we previously called “our selves at root,”
would be the first appearance
and the ground of all other appearances;
and because this would not be any error
but instead genuine truth,
it and all its modifications,
which must also be intuited as necessary,
would be valid as appearances.
On the other side, however, seeming and error
enter where appearance is taken for being itself.
This seeming and error arise necessarily from [truth's] absence,
and hence can themselves be derived
as necessary in their basis and form,
from the assumption that this absence itself is necessary.
Some have either discovered or thought they discovered,
I know not which, that in measuring the brow
they could measure people's mental capacity
on the basis of their skulls.
The science of knowing could easily claim
to possess a similar measure of inner mental capacity,
if only it could be applied.
In every case the rule is this:
tell me exactly what you do not know
and do not understand,
and I will list with total precision
all errors and illusions in which you believe,
and it will prove correct.

I.46
ta eva sabija samadhi

I.47
nirvicara-vaisaradye adhyatma-prasada

Twelfth Lecture
Monday, May 7, 1804
Honored Guests:

In the last discussion period,
it was obvious from those who attended
and allowed themselves to converse about things
not only that you have followed me well
even in the most recent deep investigations,
but that, as is ever so much more important,
a comprehensive vision of the inner spirit
and outer method of the science
we here pursue has grown in you.
Consequently, I assume that this is
even truer with the rest of you who did not speak;
and I [will] abstract from everything
that does not arise for me on this path,
with no misgivings about carrying the investigation
forward in the strength and depth with which we have begun.

A brief review in four parts:

1. production of an insight,
which may have many genetic aspects in its content,
but which at its root can only be factical,
since otherwise we would not have been able to go higher.
If there really is to be a “through,”
then (as a condition of its possibility)
we must presuppose an inner life,
independent in itself from the “through,”
and resting on itself.

2. We then made this insight,
produced within ourselves,
into an object, in order to analyze it
and consider it in its form.
There we proposed initially (in the whole of the second part)
that we saw into our concept of an actual “through,”
which appears freely created;
or rather, since everything depends on this,
[we proposed] that this concept was energetic and living,
that the inward life of this concept was
the principle of the energetic insight into a life beyond,
which grasped us, and which was intuited
in this insight as self-sufficient.
Thus, it was the principle of intuition
and of life in the intuition.
This latter need not arise except in intuition,
and its characterization as life in and for itself
is not intrinsically valid.
Instead, it can be fully explained
from the mere form of intuition
as projecting something self-sufficient
in the form of external existence.
In case another perspective is also possible,
and since it begins with the energy of reflection
and makes it a principle,
this way of looking at the insight can conveniently
be labelled the idealistic perspective,
in the terminology already provisionally adopted and explained.

3. But this other view, posited as the basis for the insight,
also proved to be possible, and was realized as well.
The presupposed life in itself should be
entirely and unconditionally in-itself;
and it is intuited as such.

Therefore, all being and life originates with it,
and apart from it there can be nothing.
The reported subjective condition of
this perspective and insight was this:
that one not stubbornly hold on to
the principle of idealism,
the energy of reflection,
but rather yield patiently
to this opposite insight.
The realistic perspective.

With this a warning!
not as if I detected traces of this misunderstanding
in someone speaking about the matter,
but rather because falling into this error is very easy,
as nearly all the philosophical public has done
in regard to the published science of knowing.
Do not think of “idealism” and “realism”
here as artificial philosophical systems
which the science of knowing wants to oppose:
having arrived in the circle of science,
we have nothing more to do with the criticism of systems.
Instead, it is the natural idealism and realism
that arise without any conscious effort
on our part in common knowing,
at least in its derived expressions and appearances:
and notwithstanding [the fact] that
both can certainly be understood in this depth
and so on the basis of their principles,
they nevertheless arise only in philosophy
and especially in the science of knowing.
It is still the latter's intention
to derive them as wholly natural disjunctions
and partialities of common knowing,
arising from themselves.

4. Both these perspectives were more closely specified
in their inner nature and character.
Thus, just as at the start,
we elevated ourselves above both
(we are not enclosed within them,
since we can move from one to the other),
[and we moved] from their facticity to the genesis of both,
out of their relative and mutual principles.
Hence, the insight which in this fourth part
we lived and were, was their genesis,
just as they were the genesis
for the previously created [terms],
in which both came together.
Thus, according to the basic law of our science,
we are constantly rising to a higher genesis
until we finally lose ourselves completely in it.

We characterized them in this way:
through its mere being,
the idealistic mode of thinking locates itself
in the standpoint of reflection,
makes this standpoint absolute by itself,
and its further development is nothing more than
the genesis of that which it already was
without any genesis other than its own absolute origin.
In its root, therefore, it is factical,
not just in relation to something outside itself
(as is, e.g., Kant's highest principle)
but rather in relation to itself.
It just posits itself unconditionally,
and everything else follows of itself;
and it frees itself from any further
accounting for its absolute positing.
The realistic mode of thinking proceeds no differently.
Abstracting entirely from [the facticity of] its thinking,
it presupposes the bare content of its thought
as solely valid and unconditionally true,
and completely consistently it denies any other
truth which is not contained in this content,
or, as would actually be the case here,
which contradicts it.
But this residing in the content is
itself an absolutely given fact,
which makes itself absolute
without wanting to give any further account of itself,
just like idealism.
Therefore, both are at their root factical;
and even ignoring that they are presented one-sidedly,
each annulling the other,
in this facticity each bears in itself
the mark of its insufficiency as
a highest principle for the science of knowing.

Let me describe it again with this formula:
at this highest point of contradiction
between the two terms absolutely demanding unification,
we find these: 0 and C,
or the form and the content,
or the forms of outer and inner existence,
or [as] in the previous lectures, essence and existence.
We appear to have obtained the absolute disjunction;
its unification promises to bring with it absolute oneness
and so to resolve our task fundamentally.

Today we will present considerations
regarding this solution
which still remain preliminary
(preliminary because we must advance
even further just to get to the point)
in order to prepare ourselves soundly
for the highest oneness.

First, it must be clear that the problem
cannot be resolved simply by
combining, rearranging, etc.,
what we know so far.
In relation to our next aim,
everything up to now is only
preparation and strengthening our spirit
for the highest insight;
and if the preceding should have
some further significance beyond this,
this significance can arise only
by deduction from the highest principle.
Now we must bring up something entirely new,
according to the insight adduced just above,
it is certain that something has remained
to some extent empirical and concrete for us.
We must investigate this and master it genetically.
The rule, therefore, is to investigate this facticity.
We have demonstrated the factical principles
of the perspectives in which we recently spent ourselves,
one after the other
(and which therefore undoubtedly contain
the highest [form of being] that we,
the scientists of knowing,
have ourselves so far attained).
One of the two must be developed.
Which shall it be?

If one grasps hold of it first,
the principle of idealism is
admitted to be absolutely incontrovertible.
Realism attacks this immediately
as idealistic stubbornness and a false maxim,
which it repudiates.
Thus, [realism] denies the principle
and cannot reason with idealism in any way.
On the other hand, idealism in turn makes
even the beginnings of realism impossible;
it ignores realism completely,
and hence can not have anything against it,
since, for idealism, [realism] does not exist.
Now realism obviously takes itself to be superior
just because of its denial of idealism's principle
and by its origin from this denial;
thus in realism at least
a negative relationship to idealism remains,
whereas even the possibility of such relatedness
is extirpated in idealism.
We must therefore attend to realism,
temporarily abstracting in every way from idealism;
since, as said before, we cannot let
realism be absolutely valid,
but rather want to correct it,
and since we cannot combat it
from the perspective of idealism,
we must fight against it on its own grounds:
catch it in self-contradiction.
Through this very contradiction,
which indeed brings a disjunction with it,
realism's empirical principle would become genetic,
and in this genesis, perhaps it will become
the principle of a higher realism and idealism [united] into one.
We have solved the first task,
finding which factical principle to develop.
This is where we grasp realism in its strength.
Its crucial point was life's in-itself and within-itself.
For now, we hold to this feature alone,
and in the mean time we could let go of life.
From this “in-itself ” realism infers
the negation of everything outside it.

But how does it bring about this very in-itself?
Let's reconstruct the process,
thinking the in-itself energetically.
I assert, and I invite you to consider this yourselves,
and see it immediately as true, I assert:
the in-itself has meaning only to the extent
that it [is] not what has been constructed,
and completely denies everything constructed,
all construction, and all constructability.
Consider well: when you say,
“thus it is in itself, unconditionally in itself,”
then you are saying [that] it exists thus
entirely independently from my asserting and thinking,
and from all asserting, thinking, and intuiting,
and from whatever other things outside
the in-itself may have a name.
This, you say, is how the in-itself must explain itself,
if it wishes to explain itself.
No other explanation yields the in-itself.
Result: the in-itself is to be described purely
as what negates thinking.
[This is] the first surprising observation:
here for the first time realism
[the perspective which, during the last lecture,
we made evident only factically
on the basis of its consequences, ]
is understood genetically.
Previously, that is, this insight arose and grasped us,
that if this life in itself is posited,
nothing apart from it can exist.
That is how it was;
we saw into it this way
and could not do otherwise.
Now we see that realism,
or we ourselves standing in its perspective,
act like the in itself,
which negates everything outside itself.
[We see] that realism therefore to some extent
(at least in its effects) is itself the in-itself,
and collapses into it;
and for this implicit reason,
in the appearance of our insight,
which grasped us in the previous lecture,
it annuls everything outside itself.
Therefore, we have comprehended genetically
something about realism which previously was only factical.

With this out of the way on one side,
let's reflect more closely about our own insight,
evoked before, and its principle.
I call on you to think the in-itself
and its meaning exactly and energetically,
whereupon you would then see into, etc.
You admit that without this exact thinking
you would not have seen it;
perhaps you even admit that for your whole life
you have thought the in-itself, faintly to be sure,
and yet this insight has not been produced in you.
(It can be shown that things have gone this way
for all philosophy without exception:
since if this insight opened itself really vividly for anyone,
then the discovery of the science of knowing
would not have taken so long!)
Thus, your insight into the negation of thinking in itself
presupposes positive thought,
and the proposition is as follows:
“In thought, thinking annuls itself
in the face of the in-itself.”

Now, to add even more consequences,
with which I only wish to make you acquainted:
the negation of thinking over against the in-itself
is not thought in free reflection,
as the in-itself ought to be thought by us,
rather it is immediately evident.
This is what we called intuition, and without doubt,
since the absolute in itself is found here,
this is the absolute intuition.
What the absolute intuition projects
would therefore be negation, absolute pure nothing,
obviously in opposition to the absolute in itself.
And thus idealism, which posits an absolute intuition of life, is
refuted at its root by a still deeper founding of realism.
It may well come up again as an appearance;
but, taken as absolute in the way it gave itself
out to be before, it is merely illusion.
Hence, we do not get past the previously mentioned
fundamental negation of ourselves over against the absolute.

The negation is intuited;
the in-itself is thought.
I ask how and in what way it is thought;
and I explain this implicitly obscure question
by the answer itself.
That is, we constructed this in-itself,
assembling it from parts, just as, for example,
at the beginning of our enterprise we constructed
oneness in the background as not being
empirically manifest identity,
and not multiplicity either,
but rather as the union of these two.
I ought not believe, instead we posit it directly,
together with its meaning,
in pure simplicity as the genuine construction:
thinking's negation is directly evident to us,
it grasps us as proceeding from it in its simplicity.
We therefore (this is very important)
didn't actually construct it,
instead it constructed itself
by means of itself.

Intuition, the absolute springing forth of light and insight,
was bound directly together with this construction.
We, however, would certainly not have produced this,
since it obviously produces itself and draws us forward with it.
Thus, the absolute's absolute self-construction
and the original light are completely and entirely one and inseparable,
and light arises from this self-construction
just as this self-construction comes from absolute light.
Hence, nothing at all remains here of a pregiven us:
and this is the higher realistic perspective.
But now we still hold on to the requirement, as with right we can,
that we should be able to think the in-itself,
and think it energetically,
thus that the living self-construction of
the in-itself within the light must have yielded to us,
and that once again this energy must be
the first condition of everything,
which results in an idealism,
which lies even higher.

But on this subject there are once again two things to consider:
first, we are also aware of this thinking, of this energy,
and apparently our claim that it exists
(without which general existence it could surely not be a principle)
is grounded solely on this awareness.
This, however, presupposes the light.
But since the light itself, at least in this its objective form,
does not exist in itself apart from the absolute
(as indeed it cannot, since nothing exists apart from the absolute)
but rather has its source in the in-itself,
we cannot appeal to it:
something which itself bears witness against us,
if we examine it more closely.

If one retains this higher presupposition of
the light in all possible deliverances of self-consciousness
(as the source of all idealistic assertions),
then the constant spirit of idealism in its highest form
and the fundamental error which contradicts and destroys it
fundamentally would be that it remains fixed on facticity,
at the objectifying light from which one can never
begin factically, but only intellectually.

So then (which is surely the same point from another angle),
one must reflect in opposition
to the idealistic objection:
You are not thinking the in-itself,
constructing it originally,
you are not thinking it out;
indeed how could you!
Nor is it known to you through something else,
which is not itself the in-itself,
rather it is merely known by you:
thus your knowing in and of itself sets it down;
or, as the matter may be more accurately put,
it sets itself down in your knowing
and as your knowing.
You have been doing this your whole life
without your will
and without the least effort
and in various forms,
as often as you expressed the judgment:
so and so exists.
And indeed philosophy has made war with you
and bound you in its circle,
not because of this procedure itself
but because of the thoughtlessness of it.
You will not give any credit to
your freedom and energy regarding the event itself.
Only now that you are aware of
this action and its significance
does your energy add anything;
likewise again with the declaration
of that which gives itself to you without any effort:
intuition.
Therefore, before we listen to you at all,
we must inquire more exactly
how far the testimony of intuition is valid.
This, additionally, by way of conclusion.
Whether it appears in its oneness
as the concept of some philosophical system,
either killed or never having lived,
as it was for us before
our realization of its significance,
or in a particular determination
as the “is” of a particular thing,
the faded in-itself is always [an object of] intuition
and is therefore dead.
For us it exists in the concept,
and is therefore living.
Hence for us there is nothing in intuition,
since everything is in the concept.
This is the most decisive distinguishing ground
between the science of knowing
and every other possible standpoint for knowing.
It grasps the in-itself conceptually:
every other mode of thought does not
conceive it, but only intuits it,
and in that way kills it to some extent.
The science of knowing grasps
each of these modes of thought
from its own perspective
as negations of the in-itself.
Not as absolute negations,
but as privative ones.
The things which, as we ascend,
we find to be not absolutely valid
(for instance, one-sided realism and idealism),
or which might be found to be so,
the science of knowing will take up again
on its descent as similar possible negations
of the absolute insight.

I.48
rtambhara tatra prajna

Thirteenth Lecture
Wednesday, May 9, 1804
Honored Guests:

Again today and even beyond I will climb on freely.
I say “freely” for you, because
I cannot provide the foundation for the distinctions
that will emerge here before using them,
rather I must first acquaint you with them in use,
even though a firm rule of ascent may well stand
at the basis of what I am doing.
If one just grasps my lecture precisely in other respects,
there is no danger of confusion despite the former circumstance,
because instead of the initially given crucial points
L (life) and C (concept)
we have the two perspectives:
realism (= genesis of life) and
idealism (= genesis of the concept).
This should be known from Friday's clear presentation
and its review the day before yesterday.
(Not generally, [but rather] in relation to the point,
as it was grasped there, one had to adhere to life in itself,
which was required to animate a “through.”
We will not move very far away from this now,
and it will easily be possible to reproduce from it
everything that needs to be presented,
or to trace the latter back to it.)
In a word, these two perspectives are our present guide,
until we find their principle of oneness
and can dispense with them directly.
Here, if anywhere, we need the capacity
to hold tight to what is presented firmly and immovably,
and to separate from everything that may
very well be rationally bound to it.
Otherwise, one will leap ahead, anticipate the inquiry,
and will not grasp the genetic process
linking what is taken up first and its higher terms,
which is what really matters;
instead the two will flow into one another factically.
“One will anticipate,” I said;
but it is not actually “one,”
not the self to whom this happens,
rather it is speculative reason,
running along automatically.
(Then let me add this remark in passing:
speculation, once aroused and brought into play,
as I partly know it has truly been brought into play in you,
is as active and vital as the empirical association of ideas ever can be,
because it is surrounded by a freer, lighter atmosphere:
and once one has entered this world,
one must be just as watchful against
leaps of speculation as previously against
the stubbornness of empiricism.
I want especially to warn those for whom the objects
of the present investigation appear very easily about this danger;
I advise you to make them a little harder for yourself,
since this appearance of ease may well arouse the suspicion
that the subject can be grasped more easily by speculative fantasy
than by pure, ever serene, reason.)

To work.

The in-itself reveals itself immediately
as entirely independent from knowledge or thought of itself,
and therefore as wholly denying the latter
in its own essential effects, in case one assigns such to it.
We did not construct the in-itself
in this immediately true and clear concept,
instead, as was immediately evident to us,
it constructed itself just as it was
in the constructive process, as denying thinking;
immediate insight, the absolute light, was
immediately united with this concept
and was evident in the same way.
Thus, the absolute in-itself revealed itself
as the source of the light;
hence the light was revealed
as in no way primordial:
which is now the first point
and which obviously bears in itself
the stamp of a higher realism.

Another idealism now tried to raise itself
against this realism, proceeding from this basis:
since we saw into the in-itself as negating vision,
we must ourselves have reflected energetically on it.
Thus, although we cannot deny that it constructed itself,
and the light as well, all of this was nevertheless
qualified by own vigorous reflection,
which therefore was the highest term of all.
As basing itself on absolute reflection,
this is obviously idealism;
and, since it does not depend,
as did the previous idealism,
on reflection about something conditioned
(actually carrying out a “through”)
as a means to realize the condition
[but instead rests on reflection
on the unconditioned in-itself]
it is a higher idealism.

We quickly struck down this idealism
with the following observation.
(“You then,” we address it in personified form,
“You think the in-itself: that is your principle.
But on what basis do you know this?
You cannot answer otherwise,
or bring up a different answer than this:
'I just see it, am immediately aware of it,'
and to be sure you do see it,
unconditionally objective and intuiting.”
(The last point is important and I will analyze it more closely.
In realism too we simply had an insight
into the self-construction of the in-itself;
but we had an insight, we saw into something living in itself,
and this living thing swept insight along with itself,
the very same relationship which we have already frequently found
in every manifestness presented genetically.
Now, however, ignoring [the fact] that
a purely objectifying intuition seems
to hover above the origin,
still this intuition is drawn at once toward,
and along with, the genesis.
In this insight, therefore,
a unification of the forms
of outer and inner existence,
of facticity and genetic development
seems to be hinted at.
Things stand completely differently with the seeing of
its thought to which idealism appeals.
That is, in that case we will certainly
not wish to report that we witness thinking as thinking
(as producing the in-itself)
in the act of producing,
as we certainly actually and in fact witness
the in-itself producing its construction;
rather intuition accommodates itself immediately only to
a thinking which is essentially opaque
and can be presented only factically.
Thus, it remains completely ambiguous
whether thinking originates from this intuition,
the intuition comes from the thinking,
or whether both might be only appearances
of a deeper hidden oneness which grounds them.

In case it is necessary to make this clearer.
Could you ever think really clearly and energetically,
which is what we are discussing here
(because faded thoughts and dreaming are
completely to be ignored),
without being aware of it;
and conversely could you possibly be
aware of such thinking without assuming
that you really and in fact were thinking?
Would the least doubt remain for you about the truth
of this testimony of your consciousness?
I think not.
It is therefore admittedly clear,
and immediately proved by the facts
that you cannot distinguish genuine thinking
from consciousness of it, and vice versa;
and that in this facticity, thinking posits its intuition,
and the intuition posits the absolute
truth and validity of its testimony.
We do not quarrel with you about that.
But you cannot provide the genetic middle term
for these two disjunctive terms.
Hence, you remain stuck in a facticity.
But, on the other hand, the genesis
which has arisen in the opposite,
realistic perspective opposes you;
that is to say, in that perspective we know nothing about
one term of your synthesis, your so called thinking.
But we recognize that to which you appeal
for verification of the latter.
Although [we do not do so] immediately,
we still do in its principle.
I say “not immediately”;
just as little do we recognize
there a simple consciousness,
expressing a fact absolutely,
as yours does according to our closer analysis.
But I also said “in its principle”:
in any event your consciousness presupposes light
and is only one of its determinations.
But light has been realized as itself originating from the in-itself
and its absolute self-construction;
however, if it originates from the in-itself,
then this latter cannot likewise originate from it, as you wish.
In your report that you are thinking actually and in fact
because you are conscious of yourself doing so,
you must posit your consciousness as absolute,
but the very source of this consciousness,
pure light, is not looked at factically,
which would bring us to the same level you occupy;
rather (which is more significant) it is
seen into genetically as itself not absolute.
And so this new idealism has been in part
determined further;
it does not, as first appears, even posit
as absolute a reflection which, according to it,
belongs simply to thinking, instead,
it posits the immediate intuition of this reflection as absolute
and is therefore different in kind from the first:
in part it is refuted as in truth valid,
although as an appearance it is not yet derived.
Assuming that it has become entirely clear to you,
hold onto this and let go of what is deeper.

By the way: in passing and while doing something else,
I have touched here on the very important distinction
between a merely factical regarding,
like our thinking of the in-itself,
and genetic insight,
like that into the in-itself's self-construction.
By means of the immediate testimony of our consciousness,
we cannot witness our thinking, as thinking,
literally as production;
we see it only so long as it exists,
or should exist, and it already is, or should be, while we see it;
on the other hand, we see the in-itself as
existing and as self-constructing,
simultaneously and reciprocally.
This point will have to come up again naturally
as the higher point of disjunction
for a still higher oneness,
and it will be very significant.
Meanwhile, let it be impressed [on you],
and, as an explanation,
let the following historical commentary be added
which may have whatever worth it can for admirers.
At the same time it may serve as an external test
of whether one has understood me.

Reinhold (or, as Reinhold claims, Bardili) wishes
to make thinking qua thinking the principle of being.
Therefore, on the most charitable interpretation,
his system would be situated within
the idealism we have just described,
and one must assume that by this he means
the thinking of the in itself
which we carried out the day before yesterday.
Now, first of all, he is very far removed from
explaining that this in-itself negates seeing,
as we have shown;
but then, which is worse,
in regard to the in-itself's real existence
(with which he generally has no dealings
and which, to be sure, he would be able to prove
only factically from the existence of individual things)
he does not appeal to consciousness
(a fact of which I reminded him, but which escaped him)
because he sees very well that doing so
would lead him to an idealism,
and he seems to have developed
an unconquerable horror of any idealism.

Hence, in the first place, his principle stands entirely in the air,
and he works to build a realism on absolutely nothing;
and he could only be driven to this by despair, following the rule:
“since it wouldn't work with anything I tried so far,
then it must work with the one thing still left in my field of vision.
”Second (and I have made this observation especially to make this point),
since, according to an absolute law of reason,
thinking does not let itself be seen into
as producing itself as thinking,
then naturally Reinhold cannot realize it this way either,
nor can he genetically deduce the least thing from it.
Hence he can only say, like Spinoza:
since everything that is lies in it,
and since now so-and-so exists,
it too must lie in it.
Because he was educated in the Kantian school,
and later by the science of knowing,
he may not now do this.
Hence he labors to deduce;
but since, if one only has a clear concept,
this appears to be completely impossible,
total darkness and obscurity arise in his system,
so that nobody grasps what he really wants.
If one considers this system from
the point of view of the science of knowing,
and indeed from the very point from
which we have just seen it,
then the obscurity of its principles is clear.)
Let's get back to business and draw the conclusion,
because in passing we have once again gained
a very clear insight into the true essence
of the science of knowing, that is,
of the principle which we are still obliged to present.

The refuted idealism makes immediate consciousness
into the absolute, the primordial source, the protector of truth;
and indeed absolute consciousness reveals itself in it
as the oneness of all other possible consciousness,
as reflection's self-consciousness.
In the first place, then, this stands firm
as one of our basic foundations.
Wherever we say “I am conscious of that,”
our testimony bears the basic formal character
of an absolute intuition,
which we have just described,
and makes a claim for the intrinsic validity of its contents.
This consciousness is now realized as self-consciousness
and reflection in its root:
all possible disjunctions and modes of consciousness
must be deduced from self-consciousness;
and we would therewith already
have achieved a comprehensive study.

It is clear that this consciousness is
completely one in itself
and capable of no inner disjunction;
because the thinking that arose in it was
that of the in-itself
which, as in-itself, is
entirely one and self-same;
thus it too was one,
and the consciousness of it was
only this one consciousness;
and therefore it was also one.
The self, or I, which arises here is
consequently the absolute I
[pure eternally self-identical, and unchanging]
but not the absolute as will soon be more precisely evident.
A specific disjunctive principle will
also have to be identified if:

a. in the course of thinking “the one in-itself ”
it should appear in a multi-faceted perspective,
even though in the background it is to remain
perpetually the same single in-itself, or categorical is, and
b. as a result there also arises a manifold view of thinking,
or reflection, and hence of the reflecting subject, or I
(all of which, just like the in-itself, are also
to remain the same one in the background).
It could well be, and indeed will turn out, that,
if we remain trapped in the absolute I
and never raise ourselves beyond it,
we may never discover this disjunctive principle
in proper genetic fashion,
and it will have to be disclosed factically.

(An historical note in passing:
even where things go the best for the science of knowing,
it has been taken for this idealistic system we have just described,
which presupposes what we have exactly characterized as
the absolute I to be the absolute
and derives everything else from it;
and no author known to me, friend or foe,
has risen to a higher conception of it.
That most have remained at an even lower level
than this conception goes without saying.
If a higher understanding is to arise
for anyone besides the originator of this science,
it could only be among the present listeners,
but it would not be written;
because what can be understood in writing
stands under the previous rule;
or it will be hit upon by you.
This remark has the consequence that nobody will get
a report about the essence of the science of knowing
from anyone other than the present one from the originator.
It will be immediately evident even from just
the clear and decisive letters of
what has been published on the topic
how false this interpretation is.)

The intrinsic validity of this idealism is refuted;
nevertheless, it may preserve its existence as an appearance,
and indeed as the foundation of all appearance,
which we have to expect:
it has been refuted on the grounds that it is factical
and that a higher development points to its origin.
One calls a fact “factical,”
and since here we are speaking of consciousness,
this fact would be a “fact of consciousness”;
or, to put it more strongly:
according to this idealistic system,
consciousness itself is a fact,
and since consciousness is for it the absolute,
the absolute would be a fact.
Now, from the first moment of its arising
the science of knowing has declared that the
primary error of all previous systems has been
that they began with something factical
and posited the absolute in this.
It, on the other hand, lays as its ground
and has given evidence for, an enactment,
which in these lectures I have called by the Greek term genesis,
since these are often more easily understood rightly than German terms.
Therefore from its first arising,
the science of knowing has gone beyond the idealism we have described.
It has shown this in another equally unambiguous way:
particularly regarding its basic point, the I.
It has never admitted that this I
[as found and perceived]
is its principle.
“As found” it is never the pure I,
but rather the individual personal being of each one,
and whoever claims to have found it as pure finds himself
in a psychological illusion of the kind
with which we have been charged by those
ignorant of this science's true principle.
So then, the science of knowing has always testified
that it recognizes the I as pure only as produced,
and that, as a science, it never places the I
at the pinnacle of its deductions,
because the productive process will always stand
higher than what is produced.
This production of the I,
and with it the whole of consciousness,
is now our task.

The idealism which has been rejected as intrinsically valid is
the same as absolute immediate consciousness.
Therefore, so that we now express
as forcefully as possible what it comes down to,
the science of knowing denies the validity of
immediate consciousness's testimony absolutely
as such and for this exact reason:
that it is this, and it proves this denial.
Solely in this way does the science of knowing
bring reason in itself to peace and oneness.
Only pure reason, which is to be grasped merely by intellect,
remains as solely valid.
So that no one is confused even for a moment
by a fancy which might easily arise here,
I immediately add a hint which must be discussed
further in what follows.

Namely, someone may say:
“But how can I grasp something in intellect without
being conscious in this intellectualizing?”
I answer: “Of course you cannot.”
But the ground of truth as truth does not rest in consciousness,
but only and entirely in truth itself.
You must always separate consciousness from truth,
as in no way making a difference to it.
Consciousness remains only an outer appearance of truth,
from which you can never escape
and whose grounds are to be given to you.
But if you believe that the grounds why truth is truth
are found in this consciousness, you lapse into illusion;
and every time something seems true to you
because you are conscious of it,
you become at the root idle illusion and error.
Here now it is obvious:

1. how in fact the science of knowing keeps its promise
and, as a doctrine of truth and reason,
expunges all facticity from itself.
The primordial fact and source of
everything factical is consciousness.
It can verify nothing, in light of the science of knowing's proof
that whenever truth is at issue
one should turn it aside and abstract from it.
To the extent that this science
in its second part is a phenomenology,
a doctrine of appearance and illusion,
which is possible only out of the first
and on the ground and basis of the latter,
to that extent it surely deduces both as existing,
but simply as they indeed exist, as factical.

2. It has become completely clear why nothing external
could be brought up against the science of knowing,
but that one always must begin with it to gain entrance into it.
The beginning point for a fight against it is
either grasped in intellect or not.
If it is grasped in intellect,
then it is either grasped intellectually immediately
[and that is the principle of the science]
or it is grasped mediately
and these must be either deductions of
the fundamental phenomenon
or phenomena derived from it.

One can come to the latter only through the former.
In this case, therefore, one would in every circumstance
be at one with the science of knowing,
be the science of knowing itself,
and in no case be in conflict with it.
If it were not grasped intellectually
but should nevertheless be true,
then one must appeal for verification
to one's immediate consciousness,
since there is no third way to get to the absolute itself
or even to an appearance of it.
But making this appeal one is straightaway turned aside
with the instruction that precisely because you are
immediately aware of it and appeal to that fact,
it must be false.
Of course thoughtlessness and drivel have
created a fancy title for themselves,
that of “skepticism,” and they believe that nothing is
so high that it cannot be forced under this rubric.
It must stay away from the science of knowing.
In pure reason doubt can no longer arise;
the former bears and holds itself
[and everyone who enters this region]
firmly and undisturbedly.
But if skepticism wishes to doubt
the implicit validity of consciousness
[and this is approximately what it
wants in some of its representatives]
and it does this provisionally in this or that corner,
although without having been able to bring about
a properly basic general doubt;
if it wants this, then with its general doubt
it has arrived too late for the science of knowing,
since the latter does not just doubt
this implicit validity in a provisional way,
but rather asserts and proves the invalidity of
what the general doubt only puts in question.
Just the possessor of this science
(who surveys all disjunctions in consciousness,
disjunctions which, if one assumes the validity
of consciousness in itself, become contradictions)
could present a skepticism which totally negate
everything assumed so far;
a skepticism to which those who have been playing
with all kinds of skeptical doubts as a pastime
might blanch and cry out: “Now the joke goes too far!”
Perhaps in this way one might even contribute
to arousing the presently stagnating philosophical interests.

I.49
sruta-anumana-prajnabhyam anya-visaya visesa-arthatvat

Fourteenth Lecture
Thursday, May 10, 1804
Honored Guests:

[Yesterday, an idealism, which made absolute consciousness
(in its actuality that is) into its principle was
presented, characterized, and refuted;
“in its actuality” I say,
since today we will uncover still
a different [idealism] in a place
where we do not expect it,
one which makes the same thing its principle,
only merely in its possibility.
Now this absolute consciousness was self-consciousness
in the energy (of reflective thinking, as it later turned out).
On this point I will add another remark
relating to the outer history of the science of knowing,
which naturally is not intended to parade my conflict
with this unphilosophical age in front of you,
but rather only to provide hints to those
who are following this science in my published writings
and who want to rediscover it in the form employed there,
telling them where they should direct their attention.]

(Let me add just this,
it is clear that just as
the form of outer existence as such perishes,
so its opposite as such perishes as well;
therefore, realism, or more accurately objectivism
perishes along with that idealism which,
because of language's ambiguity,
we might better call subjectivism.
Reality remains, as inner being;
as we must express ourselves just in order to talk;
but in no way does it remain a term of any relation,
since a second term for the relation,
and indeed all relations in general,
have been given up.
Therefore, it is not “objective,”
since this word has meaning only
over against subjectivity,
which has no meaning from our standpoint.
Only one recent philosophical writer
(I mean Schelling)
has had a suspicion about this truth,
with his so-called system of identity;
not, certainly, that he had seen into
the absolute negation of subject and object,
but that with his system he aimed
at a synthesis post factum;
and with this operation he believed
he had gone beyond the range
of the science of knowing.
Here is how things stand about that:
he had perceived this synthesizing
in the science of knowing,
which carried it out,
and he believed himself to be something more
when he said what it did.
This is the first unlucky blow that befell him:
saying, which always stems from subjectivity
and by its nature presents a dead object, is
not more but less excellent than doing,
which stands between both in
the midpoint of inwardly living being.
Moreover, he does not prove this claim,
but lets the science of knowing do it for him,
which again seems odd:
that a system which admittedly contains
the grounds for proving our own system's
basic principle should be placed below it.
Now he begins and asserts:
reason is the absolute indifference
between subject and object.
But here must first also be added
that it cannot be an absolute point of indifference
without also being an absolute point of differentiation.
That it is neither one nor the other absolutely
but both relatively;
and that therefore, however one may begin it,
no spark of absoluteness may be brought into this reason.
So then he says: reason exists;
in this way he externalizes reason from the start
and sets himself apart from it;
thus one must congratulate him
that with his definition he has not hit the right reason.
This objectification of reason is
completely the wrong path.
The business of philosophy is not
to talk around reason from the outside,
but really and in all seriousness
to conduct rational existence.
Nevertheless, this author is
the hero of all passionate,
and therefore empty and confused heads;
and especially of those who do not disavow
defects like those reproved above,
to which, when possible,
they come even more extremely
because they think either that the inferences are good,
although the principles are false,
but that the whole is still excellent,
admittedly overlooking that all
the individual parts are good for nothing;
or finally that although it is
neither true, nor good, nor beautiful,
it still remains very interesting.
For my own person, I have said all this
only in the interest of history
and to elucidate my own views,
but in no way to weaken anyone's respect
for their hero or to lead them to myself.
Because if anyone wishes to be condemned to error,
I have nothing against it.)

Further, and to the point;
[here is the] chief result:
Consciousness has been rejected in its intrinsic validity,
despite the fact that we have admitted we cannot escape it.
We absolutely, even here in the science of knowing.
Therefore,
1. if we have once seen into this fact,
although factically we could never negate consciousness,
we will not really believe it when judging truth;
instead, when judging, we will abstract from it;
indeed, on the condition that we want to get to truth,
we must do this, but not unconditionally,
since it is not necessary that we see into the truth.
Here for the first time,
we ourselves have become entwined mediately
in science and the circle of its manifestness,
and we became the topic without any effort on our part
in a new way, because we were speaking of consciousness
and we ourselves occur empirically as consciousness,
which might serve very well for
the genetic deduction of the I, which we seek.
Further, we should cultivate here a maxim for ourselves,
a rule of judging which can be appropriated only through freedom;
and this maxim should become
the absolute principle in and for us:
if never of truth itself,
then of this truth's factical appearance.
In one respect, this is generally meaningful
and may lead to a new idealism,
in a region where it alone can have value
as the principle of appearance;
and in another respect it confirms
the opinion expressed previously
when we were characterizing and refuting
the more deeply placed idealism and realism:
that, as grounded on conflicting maxims,
both could only be reunited
by means of a higher maxim.
And so notice this also at the same time:
that the present maxim is quite different
from that of the previous realism
(which allowed truth alone to be unconditionally valid)
in that the present one has a condition:
“if truth is to be valid, then it must, etc.”;
at the same time certainly acknowledging that
truth need not necessarily be valid.
Finally, freedom shows itself here
in its most original form,
in respect to its actual operation
as we have always described it,
not as affirmative, creating truth,
but rather merely as negative, averting illusion.
All of which, although very significant,
are only expositions of this insight:
if consciousness in itself has
no validity and relation to truth,
then we must abstract from
all effects of this consciousness in
the investigations which lie before us,
and whose task is to deliver truth
and the absolute purely to the light of day.

2. From what, then, do we actually need to abstract,
and what is its unavoidable effect?
This is evident from that salient point and nerve,
for whose sake consciousness was rejected as insufficient.
But, in virtue of our last investigation, the nerve was this,
that it projected something factically,
that is in its highest potency;
and in our case [it projected] the energy,
which then would become thinking,
whose genetic connection with it
it could in no way give;
thus a thing that it projected
purely and through an absolute gap.
Grasp this character exactly,
just as it has been given, and
to that end remember what was said last time:
e.g., you would not presume
that you could actually think
without being conscious of it,
and vice versa;
nor that you could be conscious of your thinking,
without in fact actually thinking,
or that this consciousness was deceiving you;
but if you were asked to provide
an explicable and explanatory ground
for the connection of these two terms,
you would never be able to provide such a ground.
Thus, changing places with you at the site
from which you conducted your proof:
your consciousness of thinking should
contain a thinking process,
actual, true and really present,
without you being able to give an accounting of it;
therefore, this consciousness projects
a true reality outward, discontinuously:
an absolute inconceivability and inexplicability.

This discontinuous projection is evidently
the same one that we have previously called,
and presently call, the form of outer existence,
which shows itself in every categorical is.
For what this means, as a projection,
concerning which no further account can be given
and which thus is discontinuous,
is the same as what we called “death at the root.”
The gap, the rupture of intellectual activity in it,
is just death's lair.
Now we should not admit the validity of this projection,
or form of outer existence,
although we can never free ourselves from it factically;
and we should know that it means nothing;
we should know, wherever it arises,
that it is indeed only the result
and effect of mere consciousness
(ignoring that this consciousness
remains hidden in its roots)
and therefore not let ourselves
be led astray by it.
This is the sense of our discovered maxim,
which is to be ours from now on in every case
where we need it.
This very is is the original appearance:
which is closely related to,
and may well be the same thing as,
the I which we presented previously
as the original appearance.

3. Thus it is decreed against the highest idealism,
and this maxim imposes the highest realism yet on us.
Before we go further under its leadership, though,
it may be advisable to test it against the law
which it itself has brought forward,
thus to draw it directly before its own seat of judgment
in order to discover whether it itself is indeed pure realism.
It proceeds from the in-itself
and proposes this as the absolute.
But what is this in-itself as such
and in its own self?
You are invited here to
a very deep reflection and abstraction.
Although the foregoing brings to an end
the thinking of this in-itself,
reflected first by consciousness,
although, we have likewise already had to
admit before that we did not construct this in-itself
but that it already is found in advance
as completely constructed and finished
and comprehensible in itself,
and thus as constructed by itself,
so that we in any case have nothing to do with this;
we may still investigate this original construction
more closely in terms of its content.

I have said that I invite you to
a very deep reflection and abstraction.
What this abstraction is
(an abstraction which may be described
in words as well as possible)
will barely be made clear from what follows;
but this will not harm anything,
and it is certain in any case that it is already clear,
and I impose on myself the task of grasping the highest in words,
and on you the task of understanding it in a pure form.
Thus, once again, as already previously,
the topic is the in-itself,
and we are called, here as before, to
a consideration of its inner meaning
and to its re-construction.
We will not complete again what has been done before,
an act which, holding us in a circle,
would not advance us from the spot.
But, if we wish [to achieve] something else,
how is this to be distinguished from the preceding?
Thus, above we presupposed the in-itself
and considered its meaning while we supplemented it with life,
or a primal fantasy, dissolved ourself in the latter,
and had our root in it.
Of course this life is not supposed to be our life,
but rather the very life and self-construction of the in-itself:
it was then an inner determination
(one which arose immediately in this context)
of the original life itself,
which still remained dominant in this case.
So it was previously.
Now, however, we elevate ourselves for the first time to
the in-itself that is presupposed in this procedure,
[knowing it] as presupposed and unconditionally immediate,
independent of this living reconstruction,
determinate, and comprehensible:
and without this original significance the reconstruction,
as a reconstruction and clarification,
would have no basis and no guide.
For this reason, I said previously that
the originally completed construction,
the enduring content, should be demonstrated.
Therefore, things must proceed in this work
so that what is absolutely presupposed remains presupposed,
so that as a result the living quality
which we bring with us will mean nothing at all
[neither as our vivacity, nor as vivacity in general]
likewise the very validity of primal fantasy,
although it cannot be withheld factically,
yet is denied as real, in which denial
the true essence of reason may well consist.
(Or, more succinctly, if only one will understand it;
what has been made perceptible resides
in this latter construction,
and the meaning of the in-itself
grasped purely intellectually
should be found there.)

So much in the way of preliminary formal description
of this new reconstruction of the in-itself.
Now to the solution.
However one may wish to take up the in-itself,
it is still always qualified by
the negation of something opposed to it,
thereby as in-itself it is itself something relative,
the oneness of a duality, and vice versa.
Certainly, it is genuinely at once
a synthetic and analytic principle,
as we have all along looked for:
but still it is no true self-sufficient oneness;
since the oneness lets itself be grasped only through duality:
although admittedly duality also lets itself be completely
grasped and explained through oneness.
In a word, the in-itself, grasped more profoundly,
is no in-itself, no absolute, because it is not a true oneness,
and even our realism has not pushed through to the absolute.
Viewed still more rigorously,
in the oneness there is in the background
a projection of in-itself and not-in-itself,
which posit one another reciprocally
for explanation and comprehensibility,
and which negate one another in reality;
and in return, the oneness is a projection of both terms.
Further, this projection happens completely immediately,
through a gap, without being able to provide
the requisite accounting of itself.
Because how an in-itself and a not-in-itself
follow from oneness as simple, pure oneness
cannot be explained.
Of course, it can be done if the oneness is already assumed to be
the oneness of the in-itself and not-in-itself;
but then the inconceivability
and inexplicability is in this determinateness of oneness,
and it itself is only a projection through an irrational gap.
This determinateness would have no warrant
other than immediate awareness;
and actually, if we will think back to
how we have arrived at everything so far,
it has no other ground.
“Think an in-itself” it began,
and this thinking, or consciousness was possible.
And this possibility has shaped our entire investigation to date;
thus we have supported ourselves on consciousness,
if not quite on its actuality,
then certainly on its possibility,
and in this quality we have had it for our principle.
Hence, our highest realism,
the highest standpoint of our own speculation,
is itself revealed here as an idealism,
which so far has just remained hidden in its roots;
it is fundamentally factical
and a discontinuous projection,
does not stand up to its own criteria,
and, according to the rules it itself established,
it is to be given up.

4. Why should it be given up?
What was the true source of the error
which we discovered in it?
Being in-itself [was discovered] as
a negation and a relational term.
Hence we must unconditionally let that go,
if it, or our entire system, is to survive.
But something is still left over for us.
I affirm this and instruct you to find it with me:
being and existence and resting,
taken as absolute, remains,
and of course I add:
“being and resting on itself,”
but I already clearly knew that
the latter would be a mere supplement
for clarification and illustration,
but would mean nothing at all
in and for itself
and would add no supplement to the
completeness and self-sufficiency of
being's inner essence.
If I wish to look back to the previous,
already discarded expression,
“being in-itself” means a being
which indeed needs no other being for its existence.
Precisely through this not-needing
it becomes intrinsically more and more
real than it was before;
and not-needing this not-needing does not also
belong to its absolute not-needing,
and so too with the not-needing of
this not-needing of not-needing,
so that this supplement in its
endless repeatability remains always the same
and always meaning nothing in relation to
the essence taken seriously and inwardly.
Thus, I see into [the fact] that
(generally, in its core,
thus indeed as the point of oneness,
which was previously tested and discarded)
the entire relation and comparison
with the not-in-itself
(from which the form of the in-itself as such arises)
is completely null in comparison to the essence.
It is without meaning or effect.
Because I see into this,
and thereby handle the addition
in almost as negative a way
as does the essence itself,
then I must, as an insight,
participate in the essence
in some manner still to be developed.

Now, to be sure, if I pay attention to myself,
I can always become aware that I objectify
and project this pure being:
but I already certainly know
that this means nothing,
alters nothing about being,
and adds nothing to it.
To be sure, in another shape
this projection is the in-itself's supplement,
whose nothingness has already been realized:
therefore I will never be deceived by it.
In brief, the entire outer existential form
has perished in this shape,
since it is the latter in the highest [element]
in which it occurs, the in-itself;
we have only the inner essence left with which to deal,
in order to work it through:
but we truly work it through
if we see into it as the genesis
for its appearance in the outer existential form;
and nothing else can lead us to that except
not allowing ourselves to be deceived by this form.
If it does deceive us, then we just are it,
dissolved and lost in it,
and we will never arrive at its origin.

I wished at least to attach this last, fourth, point here,
in order not to end the lecture with death and destruction,
as it does on first appearance.
Tomorrow its further development.

I.50
taj-ja samskara anya-samskara-pratibandhi

Fifteenth Lecture
Friday, May 11, 1804
N.B. Which Contains the Basic Proposition
Honored Guests:

My task for today is this:
first of all to work out fully and completely
the main point discovered last time;
then to present a general review of
the new material added this week,
and thus, as it were, to balance the accounts,
since with this lecture we conclude the week's work
and a discussion period intervenes

On the first matter.
This much as a reminder in advance.
To begin with, the point I must present is
the clearest and simultaneously the most hidden of all,
in the place where there is no clarity.
Not much can be said about it,
rather it must be conceived at one stroke;
even less can anything be said about it
or words used to assist comprehension,
since objectivity, the first basic twist of all language,
has already long been abandoned in our maxim,
and is to be annulled here within absolute insight.
At this point, therefore, I can rely only
on the clarity and rapidity of spirit
which you have achieved in the previous investigations.
So then, on a particular occasion I divided
the science of knowing into two main parts;
one, that it is a doctrine of reason and truth,
and second, that it is a doctrine of appearance and illusion,
but one that is indeed true and is grounded in truth.
The first part consists of a single insight and is begun
and completed with the single point which I will now present.
To work!
After the problem of absolute relation,
which appeared in the original in-itself,
which itself pointed to a not-in-itself,
nothing remained for us except
the pure, bare being by which, following the maxim,
our objectivizing intuition must be rejected as inadequate.
What then is this pure being in its abstraction from relatedness?
Could we make it even clearer to ourselves and reconstruct it?
I say yes: the very abstraction imposed on us helps.
Being is entirely of itself, in itself, and through itself;
this self is not to be taken as an antithesis,
but grasped with the requisite abstraction purely inwardly,
as it very well can be grasped,
and as I for example am most fervently conscious of grasping it.
Therefore, to express ourselves scholastically,
it has been constructed as a being in pure act,
so that both being and living, and living and being
completely interpenetrate, dissolve into one another,
and are the same, and this self-same inwardness is
the one completely unified being,
which was the first point.

This sole being and life can not exist,
or be looked for, outside itself,
and nothing at all can exist outside it.
Briefly and in a word:
duality or multiplicity does not occur at all
or under any conditions, only oneness;
because by its own agency
being itself carries self-enclosed oneness with itself,
and its essence consists in this.
Being [understood by language as a noun]
cannot literally be actively
except immediately in living;
but it [“being” understood as a noun] is only verbally;
because completely noun-like being is objectivity,
which in no way suffices:
and it is only by surrendering
this substantiality and objectivity,
not merely in pretense
but in the fact and truth of insight,
that one arrives at reason.

On the other side,
what lives immediately is the “esse,”
since only the “to be” lives,
and then it, as an indivisible oneness,
which cannot exist outside itself
and cannot go out of itself into duality,
is something to which the things
we have demonstrated immediately apply.

But we live immediately in the act of living itself,
therefore we are the one undivided being itself,
in itself, of itself, through itself,
which can never go outside itself to duality.

“We,” I say, to be sure, we are immediately conscious that,
insofar as we speak of it, we again objectify
this “We” itself with its inward life
but we already know that this objectification means
as little in this case as in any other.
And surely we know that we are not talking about this We-in-itself,
separated by an irrational discontinuity from the other We
which ought to be conscious;
rather [we are talking] purely
about the one We-in-itself,
living purely in itself,
which we conceive merely through
our own energetic negation of the conceiving
which obtrudes on us empirically here.
This We, in immediate living itself;
this We, not qualified or characterizable
by anything that might occur to someone,
but rather characterizable purely
by immediately actual life itself.

This was the surprising insight
to which I wished to elevate you,
in which reason and truth emerge purely.
If anyone should need it, I will point
to it briefly from yet another aspect.
If being is occupied with its own absolute living,
and can never emerge out from it,
then it is a self-enclosed I,
and can be nothing else besides this;
and likewise a self-enclosed I is being:
which “I” we could now call “We”
in anticipation of a division in it.
Hence, we in no way depend here on
an empirical perception of our life,
which would need to be completely rejected
as a modification of consciousness;
rather we are depending on the genetic insight into life and the I,
which emerges from the construction of the one being, and vice versa.
We already know, and abstract completely from, the fact,
that this very insight as such, together with its reversal,
is irrelevant and vanishes;
we will need to look back to it again only in deducing phenomena.

As it has been presented now,
this intrinsically cannot be made clearer by anything else;
since it itself is the original source
and ground of all other clarity.
Still, the subjective eye can get clearer,
and become more fit for this clarity,
through deeper explanation of the immediately surrounding terms;
therefore I add in this regard another consideration
which lies on the system's path anyway.
Yesterday, and again today too at the beginning of our meditation,
although we constructed being according to its inner essence,
if we only remember, we placed being objectively before ourselves,
despite the fact that, though only as a result of following our maxim,
we grant no validity to this objectivity:—factically,
even though surely not intelligibly or in reason,
being remains separated from itself.
But, just as in this reflective process we are
grasped by the insight that being itself is an absolute I, or we;
thus the first remaining disjunction,
between being and the we, is completely annulled, even in facticity,
and the first version of the form of existence is also factically negated.
Previously [we knew that] at the very least
we emerged factically out of ourselves toward being,
and [in this process] it could very well happen that
being would not come out of itself,
especially if we did not wish to be being.

We did not accept this being as valid,
merely on the basis of a maxim
that had its proof via derived terms,
and thus might well need a new proof here.
As we become being itself in the insight we have produced,
we can, as a result of this insight, no longer come out
of ourselves toward being, since we are it;
and really we absolutely cannot come out of ourselves at all,
because being cannot come out of itself.
Here the preceding maxim has received its proof,
its law, and its immediate realization in the insight:
because this insight in fact no longer objectifies being.
Now to be sure, I say, no other objectifying consciousness arises
together with this insight, because in order for that to occur,
a self-reflecting would be required to stand in between,
however the possibility does arise of an objectifying consciousness:
[namely,] our own.
Now, as regards the content of this new objectification,
it is already clear that it does not bring with itself
any disjunction in our subject matter,
as does the first, between real being and absolute non-being,
rather this content brings only the mere repetition
and repeated supposition of one and the same I, or We,
which is entirely self-enclosed, which encompasses all reality in itself,
and which is therefore entirely unalterable;
therefore, it does not contradict the original law
of not going out of oneself in essence.

But, as the first stage of our descent into phenomenology,
we will have to explore whence this
empty repetition and doubling may arise;
today it is merely a matter of establishing the insight
which is expressive of pure reason, that being or the absolute is
a self-enclosed I, in its unalterability.

Now on to the second part of the general review.
As must arise in any presentation of the science of knowing,
we have proceeded factically, inwardly completing
something useful for our purpose,
and paying attention to how we have done it,
compelled always, as is evident, by
an unconscious law of reason working in us.
On this path, which I will not repeat now,
we had elevated ourselves to a pure “through,”
as the essence of the concept;
and had understood that the latter's realization
presupposed a self-subsisting being.

Having presented this insight as a fact
and reflected further about its principle,
it turned out either that one could posit
the energy of thinking a “through” needing
completion as the absolute
and thus as the source of intuition
and of life in itself within intuition,
which was an idealism;
or that, considering that life ought to exist in itself,
one could take the latter as the principle,
with the result that everything else perishes;
this latter [was] a realism.

Both were supported by maxims.
The former on this: the fact of reflection is
to be taken as valid, and nothing else;
the latter on this:
the content of the evident proposition is
to be taken as valid, and nothing else;
and, for that very reason, both are at bottom factical,
since indeed even the contents of what is manifest,
which alone should be valid for realism, is only a fact.

Along with the necessity that arose from this
to ascend higher and to master the facts genetically,
we turned our attention to what promised to be most significant here,
[namely] to the in-itself, bound to the realistic principle,
to life in itself:
and this further deliberation was the first step we made this week.
It turned out that the in-itself manifests as an absolute
negation of the validity of all seeing directed toward itself:
that it constructs itself in immediate manifestness,
and with its own self-construction even gives
off immediate manifestness or light:
yielding a higher realism which deduces
insight and the light themselves,
items that the first realism was content to ignore.
A new idealism attempts to establish itself
against this new realism.

That is, we had to take command of ourselves
and struggle energetically to contemplate
the in-itself in its significance.
So, [we] believed we realized that this in-itself first appeared
as a result of this reflection as
simultaneously constructing itself with
immediate manifestness in the light;
and that consequently this energy of ours would be
the basic principle and first link in the whole matter.

Realism, or we ourselves, since we are nothing else than this realism,
fights very boldly against this as follows:
“If you really actually think ...”
and to what will you appeal for confirmation of this assertion:
you can adduce nothing more than that you are aware of yourself,
but you cannot derive thinking genetically in its reality and truthfulness,
as you should, from your consciousness in which you report it;
but, by contrast, we can derive the very consciousness to which you appeal,
and which you make your principle, genetically,
since this can surely be only a modification of insight and light,
but light proceeds directly out of the in-itself,
manifestness in unmediated manifestness.
The higher maxim presented in this reasoning would just be this:
to give no credence to the assertions
of simple, immediate consciousness,
even if one cannot factically free oneself from them,
but rather to abstract from them.

What is this consciousness's effect,
for the sake of which it is discarded;
and therefore what is that
which must always be removed from the truth?
Answer:
the absolute projection of an object
whose origin is inexplicable,
so that between the projective act
and the projected object
everything is dark and bare;
as I think I can express very accurately,
if a little scholastically,
a proiectio per hiatum irrationale
(projection through an irrational gap).

Let me again draw your attention to this point,
both for now and for all your future
studies and opinions in philosophy:
if my current presentation of
the science of knowing has been clearer
than all my previous versions of the same science,
and can maintain itself in this clarity,
and if clear understanding of the system is
to make a new advance by these means,
the ground for this must lie simply
in the impartial establishment of
the maxim that immediate consciousness is
in no way sufficient and that hence
it does not suffice in its basic
law of projection per hiatum.
Of course the essence of this truth has ruled
in every possible presentation of
the science of knowing from the very first hint
which I gave in a “Review of Aenesidemus”
in the Allgemeinen Literatur Zeitung;
because this maxim is identical with
the principle of absolute genesis.
If nothing that has not been realized
genetically is permitted,
then projection per hiatum will not be permitted,
since its essence consists precisely in non-genesis.
If one has not made himself explicitly aware
that this non-genesis,
which is to be restrained in thinking,
remains a factical element of the consciousness
which is unavoidable on every path
of all our investigations
and of the science of knowing itself,
then one torments and exhausts himself
trying to eliminate this illusion,
as if that were possible.
And the sole remaining way of breaking through to
truth is to divide the illusion,
and intellectually to destroy each part one at a time,
while during this procedure one actually defers the illusion to
another piece at which annihilation will arrive later,
when the first piece could once again serve
as the bearer of the illusion.
This was the science of knowing's previous path,
and it is clear that it too leads to the goal,
although with greater difficulty.
However, if one knows the origin of non-genesis in advance,
and that it always comes to nothing,
although it is unavoidable, then one no longer fights against it,
but rather allows it to work peacefully:
one simply ignores it and abstains from its results.
It is possible in this way alone to gain
access to insight immediately as we have done decisively,
and not just by inference from the nonexistence of the two halves.

Let me continue the recapitulation:
the realism which presented
the recently analyzed maxim was
itself questionable and was brought before
the judgment seat of its own maxims.
There, on closer consideration, the in-itself
(inasmuch as it is assumed to be something original
and independent of all living construction
and to possess this same guiding meaning)
turned out to be incomprehensible
without a not-in-itself.
Therefore, in the understanding
it turned out to be no in-itself
(something comprehensible by itself alone) at all,
and instead it [turned out to be] understandable
only through its correlative term.
Therefore, the unity of understanding,
which reason presupposes here,
cannot merely be a simple self-determined oneness;
instead it must be a unity-in-relation,
meaningless without two terms
which arise within it in two different connections:
in part as positing one another
and in part as negating one another,
thus the well-known “through” and
the five-foldness recognized in it.
If one must also now concede that
once oneness has been admitted,
the terms incontestably posit themselves genetically,
still oneness itself is not thereby explained genetically;
hence it is present simply by means of
a proiectio per hiatum irrationalem,
which this system, [presenting itself] as realism,
has made against its own principles.

I.51
tasyapi nirodha sarva-nirodha nirbija samadhi

With this disclosed, absolutely everything
in the in-itself which pointed to relations
was to be abandoned,
and so nothing else remained behind
except simple, pure being
as absolute, self-enclosed oneness,
which can only arise in itself,
and in its own immediate arising or life:
which therefore always arises
from a place where an arising,
a living, simply occurs, and
it does not arise except in such an arising,
and therefore occurs as absolute I,
as today's disclosure could be put briefly.
Generally, one can think this simplest
of all insights in indefinitely many forms,
if it has once become clear.
Its spirit is that being exists
immediately only in being, or life,
and that it exists only
as a whole, undivided oneness.
