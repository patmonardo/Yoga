III.53
ksana-tat-kramayo samyama viveka-jam jnanam

Twenty-sixth Lecture
Monday, June 4, 1804
Honored Guests:

Absolute knowing has been presented
objectively and in its content,
and if that were the only issue,
our work would be completed.
However, the question still remains
how we have become this knowing,
since we have become it;
and, if this has further conditions,
what are they?
The first [is] the transcendental,
in all its determinations,
the second [is] ordinary.
Our proper standpoint [is]
the reciprocal determination of both.

Based on our previous experience,
I assume that we are sufficiently
constrained by the shortness of time
from easily comprehending the difficulties.
Therefore, today we will pass immediately
into the indicated central point.
In the subsequent sessions,
this will make it easier to fill out
the remaining gaps between the center
and the designated end-points.
I tell you in advance that
this investigation is important
and raises entirely new topics.

1. How, I say, have we arrived here?
It is quite clear that we do not want
to see the steps that we have taken historically retold,
and that this entire “how” contains the de jure question
rather than any historical one.
Without further ado, it is clear that,
in the knowing just presented,
we have, and are the same as, absolute knowing
only on the condition that we are certain of this,
or that we are formally certain in general
and that we express this formal certainty
within ourselves in fulfilling this knowing.
Otherwise, it could always be certain
for someone else without being so for us,
and we would in no way have elevated
ourselves to the absolute,
although we are able to
repeat the word quite properly.
This is the first point.

The content presented should itself be
the expression of absolute knowing, or certainty.
Therefore, thought in relation to what has gone before,
the following two modes of inference are possible.
We are certain, and so what we say
in this situation of certainty is certain,
it is an expression of inward certainty;
or, this is certain, we see into it;
therefore we are certain, or expressing certainty.
Whichever of the premises one chooses,
one presupposes that the nature of certainty is known.
Without doubt, we will carry out one of the two inferences.
Hence, we must presuppose that the essence of knowing is known.

2. If it is known.
Previously we designated it as resting in itself,
thus a oneness of qualities.
We cannot assume it here as this,
since this oneness has been
completely lost to us in the foregoing.
Indeed, it has been lost in relation to
an inner essential quality, as in an image,
or in the law of an imaging,
which excluded, in the second power,
all variability and hence all qualitative similarity.
Here we must characterize certainty with this higher concept.
Thus, it is essential, immanent self-enclosure
(as absolute being was previously seen to be).

3. Now, for the judgment
that we wish to pass,
that “we are certain,”
we presuppose a primordial concept, or description,
of this immanent self-enclosure.
How is this concept or description primordially possible?
That means, the law for such a description should be presented,
(be understood by us simply as a primordial description.)

Since in fact I rise here to a higher term
than all those presented so far,
I cannot connect what is to be presented to these,
nor can I explain it from them.
Rather, I can rely only on your creative intuition,
which I merely hope to guide.
Everything will become completely clear in context,
if only one can make a beginning toward clarity
based on the parts.

I say that, description, as such,
is inward, immanent projection of the described.
Above all, this in no way means
an objective projection through a gap.
Instead, [it means] a projection
that recognizes itself as a projection
(as something superficial)
immediately in itself
and negates itself as such;
and so posits something described
only through this self-negation,
an inwardness.
In a word, it is just pure ideal seeing,
or intuition, permeating itself simply as such.
I do not say that this permeation exists in itself.
Instead, I commend it to you,
as the “We” of the science of knowing,
as that through which alone you become this “We.”
You permeate it, if you grasp it as self-negating,
yet positing in this negation something described,
positing itself and us as the external, and so negated.
“Seeing X” means not regarding the seeing as X;
thus negating it.
In this negation, seeing becomes a seeing,
and something seen arises simply
if it abstracts from X.
This, which must now be grasped in intuition
much more deeply than I can express it in words,
is what I want to point out.
It is the inner essence of pure seeing, as such,
which we have actually realized here in its essence,
as surely as you have followed my intuition.
This is the first point.

Here now should be a description of certainty,
as absolute enclosure within itself.
If we just stay with the elicited intuition of pure seeing,
or the purely formal description,
then we can derive from it alone this content:
that it is a description of an enclosure within itself.
Being is what it intuitively projects
in this negation of itself in which it still exists.
Because it projects this within, and by means of,
its own ineradicable essence, this essence is an expression,
an intrinsically self-expressive being,
an intrinsically living and powerful being.

III.54
jati-laksana-desir anyata-anavacchedat tulyayos tata pratipatti

(Because I think that I can cast
the highest clarity over the whole
I make the following remarks:

1. Seeing
(which reveals itself intrinsically
as material and qualitative externality and emanence,
as anyone can discover for himself,
if he notices what seeing means),
thereby annuls itself in the face
of absolute immanent being.
I do not say, and have not proven,
that it does this on any particular grounds,
but, according to the presupposition,
it has done so in our own person.

2.  Yet in this self-disclosure
in essence and self-negation,
it is, because we continue to know;
and it exists with its unchangeable basic property,
as expression.
Hence, the being in front of which
it negates itself is none other
than its own higher being,
in the presence of which the lower,
which is to be objectified as seeing, vanishes.
This, its being, therefore carries its
primordial feature, expression,
which, since it has now become absolute,
expresses itself.

3. Therefore, when it negates itself as seeing,
seeing becomes actual seeing,
inwardly and truly effective,
or, as is better here, pure light.
Thus, pure light does not come to be as
absolutely inward self-expression, power, and life,
instead it just is.
It emerges only in insight;
and in that case it comes to be
through seeing's absolute self-negation
in the presence of being.

4. To what extent then,
and on what grounds,
is seeing negated?
Answer: because it is the expression of another
and is set over against another as what it sees
(which other, as a result of its own self-negation,
lies within seeing itself),
therefore, what is negated
is simply absolute intuition as such.

5. This intuition is likewise present
in our recently generated insight into
absolute light as a living self-expression,
and it is to be negated in relation to
inner essence and truth,
despite the fact that it may
always factically persist.
We could point out its origin,
and we have done so in the third remark.
It lies in this insight's genesis
from the negation of another.

6. Now, to make a review of the foregoing.
The insight that was so significant
at the end of our first part
(that being was an absolutely
self-enclosed, living, and powerful esse)
was nothing other than merely
the factically fulfilled absolute insight
whose genesis we have realized here.
Indeed, it had come about factically
in just the way we have understood
here that it had to come about,
by abstracting from everything,
or factically negating all intuition.
Thus, our requirement at that time to abstract,
as regards validity, from the objectivity
that did not factically weaken as a result of intuition,
was completely correct and has now been proved.
)

We continue, and on good grounds
we stay with the standpoint originally held:
what seeing intuitively projects in this self-negation,
and has been seen into by us as intuitively projecting,
is being; and a powerful being at that.
However, being as such, or objectivity, is self-enclosure.
Moreover, a living self-enclosure (= living self-enclosing)
posits a principle of coming out of oneself,
which is annulled by the former.
Therefore, this principle is better named a “drive.”

And so we would have realized what we sought,
a description (liveliness and construction of certainty)
as a self-enclosure based simply on
the closer and deeper description of seeing,
of ideality, and of intuition as such.

In this description, four interrelated terms disclose themselves,
and this enumeration can clarify for you what has been presented.
First, [there is] seeing in its essence,
as absolute externality in relation to another,
thereby as self-negating, and, in this negation,
regarding this other as inwardly self-enclosed being.
This yields two terms.
Things would have rested with this pair of terms,
if we had not added that,
despite this partially valid self-negation,
seeing continues in the insight
into being's necessary arising,
and remains what it ineradicably is,
an inward externality.
Therefore, being must intrinsically carry
its characteristics of life and externality.
Pursue this a bit deeper.
Strictly speaking, here there is
a two-fold view of seeing,
from which there follows
a two-fold view of being;
or perhaps the other way around.
Thus, [there is] seeing, as intuition,
which therefore intuits itself,
from which mere dead being follows;
and [there is] seeing in its inward essence
as absolute externality,
whence follow the potency and life of being.
It is clear that, in being,
both should exist together completely in one,
a living self-enclosure,
or an intrinsic self-occlusion,
and therefore being must likewise
merge into knowing,
from which it has been deduced.
A self-negation resulting from
the construction of its essence as ____ ,
and a projection of this essence into being
as a result of its negation of itself.
Finally, the fourth term,
the principle of coming out of oneself
through the self-enclosure's liveliness and energy,
which is simultaneously
posited in its being
and negated in its effect;
this principle is immediately clear
and needs no further proof.

It is not difficult to add the fifth term,
which should belong here
according to historically well-known rules
so that a complete synthesis can arise.
That is to say, the arising and persistence
of these four terms in our insight depend
on the fact that one understands
seeing in its essence as externality,
as we have understood it,
and that one negates it as absolute immanence
(that is, as intrinsically valid)
in the face of the recognized being.
Still, [one must] observe it as itself
factically enduring above and within being.
This seemed to us a possible insight,
and, if it is brought about,
an absolutely evident one,
which in general we freely produced
and to which we were summoned.
Further, the continuance of these
four interconnected terms as a whole
(I say “as a whole” because if one is posited,
the others follow as well)
depends on this,
that within the insight that we have described
(and which is properly a description of certainty)
one preserves what appears there too as dependent on our freedom.

Enough about this. I commend it to your memory for future use.

1. Prevented by time from introducing new syntheses,
I will merely prepare you for them.
If being's living self-enclosing, which we mentioned,
really is literally a living enclosing,
as we express the matter,
then it posits an act, and indeed an absolute act qua act,
an effect of the drive to come out of itself.
Now it should annul this latter without resistance.
Both can very well be united if we consider that
this entire “livingness” is only a result
of seeing's inner expression and projection,
as such and as intuition.
This latter has been overthrown as valid in itself,
although factically it cannot be eradicated.
Therefore, the former can always exist
and remain in factical appearance,
while in truth the second does not occur in any way.
In this way, the unification of truth in itself
and its appearance first provides again
a true ground for the oneness of quality.
Previously, we assumed this only factically,
then lost it in the face of a higher truth,
and now have been working to reinstate it.
The drive for coming out of oneself
always emerges in appearance
and is struck down;
undoubtedly, it can form the appearance
of freedom and genesis themselves,
but it does not enter into truth.
Just so this drive provides
the true real ground for quantities.
Through the unification of these quantities
with the principle of quality,
we hope to arrive at a deduction of
the phenomenon of ordinary knowing's
primordial form.

2. This being, then, resides simply
in absolute seeing itself
and it is to be sought only there.
As much as in appearance,
it does so in the truth,
which can of course remain
inconceivable and indescribable
in its oneness.
[It remains so] not simply because
it lies outside of knowing,
which is the old fundamental error;
[it does so] rather because
absolute knowing itself is inconceivable,
and absolute conceiving is by no means
the same as absolute knowing.
According to the standpoint we have adopted, then,
where should we look for it?
This clearly confirmed itself
in the insight we completed previously.
For no reason, we saw very directly
on the basis of the matter itself
that seeing is an externality.
We perceived equally immediately
that this was clear to us,
or that we ourselves saw
in the very seeing of seeing,
and in the insight into the latter;
it was certain for us.
Yet, certainty remains to be described,
and it is described as an inner self-enclosing
against the principle of coming out of oneself.
It has come out of there, seeing sees itself,
coming out as coming out.
It sees itself just as externality;
it was enclosure.
Everything was one in the same oneness;
simply seeing into self,
which cannot be described further,
but can only be lived.
Hence, it was actually not just a re-description,
like the first one presented,
rather it is the primordial description,
authentic completion of the one certainty, or of knowing.
For the reasons given, we could later also
add to it being arising from negation,
which of course lies within knowing itself
as a part of its original description,
along with that which flows further
from that in the preceding synthesis.
But the last thing which flows from it was
the necessity that knowing hold and sustain itself
within its one unchangeable standpoint,
which now for the first time fully
points out knowing's independence,
which does not surprise us.
The one knowing would be completely enclosed
by means of this complete
synthesis in its formal oneness,
and by means of the multiplicity
which proceeds from its essence,
if this entire reflection did not appear
to be freely created and
(if we wish to pay attention to ourselves)
appear as objectified by us ourselves as was
just mentioned and as it consistently seems to us.
Objectivity and genesis are entirely one,
as we have seen previously.
However, genesis is this inner
externalization and principle of this oneness.
Hence, if our appearance is correct
(and it must of course hold good
and be explained as an appearance),
there must be yet another
specific manifesting principle,
to which everything just recapitulated is related
as to a oneness that contains it all.
It can only be derived from this
as the enclosing oneness,
and this, as enclosing oneness,
can only be derived from it.

III.55
taraka sarva-visaya sarvatha-visayam akramam ceti vivekajam jnanam

Twenty-seventh Lecture
Wednesday, June 6, 1804
Honored Guests:

Our continuing task is to show that,
if transcendental knowing
(= the existence of absolute knowing)
is to arise, then another,
ordinary knowing must be presupposed.

The permanent result of the last hour is this:
seeing permeating itself as seeing
necessarily surrenders itself
as something independent,
and posits an absolute being.
The latter enters into a further synthesis,
which yields a description of certainty
as an enclosure into itself.

I presented this as the center,
and did not fail to recall
that continuity of connection
had not been achieved
from there out to the endpoints,
but rather that gaps remained.
Today's project is to fill these gaps
from the midpoint outward,
according to absolute knowing
in its oneness.
Clearly, this requires new terms
that are not to be developed from the foregoing.
Meanwhile, this new investigation can
thus be tied to the previous one.
Undoubtedly, it was we, scientists of knowing,
who created this insight into the essence of seeing,
and therefore the other as well,
that seeing is necessarily a form of intuition, etc.
I also add that another entirely different insight,
which we have overlooked, is contained in
this mastering of seeing's essence,
and we will present it now.

1. Namely, I say:
if seeing is posited as seeing,
then it follows that seeing actually takes place;
or, seeing necessarily sees.
So that you do not overlook this sentence's importance
because of its simplicity,
I can justifiably make it more difficult,
by therefore adding the following preliminary remarks.
Evidently, this sentence is the completion
of what is required, but never achieved,
in the scholastic proof of God's existence
as the most real being
(a proof with which you are all familiar)
to infer the existence of a thing
from the bare thought of it.
Seeing posited as seeing means that
it is thought, formed and constructed
in its inner essence,
just as we have done in the last hour;
and this is done hypothetically,
as follows immediately from this construction,
without further ado.
Because, from positing what something is,
nothing at all is established regarding
its existence or non-existence,
and instead, it remains hypothetical.
As surely as this latter alone is posited,
and if “seeing-into” and certainty are enclosed in that,
nothing can be said about it definitively.
However, we reasoned as follows:
“Seeing necessarily occurs,
and so its existence is
positively established and expressed.”
This is the first point.
Thus, we genetically derive being there,
the true inner essence of existence.
Providing this derivation has been our task,
and indeed, the sole immediately derivable
existence is that of seeing.
This is the second point.

2. For either the proof in its full significance
or the insight intended here to arise
requires strenuous attention.
For instance, one could well say life lives,
which is of course correct;
but whether life is situated in seeing,
completely as such,
as we have most intimately intuited it in ourselves,
that is the question.
I handle it like this:
seeing's self-permeation is
an absolute negation of itself as independent,
and a [matter of] relating itself to something external.
It exists only in this self-negation and relating,
and otherwise not.
Yet this negating and relating is
an act that exists just in itself
and in its immediate completion.
Hence, it is necessary, immediate, and actual;
and if the whole is to be,
then it must be and exist.
Seeing cannot be posited,
except as immediately living, powerful, and active.

3. The insight that we have just completed is
now the absolute insight of reason,
that is, absolute reason itself.
In this insight, we have
immediately become absolute reason,
and have dissolved into it.
Let us analyze this insight more closely,
and with it reason as well.
First, reflect purely on the last thing,
the insight itself,
abstracting from the way we came to it:
it itself is a seeing, insight,
thus something external.
It is a seeing of seeing,
thus seeing itself as existent,
absolute and substantial,
a seeing grounded in itself
and as such self-expressive,
[a matter of] certainty and self-enclosure
by virtue of inner necessity.
Thus, as a result of its absolute inner essence,
it is a light that determines and declares itself,
that simply by itself absolutely cannot not be,
and absolutely cannot not be what it is,
just self-positing.
Self-positing, I say, but not as itself.
We have added the last in a way that must now be explained.
Reason's absolute insight sees,
just the seeing under discussion,
but not immediately again its first-named seeing,
because this is an absolute insight.
Proposition: the absolute insight of reason
brings absolute existence
(of seeing, in fact) with itself.
It does so immediately [in the course of]
performing this action,
and as the expression of doing so.
Said differently, absolute reason
permeates itself as reason,
in exactly the way we have indicated,
as absolute effect.
The proposition is important:
I think [it] is immediately clear,
and if for this one or that one it is not,
I can help in the following way.
In the previous hour,
formal seeing permeated itself
as something absolutely outward,
negated itself therein,
and thereby posited being;
not immediately, but rather by means of ____ , etc.
It is otherwise here in absolute reason;
this happens without any mediation.
Thereby, reason shows itself in us, as reason's reason,
thus as absolute reason in this self-permeation
and being permeated by itself.
A doctrine of reason,
through itself, from itself, and in itself,
as this duality can be expressed.
A doctrine of reason,
as the first and highest part
of the science of knowing,
which does not become,
but rather is unconditionally in itself,
and is that which it is.

Let's go to the other side of this insight,
that is to the genetic aspect.
On the condition that seeing has been
permeated in its inner essence,
reason posits absolute existence,
and permeates itself as positing.
I ask: were we intended to assume a seeing
and a permeatedness of this seeing,
apart from what we have here in reason itself
and on its basis?
I hope not, because then we would have two absolutes,
which certainly could not be brought together.
Thus, this itself is the expression of reason,
grounded in reason itself,
occurring in order to arrive at its result of
positing being and permeating itself in this positing.
Therefore, reason is intrinsically genetic;
and it is intrinsically completely consolidated,
necessary, and lawful in unchanging oneness.
However, it is genetic insofar as
it is actually and truly living,
and expresses itself actively.
Now, however, it exists necessarily and cannot not be,
because it is absolutely self-directing existence
and authentic existence.
Therefore, it is inwardly and secretly
what it asserts outwardly about external seeing,
and its inwardly grounded existence and life
consists in just this [act of] speaking about seeing.
But further:

4. [To say] that reason itself is
the ground of its own proper existence,
here inward, living, and active,
would mean that it posits its life and existence
unconditionally and inseparably from itself.
No living and existence is possible apart from this one,
and it is impossible to escape it.
Now, however, although we say that it cannot be escaped,
we ourselves have obviously done so.
Consequently, reason's genetic life,
which we have described in
positing seeing's absolute existence
in case this latter is merely thought,
is still not reason's primordial
and absolute existence.
That it is not this,
also follows for another reason.
We appear to ourselves as
having completed the very insight
just completed, with freedom.
We indeed are reason,
because reason is simply the I,
and cannot be anything else than I,
according to the proof conducted long since.
Thus “we seem,” or “reason seems,”
are completely synonymous.
Now we appeared here actually as
the ground of the insight's existence,
but not as absolute ground of its actuality,
but rather as the ground of its possibility,
or its free ground.
Thus, we are only a mediate,
factical appearance and insight,
and hence not absolute reason.
This mere possibility is
also evident in another way.
Evidently, this entire insight appears as
the reconstruction of an original construction in reason,
and we objectify this primordial reason,
along with its construction.
We appear to ourselves as yielding ourselves
to the original law of reason by a free act,
and now, gripped by it, as made into manifestness = certainty.
This surrender and this manifestness appear to us
as able to be repeated indefinitely.
Further, the following in particular
should be noted and made more precise.
The premise for the absolutely completed,
rational insight into seeing's inner essence
as something absolutely external
(despite the fact that we do not
let it serve as an absolute premise,
but have instead derived it from pure reason itself)
is still an intrinsic genetic power
for the rational insight into existence,
which has been brought about.
It conditions the latter just as
the latter has conditioned it.
Hence once again the circle
between reciprocal conditionings remains,
and thus the old hypothetical character.
The absolute condition has by no means been disclosed.
That hypotheticalness remains can also be shown in another way,
and it is interesting and instructive to give this demonstration.
“If seeing is, then thus and so are as well,
and from this follows ____ , etc.”
Precisely, therefore, what lies in the conclusion is
presupposed hypothetically in the premise,
and so it holds only on condition that the first is given.
Now, to be sure, we say that absolute reason posits it;
but it is only we who say it, that is, arbitrariness and freedom.
Of course, reason speaks in the connection;
but we have provided it with speech in the first place,
and should not trust it.
Reason itself must start talking directly.

5. In what we have achieved so far,
we have at least learned where this might be discovered.
“Reason itself is immediately and unconditionally
the ground of an existence, indeed of its own existence,
since it cannot be of any other.”
This means:
this existence has no further ground.
It is not possible, as before, to provide a genetic premise,
on the basis of which it can be explained further,
since otherwise it would not be grounded in absolute reason,
but in a reason that would have to be interrogated in its turn.
Instead, we must simply say, it is grounded in reason.
[It is] a pure absolute fact.
Now just what is this fact?
It has accompanied us all along,
and thus in this last investigation as well.
It is expressed in the formula:
“Reason is the absolute ground of its own existence.”
If anyone does not yet see it,
this is because it is too close to him.
We ourselves have continually objectified reason,
and therefore posited its existence, as existence,
in the “form of outer existence.”
Now we are reason.
Therefore, in us ourselves, that is in itself,
absolute reason is the absolute ground of its existence,
that is, of its existence as such.
This is a fact from which we cannot ever escape,
and that cannot be explained or understood
from any further genetic premise,
as must be the case if it is to be
an expression of absolute reason.
(Someone has encountered the enduring
objectification of the absolute,
and has opined that therefore there is no absolute.
How could that be, if, as is the case here,
true absoluteness is not found in what is objectified,
nor in the objectifying [agent],
but rather in the immediate event of objectifying itself?
From time to time, one sufficiently remembers and makes clear
the fact that the absolute must not be sought outside the absolute,
and especially that we would certainly never grasp the absolute,
if we did not even live and conduct it.)

6. Now in this case we are not just the bare fact,
but instead we are simultaneously
the insight that this fact is
reason's pure original expression and life;
that the fact is origin and the origin, fact.
The characteristic mark of the science of knowing
consists purely in this synthesis.
The bare fact, as fact, yields ordinary knowing.
So initially, this self-presentation of reason
itself occurs through reason, we say:
for us, then, how is the latter reason?
I think, if we view it properly,
[it is] objectified in the same way as the former,
and undoubtedly the same one reason as the former.
Thus our life, or the life of reason,
does not escape from its own self-objectification.
And that which is one implicitly
(and probably for ordinary knowing as well)
divides itself within the science of knowing's
insight into a twofold appearance.
Indeed, this division arises only from the insight,
or the genesis of that which in itself is an absolute fact.
This is the first point.
Further, and most significantly,
we have not just found this insight
(that reason alone is the absolute ground of its existence)
immediately in our proposition.
We could not have done so.
This insight alone is what has
turned us into scientists of knowing.
If we had, the highest fact would have required
another genetic premise above itself;
and this in turn another, and so on to infinity.
Then our system would have met the fate
of other systems and found no beginning.
Instead of that, we have found it
in the more profound insight
carried out at the beginning of today's lecture.
Now it is merely applied as
a universal and absolutely valid proposition.
Hence, the absolute insight,
on which everything depends here,
our science as well, is possible only through a deeper one,
which likewise is possible only through another, etc.
In this way, the genuine ground for the connection
between knowing's various basic determinations is
opened up and made accessible.
The deduction can start from the principle:
if the insight obtained in this way,
the science of knowing is possible in its principles,
then must ____ , etc.
This was the second point.
Finally, reason is the ground of its own existence as reason.
For, note well, this and nothing else was
the absolute fact with which we are concerned.
However, it is so only in the insight we have created,
because it is duplicated (is reason as reason)
only within this insight.
Therefore, this very insight,
or the science of knowing, is
reason's immediate expression and life:
the single life of reason unfolded
immediately in itself and permeated by itself.
Precisely because, just as one
lives in it there is seeing,
or it appears, thus reason itself lives and appears.
In its existence, the insight appears
to be possible only by means of freedom;
it is also thus actually and in fact.
That is, in this way reason shows itself
as freely self-expressing.
That freedom should appear is
simply its law and inward essence.
If it appears as conditioned, then it actually is so;
that is, it must appear thus, as it is conditioned, etc.
This is all unconditionally true and certain,
but [it is] true and certain to the exact extent
that it lies within reason
as its necessary appearance and expression.
By no means is it as certain as reason in itself,
to which it in no way comes, except in its expression.
The task on which the truthfulness and certainty
of our insight depends is simply this:
to see everything in its context,
and to articulate it in this context.
Appearing truly exists
when it is conceived
as the absolute appearing
and self-expression of absolute reason,
and without the latter qualification,
it is not true.
It is true that reason appears thus and so,
e.g., as inwardly free,
only insofar as it also appears
as inwardly necessary and actually existing.
Without this qualification, it is not true, etc.

Absolute appearance, or genesis,
has been presented.
The law for deriving it,
and for deducing inferences from it,
has also been presented,
and the deduction can proceed.
With this, I stop at the border
of a philosophia prima
(I would have wished to regard
these lectures as being this),
presenting only appearance's
first basic distinction,
which in its oneness constitutes
the concept of pure appearing as such,
without any further determination of the latter.
In this work, I can either go to work on the details
(in which case I will not end this week),
or I can present the main point
briefly and forcefully
in a single lecture.
To do the latter, I will need more time
for the preparation of this lecture.
I prefer the latter, and ask you to
permit me to defer tomorrow's lecture,
so that I can end on Friday.

III.56
sattva-purusayo suddhi-samye kaivalyam

Twenty-eighth Lecture
Friday, June 8, 1804
Honored Guests:

Here is where we stood:
reason is the unconditional ground of its own existence
and its own objectivity for itself,
and its primordial life consists just in this.

I assume that this intrinsically dull sentence has
become familiar to you in all its importance
in the preceding lectures, especially the last.
Without doubt, we see this,
ourselves intuiting and objectifying reason
(as the logical subject of the sentence,
which according to us should now
further objectify itself)
as its predicate.
Numerically reason occurs here twice,
once in us and once outside of us.
I ask, which one is the sole absolute?
That is, does our projection perchance result
from the primordial projection of the external reason,
about which we have been speaking,
so that we ourselves said what it is that we were?
Or is the reason projected outwardly the result
of its own immediate self-projection in us?
In a word, reason exists in duality,
as subject and object,
both as absolute.
This ambiguity must be removed.

1. We can ground this entire existence
most effectively with the formula
already previously used and proven:
reason makes itself unconditionally intuiting.
I say “it makes itself,” and not “it is intuiting.”
It positively does not arise with intuition's being,
and to the extent that it appears to arise there,
we must abstract from it.
I say, “it makes itself unconditionally”;
as we express ourselves meanwhile:
this making is in no way accidental for it.
Instead, it is entirely and completely necessary.
If its being is posited, the latter [intuiting]
is posited as well, and being unfolds in it.
This is its proper, immediate, and inseparable effect.
(All these propositions are easy,
if one just takes them up
in the strength of energetic thinking.
Taken weakly and inconsistently,
they are confusing and lead to nothing.)

NB: pure, intrinsically clear and transparent
intellectual activity consists in this
“making itself unconditionally intuitive”
in genuine vivacity and existence.

It is raised above all objectifying intuition
as itself the latter's ground,
and it completely fills the gap
between subject and object,
thereby negating both.
Further:
This self-making is just effectiveness,
inner life and activity;
indeed, the activity of self-making, [is]
thus a making oneself into activity.
Here arise at one and the same time
an absolute primordial activity and movement,
and also a making (or copying) of
this primordial activity as its image.
(The former fundamentally explains
the self-making and presenting manifestness
that has grasped us in every investigation;
the latter explains our reconstruction
of this [manifestness]
as it has also consistently appeared to us.)
Now, though, we have to stand
neither in one nor the other,
but rather in the midpoint between both,
that is in the absolutely inwardly
effective self-making,
real through itself,
without any other making or intuition.
We must abstract from the fact
that we go on to objectify
this midpoint as well,
and should in no way admit
the validity of that intuition.
Because otherwise we have in no way
explained the subjective and the objective,
but only added it factically.
This was the first point.
Reason is this, as certainly as it is:
but it is unconditionally.
Now, reason is an absolute, immediate, self-making;
thus, it sets itself down as existing, objectively,
and as making itself, objectively.
The permanent object and the permanent subject,
which initially come into question;
neither one [exists] by way of the other,
as we initially thought,
but instead both [exist] by means of
the same original essential effect in the midpoint.
This was the second point.
Now, the effect through which
it [reason] throws down a permanent object is
the same one through which it threw down objective living,
and therefore the primordial construction,
objective reason, devolves onto this objective life.
In addition, the effect through which
it threw down the persisting subject is
the same one through which it threw down imaging as imaging,
and so this imaging devolves onto the subject.
This was the third point.
Result:
reason, as an immediate, internal, self-intuiting making,
and to that extent an absolute oneness of its effects,
breaks down within the living of this making
into being and making;
into the making of being as made and not made;
and into the [making of] making
as likewise primordial, existing,
and not primordial, copied.
This disjunction, expressed just as we have expressed it,
is what is absolutely original.

2. Now, in the business we have just concluded,
we indeed have objectified and intuited the one reason
as the inwardly self-intuiting making;
but we realize that we must abstract from this,
if we want to recognize reason as the one.
We are also aware that we can indeed abstract from it.
That is, although we are not able
to exempt ourselves from it factically,
we can indeed think of it as not valid in itself.
In this way, reason is actually seen into.
That is, as the primordial self-making,
it fully merges into our imitative image.
Therefore, it is the very same relation
immediately in us that we have just presented objectively.
We, or reason, no longer stand
either in that objective reason
nor in subjective reason;
since we must just abstract from both
instead [we, or reason, stand] purely in
the midpoint of the absolutely effective self-making.
In the science of knowing,
reason is completely, immediately alive,
opened in itself,
and has become an inward “I,”
[both] periphery and center,
(since this is the site of the science of knowing)
as it had to come;
and all this has happened
completely through abstraction.
Absolute reason is absolute
(accomplished) thinking of oneself.
Thinking oneself as such, is reason.
I can then objectify this
very absoluteness of reason
(or the science of knowing),
produced in this way,
or I can not objectify it,
because I abstract from the objectivity,
which factically ceases without my help.
If I do the latter,
then everything ends here,
and reason is enclosed in itself.
If I abandon myself to the former,
then I give myself up to a mere fact
that is completely cut off
and without any principle.
It does not arise out of reason,
since only those things derived here arise from it;
just as little does it arise from something else,
since it is simply absolutely factically given.
Therefore, it is completely inconceivable,
without a principle;
and, once surrendered to it,
nothing remains for me
except just to abandon myself to it.
It is pure, simple appearance.

3. I want to surrender to it.
Now if I objectify this absoluteness,
then it appears to me as an objective situation,
to which I raise myself through free abstraction
from my initially objectifying reason.
“I want to surrender,”  I say;
hence, within appearance I find
an objectively complete “I”
as the original ground and original condition
for this situation.
(Consciousness, and self-consciousness,
as the original fact underlying all other facticity.
The thing whose validity has so far been completely denied [is]
once more derived and justified.)
From the preceding knowledge of reason,
I know very well what the “I” is in itself.
I need not to learn it from appearance,
which would not give me any information about its essence.
It is the result of reason's self-making.
Consequently
(a very important conclusion,
and the only possible one if appearance itself is
to be traced back to reason)
appearance itself is inaccessible only to me,
to the “I” projected absolutely through appearance;
it itself is a self-making of reason,
of reason's primordial effect,
and indeed as an I.
(The imprint of reason's effect
rests solely in the I, as such).

However, a. that I must simply abstract,
if this consciousness is to arise,
and b. that I could either do this or not,
are found purely in appearance, that is,
in reason's effect,
which is inaccessible to me
only in its principle.
Thus, that I am free.

4. What arises through abstraction for me,
as a result of the assertion of appearance?
Reason as absolute oneness:
this arises, and appears as arising.
However, all arising appears as such
only with its opposite.
The opposite of absolute oneness,
which in this oppositional relationship
in turn becomes qualitative oneness,
is absolute multiplicity and variability.
Therefore, if this oneness is to appear genetically,
then consciousness,
from which one must abstract as the beginning point,
must itself appear as absolutely changeable and multiform.
This is the first fundamental principle.
It is also the first application of our basic principle:
if the science of knowing is to arise
(if the science of knowing is to be
precisely an arising, an origin)
then such and such a consciousness must be posited.

5. The I of consciousness in appearance is
an inconceivable effect of reason,
above all materially.
This inconceivability as such
enters immediately into
the primordial consciousness,
which genesis presupposes,
which is unconditionally changeable,
and which exhausts itself in infinite multiplicity.
It does so explicitly as inconceivable, as real.
Reality in appearance is
the same as the primordial effect of reason:
the one eternally selfsame.

6. The I of consciousness is reason's effect
in the conceptual form of this effect,
which we conceived earlier in the four presented terms.
So far, we have presented these four terms
collectively only insofar as we have
penetrated reason as an inner oneness.
As a genetic abstraction, [this oneness] presupposes
external oneness just as before,
and as a primordial effect of reason within experience,
[it] presupposes inner multiplicity.
However, the inner multiplicity and separateness must
necessarily consist in the lack of correlation
in realizing these four terms,
thus in their apprehension as separate principles.
Then, in an absolutely necessary differentiation of principles,
we would have four fundamental principles:

1. In the enduring object,
and indeed in what is absolutely transient:
the principle of sensibility, belief in nature, materialism.

2. In the enduring subject:
belief in personality,
and, given the variety of personalities,
the identity and equality of personality,
the principle of legality.

3. Holding to the absolutely
real forming of the subject,
which, since this forming is
connected to the enduring subject,
conceptually makes the latter into a oneness,
leaving multiplicity just to the former.
[This is] the standpoint of morality,
as an activity that proceeds purely
from the enduring I of consciousness,
continuing through infinite time.

4. Holding to the absolute
imaging and living of the absolute object,
which becomes a oneness
for the same reason introduced in 3. above.
The standpoint of religion,
as belief in a God,
who alone is true for all lifetimes,
and alone is inwardly living.

Now all of these standpoints are effects of reason,
although not realized as such in their principle,
and reason is completely as it is, wherever it is.
Therefore, it is evident that, simply as effects of reason,
the three other standpoints insert themselves
immediately into each of the four,
without any apparent free act of abstraction.
Yet they do so colored by,
and in the spirit of,
the dominant principle.
Thus, among religious people morality exists as well,
but not, as with those having the latter as their principle,
as their proper work.
Instead, it is a divine work in them,
which brings about in them will and fulfillment,
as well as the resulting desire and pleasure.
Other people also exist besides themselves, and a sensible world.
Yet, they do so always only as an emanation of one divine life.
Likewise, too, God exists in morality as a principle,
but not for his own sake;
instead, so that he maintains the moral law.
If they had no moral law, they would not need a God.
Further, other people exist apart from themselves,
but only just so that they can be, or become, moral,
and they have a sensible world simply as a sphere for moral action.
Likewise, God is present for the legal standpoint,
but he exists only in order to manage the
higher police work that lies beyond the reach of human policing.
Legality also has a morality;
however, it coincides with outward integrity
in relation to others, and ends there.
Finally [there is] a sensible world,
for the purpose of civic industry.
Fourth, God surely has a place within the principle of sensibility,
if it is left to itself and does not become
unnaturally depraved through perverse speculation.
That is, God exists so that he can give us food in due season.
With this comes a certain morality,
namely to spread one's pleasures wisely,
that one always has something more to enjoy,
especially not to ruin things
with this God who provides.
Finally, there is also something analogous
to reason and spirituality
and that is also to enjoy these pleasures
in the right order and with prudence.
Hence, there are four basic factors in each standpoint
(five if you also add the unifying principle).
Together these yield twenty main factors
and primordial fundamental determinations of knowing;
and, if you add the science of knowing's
indicated fivefold synthesis,
which we have just completed,
this becomes twenty-five.

It has already been proved that
this division into twenty-five forms
coincides with the absolute breakdown of the real,
or [with the breakdown] into absolute multiplicity
of that effect of reason,
which is immediately inaccessible in its oneness.
This follows because multiplicity in general
arises out of the genetic nature of reflection on oneness.
However, this reflection on oneness
immediately breaks down into five-foldness.
Therefore, the manifold from which it is necessary
to abstract breaks down in the form of fivefoldness,
by the same rule of reason.

Above, we have made it immediately clear factically that
this entire consciousness is posited and determined
solely by the genesis of the science of knowing.
If we abstracted from our rational insight,
as a condition objectifying itself immediately in us,
then things would rest there,
and we would not attain anything further.
The insight into the law of all consciousness
arose for us only insofar as we reflected on it,
and thus posited the science of knowing as genesis,
as something that ought to arise,
because of our absolute decree.

The task we had assumed is therefore completely finished,
and our science has closed.
The principles have been presented
with the greatest possible clarity and determinateness.
Anyone who has truly understood and penetrated
the principles can carry out the schematism on his own.
Making many words does not contribute to clarity;
it can even obscure things for the best minds.
Perhaps there will be time and opportunity
this coming winter for applying these principles
to particular standpoints,
for example to religion,
which always should remain the highest,
not only in the partiality and sensible form
in which it was grasped previously,
but in our science's inherent spirit,
and from there to the doctrine
of virtue, and of rights.

Until then, I ask for your benevolent
remembrance for myself and for science,
and I give you my thanks for the new courage
and the new happy prospects for science,
which you have given me so amply
in the course of these lectures.
