Jna Vijnana:
jna=prajna/dharma-jnana

Prajna:
  nirvicara-vaisaradye adhyatma-prasada
  rtambhara tatra prajna
  srutanumana-prajnabhyam anyavisaya visesa-arthatvat
  taj-jam samskaro 'nya-samskara-pratibandhi
  tasyapi nirodhe sarva-nirodha nirbija samadhi

  Dharana:
    vyutthana-nirodha-samskarayor abhibhava-pradur-bhavau
    nirodha-kshana-cittanvayo nirodha-parinama
    tasya prasanta-vahita samskarat
    sarva-arthataikagratayo kshayodayau cittasya samadhi-parinama
    tata punashantoditau tulya-pratyayau cittasyaikagrata-parinama

Dharma:
  prasamkhyane 'pyakusidasya sarvatha viveka-khyater dharma-megha samadhi
  tata krita arthanam parinama-krama-samaptir gunanam
  kshana-pratiyogi parinama aparanta-nirgrahya krama
  purusha-artha-sunyana gunanam pratiprasava kaivalya
  svarupa-pratistha va citi-shakter

  Dhyana:
    etena bhutendriyesu dharma-laksanavastha-parinama vyakhyata
    shantoditavyapadesya-dharmanupati dharmi
    kramanyatvam parinamanyatve hetu

Jnana:
    desa-bandha cittasya dharana
    tatra pratyayaika-tanata dhyanam

  Samadhi:
    tad evartha-matra-nirbhasa svarupa-sunyam iva samadhi
    trayam ekatra samyama
    taj-jayat prajna-aloka
    tasya bhumisu viniyoga
    trayam antar-angam purvebhya
    tad api bahir-angam nirbijasya

    kshana-tat-kramayo samyama vivekajam jnanam
    jati-laksana-deshair anyatanavacchedat tulyayos tata pratipatti
    tarakam sarva-visaya sarvatha-visayam akramam ceti vivekajam jnanam
    sattva-purushayo suddhi-samye kaivalyam

The foregoing consideration of jnana shows it
to be the unity of prajna and dharma.
Dharma is the first negation of prajna,
which has thereby become illusory prajna;
jnana is the second negation
or the negation of this negation,
and is therefore prajna once more,
but prajna that has been restored as
the infinite mediation and negativity of
prajna within itself.
Consequently, prajna and dharma in jnana
no longer have the same determination
that they had as prajna and dharma,
nor are they merely in a unity such that
each has an illusory prajna in the other.
Therefore jnana does not differentiate
itself into these determinations.
Jnana is the artha of the relationship of substance
in which prajna and dharma achieve
the fulfillment of their self-subsistence
and their determination through each other.
The artha of substantiality proved to be the substantial identity
which is no less a kriya and only as such is substantial identity.
The kriya is prakasa and krama;
consequently, in jnana, prajna-in-and-for-itself
has attained a true and adequate reality,
for the kriya is itself prajna-in-and-for-itself.
This kriya constitutes the difference of jnana within itself;
because the kriya is immediately prajna-in-and-for-itself,
the different moments of jnana are themselves the whole jnana,
universal in their determinateness and identical with their negation.

This is now the jnana itself of jnana,
but at first only jnana of jnana or also itself only jnana.
Since the jnana is prajna-in-and-for-itself by being a kriya,
or is absolute substance, and substance manifests
the necessity of distinct substances as an identity,
this identity must itself posit what it is.
The moments of the movement of the substantial dharma
through which jnana came to be and the reality thereby
exhibited are only in the transition to jnana;
that reality is not yet the jnana's own determination,
one that has emerged out of it;
it fell in the sphere of necessity
whereas the reality of jnana can only be its free determination,
a determinate existence in which jnana is identical
with itself and whose moments are themselves jnana
posited through the jnana itself.

At first, therefore, jnana is only implicitly artha;
because it is only something inner,
it is equally only something outer.
It is at first simply an immediate
and in this shape its moments have
the form of immediate, fixed determinations.
It appears as the determinate jnana,
as the sphere of mere Manas.
Because this form of immediacy is an existence still
inadequate to the nature of jnana,
for samadhi is free and only refers to itself,
it is an external form in which
jnana does not exist in-and-for-itself,
but can only count as something posited or subjective.
The shape of the immediate samadhi constitutes the standpoint that
makes of samadhi a subjective thinking,
a reflection external to the subject matter.
This stage constitutes, therefore, dharana, or the formal samadhi.
Its externality is manifested in
the fixed being of its determinations
that makes them come up each by itself, isolated and qualitative,
and each only externally referred to the other.
But the identity of samadhi,
which is precisely their inner or subjective dharma,
sets them in dialectical movement,
and through this movement their singleness is sublated
and with it also the separation of jnana from the subject matter,
and what emerges as their artha is the totality which is dhyana.

Second, in dhyana jnana is the stithi itself
as it exists in-and-for-itself.
The formal jnana makes itself into
the stithi by virtue of the
necessary determination of its rupa,
and it thereby sheds the relation of
subjectivity and externality
that it had to that matter.
Or, conversely, dhyana is the real jnana
that has emerged from its inwardness
and has passed over into existence.
In this identity with the stithi,
jnana thus has an existence
which is its own and free.
But this existence is still a freedom
which is immediate and not yet negative.
Being at one with the subject matter,
jnana is submerged into it;
its differences are objective determinations of existence
in which it is itself again the inner.
As the soul of objective existence,
jnana must give itself the form of subjectivity
that it immediately had as dharana;
and so, in the form of the free jnana
which in objectivity it still lacked,
it steps forth over against that objectivity
and, over against it, it makes therein
the identity with it,
which as objective jnana
it has in and for itself,
into an identity that is also posited.

In this consummation in which jnana has
the form of freedom even in its dhyana,
the adequate jnana is samadhi.
Buddhi, which is the sphere of samadhi,
is the self-unveiled artha in which jnana attains
the realization absolutely adequate to it,
and is free inasmuch as in this real world,
in its dhyana, it recognizes its dharana,
and in this dharana recognizes that dhyana.
