samapatti

Scientific Knowledge vs Craft Knowledge
Jnana vs Jaya

No perceptual capacity constitutes prajna
    are in control of particular knowledge

scientific knowledge of the causes that are starting-points

question of movement

jnana of the truth

everyone has something to say about the nature of things
grasping of the whole while incapable of grasping the particular

since difficulties occur in two ways:

the problem is in us

since our eyes cannot see what is by nature most evident of all

philosophy is scientific knowledge of the truth
for theoretical science the end is truth
we do not know the truth without its cause
for practical science the end is function
relation not to something instinsic but to something and now

each attribute belongs to something when it is because of it
that a synonymous attribute also belongs to other things


the starting-points of eternal beings must be most true

eternal beings: 
not sometimes true
no cause of their being
but a cause of others being
so that as each thing is as it regards being,
so it is too regards truth

There is some starting-point:
1. not an unlimited series
2. not unlimited in kind

[1] for neither is it possible for one thing to come from another thing
as matter without limit nor is it possible for the starting-point
from which the movement comes to be such

[2] for-the-sake-of-which cannot go on without limit
and the case of the essence is the same way

Series cannot go downward without limit
With a starting-point above

One thing comes from another in two ways

It is not possible for the first cause to pass away
since it is eternal

The first thing from whose passing away something came to be must be non-eternal
since coming-to-be is not without limit in the upward direction

since the for-the-sake-of-which is an end,
and the sort of end that is not for the sake of other things
but rather other things are for its sake


The essence cannot be referred back to another definition
that is fuller in account

It is impossible to have knowledge until we come to the indivisibles

In traversing what is unlimited we cannot count the cuts
The whole line must be understood by something that does not move from cut to cuts

There can be nothing that is unlimited
or the defnition of what is unlimited is not unlimited


of medial things:

It is absurd to look for scientific knowledge 
and the way of inquiry characteristic of scientific knowledge

[2] Unlimited in kind
If the kinds of causes were unlimited in number
knowledge would again be impossible
since we think we know when we know the causes
but what is unknowable by addition cannot be gone through in a limited time


all being participate in the bad except the one itself

two-making

impossible things:
positing the coming-to-be of the eternal
immovable

for in each category of being there is something analogous
