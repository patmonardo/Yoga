Hegel on Sankhya-Yoga

Sankhya and likewise the other Indian systems of Philosophy,
occupy themselves particularly with the three qualities (Guna)
of the Absolute Idea, which are represented as substances
and as modifications of nature.

It is noteworthy that in the observing consciousness of the Indians
it struck them that what is true and in and for itself
contains three determinations,
and the Notion of the Idea is perfected in three moments.

This sublime consciousness of the trinity,
which we find again in Plato and others,
then went astray in the region of thinking contemplation
and retains its place only in Religion,
and there but as a Beyond.

Then the understanding penetrated through it,
declaring it to be senseless;
and it was Kant who broke open the road once more
to its comprehension.

The reality and totality of the Notion of everything,
considered in its substance, is absorbed
by the triad of determinations;
and it has become the business of our times
to bring this to consciousness.
