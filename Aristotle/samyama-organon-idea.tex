CHAPTER 27

The more abstract science is superior to the less abstract

One science is more precise than and prior to another,

(1) if the first studies both the fact and the reason,
the second only the fact;

(2) if the first studies what is not,
and the second what is, embodied in a subject-matter;

    Thus arithmetic is prior to harmonics

(3) if the first studies simpler and
the second more complex entities.

    Thus arithmetic is prior to geometry,
    the unit being substance without position,
    the point substance with position.

CHAPTER 28

What constitutes the unity of a science

A single science is one that is concerned with a single genus,
with all things composed of the primary elements of the genus and
being parts of the subject, or essential properties of such parts.

Two sciences are different if their first principles are not derived
from the same origin, nor those of the one from those of the other.
The unity of a science is verified when we reach the indemonstrables;
for they must be in the same genus as the conclusions from them;
and the homogeneity of the first principles can in turn be verified
by that of the conclusions.

CHAPTER 29

How there may be several demonstrations of one connection

There may be several proofs of the same proposition,

(1) if we take a premiss linking an extreme term with a middle term
not next to it in the chain;

(2) if we take middle terms from different chains.

    Pleasure is a kind of change because it is a movement,
    and also because it is a coming to rest.
    But in such a case one middle term cannot be universally deniable
    of the other, since both are predicable of the same thing.
    This problem should be considered in the other two figures as well.

CHAPTER 30

Chance conjunctions are not demonstrable

There cannot be demonstrative knowledge of a chance event;
for such an event is neither necessary nor usual,
while every syllogism proceeds from necessary or usual premisses, and
therefore has necessary or usual conclusions.

CHAPTER 31

There can be no demonstration through sense-perception

It is impossible to have scientific knowledge by perception.

For even if perception is of a such and not of a mere this,
still what we perceive must be a this here now.
For this reason a universal cannot be perceived, and,
since demonstrations are universal, there cannot be science by perception.

    Even if it had been possible to perceive that
    the angles of a triangle equal two right angles,
    we should still have sought for proof of this.
    So even if we had been on the moon and seen
    the earth cutting off the sun's light from the moon,
    we should not have known the cause of eclipse.

    Still, as a result of seeing this happen often we should
    have hunted for the universal and acquired demonstration;
    for the universal becomes clear from a plurality of particulars.
    The universal is valuable because it shows the cause, and
    therefore universal knowledge is more valuable than
    perception or intuitive knowledge,
    with regard to facts that have causes other than themselves;
    with regard to primary truths a different account must be given.

    Thus you cannot know a demonstrable fact by perception,
    unless one means by perception just demonstrative knowledge.
    Yet certain gaps in our knowledge are traceable to gaps in our perception.
    For there are things which if we had seen them we
    should not have had to inquire about,
    not that seeing constitutes knowing,
    but because we should have got the universal as a result of seeing.
    For instance, if we saw glass perforated, and the light
    passed through it, it would be also manifest why it illuminates
    in consequence of our seeing separately in each,
    and at the same time perceiving that it is thus will all.

CHAPTER 32

All syllogisms cannot have the same first principles

That the starting-points of all syllogisms are not the
same can be seen (1) by dialectical arguments.
(a) Some syllogisms are true, others false.
A true conclusion may indeed be got from false premisses,
but that happens only once.
 A may be true of
C though A is untrue of Band B of C. But if we take premisses
to justify these premisses, these will be false. because a false
conclusion can only come from false premisses; and false pre-
misses are distinct from true premisses.
27. (b) Even false conclusions do not come from the same
premisses; for there are false propositions that are contrary or
incompatible.
30. Our thesis may be proved (2) from the principles we have
laid doWIl. (a) Not even all true syllogisms have the same starting-
points. The starting-points of many true syllogisms are different
in kind, and not applicable to things of another kind (e.g. those
concerning units are not applicable to points). They would have
either to be inserted between the extreme terms, or above the
highest or below the lowest, or some would be inside and some
outside.
36. (b) Nor can there be any of the common principles, from
which everything will be proved; for the genera of things are
different. and some principles apply only to quantities, others
only to qualities, and these are used along with the common
principles to prove the conclusion.
b 3 . (c) The principles needed to prove conclusions are not
much fewer than the conclusions; for the principles are the pre-
misses, and premisses illvolve either the addition of a term from
outside or the interpolation of one.
6. (d) The conclusions are infinite in number, but the terms
supposed to be available are finite.
7. (e) Some principles are true of necessity, others are con-
1.ingent.
9. It is clear, then, that. the conclusions being infinite, the
principles cannot be a finite number of identical principles. Let
us consider other interpretations of the thesis. (I) If it is meant
that precisely these principles are principles of geometry, these
of arithmetic, these of medicine, this is just to say that the sciences
have their principles; to call the principles identical because
they are self-identical would be absurd, for at that rate all things
would be identical.-1. 32. 880I9-3I
60I
IS. Nor (2) does the claim mean that it is from all the principles
taken together that anything is proved. That would be too naIve;
for this is not so in the manifest proofs of mathematics, nor is it
possible in analysis, since it is immediate premisses that are the
principles, and a new conclusion requires the taking in of a new
immediate premiss.
20. (3) If it be said that the first immediate premisses are
the principles, we reply that there is one such peculiar to each
genus.
21. (4) If it is not the case that any conclusion requires all the
principles, nor that each science has entirely different principles,
the possibility remains that the principles of all facts are alike
in kind, but that different conclusions require different premisses.
But this is not the case; for we have shown that the principles
of things different in kind are themselves different in kind. For
principles are of two sorts, those that are premisses of demonstra-
tion, which are common, and the subject-genus, which is peculiar
(e.g. number, spatial magnitude).
88319. 1TPWTOV IJ-£v AOYLKW5 OEwpouaw. The arguments in 319-
30 are called dialectical because they take account only of the
general principles of syllogistic reasoning, and not of the special
character of scientific reasoning.
19-26. oi IJ-£V yap ••• Ta.ATJOfj. This first argument is to the
effect that all syllogisms cannot proceed fwm the same premisses,
since broadly speaking true conclusions follow from true pre-
misses and false from false. A. has to admit that there are
exceptions; a true conclusion can follow from false premisses.
But this, he claims, can only happen once in a chain of reasoning,
since the false premisses from which the conclusion follows must
themselves have false premisses, which must in turn have false
premisses, and so on.
The argument is a weak one; for not both the premisses of
a false conclusion need be false, so that there may be a con-
siderable admixture of true propositions with false in a chain of
reasoning. A. himself describes the argument as dialectical (3I9).
27-30. EC7TL yap ••• £Ao,TTOV. 'What is equal is greater' and
'what is equal is less' are offered as examples of contrary false
propositions; 'justice is injustice' and 'justice is cowardice', and
again 'man is horse' and 'man is ox' as examples of incompatible
false propositions. It is evident that no two propositions so
related can be derived from exactly the same premisses.
30-1. 'EK Se TWV KELIJ-EVWV ••• 1TI1VTWV. The dialectical arguments602
COMMENTARY
in "19-3° took account of the existence of false propositions;
the scientific arguments in "3o-b29, being based on 'Ta K.dJ-L£va, on
what has been laid down in the earlier part of the book with
regard to demonstrative science, take account only of true proposi-
tions, since only true premisses (7Ib19-26), and therefore only
true conclusions, find a place in science.
JI-6. £'TEPQ~ ya.p ..• opwv. A. considers, first, propositions
which form the actual premisses of proof, i.e. (J'an, (iJ1To(J'un,
and 0p,uJ-L0t) (72"14-16, 18-24). These, he says, are in the case of
many subjects generically different, and those appropriate to
one subject cannot be applied to prove propositions about
another subject. If we want to prove that B is A, any terms
belonging to a different field must be introduced either (1) as
terms predicable of B and having A predicable of them, or (2)
as terms predicable of A. or of which B is predicable, or (3) some
of them will be introduced as in (1) and some as in (2). In any
case we shall have terms belonging to one field predicated of
terms belonging to another field, which we have seen in ch. 7 to
be impossible in scientific proof. Such propositions could
obviously not express connexions Ka(J' am-o.
J6- b J. ci~X ouSE ••• KOWWV. A. passes now to consider an-
other suggestion, that some of the a.~,wJ-La'Ta (72016-18), like the
law of excluded middle. can be used to prove all conclusions. In
answer to this he points out that proof requires also special
principles peculiar to different subjects (i.e. those considered in
88"31-6), proof taking place through the a.~~J-La'Ta along with such
special principles. The truth rather is that the special principles
form the premisses, and the common principles the rules according
to which inference proceeds.
bJ-7. En a.t apxa.l . . . EVSEX6I'EvQL. A. has given his main
proof in "31-b3, viz. that neither can principles proper to one main
genus be llsed to prove properties of another. nor can general
principles true of everything serve alone to prove anything. He
now adds, rather hastily, some further arguments. (I) The first
is that (a) the theory he is opposing imagines that the vast
variety of conclusions possible in science is proved from a small
identical set of principles; while in fact (b) premisses are not
much fewer than the conclusions derivable from them; not much
fewer, because the premisses required for the increase of our
knowledge are got not by repeating our old premisses, but either
(if we aim at extending our knowledge) by adding a major higher
than our previous major or a minor below our former minor
, or (if we aim at making our knowledge more thorough) by interpolating a middle term between two of
our previous terms (Jp.{Ja>v'op'/yov).
(b) is a careless remark. A. has considered the subject in An.
Pr. 42bI6-26, where he points out that if we add a fresh premiss
to an argument containing n premisses or n+I terms, we get
n new conclusions. Thus (i) from two premisses 'A is B', 'B is
C we get one conclusion, 'A is C, (ii) from three premisses' A is
B', 'B is C, 'C is D', we get three conclusions, 'A is C, 'A is D'
'B is D', (iii) from four premisses 'A is B', 'B is C, 'C is D',
'D is E' we get six conclusions 'A is C, 'A is D', 'A is E', 'B is
D', 'B is E', 'C is E'-and so on. With n premisses we have
n(n-r) conclusions, and as n becomes large the disparity between
2
the number of the premisses and that of the conclusions becomes
immense. That is what happens when the new terms are added
from outside ('TTpo(rn8~p.iYov 42bI8, 'TTpou>.ap.{Jayop.iyov ggbS). The
same thing happens if new terms are interpolated (Kay £L<; T6
p.iaoy o~ 'TTap£p.'TTl'TTTfI 42b23, Jp.{Ja>V.op.lyov 88bS), and A. concludes
'so that the conclusions are much more numerous than either
the terms or the premisses' ( 42 b2S ---n). It is only if the number of
premisses is itself comparatively small that it can be said to be
'little less than the number of the conclusions'; one is tempted
to say that if A. had already known the rule which he states in
the Prior A nalytics he would nardly have written as he does here,
and that An. Pr. i. 2S must be later than the present chapter.
The next sentence (b6- 7) is cryptic enough, but can be inter-
preted so as to give a good sense. 'If the apxal of all syllogisms
were the same, the terms which, combined into premisses, have
served to prove the conclusions already drawn-and these terms
must be finite in number-are all that are available for the
proving of all future conclusions, to whose number no limit can
be set. But in fact a finite number of premisses can be combined
only into a finite number of syllogisms.'
If this interpretation be correct, the argument is an ingenious
application of A.'s theory that there is no existing infinite but
only an infinity of potentiality (Phys. iii. 6-8).
Finally (b 7-8) A. points out that some principles are apodeictic,
some problematic; this, taken with the fact that conclusions
have a modality varying with that of their premisses (cf. An. Pr.
4Ib27-3I), shows that not all conclusions can be proved from the
same premisses.
9-:19. OUTW jllV o~v • • • jl£yE8oS. A. turns now to consider
other interpretations of the phrase 'the first principles of allCOMMENTARY
syllogisms are the same'. Does it mean (I) that the first principles
of all geometrical propositions are identical, those of all arithmeti-
cal propositions are identical, and those of all medical propositions
are identical? To say this is not to maintain the identity of all
first principles but only the self-identity of each set of first
principles, and to maintain this is to maintain nothing worth
maintaining (blo- Is ).
(2) The claim that all syllogisms have the same principles can
hardly mean the claim that any proposition requires the whole
mass of first principles for its proof. That would be a foolish
claim. We can see in the sciences that afford clear examples of
proof (i.e. in the mathematical sciences) that it is not so in fact;
and we can see by attempting the analysis of an argument that
it cannot be so; for each new conclusion involves the bringing
in of a new premiss, which therefore cannot have been used in
proving the previous conclusions (b IS - 20).
(3) The sentence in b2o-1 has two peculiar features. (a) The
first is the phrase nis 7rpclrras a/daov, 7rpo-rauns. 7rpw-ro, is very
frequently used in the same sense as afl.£Go" but if that were its
meaning here A. would almost certainly have said 7rpw-ras
KaL afl.£aou, (cf. e.g. 7Ib2I). The phrase as wc have it must point
to primary immediate premisses as distinct from the immediate
premisses in general which have been previously mentioned.
(This involves putting a comma after 7rpo-raa£L, and treating
-ram-a, as a repetition for the sake of emphasis; cf. 72b7-8 and
many examples in Kiihner, Cr. Cramm. § 469.4 b.) (b) The same
point emerges in the phrase fl.ta £11 £Kaanp r£II€'. This must mean
that out of all the principles proper to a subject-matter and not
available for the study of other subject-matters, there is one that
is primary. Zabarella is undoubtedly right in supposing this to
be the definition of the subject-matter of the science in question,
e.g. of number or of spatial magnitude (cf. b 2 8---<}); for it is from
the subject's essential nature that its consequential properties
are deduced.
(4) (b2I ---<}) If what is maintained is neither (2) nor (I) but an
intermediate view, that the first principles of all proof are identi-
cal in genus but different in species, the answer is that, as we
have already proved in ch. 7, generically different subjects have
generically different principles. Proof needs not only common
principles (the axioms) but also special principles relating to the
subject-matter of the science, viz. the definitions of the terms
used in the science, and the assumptions of the existence of the
primary subjects of the scit'nce (cf. 72314--24).I. 32- 88b9-29
605
Cherniss (A.'5 Criticism of Plato and the Academy, i. 73 n.)
argues with much probability that this fourth view is that of
Speusippus, who insisted on the unity of all knowledge, the know-
ledge of any part of reality depending on exhaustive knowledge
of all reality, and all knowledge being a knowledge of similarities
(6J-LotOT7]!; = avyyEIIHU). Ct. 9786·-Il n.

CHAPTER 33

Opinion

Knowledge differs from opinion in that knowledge is universal and
reached by necessary, non-contingent, premisses.

There are things that are true but contingent.
Our state of mind with regard to them is

(1) not knowledge;  for then what is contingent would be necessary;
(2) nor intuition (which is the starting-point of knowledge) or
undemonstrated knowledge (which is apprehension of an immediate proposition).

But the states of mind capable of being true are
intuition, knowledge, and opinion;
so it must be opinion that is concerned with
what is true or false, but contingent.

Opinion is the judging of an unmediated and nonnecessary proposition.

    This agrees with the observed facts;
    for both opinion and the contingent are insecure.
    Besides, a man thinks he has opinion, not when
    he thinks the fact is necessary, he then thinks he knows,
    but when he thinks it might be otherwise.

How then is it possible to have opinion and knowledge of
the same thing? And if one maintains that anything that is
known could be opined, will not that identify opinion and know-
ledge? A man who knows and one who opines will be able to
keep pace with each other through the chain of middle terms till
they reach immediate premisses, so that if the first knows, so
does the second; for one may opine a reason as well as a fact.
16. We answer that if a man accepts non-contingent proposi-
tions as he does the definitions from which demonstration pro-
ceeds, he will be not opining but knowing; but if he thinks the
propositions are true but not in consequence of the very nature
of the subject, he will have opinion and not genuine knowledge-
both of the fact and of the reason, if his opinion is based on the
immediate premisses; otherwise, only of the fact.
23. There cannot be opinion and knowledge of what is com-
pletely the same; but as there can be false and true opinion of
what is in a sense the same, so there can be knowledge and opinion.
To maintain that true and false opinion have strictly the same
object involves, among other paradoxical consequences, that one
does not opine what one opines falsely. But since 'the same' is
ambiguous, it is possible to opine truly and falsely what is in one
sense the same, but not what is so in another sense. It is impos-
sible to opine truly that the diagonal of a square is commen-
surate with the side; the diagonal, which is the subject of both
opinions, is the same, but the essential nature ascribed to the
subjects in the two cases is not the same.
33. So too with knowledge and opinion. If the judgement be
'man is an animal', knowledge is of 'animal' as a predicate that
cannot fail to belong to the subject, opinion is of it as a predicate
that need not belong; or we may say that knowledge is of man
in his essential nature, opinion is of man but not of his essential
nature. The object is the same because it is in both cases man,
but the mode in which it is regarded is not the same.
38. It is evident from this that it is impossible to opine and
know the same thing at the same time; for that would imply
judging that the fact might be otherwise, and that it could not.
In different persons there may be knowledge and opinion of the
same thing in the sense just described, but in the same person
this cannot happen even in that sense; for then he would be
judging at the same time, for example, that man is essentially an
animal and that he is not.

The question how the remaining functions should be assigned to
understanding, intuitive reason, science, art, practical
wisdom, and philosophical knowledge belongs, rather, in part to
physics and in part to ethics.

CHAPTER 34

Quick wit

Quick wit is a power of hitting the middle term in an imperceptible time.

    If one sees that the moon always has
    its bright side towards the sun, and
    quickly grasps the reason,
    viz. that it gets its light from the sun;
    or recognizes that someone
    is talking to a rich man because he is borrowing from him; or
    why two men are friends, viz. because they have a common enemy.

On seeing the extremes one has recognized all the middle terms
