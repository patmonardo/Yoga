POSTERIOR ANALYTICS
BOOK I
CHAPTER 1
The student's need of pre-existent knowledge. Its nature

All teaching and learning by way of reasoning proceeds
from pre-existing knowledge; this is true both of the mathe-
matical and of all other sciences, of dialectical arguments by way
of syllogism or induction, and of their analogues in rhetorical
proof--enthymeme and example.
n. With regard to some things we must know beforehand
that they are (e.g. that everything may be either truly affirmed
or truly denied) ; with regard to others, what the thing referred
to (e.g. triangle) is; with regard to others (e.g. the unit) we must
have both kinds of knowledge.
17. Some of the premisses are known beforehand, others may
come to be known simultaneously with the conclusion-i.e. the
instances falling under the universal of which we have knowledge.
That every triangle has its angles equal to two right angles one
knew beforehand; that this figure in the semicircle is a triangle
one comes to know at the moment one draws the conclusion.
(For some things we learn in this way, i.e. individual things
which are not attributes-the individual thing not coming to
be known through the middle term.)
24. Before one draws the conclusion one knows in one sense,
and in another does not know. For how could one have known
that to have angles equal to two right angles, which one did not
know to exist? One knows in the sense that one knows univer-
sally; one does not know in the unqualified sense.
29. If we do not draw this distinction, we get the problem of
the M eno; a man will learn either nothing or what he already
knows. We must not solve the problem as some do. If A is
asked 'Do you know that every pair is even?' and says 'Yes',
B may produce a pair which A did not know to exist, let alone
to be even. These thinkers solve the problem by saying that the
claim is not to know that every pair is even, but that every pair
known to be a pair is even ..
34. But we know that of which we have proof, and we have
proof not about 'everything that we know to be a triangle, or a
number', but about every number or triangle.
7Ial.COMMENTARY
b S. There is, however, nothing to prevent one's knowing
already in one sense, and not knowing in another, what one learns;
what would be odd would be if one knew a thing in the same
sense in which one was learning it.
71"1. naaa. SLSa.aKa.).(a. ••• SLa.V0'lTLKTJ. ~havcnrnK7} is used to
indicate the acquisition of knowledge by reasoning as opposed to
its acquisition by the use of the senses.
2-11. CPa.VEPOV SE •.• aU).).0YLal'05. That all reasoning pro-
ceeds from pre-existing knowledge can be seen, says A., by
looking (I) at the various sciences ("3-4), or (2) at the two kinds
of argument used in dialectical reasoning (·5~), or (3) at the
corresponding kinds used in rhetoric ("c)-Il). The distinction
drawn between al £-rncnfjl-'aL and ol '\6YOL indicates that by the
latter we are to understand dialectical arguments. For the dis·
tinction cf. EV TO'S' l-'a8fJI-'aULV )( KaTa '1'01), ,\6yovS', Top. I 58b29,
159"1, and the regular use of '\0YLK6S' in the sense of 'dialectical'.
'\al-'{3d.voV'T£S' tVS' 'TTapa ,VVdVTWV ("7) is an allusion to the dialectical
method of £p<lrrr]ULS', i.e. of getting one's premisses by questioning
the opponent.
3. a.! TE yap 1'a.e"I'a.TLKa.~ TWV E-rnaT'TJl'wv, Throughout the
first book of the Posterior A nalytics A.'s examples of scientific
procedure are taken predominantly from mathematics; cf. chs. 7,
9, 10, 12, 13, 27.
3-4 TWV E-rnaT'Il'wv •.• TWV ci).).wv ••• TEXVWV. While A. does
not here draw a clear distinction between £'TTLurijl-'aL and T£xvaL,
E'TTLU'Tfjl-'aL is naturally used of the abstract theoretical sciences,
while T£)(I'aL points to bodies of knowledge that aim at production
of some kind; cf. 100"9 Eav I-'t.v 'TT£P' Y£v£aLv, T'XVTJS', Eav 8t. 'TT£P' TO
av, E'TTLU'T~I-'TJS', and the fuller treatment of the distinction in E.N.
I I39bI8-I 14°&23.
S-6. 01'0(W5 SE ••• E-rra.ywy115. The grammar is loose. 'So too
as regards the arguments, both syllogistic and inductive argu-
ments proceed from pre-existing knowledge.'
9-11. ii yap ••• aU).).0ywl'05. On the relation of 'TTap&.8£LYl-'a
to E'TTayw~ cf. A n. Pr. ii. 24, and on that of EvOJJI-'TJl-'a to av'\'\o-
yLUp.6S' cf. ib. 27.
11-17. SLXW5 S' ... TJI'LV. A. has before his mind three kinds of
proposition which he thinks to be known without proof, and to be
required as starting-points for proof: (I) nominal definitions of
the meanings of certain words (he tells us in 76&32-3 that a science
assumes the nominal definitions of all its special terms); (2)
statements that certain things exist (he tells us in 76"33-6 thatI.
1. 7I&I-2I
505
only the primary entities should be assumed to exist, e.g. in
arithmetic units, in geometry spatial figures); (3) general state-
ments such as 'Any proposition may either be truly affirmed or
truly denied'. Of these (I) are properly called OPtCTfLOl (72&21),
(2) Vrr08'CT~t; (ib. 20), (3) dgtWfLaTa (ib. 17). But here he groups
(2) and (3) together under the general name of statements OTt ;(17"£,
which by a zeugma includes both statements that so-and-so
exists (2) and statements that so-and-so is the case (3), in dis-
tinction from statements that such-and-such a word means so-
and-so (I).
14. TO SE TPLYWVOV, OT! TOSt lTIl\-1aLVEL. Elsewhere A. some-
times treats the triangle as one of the fundamental subjects of
geometry, whose existence, as well as the meaning of the word,
is assumed. Here triangularity seems to be treated as a property
whose existence is not assumed but to be proved. In that case
he is probably thinking of points and lines as being the only
fundamental subjects of geometry, and of triangularity as an
attribute of certain groups of lines. This way of speaking of it
occurs again in 92b1S-x6 and (according to the natural interpreta-
tion) in 76&33-6.
1'7-19. "EOTL SE ••• yvWCTIY. A. does not say in so many words,
but what is implied is, that the major premiss of a syllogism
must be known before the conclusion is drawn, but that the
minor premiss and the conclusion may come to be known simul-
taneously.
17. "EO'Tt SE ••• yvwpLCTaYTa. The sense requires -yl'wp{CTavra, and
the corruption is probably due to the eye of the writer of the
ancestor of all our MSS. having travelled on to >..afL{3&.vovra.
18-19' otov OCTa TUYXclVEl ••• YVWCTLY. The best that can be
made of this, with the traditional reading TO Ka86>..ov, wv ;x~' n}v
YVWCTtV, is to take it to mean 'knowledge, this latter, of the parti-
culars actually falling under the universal and therein already
virtually known' (Ox£. trans.). But this interpretation is difficult,
since the whole sentence states an opposition between the major
premiss, which is previously known, and the minor, which comes
to be known. simultaneously with the conclusion. This clearly
points to the reading TO Ka86>..ov 0 V ;x£t n}v -yl'WCTtV, which alone
appears to be known to P. (12.23) and to T. (3.16). The corruption
has probably arisen through an omission of ov after Ka86>.ov,
which a copyist then tried to patch up by inserting wv.
19-2I. aT! \-1Ev yap • • • iYVWPLO'EV. The reference is to the
proof of the proposition that the angle in a semicircle is a right
angle (Euc. iii. 31) by means of the proposition that the angles of5 0 6
COMMENTARY
a triangle equal two right angles (Euc. i. 32). There are fuller
references to the proof in 94828-34 and Met. 1051"26-33.
Heath in Mathematics in Aristotle, 37-9, makes an ingenious
suggestion. He suggests a construction such that it is only in the
course of following a proof that a learner realizes that what he
is dealing with is a triangle (one of the sides having been drawn
not as one line but two as meeting at a point).
21-4. afl-a E1TaY0fl-£vo5 ••• E1Tax8fjvaL. In a note prefixed to
A n. Pr. ii. 23 I have examined the usage of E7TClym· in A., and have
argued that afLa E7TayofL£vo<; here means 'at the very moment one
is led on to the conclusion', and that this is the main usage under-
lying the technical sense of E7Taywy~ = 'induction'. Yet the pro-
cess referred to here is not inductive. The fact referred to is the
fact that if one already knows a major premiss of the form All
M is P, knowledge of the minor premiss 5 is M may come
simultaneously with the drawing of the conclusion 5 is P; the
reasoning referred to is an ordinary syllogism. E7Tax8fivat in "25
has the same meaning; E7Tax8fivat and Aa/3£iv avAAoytUfLoV are
different ways of referring to the same thing.
21-4. EVLWV yap ... TLV05, i.e. while it is (for instance) through
the middle term 'triangle' that an individual figure is known to
have its angles equal to two right angles. it is not through a
middle term that the individual figure is known to be a triangle;
it is just seen directly to be one.
24-5. 1TPLV S' E1TQX8fjvQL .•• au~~oYLafl-0v, cf. a21 n.
26-b 8. 0 yap •.• W5. With this discussion may be compared
that in A n. Pr. ii. 21.
29. TO EV TQ MEVWVL cl.1TOP'lfl-Q• Cf. M eno 80 d Kat ·rlva TPC)7TOV
~7JT~UH<;, cl, LWKpaT£<;, TOVTO 0 fL~ ofu8a TO 7Tapa7Tav 0 n EaT{V; 7Toiov
yap cl,v OUK ofu8a 7Tpo8£fL£VOC; ~7JT~UH<;; ~ £l Kat on fLaAtaTa Ev-n.JXot<;
aUTcp, 7T<;J<; £tU€t on TOVTO Eunv 0 UV OUK i/S7Ju8a; This problem,
which Plato solved by his doctrine that all learning is reminiscence,
A. solves by pointing out that in knowing the major premiss one
already knows the conclusion potentially.
3o- b 5. ou yap Stl ... 1TQVT05. The question is whether a man
who has not considered every pair of things in the world and
noticed its number to be even can be said to know that every pair
is even. It would seem absurd to deny that one knows this; but
if one claims to know it, one might seem to be refuted by being
confronted with a pair which one did not even know to exist.
A solution which had evidently been offered by certain people
was that what one knows is that every pair that one knows to be
a pair is even; but A. rightly points out that this is a completely5 0 7
unnatural limitation to set on the claim to know that every pair
is even. His own solution (bS -8) is that we must distinguish two
modes of knowledge and say that one knows beforehand in a
sense (i.e. potentially) that the particular pair is even, but does
not know it in another sense (i.e. actually).
CHAPTER 2
The nai1ere of scientific knowledge and of its premisses
71b9_ We think we know a fact without qualification, not
in the sophistical way (i.e. per accidens), when we think that we
know its cause to be its cause, and that the fact could not be
otherwise; those who think they know think they are in this con-
dition, and those who do know both tnink they are, and actually
are, ir:. it.
16. We will discuss later whether there is another way of
knowing; but at any rate there is knowledge by way of proof, i.e.
by way of scientific syllogism.
19- If knowledge is such as we have stated it to be, demonstra-
tive knowledge must proceed from premisses that are (I) true,
(2) primary and immediate, (3) (a) better known than, (b) prior
to, and (c) causes of, the conclusion. That is what will make
our starting-points appropriate to the fact to be proved. There
can be syllogism without these conditions, but not proof, because
there cannot be scientific knowledge.
25. (I) The premisses must be true, because it is impossible
to know that which is not.
26. (2) They must be primary, indemonstrable premisses be-
cause otherwise we should not have knowledge unless we had
proof of them <which is impossible); for to know (otherwise than
per accidens) that which is provable is to have proof of it.
29- (3) They must be (a) causes, because we have scientific
knowledge only when we know the cause; (b) prior, because they
are causes; (c) known beforehand, not only in the sense that we
understand what is meant, but in the sense that we know them
to be the case.
33. Things are prior and better known in two ways: for the
same thing is not prior by nature and prior to us, or better known
by nature and better known to us. The things nearer to sense are
prior and better known relatively to us, those that are more
remote prior and better known without qualification. The most
universal things are farthest from sense, the individual things
nearest to it; and tliese are opposed to each other.COMMENTARY
508
72.5. To proceed from what is primary is to proceed from the
appropriate starting-points. A starting-point of proof is an im-
mediate premiss, i.e. one to which no other is prior. A premiss is
a positive or negative proposition predicating a single predicate
of a single subject; a dialectical premiss assumes either of the pair
indifferently, a demonstrative premiss assumes one definitely to be
true. A proposition is either side of a contradiction. A contra-
diction is an opposition which of itself excludes any intennediate.
A side of a contradiction is, if it asserts something of something,
an affirmation; it it denies something of something, a negation.
14. Of immediate syllogistic starting-points, I give the name
of thesis to one that cannot be' proved, and that is not such that
nothing can be known without it; that of axiom to one which
a man needs if he is to learn anything. Of theses, that which
assumes a positive or negative proposition, i.e. that so-and-so
exists or that it does not exist, is an hypothesis; that which does
not do this is a definition. For a definition is a thesis, since it
lays it down that a unit is that which is indivisible in quantity;
but it is not an hypothesis, since it is not the same thing to say
what a unit is and that a unit exists.
25. Since what is required is to believe and know a fact by
having a demonstrative syllogism, and that depends on the truth
of the premisses, we must not only know beforehand the first
principles (all or some of them), but also know them better; for
to that by reason of which an attribute belongs to something,
the attribute belongs still more--e.g. that for which we love some-
thing is itself more dear. Thus if we know and believe because
of the primary facts, we know and believe them still more. But
if we neither know a thing nor are better placed with regard
to it than if we knew it, we cannot believe it more than the
things we know; and one who believed as a result of proof would
be in this case if he did not know his premisses beforehand; for
we must believe our starting-points (all or some) more than our
conclusion.
37. One who is to have demonstrative knowledge must not
only know and believe his premisses more than his conclusion,
but also none of the opposite propositions from which the
opposite and false conclusion would follow must be more credible
to or better known by him, since one who knows. must be abso-
lutely incapable of being convinced to the contrary.
7IbC;-IO.
P.
lLAAa' 11-iJ . . . aUI1~E~TJKOS. The reference is not, as
supposes, to sophistical arguments employing the
21. 15-28fallacy of accident. The meaning is made plain by 74a2S-30,
where A. points out that if one proves by separate proofs that the
equilateral, the isosceles, and the scalene triangle have their angles
equal to two right angles, one does not yet know, except TOV
UO</HCTTtKOV TPo-rrOV, that the triangle has that property, since one
does not know the triangle to have it as such, but only the triangle
when conjoined with any of its separable accidents of being equi-
lateral, being isosceles, or being scalene. In such a case, as A. says
here, one does not know the cause of its having the property,
nor know that it could not fail to have it.
16-17. Et ~Ev ow ..• ipou!'€v. In 72bl9-22 A. recognizes the
existence of bTtD"T'7/fLT} TWV afLEuwv ava1TOS£tKToS" as well as of
l1TtD"T'7/fLT} a1TOS£tKTtJO), and in 76°16-22 he describes it as the higher
of the two kinds. But in ii. 19 he discusses the question at length,
and gives the name of vous- to the faculty by which we know the
apXal, distinguishing this from l1TtD"T'7/fLT} , which is thus finally
identified with l1TtD"T'7/fLT} a1TOS£tKTtJO) (IOO bS-I7).
19-z3. (t TO£VUV • • • !)UKVU.uvOU. A. states first the charac-
teristics which the ultimate premisses of demonstration must
have in themselves. They must be (1) true, (2) primary, immedi-
ate, or in demonstrable (b2I , 27). 1TpWTa here does not mean 'most
fundamental', for A. could not, after saying that the premisses
must be fundamental in the highest degree, go on to make the
weaker statement that they must be more fundamental (1TpO-
TEPWV, a22) than the conclusion. To say this would be to confuse
the characteristics of the premisses in themselves (aAT}8wv Ka,
1TpWTWV) with their characteristics in relation to the conclusion
(YVWptfLCdTEPWV Ka, 1TPOTEPWV Ka, alTlwv TOV CTVfL1TfipaufLaTos-).
1TPWTWV, then, means just the same as afLEuwv or ava1ToSdKTwv
(b 27 )-that the premisses must be such that the predicate at-
taches to the subject directly as such, not through any middle
term.
A. next states the characteristics which the ultimate premisses
must have in relation to the conclusion. He states these as if
they were three in number---yvwptfLwT£pa, 1TPOT(pa, a'TLa (b2I , 29).
But in fact they seem to be reducible to two. (1) The facts stated
in the premisses must be objectively the grounds (ahta) of the
fact stated in the conclusion; it is only another way of saying
this to say that they must be objectively prior to, i.e. more
fundamental than, the fact stated in the conclusion (1TpoT£pa,
£'1T£P ahta, b 31 ). (2) It follows from this that they must be more
knowable in themselves; for if C is A only because B is A and
C is B, we can know (so A. maintains) that C is A only if we5 10
COMMENTARY
understand why it is so, i.e. only if we know that B is A, that
C is B, and that Cs being A is grounded in B's being A and in
Cs being B. It must be possible to know that B is A and that
C is B without already knowing that C is A, while it will be
impossible to know that C is A without already knowing that
B is A and that C is B. Further, the premisses must be known
beforehand not only in the sense that their meaning must be
grasped, but that they mw;t be known to be true (b 31 - 3, cf.
aII-q).
The fact that C is A may well be more familiar to us (~fL'v
YVWP'fLWT£POV, 72'1). I.e. it may be accepted as true, as being a
probable inference from the data of perception. But it will not
be known in the proper sense of the word, unless it is known on
the basis of the fact on which it is objectively grounded.
If these conditions (especially that indicated by the word
aLna) are all satisfied, the premisses that satisfy them will ipso
facto be the principles appropriate to the proof of the fact to be
proved; no further condition is necessary (7Ib22-3).
28. TO yap E11"LaTQa8Q~ ••• JLT] KCl.Ta aUjL~E~TJKO~J cf. b 9- Io n.
72"5-'7. EK 1TPWTWV 8' ... a.PX"v. This seems to be intended
to narrow down the statement that demonstration must proceed
£1( 7TPWTWV (7Ib2I). Not any and every immediate proposition
will serve; the premisses must be appropriate to the science.
This does not mean that they must be peculiar to the science
(though 0[1(£'i0> often implies that) ; for among them are included
premisses which must be known if anything is to be known
(aI6-18)-the axioms which lie at the root of all proof, e.g. the
law of contradiction. What is excluded is the use of immediate
propositions not appropriate to the subject-matter in hand, in
other words the fL£TCi/3aat> £~ lliov y€vov>, the use of arithmetical
propositions, for instance, to prove a geometrical proposition
(cf. chs. 7 and 9).
8-cJ. 11"poTa.a~~ 8' EaTlv •.• jLop~ov, i.e. a premiss is either an
affirmative or a negative proposition.
9-10. l)~QAEKTLKT] jLEV •.• 01TOTEpOVOUV. The method of dia-
lectic is to ask the respondent a well-chosen question and, what-
ever answer he gives, to prove your own case with his answer
as a basis; cf. De I nt. 20b22-3.
14-24. 'AjLEaOu l)' a.pxij~ ••• Ta.UTOV. It must be noted that the
definitions here given of ()€m>, d~{wfLa, lnro()wt> are defmitions of
them as technical terms, and that this does not preclude A. from
often using these words in wider or different senses. The variolls
kinds of dpX~ are dealt with more fully in ch. 10. On the partialSI!
correspondence which exists between A.'s dtuVP.UTU (KOLVa. 76"38,
77a27, 30, KOLVUL dpxut 88 b28, KOLVUL 80tUL Alet. 996b28), 1nr0(UO£LS,
and 0pLop.ol, and Euclid's KOLVUL EVVOLUL, uLT~p.uTa, and OPOL, cf.
H. D. P. Lee in C.Q. xxix. 113-18 and Heath, Mathematics in
Aristotle, 53-7.
17-18. Toi:iTo yap • . . >'EYELV, i.e. A. here strictly (p.aALoTu)
restricts the name ritlI.JJp.u to propositions like the 'laws of thought'
which underlie all reasoning, while implicitly admitting that it
is often applied to fundamental propositions relating only to
quantities-what in Met. 1005'20 A. calls Tt1 £V TO'S p.U(J~P.UOL
KUAoVP.€VU dtLWP.UTU (implying that the word is borrowed from
mathematics), the KOLVUL EVVOLUL which are prefixed to Euclid's
Elements, and probably also were prcfixed to the books of
Elements that existed in A.'s time. Thus in 77"30-1 both the law
of excluded middle and the principle that if equals are taken
from equals, equals remain are quoted as instances of Ta KOLva.
IS-ZO. 9ECJEWC; S' ... ll"rro9EeJLC;. The present passage is the only
one in which V7TO(JEOLS has this strict sense. In 76b3S-9, 77 8 3-4 the
distinction of InrOOEOL'i from definition is maintained, but in that
context (76b23-31) l)7To(JEoLs is said to be, not a self-evident truth,
but something which, though provable, is assumed without
proof. That corresponds better with the ordinary meaning of
the word.
28. ;; Tl"ana. ;; EVLa.. The discussion in 71b29-72aS has stated
that the premisses of demonstration must all be known in advance
of the conclusion. But A. remembers that he has pointed out
in 71°17-21 that the minor premiss in a scientific proof need not
be known before the conclusion; and the qualification ~ 7TavTu
~ EvLa is introduced with reference to this.
zcr30' a.iEt yap ••. l1u>,>,ov. What A. is saying is evidently
that if the attribute A belongs to C because it belongs to Band
B to C, it belongs to B more properly than to C. I have therefore
read €KElv'P for the MS. reading £KELVO. T. evidently read £KEtv'P
(8.6), and so did P. (38. IS).
36. ;; Tl"aaa.LC; ;; TLa(, cf. 328 n.
bl - 3 . 6.>'>'0. 11"'18' ••• 6.TI"aT"'1C;' This may mean (I) 'but also
nothing else, i.e. none of the propositions opposed to the first
principles, from which propositions the opposite and false con-
clusion would follow, must be more credible or better known to
him than the first principles', or (2) 'but also nothing must be
more credible or bctter known to him than the propositions
opposed to the principles from which the opposite and false
conclusion would follow', i.e. than the true principles. T. 8. 16-20COMMENTARY
5 1 2
and P. 41. 21-42. 2 take the words in the first sense; Zabarella
adopts a third interpretation-'but also nothing must be more
credible or better known to him than the falsity of the propositions
opposed to the principles, from which propositions the opposite
and false conclusion would follow'; but this is hardly a defensible
interpretation. Between the other two it is difficult to choose.
CHAPTER 3
Two errors-the view that knowledge is impossible because it involves
an infinite regress, and the view that circular demonstration is
satisfactory
b
7:z S. Because the first principles need to be known, (I) some
think knowledge is not possible, (2) some think it is but everything
is provable; neither view is either true or required by the facts.
(I) The fonner school think we are involved in an infinite regress,
on the ground that we cannot know the later propositions because
of the earlier unless there are first propositions (and in this they
are right; for it is impossible to traverse an infinite series) ; while
if there are, they are unknowable because there is no proof of
them, and if they cannot be known, the later propositions cannot
be known simply, but only known to be true if the first .principles
are.
IS. (2) The latter school agree that knowledge is possible only
by way of proof, but say there can be proof of all the propositions,
since they can be proved from one another.
18. <Repudiation of the underlying assumption that all know-
ledge is demonstrative.) We maintain that (a) not all knowledge
is demonstrative, that of immediate premisses not being so (this
must be true; for if we need to know the earlier propositions,
and these reach their limit in immediate propositions, the latter
must be indemonstrable) ; and (b) that there is not only scientific
knowledge but also a starting-point of it, whereby we know the
limiting propositions.
:zS. <Refutation of second view.) (a) That proof in the proper
sense cannot be circular is clear, if knowledge must proceed from
propositions prior to the conclusion; for the same things cannot
be both prior and posterior to the same things, except in the
sense that some things may be prior for us and others prior with-
out qualification-a distinction with which induction familiarizes
us. If induction be admitted as giving knowledge, our definition
of unqualified knowledge will have been too narrow, there being513
two kinds of it; or rather the second kind is not demonstration
proper, since it proceeds only from what is more familiar to us.
3z. (b) Those who say demonstration is circular make the
further mistake of reducing knowledge to the knowledge that a
thing is so if it is so (and at that rate it is easy to prove anything).
We can show this by taking three propositions; for it makes no
difference whether circular proof is said to take place through
a series of many or few, but it does matter whether it is said to
take place through few but more than two, or through two
(i) When A implies B and B implies C, A implies C. Now if
(ii) A implies Band B implies A, we may represent this as a
special case of (i) by putting A in the. place of C. Then to say (as
in (ii)) 'B implies A' is a case of saying (as in (i)) 'B implies C.
and this <together with 'A implies B') amounts to saying
'A implies C; but C is the same as A. Thus all they are saying
is that A implies A ; but at that rate it would be easy to prove
anything.
73"6. But indeed (c) even such proof as this is possible only in
the case of coextensive terms, i.e. of attributes peculiar to their
subjects. We have shown that from the assumption of one term
or one premiss nothing follows; we need at least two premisses,
as for syllogism in gp-neral. If A is predicable of Band C, and
these of each other and of A, we can prove, in the first figure, all
of these assumptions from one another, but in the other figures
we get either no conclusion or one different from the original
assumptions. When the terms are not mutually predicable
circular proof is impossible. Thus, since mutually predicable
terms are rare in demonstration, it is a vain claim to say that
proof is circular and that in that way there can be proof of every-
thing.
7zbS-6. 'EY(o~s flEY O~Y ••• dya~. There are allusions to this
view in Met. Ion"3-13 (£LaL 8£ nll£, oi a:rropoua, KaL TWY Tav-ra
'TT£'TTnafLEIIWII KaL TWII TOUS AOYOVS TOVrOVS fLOIIOY A£yOVTWY' ~7)Toua,
yap TLS <> KpWWY TOll Uy,aLIIOVTa KaL OAW, TOY 'TT£pL EKaaTa Kp'YOUYTa
op8ws. Ta S£ To,av-ra (hop~fLaTa ofLO'a. Ean T{jJ a'TTOp£LII 'TTOT£pOY Ka8d-
SOfL£1I IIUII ~ EYP7)yopafLOI, OUyaYTa, 0' at a'TTOpLa, at To,av-ra, miaa, TO
aVro· 'TTa.IITWY yap AOYOII a~WUaLY £fYaL O~TOL· apx~Y yap ~7)TouaL, KaL
TaUT7)1I SL' a'TTOO£L~£WS AafL~a.YHY, E'TT£L on Y£ 'TT£'TTHafLEYOL OVK £lal,
cpall£poL £LaLY EY TaLS 'TTpa.~£aLY. aAA' ()'TT£P £t'TTOfL£Y, TOUTO aVTWY TO
'TTa.80s EUTLy· AOyOY yap ~7)Touaw Jjy OUK Ean AOyo,· a'TTOO£'~£WS yap
apxTI OVK a'TTOOH~" EUTLII), loo6a5~ (a~wua, O~ KaL TOv-rO a'TTOOHK\lWaL
nll£, SL' a'TTaLOEVaLay· Ean yap a'TTaLOWaLa TO fL~ YLYYWaKHY TLYWY
4985
L 1COMMENTARY
OEE {"7TEEv ci1rOOELgLv KaL Tlvwv O~ OEE' oAWS' p,iJJ yap a.1TCfVTWV dOOvaTov
514
ci1TOOEL~LV £tvaL (£lS' a.1TELPOV yap
ci1T6o£L~LV), 1012"20--1 (ot P,tv O~V
av /3aol{oL, WUT£ p'''7o' oVrwS' £tvaL
OLa TOLatfn)v alTta)' UYOVCTLV, ot Ot
OLa TO 1TCfVTWV {"7T£Ev A6yov). It is not improbable that the school
of Antisthenes is referred to (cf. l006"S-<) quoted above with
o{ ~VTw8b£LOL KaL ol oVrwS' d1TalS£VToL (1043b24), 'AVTw81vryS' C;;£TO
Wrl8wS' (1024b3Z). The arguments for supposing Antisthenes to be
referred to are stated by Maier (2 b. IS n. 2); he follows Diimmler
too readily in scenting allusions to Antisthenes in Plato, but he
is probably right in saying that A.'s allusions are to Antisthenes.
Cf. my note on Met. looSbz-S.
We cannot certainly identify the second school, referred to (in
b6-7) as having held that knowledge is possible because there
is no objection to circular proof. P. offers no conjecture on the
subject. Cherniss (A.'s Criticism of Plato and the Academy,
i. 68) argues that 'it is probable that the thesis which A. here
criticizes was that of certain followers of Xenocrates who had
abandoned the last vestiges of the theory of ideas and therewith
the objects of direct knowledge that served as the principles of
demonstrative reason' ; and he may well be right.
6. 'II'clVTWV I'EvTOL ti'll'6SEL~L5 E!vaL. ci1T6o£",LS' is more idiomatic
than ci1TOO£~£LS' (cf. b12 , 17, 73"zo), which is easily accounted for
by itacism.
7-IS. ot I'EV ya.p U1I'09EI'EVOL ••• EO'TLV. The argument is a
dilemma: (I) If there are not primary propositions needing no
proof, the attempt to prove any proposition involves an infinite
regress, which necessarily cannot be completed; (z) if it is claimed
that there are such propositions, this must be denied, since the
only knowledge is by way of proof.
7-8. ot I'Ev ya.p U1I'09EI'EvOL ••• E1I'L<7Ta0'9aL. Bekker and Waitz
are right in reading OAWS' with n, against the evidence of most of
the MSS.; for these words answer to £vloLS' p,/v (b S) and refer to
those who believe knowledge to be impossible, while o{ III (b IS )
answers to ToES' O· (b6) and refers to those who hold that circular
reasoning gives knowledge. That knowledge is not possible other-
wise than by proof is common ground to both schools (bIS-16),
so that cVv\wS' would not serve to distinguish the first school from
the second. P. read OAWS' (4 2 •II , 4S. 17).
22. t<7TaTaL SE 1I'OTE Ta. Ci.I'EO'a. This is a rather careless account
of the situation more accurately expressed in 9Sbz2 by the words
UT?/O'£Tal ?TOV £lS' a.p,£O'ov. What A. means is that in the attempt
to prove what we want to prove, we must sooner or later come
to immediate premisses, not admitting of proof.5 I 5
23-4' Kat ou IlOVOV ••. yvwPLtoloLEV. A.'s fullest account of the
faculty by which apxal come to be known is to be found in An.
Post. ii. 19.
29. OV'II'EP TP0'll'OV ••• yvwPLlloV is rather loosely tacked on-
'a distinction of senses of "prior" with which induction familiar-
izes us', since in it what is prior in itself is established by means of
what is prior to us.
30-2. Et S' o\hw~ ••• YVWPLIlWTEpWV. If the establishment of
what is prior in itself by what is prior to us be admitted, (a)
knowledge in the strict sense will not have been correctly defined
by us in ch. 2 as proof from what is prior in itself, since there is
another kind of it, or rather (b) the other is not strictly proof
(nor strictly knowledge).
31. YLvollEv'l y'. Neither Bekker's reading YLVOfL£VT) , Waitz's
reading YLVOfL£VT) ~, nor P.'s reading ~ YWOfL£VT) is really satisfac-
tory; a more idiomatic text is produced by reading YLVOfL£VT) y'-
'or should we say that one of the two processes is not demonstra-
tion in the strict sense, since it arises from what is more familiar
to us?', not from what is more intelligible in itself.
3:z-']386. aUIl~aLvEL SE • • • p4SLOV. The passage is difficult
because it is so tersely expressed. The sense is as follows: 'The
advocates of circular reasoning cannot show that by it any pro-
position can be known to be true, but only that it can be known
to be true if it is true-which is clearly worthless, since if this
were proof of the proposition, any and every proposition could
be proved. This becomes clear if we take three OpOL; it does not
matter whether we take many or few, but it does matter whether
we take few or two' ('few' being evidently taken to mean 'three
or more'). One's first instinct is to suppose that A. is asserting
the point, fundamental to his theory of reasoning, that there
must be three terms-two to be connected and one to connect
them. But it is clear that in 72b37-73a6 A, B, and C are pro-
positions, not terms; and in fact A. very often uses opo<; loosely in
this sense. What he goes on to say is this: The advocates of
circular proof claim that if they can show that if A is the case
B is the case, and that if B is the case A is the case, they have
shown that A is the case. But, says A., the situation they en-
visage is simply a particular case of a wider situation-that in
which if A is the case B is the case, and if B is the case C is the
case; and just as there what is proved is not that C is the case,
but that C is the case if A is the case, so here what is proved
is not that A is the case, but only that A is the case if A is the
case.5 I6
COMMENTARY
TOUTO S' (In TOU A OVTOS TO
r
Ecrn (73 a 3) is difficult, and may be
corrupt. If it is genuine, it must be supposed to mean 'and <since
if A is true B is true> this implies that if A is true C is true'.
73"6-20. Ou ~;'V a.)'}" • • • a.'TI'6SEL~LV. A. comes now to his
third argument against the attempt to treat all proof as being
circular. He has considered circular proof in An. Pr. ii. S-7. He
has shovm that if we have a syllogism All B is A, All C is B,
Therefore all C is A, we can prove the major premiss from the
conclusion and the converse of the minor premiss (All C is A,
All B is C, Therefore all B is A), and the minor premiss from the
conclusion and the converse of the major premiss (All A is B,
All C is A, Therefore all C is B) (S7bz1---<)). But these proofs
are valid only if the original minor and major premiss, respec-
tively, are convertible. And that can be proved only if we add
to the original data (All B is A, All C is B) the datum that the
original conclusion is convertible. Then we can say All A is C,
All B is A, Therefore all B is C, and All C is B, All A is C, There-
fore all A is B. Thus, he maintains, we can prove each of the
original premisses by a circular proof in the first figure, only if
we know all three terms to be convertible (73"II-14, S7b3S-S8"1S).
The words USHKTaL SE Kat on £V TOLS" c:iAAo,S" crX~fLaaLv ~ o~
YLV€Ta' avAAoy,afL0<; ~ o~ 7T€pt T(J)V A7]CP()£VTWV (73"15-16) rather over-
state the results reached in An. Pr. ii. 6, 7. What A. has shown
there is that there cannot be in those figures a perfect circular
proof, i.e. a pair of arguments proving each premiss from the
conclusion
the converse of the other premiss, because (1) in
the second figure, the original conclusion being always negative,
it is impossible to use it to prove the affirmative original premiss,
and (2) in the third figure, the original conclusion being always
particular, it is impossible to use it to prove the universal original
premiss (or either premiss if both were universal). The discrepancy
is, however, unimportant; for A.'s main point is that, even where
the form of a syllogism does not make circular proof impossible,
the matter usually does, since most propositions are not in fact
convertible. A proposition will assert of a subject either its
essence, or part of its essence, or some other attribute of it. Now
if it states any part of the essence other than the lowest differ-
entia, the proposition will not be convertible; and of non-essential
attributes the great majority are not coextensive with their
subjects; thus only propositions stating the whole essence, or
the last differentia, or one of a comparatively small number out
of the non-essential attributes, are convertible (73"6-7, 16-18).
7-11. EVOS ~EV o~v ••• cru},},0yLcra.cr9a.L, cf. An. Pr. 34"16-21,
+I. 3· 73"6- 1 4
5 1 7
40b3O--7. Two premisses and three terms are necessary for demon-
strative syilogism, since they are necessary for any syllogism
(£t7r£P Ka~ av),),oy,aaa8a" all).
7. W(71f£P Ta. Hha.. r8,a may be used as in Top. 102 a 18 of
attIibutes convertible with the subject and non-essential, or, as
it is sometimes used (e.g. in 92a8), as including also the whole
definition and the lowest differentia, both of which are convertible
with the subject.
11-14. ~a.v ..,.EV o~v . . . UUAAOY'U!-1ou. A. has shown in An.
Pr. ii. 5 that if we have the syllogism All B is A, All C is B,
Therefore all C is A, then by assuming Band C convertible we
can say All C is A, All B is C, Therefore All B is A, and by
assuming A and B convertible we can say All A is B, All C is A,
Therefore all C is B. Thus the assumptions All C is A, All B is
C, All A is B are all that is needed to prove the two original
assumptions. In the present passage A. names six assumptions-
All B is A, All C is A, All B is C, All C is B, All A is B, All A is
C-and speaks of proving all the alTTJ8/vTa. What he means, then,
must be that we can prove any of these six propositions by taking
a suitable pair out of the other five; which is obviously true.
CHAPTER 4-
The premisses of demonstration must be such that the predicate is
true of every instance of the subject, true of the subject per se, and
and true of it precisely qua itself
73aZI. Since that which is known in the strict sense is in-
capable of being otherwise, that which is known demonstratively
must be necessary. But demonstrative knowledge is that which
we possess by having demonstration; therefore demonstration
must proceed from what is necessary. So we must examine the
nature of its premisses; but first we must define certain terms.
z8. I call that 'true of every instance' which is not true of one
instance and not of another, nor at one time and not at another.
This is supported by the fact that, when we are asked to admit
something as true of every instance, we object that in some
instance or at some time it is not.
34. I describe a thing as 'belonging per se' to something else
if (I) it belongs to it as an element in its essence (as line to
triangle, or point to line; for the being of triangles and lines con-
sists of lines and points, and the latter are included in the defini-
tion of the former) ; or (2) it belongs to the other, and the other
is included in its definition (as straight and curved belong to line,SIB
COMMENTARY
or odd and even, prime and composite, square and oblong, to
number). Things that belong to another but in neither of these
ways are accidents of it.
b5. (3) I describe as 'existing per se' that which is not predicated
of something else; e.g. that which is walking or is white must
first be something else, but a substance-an individual thing-is
what it is without needing to be something else. :rhings that are
predicated of something else I call accidents.
10. (4) That which happens to something else because of that
thing's own nature I describe as per se to it, and that which
happens to it not because of its own nature, as accidental; e.g. if
while a man is walking there is a flash of lightning, that is an
accident; but if an animal whose throat is being cut dies, that
happens to the animal per se.
16. Things that are per se, in the region of what is strictly
knowable, i.e. in sense (I) or (2), belong to their subjects by the
very nature of their subjects and necessarily. For it is impossible
that such an attribute, or one of two such opposite attributes
(e.g. straight or curved), should not belong to its subject. For
what is contrary to anpther is either its privation or its contra-
dictory in the same genus; e.g., that which is not odd, among
numbers, is even, in the sense that the one follows on the other.
Thus if it is necessary either to affirm or to deny a given attribute
of a given subject, per se attributes must be necessary.
25. I call that 'universally true' of its subject which is true of
every case, and belongs to the subject per se, and as being itself.
Therefore what is universally true of its subject belongs to it
of necessity. That which belongs to it per se and that which
belongs to it as being itself are the same. Point and straight belong
to the line per se, for they belong to it as being itself; having
angles equal to two right angles belongs to triangle as being
itself, for they belong to it per se.
32. A universal connexion of subject and attribute is found
when (1) an attribute is proved true of a~y chance instance of the
subject and (2) the subject, is the first (or widest) of which it is
proved true. E.g. (1) possession of angles equal to two right
angles is not a universal attribute of figure (it can be proved true
of a figure, but not of any chance figure); (2) it is true of any
chance isosceles triangle, but triangle is the first thing of which
it is true.
73a34-bI6. Ka,8' a,UTo. •.• a:rro8a,V(LV. Having in "28-34 dealt
with the first characteristic of the premisses of demonstration,5 I 9
that they must be true of every instance of their subject without
exception, A. now turns to the second characteristic, that they
must be true of it KaO' aVTo, in virtue of its own nature. He pro-
ceeds to define four types of case in which the phrase is applicable,
but of these only the first two are relevant to his theme, the
nature of the premisses of demonstration (cf. b16-18 n.); the
others are introduced for the sake of completeness. (I) The first
case (a34-7) is this: that which V7To.pX"-' to a thing as included in
its essence is KaO' aVTo to it. v7TlipX€w is a word constantly used
by A. in describing an attribute as belonging to a subject, and
the type of proposition he has mainly in mind is a proposition
stating one or more attributes essential to the subject and in-
cluded in its definition. But V7To.pX€W is a non-technical word. Not
only can an attribute be said VTTapx€W to its subject, but a con-
stituent can be said VTTapXHv to that of which it is a constituent,
and the instances actually given of KaO' aVTo. VTTapXOV'Ta are limits
involved in the being of complex wholes-lines in the triangle,
points in the line ("3S). These two types of KaO' aVTo. v7TapxoVTa
can be included under one formula by saying that KaO' aVTo.
VTTo.pXOV'Ta in this sense are things that are mentioned in the
definition of the subject (whether as necessary attributes or as
necessary elements in its nature).
(2) The second case ("37- bS) is that of attributes which while
belonging to certain subjects cannot be defined without mention-
ing these subjects. In all the instances A. gives of this sort of
situation ("38-bI, bI<}-2I) these attributes occur in pairs such that
every instance of the subject must have one or other of the
attributes; but there is no reason why they should not occur in
groups of three (e.g. equilateral, isosceles, scalene as attributes
of the triangle) or of some larger number.
For the sake of completeness A. mentions two other cases
in which the expression KaO' aVTo is used. (3) (b S- IO) From pro-
positions in which an attribute belonging KaO' aVTo to a subject
is asserted of it, he turns to propositions in which a thing is said
to exist KaO' aVTo. It is only individual substances (1'7) that exist
KaO' aVTo., not in virtue of some implied substratum. When on
the other hand we refer to something by an adjectival or partici-
pial phrase such as TO A€VKOV or TO f3a'O{~ov, we do not mean that
the quality or the activity referred to exists in its own right; it
can exist only by belonging to something that has or does it;
what is white must be a body (or a surface), what.is walking an
animal.
Finally (4) (b r o- r 6) we use the phrase to describe a necessaryCOMMENTARY
5 20
connexion not between an attribute and a subject, but between
two events, viz. the causal relation, as when we say that a thing
to which one event happened became Ka6' am-6 involved in an-
other event, KaTo. standing for 15,0., which more definitely refers
to the causal relation. This fourth type of Ka6' am-6 is akin to
the first two in that it points to a necessary relation between
that which is Ka6' am-6 and that to which it is Ka6' am-6, but the
relation here involves temporal sequence, as distinguished from
the timeless connexions between attribute and subject that are
found in the first two types.
34. aaa U1Ta.PXEL TE ~V Te;, TL ~aTlv. If the position of T£ be
stressed, A. should be here giving the first characteristic of a
certain kind of Ka6' aVT6, to be followed by another character-
istic introduced by Kat; and this we can actually get if we ter-
minate the parenthesis at £rrr[, "36. But then the second clause,
Ka, £V nfJ >"6yep Tip Myovn T[ £rrrLv £VVTTo.PX£L, would be practically
a repetition of the first. It is better therefore to suppose that T£
is, as often, slightly misplaced, and that what answers to the
present clause is Kai auo" ... S1)>"oiivn, "37-8.
Zarabella (In Duos Arist. Libb. Post. Anal. Comm. 3 23 R-V)
points out that aua tnro.PX£L £V Tip Tt £unv does not mean, strictly,
'those that are present in the Tt £rrrLv'. The construction of
V7ro.PX£LV is not with £V (as is that of £VV7ro.PX£LV) but with a simple
dative, and the proper translation is 'those things which belong
to a given subject, as elements in its essence'. The full construc-
tion, with both the dative and iv, is found in 74b8 TOL, 15' aVTd EV
Tep
...
n "
£unv V7rapX£L KaT1JyopOVf.'£VOL, aVTWV.
,
I
"
....
37-8. Kat 8aOLS TWV U1TapXOVTWV • • • ST)~OQVTL. V7ro.PXELV, in
A.'s logic, has a rather general significance, including the 'be-
longing' of a predicate to its subject, as straight and curved
belong to a line, and the 'belonging' to a thing of an element in
its nature, as a line belongs to a triangle. £VV7ro.PX£LV on the other
hand is a technical word used to denote the presence of something
as an element in the essence (and therefore in the definition) of
another thing. In certain passages the distinction is very clearly
marked: "38-bz orOV TO £u6l! V7To.pXU ypaf.'f.'fi . • . Kai 7Tau,
TOm-OL, £VV7ro.PXOVULV EV Tip >"6yep Tip Tt ErrrL MYOI'7"L (v6a f.'£v
ypaf.'f.'~ KT>"., 84"12 Ka6' am-d 15£ ~bTW" aaa T£ ydp £V £KEtvOL,
£VV7To.p XEL £V Tip Tt £rrrL, Ka, ot, atlTd £V Tip Tt £unv (sc. EVV7To.p xu)
,'J7To.PXOVULV (participle agreeing with or,) aUTO;:" orOV Tip apL6f.'ip
TO 7r£PLTT6v, 0 V7To.PXEL f.'£V apL6f.'ip, £VV7ro.PXEL S' mho, 0 apL6f.'o,
£V Tip >"6yep aUToii, ib. 20 7TPWTOV 0 apL6f.'o, EVV7To.P~£L iJ7To.PXOVULV
aVTtfJ. For other instances of tvv7Taf'X£I.V cf. 73~I7, 18,84&25. Any-5 21
thing that EVV7TCipXH €v something may be said iJ7TriPXHV to it,
but not vice versa. In view of these passages I concluded that
iJ7TaPXOVTwV should be read here, and afterwards found that I had
been anticipated by Bonitz (Arist. Stud. iv. 21). The emendation
derives some support from T. 10. 30 ouwv S~ UVP.{3E{3T)KOTWV
nUL TOV AOYOV d7TOS,SOlJ7"E, Ta lJ7ToKdp.EVa aUTO', avVErpEAKop.E()a £V Tep
AOycp, TaU-ra Ka()' atn-a lJ7TriPXHV TOVTO', AiYETa, TO', IJ7TOKHP.ivo".
The MSS. are similarly confused in "38, 84"13, 19, 20, and in
An. Pr. 6S"IS.
39-b1. Kat TO 1TEPLTTOV ••• ETEPO .... "lKES. Not only odd and
even, but prime and composite, square and oblong (i.e. non-square
composite), are Ka()' aUTO to number in the second sense of Ka()'
,
.
aV'To.
b 7 . Kat TO XEUKOV (XEUKOV).
The editions have KaL AEVKOV.
But this does not give a good sense, and n's KaL TO AWKOV points
the way to the true reading.
16-18. Ta. o.po. XEyO .... EVa ••• avo.YK"lS' A. seems here to be
picking out the first two senses of Ka()' aUTO as those most per-
tinent to his purpose (the other two having been mentioned in
order to give an exhaustive account of the senses of the phrase).
Similarly it is they alone that are mentioned in 84311-28. They
are specially pertinent to the subject of the Posterior Analytics
(demonstrative science). Propositions predicating of their subject
what is Ka()' aUTO to it in the first sense (viz. its definition or some
element in its definition) occur among the premisses of demon-
stration. With regard to propositions predicating of their subject
something that is Ka()' aUTO to it in the second sense, A. seems not
to have made up his mind whether their place is among the
premisses or among the conclusions of scientific reasoning. In
74bS-12 they are clearly placed among the premisses. In 7S"28-31
propositions asserting of their subjects something that is Ka()' atn-o
to them are said to occur both as premisses and as conclusions,
but A. does not there distinguish between the two kinds of Ka()'
atn-o proposition. In 76"32-6 TO EU()V (a Ka()' atn-o attribute of the
second kind) appears to be treated, in contrast to p.ovri, and
P.iYE()O" as something whose existence has to be proved, not to
be assumed; and 7TEP'TTOV and apnov are clearly so treated in
76b6-lI. In 75 340-1 and in 84aII-q propositions involving Ka()'
atn-o attributes are said to be objects of proof, and this must refer
to those which involve Ka()' aUTO attributes of the second kind,
since A. says consistently that both the essence and the existence
of Ka()' al\TO attributes of the first kind are assumed, not proved.
The truth is that A. has not distinguished between two typesCOMMENTARY
of proposition involving Ka(J' alh6 attributes of the second kind.
That every line is either straight, crooked, or curved, or that
every number is either odd or even, must be assumed; that a
particular line is straight (i.e. that three particular points are
collinear), or that a number reached by a particular arithmetical
operation is odd, must be proved. Thus to the two types of
rS,a, apxat recognized by A. in 72318-24 he ought to have added
a third type, disjunctive propositions such as 'every number must
be either odd or even'.
5 22
17. O~TWS
ws EVU1fa.PXELV TOLS ICa.TTlYOPOUIl€vOLS ;; EVU1fa.PXEa6a.L,
'as inhering in (i.e. being included in the essence of) the subjects
that are accused of possessing them' (mode (1) of the Ka(J' a&6
("34-7)), 'or being inhered in by them (i.e. having the subjects
included in their essence): (mode (2) of the Ka(J' aVT6 (a37-b3)).
KaTTJyopovp.£IJOIJ, generally used of the predicate, is occasionally,
as here, used of the subject 'accused', i.e. predicated about (cf.
An. Pr. 47b1).
For cV!; £VV'TTCiPXELIJ = cV!; lVWTaPXOIJTa, cf. 7SA2S.
19. ;; cl.1fAWS fJ Ta. cl.VTLlCdIlEVa.. a.1TAW!; applies to the attributes
that are Ka(J' aVT6 in the first sense, Ta aIJTLKEtp.£IJa to those that
are Ka(J' aVT6 in the second sense.
21-2. EaTL ya.p ••• i1fETa.L. Of two contrary terms, i.e. two
terms both positive in form but essentially opposed, either one
stands for a characteristic and the other stands for the complete
absence of that characteristic, while intermediate terms standing
for partial absences of it are possible (as there are colours between
white and black), or one term is 'identical with the contradictory
of the other, within the same genus'. In the latter case, while
the one term is not the bare negation of the other (if it were,
they would be contradictories, not contraries), yet within the
only genus of which either is an appropriate predicate, every term
must be characterized either by the one or by the other. Not
every entity must be either odd or even; but the only entities
that can be odd or even (i.e. numbers) must be one or the other.
The not-odd in number is even, not in the sense that 'even' means
nothing more than 'not odd', but inasmuch as every number that
is not odd must in consequence be even (?i £1TETa,).
25-32. To IlEv o~v ••• ta-ov. A. has in "28-b24 stated the first
two conditions for a predicate's belonging Ka(J6Aov to its subject-
that it must be true of every instance (KaTa 1TaIJT6!;) and true in
virtue of the subject's nature (Ka(J' a&6). He now adds a third
condition, that it must be true of the subject ?i av-r6, precisely
as being itself, not as being a species of a certain genus. It is5 2 3
puzzling, then, to find A. saying TO Ka()' aUTO Ka~ 17 aUTO TaUTOV
(b28). It must be remembered, however, that he is making his
terminology as he goes. Having first used Ka()' aUTO and 17 ath-o as
standing for different conditions, he now intimates that Ka()' aUTO
in a stricter sense means the same as 17 aUTO; that which belongs
to a subject strictly Ka()' aUTO is precisely that which belongs to
it qua itself, not in virtue of a generic nature which it shares
with other things; cf. 74"2 n.
This strict sense of Ka()o>..ov is, perhaps, found nowhere else
in A. ; usually the word is used in the sense of KaTa 1TaVTos simply;
e.g. in 99 8 33-4.
32-74"3. TO Ko,OoAOU SE . . . 1I'AEOV. Universality is present
when (1) the given predicate is true of every chance instance of
the subject, and (2) the given subject is the first, i.e. widest, class,
such that the predicate is true of every chance instance of it.
As a subject of 'having angles equal to two right angles', figure
violates the first condition, isosceles triangle the second; only
triangle satisfies both.
34. OUTE T~ aXTJllo,TL ~<7TL KaOoAou is answered irregularly by
TO S' laoaK£>"£S KT>"., b38.
7482. TWV S' aAAwv ••• o,UTO,. i.e. not Ka()' aUTO in the stricter
sense of Ka()' auTO in which it is identified with 17 aUTO (73b28--9).
CHAPTER 5
How we fall into, and how we can avoid, the error of thinking our
conclusion a true universal proposition when it is not
74 8 4. We may wrongly suppose a conclusion to be universal,
when (1) it is impossible to find a class higher than the sub-class
of which the predicate is proved, (2) there is such a class but it
has no name, or (3) the subject of which we prove an attribute
is taken only in part of its extent (then the attribute proved will
belong to every instance of the part taken, but the proof will not
apply to this part primarily and universally, i.e. qua itself).
13. (3) If we prove that lines perpendicular to the same line
do not meet, this is not a universal proof, since the property
belongs to them not because they make angles equal in this
particular way, but because they make equal angles, with the
single line.
16. (I) If there were no triangle except the isosceles triangle,
some property of the triangle as such might have been thought
to be a property of the isosceles triangle.COMMENTARY
17. (2) That proportionals alternate might be proved separ-
ately in the case of numbers, lines, solids, and times. It can be
proved of all by a single proof, but separate proofs used to be
given because there was no common name for all the species.
Now the property is proved of all of these in virtue of what they
have in common.
25. Therefore if one proves separately of the three kinds of
triangle that the angles equal two right angles, one does not yet
know (except in the sophistical sense) that the triangle has this
property-even if there is no other species of triangle. One knows
it of every triangle numerically, but not of every triangle in
respect of the common nature of all triangles.
32. When, then, does one know universally? If the essence of
triangle had been the same as the essence of equilateral triangle,
or of each of the three species, or of all together, we should have
been knowing, strictly. But if the essence is not the same, and
the property is a property of the triangle, we were not knowing.
To find whether it is a property of the genus or of the species, we
must find the subject to which it belongs directly, as qualifica-
tions are stripped away. The brazen isosceles triangle has the
property, but the property remains when 'brazen' and 'isosceles'
are stripped away. True, it does not remain when 'figure' or
'closed figure' is stripped away, but these are not the first quali-
fication whose removal removes the property. If triangle is the
first, it is of triangle that the property is proved universally.
74"6-13' Q1rQTWIlE9Q SE . . . J(Q96Xou. The three causes of
error (i.e. of supposing that we have a universal proof when we
have not) are (1) that in which a class is notionally a specification
of a genus, but it is impossible for us to detect the genus because
no examples of its other possible species exist ("7-8, illustrated
"16-17); (2) that in which various species of a genus exist, but
because they have no common name we do not recognize the
common nature on which a property common to them all depends,
and therefore offer separate proofs that they possess the property
("8-9, illustrated "17-32); (3) that in which various species exist
but a property common to all is proved only of one ("9-13,
illustrated "13-16).
Most of the commentators take the first case to be that in
which a class contains in fact only one individual (like the class
('earth', 'world', or 'sun'), and we prove a property of the
individual without recognizing that it possesses the property not
qua this individual but qua individual of this species. But (a)5 2 5
the only instance given ('16-17) is that in which we prove some-
thing of a species without recognizing that it is a property of the
genus, and (b) in the whole of the context the only sort of proof
A. contemplates is the proof that a class possesses a property.
The reference, therefore, cannot be to unique individuals.
i1 Ta. KaS' ~Kaa,-a (a8) can hardly be right. In the illustration
(816-17) A. contemplates only the case in which there is no more
than one species of a genus; and if more than one were referred
to here, the case would be identical with the second, in which
several species are considered but the attribute is not detected
as depending on their generic character, or else with the third,
in which only one out of several species is considered. The words
are omitted by C, and apparently by T. (13. 12-29) and by P.
(72. 23-73. 9); they are a mistaken gloss.
What is common to all three errors is that an attribute which
belongs strictly to a genus is proved to belong only to one, or
more than one, or all, of the species of the genus. In such a case
the attribute is true of the species Ka'nl 7TaVT6S' and KaO' aVT6, but
not fJ aVT6 (as this is defined in ch. 4).
13-16. Et o~V • • • taa~. The reference is to the proposition
established in Euc. El. i. 28, 'if a straight line intersecting two
straight lines makes the exterior angle equal to the interior and
opposite angle falling on the same side of it ... the two straight
lines will be parallel'. The error lies in supposing that the par-
allelness of the lines follows from the fact that the exterior angle
and the interior and opposite angle are equal by being both of
them right angles, instead of following merely from their equality.
17-25. Kat TO civciAOYOV ... ulI'cipxuv. A. refers here to a pro-
position in the general theory of proportion established by
Eudoxus and embodied in Euc. El. v, viz. the proposition that if
A : B = C: D, A: C = B: D, and points out the superiority of
Eudoxus' proof to the earlier proofs which established this pro-
position separately for different kinds of quantity; cf. 8S 8 36- b I.
and Heath, Mathematics in Aristotle, 43-6.
25-32. 5~a. '-00'-0 . . • 015Ev. Geminus (apud Eutocium in
Apollonium (Apollonius Pergaeus, ed. Heiberg, ii. 170)) says that
oi apxaZot actually did prove this proposition separately for the
three kinds of triangle. But Eudemus (apudProc1um, in Euclidem,
379), while he credits the Pythagoreans with discovering the
proposition, gives no hint of an earlier stage in which distinct
proofs were given. Geminus' statement may rest on a misunder-
standing of the present passage, This example does not precisely
illustrate the second cause of error (88-9); for the genus triangleCOMMENTARY
was not avwvvfLoV. But it illustrates the same general principle,
that to prove separately that an attribute belongs to several
species, when it really rests upon their common nature, is not
universal proof.
28. Et 11~ TOV aoc\lLC7TLKOV Tp611"ov. A sophist might well say
'You know that all triangles are either equilateral, isosceles, or
scalene. You have proved separately that each of them has its
angles equal to two right angles. Therefore you know that all
triangles have the property.' A. would reply 'Yes, but you do
not know that all triangles as such have this property; and only
knowledge that B as such is A is real scientific knowledge that
all B is A'.
29. ouS~ Ka.B' OAOU TPLYWVOU, 'nor does he know it of triangle
universally', should clearly be read instead of the vulgate reading
ov8~ Ka86110v Tplywvov. Cf. 7Sb2S n.
33-4' SfiAOV S~ ... 1I"ciaLV. It is possible to translate ~ JKacrrcp
~ 1Tacnv 'either for each or for all' but there is no obvious point
in this. A better sense seems to be got ·if we translate the whole
sentence 'we should have had true knowledge if it had been the
same thing to be a triangle and (a) to be equilateral, or (b) to
be each of the three severally (equilateral, isosceles, scalene), or
(c) to be all three taken together' (i.e. if to be a triangle were the
same thing as to be equilateral, isosceles, or scalene).
CHAPTER 6
The premisses of demonstration must state necessary connexions
74 b S. (I) If, then, -demonstrative knowledge proceeds from
necessary premisses, and essential attributes are necessary to
their subjects (some belonging to them as part of their essence,
while to others the subjects belong as part of their essence, viz.
to the pairs of attributes of which one or other necessarily belongs
to a given subject), the demonstrative syllogism must proceed
from such premisses; for every attribute belongs to its subject
either thus or per accidens, and accidents are not necessary to
their subjects.
13. (2) Alternatively we may argue thus: Since demonstration
is of necessary propositions, its premisses must be necessary.
For we may reason from true premisses without demonstrating,
but not from necessary premisses, necessity being the charac-
teristic of demonstration.
18. (3) That demonstration proceeds from necessary premisses
is shown by the fact that we object to those who think they aredemonstrating, by saying of their premisses 'that is not neces-
sary' -whether we think that this is so or that it may be so,
as far as the argument goes.
:u. Plainly, then, it is folly to be satisfied with premisses that
are plausible and true, like the sophistical premiss 'to know is to
possess knowledge'. It is not plausibility that makes a premiss;
it must be true directly of the subject genus, and not anything
and everything that is true is peculiar to the subject of which
it is asserted.
26. (4) That the premisses must be necessary may also be
proved as follows: If one who cannot show why a thing is so,
though demonstration is possible, has no scientific knowledge of
the fact, then if A, is necessarily true of C, but E, his middle term,
is not necessarily connected with the other terms, he does not
know the reason; for the conclusion is not true because of his
middle term, since his premisses are contingent but the conclusion
is necessary.
32. (5) Again, if someone does not know a certain fact now,
though he has his explanation of it and is still alive, and the fact
still exists and he has not forgotten it, then he did not know the
fact before. But if his premiss is not necessary, it might cease
to be true. Then he will retain his explanation, he will still exist,
and the fact will still exist, but he does not know it. Therefore
he did not know it before. If the premiss has not ceased to be
true but is capable of ceasing to be so, the conclusion will be
contingent; but it is impossible to know, if that is one's state of
mind.
75.1. (When the conclusion is necessary, the middle term used
need not be necessary; for we can infer the necessary from the
non-necessary, as we can infer what is true from false premisses.
But when the middle term is necessary, the conclusion is so, just
as true premisses can yield only a true conclusion; when the con-
clusion is not necessary, the premisses cannot be so.)
12. Therefore since, if one knows demonstratively, the facts
known must be necessary. the demonstration must use a neces-
sary middle term-else one will not know either why or tha,t the
fact is necessary; he will either think he knows when he does not
(if he takes what is not necessary to be necessary), or he will not
even think he knows-whether he knows the fact through middle·
terms or knows the reason, and does so through immediate
premisses.
18. Of non-essential attributes there is no demonstrative
knowledge. For we cannot prove the conclusion necessary, sinceCOMMENTARY
5 28
such an attribute need not belong to the subject. One might
ask why such premisses should be sought, for such a conclusion,
if the conclusion cannot be necessary; one might as well take any
chance premisses and then state the conclusion. The answer is
that one must seek such premisses not as giving the ground on
which a necessary conclusion really rests but as forcing anyone
who admits them to admit the conclusion, and to be saying what
is true in doing so, if the premisses are true.
~8. Since the attributes that belong to a genus per se, and as
such, belong to it necessarily, scientific demonstration must pro-
ceed to and from propositions stating such attributes. For
accidents are not necessary, so that by knowing them it is not
possible to know why the conclusion is true-not even if the
attributes belong always to their subjects, as in syllogisms
through signs. For with such premisses one will not know the
necessary attribute to be a necessary attribute, or know why it
belongs to its subject. Therefore the middle term must belong
to the minor, and the major to the middle, by the nature of the
minor and the middle term respectively.
74b7-IO. Ta. ~(V ya.p ... 01rclpXELV. Cf. the fuller statement in
73 a 34- b 3·
13. ~hL ,; ci1rOS(L~L~ ciVa.YKa.LWV €(nL. The sense is much im-
proved by reading avaYKa{wv or avaYKa{ov. P.'s paraphrase
(84. 18) £L yap ~ a1ToSngLS' TWV Eg avaYK1)S' EaTlv {mapxoVTwv points to
avaYKalwv. A. is arguing that demonstration, which is of necessary
truths, must be from necessary premisses.
21. (V(Kcl Y( TOU AOYou, not 'for the sake of the argument'
(which would be inappropriate with OLw/-L£8a.), but 'so far as the
argument goes' (sense 2 of EV£Ka in L. and S.)
z3-4' otov OL (70CPLaTa.t ••• €XELv. The reference must be to
PI. Euthyd. 277 b, where this is used as a premiss by the sophist
Dionysodorus.
34. 4»9a.P(LT) S' a.v TO ~((70V, i.e. the connexion of the middle
term with the major or with the mir,or might cease to exist.
75"1-17. "OTa.V ~(y o~V ••• a.~((7wv. This is usually printed as
a single paragraph, but really falls into two somewhat uncon-
nected parts. The first part (aI-Il) points out that the conclusion
of a syllogism may state something that is in fact necessarily
true, even when the premisses do not state such facts, while,
on the other hand, if the premisses state necessary facts, so will
the conclusion. This obviously does not aid A.'s main thesis,
that since the object of demonstration is to infer necessary facts,5 2 9
it must use necessary premisses. It is rather a parenthetical com-
ment, and the conclusion drawn in "I2 (bTf;' Tolvvv KT>'.) does not
follow from it, but sums up the result of the arguments adduced
in 74 bS-39, and especially of that in 74b26-32 (cf. OVT€ 8ton
7S"14 with OUI( 0[8£ 8ton 74b30). 7S"I-II points out the com-
patibility of non-necessary premisses with a necessary conclusion;
but the fact remains that though you may reach a necessary
conclusion from non-necessary premisses, you will not in that
case know either why or even that the conclusion is necessary.
3-4' w0"1I"€P Ka.t ciA"OE; ... ciA"Owv, cf. A n. Pr. ii. 2-4.
12-17. 'E1I"€t TOCVUV • • • a.ILEO"wv. The conclusion of the sentence
is difficult. The usual punctuation is 1j ou8' Ol1}O"£Tat ojLolw>, M.v
Tt: KT>'. One alteration is obvious; OjLo{w> must be connected with
what follows, not with what precedes. But the main difficulty
remains. A. says that 'if one is to know a fact demonstratively,
it must be a necessary fact, and therefore he must know it by
means of premisses that are necessary. If he does not do this,
he will not know either why or even that the fact is necessary, but
will either think he knows this (if he thinks the premisses to be
necessary) without doing so, or will not even think this (sc. if
he does not think the premisses necessary)-alike whether he
knows the fact through middle terms, or knows the reason, and
does so through immediate premisses.' There is an apparent
contradiction in representing one who is using non-necessary pre-
misses, and not thinking them to be necessary, as knowing the
conclusion and even as knowing the reason for it. Two attempts
have been made to avoid the difficulty. (I) Zabarella takes A.
to mean 'that you may construct a formally perfect syllogism,
inferring the fact, or even the reasoned fact, from what are
actually true and necessary premisses; yet because you co not
realize their necessity, you have not knowledge' (Mure ad loc.).
But (a), as Mure observes, in that case we should expect O"v>.-
>'0Y{OTJTaL for d8n. This might be a pardonable carelessness; what
is more serious is (b) that any reference to a man whose premisses
are necessary, but not known by him to be such, has no relevance
to the rest of the sentence, since the words beginning 1j OUI(
bTLt7T1}a£TaL deal with a person whose premisses are non-necessary.
(2) Maier (2 b. 2S0) takes £O.v T£ TO on ... ajL£awv to mean 'when
through other middle terms he knows the fact, or even knows the
reason of the necessity, and knows it by means of other premisses
that are immediate'. But there is no hint in the Greek of refer-
ence to a second syllogism also in the possession of the same
thinker.
MmCOMMENTARY
53 0
The solution lies in stressing dvaYK'I] in 814. A. is saying that
if someone uses premisses that are not apodeictic (e.g. All B is
A, All C is B), and does not think he knows that all B must be
A and all C must be B, he will not know why or even that all
B must be A-alike whether he knows by means of premisses
simply that all C is A, or knows why all C is A, and does so by
means of immediate premisses-since his premisses are in either
case ex hypothesi assertoric, not apodeictic.
18-19. ()V TP01rOV ••• a.UTa., cf. 73 8 37- b 3, 74bS-IO.
21. 1rEpt Toil TOlOIJTOU ya.p AEyW UU .... ~E~"lKOTOS, in distinction
from a O'uJLf3(;f3'1]KO, Ka(j' aUTO (i.e. a property).
22-3. Ka.LTOl a.1r0PtlUUEV ••• dVa.l. The word iPWTfiv, as well as
the substance of what A. says, shows that the reference is to
dialectical arguments.
25-']. SEl S' ••• U1ra.pXOVTa.. A. points here to the distinction
between the fonnal necessity which belongs to the conclusion
of any valid syllogism, and the material necessity which belongs
only to the conclusion of a demonstrative syllogism based on
materially necessary premisses.
For cfJ, . .. "fvat = cfJ, . •. cf. 73 b q.
33. otov ot Sla. CTTJ .... ELWV uuAAoylU .... OL. For these cf. An. Pr.
7oa7-b6. These are, broadly speaking, arguments that are neither
from ground to consequent nor from cause to effect, but from
effect to cause or from one to another of two attributes inci-
dentally connected.
av
CHAPTER 7
The premisses of a demonstration must state essential attributes of
the same genus of which a property is to be proved
75838. Therefore it is impossible to prove a fact by transition
from another genus, e.g. a geometrical fact by arithmetic. For
there are three elements in demonstration-(l) the conclusion
proved, i.e. an attribute's belonging to a genus per se, (2) the
axioms from which we proceed, (3) the underlying genus, whose
per se attributes are proved.
b2. The axioms may be the same; but where the genus is
different, as of arithmetic and geometry, the arithmetical proof
cannot be applied to prove the attributes of spatial magnitudes,
unless spatial magnitudes are numbers; we shall show later that
such application may happen in some cases. Arithmetical proof,
and every proof, has its own subject-genus. Therefore the genus
must be either the same, or the same in some respect, if proof53!
is to be transferable; otherwise it is impossible; for the extremes
and the middle term must be drawn from the same genus, since
if they are not cOlUlected per se, they are accidental to each other.
12. Therefore geometry cannot prove that the knowledge of
contraries is single, or that the product of two cubic numbers is a
cubic number, nor can one science prove the propositions of
another, unless the subjects of the one fall under those of the
other, as is the case with optics and geometry, or with harmonics
and arithmetic. Nor does geometry prove any attribute that
belongs to lines not qua lines but in virtue of something common
to them with other things.
75.41-2. Ev SE ... wv. The dgLwfLaTa are the KOLvaL dpXat, the
things one must know if one is to be able to infer anything
(72&16-17). It is rather misleading of A. to describe them as
the Eg Jiv; any science needs also ultimate premisses peculiar to
itself (81CTfLS') , viz. 6PLCTfLot, definitions of all its terms, and V'TrO-
81CTfLS', assumptions of the existence in reality of things answering
to its fundamental terms (72a14-24). But the axioms are in a
peculiar sense the Eg wv, the most fundamental starting-points
of all. The 6PLCTfLot and lnr081CTfL" being concerned with the
members of the ylvoS', are here included under the term ylvo,.
A.'s view here seems to be that axioms can be used as actual
premisses of demonstration (which is what Eg tilv naturally sug-
gests); and such axioms as 'the sums of equals are equal' are
frequently used as premisses in Euclid (and no doubt were used
in the pre-Euclidean geometry A. knew). But the proper function
of the more general (non-quantitative) axioms, such as the laws
of contradiction and excluded middle, is to serve as that not
from which, but according to which, argument proceeds; even
if we insert the law of contradiction as a premiss, we shall still
have to use it as a principle in order to justify our advance from
that and any other premiss to a conclusion. This point of view
is hinted at in 88a36-b3 (aAA' o~S£ TWV KOLVWV apxwv orov T' frvat
TLva, Eg tilv cL'TraV'Ta ofLx8~CTfTaL' Aiyw SE KOLVa, orov TO miv c/>dvaL
~ a'Troc/>avaL' Ta yap ylVTJ TWV OVTWV £Tfpa, Kat Ta- fLEV TOL, 'TrOCTOLS'
O£ TOL, 'TrOLOL, V'TrapXfL fLOVOL" fLf8' tilv OftKVlITaL SLa TWV KOLVWV).
The conclusion is arrived at by means of (SLa) the axioms with the
help of (fLfTa) the ,SLaL dpxat. 76bIO puts it still better-SfLKVvovaL
SLd Tf TWV KOLVWV KaL E K TWV a'TrOSfSHYfL1vwv. In accordance with
this, A. points out that the law of contradiction is not expressly
assumed as a premiss unless we desire a conclusion of the form
'C is A and not also not A' (77310--21). He points out further that
TaCOMMENTARY
53 2
the most universal axioms are not needed in their whole breadth
for proof in any particular science, but only oaov lKav6v, lKavov
S' br~ TOU
(ib. 23-4. cf. 76a42-b2).
bS-6. et "Tt . • . AEx9tiaETaL. fLf:,,'8TJ are not a.p,8P.OL, p'ry'87J
being ?Toao. avv£x77, a.p£8p.0£ ?Toao. Sl.WpwfLlva (Cat. 4b22-4). TOVrO S'
••• AE)(67)a£Ta, does not, then, mean that in some cases spatial
magnitudes are numbers, but that in some cases the subjects
of one science are at the same time subjects of another, or, as
A. puts it later, fall under those of another, are complexes formed
by the union of fresh attributes with the subjects of the other
,,'vov,
(b I4- I7 ).
6. uaTEpov AEx9tiaETaL, 76a9-15, 23-5, 78b34-79aI6.
9. ;; 'll'TI, i.e. in the case of the subaltern sciences referred to
in b6.
13. on ot Suo KUj30L KUj30S' This refers not, as P. supposes, to
the famous problem of doubling the cube (i.e. of finding a cube
whose volume is twice that of a given cube), but to the proposition
that the product of two cube numbers is a cube number, a purely
arithmetical proposition, proved as such in Euc. El. ix. 4.
CHAPTER 8
Only eternal connexions can be demonstrated
7S b :n. If the premisses are universal, the conclusion must be
an eternal truth. Therefore of non-eternal facts we have demon-
stration and knowledge not strictly, but only in an accidental way,
because it is not knowledge about a universal itself, but is limited
to a particular time and is knowledge only in a qualified sense.
:z6. When a non-eternal fact is demonstrated, the minor pre-
miss must be non-universal and non-eternal-non-eternal because
a conclusion must be true whenever its premiss is so, non-universal
because the predicate will at any time belong only to some
instances of the subject; so the conclusion will not be universal,
but only that something is the case at a certain time. So too with
definition, since a definition is either a starting-point of demon-
stration, or something differing from a demonstration only in
arrangement, or a conclusion of demonstration.
33. Demonstrations of things that happen often, in so far as
they relate to a certain type of subject, are eternal, and in so far
as they are not eternal they are particular.
In chapter 7 A. has shown that propositions proper to one
science cannot be proved by premisses drawn from another; in533
ch. 9 he shows that they cannot be proved by premisses applying
more widely than to the subject-matter of the science. There is
a close connexion between the two chapters, which is broken by
ch. 8. Zabarella therefore wished to place this chapter immedi-
ately after ch. 10. Further, he inserts the passage 77 a S-9, which
is clearly out of place in its traditional position, after ill' ~n
viiv in 7Sb30. In the absence, however, of any external evidence
it would be rash to effect the larger of these two transferences;
and as regards the smaller, I suggest ad loco a transference of
77 8 S-9 which seems more probable than that adopted by Zabarella.
The order of the work as a whole is not so carefully thought
out that we need be surprised at the presence of the present
chapter where we fmd it. A. is stating a number of corollaries
which follow from the account of the premisses of scientific
inference given in chs. 1-6. The present passage states one of
these corollaries, that there cannot strictly speaking be demon-
stration of non-eternal facts. And, carefully considered, what he
says here has a close connexion with what he has said in ch. 7.
In the present chapter A. turns from the universal and eternal
connexions of subject and attribute which mathematics discovers
and proves, to the kind of proof that occurs in such a science as
astronomy (ofov (1€A~V7]s- ;'KAdr/1€ws, 7S b 33). Astronomy differs in
two respects from mathematics; the subjects it studies are in
large part not universals like the triangle, but individual heavenly
bodies like the sun and the moon, and the attributes it studies
are in large part attributes, like being eclipsed, which these sub-
jects have only at certain times. A. does not clearly distinguish
the two points; it seems that only the second point caught his
attention (cf. TrOT€ 7Sb26, viiv ib. 30, TrOA,\dKLS- ib. 33). The gist
of what he says is that in explaining why the moon is eclipsed,
or in defining eclipse, we are not offering a strictly scientific
demonstration or definition, but one which is a demonstration
or definition only KUTa (1UILfJ€fJ'7K6S- (ib. 2S). There is an eternal
and necessary connexion involved; it is eternally true that that
which has an opaque body interposed between it and its source
of light is eclipsed; when we say the moon is eclipsed when (and
then because) it has the earth interposed between it and the sun,
we are making a particular application of this eternal connexion.
In so far as we are grasping a recurrent type of connexion, we
are grasping an eternal fact; in so far as our subject the moon
does not always have the eternally connected attributes, we are
grasping a merely particular fact (ib. 33-6).
7SbZS. n~X OUTW5 ••• C7U .... ~E~..,K05. We do not strictly speaking534
COMMENTARY
prove that or explain why the moon is eclipsed, because it
is not an eternal fact that the moon is eclipsed, but only that
that which has an opaque body interposed between it and its
source of light is eclipsed; the moon sometimes incidentally
has the latter attribute because it sometimes incidentally has
the former.
25. liTL ou Ka8' liAou aUTou ~(n"LV. Bekker's reading 00 Ka8oAou
is preferable to TOU Ka8oAov, which P. 107. 18 describes as occurring
in most of the MSS. known to him. (T. apparently read Ka86Aov
simply (21. 18).) But, with Bekker's reading, aVToii is surprising,
since we should expect aOTwv. I have therefore read Ka8' oAov
aVToii, 'not about a whole species itself'; cf. 74&29 n. What A.
means is that strict demonstration yields a conclusion asserting
a species to have an attribute, but that if we know a particular
thing to belong to such a species, we have an accidental sort of
knowledge that it has that attribute.
27. niv ET€paV ••• lI'pOTaO"LV, the minor premiss, which has for
its subject an individual thing.
27-8. q,8apTiJv I'EV ••• oUCMJ'i. Bonitz (Arist. Stud. iv. 23-4)
argues that the received text on KaL TO CTUfL7TlpaUfLa oiJU7]r makes
A. reason falsely 'The premiss must be non-eternal if the con-
clusion is so, because the conclusion must be non-eternal if the
premiss is so.' He therefore conjectures TOtOVTOV for oiJU7]r. This
gives a good sense, and is compatible with T.'s £t7T£P TO uVfL7TlpaufLa
.p8apTov £UTat (21. 22) and P.'s OLon KaL TO CTUfL7TlpaUfLa .p8apTov
(108. 17). But it is hard to see how TOtOVTOV could have been
corrupted into oiJU7]r. and the true reading seems to be provided
by n-oTt £UTat KaL TO CTUfL7TlpaUfLa oiJU7]r, 'because the conclusion
will exist when the premiss does'. so that if the premiss were
eternal, the conclusion would be so too, while in fact it is ex hy-
pothesi not so. For the genitive absolute without a noun, when
the noun can easily be supplied, cf. Kiihner, Cr. Cramm. ii. 2.
81, Anm. 2.
28-9. I'TJ KI18oAou ••• €~' .r,v. With the reading adopted by
Bekker and Waitz and printed in our text, the meaning will be
that the minor premiss must be particular because the middle
term is at any time true only of some instances of the subject-
genus; with the well-supported reading fL~ Ka8oAov o~ OTt TO fL~V
£UTat TO O~ OOK £UTaL ~(f wv, the meaning will be that the minor
premiss must be particular because at any time only some
instances of the subject term are in existence. The former sense
is the better, and it is confirmed by the example of eclipse of the
moon (b34); for the point there is not that there is a class of535
moons of which not all exist at once, but that the moon has not
always the attribute which, when the moon has it, causes eclipse.
3()-2. 0llo(wS 8' ... ll.1To8€(~€ws. The three kinds of definition
are: (I) a verbal definition of a subject-of-attributes, which needs
no proof but simply states the meaning that everyone attaches
to the name; (z) a causal definition of an attribute, which states
in a concise form the substance of a demonstration showing why
the subject has the attribute; (3) a verbal definition of an attri-
bute, restating the conclusion of such a demonstration without
the premisses (94aII-14). An instance of (I) would be 'a triangle
is a three-sided rectilinear figure' (93b3O-Z). An instance of (z)
would be 'thunder is a noise in clouds due to the quenching of
fire', which is a recasting of the demonstration 'Where fire is
quenched there is noise, Fire is quenched in clouds, Therefore
there is noise in clouds (93b38-94a7). An instance of (3) would be
'thunder is noise in clouds' (94 3 7-9).
Since a definition is either a premiss (i.e. a minor premiss
defining one of the subjects of the science in question), or a
demonstration recast, or a conclusion of demonstration, it must
be a universal proposition defining not an individual thing but
a species.
CHAPTER 9
The premisses of demonstration must be peculiar to the science in
question, except in the case of subaltern sciences
b
7S 37. Since any fact can be demonstrated only from its own
proper first principles, i.e. if the attribute proved belongs to the
subject as such, proof from true and immediate premisses does
not in itself constitute scientific knowledge. You may prove
something in virtue of something that is common to other sub-
jects as well, and then the proof will be applicable to things
belonging to other genera. So one is not knowing the subject to
have an attribute qua itself, but per accidens; otherwise the proof
could not have been applicable to another genus.
76a4. We know a fact not per accidens when we know an
attribute to belong to a subject in virtue of that in virtue of which
it does belong, from the principles proper to that thing, e.g. when
we know a figure to have angles equal to two right angles, from
the principles proper to the subject to which the attribute belongs
per se. Therefore if that subject also belongs per se to its subject.
the middle term must belong to the same genus as the extremes.
9. When this condition is not fulfilled, we can still demonstrateCOMMENTARY
as we demonstrate propositions in harmonics by means of arith-
metic. Such conclusions are proved similarly, but with a differ-
ence; the fact belongs to a different science (the subject genus
being different), but the reason belongs to the superior science,
to which the attributes are per se objects of study. So that from
this too it is clear that a fact cannot be demonstrated, strictly,
except from its own proper principles; in this case the principles
of the two sciences have something in common.
16. Hence the special principles of each subject cannot be
demonstrated; for then the principles from which we demon-
strated them would be principles of all things, and the knowledge
of them would be the supreme knowledge. For one who knows
a thing from higher principles, as he does who knows it to follow
from uncaused causes, knows it better; and such knowledge would
be knowledge more truly-indeed most truly. But in fact
demonstration is not applicable to a different genus, except in
the way in which geometrical demonstrations are applicable to
the proof of mechanical or optical propositions, and arithmetical
demonstrations to that of propositions in harmonics.
26. It is hard to be sure whether one knows or not; for it is
hard to be sure that one is knowing a fact from the appropriate
principles. We think we know, when we can prove a thing from
true and immediate premisses; but in addition the conclusions
ought to be akin to the immediate premisses.
7Sb40' WGTrEP Bpu<1wV TOV TETpaywVLGI'0V. A. refers twice
elsewhere to Bryson's attempt to square the circle-Soph. El.
I7J b I2-J8 Ta yap t/;wSoypa4n1fLaTa OUK £pLUnKC!. (KaTa yap Ta IJ'TTO
rTJV T'XVTJV O{ 1TapaAOYWJ.l-Ot) ,
Y' £ L T{ £un t/;wSoyparPTJJLa 7T£P~
ri).-T}fJf" ofov TO 'Tr:7T0KpaTou, [~ 0 T£Tpayw~'LUJLO, 0 SLa TWV JLTJV{UKWVJ.
riM'
Bpvuwv £T£TpaywvL~£ TOV KVKAov, £l Ka~ T£Tpaywv{~(TaL 0
KJKAo" d>.>.'
ov KaTa TO 7TpiiYfLa, S,a TOiYrO U0rPLUTLKO" 172"2-7
ofov 0 T£TpaywvLUJLO, 0 JLf:V S,a TWV JLTJI'{UKWV OUK £PLaTLKO" 0 Sf:
Bpvuwvo, £P'UTLKO,· KaL TOV JLfv OUK Eun JLET£V£yK~Lv d.).A' ~ 7TPO,
YfWJL£Tp{av JLOVOV "La TO £K TWV lstwv £fvaL dpxwv, TOV Sf: 7TPO,
7T0Mov" aUOL JL~ Luau, TO SuvaTOV £V €KaUTCtJ KaL TO dSvvaTov' apJLouEL
ya.p. The point made in all three passages is the same, that
ouS'
w,
ern
Bryson's attempt is llot scientific but sophistical, or eristic,
because it does not start from genuinely geometrical assumptions,
but from one that is much more general. This was in fact the
assumption that two things that arc greater than the same thing,
and less than the same thing, arc equal to one another (T. 19. 8,
;Po I l l . 27). Bryson's attempt is discussed in T. 19.6-20, P. I l l .537
17-114, 17, ps.-Al. in Soph. El. 90. 10-2!, and in Heath's Hist.
of Gk. M ath. i. 223-S, and in his Mathematics in Aristotle, 48-so.
76"4-9- ~EK(1<TTOV S' _ .. EtVa.L. This difficult passage may be
expanded as follows: 'We know a proposition strictly. not per
accidens, when we know an attribute A to belong to a subject C
in virtue of the middle term B in virtue of which A really belongs
to C, as a result of more primary propositions true of B precisely
as B; e.g. we know a certain kind of figure C to have angles equal
to two right angles (A) when we know it as a result of more
primary propositions true precisely of that (E) to which A belongs
per se. And if, as we have seen, A must belong to B simply as B,
it is equally true that B (KaK€Ll'o) must belong to its subject C
(c[J Imd.pXH) precisely as C. Thus the middle term must belong
to the same family as both the extreme terms; i.e. both premisses
must be propositions of which the predicate belongs to the
subject not for any general reason but just because of the specific
nature of the subject: A. has in mind such a proof as 'The angles
made by a line when it meets another line (not at either end of
the second line) equal two right angles, The angles of a triangle
equal the angles made by such a line, Therefore the angles of a
triangle equal two right angles', where the predicate of each
premiss belongs to that subject precisely as that subject.
16-18. Et SE .•. 1Tc1VTWV. Zabarella supposes A. not to be
denying that metaphysics can prove the apxat of the sciences,
but only that the sciences can prove their own apxat. But it is
impossible to reconcile this interpretation with what A. says.
What he says amounts to denying that there can be a master-
knowledge (aI8) which, like Plato's dialectic, proves the principles
of the special sciences. There is, so far as I know, no trace in A.
of the doctrine Zabarella suggests as his; in the Metaphysics no
attempt is made to prove the apxat of the sciences.
22-4' T] S' a:TT6SEL~L~ ••• a.pjl.OVLKc1~. The connexion of thought
is: If it were possible to prove the first principles of the sciences,
the science that did so would be the supreme science (°16-22);
but in fact no such use of the conclusions of one science as first
principles for another is possible, except where there is something
common to the subject-matters of the two sciences (cf. °IS).
23. w~ E'{Pl1Ta.L, 7SbI4-17, 76a9-1S·COMMENTARY
CHAPTER 10
The different kinds of ultimate premiss required by a science
76-31. The first principles in each genus are the propositions
that cannot be proved. We assume the meaning both of the
primary and of the secondary terms; we assume the existence
of the primary and prove that of the secondary terms.
37. Of the first principles some are special to each science,
others common, but common in virtue of an analogy, since they
are useful just in so far as they fall within the genus studied.
Special principles are such as the· definition of line or straight,
common principles such as that if equals are taken from equals,
equals remain. It is sufficient to assume the truth of such a
principle within the genus in question.
b3. There are also special principles which are assumptions
of the existence of the subjects whose attributes the science
studies; of the attributes we assume the meaning but prove the
existence, through the common principles and from propositions
already proved.
I I . For every demonstrative science is concerned with three
things-the subjects assumed to exist (i.e. the genus), the common
axioms, and the attributes.
16. Some sciences may omit some of these; e.g., we need not
expressly assume the existence of the genus, or the meaning of
the attributes, or the truth of the axioms, if these things are
obvious. Yet by the nature of things there are these three
elements.
z3. That which must be so by its own nature, and must be
thought to be so, is not an hypothesis nor a postulate. There are
things which must be thought to be so; for demonstration does
not address itself to the spoken word but to the discourse in the
soul; one can always object to the former, but not always to the
latter.
z7. Things which, though they are provable, one assumes
without proving are hypotheses (i.e. hypotheses ad hominem) if
they commend themselves to the pupil, postulates if he has no
opinion or a contrary opinion about them (though 'postulate'
may be used more generally of any unproved assumption of what
can be proved).
35. Definitions are not hypotheses (not being assumptions of
existence or non-existence). The hypotheses occur among the
expressed premisses, but the definitions need only be understood;539
and this is not hypothesis, unless one is prepared to call listening
hypothesis.
39- (Nor does the geometer make false hypotheses, as he has
been charged with doing, when he says the line he draws is a foot
long, or straight, when it is not. He infers nothing from this; his
conclusions are only made obvious by this.)
77·3- Again, postulates and hypotheses are always expressed
as universal or particular, but definitions are not.
76&34-5' otov TC ~OVQ~ ••• T9CYIalVov. p.ovas is an example of
7TpwTa (the subjects whose definition and existence are assumed
by arithmetic). £1;00 is put forward as an example of Ta €K TOO-rWV
(whose definition but not their existence is assumed by geometry) ;
this is implied by its occurrence as an instance of Ta. KaO' aUTa
in the second sense of KaO' aVTa (i.e. essential attributes) in 73&38.
Tp[ywvov might have been put forward as an example of Ta 7TpwTa
assumed by geometry; for in 73"35 it occurs among the subjects
possessing KaO' aUTO. in the first sense (Le. necessary elements in
their being). But here it is treated as one of Ta €K TOO-rWV (i.e.
attributes), as being a particular arrangement of lines. This way
of thinking of it occurs clearly in 71&14 and 92bI5. The genus
whose existence arithmetic presupposes is that of p.ovciBfs
(76"35, b4 ) or of ap,Op.ol (75 b5, 76b2, 18, 88b28) ; that whose existence
geometry presupposes is that of 1'0*=1'(01] (75 b5, 76"36, bI , 88b29),
or of points and lines (76b5, cf. 75bI7).
b 9 . il TO KEKA6.a9Q~ il VEUUV. KAiiaOa, is used of a straight line
deflected at a line or surface; cf. Phys. 228b24, Pr. 9I2b29, Euc.
El. iii. 20, Data 89, Apollon. Perg. Con. ii. 52, 3. 52, etc. A. dis-
cusses the problem of avo.KAaa,S" in Mete. 372b34-373"I9, 375bl6--
377328. vEvnv is used of a straight line tending to pass through
a given point when produced; cf. Apollon. Perg. Con. i. 2. al
VEoaE'S" was the title of a work by Apollonius, consisting of pro-
blems in which a straight line of given length has to be placed
between two lines (e.g. between two straight lines, or between
a straight line and a circle) in such a direction that it 'verges
towards' (i.e. if produced, would pass through) a given point
(Papp. 670. 4). It is remarkable that A. should refer to 'verging'
as one of the terms whose definitions must be presupposed in
mathematics; for it played no part in elementary Greek mathe-
matics as it is known to us. Oppermann and Zeuthen (Die Lehre
'v. d. Kegelschnitten im Alterthum, 261 ft.) conjecture that vEoa*=lS"
were in earlier times produced by mechanical means and thus
played a part in elementary mathematics.
TaCOMMENTARY
54 0
10. SUl TE Ti;)V KOWi;)V ••• 6:rroSESE'YIl€VWV, cf. 7 S"41-2 n.
14. Ta. KOWa. AEYOIlEVa. cl.~'~tJ-a.Ta., the axioms which the mathe-
maticians call common (cf. Met. 100S 8 20 Ta €v TO'!j ILa8~ILarn
KaAOVIL£Va d~LwILaTa), though in truth they are common only
KaT' dvaAoy{av, as explained in "38-b2.
23-'7' OUK ian S' ••• cl.EL. A. here distinguishes d~LwILaTa
from V1TO(}£U£L!j and alT~ILaTa. The former are propositions that
are necessarily and immediately (8L' aV-ro) true, and are neces-
sarily thought to be true. They may indeed be denied in words;
but demonstration addresses itself not to winning the verbal
assent of the learner, but to winning his internal assent. He may
always verbally object to our verbal discussion, but he cannot
always internally object to our process of thought.
The phrase 6 ~uw AOYOS was suggested by Plato's Myov OV aV-r~
1TPOS aVT~v ~ .pvx71 8LE~£pXfiTaL 1TEPL ~v av UK07rfi (Theaet. 189 e).
The distinction between aL7TjILa and d~{wILa corresponds (as
B. Einarson points out in A.J.P. lvii (1936), 48) with that between
alTw, 'request', and d~LW, 'request as fair and reasonable' ..
On the terms v1To8EaLS and a'iT'T}p.a. cf. Heath, Mathematics in
Aristotle, 54-7.
2']-9. oaa. tJ-EV o~v ••• U'II'OTL9ETa.,. This sense of V1TO(}EULS, as
the assumption of something that is provable (which is scienti-
fically improper), is to be distinguished from the other sense of
the word in the Posterior A nalytics, in which it means the assump-
tion of something that cannot and need not be proved, viz. of
the existence of the primary objects of a science; cf. 72°18-20,
where it is one kind of aILEuo!j dpxrJ, i.e. of unprovable first prin-
ciple. A.'s logical terminology was still in process of making.
It is probably to distinguish the kind of Inr08EaL!j here referred
to from the other that A. adds KaL EaTLV ovx a1TAw!j lnro8EaL!j ilia
1TpO!j lKELVOV ILoVOV. Such an hypothesis is not something to be
assumed without qualification, since it is provable (presumably
by a superior science (cf. 7 Sb I4- q ) ; but it is a legitimate hypo-
thesis in face of a student of the inferior science who is prepared
to take the results of the superior science for granted.
32-4. Ean ya.p ••• Sd~a.s. The fact that two definitions of
a'l7TjILa are offered indicates that it, like Inr08EaL!j, has not yet
hardened into a technical term.
M. Hayduck (Obs. erit. in aliquot locos Arist. 14), thinking that
a reference to the state of mind of the learner is a necessary part
of the definition of an aL7TjILa, and pointing out that the second
definition given of aLT'T}ILa is equivalent to that given in b27 -8
of the genus which includes V1TO(}EaL!j as well, omits ~ in b 33 . ButI.
10. 76bIO-77a2
541
it is read by P. (129. 8-17) as well as by all the 1\1SS., and 0 av
... Aap-f3&'vTl suggests that a wider sense than that indicated in
b 3 O- 3 is being introduced.
The sense given by A. to atT'I)p-a is quite different from that
given by Euclid to it. Euclid's first three postulates are practical
claims-claims to be able to do certain things-to draw a straight
line from any point to any other, to produce a finite straight line,
to draw a circle with any centre and any radius. The other two,
which Euclid illogically groups with these, are theoretical assump-
tions-the assumptions that all right angles are equal, and that
if a straight line falling on two other straight lines makes interior
angles on the same side of it less than two right angles, the two
straight lines if produced indefinitely will meet-the famous
postulate of parallels. .
35-6. OUSEV ya.p • • • A~y(Tal. Neither oOOt . . . '\£yOVTa~
(Bekkerj nor ooOtv ... >'oiyoVTa~ (Waitz) gives a good sense; it
seems necessary to read ouOtv . . . '\£Y£Ta~. When oooE'y had once
been corrupted into ouoE', the corruption of '\£Y£Ta~ naturally
followed.
3~. la.AA' Cv To.is 'II'pOTO,aEalv ••• au .... 'II'~PCla .... Cl. Hypotheses
must be definitely stated in the premisses (h36), and the conclu-
sions follow from them (h38-<)). Definitions have only to be under-
stood by both parties, and they should not be called hypotheses
unless we are prepared to call intelligent listening a form of
hypothesis or assumption.
39-77"2. ouS' (, y'w .... ~TPT1S ••• ST1Aou .... 'VCl. The statement that
definitions a1'e not hypotheses, because they do not occur among
the premisses on which proof depends, leads A. to point out
parenthetically that the same is true of the geometer's 'let AB be
a straight line'. It does not matter if what he draws is not a
straight line, for what he draws Sffves for illustration, not for
proof. In 77 8 3 A. returns to his main theme.
77"I-z. T~ TtlVS€ ••• i4»8£yKTCll, 'from the line's being the kind
of line he has called it'. The omission of the article between -n)vo£
and ypap-p-~y is made possible by the fact that a relative clause
follows; cf. Kiihner, Gr. Gramm. ii. 1. 628, Apm. 6 (a), which
quotes Thuc. ii. 74 (.rrl. yi]v T~YO£ 71>.8op-£v iv fJ KT>'., and other
passages. But it may be conjectured that we should read Oray
for -rjv and translate 'The geometer infers nothing from this
particular line's being a line such as he has described it as being'.54 2
COMMENTARY
CHAPTER 11
The function of the most general axioms in demonstration
77"5. Proof does not require the existence of Forms-i.e. of
a one apart from the many-but of one predicable of many, i.e.
of a universal (not a mere ambiguous term) to serve as middle
term.
10. No proof asserts the law of contradiction unless it is
desired to draw a conclusion in the form 'C is A and not not-A';
such a proof does require a major premiss' B is A and not not-A '.
It would make no difference if the middle term were both true
and untrue of the minor, or the minor both true and untrue of
itself.
18. The reason is that the major term is assertible not only
of the middle term but also of other things, because it is wider,
so that if both the middle and its opposite were true of the minor,
it would not affect the conclusion.
22. The law of excluded middle is assumed by the reductio ad
impossibile, and that not always in a universal form, but in the
form that is sufficient, i.e. as applying to the genus in question.
26. All the sciences are on common ground in respect of the
common principles (i.e. the starting-points, in distinction from
the subjects and the attributes proved). Dialectic too has com-
mon ground with all the sciences, and so would any attempt to
prove the common principles. Dialectic is not, like the sciences,
concerned with a single genus; if it were, it would not have
proceeded by asking questions; you cannot do that in demonstra-
tion because you cannot know the same thing indifferently from
either of two opposite premisses.
77°5-9. EtS" .... €Y O~Y ••• o . . WYu .... ~Y. T. (21. 7-15) apparently
found this passage, in the text he used, between 75b24 a.'TroSd~£w>
and ib. 25 OVK iu'TLv, and Zabarella transfers it to 75b30. But at
both these points it would somewhat break the connexion. On
the other hand, it would fit in thoroughly well after 83"32-5.
It is clearly out of place in its present position.
12-18. 8dICYUTa.L 8€ .•• QU. A. points out (1) that in order to
get the explicit conclusion 'C is A and not non-A', the major
premiss must have the explicit form • B is A and not non-A'
("12-13). (2) As regards the minor premiss it would make no
difference if we defied the law of contradiction and said 'C is
both Band non-B' (°13-14), since if B is A and not non-A, thenI. II. 7rS-35
543
if C is B (even if it is also non-B), it follows that C is A and not
non-A. To this A. adds (ws S' aVTws Ka~ TO TplTOV, 814-IS) the
further point (3) that it would make no difference if the opposite
of the minor tenn were predicable of the minor tenn, since it
would still follow that C is A and not non-A.
Et ya.p ••• 011 (aIS-I8), 'if it was given that that of which 'man'
can truly be asserted--even if not-man could also be truly
asserted of it (point (2) above)-if it was merely given, I say, that
man is an animal, and not a not-animal (point (I) above), it
will be correct to infer that Callias-even if it is true to say that
he is also not-Calli as (point (3) above)-is an animal and not a
not-animal' .
20-1. ouS' Et ....... 1) aUTO, 'not even if the middle tenn were
both itself and not itself-so that both it and its opposite could
be predicated of the minor tenn.
25. WcnrEP Eip"1Ta~ Kat 1TpOTEPOV, cf. 76342-b2.
27. ICO~Va. SE ••• a.1TOSE~KVUVTES, cf. 7S341-2 n.
29. Kat '" S~CI."'EKnK1) 1TClaa~s. It is characteristic of dialectic
to reason not from the principles peculiar to a particular genus
(as the sciences do) but from general principles. These include
both the axioms, which are here in question, and the vaguer
general maxims called T67TO~, with the use of which the Topics
are concerned.
29-31. Kat Ei ns ••• liTTCI.. Such an attempt would be a meta-
physical attempt, conceived after the manner of Plato's dialectic,
to deduce hypotheses from an unhypothetica1 first principle.
A. calls it an attempt, for there can be no proof, in the strict
sense, of the axioms, since they are G.p-Ea-a. What A. tries to do
in Met. ris rather to remove difficulties in the way of acceptance
of them than to prove them, strictly. It is obvious that no proof
of the law of contradiction, for example, is possible, since all
proof assumes this law.
32. Ol/TIIlS, like a science, or even like metaphysics.
34-5' SESE~KTa~ SE ••• au"''''oy~a .... oQ. The reference is not, as
Waitz and Bonitz's Index say, to An. Pr. 64b7-I3, which deals
with quite a different point, but to An. Pr. S7a36-bI7.544
COMMENTARY
CHAPTER 12
Error due to assttming answers to questions inappropriate to the
science distinguished from that due to assuming wrong answers to
appropriate questions or to reasoning wrongly from true and appro-
priate assumptions. How a science grows
77"36. If that which an opponent is asked to admit as a basis
for syllogism is the same thing as a premiss stating one of two
contradictory propositions, and the premisses appropriate to a
science are those from which a conclusion proper to the science
follows, there must be a scientific type of question from which
the conclusions proper to each science follow. Only that is a
geometrical question from which follows either a geometrical
proposition or one proved from the same premisses, e.g. an optical
proposition.
b 3 . Of such propositions the geometer must render account, on
the basis of geometrical principles and conclusions, but of his
principles the geometer as such must not render account. There-
fore a man who knows a particular science should not be asked,
and should not answer, any and every kind of question, but only
those appropriate to his science. If one reasons with a geometer,
qua geometer, in this way, one will be reasoning well-viz. if one
reasons from geometrical premisses.
I I . If not, one will not be reasoning well, and will not be
refuting the geometer, except per accidens; so that geometry
should not be discussed among ungeometrical people, since among
such people bad reasoning will not be detected.
16. Are there ungeometrical as well geometrical assumptions?
Are there, corresponding to each bit of knowledge, assumptions due
to a certain kind of ignorance which are nevertheless geometrical
assumptions? Is the syllogism of ignorance that which starts from
premisses opposite to the true premisses, or that which is formally
invalid but appropriate to geometry, or that which is borrowed from
another science? A musical assumption applied to geometry is
ungeometrical, but the assumption that parallels meet is in one
sense geometrical and in another not. 'Ungeometrical' is am-
biguous, like 'unrhythmical'; one assumption is ungeometrical
because it has not geometrical quality, another because it is
bad geometry; it is the latter ignorance that is contrary to
geometrical knowledge.
27. In mathematics formal invalidity does not occur so often,
because it is the middle term that lets in ambiguity (having the1. I2
545
major predicated of all of it, and being predicated of all of the
minor-we do not add 'all' to the predicate in either premiss).
and geometrical middle terms can be seen, as it were, by in-
tuition, whereas in dialectical argument ambiguity may escape
notice. Is every circle a figure? You have only to draw it to
see that it is. Are the epic poems a circle in the same sense?
Clearly not.
34. We should not meet our opponent's assumption with an
objection whose premiss is inductive. For as that which is not
true of more things than onc is not a premiss (for it would not be
true of 'all so-and-so', and it is from universals that syllogism
proceeds), neither can it be an objection. For anything that is
brought as an objection can become a premiss, demonstrative
or dialectical.
40. People sometimes reason invalidly because they assume
the attributes of both the extreme terms, as Caeneus does when
he reasons that fire spreads in geometrical progression, since both
fire and this progression increase rapidly. That is not a syllogism;
but it would be one if we could say 'the most rapid progression
is geometrical. and fire spreads with the most rapid progression
possible to movement'. Sometimes it is impossible to reason
from the assumptions; sometimes it is possible but the possibility
is not evident from the form of the premisses.
78a6. If it were impossible to prove what is true from what is
false, it would be easy to resolve problems; for conclusions would
necessarily reciprocate with the premisses. If this were so, then
if A (the proposition to be proved) entails a pair of propositions
B. which I know to be true, I could infer the truth of A from
that of B. Reciprocity occurs more in mathematics, because
mathematics assumes no accidental connexions (differing in this
also from dialectic) but only definitions.
14. A science is extended not by inserting new middle terms.
but (I) by adding terms at the extremes (e.g. by saying' A is
true of B, B of C, C of D', and so ad infinitum); or (2) by lateral
extension, e.g. if A is finite number (or number finite or infinite),
B finite odd number, C a particular odd number, then A is true
of C, and our knowledge can be extended by making a similar
inference about a particular even number.
The structure of this chapter is a very loose one. There is a
main theme-the importance of reasoning from assumptions
appropriate to the science one is engaged in and not borrowing
assumptions from another sphere; but in addition to that source
4485
NnCOMMENTARY
of error A. mentions two others-the use of assumptions appro-
priate to the science but false, and invalid reasoning (77bI8-21)-
and devotes some space to the latter of these two (b 27 - 33 , 40-
78&6). Finally, there are three sections which are jottings having
little conneJ..;on with the rest of the chapter (77b34~, 7886-13,
14-21).
77 8 38--fJ. £r1l a.v ••• £'naT1l1l0VlKOV. A. has just said (833)
£a7'W EPWTa.y; there must therefore be some
change in the meaning of EPWTa.y. When he says the scientist
does not ask questions, he means that the scientist does not, like
the dialectician, ask questions with the intention of arguing from
either answer indifferently ("33-4). The only kind of question
he should ask is one to which he can count on a certain answer
being given, and EpWTTJfLa in this connexi.:Jn therefore = 'as-
sumption'.
bI. ;; a. EK TWV a.UTWV. The received text omits a., and Waitz
tries to defend the ellipse by such passages as A n. Pr. 2Sb3S
KaAW BE fLEUOY fLEY 0 Ka~ aVro EY o.,uCfJ Ka~ llio £y TOVrCfJ EUTly (cf.
An. Post. 81b39, 82"1, b l , 3). 7T£P~ ~y stands for TODTWY 7T£P~ ~Y,
and he takes Tj to stand for ~ TOVrWY a.. But Bonitz truly remarks
(Ar. Stud. iv. 33) that where a second relative pronoun is irregu-
larly omitted or replaced by a demonstrative, the relative pronoun
omitted would have had the same antecedent as the earlier
relative pronoun-a typical instance being 8 1b 39 0 fL1j3£YI inrapXH
£TEPCfJ a.,u' llio EK£lyCfJ (= a,u' c[J llio). a. is necessary; it stands
for TOVrWY a., as 7T£P~ ~y stands for TOVrWY 7T£.p~ ~Y.
3-6. Ka.t WEpt IlEV TOUTlolV ••• YElolll£TP1lS, i.e. when it is possible
to prove our assumptions from the first principles of geometry and
from propositivns already proved, we must so prove them; but
we must not as geometers try to prove the first principles of
geometry; that is the business, iJ oJ anyone, of the metaphysician
a7T03HKYDYTa 01)1<
(cf. 829-31).
18-21. Ka.t WOTEPOV ••• TEXV1lS. P. (following in part 79b23)
aptly characterizes the three kinds of o.YYOLa as follows: (I) ~ KaTa
BLa8£Gw, involving a positive state of opinion about geometrical
questions but erroneous (a) materially or (b) formally, according
as it (a) reasons from untrue though geometrical premisses, or
(b) reasons invalidly from true geometrical premisses, (2) ~ KaTa
a.7TocpaULY, a complete absence of opinion about geometrical
questions, with a consequent borrowing of premisses E~ ti,u1jS"
TEXY1]S"·
The sense requires the insertion of 0 before E~ ti,u1jS" T£XY1]S".
25. Cla1TEp TO a.ppu9Ilov. It is difficult to suppose that A. can547
have written these words here as well as in the previous line; the
repetition is probably due to the similarity of what follows-Ka~
TO f'£V (TEPOV ••. TO S' (TEPOV.
26--]. Ka.l tl a.yvo~a. ••• (va.vTla., 'and this ignorance, i.e. that
which proceeds from geometrical' (though untrue) 'premisses, is
that which is contrary to scientific knowledge'.
32-3. Tt Si ... icnw. The Epic poems later than Homer were
designated by the word KUKAO" but if you draw a circle you see
they are not a circle in that sense, and therefore you will be in
no danger of inferring that they are a geometrical figure.
34-9. Oil S£L S' •.• S~a.}.IKT~KTJ. The meaning of this passage
and its connexion with the context have greatly puzzled com-
mentators, and Zabarella has' dealt with the latter difficulty
by transferring the passage to 17. 81"37, basing himself on T.,
who has no reference to the passage at this point but alludes
rather vaguely to it at the end of his commentary on chs. 16 and
17 (31. 17-24). There is, however, no clear evidence that T. had
the passage before him as part of ch. 17, and nothing is gained by
transferring it to that chapter; all the MSS. and P. have it here.
If we keep the MS. reading in b34 , av ~ 7Tp6Taa,> (7TaKTtK7/, the
connexion must be supposed to be as follows: A. has just pointed
out (bl 6-33) three criticisms that may be made of an attempted
syllogistic argument-that the premisses, though mathematical
in fonn, are false, that the reasoning is invalid, that the premisses
are not mathematical at all. He now turns to consider arguments
that are not ordinary syllogisms (with at least one universal
premiss) but are inductive, reasoning to a general conclusion
from premisses singular in form, and says that in such a case
we must not bring an €VaTaaL, to our opponent's lpwTTJf'a (lil,
atiT6, b 34 ). This is because a proposition used in an €VaTaaL, must
be capable of being a premiss in a positive argument (b 37 -8; cf.
An. Pr. 69 b 3), and any premiss in a scientific argument must be
universal (b3 6-7), while a proposition contradicting a singular
proposition must be singular.
A difficulty remains. In b 34 a singular statement used in induc-
tion is called a 7Tp6TaaL" but in b 3S- 7 it is insisted that a 7Tp6TaaL,
must be universal. The explanation is that a singular proposition,
which may loosely be called a premiss as being the starting-point
of an induction, is incapable of being a premiss of a syllogism
whether demonstrative or dialectical.
The relevance of the passage to what precedes will be the
greater if we suppose the kind of induction A. has in mind to be
that used in mathematics, where a proposition proved to be true
nCOMMENTARY
of the figure on the blackboard is thereupon seen to be true of all
figures of the same kind.
I have stated the interpretation which must be put on the
passage if the traditional reading be accepted. But the text
is highly doubtful. The reference of atJ.r6 is obscure, and we should
expect TTp6s rather than IOZS (cf. 74b19, 76b26). Further, the meaning
is considerably simplified if we read ;v fJ ~ TTp6TaUts ;TTaKn,q and
suppose the TTp6-rauts in question to be not the original premiss but
that of the objector. This seems to have been the reading which
both T. and P. had before them: T. 31. 18 TaS €vcrra.uns TTOt7}T/OV
OVK ;TT' p./pOIJS ouo' dvn,panlCws dvnKnp.tvas d,u' €vavTlas Ka,
Ka86AolJ, P. 157. 22 010' .•. £vtUTap./vovs TaS €vcrra.uns p.~ Ot' ;TTaywyijs
,p/pnv ... dAAa Ka80AtKWS ;vlcrrau8at.
If this reading be accepted the paragraph has much more
connexion with what precedes. There will be no reference to
inductive arguments by the opponent; the point will be that
syllogistic arguments by him must be met not by inductive
arguments, which cannot justify a universal conclusion, but by
syllogistic arguments of our own.
40-1. Iu .... I3a.LVU S' . . . E1I"O .... EVa., i.e. by trying to form a
syllogism in the second figure with two affirmative premisses,
they commit the fallacy of undistributed middle (of which one
variety, ambiguous middle, has already been referred to in
b z7 -
33 ).
41. otov Ka.t;, K,"VEU5 1I"O~EL. P. describes Caeneus as a sophist,
but no sophist of the name is known and P. is no doubt merely
guessing. The present tense implies that Caeneus was either a
writer or a character in literature, and according to Fitzgerald's
canon 0 Katv€us should be the latter. The reference must be to
the KawEVS of A.'s contemporary the comic poet Antiphanes;
A. quotes from the play (fr. II2 Kock) in Rhet. 1407"17, 1413"1,
and in Poet. 1457bz!. The remark quoted in the present passage
is a strange one for a Lapith, but in burlesque all things are
possible.
78aS-6. EvLOTE .... EV o~v ... ;'piiTa.~. Though a syllogism with
two affirmative premisses in the second figure is always, so far
as can be seen from the form (opaTat), fallacious, yet if the pre-
misses are true and the major premiss is convertible, the con-
clusion will be true.
6-13. Et S· ... ;'p~a""ou5. Pacius and Waitz think the move-
ment of thought from A to B here represents an original syllogism,
and that from B to A the proof (Pacius) or the discovery (Waitz)
of the premisses of the original syllogism from its conclusion.549
This interpretation is, however, negated by the fact that A is
represented as standing for one fact (TOVrOV) and B for more than
one (TaSl). Two premisses might no doubt be thought of as a
single complex datum, but since from two premisses only one
conclusion follows, it is impossible that the conclusion of an
ordinary syllogism should be expressed by the plural TaSl. There
must be some motive for the use of the singular and the plural
respectively; and the motive must be (as P. and Zabarella recog-
nize) that the movement from A to B is a movement from a
proposition to premisses-from which, in turn, it may be
established.
When this has been grasped, the meaning of the passage be-
comes clear. dva'\vl!w means not the analysis or reversal of a
given syllogism but the analysis of a problem, i.e. the discovery
of the premisses which will establish the truth of a conclusion
which it is desired to prove. This is just the sense which dVaAvu,S'
bears in a famous passage of the Ethics, III2 b 20. A. says there
o yap {3ovA£v6p.£voS' £O'K£ ~7JT.£rV Ka, dVaAVHV TOV £lp7Jp.lvov Tp6TroV
WO'Tr£P S,,;'ypap.p.a (cpatv£Ta, S' ~ p.£v ~~'T7)a,S' ov Traaa £lvat {3ov'\£va,S',
olov a{ p.a.8T}p.aT'Kat, ~ S( {3ov'\£va,S' Traua ~~T7JU'S'), Ka, TO £axaTOV iv
rfj dvaAvu£, TrPWTOV £lva, iv Tfi y£vlaH. In deliberation we desire
an end; we ask what means would produce that end, what means
would produce those means, and so on, till we find that certain
means we can take forthwith would produce the desired end.
This is compared to the search, in mathematics, for simpler
propositions which will enable us to prove what we desire to
prove-which is in fact the method of mathematical discovery,
as opposed to that of mathematical proof.
This gives the clue to what A. is saying here, viz. If true con-
clusions could only follow from true premisses, the task of
analysing a problem would be easy, since premisses and con-
clusion could be seen to follow from each other (a6-8). We should
proceed as follows. We should suppose the truth of A, which
we want to prove. We should reason 'if this is true, certain
other propositions are true', and if we found among these a pair
B, which we knew to be true, we could at once infer that A is
true (a8-IO). But since in fact true conclusions can be derived
from false premisses (An. Pr. ii. 2-4), if A entails B and B is true
it does not follow that A is true, and so the analysis of problems
is not easy, except in mathematics, where it more often happens
that a proposition which entails others is in turn entailed by
them. This is because the typical propositions of mathematics
are reciprocal; the predicates being necessary to the subjects andCOMMENTARY
55 0
the subjects to the predicates (as in definitions) (aI0-13). Thus, for
instance, since it is because of attributes peculiar to the equilateral
triangle that it is proved to be equiangular, the equiangular
triangle can equally well be proved to be equilateral. This con-
stitutes a second characteristic in which mathematics differs from
dialectical argument (112 ; the first was mentioned in 77b27-33).
The passage may usefully be compared with another dealing
with the method of mathematical discovery, M et. IOSla21-33,
where A. emphasizes the importance of the figure in helping the
discovery of the propositions which will serve to prove the
demonstrand urn.
For a clear discussion of analysis in Greek geometry, see
R. Robinson in Mind, xlv (1936). 464-73.
14-21. A1J€ETa~ S' ... TOO E. The advancement of a science,
says A., is not achieved by interpolating new middle terms. This
is because the existing body of scientific knowledge must already
have based all its results on a knowledge of the immediate pre-
misses from which they spring; otherwise it would not be science.
Advancement takes place in two ways: (I) vertically, by extra-
polating new terms, e.g. terms lower than the lowest minor term
hitherto used (814-16), and (2) laterally, by linking a major term,
already known to be linked with one minor through one middle
term, to another minor through another middle; e.g. if we
already know that 'finite number' (or 'number finite or infinite')
is predicable of a particular odd number, through the middle
term 'finite odd number', we can extend our knowledge by making
the corresponding inference about a particular even number,
through the middle term 'finite even number'. What A. is
speaking of here is the extension of a science by the taking up of
new problems which have a common major term with a problem
already solved; when he speaks of science as coming into being
(not as being extended) by interpolation of premisses, he is think-
ing of the solution of a single problem of the form 'why is BA?'
(cf. 84bI9-8Sa12).
CHAPTER 13
Knowledge of fact and knowledge of reasoned fact
78au. Knowledge of a fact and knowledge of the reason for
it differ within a single science. (I) if the syllogism does not pro-
ceed by immediate premisses (for then we do not grasp the
proximate reason for the truth of the conclusion); (2) (a) if it
proceeds by immediate premisses, but infers not the consequent55!
from the ground but the less familiar from the more familiar of
two convertible terms.
28. For sometimes the term which is not the ground of the
other is the more familiar, e.g. when we infer the nearness of the
planets from their not twinkling (having grasped by perception
or induction that that which does not twinkle is near). We have
then proved that the planets are near, but have proved this not
from its cause but from its effect.
39. (b) If the inference were reversed-if we inferred that the
planets do not twinkle from their being near-we should have
a syllogism of the reason.
So too we may either infer the spherical shape of the moon
from its phases, or vice versa.
n. (3) Where the middle terms are not convertible and the
non-causal term is the more familiar, the fact is proved but not
the reason.
13. (4) (a) So too when the middle term taken is placed outside
the other two. Why does a wall not breathe? 'Because it is not
an animal.' If this were the cause, being an animal should be the
cause of breathing. So too if the presence of a condition is the
cause of an attribute, its absence is the cause of the absence of
the attribute.
21. But the reason given is not the reason for the wall's not
breathing; for not every animal breathes. Such a syllogism is
in the second figure-Everything that breathes is an animal, No
wall is an animal, Therefore no wall breathes.
28. Such reasonings are like (b) far-fetched explanations, which
consist in taking too remote a middle term-like Anacharsis' 'there
are no female flute-players in Scythia because there are no vines'.
32. These are distinctions between knowledge of a fact and
knowledge of the reason within one science, depending on the
choice of middle term; the reason is marked off from the fact
in another way when they are studied by different sciences-
when one science is subaltern to another, as optics to plane
geometry, mechanics to solid geometry, harmonics to arithmetic,
observational astronomy to mathematical.
39. Some such sciences are virtually 'synonymous', e.g. mathe-
matical and nautical astronomy, mathematical harmonics and
that which depends on listening to notes. Observers know the
fact, mathematicians the reason, and often do not know the fact,
as people who know universal laws often through lack of observa-
tion do not know the particular facts.
7~6. This is the case with things which manifest forms but
b...COMMENTARY
55 2
have a distinct nature of their own. For mathematics is concerned
with fonns not characteristic of any particular subject-matter;
or if geometrical attributes do characterize a particular subject-
matter, it is not as doing so that mathematics studies them.
10. There is a science related to optics as optics is to geometry,
e.g. the theory of the rainbow; the fact is the business of the
physicist, the reason that of the student of optics, or rather of
the mathematical student of optics. Many even of the sciences
that are not subaltern are so related, e.g. medicine to geometry;
the physician knows that round wounds heal more slowly, the
geometer knows why they do so.
7Sa:z:z-b31. Tb S' on ... a .... 'IT(Ao~. The distinction between
knowledge of a fact and knowledge of the reason for it, where
both fall within the same science, is illustrated by A. with refer-
ence to the following cases:
(1) (az3-{i) 'if the syllogism is not conducted by way of im-
mediate premisses'. I.e. if D is A because E is A, C is E, and
D is C, and one says 'D is A because E is A and D is E' or
'because C is A and D is C', one is stating premisses which entail
the conclusion but do not fully explain it because one of them
(' D is E', or 'C is A ') itself needs explanation.
(z) Where 'E is A' stands for an immediate connexion and is
convertible and being A is in fact the cause of being E, then (a)
("z6-39) if you reason 'C is A because E is A and C is E' (e.g. 'the
planets are near because stars that do not twinkle are near and
the planets do not twinkle'), you are grasping the fact that C is A
but not the reason for it, since in fact C is E because it is A, not
A because it is E. But (b) (a39- bII ), since 'stars that do not
twinkle are near' is (ex vi materiae, not, of course, ex vi jormae)
convertible, you can equally well say 'the planets do not twinkle,
because stars that are near us do not twinkle and the planets
are near us', and then you are grasping both the fact that the
planets do not twinkle and the reason for the fact.
A. describes this as reasoning St' dfLEUWV (and in this respect
correctly), but only means that the major premiss is afLwo,.
(3) (b lI - I3 ) The case is plainly not improved if, of two non·-
convertible tenns which might be chosen alternatively as middle
tenn, we choose that which is not the cause but the effect of the
other. Here not only does our proof merely prove a fact without
giving the ground of it, but we cannot by rearranging our tenns
get a proof that does this. Pacius illustrates the case by the
syllogism What is capable of laughing is an animal, Man is553
capable of laughing, Therefore man is an animal. Such terms
will not lend themselves to a syllogism Toii ot6n, i.e. one in which
the cause appears as middle term; for we cannot truly say All
animals are capable of laughing, Man is an animal, Therefore
man is capable of laughing.
(4) (a) (b r3- z8) 'When the middle term is placed outside.' In
An. Pr. z6b39, 28"14 A. says that in the second and third figures
TtBfiTat T6 t-daov ;~w TWV aKpwv, and this means that it does not
occur as subject of one premiss and predicate of the other, but
as predicate of both or subject of both. But the third figure is
not here in question, since the Posterior A nalytics is concerned
only with universal conclusions; what A. has in mind is the second
figure (bZ3 - 4). And the detail of the passage (b rs - r 6, z4-8)
('Things that breathe are animals, Walls are not animals, There-
fore walls do not breathe') shows that the case A. has in mind
is that in which the middle term is asserted of the major and
denied of the minor (Camestres)-the middle, further, not being
coextensive with the major but wider in extension than it. Then
the fact that the middle term is untrue of the minor entails
that the major term is untrue of the minor, but is not the precise
ground of its being so. For if Cs non-possession of attribute
A were the cause of its non-possession of attribute B, its pos-
session of A would entail its possession of B; but obviously the
possession of a wider attribute does not entail the possession of
a narrower one.
(b) (bz8-31) A. says that another situation is akin to this, viz.
that in which people, speaking Ka8' iJ7rt,pf3o>"~v, in an extravagant
and epideictic way, explain an effect by reference to a distant
and far-fetched cause. So Anacharsis the Scythian puzzled his
hearers by his riddle 'why are there no female flute-players in
Scythia?' and his answer 'because there are no vines there'. The
complete answer would be: 'Where there is no drunkenness there
are no female flute-players, Where there is no wine there is no
drunkenness, Where there are no vines there is no wine, In
Scythia there are no vines, Therefore in Scythia there are no
female flute-players.' The resemblance of this to case (4 a) is
that in each case a super-adequate cause is assigned; a thing
might be an animal, and yet not breathe, and similarly there
might be drunkenness and yet no female flute-players, wine and
yet no drunkenness, or vines and yet no wine.
Thus the whole series of cases may be summed up as follows:
(I) explanation of effect by insufficiently analysed cause; (z a)
inference to causal fact from coextensive effect; (2 b) explanationCOMMENTARY
554
of effect by adequate (coextensive) cause (scientific expla1tation) ;
(3) inference to causal fact from an effect narrower than the
cause; (4 a) explanation of effect by super-adequate cause, (4 b)
explanation of effect by super-adequate and remote cause.
34-5. TOIITO li' ... aLu9"u£ws. Sometimes. a single observation
is enough to establish, or at least to suggest, a generalization like
this (cf. 9°"26-3°); more often induction from a number of
examples is required.
38. liul. TO EyyVS £tvaL 0':' aT(X130UULV. A. gives his explanation
more completely in De Caelo 29°"17-24.
b z . KaL TO A ••• aT(X13£LV. The sense requires the adoption of
n's reading; the MSS. have gone astray through Kat TO A Tip B
having been first omitted and then inserted in the wrong place.
30. otov TO TOG 'AvaX6.PULOS. Anacharsis was a Scythian who
according to Hdt. iv. 76-7 visited many countries in the sixth
century to study their customs. Later tradition credits him with
freely criticizing Greek customs (Cic. Tusc. v. 32. 90; Dio, Dr.
32.44; Luc. Anach., Scyth.). See also Plut. Solon 5.
32-4. KaTa. TT)v TWV .uuwv 9£ULV • • • uuXXoYLufoL6v, i.e. the
different cases differ in respect of the treating of the causal or
the non-causal term as the middle term, and of the placing of the
middle term as predicate of both premisses (as in case (4 a)) or
as subject of the major and predicate of the minor (as in the
other cases).
34-'79"16. a.XXov liE Tp61TOV ..• Y£WfoL£TpOU. A. recurs here to
a subject he has touched briefly upon in 7 Sb 3- I7 , that of the
relation between pure and applied science. He speaks at first as
if there were only pairs of sciences to be considered, a higher
science which knows the reasons for certain facts and a lower
science which knows the facts. Plane geometry is so related to
optics, solid geometry to mechanics, and arithmetic to harmonics.
Further, he speaks at first as if astronomy were in the same rela-
tion to Ta cpatVo/-,Eva, i.e. to the study of the observed facts about
the heavenly bodies. But clearly astronomy is not pure mathe-
matics, as plane geometry, solid geometry, and arithmetic are.
It is itself a form of applied mathematics. And further, A. goes
on to point out a distinction within astronomy, a distinction
between mathematical astronomy and the application of astro-
nomy to navigation; and a similar distinction within harmonics,
a distinction between mathematical harmonics and ~ KaTa T~V
aKm)v, the application of mathematical harmonics to facts which
are only given us by hearing. The same distinctions are pointed
out elsewhere. In An. Pr. 46"19-21 A. distinguishes astronomical555
experience of ni q,au,6p.Eva from the astronomical science which
discovers the reasons for them. Thus in certain cases A. recog-
nizes a threefold hierarchy, a pure mathematical science, an
applied mathematical science, and an empirical science--e.g.
arithmetic, the mathematical science of music, and an empirical
description of the facts of music; or solid geometry, the mathe-
matical science of astronomy, and an empirical description of
the facts about the heavenly bodies (which is probably what he
means by VaVTLK-f] dQ'TpO>'OYLK~) ; or plane geometry, the geometrical
science of optics, and the study of the rainbow (79"10-13). Within
such a set of three sciences, the third is to the second as the
second is to the first (ib. 10-II); in each case the higher science
knows the reason and the lower knows the fact (78b34~, 79"2--6,
II-I3). Probably the way in which A. conceives the position is
this: The first science discovers certain very general laws about
numbers, plane figures, or solids. The third, which is only by
courtesy called a science, collects certain empirical facts. The
second, borrowing its major premisses from the first and its minor
premisses from the third, explains facts which the third discovers
without explaining them. Cf. Heath, Mathematics in Aristotle,
5 8--6r.
35. T~ SL' liAA"s ••• OEWpELV. TcfJ (read by nand p) is obviously
to be read for the vulgate T6.
39-40. aXESov SE ••• ~1TLaT"f.LWv. avvwvvp.a are things that
have the same name and the same, definition (Cat. 1'(6), and T.
rightly remarks that in the case of the pure and applied sciences
mentioned by A. TO ovop.a TO aUTO KaL & >'6yos- ou 1T(lVTTl ET£pOS-.
79a4~'
KO,Oc11TEP ot TO KaOoAou OEWPOUVTES ••• a,vE1TLaK~Lav.
The possibility of this has been examined in An. Pr. 67a8_bII.
8-9. ou ya,p ••• U1TOKELf.LEVOU, 'for mathematics is not about
forms attaching to particular subjects; for even if geometrical
figures attach to a particular subject, mathematics does not study
them qua so doing'.
n. TO 1TEpl Tils LPLSOS, not, as Waitz supposes, the study of the
iris of the eye, but the study of the rainbow (so T. and P.).
12-13. TO SE SLon ••• f.LaO"f.La, 'while the reason is studied by
the student of optics-we may say "by the student of optics"
simply, or (taking account of the distinction between mathe-
matical and observational optics, cf. 78b4Q-79"2) "by one who
is a student of optics in respect of the mathematical theory of
the subject" '.
14-16. on f.LEV yap ••• YEWf.L(TPOU. P. gives two conjectural
explanations: (I) 'because circular wounds have the greatestCOMMENTARY
55 6
area relatively to their perimeter' ; (2) (which he prefers) 'because
in a circular wound the parts that are healing are further separated
and nature has difficulty in joining them up' (sc. by first or second
intention as opposed to granulation) (182. 21-3). He adds that
doctors divide up round wounds and make angles in them, to
overcome this difficulty.
CHAPTER 14
The first figure is the figure of scientific reasoning
79"17. The first figure is the most scientific; for (r) both the
mathematical sciences and all those that study the why of things
couch their proofs in this figure.
24. (2) The essence of things can only be demonstrated in this
figure. The second figure does not prove affirmatives, nor the
third figure universals; but the essence of a thing is what it is,
universally.
29. (3) The first figure does not need the others, but the inter-
stices of "a proof in one of the other figures can only be filled up
by means of the first figure.
79"25-6. EV .... EV ya.p
T~ .... EO''l:! ••• O'U~~OyLO'''''OS,
proved in An.
Pr. i. 5.
27-8. EV Se T~ EO'Xcl.T'l:! .•• ou l(aeO~OU, proved in An. Pr. i. 6.
29-31. ETL TOUTO • • • EMU' With two exceptions, every valid
mood in the second or third figure has at least one universal
affirmative premiss, which can itself be proved only in the first
figure. The two exceptions, Festino and Ferison, have a major
premiss which can be proved only by premisses of the form AE
or EA, and a minor premiss which can be proved only by pre-
misses of the form AA, lA, or AI, and an A proposition can itself
be proved only in the first figure.
30. l(aTa1l"UI(VOUTaL. B. Einarson in A.J.P.lvii (1936),158, gives
reasons for supposing that this usage of the term was derived
from the use of it to denote the filling up of a musical interval
with new notes.
CHAPTER 15
There are negative as well as affirmative propositions that are
immediate and indemonstrable
79"33. As it was possible for A to belong to B atomically, i.e.
immediately, so it is possible for A to be atomically deniable of
B. (I) When A or B is included in a genus, or both are, A cannot557
be atomically deniable of B. For if A is included in rand B is
not, you can prove that A does not belong to B: 'All A is r, No
B is r, Therefore no B is A.' Similarly if B is included in a genus,
or if both are.
b S• That B may not be in a genus in which A is, or that A
may not be in a genus in which B is, is evident from the existence
of mutually exclusive chains of genera and species. For if no
term in the chain Ar.::l is predicable of any term in the chainBEZ,
and A is in a genus e which is a member of its chain, B will
not be in e; else the chains would not be mutually exclusive.
So too if B is in a genus.
12. But (2) if neither is in any genus, and A is deniable of B,
it must be atomically deniable of it; for if there were a middle term
one of the two would have to be in a genus. F:or the syllogism
would have to be in the first or second figure. If in the first, B
will be in a genus (for the minor premiss must be affirmative) ;
if in the second, either A or B must (for if both premisses are
negative, there cannot be a syllogism).
79&33. "nCJ1rEp SE ••• CLT6I'WS' This was proved in ch. 3.
36-7. 3TClV j.1Ev o~v ••• o.j.1'w. The reasoning in &38-bI2 shows
that by these words A. means 'when either A is included in a
genus in which B is not, or B in a genus in which A is not, or
A and B in different genera'. He omits to consider the case in
which both are in the same genus. The only varieties of this
that need separate consideration are the case in which A and B
are inftmae species of the same genus, and that in which they are
members of the same inftma species; for in all other cases they
will be members of different species, and the reasoning A. offers
in &38-bI2 will apply. If they are inftmae species of the same
genus, they will have different differentiae E and F, and we can
infer No B is A from All A is E, N.o B is E, or from No A is F,
All B is F. A. would have, however, to admit that alternative
differentiae, no less than summa genera or categories, exclude
e:ich other immediately. The case in which A and B are members
of the same inftma species would not interest him, since through-
out the Posterior A nalytics he is concerned only with relations
between universals.
b l - 2 • oj.1olws SE ••• A, sc. Ka, 'TO A ,.,.~ ;O"TW ill o'\<fJ 'TijI .::I.
7. EK TWV CJUCJTOLXLWV. CTVO"TO,xta is a word of variable meaning
in A., but stands here, and often, for a chain consisting of a genus
and its species and sub-species.
15-20. 1i yap •.. ECJTClL. Only the first and second figure canCOMMENTARY
558
prove a universal negative, and in these only Celarent and Cesare,
in which the minor premiss includes the minor term in the
middle term, and Camestres, in which the major premiss includes
the major term in the middle term.
CHAPTER 16
Error as inference of conclusions whose opposites are
immediately true
79b23. Of ignorance, not in the negative sense but in that in
which it stands for a positive state, one kind is false belief formed
without reasoning, of which there are no determinable varieties;
another is false belief arrived at by reasoning, of which there are
many varieties. Of the latter, take first cases in which the terms
of the false belief are in fact directly connected or directly dis-
connected.
(A) A directly deniable of B
Both premisses may be false, or only one.
33. If we reason All C is A, All B is C, Therefore all B is A,
(a) both premisses will be false if in fact no C is A and no B is C.
The facts may be so; since A is directly deniable of B, B cannot
(as we have seen) be included in C, and since A need not be true
of everything, in fact no C may be A.
40. (b) The major premiss cannot be false and the minor true;
for the minor must be false, because B is included in no genus.
80 8 2. (c) The major may be true and the minor false, if A is
in fact an atomic predicate of C as well as of B; for when the
same term is an atomic predicate of two terms, neither of these
will be included in the other. It makes no difference if A is
not an atomic predicate of C as well as of B.
6.
(B) A directly assertible of B
While a false conclusion All B is A can only be reached, as
above, in the first figure, a false conclusion No B is A may be
reached in the first or second figure.
9. (r) First figure. If we reason No C is A, All B is C, Therefore
no B is A, (a) if in fact A belongs directly both to C and to B,
both premisses will be false.
J4. (b) The major premiss may be true (because A is not true
of everything). and the minor false, because (all B being A)
all B cannot be C if no C is A ; besides, if both premisses were
true, the conclusion would ·be so.559
21.
(c) If B is in fact included in C as well as in A, one of the
two (C and A) must be under the other, so that the major premiss
will be false and the minor true.
27. (z) Second figure. (a) The premisses must be All A is C,
NoB is C, or No A is C, All B is C. Both premisses cannot be
wholly false; for if they were, the truth would be that no A is C
and all B is C, or that all A is C and no B is C, but neither of
these is compatible with the fact that all B is A.
33. (b) Both the premisses All A is C, No B is C may be partly
false; some A may not be C, and some B be C.
37. So too if the premisses are No A is C, All B is C ; some A
may be C, and some B not C.
38. (c) Either premiss may be wholly false. All B being A,
(a) what belongs to all A will belong to B, so that if we reason
All A is C, NoB is C, Therefore no B is A, if the major premiss
is true the minor will be false.
b2 • (f3) What belongs to no B cannot belong to all A, so that
(with the same premisses) if the minor premiss is true the major
will be false.
6. (y) What belongs to no A will belong to no B, so that if we
reason No A is C, All B is C, if the major is true the minor must
be false.
10. (8) What belongs to all B cannot belong to no A, so that
(with the same premisses) if the minor is true the major must be
false.
14. Thus where the major and minor terms are in fact directly
connected or disconnected, a false conclusion can be reached from
two false premisses or from one true and one false premiss.
A. begins with a distinction between aYV0L<l as the negation of
knowledge, i.e. as nescience, and ayvota as a positive state, i.e.
as wrong opinion-a distinction already drawn in 77bZ4 TO p.;'v
£TEPOV ay£wp.£TP'T}TOV Tep p.~ EXEtV ... TO 8' £TEPOV Tep q,avAw!> EXEtV'
Ka~ ~ ayvota aiJ-rry ... lvav-ria. He first (79bz4) identifies the latter
with wrong opinion reached by reasoning, but later (1)z5-8) cor-
rects himself by dividing it into wrong opinion so reached and
that formed without reasoning. Wrong opinion of the former
kind admits of different varieties; that of the latter kind is
d,1TA1j, i.e. does not admit of varieties of which theory can take
account (b z 8) ; and A. says nothing more about it. Finally, wrong
opinion based on reasoning is divided according as the term
which forms the predicate of our conclusion is in fact directly,
or only indirectly, assertible or deniable of the term which formsCOMMENTARY
560
our subject. The case of tenns directly related is discussed in
this chapter, that of tenns indirectly related in the next, aYVOtCl
in the sense of nescience in ch. 18.
79bJ']-8. TO fla. yelp B ... U1TelPXEIV. That the subject of an
unmediable negative proposition cannot be included in a whole,
i.e. must be a category, was argued in b 1- 4.
80"2-5. Tt,V SE Ar ... U1TelPXEL, 'but the premiss All C is A
may be true, i.e. if A is an atomic predicate both of C and of B
(for when the same tenn is an atomic predicate of more than one
tenn, neither of these will be included in the other). But it makes
no difference if A is not an atomic predicate of both C and B:
The case in question is that in which in fact All C is A, no B is C.
and no B is A; therefore Inrapxn in "3 and 5 and KaT7)yopTjTat in
33 must be taken to include the case of deniability as well as that
of assertibility; and this usage of the words is not uncommon in
the A nalytics ; cf. 82814 n. And in fact, whether A is immediately
assertible of both C and B, immediately deniable of both, or
immediately assertible of one and deniable of the other, C cannot
be included in B, or B in C; in the first case they will be coordinate
classes immediately under A, in the second case genera outside
it and one another; in the third case one will be a class under A
and one a class outside A.
In "2-4 A. assumes that A is directly assertible of C and directly
deniable of B. But, he adds in a4-5, it makes no difference if it
is not directly related to both. That it is directly deniable of B
is the assumption throughout 79b29-8oa5; what A. must mean is
that it makes no difference if it is not directly assertible of C (i.e.
if C is a species of a genus under A, instead of a genus directly
under A). And in fact it does not; the facts will still be that all
C is A, no B is C, and no B is A.
In "4 (V should be read before oVO£Tl.pcp, as it is by one of the
best MSS. and by P. (196. 28).
']-8. ou yap ••• auAAoYLafl6~. IJ7rapxuv stands for KafJoAov
inrapxuv; for it is with syllogisms yielding the false conclusion
All B is A that A. has been concerned. He has shown in An. Pr.
i. 5 that the second figure cannot prove an affirmative, and
ib. 6 that the third cannot prove a universal.
15-~0. EYXWPEL yap ••• ciATJ8E~. The situation that is being
examined in "6-b16 is that in which A is directly true of all B,
and we try to prove that no B is A. If we say No C is A, All
B is C, Therefore no B is A, the major premiss may be true
because A is not true of everything and there is no reason why
it need be true of C; and if the major is true, the minor not only56r
may but must be false, because, all B being A, if all B were C
it could not be true that no C is A. Or, to put it otherwise, if
both No C is A and All B is C were (fl Kat, "19) true, it would
follow that No B is A is true, which it is not.
23. Q.VclYICTJ yap ••• dvcu. A. must mean that A is included
in C; for (I) A cannot fall outside C, since ex hypothesi B is in-
cluded in both, and (2) A cannot include C, since if all C were A,
then, all B being C, All B is A would be a mediable and not (as
it is throughout a8- b r6 assumed to be) an immediate proposition.
A. ignores the possibility that A and C should be overlapping
classes, with B included in the overlap.
2']-33. (lAl1s f1£Y EtVI1' Ta,> 1TpOTclaEL'> Q.f1tOTEPI1S IjtEUSEL'> •••
E1TL n 5' EKI1TEPI1V OUSEV KWAllI;:~ IjtwSij etVI1~. 'All B is A' is wholly
false when in fact no B is A ; 'No B is A' wholly false when in
fact all B is A ; 'All B is A' and 'No B is A' are partly false when
in fact some B is A and some is not (cf. An.Pr. 53b28-30 n.).
32-3' Et OOV ••• Q.SUVI1TOV, 'if, then, taken thus (i.e. being
supposed to be All A is C, No B is C, or No A is C, All B is C), the
premisses were both wholly false, the truth would be that no A
is C and all B is C, or that all A is C and no B is C; but this is
impossible, because in fact all B is A ("28).
b 9 . ~ f1£Y r A 1TpOTl1alS. TA must be read, as in bl and 14;
for A. always puts the predicate first, TA standing for ~TL T~ T
T<fJ A oUX lnraPXH. Cf. 81 8 11 n., r9 n.
CHAPTER 17
Error as inference of conclusions whose opposites can be
proved to be true
(A) A assertible of B through middle term C
(I) First figure. (a) When the syllogism leading to a false
conclusion uses the middle term which really connects the terms,
both premisses cannot be false. To yield a conclusion, the minor
premiss must be affinnative, and therefore must be the true
proposition All B is C. The major premiss will be the false
proposition No C is A.
26. (b) If the middle tenn be taken from another chain of
predication, being a tenn D such that all D is in fact A and all
B is D, the false reasoning must be No D is A, All B is D, There-
fore no B is A ; major premiss false.
32. (c) If an improper middle tenn be used, to give the false
conclusion NoB is A th e premisses used must be NoD is A,
SObI7.
00COMMENTARY
All B is D. Then (a) if in fact all D is A and no B is D, both
premisses will be false.
40. (f3) If in fact no D is A and no B is D, the major will be
true, the minor false (for if it had been true the conclusion No
B is A would have been true).
8I a s. (2) Second figure. (a) Both premisses (All A is C, No
B is C, or No A is C, All B is C) cannot be wholly false (for when
B in fact falls under A, no predicate can belong to the whole
of one and to no part of the other).
9. (b) If all A is C and all B is C, then (a) if we reason All
A is C, No B is C, Therefore no B is A, the major will be true
and the minor false.
12. (f3) If we reason No A is C, All B is C, Therefore no B is
A, the minor will be true and the major false.
(B) A deniable of B through C
(a) If the proper middle term be used, the two false premisses
all C is A, No B is C, would yield no conclusion. The premisses
leading to the false conclusion must be All C is A, All B is C;
major false.
20. (b) If the middle term be taken from another chain of
predication, to yield the false conclusion All B is A the premisses
must be All D is A, All B is D, when in fact no D is A and all
B is D; major false.
24. (c) If an improper middle term be used, to yield the false
conclusion All B is A the premisses must be All D is A, All B is
D. Then in fact (a) all D may be A, and no B be D; minor false;
29 or (f3) no D may be A, and all B be D; major false;
31 or (y) no D may be A, and no B be D; both premisses
false.
3S. Thus it is now clear in how many ways a false conclusion
may be reached by syllogism, whether the extreme terms be
in fact immediately or mediately related.
80bI7-8Ia4' 'Ev l)~ TOL!; (.1'; ciTo(.1W!; ••• V€ul)o<;. A. considers
here the case in which All B is in fact A because it is C. The
possible ways in which we may then reach a false negative con-
clusion, in the first figure, are the following:
(r) (80br8-26) We may misuse the OlKE'iOJl piaoJl C by reasoning
thus: No C is A, All B is C, Therefore no B is A. We use a major
premiss which is the opposite of the truth, but there is no dis-
torting of the minor premiss (ov yap Q.V'T'a7'p€cpera,, b24 ; for this
use of Q.V'T'a7'P€CPEtJl cf. An. Pr. 4Sb6 and ii. 8-ro passim) ; for in the1. 17. 80 b 17- 818 34
first figure the minor premiss must be affirmative and the affirma-
tion All B is C is true.
(2) (b26-32) We may use a middle term Jg n,U7J' ava'To,xlo.., i.e.
one which is not the actual ground of the major term's being true
of the minor, but yet entails the major and is true of the minor.
The facts being that all D is A, all B is D, and therefore all B
must be A, we reason No D is A, All B is D, Therefore no B is
A; as before, our major is false and our minor true (b 3r - 2).
(3) (b32-8ra4) We may reason JL~ S,d 'TOU OlKdov JLEaOV, use a
middle term which is not in fact true of the minor term. Our
reasoning is again No D is A, All B is D, Therefore no B is A
(the only form of reasoning which gives a universal negative
conclusion in the first figure), while the facts may be either that
all D is A and no B is D, in which case both our premisses are
false (b 33- 40 ), or that no D is A and no B is D, in which case our
major is true and our minor false (b4o-8r84).
35-7. A"11rT(a.~ yo.p ••• IjIEUSELS. D in fact entails A, and B
in fact does not possess the attribute D. But to get the conclusion
No B is A we must (to fall in with the rules of the first figure,
as stated in An. Pr. i. 4) have as premisses No D is A and All
B is D-both false.
81"5-8. lJuo. S( TOU !-1(aou aX';!-1a.ToS ••• 1rponpov. The situa-
tion is this: In fact all B is A. To reach the false conclusion
No B is A in the second figure, we must use the premisses All
A is C, No B is C, or No A is C, All B is C. If in either case both
premisses were wholly false (i.e. contrary, not contradictory, to
true propositions), in fact no A would be C and all B would be
C, or all A would be C and no B would be C. But, all B being in
fact A, neither of these alternatives can be the case. Ko.8a.7T£p
J>..Ex81] KO.' 7TPO'T£POV refers to 80"27-33, where the same point was
made about the case in which A is immediately true of B.
II. ij !-1£v r A. r A must be read; cf. 80 b9 n., 8r"r9 n.
19. Ka.9a.1r(p (A(X9"1 Ka.t 1rponpov, i.e. in 80 b22-S.
11)-20. wan ij Ar ... Q.vT~aTp(4)o!-1(v''1' Ar must be read; cf.
8ob9 n., 8r a I l n. For the meaning of ~ dV'TLcrrP£<P0JLEV7] cf. 80 bq-
8r"4 n.
20-4. 0!-10LWS S( ••• 1rponpov. For the meaning of 'taking the
middle term from another chain of predication', cf. 80bq-8ra4 n.
21-2. W(71r(P (AiX9"1 ••• Q.1ra.T"1S cf. 80b26-32.
24. Tn 1rpOTEPOV, i.e. that described in 8r9-2o.
24-34. OTa.V S( .•. ;TUX(V. A. here recognizes three cases of
reasoning JL~ S,d 'TOU OlK£{OV. The reasoning in all three is All D
is A, All B is D, Therefore all B is A. The facts are (I ) that allCOMMENTARY
D is A, no B is D, and no B is A ("24-7), (2) that no D is A, all
B is D, and no B is A ("29-31), (3) that no D is A, no B is D,
and no B is A ("31'-2). The second of these cases, however, is
identical with that described in a2cr-4 as reasoning with a middle
Uvu'TOLx[a>, but there ought to be this difference between
reasoning fL~ s,a TOU OlK€[OV and reasoning with a middle term
ig lli1» UVu'To,x[a>, that in the latter by correcting the false
premiss we should get a correct (though unscientific) syllogism
giving a true conclusion, whereas in the former if we correct the
false premiss or premisses we do not get a conclusion at all (cf.
the distinction between the two types of error in 80b26-32, 32-
8184). It will be seen that the first and third cases cited as cases
of reasoning fL~ o,a 'TOU OlK€[OV are really cases of it (answering
to the two cited in 80b32-81a4), while the second is really a case
of reasoning with a middle term ig a'\'\1» avuTo,x[a>.
The final sentence betrays still greater confusion. It says that
if the middle term does not in fact fall under the major term,
both premisses or either may be false. But if the middle term does
not in fact fall under the major, the major premiss is inevitably
false, since (the conclusion being All B is A) the major premiss
must be All D is A. So great a confusion within a single sentence
can hardly be ascribed to A., and there is no trace of this sentence
in P.'s commentary (Kat 'Ta ig.Tj> in P. 213. 12 is omitted by one
of the two best MSS.).
26-7. EYXIalPEL yap . .. lI.AATJACI., i.e. A may be truly-assertible-
or-deniable of two terms (in this case assertible of D, deniable
of B) without either of them falling under the other. IJ7Tapx£U>
has the same significance as in 8083 and 5.
;g a""1»
CHAPTER 18
Lack of a sense must involve ignorance of certain universal pro-
positions which can only be reached by induction from particular
facts
.
81"38. If a man lacks any of the senses, he must lack some
knowledge, which he cannot get, since we learn either by induc-
tion or by demonstration. Demonstration is from universals,
induction from particulars; but it is impossible to grasp universals
except through induction (for even abstract truths can be made
known through induction, viz. that certain attributes belong to
the given class as such-even if their subjects cannot exist
separately in fact), and it is impossible to be led on inductively to1. 17. 81"26--]
the universals if one has not perception. For it is perception
that grasps individual facts; you cannot get scientific knowledge
of them; you can neither deduce them from universal facts
without previous induction, nor learn them by induction without
perception.
The teaching of this chapter is that sensuous perception is the
foundation of science. The reason is that science proceeds by
demonstration from general propositions, themselves in demon-
strable, stating the fundamental attributes of a genus, and that
these propositions can be made known only by intuitive induction
from observation of particular facts by which they are seen to
be implied. The induction must be intuitive induction, not
induction by simple enumeration nor even 'scientific' induction,
since neither of these could establish propositions having the
universality and necessity which the first principles of science
have and must have.
The induction in question is. said to be £K TWV KaTd /.dpo, (b I ),
and this leaves it in doubt whether A. is thinking of induction
from species to the genus, or from individuals to the species. But
since induction is described as starting from perception, it is clear
that the first stage of it would be from individual instances, and
that induction from species to genus is only a later stage of the
same process.
Even abstract general truths, says A. (b 3), can be made known
by induction. He treats it as obvious that general truths about
classes of sensible objects must be grasped by induction from
perceived facts, but points out that even truths about things (like
geometrical figures) which have no existence independent of
sensible things (Ka1 £l p.TJ XWPLCTTa. (CTTLV, b 4 ) are grasped by means of
an induction from perceived facts, which enables us to grasp,
e.g. that a triangle, whatever material it is embodied in, must
have certain attributes. By these he means primarily, perhaps,
the attributes included in its definition. But the apxa{ referred
to include also the a~Lwp.aTa or KOLva1 apxa{ which state the
fundamental common attributes of all quantities (e.g. that the
sums of equals are equal), and even those of all existing things
(like the law of contradiction or that of excluded middle); and
also the imo8lu£L, in which the existence of certain simple
entities like the point or the unit is assumed. For since no apX'7 of
demonstration can be grasped by demonstration, all the kinds
of apX'7 of science (72&14-24) must be grasped by induction from
sense-perception.COMMENTARY
566
The passage contains the thought of a teacher instructing
pupils-that at least is the most natural interpretation of yvwP1/La
7TOIE'V (b 3); and the same thought is carried on in the word
E7Tax8fjval (b S). 'It is impossible for learners to be carried on to
the universal unless they have sense-perception.' The passage
is one of those that indicate that the main idea underlying A.'s
usage of the word £7TaywY17 is that of this process of carrying on,
not that of adducing instances. Other passages which have the
same implication are 71'2 I, 24, M et. 989'33; cf. PI. Polit. 278 as,
and E7TavaywY17 in Rep. 532 cS; cf. also my introductory note on
An. Pr. ii. 23. The process of abstracting mathematical entities
from their sensuous embodiment (which is what A. has at least
chiefly in mind when he speaks of Ta E{ d.paLp£UEW,) is most fully
described in M et. 1061'28-b3.
The sum of the whole matter is that sense-perception is the
necessary starting-point for science, since 'we can neither get
knowledge of particular facts from universal truths without
previous induction to establish the general truths, nor through
induction without sense-perception for it to' start from' (b 7 -9)'
CHAPTER 19
Can there be an infinite chain of premisses in a demonstration,
(1) if the primary attribute is fixed, (2) if the ultimate subject is fixed,
(3) if both terms are fixed?
8 I b I o. Every syllogism uses three terms; an affirmative
syllogism proves that r is A because B is A and r is B ; a negative
syllogism has one affirmative and one negative premiss. These
premisses are the starting-points; it is by assuming these that
one must conduct one's proof, proving that A belongs to r
through the m('diation of B, again that A belongs to B through
another middle term, and B to r similarly.
18. If we are reasoning dialectically we have only to consider
whether the inference is drawn from the most plausible premisses
possible, so that if there is a middle term between A and B but
it is not obvious, one who uses the premiss 'B is A' has reasoned
dialectically; but if we are aiming at the truth we must start from
the real facts.
23. There are things that are predicated of something else
not per accidens; by per accidens I mean that we can say 'that
white thing is a man', which is not like saying 'the man is white;'
for the man is white without needing to be anything besides beinga man, but the white is a man because it is an accident of the
man to be white.
30. (I) Let be something that belongs to nothing else, while
B belongs to it directly, Z to B, and E to Z; must this come to
an end, or may it go on indefinitely? (z) Again, if nothing is
assertible of A per se, and A belongs to e directly, and 8 to H,
and H to B, must this come to an end, or not?
37. The two questions differ in that (I) is the question whether
there is a limit in the upper direction, (z) the question whether
there is a limit in the lower.
8z·z. (3) If the ends are fixed, can the middle terms be indeter-
minate in number? The problem is whether demonstration pro-
ceeds indefinitely, and everything can be proved, or whether
there are terms in immediate contact.
9. So too with negative syllogisms. If A does not belong to
any B, either B is that of which A is immediately untrue or there
intervenes a prior term H, to which A does not belong and which
belongs to all B, and beyond that a term e to which A does not
belong and which belongs to all H.
15. The case of mutually predicable terms is different. Here
there is no first or last subject; all are in this respect alike, no
matter if our subject has an indefinite number of attributes,
or even if there is an infinity in both directions; except where
there is per accidens assertion on one side and true predication
on the other.
r
Chs. I9-Z} form a continuous discussion of the question whether
there can be an infinite chain of premisses in a demonstration.
In ch. 19 this is analysed into the three questions: (I) Can there
be an infinite chain of attributes ascending from a given subject?
(z) Can there be an infinite chain of subjects descending from a
given attribute? (3) Can there be an infinite number of middle
terms between a given subject and a given attribute? Ch. zo
proves that if (I) and (z) are answered negatively, (3) also must
be so answered. Ch. Zl proves that if an affirmative conclusion
always depends on a finite chain of premisses, so must a negative
conclusion. Ch. zz proves that the answers to (I) and (z) must be
negative. Ch. Z3 deduces certain corollaries from this.
8 I bZ O--Z. t:Jcrr' Et • • • SUJ.~EKnKws. There is here a disputed
question of reading. A2 B2 Cdn 2 and P. (zI8. 14) have £crn,
Al BI n l p.~ E(rn. B2 dn have SE p.~, A2 C2 (apparently) SE p.~
£lvo." Bl S£, Al Cl S£ t:lvo.t. The presence or absence of t:lvo., does
not matter; what matters is, where p.~ belongs. The readingCOMMENTARY
568
with f..'~ in the earlier position has the stronger MS. support, but
the clear testimony of P. may be set against this. oui "TOVTOU,
however, is decisive in favour of the reading f..'~ [un . . . OOKI:'
O~
I:lVaL.
24-9. E1TELSTJ EO"TLV •.. Ka.TTlYopELa9a.L. A. is going to assume
in b 3 O- 7 that there are subjects that are not attributes of any-
thing, and attributes that are not subjects of anything. But he
first clears out of the way the fact that we sometimes speak as if
each of two things could be predicated of the other, as when we
say 'that man is white' and 'that white thing is a man'. These,
he says, are very different sorts of assertion. The man does not
need to be anything· other than a man, in order to be white; the
white thing is a man ("TO AWKOV in b 2 8 is no doubt short for this)
in the sense that whiteness inheres in the man. A. is hampered
by the Greek idiom by which "TO AWKOV may mean either 'white
colour' or 'the white thing'. What he is saying is in effect that
'man' is the name of a particular substance which exists in its
own right, 'white' the name of something that can exist only by
inhering in a substance. At the end of the chapter ("TO S' w!;
lCa"TTJyopi.av, 82·20) he implies that 'the white thing is a man' is
not a genuine predication, and he definitely says so in 83"14-17.
8za6-S. Eo"TL SE ••• 1Tl:pa.LV€Ta.L. This seems to refer to the last
of the three questions stated in 8rb30-82"6. I:l a{ d7TOSd~H!; I:l!;
a.7Tnpov [PXOV"TaL might refer to any of the three; but 1:4 [any
a7TOOH~L!; Q,7TaV"TO!; refers to the third, for the absence of an ultimate
subject or of an ultimate predicate would not imply that aU
propositions are provable; there might still be immediate con-
nexions between pairs of terms within the series. 7TPO!; Ci.,\ATJAa
7Tl:patVI:"TaL means that some terms are bounded at each other.
'touch' each other; in other words that there are terms with no
term between them. Fina.lly, it is the third question that is
carried on into the next paragraph.
9-14. 'OI'OLWS SE ••• to"Ta.Ta.L. Take the proposition No B is A.
Either this is unmediable, or there is a term H such that no H js
A, and all B is H. Again either No H is A is unmediable, or there
is a term e such that no e is A, and all His e. The question
is whether an indefinite number of terms can always be inter-
polated between B and A, or there are immediate negative
propositions.
14. ;; a.1TI:Lpa. ots U1TI1.PXEL 1TPOTEpOLS. The question is whether
there is an infinite number of terms higher than B to which A
cannot belong. We must therefore either read ot!; o~X iJ7TapXH
with n, or more probably take iJ7Tapxn to be used in the sense inwhich it means 'occurs as predicate' whether in an affirmative
or a negative statement; cf. 80"2-5 n.
15-20. 'E1Tl
TWV Q.vTlCTTp,+6vTWV . . . KQ.T1lYOp(Q.v. A. now
recurs to the first two questions, and points out that the situation
with regard to these is different if we consider not terms related
in linear fashion so that one is properly predicated of the other
but not vice versa, but terms which are properly predicated of
each other. Here there is no first or last subject. Such terms
form a shuttle service, if there are but two, or a circle if there are
more, of endless predication, whether you say that each term is
subject to an infinite chain of attributes, or is that and also
attribute to an infinite chain of subjects (t:iT' a.Jl-q,6Tfpa iCJ"n Ta
S,
a.7ToPTJ8£VTa a.7THpa, 818-19).
The best examples of a.VTtKaTTJyopOVJl-t:Va are not, as Zabarella
suggests, correlative terms, or things generated in circular
fashion from each other (for neither of these are predicable of
each other), but (to take some of P.'s examples) terms related as
TO yt:AaO"TtK6v, TO VOV Kat imcrrr)Jl-TJ<;; O€KTtK6v, TO dp807Tt:pmaTTJTtK6v,
TO 7TAaTVWVVXOV, TO iv AOytKOi<;; 8v-r]T6v (all of them descriptions
of man) are to one another.
Finally, A. points out ("19-20) that what he has just said does
not apply to pairs of terms that are only in different ways assert-
ible of each other (cf. 81b25---9). the one ass~rtion (like 'the man
is white') being a genuine predication, the other (like 'that white
thing is a man') being an assertion only per accidens. For this
way of expressing the distinction cf. 83314-18.
CHAPTER 20
There cannot be an infinite chain of premisses if both extremes
are fixed
82"21. The intermediate terms cannot be infinite in number,
if predication is limited in the upward and downward directions.
For if between an attribute A and a subject Z there were an
infinite number of terms B l , B 2 , • • • , B n , there would be an
infinite number of predications from A downwards before Z is
reached, and from Z upwards before A is reached.
30. It makes no difference if it is suggested that some of the
terms A, B l , B 2 , ••• , B n , Z are contiguous and others not. For
whichever B I take, either there will be, between it and A or Z,
an infinite number of terms, or there will not. At what term the
infinite number starts, be it immediately from A or Z or not,
makes no difference; there is an infinity of terms after it.57 0
COMMENTARY
8za3O-Z. ouSE ya.p . . . ~ha"'Epn. Waitz's reading ABZ is
justified; for in "25 and 32 all the middle terms are designated B,
and there is no place for a term r. ABZ stands for ABl B2 ...
BnZ. Waitz may be right in supposing the reading ABr to have
sprung from the habit of the Latin versions of translating Z by
C (which they do in "25,27,28,29,33).
34. E'iT' Eu8us ErTE 1-11) EMus, i.e. whether we suppose the
premiss which admits of infinite mediation to have A for its
predicate or Z for its SUbject, or to have one of the B's for its
predicate and another for its subject.
CHAPTER 21
If there cannot be an infinite chain of premisses in affirmative
demonstration, there cannot in negative
8za36. If a series of affirmations is necessarily limited in both
directions, so is a series of negations.
b 4 . For a negative conclusion is proved in one of three ways.
(1) The syllogism may be No B is A, All C is B, Therefore no
C is A. The minor premiss, being affirmative, ex hypothesi
depends, in the end, on immediate premisses. If the major
premiss has as its major premiss No D is A, it must have as its
minor All B is D; and if No D is A itself depends on a negative
major premiss, it must equally depend on an affirmative minor.
Thus since the series of ascending affirmative premisses is limited,
the series of ascending negative premisses will be limited; there
will be a highest term to which A does not belong.
13. (2) The syllogism may be All A is B, No C is B, Therefore
no C is A. If No C is B is to be proved, it must be either by the
first figure (as No B is A was proved in (1)), by the second, or by
the third. If by the second, the premisses will be All B is D,
No C is D; and if No C is D is to be proved, there will have to
be something else that belongs to D and not to C. Therefore
since the ascending series of affirmative premisses is limited, so
will be the ascending series of negative premisses.
u. (3) The syllogism may be Some B is not C, All B is A,
Therefore some A is not C. Then Some B is not C will have to be
proved either (a) as the negative premiss was in (1) or in (2), or (b)
as we have now proved that some A is not C. In case (a), as we
have seen, there is a limit; in case (b) we shall have to assume
Some E is not C, All E is B; and so on. But since we have assumed
that the series has a downward limit, there must be a limit to
the number of negative premisses with C as predicate.57!
I. 20. 82"30-4
29. Further, if we use all three figures in turn, there will still
be a limit; for the routes are limited, and the product of a finite
number and a finite number is finite.
34. Thus if the affirmative series is limited, so is the negative.
That the affirmative series is so, we shall now proceed to show by
a dialectical proof.
A.'s object in this chapter is to prove that if there is a limit to
the number of premisses needed for the proof of an affirmative
proposition, there is a limit to the number of those needed for the
proof of a negative (82"36-7). He assumes, then, that if we start
from an ultimate subject, which is not an attribute of anything,
there is a limit to the chain of predicates assertible of it, and that
if we start from a first attribute, which has no further attribute,
there is a limit to the chain of subjects of which it is an attribute
("38- b 3). Now the proof of a negative may be carried out in any
of ,the three figures; A. takes as examples a proof in Celarent
(b S- I3 ), one in Camestres (b I3 - 2I ), and one in Bocardo (b2I-8).
The point he makes is that in each case, if we try to insert a
middle term between the terms of the negative premiss, we shall
need an affirmative premiss as well as a negative one, so that if
the number of possible affirmative premisses is limited, so must
be the number of negative premisses.
First figure
No B is A
All C is B
:.NoCisA
No D is A
All B is D
:.No B is A
Second figure
All A is B
No C is B
:.NoCisA
All B is D
No C is D
:.No C is B
Third figure
Some B is not C
All B is A
:.Some A is not C
Some E is not C
AllEisB
:.Some B is not C
If we try to carry the process of mediation further, it will take
the following three forms, respectively (bIO- II , 19-20, 26-7).
No E is A
All D is E
:.NoDisA
All D is E
No C is E
:.No C is D
Some F is not C
All F is E
:.Some E is not C
In the second figure the regress from the original syllogism to
J
the prosyllogism is said to be in the upward direction C
2o-1);
and this is right, because the new middle term D is wider than
the original middle term B. In the third figure the movement is
said to be in the downward direction (b 27 ); and this is right,
because the new middle term E is narrower than the original
middle term B. In the first figure the new middle term D isCOMMENTARY
57 2
wider than the original middle term B, so that here too the move-
ment is upward, and r'1.lIw, not Kd:rw, must be read in b II . But in
bl2 neither Bekker's r'1.IIW nor Waitz's Kd:rw will do; obviously not
tillW, because that stands or falls with the reading Kd:rw in b II ;
not KCl-rW, for three reasons: (I) The regress of negative premisses,
as well as of affirmative, is in the first figure upwards; for we
pass from No B is A in the original syllogism to No D is A in
the prosyllogism, and the latter proposition is the wider (B being
included in D, as stated in the minor premiss of the prosyllogism).
(2) The last words of the sentence, Ka, Ecnm n 7TPUJTOII 0/ OUK
imapXH, are clearly meant to elucidate the previous clause; but
what they mean is not that there is a lowest term of which A
is deniable (for it is assumed that C is that term), but that there
is a highest term, of which A is immediately deniable. Thus what
the sense requires in b l2 is 'the search for higher negative pre-
misses also must come to an end'. (3) A comparison of b II- I2
with the corresponding words in the case of the other two figures
(OOKOVV €-1T£1. 'TO VrraPXELv aE:L 7ep aVW'TEpW icrra'TaL, O"7'7jUE:'TaL Kat 'TO
p.~ iJ7T6.pX~w b 20- I , €7TE' S' imoKH'Ta, ,crraC18a, Kat. €7Tt. 'T6 K6.'TW,
S-ry'\OIl on O"7'7jC1E'Ta, Kat. 'T6 r OUK iJ7T6.pxolI b 27 -8) would lead us to
expect in the present sentence not a contrast between an upward
and a downward movement, but a comparison between the
search for affirmative premisses and the search for negative.
The right sense is given by n's reading, Kat. ~ €7Tt. 'T6 A O"7'7jC1E'Ta'.
These words mean 'the attempt to mediate the negative premiss
NoB is A will come to an end, no less than the attempt to mediate
the affirmative premiss All C is B' (dealt with in b6-8). The
passage from the original major premiss No B is A to the new
major premiss No D is A is a movement 'towards A'; for if in
fact no D is A and all B is D, in passing from No B is A to No
D is A we have got nearer 'to finding a subject of which not being
A is true au'TO, not merely Ka8' au'To.
At an early stage some scribe, having before him r'1.IIW in b ll ,
must have yielded to the temptation to write K6.'TW in b I2 , and
a later (though still early) scribe. seeing that this would not
work, must have reversed the two words; for P. clearly read
Ka'TW . .. allW.
8z b 6-J. TOU .. ~v ... 5LQ.CT-n1J1Q.Tos. For the use of the genitive
at the beginning of a sentence in the sense of 'with regard to .. .'
cf. Kiihner, Gr. Gramm. ii. I. 363 n. 11.
14. TOUTO, i.e. that no C is B.
18-19. £t a.V6.YKTJ .•• B. 'if in fact there is any particular term
D that necessarily belongs to B'.
n
I
~573
a.d T~ o.vwTipw, 'the belonging to higher and
higher terms', i.e. the movement from All A is B to All B is D,
and so on. "TO fL~ lJ7TclPX€LV, 'the movement from No C is B to No
C is D, and so on'.
24. TOUTO, i.e. that some B is not C.
35-6. AOyLKWS
cpo.v~p6v. A. describes his first two
arguments (that drawn from the possibility of definition, b 37 -
83b3I, and that drawn from the possibility of knowledge by
inference, 83b32--8436) as being conducted '\0YLlav, (cf. 84 37) be-
cause they are based on principles that apply to all reasoning,
not only to demonstrative science. His third argument is called
analytical (8438) because it takes account of the special nature of
demonstrative science, which is concerned solely with proposi-
tions predicating attributes of subjects to which they belong
per se (ib. II-I2).
20. TO U'lTclPXnV
.uv . . .
CHAPTER 22
There cannot be an infinite chain of premisses in affirmative
demonstration, if either extreme is fixed
82b37. (A) (First dialectical proof.) That the affirmative series
of predicates is limited is clear in the case of predicates included
in the essence of the subject; for otherwise definition would be
impossible. But let us state the matter more generally.
8331. (First preliminary observation.) You can say truly (1)
(a) 'the white thing is walking' or (b) 'that big thing is a log' or
(2) 'the log is big' or 'the man is walking'. (1 b) 'That white thing
is a log' means that that which has the attribute of being white
is a log, not that the substratum of the log is white colour; for
it is not the case that it was white or a species of white and
became a log, and therefore it is only per accidens that 'the white
thing is a log'. But (2) 'the log is white' means not that there is
something else that is white, and that that has the accidental
attribute of being a log, as in (1 a) ; the log is the subject, being
essentially a log or a kind of log.
14. If we are to legislate, we must say that (2) is predication,
and (1) either not predication, or predication per accidens; a term
like 'white' is a genuine predicate, a term like 'log' a genuine
subject. Let us lay it down that the predications we are con-
sidering are genuine predications; for it is such that the sciences
use. Whenever one thing is genuinely predicated of one thing, the
predicate will always be either included in the essence of theCOMMENTARY
574
subject, or assign a quality, quantity, relation, adion, passivity,
place, or time to the subject.
24. (Second preliminary observation.) Predicates indicating
essence express just what the subject is, or what it is a species of;
those that do not indicate substance, but are predicated of a sub-
ject which is not identical with the predicate or with a specifica-
tion of it, are accidents (e.g. man is not identical with white,
or with a species of it, but presumably with animal). Predicates
that do not indicate substance must be predicated of a distinct
subject; there is nothing white, which is white without being
anything else. For we must say good-bye to the Platonic Forms;
they are meaningless noises, and if they exist, they are nothing
to the point; science is about things such as we have described.
36. (Third preliminary observation.) Since A cannot be a
quality of Band B of A, terms cannot be strictly counter-
predicated of each other. We can make such assertions, but they
will not be genuine counter-predications. For a term counter-
predicated of its own predicate must be asserted either (1) as
essence, i.e. as genus or differentia, of its own predicate; and
such a chain is not infinite in either the downward or the upward
direction; there must be a widest genus at the top, and an
individual thing at the bottom. For we can always define the
essence of a thing, but it is impossible to traverse in thought an
infinity of terms. Thus terms cannot be predicated as genera
of each other; for so one would be saying that a thing is identical
with a species of itself.
blO. Nor (2) can a thing be predicated of its own quality, or
of one of its determinations in any category other than substance,
except per accidens; for all such things are concomitants, ter-
minating, in the downward direction, in substances. But there
cannot be an infinite series of such terms in the upward direction
either-what is predicated of anything must be either a quality,
quantity, etc., or an element in its essence; but these are limited,
and the categories are limited in number.
17. I assume, then, that one thing is predicated of one other
thing, not things of themselves, unless the predicate expresses
just what the subject is. All other predicates are attributes, some
per se, some in another way; and all of these are predicates of a
subject, but an attribute is not a subject; we do not class as an
attribute anything that without being anything else is said to be
what it is said to be (while other things are what they are by being
it); and the attributes of different subjects are themselves different.
24. Therefore there is neither an infinite series of predicates nor1.
575
an infinite series of subjects. To serve as subjects of attributes
there are only the elements in the substance of a thing, and these
are not infinite in number; and to serve as attributes of subjects
there are the elements in the substance of subjects, and the con-
comitants, both finite in number. Therefore there must be a
first subject of which something is directly predicated, then a
predicate of the predicate, and the series finishes with a term
which is neither predicate nor subject to any term wider than
itself.
3z. (B) (Second dialectical proof.) Propositions that have
others prior to them can be proved; and if things can be proved,
we can neither be better off with regard to them than if we knew
them, nor know them without proof. But if a proposition is
capable of being known as a result of premisses, and we have
neither knowledge nor anything better with respect to these, we
shall not know the proposition. Therefore if it is possible to know
anything by demonstration absolutely and not merely as true
if certain premisses are true, there must be a limit to the inter-
mediate predications; for otherwise all propositions will need
proof, and yet, since we cannot traverse an infinite series, we shall
be unable to know them by proof. Thus if it is also true that we
are not better off than if we knew them, it will not be possible
to know anything by demonstration absolutely, but only as
following from an hypothesis.
84"1. (C) (Analytical proof.) Demonstration is of per se attri-
butes of things. These are of two kinds: (a) elements in the
essence of their subjects, (b) attributes in whose essence their
subjects are involved (e.g. 'odd' is a (b) attribute of number,
plurality or divisibility an (a) attribute of it).
11. Neither of these two sets of attributes can be infinite in
number. Not the (b) attributes; for then there would be an
attribute belonging to 'odd' and including 'odd' in its own essence;
and then number would be involved in the essence of all its (b)
attributes. So if there cannot be an infinite number of elements
in the essence of anything, there must be a limit in the upward
direction. What is necessary is that all such attributes must
belong to number, and number to them, so that there will be
a set of convertible terms, not of terms gradually wider and
wider.
z5. Not the (a) attributes; for then definition would be im-
possible. Thus if all the predicates studied by demonstrative
science are per se attributes, there is a limit in the upward direc-
tion, and therefore in the lower.
22COMMENTARY
29. If so, the tenns between any two tenns must be finite in
number. Therefore there must be first starting-points of demon-
stration, and not everything can be provable. For if there are
first principles, neither can everything be proved, nor can proof
extend indefinitely; for either of these things implies that there
is never an immediate relation between terms; it is by inserting
terms, not by tacking them on, that what is proved is proved,
and therefore if proof extends indefinitely, there must be an in-
finite series of middle terms between any two tenns. But this
is impossible, if predications are limited in both directions; and
that there is a limit we have now proved analytically.
In this chapter A. sets himself to prove that the first two
questions raised in ch. 19-Can demonstration involve an infinite
regress of premisses, (I) supposing the primary attribute fixed,
(2) supposing the ultimate subject fixed?-must be answered
in the negative. The chapter is excessively difficult. The con-
nexion is often hard to seize, and in particular a disproportionate
amount of attention is devoted to proving a thesis which is at
first sight not closely connected with the main theme. A. offers
two dialectical proofs-the first, with its preliminaries, extending
from the beginning to 83b3I, the second from 83b32 to 84 a 6-and
one analytical proof extending from 8437 to 84828.
He begins (82b37--8331) by arguing that the possibility of
definition shows that the attributes predicable as included in
the definition of anything cannot be infinite in number, since
plainly we cannot in defining run through an infinite series. But
that proof is not wide enough; he has also to show that the
attributes predicable of anything, though not as parts of its
definition, must be finite in number. But as a preliminary to
this he delimits the sense in which he is going to use the verb
'predicate' (83"1-23). He distinguishes three types of assertion,
and analyses them differently: (I a) assertions like TO AWKOV
f3a8t~H or TO j1.0VULKCW £UTL AWKOV; (I b) assertions like TO Idya
£K£LVO (or TO AWKOV) £un gVAOV; (2) assertions like TO gVAOV £un
j1.£ya (or AWKOV) or 0 av8pW1TOS fJaS{~H. (1 b) When we say TO
A£VKOV £un gVAOV, we do not mean that white is a subject of
which being a log is an attribute, but that being white is an
attribute of which the log is the subject. And (I a) when we say
TO J1.0VULKOV £UTL AWKOV, we do not mean that musical is a subject
of which being white is an attribute, but that someone who has the
attribute of being musical has also that of being white. But (2)
when we say TO gVAOV £un A£VKOV, we mean that the log is a genuine1.
577
subject and whiteness a genuine attribute of it. This last type
of assertion is the only type that A. admits as genuine predication;
the others he dismisses as either not predication at all, or predica-
tion only Ka'Ta uvfL{3£{371K6" predication that is possible only as
an incidental consequence of the possibility of genuine predica-
tion. As a logical doctrine this leaves much to be desired; it must
be admitted that all these assertions are equally genuine predica-
tions, that in each we are expressing knowledge about the subject
beyond what is contained in the use of the subject-term; and in
particular it must be admitted that A. is to some extent confused
by the Greek usage-one which had very unfortunate results
for Greek metaphysics-by which a phrase like 'T6 A£VK6v, which
usually stands simply for a thing having a quality, can be used
to signify the quality; it is this that ma,kes an assertion like 'T6
A£VK6v £o-n ~vAov or 'T6 fLOVatK6v EaTt AWK6v seem to A. rather
scandalous. But A. is at least right in saying ("20--1) that his
'genuine predications' are the kind that occur in the sciences.
The only examples he gives here of genuine subjects are 'the log'
and 'the man', which are substances. The sciences make, indeed,
statements about things that are not substances, such as the
number seven Of the right-angled triangle, but they at least
think of these as being related to their attributes as a substance
is related to its attributes (cf. 87336), and not as 'T6 A£VK6v is
related to ~vAov, or 'T6 fLOVatK6v to A£VK6v. He concludes (83321-3)
that the predications we have to consider are those in which there
is predicated of something either an element in its essence or that
it has a certain quality or is of a certain quantity or in a certain
relation, or doing or suffering something, or at a certain place,
or occurs at a certain time.
He next (83824-35) distinguishes, among genuine predications,
those which 'indicate essence' (i.e. definitions, which indicate
what the subject is, and partial definitions, which indicate what
it is a particularization of, i.e. which state its genus) from those
which merely indicate a quality, relation, etc., of the subject,
and groups the latter under the term uvfL{3£{371K6'Ta. But it must
be realized that these include not only accidents but also pro-
perties, which, while not included in the essence of their subjects,
are necessary consequences of that essence. The predication of
UVfL{3£{371K6'Ta is of course to be distinguished from the predication
Ka'Ta aVfL{3£{371K6, dealt with in the previous paragraph. A. repeats
here ("30--2) what he has already pointed out, that UVfL{3£{3TJK6'Ta
depend for their existence on a subject in which they inhere-
that their esse (as we might say) is inesse-and takes occasion
4985
PP
22COMMENTARY
to denounce the Platonic doctrine of Forms as sinning against
this principle.
Now follows a passage (a36-bI2) whose connexion with the
general argument is particularly hard to seize; any interpretation
must be regarded as only conjectural. 'If B cannot be a quality
of A and A a quality of B-a quality of its own quality-two
terms cannot be predicated of each other as if each were a genuine
subject to the other (cf. "31), though if A has the quality B, we
can truly say "that thing which has the quality B is A" (as has
been pointed out in al-23). There are two possibilities to be
considered. (I) ("39-b10) Can A be predicated as an element in
the essence of its own predicate (i.e. as its genus or differentia)?
This is impossible, because (as we have seen in 82b37-838r) the
series which starts with "man" and moves upwards through the
differentia "biped" to the genus "animal" must have a limit,
since definition of essence is possible and the enumeration of an
infinity of elements in the essence is impossible; ju.st as the series
which starts with "animal" and moves downwards through
"man" must have a limit in an individual man. Thus a term
cannot be predicated as the genus of its own genus, since that
would make man a species of himself. (2) (b1 Q-q) The second
possibility to be examined is that a term should be predicated
of its own quality or of some attribute it has in another category
other than substance. Such an assertion can only be (as we have
seen in 81-23) an assertion KaTa UV/-L{3£{37fKtk All attributes in
categories other than substance are accidents and are genuinely
predicable only of substances, and thus limited in the downward
direction. And they are also limited in the upward direction,
since any predicate must be in one or other of the categories,
and both the attributes a thing can have in any category and the
number of the categories are limited.'
A.'s main purpose is to maintain the limitation of the chain
of predication at both ends, beginning with an individual sub-
stance and ending with the name of a category. But with this
is curiously intermingled a polemic against the possibility of
counter-predication. We can connect the two themes, it seems,
only by supposing that he is anxious to exclude not one but two
kinds of infinite chain; not only a chain leading ever to wider
and wider predicates, but also one which is infinite in the sense
that it returns upon itself, as a ring does (Phys. 20782). Such
a chain would be of the form 'A is B, B is C . . . Y is Z,
Z is A', and would therefore involve that A is predicable of
B as weB as B of A ; and that is what he tries in this sectionI. 22
579
to prove to be impossible, if 'predication' be limited to genuine
predication.
There follows a passage (b q - 3I ) in which A. sums up his theory
of predication. The main propositions he lays down are the
following: (I) A term and its definition are the only things that
can strictly be predicated of each other (bI8-19). (2) The ultimate
predicate in all strict predication is a substance (b2 o--- 2 ). (3)
Upwards from a substance there stretches a limited chain of
predications in which successively wider elements in its essence
are predicated (b 27 -8). (4) Of these elements in the definition of
a substance can be predicated properties which they entail, and
of these also the series is limited (b2 6-8). (5) There are thus sub-
jects (i.e. individual substances) from which stretches up a
limited chain of predication, and attributes (i.e. categories) from
which stretches down a limited chain of predication, such attri-
butes being neither predicates nor subjects to anything prior
to them, se. because there is no genus prior to them (i.e. wider
than they are) (b28-3I). Thus A. contemplates several finite
chains of predication reaching upwards from an individual subject
like Callias. There is a main chain of which the successive terms
are Callias, injima species to which Callias belongs, differentia
of that species, proximate genus, differentia of that genus, next
higher genus ... category (i.e. substance). But also each of these
elements in the essence of the individual subject entails one or
more properties and is capable of having one or more accidental
attributes, and each of these generates a similar train of differ-
entiae and genera, terminating in the category of which the
property or accident in question is a specification--quality,
quantity, relation, etc.
The second dialectical proof (b 32 -84 a 6) runs as follows: Wherever
there are propositions more fundamental than a given proposition,
that proposition admits of proof; and where J. proposition admits
of proof, there is no state of mind towards it that is better than
knowledge, and no possibility of knowing it except by proof.
But if there were an infinite series of propositions more funda-
mental than it, we could not prove it, and therefore could not
know it. The finitude of the chain is a necessary precondition
of knowledge; nothing can be known by proof, unless something
can be known without proof.
The analytical proof (B4"7-28) runs as follows: Demonstration
is concerned with propositions ascribing predicates to subjects
to which they belong per se. Such attributes fall into two c1asses-
the two which were described in 73 a 34- b3, viz. (I) attributesCOMMENTARY
580
involved in the definition of the subject (illustrated by plurality
or divisibility as belonging per se to number), (z) attributes
whose definition includes mention of the subjects to which they
belong. The latter are illustrated by 'odd' as belonging per se
to number; but since such KaO' aimL attributes are said to be
convertible with their subjects (84aZZ-5), 'odd' must be taken to
stand for 'odd or even', which we found in 73339-40. The original
premisses of demonstration (if we leave out of account dgtwp.aTa
and iJ1ToO€an,) are definitions (72aI4-Z4), which ascribe to subjects
predicates of the first kind. From these original premisses (with
the help of the dgtwp.aTa and inroO€an,) are deduced propositions
predicating of their subjects attributes KaO' am-o of the second
kind; and by using propositions of both kinds further proposi-
tions of the second kind are deduced.
KaO' alho attributes of the second kind are dealt with in 84a18-
Z5, those of the first in 3z5-8. There cannot be an infinite chain
of propositions asserting KaO' alho attributes of the second kind,
e.g. 'number is either odd or even, what is either odd or even is
either a or b, etc.'; for thus, number being included in the
definition of 'odd' and of 'even', and 'odd or even' being included
in that of 'a or b', number would be included in the definition of
'a or b', and of any subsequent term in the series, and-the defini-
tion of the term at infinity would include an infinity of preceding
terms. Since this is impossible (definition being assumed to be
always possible, and the traversing of an infinite series impossible;
cf. 8zb37--8331), no subject can have an infinite series of KaO' auro
attributes of the second kind ascending from it (84aI8-zz). It
must be noted, however (azz-5), that, since in such predications
the predicate belongs to the subject precisely in virtue of the
subject's nature, and to nothing else, in a series of such terms all
the terms after the first must b~ predicable of the first, and the
first predicable of all the others, so that it will be a series of
convertible terms, not of terms of which each is wider than the
previous one, i.e. not an ascending but what may be called a
neutral series; thus it will be infinite as the circumference of a
circle is infinite, in the sense that it returns on itself, but not an
infinite series of the kind whose existence we are denying.
Again (3Z5-8) KaO' am-o terms of the first kind are all involved
in the essence of their subject, and these for the same reason
cannot be infinite in number.
We have already seen (in ch. zo) that if the series is finite in
both directions, there cannot be an infinity of terms between any .
two terms within the series. We have now shown, therefore, that58!
there must be pairs of terms which are immediately connected,
the connexion neither needing nor admitting of proof (84aZg--bz).
83"7. 01TEP AEUKOV n, 'identical with a species of white'.
13. 01TEP Ka.t EYEvETO, 'which is what we made it in our assettion'.
24-5. "En Ta. I'EV ••• Ka.T"yOpEi:Ta.~, 'further, predicates that
indicate just what their subject is, or just what it is a species of'.
8TT£P £K£L))O n is to be explained differently from 8TT£P A£UKO)) 'Tt
in "7 and the other phrases of the form OTT£P •.• n which occur in
the chapter. It plainly means not 'just what a species of that
subject is', but 'just what that subject is a species of', n going
not with £K£L))O but with 8TT£p.
30. 01TEP ya.p t'i'ov Eanv (, av9plll1ToS. More strictly OTT£P ~cfio))
n, 'identical with a species of animal'. But A.'s object here is not
to distinguish genus from species, but both from non-essential
attributes.
32-5. Ta. ya.p ErS" ••• EtaLv. 'T£pe,LufLa-ra is applied literally to
buzzing, twanging, chirruping, twittering; metaphorically to
speech without sense. This is the harshest thing A. ever says
about the Platonic Forms, and must represent a mood of violent
reaction against his earlier belief. The remark just made (832),
that there is nothing white without there being a subject in
which whiteness inheres. leads him to express his disapproval
of the Platonic doctrine, which in his view assigned such a
separate existence to abstractions. Even if there were Platonic
Forms. he says. the sciences (whose method is the subject of the
Posterior A nalyties) are concerned only with forms incorporated
in individuals.
I conjecture that after these words we should insert £i8") fLE))
00')) • • • OfLWVVfLO)) (77 8 5-9). which is out of place in its present
position. It seems impossible to say what accident in the history
of the text has led to the misplacement.
3~. "En Et ••• ollTlIIs. TTOW'T")S' is here used to signify an
attribute in any category. TTOLOrr]'T£S' are then subdivided into
essential attributes (a3g--bro) and non-essential attributes (bro- q ).
as in Met. rozobr3-r8).
39-bI . .;; yelp ••• Ka.TTJYOPOUI'EVOU. These words are answered
irregularly by ou8£ fL~)) 'TOU TTOtOU ~ 'TW)) aAAw)) 01~8£)). bIO.
bU-17. a.AAa. ST] ••• 1TO'TE, 'but now to prove that ... ; the
proof is contained in the fact that. .. .' For this elliptical use
of on cf. An. Pr. 6283z, 40,br4. n may be right in reading aA).a
8iJAO)) on (the reading 8~ being due to abbreviation of OijAO))), but
the leetio diffieilior is preferable.
17. ·Y1TOKELTa.L ••. EVOS Ka.T"YOPELU9a.L is to be interpreted in5 82
COMMENTARY
the light of the remainder of the sentence, 'we assume that one
thing is predicated of one other thing'. The only exception is
that in a definitory statement a thing is predicated of itself (Clua
J-LTJ ·d £a'Tt,
b I
8).
These words seem to make a fresh start, and I have accordingly
written o~ as the more appropriate particle.
19-2.4. au .... ~E~TJK6Ta. ya.p . . . ETEpou. In all non-definitory
statements we are predicating concomitants of the subject-
either per se concomitants, i.e. properties (attributes Ka(J' atl'r6 of
the second of the two kinds defined in 73 334-b3) or accidental
concomitants. Both alike presuppose a subject characterized by
them. Not only does 'straight' (a typical Ka(J' aVT6 attribute) pre-
suppose a line, but 'white' (a typical accidental concomitant)
presupposes a body or a surface (8331-23). 'For we do not class
as a concomitant anything that is said to be what it is said to be,
without being anything else' (b22- 3).
84 8 7-8. AOYLKWS .... EV o~v ••• a.Va.AUTLKWS SE, cf. 8 2 b 3S -6 n.
II-I2.. TJ .... Ev ya.p ... 1Tpa.y .... a.aw. The use of the article (TWV)
as a demonstrative pronoun, with a relative attached, is a relic
of the Homeric usage, found also in 8Sb36 and not uncommon in
Plato (cf. esp. Prot. 320d3, Rep. 469 b3, SIoa2, Parm. I30CI,
Theaet. 20 4 dI ).
13. aaa. TE ya.p . . . EaTL. ]aeger (Emend. Arist. Specimen,
49-S2) points out that while the implication of one tenn in the
definition of another is expressed by £VV7TCLPXH, or v7TapXH, £V Tip
Tt £un (73334,36, 74 b7, 843IS), the inherence of an attribute in a
subject is expressed by V7rapXH, or £VV7Td.PXH, nvt (without €v),
and that when A. wants to say' A inheres in B as being implied
in its definition', he says nvt £V Tip Tt £unv £vV7Tapx£t, or v7TaPXH
(73 337, b I , 74b8). He therefore rightly excises £v.
16-17. Ka.L1TCI.ALV ••• EVU1TCI.PXEL. Mure reads aOtatp£Tov, on the
ground that number is 7TAij(JO, aotatpt-rwv (M et. I08S b22). But
otatp£T6v is coextensive with 7Tou6v in general (M et. 102087).
Quantity or the divisible has for its species J-Lly£(Jo, or TO UVV€Xl"
and 7TAij(Jo, or TO OtwptuJ-Llvov (phys. 2048II), i.e. what is infinitely
divisible (De Caelo 268"6) and what is divisible into indivisibles,
i.e. units (Met. I02087-II). Thus Otatp£T6v is in place here, as an
element in the nature of number.
18-19. 1Ta.Aw ya.p ... ELTJ. Bonitz (Arist. Stud. iv. 21-2) points
out that, as in 73 8 37 the sense requires v7Tapx6vTwV, not £vv7Tapx6v-
TWV (cf. n. ad loc.), so here we do not want £V before Tip 7T£ptTTip.
The lectio recepta av €v is due to a conflation of the correct av
with the corrupt £v.%I. U1I'a.PXELY EY T4I EyL. fl, should perhaps be omitted, as in
"I3 and I9. But on the whole it seems permissible here. tnrd.PXf.LV
£V Tip Evi stands for iJ7Td.PXHV Tip Ev~ £V Tip Tt £a-nv.
:U-5.
!lofty ••• U1I'EPTELYOYTa. Having rejected in "I8-22
the possibility of an infinite series of terms, each KaO' aUTO in the
0.,,"«
second sense to its predecessor, A. now states the real position-
that, instead, there is a number of terms, each KaO' aUTO in this
sense to a certain primary subject (in the case in question, to
number) ; but these will be convertible with one another and with
the subject, not a series in which each term is wider than its
predecessor.
%9. Et 8' O(lTW • • • 1I'E1I'Epau!l€va. This has been proved in
ch. 20.
3%. 01l'EP i~a!lEY ••• o.PXa.s, in 7 2b6 -7·
36. ~!lI3a.""EaeaL. Ct. TTapf.p.TTiTTTHV in An. Pr. 42 b 8 (where
see n.).
CHAPTER 23
Corollaries from the foregoing propositions
b
84 3. It follows (I) that if the same attribute belongs to two
things neither of which is predicable of the other, it will not
always belong to them in virtue of something common to both
(though sometimes it does, e.g. the isosceles triangle and the
scalene triangle have their angles equal to two right angles in
virtue of something common to them).
9. For let B be the common term in virtue of which A belongs
to C and D. Then (on the principle under criticism) B must
belong to C and D in virtue of something common, and so on,
so that there would be an infinite series of middle terms between
two terms.
14. But the middle terms must fall within the same genus,
and the premisses be derived from the same immediate premisses,
if the common attribute to be found is to be a per se attribute;
for, as we saw, what is proved of one genus cannot be transferred
to another.
19. (2) When A belongs to B, then if there is a middle term, it
is possible to prove that A belongs to B; and the elements of the
proof are the same as, or at least of the same number as, the
middle terms; for the immediate premisses are elements-either
all or those that are universal. If there is no middle term, there
is no proof; this is 'the way to the first principles'.
%4. Similarly if A does not belong to B, then if there is a middle
term, or rather a prior term to which A does not belong, there isCOMMENTARY
a proof; if not, there is not-No B is A is a first principle; and
there are as many elements of proof as there are middle tenns;
for the propositions putting forward the middle tenns are the
first principles of demonstration. As there are affinnative in-
demonstrable principles, so there are negative.
31. (a) To prove an affinnative we must take a middle term
that is affinned directly of the minor, while the major is affirmed
directly of the middle tenn. So we go on, never taking a premiss
with a predicate wider than A but always packing the interval
till we reach indivisible, unitary propositions. As in each set of
things the starting-point is simple-in weight the mina, in melody
the quarter-tone, etc.-So in syllogism the starting-point is the
immediate premiss, and in demonstrative science intuitive
knowledge.
8S"I. (b) In negative syllogisms, (i) in one mood, we use no
middle term that includes the major. E.g., we prove that B is
not A from No C is A, All B is C; and if we have to prove that
no C is A, we take a tenn between A and C, and so on. (ii) In
another mood, we prove that E is not D from All D is C, No E is
C; then we use no middle tenn included in the minor tenn. (iii)
In the third available mood, we use no middle tenn that either
is included in the minor or includes the major.
n
84b8. ya.p ax" . . O' n, i.e. qua triangle .
. n. i .... 1T(1TTOLEV. Cf. 1TUP£fL1T[7TT£LV in An. Pr. 42 b 8 (where see n.).
ciAA' ciMva.Tov, as proved in chs. 19-22.
14. EL1TEP lUTa.L a. . . EUa. liLa.UTiJ .... a.Ta.. Jaeger (Emend. Arist.
Specimen, 53-7) points out that the MS. reading (with £1T£L1T£p)
could only mean 'since it would follow that there are immediate
intervals'. I.e. the argument would be a reductio ad absurdum.
But it is not absurd, but the case, that there are immediate
intervals (bII-13). He cures the passage by reading £L1T£P, which
gives the sense 'if there are to be' (as there must be) 'immediate
intervals'. For the construction cf. b1 6, 80"30-2, 81"18-19; £L •••
£aTaL in 77"6; £L fL£>..>..£L £aw(JuL in 80 b35 ; £r7T£p 8£L • •• £lvuL in 72b3, 26.
14-17. EV .... iVTOL T<\I a.UT<\I YEvEL ••• 5ELICVU .... EVa.. The point of
this addition is to state that while the middle tenns used to prove
the possession of the same KU(J' um-o attribute by different subjects
need not be identical, all the middle tenns so used must fall
within the same genus (e.g. be arithmetical, or geometrical),
and all the premisses must be derived from the same set of ultimate
premisses, since, as we saw in ch. 7, propositions appropriate to one
genus cannot be used to prove conclusions about another genus.585
~.
KW aTOlX€la. ••• Ka.9bAou. The sentence is improved by
reading TauTlf in b ZI , but remains difficult; to bring out A.'s
meaning, his language must be expanded. 'And there are ele-
ments of the proof the same as, or more strictly as many as, the
middle terms; for the immediate premisses are elements--either
all of them (and these are of course one more numerous than the
middle terms) or those that are major premisses (and these are
exactly as many as the middle terms).' The suggestion is that in
a chain of premisses such as All B is A, All C is B, All D is ConI y
the first two are elements of the proof, since in a syllogism the
major premiss already contains implicitly the conclusion (cf.
86"2Z-«), and 86h30 £l apx~ O1JAAOYLUP.OV ~ Ka80AOV 1TP0TaULS ctp.mos).
For Kat (bzI ) = 'or more strictly' cf. Denniston, Greek Particles,
z9 2 (7).
23-4' a.A)" ti l1Tt Ta.S a.PXa.s ••• laTLV. Cf. E.N. I09S"32 £0 yap
Ka~ 0 IDa.TWV ~1T()P£L TOVrO Ka~ £~~T£L, 1TOTt:pOV a1T() TWV apxwv ~
£1T~ TaS apxa.s £aTLV ~ ooos. As the imperfect tenses imply, the
reference is to Plato's oral teaching rather than to Rep. SIO b--
511 c.
25. €t .... iv ... U1TclPXU. ~ 1Tponpov cp OUX lJ1TapXH is a correction.
p./uov suggests something that links two extremes, and something
intermediate in extent between them; and in a syllogism in
Barbara the middle term must at least be not wider than the
major and not narrower than the minor. But in a negative
syllogism the middle term serves not to link but to separate the
extremes, and in a syllogism in Celarent nothing is implied about
the comparative width of the major and middle terms; they are
merely known to exclude each other. But the middle term at
least more directly excludes the major than the minor does.
31-85a3. "OTa.V S~ . . . 1TL1TT€l. A. here considers affirmative
syllogisms, and takes account only of proof in the first figure,
ignoring the second, which cannot prove an affirmative, and the
third, which cannot prove a universal. If we want to prove that
all B is A, we can only do so by premisses of the form All C is
A, All B is C. If we want to prove either of these premisses, we
can only do so by a syllogism of similar form. Clearly, then, we
never take a middle wider than and inclusive of A, nor (though
A. does not mention this) one narrower than and included in B;
all the middle terms will fall within the 'interval' that extends
from B to A, and will break this up into shorter, and ultimately
into unitary, intervals.
31-3. "OTa.V S~ ... ll.. The editions have op.otws TO A. But if
we start with the proposition All B is A, there is no guarantee thatCOMMENTARY
5 86
we can find a term 'directly predicable of B, and having A directly
predicable of it'; and in the next sentence A. contemplates a
further packing of the interval between B and A. n must be
right in reading OjLotw,> TO Lt; the further packing will then be
of the interval between Lt and A.
35-6. ~O'TL 5' ••• lif.L~O'os. A comma is required after Y£V7JTaL,
and none after lv.
85"1. EV 5' Q1Ta5~(~EL KQt E1TLO'T'lf.Ln b vaus, 'in demonstrative
science the unit is the intuitive grasp of an unmediable truth'.
3-12. EV 5E TOLS O'T~P'lTLKOLS • • • I3Q5L~LTQL. The interpreta-
tion of this passage depends on the meaning of ;gw in 84, 9, II.
Prima facie, ;gw might mean (a) including or (b) excluded by.
But neither of these meanings will fit A.'s general purpose, which
is to show that a proposition is justified not by taking in terms
outside the 'interval' that separates the subject and predicate,
but by breaking the interval up into minimal parts (84b33-S).
The only meaning of ;gw that fits in with this is that in which
a middle term would be said to be outside the major term if it
included it, and outside the minor term if it were included in it.
Further, this is the only meaning that fits the detail of the
passage. Finally, it is the sense that ;gw bears in 88 8 3S ~ TOV,>
jLEV daw ;XHV TOV,> 0' ;gw TWV opwv.
A. considers first (°1-7) the justification of a negative proposi-
tion by successive syllogisms in the first figure (i.e. Celarent).
'No B is A' will be justified by premisses of the form No C is A,
All B is C. Here the middle term plainly does not include the
major. Further, if All B is C needs proof, the middle term to be
used will not include the major term C (shown in 84b31-S and now
silently assumed by A.). And if No C is A is to be proved, it
will be by premisses of the form NoD is A, All C is D, where
again the middle term does not include the major. Thus in a
proof by Celarent no middle term used includes the major term
(8S"3-S). We may add, though A. does not, that no middle term
used is included in the minor.
A. next (a7-10) considers a proof in Camestres. If we prove
No E is D from All D is C, No E is C, we see at once that here it
is not true that no middle term used includes the major; for here
the very first middle does so. But it is true that no middle term
used is included in the minor. The first middle term plainly is
not. And if we have to prove the minor premiss by Camestres,
it will be by premisses of the form All C is F, No E is F, where F
is not included in E.
.
The last case (alo-12) is usually taken to be that of a proof inthe third figure. But a reference to the third figure would be
irrelevant; for A. is considering only the proof of a universal
proposition, and that is why he ignored the third figure when
dealing with proofs of an affirmative proposition (84b31-S).
Further, what he says, that the middle term never falls outside
either the minor or the major term, i.e. never is included in the
minor or includes the major, would not be true of a proof in the
third figure. For consider the proof of a negative in that figure,
say in Fesapo-No M is P, All M is 5, Therefore some 5 is not P;
the very first middle term used is included in the minor.
~1Ti TOU Tp(TOU TP01TOU refers not to the third figure, but to the
third (and only remaining) way of proving a universal negative,
viz. by Cesare in the second figure. (Cf. A n. Pr. 42b32 TO P.€II 0011
KaTa.panKOII TO Ka86>.ov Sw. TOV TTPWTOV ax~p.aTo~ S€{KIIVTa, p.6110V,
Kat S,a Tothov p.Ollaxw,· TO S€ CTT€pr]nKOII S,ci U TOV TTPWTOV Kat S,a
TOV p.iaov, Kat Sw. p.€Y TOV TTPWTOV p.ollaxw~, Sw. S€ TOV p.€aov S,xw~.
Further, the three modes of proving an E proposition have been
mentioned quite recently in An. Post. 79bl6-20.) The form of
Cesare is No D is C, All E is C, Therefore No E is D. The middle
term neither includes the major nor is included in the minor.
Further, if we prove the premiss No D is C by Cesare, it will be
by premisses of the form No C is F, All D is F, and if we
prove the premiss All E is C, it will be by premisses of the form
All C is C, All E is C; and neither of the middle terms, F, C, in-
cludes the corresponding major or is included in the correspond-
ing minor.
Thus the general principle, that in the proof of a universal
proposition we never use a middle term including the major
or included in the minor, holds good with the exception (tacitly
admitted in 89-10) that in a proof in Camestres the middle term
includes the major.
One point remains in doubt. The fact that A. ignores the third
figure when dealing with affirmative syllogisms (84b31-S) and the
fact that he ignores Ferio when dealing with negative syllogisms
in the first figure (8S 3 S-7) imply that he is considering only
universal conclusions and therefore only universal premisses.
But in 8S"9 (~ p.~ TTaJIT{) the textus receptus refers to a syllogism
in Baroco. It is true enough that in a proof or series of proofs
in Baroco the middle term is not included in the minor; but
either the remark is introduced per incuriam or more probably
it is a gloss, introduced by a scribe who thought that "10--12 re-
ferred to the third figure, and therefore that A. was not confining
himself to syllogisms proving universal conclusions.COMMENTARY
588
3. iv9a. jL€v 8 5Ei U1Ta.PXELV. 8 8Er lJ7Tapx£w can stand for the
predicate of the conclusion even when the conclusion is negative
(cf. 80"2-5 n.).
s. Et ya.p ... A, 'for this proof is effected by assuming that
all B is C and no C is A'.
10. ~ 5Ei U1Ta.PXEW. This reading is preferable to the easier
0/ ov 8£, lJ7Tapx£w. Cf. a 3 n.
CHAPTER 24
Universal demonstration is superior to particular
85"13. It may be inquired (I) whether universal or particular
proof is the better, (2) whet.her affirmative or negative, (3)
whether ostensive proof or reductio ad impossibile.
~O. Particular proof might be thought the better, (I) because
the better proof is that which gives more knowledge, and we know
a thing better when we know it directly than when we know it
in virtue of something else; e.g. we know Coriscus the musician
better when we know that Coriscus is musical than when we
know that man is musical; but universal proof proves that some-
thing else, not the thing itself, has a particular attribute (e.g.
that the isosceles triangle has a certain attribute not because it
is isosceles but because it is a triangle), while particular proof
proves that the particular thing has it:
3I. (2) because the universal is not something apart from its
particulars, and universal proof creates the impression that it
is, e.g. that there is a triangle apart from the various kinds of
triangle; now proof about a reality is better than proof about
something unreal, and proof by which we are not led into error
better than that by which we are.
b4. In answer to (I) we say that the argument applies no more
to the universal than to the particular. If possession of angles
equal to two right angles belongs not to the isosceles as such
but to the triangle as such, one who knows that the isosceles has
the attribute has not knowledge of it as belonging essentially
to its subject, so truly as one who knows that the triangle has
the attribute. If 'triangle' is wider and has a single meaning,
and the attribute belongs to every triangle, it is not the triangle
qua isosceles but the isosceles qua triangle that has the attribute.
Thus he who knows universally, more truly knows the attribute
as essentially belonging to its subject.
IS. In answer to (2) we say (a) that if the universal term is
univocal, it will exist not less, but more, than some of its part i-589
culars, inasmuch as things imperishable are to be found among
universals, while particulars tend to perish; and (b) that the fact
that a universal term has a single meaning does not imply that
there is a universal that exists apart from particulars. any more
than do qualities, relations, or activities; it is not the demonstra-
tion but the hearer that is the source of error.
23. (Positive arguments.) (r) A demonstration is a syllogism
that shows the cause, and the universal is more causal than the
particular (for if A belongs to B qua B, B is its own reason for
its having the attribute A; now it is the universal subject that
directly owns the attribute, and therefore is its cause) ; and there-
fore the universal demonstration is the better.
27. (2) Explanation and knowledge reach their term when we
see precisely why a thing happens or exists, e.g. when we know
the ultimate purpose of an act. If this is true of final causes, it
is true of all causes, e.g. of the cause of a figure's having a certain
attribute. Now we have this sort of knowledge when we reach
the universal explanation; therefote universal proof is the better.
8683. (3) The more demonstration is particular, the more it
sinks into an indeterminate manifold, while universal demonstra-
tion tends to the simple and determinate. Now objects are
intelligible just in so far as they are determinate, and therefore
in so far as they are more universal; and if universals are more
demonstrable, demonstration of them is more truly demonstration.
10. (4) Demonstration by which we know two things is better
than that by which we know only one; but he who has a universal
demonstration knows also the particular fact, but not vice versa.
13. (5) To prove more universally is to prove a fact by a middle
term nearer to the first principle. Now the immediate proposition,
which is the first principle, is nearest of all. Therefore the more
universal proof is the more precise, and therefore the better.
zz. Some of these arguments are dialectical; the best proof
that universal demonstration is superior is that if we have a more
general premiss we have potentially a less general one (we know
the conclusion potentially even if we do not know the minor
premiss); while the converse is not the case. Finally, universal
demonstration is intelligible, while particular demonstration
verges on sense-perception.
1. 23. 85"3-24. 858I6
85"13-16. Ou'"1S s· ... ci.1I"OSEL~EWS. The three questions are
discussed in chs. 24, 25, 26. In the first question the contrast is
not between demonstrations using universal propositions and
those using particular or singular propositions; for demonstration59 0
COMMENTARY
always uses universal propositions (the knowledge that Coriscus
is musical (a25) is not an instance of demonstration, but an
example drawn from the sphere of sensuous knowledge, in a
purely dialectical argument in support of the thesis which A.
rejects, that particular knowledge is better than universal). The
contrast is that between demonstrations using universal pro-
positions of greater and less generality.
34. TOY flOUULKOY KOp(UKOY. Coriscus occurs as an example
also in the Sophistici Elenchi, the Physics, the Parva Natllralia,
the De Partibus, the De Generatione Animalium, the Metaphysics,
and the Eudemian Ethics. Coriscus of Scepsis was a member of
a school of Platonists with whom A. probably had associations
while at the court of Hermeias at Assos, c. 347-344. He is one of
those to whom the (probably genuine) Sixth Letter of Plato is
addressed. From Phys. 2I9b20 wa7Tff-P ol aocfmrraL AafLf3dvovaw
£upov 'TO Kop{aKov £V AVKdCfJ ElvaL KaL 'TO KOPWKOV £V ayopij. we may
conjecture that he became a member of the Peripatetic school,
and he was the father of Neleus, to whom Theophrastus left
A.'s library. The reference to him as 'musical Coriscus' recurs in
Met. IOI5bI8, I026bI7. On A.'s connexion with him cf. Jaeger.
Entst. d. Met. 34 and A'rist. II2-I7. 268.
37-8. otOY on ... Tp(yWYOY, 'e.g. it proves that the isosceles
triangle has a certain attribute not because it is isosceles but
because it is a triangle'.
37-bI. 1TPO'ioYTE5 ya.p ••• n. A. illustrates the point he is here
putting dialectically. by reference to a development of mathe-
matics which he elsewhere (74aI7-25) describes as a recent dis-
covery. viz. the discovery that the properties of proportionals
need not be proved separately for numbers. lines, planes, and
solids, but can be proved of them all qua sharing in a common
nature, that of being quanta. The Pythagoreans had worked out
the theory of proportion for commensurate magnitudes; it was
Eudoxus that discovered the general theory now embodied in
Euc. El. v. vi. In the present passage the supposed objector
makes a disparaging reference to the general proof-'if they carry
on in this course they come to proofs such as that which shows
that whatever has a certain common character will be propor-
tional, this character not being that of being a number, line.
plane. or solid, but something apart from these'.
bS. CiTEp05 XOY05, 'the other argument', i.e. that in -21-31.
11. TO Suo, i.e. 'TO 'TaS' ywv{aS' Ova op(JaLS' LaaS' £XELv.
33. "En Et KT).., 'the same conclusion follows from the fact
that'. etc. ; cf. 86-ro n .• b30- I n.as. TO 8E Ka.eO~OU 'lrpWTOV. 'and the universal is primary', i.e.
if the proposition All B is A is commensurately universal, the
presence of B'ness is the direct cause of the presence of A 'ness.
36. E'lrL 8E TWV aaa. a.LTLa.. For the construction cf. 84all-12 n.
38-86a1. OTa.V IlEv o~v •.• Eu9uYPa.llllov. This is interesting
as being one of the propositions known to A. but not to be found
in Euclid a generation later; for other examples cf. De Caelo
287a27-8, Meteor. 376"1-3, 7"""'), bI - 3 , 10-12, and Heiberg, Math.
zu Arist. in Arch. z. Gesch. d. Math. Wissensch. xviii (1904), 26--7.
Cf. Heath, Mathematics in Aristotle, 62-4.
8619. Cilla. ya.p IlCi~~ov Ta. 'lrpOS TL, 'for cor relatives increase
concomitantly' .
10. ~ETL Et a.tPETWTEpa. KT~ •• cf. 85b23 n.
22.-9 . • A~~a. TWV IlEV Etp,,~VWV ••• EvEPYEl~. This is not a new
argument; it is the argument of a10-13 expanded, with explicit
introduction of the distinction of oUllall-l!> and £v£fYYHa. Thus A.
is in fact saying that while some of the previous arguments are
dialectical, one of them is genuinely scientific.
Zabarella tries to distinguish this argument from that of "10-13
by saying that whereas the present argument rests on the fact
that knowledge that all B is A involves potential knowledge that
particular B's are A, the earlier argument rests on the fact
that knowledge that all B is A presupposes actual knowledge that
some particular B's are A. But this is not a natural reading of
"10- 1 3.
A. does not mean by T~II 7TpoT£pall, TIjIl va-r£pall the major and
minor premiss of a syllogism; for (a) what he is comparing in
general throughout the chapter is not two premisses but two
demonstrations, or the conclusions of two demonstrations; (b)
it is not true that knowledge of a major premiss implies potential
knowledge of the minor, though it is true to say that in a sense
it implies potential knowledge of the conclusion; (c) in the example
(a25""",)) it is with knowledge of the conclusion that A. contrasts
knowledge of the major premiss. ~ 7TpoT£pa is the premiss of a
more general demonstration, ~ va'dpa the premiss of a less general
demonstration. A. is comparing the first premiss in a proof of the
form All B is A, All C is B, All D is C, Therefore All D is A, with
the first premiss in a proof of the form All C is A, All D is C, There-
fore all D is A.
It follows that TaV-r7]1I TIjIl 7TpoTaalll in a27--8 means TO laoaK£i\~!>
~Tl ovo op(Ja'i!>, not TO laoaKEA€!> ~n TptYWIIOII.
3~30' " 8E Ka.Ta. IlEPOS ... TE~EUT~. If we imagine a series of
demonstrations of gradually lessening generality, the last memberCOMMENTARY
59 2
of such a series would be a syllogism with an individual thing
as its minor term, and in that case the conclusion of the syllogism
is a fact which might possibly be apprehended by sense-per-
ception, as well as reached by inference.
CHAPTER 25
Affirmative demonstration is superior to negative
86"31. That an affirmative proof is better than a negative
is clear from the following considerations. (I) Let it be granted
that ceteris paribus, i.e. if the premisses are equally well known,
a proof from fewer premisses is better than one from more pre-
misses, because it produces knowledge more rapidly.
36. This assumption may be proved generally as follows: Let
there be one proof that E is A by means of the middle terms
B, r, .1, and another by means of the middle terms Z, H. Then
the knowledge that .1 is A is on the same level as the knowledge,
by the second proof, that Eis A. But that .1 is A is known better
than it is known, by the first proof, that E is A; for the latter
is known by means of the former.
b7. Now an affirmative and a negative proof both use three
terms and two premisses ~ but the former assumes only that
something is, and the latter both that something is and that
something is not, and therefore uses more premisses, and is
therefore inferior.
10. (2) We have shown that two negative premisses cannot
yield a conclusion; that to get a negative conclusion we must
have a negative and an affirmative premiss. We now point out
that if we expand a proof, we must take in several affirmative
premisses but only one negative. Let no B be A, and all r be B.
To prove that no B is A, we take the premisses No .1 is A, All
B is .1 ; to prove that all r is B, we take the premisses All E is B,
All r is E. So we take in only one negative premiss.
22. The same thing. is true of the other syllogisms; an affirma-
tive premiss needs two previous affirmative premisses; a negative
premiss needs an affirmative and a negative previous premiss.
Thus if a negative premiss needs a previous affirmative premiss,
and not vice versa, an affirmative proof is better than a negative.
30. (3) The starting-point of a syllogism is the universal im-
mediate premiss, and this is in an affirmative proof affirmative,
in a negative proof negative; and an affirmative premiss is prior
to and more intelligible than a negative (for negation is known
on the ground of affirmation, and affirmation is prior, as being593
is to not-being). Therefore the starting-point of an affirmative
proof is better than that of a negative; and the proof that has
the better starting-point is the better. Further, the affirmative
proof is more primary, because a negative proof cannot proceed
without an affirmative one.
86a33-b9' EO'TW ya.p ••• XE[pWV. This argument is purely dia-
lectical, as we see from two facts. (I) What A. proves in a33-b 7 is
that an argument which uses fewer premisses is superior to one
that uses more, if the premisses are equally well known. But
what he points out in b 7 --9 is that a negative proof uses more
kinds of premiss than an affirmative, since it needs both an
affmnative and a negative premiss. (2) The whole conception
that there could be two demonstrations of the same fact using
different numbers of equally well-known premisses (i.e. immediate
premisses, or premisses approaching equally near to immediacy)
is inconsistent with his view of demonstration, namely that of
a single fact there is only one demonstration, viz. that which
deduces it from the unmediable facts which are in reality the
grounds of the fact's being a fact.
34. a.tTTII-I.a.TWV 1\ Un09€O'EW\,. iJ7To8wts and ah7)J.La are defined
in contradistinction to each other in 76b27-34; there is no allusion
here to the special sense given to lmo8wts in 72aI8-20.
bI o- I2 • EnELSi} SEliELKTa.L ••• una.pXEL. The proof is contained
in the treatment of the three figures in An. Pr. i. 4-6, and summed
up ib. 24. 4Ib6-7.
IS. EV ana.VTl. O'U)..)..OYLO'jL~, not only in each syllogism but in
each sorites, as A. goes on to show.
22-3. 0 S' a.UTO'S Tpono'S ••• O'U)..)..0YLO'jLWV. This may refer
either (a) to further expansions of an argument by the inter-
polation of further middle terms, or (b) to arguments in the second
or third figure. But in b 30- 3 A. contemplates only first-figure
syllogisms; for in the second figure a negative conclusion does
not require a negative major premiss; so that (a) is probably the
true interpretation here.
30-1. ET!. Et ••• a.jLEO'O'S. El is to be explained as in 8S b 23,
86 K IO. The major premiss is called the starting-point of the
syllogism because knowledge of it implies potential knowledge
of the conclusion (322--9).
38. a.vEli ya.p Tfj'S OELlCVUOUaTI'i •.• O'TEpTJTLKTJ, because, as we
have seen in bID-30, a negative proof requires an affirmative
premiss, which (if it requires proof) requires proof from affirma-
tive premisses.
Qq594
COMMENTARY
CHAPTER 26
Ostensive demonstration is superior to reductio ad impossibile
87"1. Since affirmative proof is better than negative, it is
better than reductio ad impossibile. The difference between
negative proof and reductio is this: Let no B be A, and all C be B.
Then no C is A. That is a negative ostensive proof. But if we
want to prove that B is not A, we assume that it is, and that C is
B, which entails that C is A. Let this be known to be impossible;
then if C is admittedly B, B cannot be A.
12. The terms are similarly arranged; the difference depends
on whether it is better known that B is not A or that C is not A.
When the falsity of the conclusion ('C is A ') is the better known,
that is reductio; when the major premiss (' B is not A ') is the
better known, ostensive proof. Now 'B is not A' is prior by
nature to 'C is not A'. For premisses are prior to the conclusion
from them, and 'C is not A' is a conclusion, 'B is not A' a premiss.
For if we get the result that a certain proposition is disproved,
that does not imply that the negation of it is a conclusion and
the propositions from which this followed premisses; the pre-
misses of a syllogism are propositions related as whole and part,
but 'C is not A' and 'C is B' are not so related.
25. If, then, inference from what is prior is better, and the
conclusions of both kinds of argument are reached from a nega-
tive proposition, but one from a prior proposition, one from
a later one, negative demonstration is better than reductio, and
a fortiori affirmative demonstration is so.
87"10. OUK a.pa. ... U1ra.PXELV. Maier (SyU. d. Arist. 2 a. Z3I n.)
conjectures r for the MS. reading B, on the ground that otherwise
this sentence would anticipate the result reached in the next
sentence. But with his emendation the present sentence becomes
a mere repetition of the previous one, so that nothing is gained.
The next sentence simply sums up the three that precede it.
12-25. OL I-LEV o3v OpOL ••• 0.>'>'1]>'0.<;. The two arguments, as
stated in "3-12, are (1) (Ostensive) No B is A, All C is B, There-
fore no C is A. (2) (Reductio) (a) If B is A, then-since C is B-C
is A. But (b) in fact C is not A ; the conclusion of a syllogism
cannot be false and both its premisses true; 'C is B' is true; there-
fore' B is A' is false. A. deliberately (it would seem) chooses a
reductio the effect of which is to establish not the conclusion to
which the ostensive syllogism led, but the major premiss of that1. 26. 8]"10-28
595
syllogism. At the same time, to avoid complications about the
quantity of the propositions, he introduces them in an unquanti-
fied form. The situation he contemplates is this: (1) There may
be a pair of known propositions of the form 'B is not A', 'C is
B', which enable us to infer that C is not A. But (2), on the other
hand, we may, while knowing that C is B and that C is not A, not
know that B is not A, and be able to establish this only by con-
sidering what follows from supposing it to be false; then we use
reductio. The arrangement of the terms is as before (112); i.e.
in fact in both cases B is not A, C is B, and C is not A; the
difference is that we use' B is not A' to prove 'C is not A' (as in
(I)) when' B is not A' is to us the better known proposition, and
'C is not A' to prove 'B is not A' (as in (2 b)) when 'C is not A'
is the better known. But the two processes are not equally
natural (°17-18); 'B is not A' is in itself the prior proposition,
since it, with the other premiss 'C is B', constitutes a pair of
premisses related to one another as whole to part (a22-3, cf.
An. Post. 42a8-13, 47aI0-14, 49b37-5oal), the one stating a general
rule, the other bringing a particular case under it; while 'C is
not A', with 'C is B', does not constitute such a pair (and in fact
does not prove that B is not A, but only that some B is not A).
The second part of the red~tctio process is, as A. points out in
An. Pr. 41"23-30, 50"29-38, not a syllogism at all, but an argument
if lmO (J£u£w" involving besides the data that are explicitly
mentioned CC is not A' and 'C is B') the axiom that premisses
(e.g. 'B is A' and 'C is B') from which an impossible conclusion
(e.g. 'C is A') follows cannot both be true.
It seems impossible to make anything of the MS. reading Ar
KaL ~ in "24. For what A. says is 'the only thing that can be
a premiss of a syllogism is a proposition which is to another'
(Le. to the other premiss) 'either as whole to part or as part to
whole', and it would be pointless to continue 'but the propositions
Ar (HC is not A ") and AB (" B is not A ") are not so related' ; for
in the reductio there is no attempt to treat these propositions as
joint premisses; 'C is not A' is datum, 'B is not A' conclusion.
Accordingly we must read Ar KUL Br, which do appear as joint
data in (2 b). The corruption was very likely to occur, in view
of the association of the propositions 'C is not A' and' B is not A'
in 814, 17-18, 19-20.
z8. tl TQ.UT1'j" ~EAT(WV tl Ka.T1'jYOPLKTJ. That affirmative proof
is superior to negative was proved in ch. 25.COMMENTARY
CHAPTER 27
The more abstract science is superior to the less abstract
87831. One science is more precise than and prior to another,
(I) if the first studies both the fact and the reason, the second
only the fact; (2) if the first studies what is not, and the second
what is, embodied in a subject-matter (thus arithmetic is prior
to hannonics) ; (3) if the first studies simpler and the second more
complex entities (thus arithmetic is prior to geometry, the unit
being substance without position, the point substance with
position).
87'31-3' ' AlCpl~EaTEf'a. S' ... SLOTL. At first sight it looks as
if we should put a comma after xwpiS" TOO on, and suppose A.
to be placing a science which studies both the fact and the reason,
and not the fact alone (if we take XWPIS adverbally), or not the
reason without the fact (if we take xwp{<; as a preposition), above
one which studies the reason alone. But it seems impossible to
reconcile either of these interpretations with A.'s general view,
and there is little doubt that T. 37. 9-1I,P. 299. 27-8, and
Zabarella are right in taking d'\'\a fL~ xwpiS" 'ToO on 'TiiS" 'TOV S,on to
mean, by hyperbaton, 'but not of the fact apart from the know-
ledge of the reason'. A. will then be referring to such a situation
as is mentioned in 78b39-79aI3, where he distinguishes mathe-
matical astronomy, which knows the reasons, from nautical
astronomy, which knows the facts, and similarly distinguishes
mathematical harmonics from ~ KaTa TIJIJ aK0l)IJ, and mathematical
optics from TO TrEpt TfjS" ,p,SoS", the empirical study of the rainbow.
The study of the facts without the reasons is of course only by
courtesy called a science at all, being the mere collecting of
unexplained facts.
Thus A. in the first place ranks genuine sciences higher than
mere collections of empirical data. He then goes on to rank pure
sciences higher than applied sciences ("33-4)' and pure sciences
dealing with simple entities higher than those that deal with
more complex entities ('34-7).
36. otOY .... Oycl.S ••• Ono!:. The definition of the point is taken
from the Pythagoreans; cf. Prod. in Euc. El. 95. 21 o{ IIv8ayopHo,
TO I77JfLE'iOIJ dcpOp{'OIJTa, fLoIJciSa. TrpoCT'\a/300CTaIJ 8ECT'IJ. A.'s use of the
term oVCT{a in defining the unit and the point is not strictly
justified, since according to him mathematical entities have no
existence independent of subjects to which they attach. But he597
can call them ouaiaL in a secondary sense, since in mathematics
they are regarded not as attributes of substances but as subjects
of further attributes.
CHAPTER 28
W h~t constitutes the unity of a science
87838. A single science is one that is concerned with a single
genus, i.e. with all things composed of the primary elements of
the genus and heing parts of the subject, or essential properties
of such parts. Two sciences are different if their first principles
are not derived from the same origin, nor those of the one from
those of the other. The unity of a science is verified when we
reach the indemonstrables; for they must be in the same genus
as the conclusions from them; and the homogeneity of the first
principles can in turn be verified by that of the conclusions.
87 a 38--c). MLa S' E'II"lcrn;I'TJ •.• ailTCl. A science is one when its
subjects are species (p.£P7J) of a single genus, composed of the
same ultimate elements, and when the predicates it ascribes to
its subjects are per se attributes of those species.
3C)-b I • €TEpa S' ... ETEpwv, When the premisses of two pieces
of reasoned knowledge are derived from the same ultimate
principles, we have two coordinate parts of one science; when the
premisses of one are derived from the premisses of the other, we
have a superior and a subaltern branch of the same science;
cf. 78h34-79aI6.
hI. I'tl 9' a.TEpal El( TWV €TEpwv, The grammar requires aT£paL.
The MSS. of T. and P. are divided between £T£paL and at £T£paL,
but P. seems to have read aT£paL or at £T£paL (TOts" 8£ rijS" £T£paS"
8£wp~p.aaLv apxa'iS" ~ £T£pa XPo/TO, 303· 9- 10 ).
1-4. TOUTOU SE ... O"uYY(Vil. Since the conclusions of a science
must fall within the same genus (deal with the same subject-
matter) as its premisses, the homogeneity of the conclusions can
be inferred from that of the premisses, or vice versa.
CHAPTER 29
How there may be several demonstrations of one connexion
87 h S. There may be several proofs of the same proposition,
(I) if we take a premiss linking an extreme term with a middle
term not next to it in the chain; (2) if we take middle terms from
different chains, c.g. pleasure is a kind of change because it is59 8
COMMENTARY
a movement, and also because it is a coming to rest. But in such
a case one middle term cannot be universally deniable of the
other, since both are predicable of the same thing. This problem
should be considered in the other two figures as well.
87 b 5-'1' ou I'0YOY ••• Z, i.c. if all r is A, all ..1 is r, all Z is ..1,
and all B is Z, we may omit any two of the middle terms and use
as premisses for the conclusion All B is A (I) All r is A, All B is
r, (2) All..1 is A, All B is..1, or (3) All Z is A, All B is Z, using
in (1) a middle term not directly connected with B, in (3) one not
directly connected with A, and in (2) one not directly connected
with either extreme.
14-15. ou I'~v .•• l'iUIIlY, i.e. each of the middle terms must
be predicable of some part of the other, since both are predicable
of pleasure.
16-18. €1I'LGKi",aaeaL S~ . . . GU).).0YLGI'0V. For infinitivus vi
imperativa, cf. Bonitz, Index, 343"22-34.
CHAPTER 30
Chance conjunctions are not demonstrable
87b19. There cannot be demonstrative knowledge of a chance
event; for such an event is neither necessary nor usual, while
every syllogism proceeds from necessary or usual premisses, and
therefore has necessary or usual conclusions.
For A.'s doctrine of chance cf. Phys. ii. 4--6, A. Mansion, Intro-
duction a la Physique Aristotiiicienne, ed. 2 (1946), and the Intro-
duction to my edition of the Physics, 38-41.
CHAPTER 31
There can be no demonstration through sense-perception
87bz8. It is impossible to have scientific knowledge by per-
ception. For even if perception is of a such and not of a mere
this, still what we perceive must be a this here now. For this
reason a universal cannot be perceived, and, since demonstrations
are universal, there cannot be science by perception. Even if it
had been possible to perceive that the angles of a triangle equal
two right angles, we should still have sought for proof of this.
So even if we had been on the moon and seen the earth cutting
off the sun's light from the moon, we should not have known the
cause of eclipse.599
88·~.
Still, as a result of seeing this happen often we should
have hunted for the universal and acquired demonstration; for
the universal becomes clear from a plurality of particulars. The
universal is valuable because it shows the cause, and therefore
universal knowledge is more valuable than perception or intuitive
knowledge, with regard to facts that have causes other than
themselves; with regard to primary truths a different account
must be given.
9. Thus you cannot know a demonstrable fact by perception,
unless one means by perception just demonstrative knowledge.
Yet certain gaps in our knowledge are traceable to gaps in our
perception. For there are things which if we had seen them we
should not have had to inquire about-not that seeing con-
stitutes knowing, but because we should have got the universal
as a rest{lt of seeing.
87 b 37. ClcTTrEp 4»aa( TLYES. The reference is to Protagoras'
identification of knowledge with sensation; cf. PI. Theaet. ISI e-
152 a.
88"1. Kat OV 5Lcm 3Xws, 'and not at all why it happens'. For
this usage of oAw, with a negative cf. Bonitz, Index, 506. 1-10.
Z-4. ov fL~Y «XX' ••• ELXOfLEV. The knowledge of a universal
principle which supervenes on perception of particular facts is
not itself deduction but intuitive knowledge, won by induction
("16-17); but the principles thus grasped may become premisses
from which the particular facts may be deduced.
6-8. ClaTE 1TEpt TWV TOLOUTWV ••• Xoyos. What A. is saying
here is that where there is a general law that depends on a still
more general principle, the only way of really knowing it is to
derive it by demonstration from the more general principle. It
cannot be grasped by sensation, which can only yield awareness
of particular facts; nor by intellectual intuition, which grasps
only the most fundamental general principles. For the latter
point cf. ii. 19, especially IOO b I2 vou, av £LT) 7'WV o.pXwv.
14-16. otOY EL ••• Ka(EL. This is a reference to Gorgias' explana-
tion of the working of the burning-glass-fr. 5 Diels (= Theophr.
de I gne 73) ;"a.7T7'eraL SE 0.1T0 T£ Tij, MAov .•. oVX, wa1Tf,p ropylar:
<I"la, Ka,
a.>..AoL S£ TLV£,
otOV'TaL,
SLd 7'6 o.1TL£VaL T6 1TUP SLd 7'WV 1T/JPwv.600
COMMENTARY
CHAPTER 32
AU syllogisms cannot have the same first principles
88°18. That the starting-points of all syllogisms are not the
same can be seen (I) by dialectical arguments. (a) Some syllogisms
are true, others false. A true conclusion may indeed be got from
false premisses, but that happens only once. A may be true of
C though A is untrue of Band B of C. But if we take premisses
to justify these premisses, these will be false. because a false
conclusion can only come from false premisses; and false pre-
misses are distinct from true premisses.
27. (b) Even false conclusions do not come from the same
premisses; for there are false propositions that are contrary or
incompatible.
30. Our thesis may be proved (2) from the principles we have
laid doWIl. (a) Not even all true syllogisms have the same starting-
points. The starting-points of many true syllogisms are different
in kind, and not applicable to things of another kind (e.g. those
concerning units are not applicable to points). They would have
either to be inserted between the extreme terms, or above the
highest or below the lowest, or some would be inside and some
outside.
36. (b) Nor can there be any of the common principles, from
which everything will be proved; for the genera of things are
different. and some principles apply only to quantities, others
only to qualities, and these are used along with the common
principles to prove the conclusion.
b 3 . (c) The principles needed to prove conclusions are not
much fewer than the conclusions; for the principles are the pre-
misses, and premisses illvolve either the addition of a term from
outside or the interpolation of one.
6. (d) The conclusions are infinite in number, but the terms
supposed to be available are finite.
7. (e) Some principles are true of necessity, others are con-
1.ingent.
9. It is clear, then, that. the conclusions being infinite, the
principles cannot be a finite number of identical principles. Let
us consider other interpretations of the thesis. (I) If it is meant
that precisely these principles are principles of geometry, these
of arithmetic, these of medicine, this is just to say that the sciences
have their principles; to call the principles identical because
they are self-identical would be absurd, for at that rate all things
would be identical.-1. 32. 880I9-3I
60I
IS. Nor (2) does the claim mean that it is from all the principles
taken together that anything is proved. That would be too naIve;
for this is not so in the manifest proofs of mathematics, nor is it
possible in analysis, since it is immediate premisses that are the
principles, and a new conclusion requires the taking in of a new
immediate premiss.
20. (3) If it be said that the first immediate premisses are
the principles, we reply that there is one such peculiar to each
genus.
21. (4) If it is not the case that any conclusion requires all the
principles, nor that each science has entirely different principles,
the possibility remains that the principles of all facts are alike
in kind, but that different conclusions require different premisses.
But this is not the case; for we have shown that the principles
of things different in kind are themselves different in kind. For
principles are of two sorts, those that are premisses of demonstra-
tion, which are common, and the subject-genus, which is peculiar
(e.g. number, spatial magnitude).
88319. 1TPWTOV IJ-£v AOYLKW5 OEwpouaw. The arguments in 319-
30 are called dialectical because they take account only of the
general principles of syllogistic reasoning, and not of the special
character of scientific reasoning.
19-26. oi IJ-£V yap ••• Ta.ATJOfj. This first argument is to the
effect that all syllogisms cannot proceed fwm the same premisses,
since broadly speaking true conclusions follow from true pre-
misses and false from false. A. has to admit that there are
exceptions; a true conclusion can follow from false premisses.
But this, he claims, can only happen once in a chain of reasoning,
since the false premisses from which the conclusion follows must
themselves have false premisses, which must in turn have false
premisses, and so on.
The argument is a weak one; for not both the premisses of
a false conclusion need be false, so that there may be a con-
siderable admixture of true propositions with false in a chain of
reasoning. A. himself describes the argument as dialectical (3I9).
27-30. EC7TL yap ••• £Ao,TTOV. 'What is equal is greater' and
'what is equal is less' are offered as examples of contrary false
propositions; 'justice is injustice' and 'justice is cowardice', and
again 'man is horse' and 'man is ox' as examples of incompatible
false propositions. It is evident that no two propositions so
related can be derived from exactly the same premisses.
30-1. 'EK Se TWV KELIJ-EVWV ••• 1TI1VTWV. The dialectical arguments602
COMMENTARY
in "19-3° took account of the existence of false propositions;
the scientific arguments in "3o-b29, being based on 'Ta K.dJ-L£va, on
what has been laid down in the earlier part of the book with
regard to demonstrative science, take account only of true proposi-
tions, since only true premisses (7Ib19-26), and therefore only
true conclusions, find a place in science.
JI-6. £'TEPQ~ ya.p ..• opwv. A. considers, first, propositions
which form the actual premisses of proof, i.e. (J'an, (iJ1To(J'un,
and 0p,uJ-L0t) (72"14-16, 18-24). These, he says, are in the case of
many subjects generically different, and those appropriate to
one subject cannot be applied to prove propositions about
another subject. If we want to prove that B is A, any terms
belonging to a different field must be introduced either (1) as
terms predicable of B and having A predicable of them, or (2)
as terms predicable of A. or of which B is predicable, or (3) some
of them will be introduced as in (1) and some as in (2). In any
case we shall have terms belonging to one field predicated of
terms belonging to another field, which we have seen in ch. 7 to
be impossible in scientific proof. Such propositions could
obviously not express connexions Ka(J' am-o.
J6- b J. ci~X ouSE ••• KOWWV. A. passes now to consider an-
other suggestion, that some of the a.~,wJ-La'Ta (72016-18), like the
law of excluded middle. can be used to prove all conclusions. In
answer to this he points out that proof requires also special
principles peculiar to different subjects (i.e. those considered in
88"31-6), proof taking place through the a.~~J-La'Ta along with such
special principles. The truth rather is that the special principles
form the premisses, and the common principles the rules according
to which inference proceeds.
bJ-7. En a.t apxa.l . . . EVSEX6I'EvQL. A. has given his main
proof in "31-b3, viz. that neither can principles proper to one main
genus be llsed to prove properties of another. nor can general
principles true of everything serve alone to prove anything. He
now adds, rather hastily, some further arguments. (I) The first
is that (a) the theory he is opposing imagines that the vast
variety of conclusions possible in science is proved from a small
identical set of principles; while in fact (b) premisses are not
much fewer than the conclusions derivable from them; not much
fewer, because the premisses required for the increase of our
knowledge are got not by repeating our old premisses, but either
(if we aim at extending our knowledge) by adding a major higher
than our previous major or a minor below our former minor
, or (if we aim at making our knowledge more thorough) by interpolating a middle term between two of
our previous terms (Jp.{Ja>v'op'/yov).
(b) is a careless remark. A. has considered the subject in An.
Pr. 42bI6-26, where he points out that if we add a fresh premiss
to an argument containing n premisses or n+I terms, we get
n new conclusions. Thus (i) from two premisses 'A is B', 'B is
C we get one conclusion, 'A is C, (ii) from three premisses' A is
B', 'B is C, 'C is D', we get three conclusions, 'A is C, 'A is D'
'B is D', (iii) from four premisses 'A is B', 'B is C, 'C is D',
'D is E' we get six conclusions 'A is C, 'A is D', 'A is E', 'B is
D', 'B is E', 'C is E'-and so on. With n premisses we have
n(n-r) conclusions, and as n becomes large the disparity between
2
the number of the premisses and that of the conclusions becomes
immense. That is what happens when the new terms are added
from outside ('TTpo(rn8~p.iYov 42bI8, 'TTpou>.ap.{Jayop.iyov ggbS). The
same thing happens if new terms are interpolated (Kay £L<; T6
p.iaoy o~ 'TTap£p.'TTl'TTTfI 42b23, Jp.{Ja>V.op.lyov 88bS), and A. concludes
'so that the conclusions are much more numerous than either
the terms or the premisses' ( 42 b2S ---n). It is only if the number of
premisses is itself comparatively small that it can be said to be
'little less than the number of the conclusions'; one is tempted
to say that if A. had already known the rule which he states in
the Prior A nalytics he would nardly have written as he does here,
and that An. Pr. i. 2S must be later than the present chapter.
The next sentence (b6- 7) is cryptic enough, but can be inter-
preted so as to give a good sense. 'If the apxal of all syllogisms
were the same, the terms which, combined into premisses, have
served to prove the conclusions already drawn-and these terms
must be finite in number-are all that are available for the
proving of all future conclusions, to whose number no limit can
be set. But in fact a finite number of premisses can be combined
only into a finite number of syllogisms.'
If this interpretation be correct, the argument is an ingenious
application of A.'s theory that there is no existing infinite but
only an infinity of potentiality (Phys. iii. 6-8).
Finally (b 7-8) A. points out that some principles are apodeictic,
some problematic; this, taken with the fact that conclusions
have a modality varying with that of their premisses (cf. An. Pr.
4Ib27-3I), shows that not all conclusions can be proved from the
same premisses.
9-:19. OUTW jllV o~v • • • jl£yE8oS. A. turns now to consider
other interpretations of the phrase 'the first principles of allCOMMENTARY
syllogisms are the same'. Does it mean (I) that the first principles
of all geometrical propositions are identical, those of all arithmeti-
cal propositions are identical, and those of all medical propositions
are identical? To say this is not to maintain the identity of all
first principles but only the self-identity of each set of first
principles, and to maintain this is to maintain nothing worth
maintaining (blo- Is ).
(2) The claim that all syllogisms have the same principles can
hardly mean the claim that any proposition requires the whole
mass of first principles for its proof. That would be a foolish
claim. We can see in the sciences that afford clear examples of
proof (i.e. in the mathematical sciences) that it is not so in fact;
and we can see by attempting the analysis of an argument that
it cannot be so; for each new conclusion involves the bringing
in of a new premiss, which therefore cannot have been used in
proving the previous conclusions (b IS - 20).
(3) The sentence in b2o-1 has two peculiar features. (a) The
first is the phrase nis 7rpclrras a/daov, 7rpo-rauns. 7rpw-ro, is very
frequently used in the same sense as afl.£Go" but if that were its
meaning here A. would almost certainly have said 7rpw-ras
KaL afl.£aou, (cf. e.g. 7Ib2I). The phrase as wc have it must point
to primary immediate premisses as distinct from the immediate
premisses in general which have been previously mentioned.
(This involves putting a comma after 7rpo-raa£L, and treating
-ram-a, as a repetition for the sake of emphasis; cf. 72b7-8 and
many examples in Kiihner, Cr. Cramm. § 469.4 b.) (b) The same
point emerges in the phrase fl.ta £11 £Kaanp r£II€'. This must mean
that out of all the principles proper to a subject-matter and not
available for the study of other subject-matters, there is one that
is primary. Zabarella is undoubtedly right in supposing this to
be the definition of the subject-matter of the science in question,
e.g. of number or of spatial magnitude (cf. b 2 8---<}); for it is from
the subject's essential nature that its consequential properties
are deduced.
(4) (b2I ---<}) If what is maintained is neither (2) nor (I) but an
intermediate view, that the first principles of all proof are identi-
cal in genus but different in species, the answer is that, as we
have already proved in ch. 7, generically different subjects have
generically different principles. Proof needs not only common
principles (the axioms) but also special principles relating to the
subject-matter of the science, viz. the definitions of the terms
used in the science, and the assumptions of the existence of the
primary subjects of the scit'nce (cf. 72314--24).I. 32- 88b9-29
605
Cherniss (A.'5 Criticism of Plato and the Academy, i. 73 n.)
argues with much probability that this fourth view is that of
Speusippus, who insisted on the unity of all knowledge, the know-
ledge of any part of reality depending on exhaustive knowledge
of all reality, and all knowledge being a knowledge of similarities
(6J-LotOT7]!; = avyyEIIHU). Ct. 9786·-Il n.
CHAPTER 33
Opinion
88b30. Knowledge differs from opinion in that knowledge is
universal and reached by necessary, i.e. non-contingent, premisses.
There are things that are true but contingent. Our state of mind
with regard to them is (r) not knowledge; for then what is con-
tingent would be necessary; nor (2) intuition (which is the start-
ing-point of knowledge) or undemonstrated knowledge (which is
apprehension of an immediate proposition). But the states of
mind capable of being true are intuition, knowledge, and opinion;
so it must be opinion that is concerned with what is true or false,
but contingent.
89 8 3. Opinion is the judging of an unmediated and non-
necessary proposition. This agrees with the observed facts; for
both opinion and the contingent are insecure. Besides, a man
thinks he has opinion, not when he thinks the fact is necessary-
he then thinks he knows-but when he thinks it might be other-
wise.
11. How then is it possible to have opinion and knowledge of
the same thing? And if one maintains that anything that is
known could be opined, will not that identify opinion and know-
ledge? A man who knows and one who opines will be able to
keep pace with each other through the chain of middle terms till
they reach immediate premisses, so that if the first knows, so
does the second; for one may opine a reason as well as a fact.
16. \Ve answer that if a man accepts non-contingent proposi-
tions as he does the definitions from which demonstration pro-
ceeds, he will be not opining but knowing; but if he thinks the
propositions are true but not in consequence of the very nature
of the subject, he will have opinion and not genuine knowledge-
both of the fact and of the reason, if his opinion is based on the
immediate premisses; otherwise, only of the fact.
23. There cannot be opinion and knowledge of what is com-
pletely the same; but as there can be false and true opinion of
what is in a sense the same, so there can be knowledge and opinion.606
COMMENTARY
To maintain that true and false opinion have strictly the same
object involves, among other paradoxical consequences, that one
does not opine what one opines falsely. But since 'the same' is
ambiguous, it is possible to opine truly and falsely what is in one
sense the same, but not what is so in another sense. It is impos-
sible to opine truly that the diagonal of a square is commen-
surate with the side; the diagonal, which is the subject of both
opinions, is the same, but the essential nature ascribed to the
subjects in the two cases is not the same.
33. So too with knowledge and opinion. If the judgement be
'man is an animal', knowledge is of 'animal' as a predicate that
cannot fail to belong to the subject, opinion is of it as a predicate
that need not belong; or we may say that knowledge is of man
in his essential nature, opinion is of man but not of his essential
nature. The object is the same because it is in both cases man,
but the mode in which it is regarded is not the same.
38. It is evident from this that it is impossible to opine and
know the same thing at the same time; for that would imply
judging that the fact might be otherwise, and that it could not.
In different persons there may be knowledge and opinion of the
same thing in the sense just described, but in the same person
this cannot happen even in that sense; for then he would be
judging at the same time, for example, that man is essentially an
animal and that he is not.
The question how the remaining functions should be as-
signed to understanding, intuitive reason, science, art, practical
wisdom, and philosophical knowledge belongs, rather, in part to
physics and in part to ethics.
b,.
88b35-7.
ci~~Q
Il-Ttv . . . 1TPOTQO'(WS. Though the phrase E'TTt-
is common in A., the phrase which is implied
as its opposite, E'TTtO'T~fLTJ aVa7T()i)HKTo" occurs only here and in
72bI9-20. Where E'TTtO'nJfLTJ is used without qualification it means
demonstrative knowledge; with the qualification aVQ7T()i)HKTo, it
means mental activity which shares with demonstrative know-
ledge the characteristics of possessing subjective certainty and
grasping necessary truth, but differs from it in being immediate,
not ratiocinative. Nuw this is exactly the character which A.
constantly ascribes to VDU" and which the identification of VDU,
with the dpX~ E'TTtO'T~fLTJ' (b 36) implies VDU, to possess. Finally, in
89aI E'TTtO'T~fLTJ aVa7T()i)ELKTO, does not appear alongside of VDU"
E'TTtaT~fLTJ (i.e. E'TTtanJfLTJ a7Toi)HKTLK~), and i)o~Q. It must therefore
be mentioned here not as anything distinct from VDU, but as
O'T~fLTJ a7Toi)HKTtK'7janother name for it; and I have altered the punctuation accord-
ingly. Just as Kat in an affirmative statement can have explicandi
magis quam copulandi vim (Bonitz, Index, 3S7bI3-20), so can ovo'
in a negative sentence.
89"3-4' TO(iTO 8' . . . o.va.YKa.ia.c;. bwrnlp.1] aVa7TOOUIM"O, has
been defined as imoA1].pL' T1i. ap.£CTov 7TpOTaCTf.w., i.e. of a premiss
which is unmediable because its predicate belongs directly and
necessarily to its subject. oo~a is imoA1].pL' rT)' a,.dCTov 7TpOTaCTf.w.
Kal p.~ avaYKata" i.e. of a premiss which is ap.f.C1o, for another
reason, viz. that (whether it has been reached by incorrect reason-
ing or without reasoning; for opinion may occur in either case),
it has not been mediated, i.e. derived by correct reasoning from
necessary premisses.
17-18. WcnrEP [iXEL] TOUS OpLCT!10US. Neither £Xf.L, the reading
of the best MSS., nor £XHV, which is adopted by Bekker and
Waitz, gives a tolerable sense, and I have treated the word as an
intruder from the previous line.
:15-8. Ka.t ya.p ... 1JIEu8ws. The view referred to is the sceptical
view discussed in M et. r which denies the law of contradiction.
In holding that a single thing B can both have a certain attribute
A and not have it, such thinkers imply that there can be both
a true and a false opinion that B is A (or that B is not A). This
was not the doctrine of a single school; it was rather a view of
which A. found traces in many of his predecessors-Heraclitus
(Met. lor2"24, 34) and his school (Ioro"lo), Empedocles (ro09 brs),
Anaxagoras (1009"27, b2S ), Democritus (1009327, b u , IS), Prota-
goras (r009"6).
Besides the many paradoxical consequences which A. shows in
the Metaphysics to follow from this view, there is (he here says)
the self-contradictory consequence that what a man opines
falsely he does not opine at all. This consequence arises in the.
following way: if the object of true and of false opinion is (as
these thinkers allege) the same, anyone who entertains this object
must be thinking truly; so that if a man be supposed to be
thinking falsely, it turns out that he cannot really be thinking
what he was supposed to be thinking falsely.
:19-3:1. TO .... EV ya.p •.. a.{ITC). There cannot be a true opinion
that the diagonal of a square is commensurate with the side.
There can indeed be a true opinion that the diagonal is not
commensurate, and a false opinion that it is commensurate, and
these opinions are 'of the same thing' in so far as they are both
about the diagonal. But the essential nature (as it would be
stated in a definition) ascribed to the subject is different in the608
COMMENTARY
two cases; not that 'commensurate' or 'not commensurate' is
included in the definition, but that since properties follow from
essence, it would only be by having a different essence that the
diagonal, which is in fact not commensurate, could be com-
mensurate.
33-7. 0llo(wS S€ ... o,lho. A. has pointed out that a true and
a false judgement with the same subject and the same predicate
must differ in qUality. He now insists that knowledge and opinion
about the same subject and the same predicate differ in modality.
He takes as his example the statement 'man is an animal' (cf.
b 4). The knowledge that man is an animal is 'of animal', but of
it as a predicate that cannot fail to belong to man; the opinion
that man is an animal is also 'of animal' but of it as a predicate
that belongs, but need not belong, to man. Or, to put the matter
with reference to the subject, the one is 'of what man essentially
is', the other is 'of man', but not 'of what man essentially is'.
For the phrase ~ /LEII om,p dllBpunrov £UTtll, strict grammar would
require ~ /LEII TOVTOV £UT'" om,p all8pw7T(J.; £unll. But 07TEP /illBpw7TO,
has through constant usage almost coalesced into one word, so
that the genitive inflection can come at the end. Cf. Met. 1007822,
23, 28 07TEP dllBpunrqJ Eflla,.
b2-3. EV &AA~ ... otov TE. Two people can respectively know
and opine what is the same proposition in the sense explained
in a33-7 (there should be no comma before W, E'ifY']Ta, in b 2 ) ; i.e.
two propositions with the same subject and predicate but different
modalities; one person cannot at one time know and opine what
is the same proposition even in this sense, still less a strictly self-
identical proposition.
7-9. T e. S€ Aome. ... EUTLV. A. has in this chapter considered
the difference between knowledge and opinion, because know-
ledge (i.e. demonstrative knowledge) is the subject of the Posterior
Analytics. But a full discussion of how the operations of thought
are to be assigned respectively to 8,dllo,a (discursive thought)
and its species-£maT~/L7J (knowledge pursued for its own sake),
TIXVTJ (knowledge applied to production), and cf>p6VTJu,,> (knowledge
applied to conduct)-and to I'OU, (intuitive reason) and uocf>ta
(metaphysical thought, the combination of IIOU, and £m~/L7J),
is a matter for the sciences that study the mind itself-psy-
chology (here included under physical science) and ethics. IIOU,
is in fact discussed in De An. iii. 4-7 and in E.N. vi. 6, £maT~/L7J
in E.N. vi. 3, TIXVTJ ib. 4, cf>p6VTJ(M ib. 5, uocf>ta ib. 7·60 9

CHAPTER 34

Quick wit

Quick wit is a power of hitting the middle term in an imperceptible time;

    if one sees that the moon always has
    its bright side towards the sun, and
    quickly grasps the reason,
    viz. that it gets its light from the sun;
    or recognizes that someone
    is talking to a rich man because he is borrowing from him; or
    why two men are friends, viz. because they have a common
    enemy.

On seeing the extremes one has recognized all the middle terms
