CHAPTER 1
The student's need of pre-existent knowledge. Its nature

CHAPTER 2
The nature of scientific knowledge and of its premises

CHAPTER 3
Two errors-the view that knowledge is impossible because it involves
an infinite regress, and the view that circular demonstration is
satisfactory

CHAPTER 4
The premises of demonstration must be such that the predicate is
true of every instance of the subject, true of the subject per se,
and true of it precisely qua itself

CHAPTER 5
How we fall into, and how we can avoid, the error of thinking our
conclusion a true universal proposition when it is not

CHAPTER 6
The premises of demonstration must state necessary connections

CHAPTER 7
The premises of a demonstration must state essential attributes of
the same genus of which a property is to be proved

CHAPTER 8
Only eternal connections can be demonstrated

CHAPTER 9
The premises of demonstration must be peculiar to the science in
question, except in the case of subaltern sciences

CHAPTER 10
The different kinds of ultimate premiss required by a science

CHAPTER 11
The function of the most general axioms in demonstration

CHAPTER 12
Error due to assuming answers to questions inappropriate to the
science distinguished from that due to assuming wrong answers to
appropriate questions or to reasoning wrongly from true and
appropriate assumptions. How a science grows

CHAPTER 13
Knowledge of fact and knowledge of reasoned fact

CHAPTER 14
The first figure is the figure of scientific reasoning

CHAPTER 15
There are negative as well as affirmative propositions that are
immediate and indemonstrable

CHAPTER 16
Error as inference of conclusions whose opposites are
immediately true

CHAPTER 17
Error as inference of conclusions whose opposites can be
proved to be true

CHAPTER 18
Lack of a sense must involve ignorance of certain universal propositions
which can only be reached by induction from particular facts

CHAPTER 19
Can there be an infinite chain of premises in a demonstration,
(1) if the primary attribute is fixed, (2) if the ultimate subject is fixed,
(3) if both terms are fixed?

CHAPTER 20
There cannot be an infinite chain of premises if both extremes
are fixed

CHAPTER 21
If there cannot be an infinite chain of premises in affirmative
demonstration, there cannot in negative

CHAPTER 22
There cannot be an infinite chain of premises in affirmative
demonstration, if either extreme is fixed

CHAPTER 23
Corollaries from the foregoing propositions

CHAPTER 24
Universal demonstration is superior to particular

CHAPTER 25
Affirmative demonstration is superior to negative

CHAPTER 26
Ostensive demonstration is superior to reductio ad impossibile

CHAPTER 27
The more abstract science is superior to the less abstract

CHAPTER 28
What constitutes the unity of a science

CHAPTER 29
How there may be several demonstrations of one connection

CHAPTER 30
Chance conjunctions are not demonstrable

CHAPTER 31
There can be no demonstration through sense-perception

CHAPTER 32
All syllogisms cannot have the same first principles

CHAPTER 33
Opinion

CHAPTER 34
Quick wit
