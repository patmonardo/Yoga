CHAPTER 1
There are four types of inquiry

CHAPTER 2
They are all concerned with a middle term

CHAPTER 3
There is nothing that can be both demonstrated and defined

CHAPTER 4
It cannot be demonstrated that a certain phrase is the definition
of a certain term

CHAPTER 5
It cannot be shown by division that a certain phrase is the
definition of a certain term

CHAPTER 6
Attempts to prove the definition of a term by assuming the definition
either of definition or of the contrary term beg the question

CHAPTER 7
Neither definition and syllogism nor their objects are the same;
definition proves nothing; knowledge of essence cannot be got either
by definition or by demonstration

CHAPTER 8
The essence of a thing that has a cause distinct from itself cannot be
demonstrated, but can become known by the help of demonstration

CHAPTER 9
What essences can and what cannot be made known by demonstration

CHAPTER 10
The types of definition

CHAPTER 11
Each of four types of cause can function as middle term

CHAPTER 12
The inference of past and future events

CHAPTER 13
The use of division (a) for the finding of definitions

CHAPTER 14
The use of division (b) for the orderly discussion of problems

CHAPTER 15
One middle term will often explain several properties

CHAPTER 16
Where there is an attribute commensurate with a certain subject,
there must be a cause commensurate with the attribute

CHAPTERS 17, 18
Different causes may produce the same effect, but not in things
specifically the same

CHAPTER 19
How we come by the apprehension of first principles
