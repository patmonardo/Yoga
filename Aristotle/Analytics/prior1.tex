CHAPTER 1
Subject and scope of the Analytics. Definitions of fundamental terms

CHAPTER 2
Conversion of pure propositions

CHAPTER 3
Conversion of modal propositions

CHAPTER 4
Assertoric syllogisms in the first figure

CHAPTER 5
Assertoric syllogisms in the second figure

CHAPTER 6
Assertoric syllogisms in the third figure

CHAPTER 7
Common properties of the three figures

CHAPTER 8
Syllogisms with two apodeictic premisses

CHAPTER 9
Syllogisms with one apodeictic and one assertoric premiss, in the
first figure

CHAPTER 10
Syllogisms with one apodeictic and one assertoric premiss, in the
second figure

CHAPTER 11
Syllogisms with one apodeictic and one assertoric premiss, in the
third figure

CHAPTER 12
The modality of the premisses leading to assertoric or apodeictic
conclusions

CHAPTER 13
Preliminary discussion of the contingent

CHAPTER 14
Syllogisms in the first figure with two problematic premisses

CHAPTER 15
Syllogisms in the first figure with one problematic and one assertoric
premiss

CHAPTER 16
Syllogisms in the first figure with one problematic and one apodictic
premiss

CHAPTER 17
Syllogisms in the second figure with two problematic premisses

CHAPTER 18
Syllogisms in the second figure with one problematic and one assertoric
premiss

CHAPTER 19
Syllogisms in the second figure with one problematic and one
apodeictic premiss

CHAPTER 20
Syllogisms in the third figure with two problematic premisses

CHAPTER 22
Syllogisms in the third figure with one problematic and one apodeictic
premiss

CHAPTER 23
Every syllogism is in one of the three figures, and reducible to a
universal mood of the first figure

CHAPTER 24
Quality and quantity of the premisses

CHAPTER 25
Number of the terms, premisses, and conclusions

CHAPTER 26
The kinds of proposition to be proved or disproved in each figure

CHAPTER 27
Rules for categorical syllogisms, applicable to all problems

CHAPTER 28
Rules for categorical syllogisms, peculiar to different problems

CHAPTER 29
Rules for reductio ad impossibile, hypothetical syllogisms, and
modal syllogisms

CHAPTER 30
Rules proper to the several sciences and arts

CHAPTER 31
Division

CHAPTER 32
Rules for the choice of premisses, middle term, and figure

CHAPTER 33
Error of supposing that what is true of a subject in one respect is
true of it without qualification

CHAPTER 34
Error due to confusion between abstract and concrete terms

CHAPTER 35
Expressions for which there is no one word

CHAPTER 36
The nominative and the oblique cases

CHAPTER 37
The various kinds of attribution

CHAPTER 38
The difference between proving that a thing can be known, and
proving that it can be known to be so-and-so

CHAPTER 39
Substitution of equivalent expressions

CHAPTER 40
The difference between proving that B is A and proving that B is the A

CHAPTER 41
The difference between' A belongs to all of that to which B belongs'
and 'A belongs to all of that to all of which B belongs'. The 'setting
out' of terms is merely illustrative

CHAPTER 42
Analysis of composite syllogisms

CHAPTER 43
In discussing definitions, we must attend to the precise point at issue

CHAPTER 44
Hypothetical arguments are not reducible to the figures

CHAPTER 45
Resolution of syllogisms in one figure into another

CHAPTER 46
Resolution of arguments involving the expressions 'is not A' and
'is not-A'
