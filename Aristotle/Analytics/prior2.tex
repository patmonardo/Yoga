CHAPTER 1
More than one conclusion can sometimes be drawn from the same
premisses

CHAPTER 2
True conclusions from false premisses, in the first figure

CHAPTER 3
True conclusions from false premisses, in the second figure

CHAPTER 4
True conclusions from false premisses, in the third figure

CHAPTER 5
Reciprocal proof applied to first-figure syllogisms

CHAPTER 6
Reciprocal proof applied to second-figure syllogisms

CHAPTER 7
Reciprocal proof applied to third-figure syllogisms

CHAPTER 8
Conversion of first-figure syllogisms

CHAPTER 9
Conversion of second-figure syllogisms

CHAPTER 10
Conversion of third-figure syllogisms

CHAPTER 11
'Reductio ad impossibile' in the first figure

CHAPTER 12
'Reductio ad impossibile' in the second figure

CHAPTER 13
'Reductio ad impossibile' in the third figure

CHAPTER 14
The relations between ostensive proof and 'reductio ad impossibile'

CHAPTER 15
Reasoning from a pair of opposite premisses

CHAPTER 16
Fallacy of 'Petitio principii'

CHAPTERS 17, 18
Fallacy of false cause

CHAPTERS 19, 20
Devices to be used against an opponent in argument

CHAPTER 21
How ignorance of a conclusion can coexist with knowledge of the
premisses

CHAPTER 22
Rules for the use of convertible terms and of alternative terms, and for
the comparison of desirable and undesirable objects

CHAPTER 23
Induction

CHAPTER 24
Argument from an example

CHAPTER 25
Reduction of one problem to another

CHAPTER 26
Objection

CHAPTER 27
Inference from signs
