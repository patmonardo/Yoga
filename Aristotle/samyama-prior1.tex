ANALYTICA PRIORA
BOOK I
CHAPTER 1
Subject and scope of the A nalytics. Definitions of fundamental
terms
24"10. Our first task is to state our subject, which is demonstra-
tion; next we must define certain terms.
16. A premiss is an affirmative or negative statement of some-
thing about something. A universal statement is one which says
that something belongs to every, or to no, so-and-so; a particular
statement says that something belongs to some, or does not
belong to some (or does not belong to every), so-and-so; an
indefinite statement says that something belongs to so-and-so,
without specifying whether it is to all or to some.
22. A demonstrative premiss differs from a dialectical one, in
that the former is the assumption of one of two contradictories,
while the latter asks which of the two the opponent admits; but
this makes no difference to the conclusion's being drawn, since in
either case something is assumed to belong, or not to belong, to
something.
28. Thus a syllogistic premiss is just the affirmation or denial
of something about something; a demonstrative premiss must in
addition be true, and derived from the original assumptions; a
dialectical premiss is, when one is inquiring, the ~sking of a pair
of contradictories, and when one is inferring, the assumption of
what is apparent and probable.
b 1 6. A term is that into which a premiss is analysed (i.e. a
subject or predicate), 'is' or 'is not' being tacked on to the terms.
18. A syllogism is a form of speech in which, certain things
being laid down, something follows of necessity from them, i.e.
because of them, i.e. without any further term being needed to
justify the conclusion.
22. A perfect syllogism is one that needs nothing other than the
premisses to make the conclusion evident; an imperfect syllogism
needs one or more other statements which are necessitated by the
given terms but have not been assumed by way of premisses.
26. For H to be in A as in a whole is the same as for A to be288
COMMENTARY
predicated ot all B. A is predicated of all B when there is no B
of which A will not be stated; 'predicated of no B' has a corre-
sponding meaning.
2481~II. npWTOY ••• cufoliELKTLKT]S, A. here treats the Prior and
the Posterior Analytics as forming one continuous lecture-course
or treatise; for it is not till he reaches the Posterior A nalytics that
he discusses demonstration; in the Prior A nalytics he discusses
syllogism, the form common to demonstration and dialectic.
Tlyos ••• C7K€1jILS might mean either 'what the study is a study of'
('rlvo~ being practically a repetition of 7T€pt Tt), or 'to what science
the study belongs. Maier (z a. I n.), taking Ttvo~, and therefore
also €7Ttarr}I-'TJ~ a.7TOSHKTUdj~, in the latter way, as subjective
genitives, renders the latter phrase 'the demonstrative science'.
But to name logic by this name would be quite foreign to A.'s
usage; €7TI'crn7I-'TJ a7TOS€tK'TL1O) is demonstrative science in general
(cf. An. Post. 99b1S-17), and the genitives must be objective.
Et1l'ELY •• . liLOpLC7Ut. A. not infrequently uses the infinitive
thus, to indicate a programme he is setting before himself, the
infinitive taking the place of a gerund; cf. Top. 106 a lO, b13 , 21, etc.
The imperatival use of the infinitive is explained by Kiihner,
Cr. Cramm. ii. 2. 19-zo.
16. npOTUC7LS. The word apparently does not occur before A.
In A. it is found already in De Int. 20b z3, 24, Top. 10I b1S-37,
10433-37, etc. A 7TpoTaat~ is defined, as here, as one of a pair of
contradictory statements (avTL<p6.a€W~ I-'tOS I-'opwv, De Int. 20b24).
That is its form, and as for its function, it is something to which
one party in a discussion asks the other whether he assents (De
Int. zobZ2-3). Strictly, it differs from a 7Tpo{3ATJl-'a in that it is
stated in the form 'Is AB?', while a 7Tpo{3ATJl-'a is in the form
'Is A B, or not?' (Top. I01 b28-34); but in some of the other
passages of the Topics 7TpoT6.aH~ are stated in the form said to be
proper to 7Tpo{3A~l-'aTa. Further, it appears that the function of
7TpoT6.a€t~ is to serve as starting-points for argument. Thus the
Aristotelian usage of the term 7TpoTaatC; is already to be found in
works probably earlier than the Prior Analytics, though it is
only now that constant use begins to be made of the term.
The usage is derived from a usage of 7TPOTdvHV as meaning
'put forward for acceptance'; but of this again as applied to
statements we have no evidence earlier than A. In A. it is not
uncommon, especially in the Topics; 7TpoTdvwOat occurs once in
the same sense (I64b4). The only other usage of 7Tp6Taat~ which
it is worth while to compare (and contrast) with this is the useof it in the astronomer Autolycus 2. 6 (c. 310 B.c.) and in later
writers, to denote the enunciation of a proposition to be proved.
17. O~TOS 5€ ;; ICQ8o).ou ;; €V ~(P€L ;; ci5LOpLC7TOS. In De Int. 7 a
different classification of propositions in respect of quantity is
given. Entities (Ta 7Tpct.yp.aTU) are divided into Ta KaOOAOU and
Ta KaO' £KauTov, and propositions are divided into (1) those about
universals; (a) predicated universally, (b) predicated non-univer-
sally; (2) those about individuals. This is the basis of the common
doctrine of formal logic, that judgements are universal, parti-
cular, or singular. The treatment of the matter in the Prior
A nalytics is by comparison more formal. It ignores the question
whether the subject of the judgement is a universal or an indivi-
dual, and classifies judgements according as the word 'all', or the
word 'some', or neither, is attached to the subject; and the judge-
ments in which neither 'all' nor 'some' appears are not, as might
perhaps be expected, those about individuals, but judgements
like 'pleasure is not good', where the subject is a universal. In
fact the Prior A nalytics entirely ignores judgements about indi-
viduals, and the example of a syllogism which later was treated
as typical-Man is mortal, Socrates is a man, Therefore Socrates
is mortal-is quite different from those used in the Prior A nalytics,
which are all about universals, the minor term being a species.
A.'s reason for confining himself to arguments about universals
probably lies in the fact mentioned in 43"42-3, that 'discussions
and inquiries are mostly about species'.
21. TO TWV €VQVTLWV ••• €1rLCTTTJ~TlV. The Greek commentators
rightly treat not 'the same science' but 'contraries' as the logical
subject of the statement, which is MWpWTO. because it says T(Jiv
£vaYT{wv and not 7Tct.VTWV TWV (or TLVWV) lvaVTLwv (Am. 18. 28-33,
P.
20. 25).
22-5. 5LQ~Ep€L ••• €anv.
Demonstration firmly assumes the
truth of one of two contradictories as self-evident (or following
from something self-evident); in dialectic the person who is
trying to prove something asks the other party 'Is A B?', and
is prepared to argue from 'A is B' or from 'A is not B', according
as the interlocutor is willing to admit one or the other.
26. f:KQTEPOU, i.e. TOU T£ a7To1)£LKVUoVTO' KaL TOU lPWTWYTOS.
b 12 . €V TOLS T01rLKOLS €lPTlTQL, i.e. in 100°27-30, 104°8.
13-14. TL 5LQ~Ep€L ••• 5LQ).€KTLKT). auAAOYWTLKTt 7TpoTauL. is the
genus of which the other two are species.
14. 5L' a.KPL~€LQS ••• PTl8T)a€TQL. What distinguishes demon-
strative from dialectical premisses is discussed in the Posterior
Analytics (especially 1. 4-12).
4985
uCOMMENTARY
16. "Opov. opo~ in the sense of 'term of a proposition' seems not
to occur before A., nor, in A., before the Analytics. It was
probably used in this sense by an extension from its use to
signify the terms of a ratio, as in Archytas 2 OKKa lWIJ'n Tp£'i~ OPOL
KaTa TaIJ TOLaIJ V7r£POXaIJ aIJa ..\6yoIJ. This arithmetical usage may
itself have developed from the use of 0pos for the notes which
form the boundaries of musical intervals, as in PI. Rep. 443 d
Wa7r£P opov~ Tp£'i~ ap/LOIJLa~ ••. , IJdT7I~ T£ Kat V7raT71~ Kat /L£C1T)~,
Phileb. 17 d TOV~ opOV~ TCVIJ SLaC7T1}/LaTWIJ. The arithmetical usage
is found in A. (e.g. E.N. II31b5 laTaL apa W~ cl a opo~ 7Tpd~ TdIJ
/3, OVTW~ cl Y 7Tpd~ TdIJ S, cf. ib. 9, 16). It also occurs in Euclid
(e.g. V, Def. 8), and if we had moreofthe early Greek mathematical
writings we might find it established before A.'s time. His logical
usage of the word is no doubt original, as, indeed, 0POIJ oE KaAw
suggests. It belongs to the same way of thinking as his use of
aKpa for the terms and of oLaaT7I/La for the proposition, of E/L7TL7TTHIJ,
7Tap£/L7TL7TTHIJ, E/Lj3aAAw(}aL, and KaTa7TVKIJOva(}aL, of /L£L~WIJ and
EAaTTwIJ (opo~), of 7TPWTOIJ, /L£aoIJ, and laxaToIJ.
The probable development of the logical usage of these
words from a mathematical usage as applied to progressions
is discussed at length by B. Einarson in A.J.P. lvii (1936),
290
155--64.
16-17. otov . . . KQ.T'lYOPELTQ.L. The technical sense of KaT71-
is already common in the Categories and in the Topics.
It does not occur before A., but is an easy development from the
use of KaT71yop£!.IJ TL T"'O~ (KaTa T"'O~, 7T£p{ nIJo~), 'to accuse some-
one of something'.
17-18. 1I'POO'TL9EIlEvOU ....... ~ dVQ.L. The vulgate reading ~
7Tpoan(}£/LEIJOV ~ SLaLpOV/LEIJOV TOV EtIJaL Kat /L~ £lIJaL betrays its
incorrectness at two points. (I) The true opposite of 7Tpoan-
(}£/LEIJOV, both according to A.'s usage and according to the nature
of things, is not oLaLpOV/LEIJOV but a,paLpoV/LEIJOV; (2) even if a,paLpov-
/L£IJOV be read, the text would have to be supposed to be an
illogical confusion of two ways of saying the same thing, ~
7Tpoan(}£/L£IJoV ~ acpaLpov/LlIJov TOV £lIJaL, and 7Tpoan(}£/L£IJoV ~ TOV
£[IJaL ~ TOV /L~ £lIJaL. A. can hardly be credited with so gross a
confusion, and though the Greek commentators agree in having,
substantially, the vulgate reading, they have great difficulty in
defending it. There are many other traces of interpolations which
were current even in the time of the Greek commentators (cf.
the apparatus criticus at "17, 29, 32"21-9, 34b2--6). The text as
emended falls completely into line with such passages as De Int.
YOP£!.IJ
16"16 Kat yap cl Tpo.yE..\acpo~ C1T)/La[IJH /LEIJ n, oVrrw SE aA7I(}E~ ~ tP£vSo~,£av fL~ 'TO frVaL ~ fL~ £lVaL 7TPO(7'Tf8fj,
p.~ £ lvat TTPOO'(J£O'£t,.
2Ib27
£7T' £K£tVWV 'TO flva, Kat
TO
IS-ZO. O'U~~OYLO'I-10" ••• dvaL.
The original meaning of CTVAAO-
is 'to compute, to reckon up', as in Hdt. 2. I4S Ta. £g
'EAA~VWV ntx£Ci T£ Ka~ £pywv a7TCIS£gtV O'uAAoytO'atTo. But in Plato
the meaning 'infer' is not uncommon, e.g. Grg. 479 c Ta. O'up.-
/1atvOV'Ta £K TOU AOYOU ... 0'., R. 516 b 0'. TT£p~ aUTou OTt KTA. So too
in Plato we have CTVAAoytO'p.o, in the sense of 'reasoning', in Crat.
412 a aVv£O't, . . . Sog£t£V av WO'7T£P 0'. £[vat, and in Tht. IS6 d £v
p.£v ... TO', TTa(J~p.aO'tv OUK ivt £7Tt~p.T), £v S£ Tcjj TT£pi £K£tvwV u. In
A. O'UAAoyt~£O'(Jat and O'uAAoytUP.O', in the sense of 'reasoning' are
both rare in the Topics (O'VAAoyt~£O'(Jat IOIa4, 153"8, I57 b35-9,
I60 b23, CTVAAOytO'p.o, i. 1 and 12 passim, 130"7, I39b30, 156"20, 21,
I57"IS, b3S, 15S8S-3o), but common in the SoPhistici Elenchi. It has
sometimes been thought that the parts of the Topics in which the
words occur were added later, after the doctrine of the syllogism
had been discovered; but this is not necessary, since the words
occur already in Plato, and the developed Aristotelian doctrine
is not implied in the Topics passages.
The definition here given of CTVAAOYWP.()' is wide enough to cover
all inference. Thus A. does not give a new meaning to the word;
but the detailed doctrine which follows gives an account of some-
thing much narrower than inference in general, since it excludes
both immediate inference and constructive inference in which
relations other than that of subject and predicate are used, as
in' A = B, B = C, Therefore A = C.
ZI-Z. TO SE SLcl. TaUTa. ••• a.Va.YKa.LOV. This excludes, as AI.
points out (2I. 21-23. 2), (I) P.OvoA~p.p.aTOt O'vAAoytO'P.ot, enthymemes
in the modem sense of that word, such as 'A is B, Therefore it
is C; (2) what the Stoics called ap.£(}oSot AOYOt, such as 'A is
greater than B, B is greater than C, Therefore A is greater than
C, where (according to Al.) another premiss is implied-'that
which is greater than that which is greater than a third thing is
greater than the third thing'; (3) arguments of which the pre-
misses need recasting in order to bring them into syllogistic form,
e.g. 'a substance is not destroyed by the destruction of that which is
not a substance, A substance is destroyed by the destruction of its
parts, Therefore the parts of a substance are substances' (47"22-S).
zZ-4. TE~(LOV ••• a.vaYKa.Lov. Superficially this definition of a
perfect syllogism looks as if it were identical with the definition of
a syllogism given in bI S-20 . But if it were identical, this would
imply that so-called anA£" CTVAAoytO'P.ot (i.e. inferences in the
second and third figures) are not CTVAAoytO'p.ot, while both "12-13
yt~£O'(}atCOMMENTARY
and b22-{, imply that they are. The solution of the difficulty lies
in noticing that q,avfivaL TO dvaYKaLov is used in b 24 in contrast
with YH€u8aL TO dvaYKaLov in the definition of syllogism. An
imperfect syllogism needs the introduction of no further pro-
position (;gw8"v 0pou) to guarantee the truth of the syllogism,
but it needs it to make the conclusion obvious. The position of
imperfect syllogisms is quite different from that of the non-
syllogistic inferences referred to in b 2I - 2 n. The latter need
premisses brought in from outside; the former need, in order that
their conclusions may be clearly seen to follow, the drawing out
(by conversion) of premisses in1plicit in the given premisses, or
an indirect use of the premisses by reductio ad impossibile.
26. ou I'T)V EtAT)"lTTUL SLcI. "lTPOTa.O'EWV, 'but have not been secured
by way of premisses'.
26-8. TO SE ••• EO'TLV, 'for A to be in B as in a whole is the
same as for B to be predicated of every A'. If 'animal' is pre-
dicated of every man, man is said to be in animal as in a whole
to which it belongs. That this is the meaning of lv OA't' ElvaL is
clear from 2Sb32-S.
29. TOU \I1TOKELI'EVOU. Al.'s commentary (24. 27-30) implies
that he did not read these words (which are absent also from his
quotations of the passage in 167. 17, 169. 25) ; and their presence
in the MSS. is due to Al.'s using the phrase TOU lJ1TOKup.lvou in his
interpretation. The sense is conveyed sufficiently without these
words.
CHAPTER 2
Conversion of pure propositions
25'1. Every proposition (A) states either that a predicate
belongs, that it necessarily belongs, or that it admits of belonging,
to a subject, (B) is either affirmative or negative, and (C) either
universal, particular, or indefinite.
S. Of assertoric statements, (1) the universal negative is con-
vertible, (2) the universal affirmative is convertible into a par-
ticular, (3) so is the particular affirmative, (4) the particular
negati ve is not convertible.
14. (1) If no B is A, no A is B. For if some A (say C) is B, it
will not be true that no B is A ; for C is a B.
17. (2) If all B is A, some A is B. For if no A is B, no B is A;
but ex hypothesi all B is A.
20. (3) IfsomeBisA,someA isB. ForifnoAisB,noBisA.
22. (4) If some B is not A, it does not follow that some A is
not B. Not every animal is a man, but every man is an animal.1. 1. 24b26-2. 25"34
293
25 3 3. Ka.8' EKa.aT1lv 1rpoaP1law, in respect of each of these
phrases added to the terms, i.e. imripXH, Eg avriYK7)S" imapXH,
Ev8iX£Tut imaPXHv. 7rp6u6wt, is used similarly in De Int. 2Ib27, 30.
6. c:i.vnaTpecl>Ew. Six usages of this word may be distinguished
in the Analytics. (I) It is used, as here, of the conversion or
convertibility of premisses. (2) It is used in the closely associated
sense of the conversion or convertibility of terms. (3) It is used
of the substitution of one term for another, without any suggestion
of convertibility. (4) It is used of the inference (pronounced to
be valid) from a proposition of the form' B admits of (€V8€X£TUt)
being A' to one of the form 'B admits of not being A', or vice
versa. (5) It is used of the substitution of the opposite of a
proposition for the proposition, without (of course) any suggestion
that this is a valid inference. (6) By combining the meaning 'change
of direction' (as in (I) and (2)) with the meaning 'passage from
a proposition to its opposite', we find the word used of an argu-
ment in which from one premiss of a syllogism and the opposite of
the conclusion the opposite of the other premiss is proved.
Typical examples of these usages are given in the Index.
14-17. npWTOV ••• EaTw. The proof that a universal negative
can be simply converted is by €K6wts", i.e. by supposing an imagin-
ary instance, in this case a species of A of which B is predicable.
'If no B is A, no A is B. For if there is an A, say C, which is B,
it will not be true that no B is A (for C is both a B and an A);
but ex hypothesi no B is an A.'
15-34' Et OOV .•. U1ra.PXo~. In this and in many other passages
the manuscripts are divided between such forms as TcjJ A and
TWV A before or after nv[, o138£v[, or fL7)8€VL tmapXH. The sense
affords no reason why A. should have written sometimes TcjJ
and sometimes TWV; we should expect one or other to appear
consistently. The following points may be noted: (I) in still
more passages the early manuscripts agree in reading TcjJ. (2) AI.
has TcjJ almost consistently (e.g. in 31. 2, 3, 7, 21, 23, 24, 26; 32.
12 (bis), 13, 19, 24 (bis), 28; 33. 20; 34. 9, Il, 18, 19 (bis), 26 (bis),
27,28,29,31 ; 35. I, 16,25,26,27; 36. 4, 6; 37· 10 (bis), 13)· (3) The
reading TcjJ is supported by such parallels as fL7)80'6, TaU B
(25 b40, 26b9, 2736, 21, b6 (bis), 28333, 60 31, or as fL7)8€vt T<.i' €UXriTIp
(2633,S). (4) TcjJ is more in accord with A.'s way of thinking of the
terms of the syllogism; the subject he contemplates is A, the
class, not the individual A's. I have therefore read TcjJ wherever
there is any respectable ancient authority for doing so.294
COMMENTARY
CHAPTER 3
Conversion of modal propositions
25"27. So too with apodeictic premisses; the universal negative
is convertible into a universal, the affirmative (universal or parti-
cular) into a particular. For (1) if of necessity A belongs to no B,
of necessity B belongs to no A ; for if it could belong to some A,
A would belong to some B. If of necessity A belongs (2) to all or
(3) to some B, of necessity B belongs to some A; for if this were
not necessary, A would not of necessity belong to any B. (4) The
particular negative cannot be converted, for the reason given
above.
37. What is necessary, what is not necessary, and what is
capable of being may all be said to be possible. In all these cases
affirmative statements are convertible just as the corresponding
assertoric statements are. For if A may belong to all or to some
B, B may belong to some A; for if not, A could not belong to
any B.
b 3 . Among statements of negative possibility we must distin-
guish. When a non-conjunction of an attribute with a subject is
said to be possible (1) because it of necessity is the case or (2)
because it is not of necessity not the case (e.g. (1) 'it is possible
for a man not to be a horse' or (2) 'it is possible for white to belong
to no garment'), the statement is convertible, like the correspond-
ing assertoric proposition; for if it is possible that no man should
be a horse, it is possible that no horse should be a man; if it is
possible that no garment should be white, it is possible that
nothing white should be a garment; for if 'garment' were neces-
sarily predicable of something white, 'white' would be necessarily
predicable of some garment. The particular negative is incon-
vertible, like an assertoric 0 proposition.
14- But (3) when something is said to be possible because it
usually is the case and that is the nature of the subject, negative
statements are not similarly convertible. This will be shown later.
19. The statement 'it is contingent for A to belong to no B' or
'for A not to belong to some B' is affirmative in form ('is contin-
gent' answering to 'is', which always makes an affirmation, even
in a statement of the type' A is not-B'), and is convertible on the
same terms as other affirmatives.
25"29. EKClTlPCl, i.e. both the universal and the particular
affirmative proposition.
29-34- Ei P.EV ya.p ___ l,.rrcipxo~_ Becker (p. 90) treats this295
section as spurious on the ground that in "29-32 (1) 'Necessarily
no B is A' is said to entail (2) 'Necessarily no A is B' because (3)
'Some A may be B' would entail (4) 'Some B may be A', while
in "4o--h3 (3) is said to entail (4) because (1) entails (2); and that
there is a similar circulus in probando in "32-4 when combined
with hI o-- I3 . The charge of circulus must be admitted, but the
reasoning is so natural that the contention that A. could not have
used it is not convincing.
36. 1fPOTEpOV ~~a.~€V, cf. "10-14.
37-hI9. 'E1ft SE TWV €VSEXO~€vWV ••• )'EYW~EV. The difficulties
of this very difficult passage are largely due to the fact that A.,
in order to complete his discussion of conversion, discusses the
conversion of problematic propositions without stating clearly
a distinction between two senses of EIISEXECT8aL which he states
clearly enough in later passages. He has pointed out in ch. 2
that, of assertoric propositions, A propositions are convertible
per accidens, E and I propositions simply, and 0 propositions
not at all; and in 25"27-36 that the same is true of apodeictic
propositions. He now turns to consider the convertibility of
problematic propositions, i.e. whether a proposition of the form
EIISEXETaL TTallTt (or nlli) TifJ B TO A ImaPXEtIl (or p.~ &rapXEtIl) entails
one of the form EIISEXETat TTallTt (or nil':) TifJ A TO B &rapXEtIl (or
p.~ &rapXEw). This depends, he says, on the sense in which
EIlSEXETat is used. At first sight it looks as if he distinguished three
senses, TO allaYKatOll, TO p.~ allaYKaWII, TO SUllaTolI. But these are
plainly not three senses of EIISEX0P.flIOII, which could not be said
ever to mean either 'necessary' or 'not necessary'. He can only
mean that there are three kinds of case to which EIISEXOP.EVOII can
be applied. When he says TO allaYKatOIl EllSEXECT8at Myop.EII, he
clearly means that that which is necessary may a fortiori be said
to be possible. The reference of TO p.~ allaYKatOIl is less clear.
AI. and P. suppose it to refer to the existent, which can similarly
be said a fortiori to be possible. But that interpretatioIl does not
square with the example given in b6-7, EIlSEXECTBat TO AEVKOII
P.."SEllt tp.aTtl.p ImaPXELII. It is not a fact that no garment is white;
there is only a possibility that none should be so. What the
example illustrates is that which, without being necessary, is
possible in the sense of being not impossible. KaL yap TO allaYKatOIl
Kat TO p.~ allaYKatOIl EIISEXECT8at Myop.EII must be a brachylogical
way of saying 'Not only can we say of what is necessary that it
is possible, but we can (in the same sense, viz. that they are not
impossible) say this of things that are not necessary'.
These two applications 0' .'SEXECT8at are what is illustrated in29 6
COMMENTARY
b S- 13 . We say, 'For all men, not being horses is possible', because
necessarily no man is a horse; and we say 'For all garments, not
being white is possible', because no garment is necessarily white.
In b4- S the evidence is pretty equally divided between Tcf £g
dvaYK7JS" InraPXfLv 17 Tcf /L~ £g dvayK7}S" InraPXfLv and Tcf £g dvaYK7JS"
/L~ lnraPXfLV 17 Tcf /L~ fg dvayK7}S" InraPXfLv. The former reading
brings the text into line with a38; the latter brings it into line
with b 7 -8. But neither reading gives a good sense. While TO /L~
dvaYKaiov in a38 may serve as a brachylogical way of referring
to one kind of case in which lvUXfTat may be used, Tcf /L~ fg dvaYK7JS"
InraPXfLv cannot serve as a reason for using it in that case.
Becker's insertion of /L~ (p. 87), in which a late hand in B has
anticipated him, alone gives the right sense. In b 4- S A. says that
some things are said to be possible because they are necessary,
others because they are not necessarily not the case; in b S -8 he
illustrates this by saying that it is said to be possible that no
man should be a horse because necessarily no man is so, and that
it is said to be possible that no garment should be white because
it is not necessary that any should. The variation of reading in
b 4 and the omission in b S are amply accounted for by the fact that
these two applications of £v8£XfTat are in b S -8 illustrated only by
examples of the possibility of not being something-these alone
being relevant to the point he is making about convertibility. CL
a similar corruption in 37 a 3S--6.
TO avaYlCa'lov and TO fl.~ civaYlCa'lov (a38) refer to two applica-
tions of one sense of lvUXfTat, that in which it means 'is possible',
i.e. 'is not impossible'; to what does TO 8vvaT6v refer? For this
we turn to A.'s main discussion of TO £V8fX6/LfVOV. In 32"18
he defines it as 00 /L~ OVTOS" dvaYKa{ov, uBlvTOS" 8' lnrapXfLV, ov8£v
fcrrat 8ul ToiiT' d8VvaTov. Since that, and only that, which is im-
possible has impossible consequences, this amounts to defining TO
£V8fX6/L£I'OV as that which is neither impossible nor necessary. (He
adds that in another sense (as we have already seen) the necessary
is said to be £V8fX6/LfVOV.) It is to this that TO 8vvaT6v must point,
and that is quite in accord with the doctrine of 8vva/LtS" /Lia
fvavT{wv, in which a 8wa/Lts" is thought of as a possibility of opposite
realizations, neither impossible and neither necessary. When A.
uses £V8fX6/LfVOV in this sense I translate it by 'contingent' ; when
he uses it in the other, by 'possible'.
What A. maintains in the present passage is the following
proposi tions :
(r) 'For all B, being A is possible' entails 'For some A, being
B is possible'.297
(2) 'For all B, being A is contingent' entails 'For some A,
being B is contingent'.
(3) 'For some B, being A is possible' entails 'For some A,
being B is possible'.
(4) 'For some B, being A is contingent' entails 'For some A,
being B is contingent'.
(s) 'For all B, not being A is possible' entails 'For all A, not
being B is possible'.
(6) 'For all B, not being A is contingent' entails 'For some A,
not being B is contingent' (OUK civ·nuTpEr/>Et in bI? means
'is not simply convertible').
(7) 'For some B, not being A is possible' is inconvertible.
(8) 'For some B, not being A is contingent' entails 'For some
A, not being B is contingent'.
A. argues for propositions (1)-(4) in a4o-- b3. 'If for all or some B
being A is possible or contingent, for some A being B is (respec-
tively) possible or contingent; for if it were so for no A, neither
would A be so for any B.' The argument is sound when £vSiXera.t
means 'is possible', but not when it means 'is contingent'. For
then what A. is saying is that if for all (or some) B being A is
neither impossible nor necessary, for some A being B is neither
impossible nor necessary, since if for all A being B were impossible
or necessary, for all B being A would be impossible or necessary.
Now if for all A being B is impossible, for all B being A is im-
possible; but if for all A being B is necessary, it only follows that
for some B being A is necessary. Thus the conclusion of the
reductio should run 'Either for all B being A would be impossible
or for some B it would be necessary'. The error is, however, not
important, since this proposition would still contradict the original
assumption that for all B being A is neither impossible nor
necessary. b2 - 3 d ... 7TPOTEPOV need not be excised (as it is by
Becker, p. 90), since the mistake is <I. natural and venial one.
For propositions (s) and (7) A. argues correctly in b3- 14 . To
propositions (6) and (8) he turns in bI4 - 19 . In 32b4-I.3 (cf. De Int.
19"18-22) A. distinguishes two cases of contingency-one in which
the subject has a natural tendency to have a certain attribute and
has it more often than not, and one in which its possession of the
attribute is a matter of pure chance. It is by an oversight that
in 2SbI4-IS A. paraphrases TO SwaTov of '39 by a reference to the
first alone of these two cases. The essential difference he has in
mind turns not at all on the difference between the two cases, but
on the difference between the sense in which both alike may be
said £vSixEa8at (viz. that they are neither impossible nor necessary)COMMENTARY
and the other sense of Ev8Exw8at, in which it means simply 'not
to be impossible'. It is on this alone that (as we shall see) A.'s
point about convertibility (his whole point in the present passage)
turns. The oversight may to some extent be excused by the fact
that A. thinks contingency of the second kind (where neither
realization is taken to be more probable than the other) no
proper object of science (32bI8-22).
Proposition (6) has sometimes been treated as a curious error
on A.'s part, and Maier, for instance (2 a. 36 n.), has an elaborate
argument in which he tries to account psychologically for the
supposed error. But really there is no error. For the reason for
the statement A. refers us (2SbI8-19) to a later passage, viz.
36b3S-37a3I. But in order to understand that passage we must
first turn to an intervening passage, 32a29-bI. A. there points
out, obviously rightly, that where Ev8Exerm is used in the strict
sense, propositions stating that something £v8Exerat are capable
of a special kind of conversion, which I venture to call comple-
mentary conversion.
'For all B, being A is contingent' entails 'For all B, not being
A is contingent' and 'For some B, not being A is contingent'.
'For all B, not being A is contingent' entails 'For all B, being
A is contingent' and 'For some B, being A is contingent'.
'For some B, being A is contingent' entails 'For some B, not
being A is contingent'.
'For some B, not being A is contingent' entails 'For some B,
being A is contingent'.
With this in mind, let us turn to 36b3S-37a31. A. there gives
three arguments to show that 'For all B, not being A is contin-
gent' does not entail 'For all A, not being B is contingent'. His
first argument (36b37-37a3) is enough to prove the point. The
argument is: (i) 'For all H, being A is contingent' entails (as we
have seen) (ii) 'For all B, not being A is contingent'. (iii) 'For
all A, not being B is contingent' entails (iv) 'For all A, being B
is contingent'. Therefore if (ii) entailed (iii) , (i) would entail
(iv), which it plainly does not. Therefore (ii) does not entail (iii).
Two things may be added: (I) 'For all B, not being A is con-
tingent' does entail 'For some A, not being B is contingent';
(2) as A. says in 2S b q-I8, 'For some B, not being A is contingent'
does entail 'For some A, not being B is contingent'. Both of these
entailments escape the objection which A. shows to be fatal to
any entailment. of 'For all A, not being B is contingent' by 'For
all B, not being A is contingent'.
bZ -3. SESELKTal ya.p ••• 1TpOTEPOV, cL "29-32.299
IZ-13. ToilTo ••• 1TPOTEPOV, cf. "32-4.
13. 0fl-0LIIlS oE .•• a.1TOCPa.TLKfjS. i.e. 'For some B, not being A is
possible' is inconvertible, as 'Some B is not A' and 'Some B is
necessarily not A' are.
14. hiS E1Tt TO 1ToM. ABd' have ws (7Tt 7TOAV, and this form
occurs in some or all of the MSS. in a few other passages (in E
in Phys. 196bII, 13,20, in all MSS. in Probl. 902a9). But the Greek
commentators read ws (7Tt T6 7TOAV pretty consistently, and the
shorter form is probably a clerical error.
15. Ka.O' QV TP01TOV ••• EVOEX0fl-EVOV, 'which is the strict sense
we assign to "possible" '.
19-z4. VUV oE • • • €1T0tJ.EvIllV. Though A. has distinguished
judgements of the forms 'For B, being A is contingent', 'For B,
not being A is contingent' as affirmative and negative (a39, b3),
he now points out that in form they are both affirmative. In both
cases something is said to be contingent, just as, both in 'B is A'
and in 'B is not-A', something is said to be something else.
Maier (2 a. 324 n. I) thinks that this section, which in its final
sentence refers forward to ch. 46, is probably, with that chapter,
a late addition, by A. himself. But cf. my introductory n. to
that chapter. Becker's contention (p. 91) that this section is a
late addition by some writer familiar with De Int. 12 seems to me
unconvincing; I find nothing here that A. might not well have
written.
z4. oElxO~a'ETO-I oE ••• €1T0fl-EVIIlV. The point is discussed at
length in ch. 46, where A. points out the difference between' A is
not equal' and' A is not-equal', viz. that To/ {I.~ inrOK£L'Tat n, To/
OV'TL {I.~ ialfJ, Kat TO&r' ;an T6 ci:vLaov, To/ S' ouSiv (Sl b26-7). I.e.,
'A is not-equal' is not a negative proposition, merely contradicting
'A is equal'; it is an affirmative proposition asserting that A
possesses the attribute which is the contrary of 'equal'.
Z5. KO-TO. OE To.S o.VTL(YTpoCPo.s ••• a~~O-ls. We have to ask
whether the present statement refers to (a) the first two applica-
tions of (VSiXE'TaL or (b) to the third, and what Tais ci:AAaLS means.
If the statement refers to (a), 'TaLS ci:AAaLS means negative assertoric
and apodeictic propositions, and A. is saying that, in spite of
their affirmative form (b I9 - 2S ), negative problematic propositions
of type (a) are, like negative assertoric and apodeictic proposi-
tions, convHtible if universal and inconvertible if particular (as
he has said in b3- 14). If it refers to (b), 'Tais ci:AAaLS means affirma-
tive problematic propositions of type (b), and A. is saying that
the corresponding negative propositions, like these, are incon-
vertible (i.e. not simply convertible) if universal, and convertibleCOMMENTARY
3 0 0
if particular. Maier (2 a. 27 and n.) adopts the first view, AI., P.,
and Waitz the second. The question is, I think, settled in favour
of the second view by the fact that the natural noun to be sup-
plied with 'Tat, rua,S' is Ka'Taq,au£a'v (cf. b 22 ).
CHAPTER 4
A ssertoric syllogisms in the first figttre
Z5bZ6. Let us now state the conditions under which syllogism
is effected. Syllogism should be discussed before demonstration,
because it is the genus to which demonstration belongs.
3z. When three terms are so related that the third is included
in the middle term and the middle term included in or excluded
from the first, the extremes can be connected by a perfect syllo-
gism.
37.
(A) Both premisses tmiversal
AAA (Barbara) valid.
40. EAE (Celarent) valid.
z6'z. AE proves nothing; this shown by contrasted instances.
9. EE proves nothing; this shown by contrasted instances.
13. We have now seen the necessary and sufficient conditions
for a syllogism in this figure with both premisses universal.
(B) One premiss particular
If one premiss is particular, there is a syllogism when and only
when the major is universal and the minor affirmative.
z3. (a) Major premiss universal, minor particular affirmative.
All (Darii) valid.
z5. EIO (Ferio) valid.
30. (b) Major premiss particular, minor universal. lA and OA
prove nothing; this shown by contrasted instances.
36. lE and OE prove nothing; this shown by contrasted in-
stances.
39. (c) Major premiss universal, minor particular negative.
AO proves nothing; this shown by contrasted instances.
b lO • EO proves nothing; this shown by contrasted instances.
14. That AO and EO prove nothing can also be seen from the
facts that the minor premiss Some C is not B is true even if Ko C
is B is true, and that AE and EE have already been seen to
prove nothing.I. 4. 2S b 26-36
3 0r
(C) Both premisses particular
:n.
n. 00, 10, 01 prove nothing; this shown by contrasted in-
stances.
26. Thus (I) to give a particular conclusion in this figure. the
terms must be related as described; (2) all syllogisms in this figure
are perfect, since the conclusion follows directly from the pre-
misses; (3) all problems can be dealt with in this figure, since it
can prove an A, an E. an I, or an conclusion.
°
2Sb26. dLWp~aI'EYWY SE TOIJTWY AEywIUY.
Here. and in 32"17.
the evidence is divided between Mywp.f.v and Myop.f.v. but
the sense demands Mywp.f.v. There are many passages in A. in
which the MSS. give only Uyop.f.V (in similar contexts). but
Bonitz rightly pronounces that Uywp.f.V should always be read
(Index, 424bS8-42SaIO).
27-8. uanpoy SE •.• ci1l"OSf.L~f.WS, in the Posterior Analytics.
28-31. 1I"ponpoy SE .•• ci1l"oSf.~~~s. The premisses of demon-
stration, in addition to justifying the conclusion, must be d),7]6ij,
b4 • 24,
'1TpWTa KO.' ap.f.aa, YVWpLp.WTf.pa KO.' '1Tpchf.pa
'1T£pa.ap.a-ros (An. Post. 7IbI9-72a7).
KO.'
a'TLa -rov uvp.-
32-4. wan TOY iaxaTOY •.. 1''; f.tya~, i.e. so that the minor
term is contained in the middle term as in a whole (i.e. as species
in genus). and the middle term is (se. universally) included in or
excluded from the major as in or from a whole.
36. 0 Kal TU 9Eaf.~ YLYna~ I'EaOY points to the position of the
middle term in a diagram. B. Einarson in A.].P. lvii (1936).
166-9 gives reasons for thinking that, on the model of the dia-
grams used by the Greeks to illustrate the theory of proportion,
A. illustrated the three figures by the following diagrams:
First figure
Second figure
major
A
middle
major
B (or N)
minor
r
major
A (or IT)
A (or M)
B
Third figure
middle
mmor
B (or P)
mmor
r
(or B)
middle
r
(or 1:)
where the length of the lines answers to the generality of the
terms. The principle on which these lines of varying length were
assigned to the three terms is this: In the primary kind of pro-
position, the universal affirmative, the predicate must be at least
as general as the subject and is usually more general; and
negative and particular propositions are by analogy treated asCOMMENTARY
3 02
if this were equally true of them. Thus any term which in any
of the three propositions appears as predicate is treated as being
more general than the term of which it is predicated. The para-
digms of the three figures being (first figure) B is A, C is B,
Therefore C is A; (second figure) B is A, C is A, Therefore C is B;
(third figure) C is A, C is B, Therefore B is A, the comparative
length of the lines to be assigned to the terms becomes obvious.
Alternatively it might be thought that the diagrams took the
form:
Secrmd figure
First figure
B
r
major
major
major
A
B
Third figure
I middle
A --------
\
;~7
r
middle
mmor
B
This would serve better to explain the use of uxfiIJ-a, as meaning
the distinctive shape of each of the three modes of proof. But
it is negated by the fact that A. describes the middle term as
coming first in the second figure and last in the third figure
(26b39, 28315).
39-40. 1TpOTEPOV ••• AEyO .... EV, 24b28-30.
26a~. Et BE ••• AL905. It is noticeable that in this and follow-
ing chapters, where A. states that a particular combination of
premisses yields no conclusion he gives no reason for this, e.g. by
pointing out that an undistributed middle or an illicit process
is involved; but he often points to an empirical fact which shows
that the conclusion follows. E.g. here, instead of giving the
reason why All B is A, No C is B yields no conclusion, he simply
points to one set of values for A, B, C (animal, man, horse) for
which, all B being A and no C being B, all C is in fact A, and to
another set of values (animal, man, stone) for which, all B being
A and no C being B, no C is in fact A. Since in the one case all
C is A, a negative conclusion cannot be valid; and since in the
other case no C is A, an affirmative conclusion cannot be valid.
Therefore there is no valid conclusion (with C as subject and A
as predicate). This type of proof I call proof by contrasted
instances.
In giving such proofs by opaL A. always cites them in the follow-
ing order: first figure, major, middle, minor; second figure, middle,
major, minor; third figure, major, minor, middle.
2. Et BE .•• cl.KOAOu9EL. AI. plainly read aKoAov8EL (SS. 10),3 0 3
and the much commoner lJ7raPXEL is much more likely to have been
substituted for ciKOAovBE' than vice versa.
II-I2. OPOL TOU U'II"clPXELV ••• l1ovcls. I.e., no line is scientific
knowledge, no medical knowledge is a line, and in fact all medical
knowledge is scientific knowledge. On the other hand, no line
is a science, no unit is a line; but in fact no unit is a science.
Therefore premisses of this form cannot prove either a negative
or an affirmative.
17-ZI. EL 5' ••• ci.5UVa.TOV. "18-20 refers to combinations of
a universal major premiss with a particular affirmative minor,
820 .nav SE 7fpO<; 'n) £AaTTov to combinations of a particular major
with a universal minor, "20 ~ Ka~ lliw<; 7fW<; £Xwuw ol opo, to com-
binations of a universal major with a particular mgative minor.
Comparison with 27&26-8 (second figure) and 28bS (third figure)
shows that 26"17 El S' <> fL£v Ka(J6Aov T(OV opwv <> S' ;v fLlpEL 7fpo<; TOV
£npov means 'if the predicate of one premiss is predicated univer-
sally of its subject, and that of the other non-universally of its
subject'. Maier's TO 0' for <> S' (2 a. 76 n. 3) finds no support in
the evidence and is far from being an improvement.
Z4. TO €v ci.pxn }.EX8Ev, cf. 24b28-30.
z7. wpLaTa.L ••• }.EYOI1EV, 24b30.
Z9. TO Br, i.e. the premiss' B belongs to C.
3z. TOU ci.5LOpLaTOU ~ Ka.Ta. I1EpoS OVTOS. The MSS., except f,
have oun ciSop[rrrov ~ KaTlt fLlpo<; OVTO<; (sc. TOU (Tlpov, i.e. the major
premiss). But (I) the ellipse of TOU ETlpov is impossible, and (2)
cio,op[rrrov arid KaTa fLtPO<; are no true alternatives to ci7forPanKou
and KaTarPanKofi. Waitz is no doubt right in reading Tofi, which
derives support from AI. ; for, ignoring ciSwp[rrrov ~ as introducing
an unimportant distinction, he says (61. 20-1) Tofi (so the MSS.;
Wallies wrongly emends to TO) OE KaTa fLtpo<; OVTO<; Et7fEV ciVT~
Tofi 'Tfj<; fLd~ovo<;"
aVrrj
yap Y[VETa, KaTa fLlpo<;.
34-6. OPOL ••• ci.11a.8La.. I.e., some states are good, and some
not good, all prudence is a state; and in fact all prudence is good.
On the other hand, some states are good, and some not good,
all ignorance is a state; and in fact no ignorance is good. Thus
premisses of the form lA or OA do not warrant either a negative
or an affirmative conclusion.
38. OPOL ••• KOpa.~. I.e., some horses are white, and some not
white, no swans are horses; and in fact all swans are white. On
the other hand, some horses are white, and some not white, no
ravens are horses; and in fact no ravens are white. Thus pre-
misses of the form lE or OE do not warrant either a negative
or an affirmative conclusion.3~
COMMENTARY
b3. a.SLOpLO'TOU TE Kal. EV f1EPEL ~TJ.p9EVTOS. These words are a
pointless repetition of the previous line, and should be omitted.
There is no trace of them in Al.'s or in P.'s exposition.
6-10. U'I1'OKEL0'911l0'av ••• O'U~~OYLO'f10S. The fact that, all men
being animals, and some white things not being men, some white
things are animals and some are not, shows that premisses of
the form AO do not warrant a universal conclusion; but it does
not show that a particular conclusion cannot be drawn. Therefore
here A. falls back on a new type of proof. Within the class of
white things that are not men we can find a part A, e.g. swans,
none of whose members are (and a fortiori some of whose members
are not) men, and all are animals; and another part none of
whose members are (and therefore a fortiori some of whose
members are not) men, and none are animals. If the original
premisses (All men are animals, Some white things are not men)
warranted the conclusion Some white things are not animals,
then equally All men are animals, Some swans are not men,
would warrant the conclusion Some swans are not animals; but
all are. And if the original premisses warranted the conclusion
Some white things are animals, then equally All men are animals,
Some snow is not a man, would warrant the conclusion Some
snow is an animal; but no snow is. Therefore the original pre-
misses prove nothing.
10-14. 'I1'a.~LV ••• OUSEVOS. The proof that premisses of the
form EO prove nothing is exactly like the proof in b 3- IO that
premisses of the form AO prove nothing. The fact that, no men
being inanimate, and some white things not being men, some
white things are and others are not inanimate, shows that a
universal conclusion does not follow from EO. And the further
fact that, no men being inanimate, and some swans not being
men, no swans are inanimate, shows that EO does not yield a
particular affirmative conclusion; and the fact that, no men being
inanimate, and some snow not being a man, all snow is inanimate,
shows that EO does not yield a particular negative conclusion.
14-20. in ••• TOUTIIlV. A. gives here a second proof that AO
yields no conclusion. Some C is not B, both when no C is Band
when some is and some is not. But we have already proved
(32-9) that All B is A, No C is B, proves nothing. It follows that
All B is A, Some C is not B, proves nothing. This is the argument
£K TOV dSWPL<rTOV (from the ambiguity of a particular proposition)
which is used in 27b2D-3, 27-8, 28b28-3I, 2936, 3SblI.
23. 11 TO f1EV ••• SLIIlPLO'f1EVOV, 'or one indefinite and the other
a definite particular statement'.24-5' 8pOL SE ... X£905. Some white things are animals, and
some not, some horses are white, and some not; and all horses
are animals. On the other hand, Some white things are animals,
and some not, some stones are white, and some not; but in fact
no stones are animals. Thus premisses of the form 11, 01, 10,
or 00 cannot prove either a negative or an affirmative.
2c.-s. 41QVEpoV ••• yLVETQL. This sums ,up the argument in
817-b25. To justify a particular conclusion, the premisses must
be of the form AI (823-5) or El ("25-30). A. ignores the fact that
AA, EA, which warrant universal conclusions, a fortiori warrant
the corresponding particulars.
CHAPTER 5
Assertoric syUogisms in the second figure
26b34. When the same term belongs to the whole of one class
and to no member of another, or to all of each, or to none of either,
I call this the second figure; the common predicate the middle
term, that which is next to the middle the major, that which is
farther from the middle the minor. The middle is placed outside
the extremes, and first in position. There is no perfect syllogism
in this figure, but a syllogism is possible whether or not the
premisses are universal.
(A) Both premisses universal
There is a syllogism when and only when one premiss is
affirmative, one negative. (a) Premisses differing in quality.
EAE (Cesare) valid; this shown by conversion to first figure.
9. AEE (Camestres) valid; this shown by conversion.
14. The validity of EAE and AEE can also be shown by
reductio ad impossibile. These moods are valid but not perfect,
since new premisses have to be imported.
18. (b) Premisses alike in quality. AA proves nothing; this
shown by contrasted instances.
20. EE proves nothing; this shown by contrasted instances.
(B) One premiss particular
26.
(a) Premisses differing in quality. (a) Major universal. EIO
(Festino) valid; this shown by conversion.
36. AOO (Baroco) valid; this shown by redHctio ad impossibile.
b 4 . (fJ) Minor universal. OA proves nothing; this shown by
contrasted instances.
6. lE proves nothing; this shown by contrasted instances.
x306
COMMENTARY
10. (b) Premisses alike in quality. (a) Major universal. EO
(No N is M, Some E is not M) proves nothing. If both some Eis
not M and some is, we cannot show by contrasted instances that
EO proves nothing, since all E will never be N. We must there-
fore fal! back upon the indefiniteness of the minor premiss; since
o is true even when E is true, and EE proved nothing, EO proves
nothing.
23. AI proves nothing; this must be shown to follow from the
indefiniteness of the minor premiss.
28. (f3) Minor universal. OE proves nothing; this shown by
contrasted instances.
32. lA proves nothing; this shown by contrasted instances.
34. Thus premisses alike in quality and differing in quantity
prove nothing.
36.
(B) Both premisses particular
n, 00, 10, 01 prove nothing; this shown by contrasted in-
stances.
28 a l. It is now clear (I) what are the conditions of a valid
syllogism in this figure; (2) that all syllogisms in this figure are
imperfect (needing additional assumptions that either are im-
plicit in the premisses or-in reductio ad impossibile-are stated
as hypotheses; (3) that no affirmative conclusion can be drawn in
this figure.
26b34-6. "OTClV SE • . . SEUTEpOV. This is not meant to be a
definition of the second figure, since it mentions only the case
in which both premisses are universal. But it indicates the general
characteristic of this figure, that in it the premisses have the
same predicate.
37-8. flELbOV SE ••• ICELflEVOV. It is not at first sight clear why
A. should say that in the second figure the major term is placed
next to the middle term, while in the third figure the minor is
so placed (28aI3-14). AI. criticizes at length (72. 26-75. 9) an
obviously wrong interpretation given by Herminus, but his own
further observations (75. 10-34) throw no real light on the ques-
tion. P. (87. 2-19) has a more plausible explanation, viz. that in
the second figure (PM, SM, SP) the major term is the more akin
to the middle, because while the middle term figures twice as
predicate, the major term figures so once and the minor term not
at all. On the other hand, in the third figure (MP, MS, SP), the
minor term is the more akin to the middle because, while the
middle term occurs twice as subject, the minor .occurs once as
subject and the major term never.This explanation is open to two obj ections. (1) It is far from
obvious, and A. could hardly have expected an ordinary hearer
or reader to see the point in the complete absence of any explana-
tion by himself. (2) 'n) TrPO, T0 fduC{J K€tfL€VOV naturally suggests
not affinity of nature but adjacent position in the formulation
of the argument. The true explanation is to be found in the
diagram used to illustrate the argument-the first of the two
diagrams in 2Sb36 n. It may be added that in A.'s ordinary
formulation of a second-figure argument (e.g. KaT'Y}yop€tu8w TO
M TOU fL~v N fL'Y}8€v6S', TOU 8~ E TravT6S', 27'S--D) the major term N
is named next after the middle term M, while in the ordinary
formulation of the third figure (e.g. OTav Kat TO IT Kat TO P Trav-rt
T0 E umi.pXTI, 28'18) the minor term P is named next before the
middle term E.
39. TL9£Ta.~ ••• 9ia£L. In 28'14-1S A. says that in the third
figure Tt8€TaL TO fLluov €gw fLEV TWV O.KPWV, €UXaTOV 8~ Tfj 8IaEL.
When he says of the middle term in the second figure that it is
placed outside the extremes, we might suppose that it was
because it is the predicate of both premisses (the subject being
naturally thought of as included in the predicate, because it is
so in an affirmative proposition). But that would not account
for his saying that in the third figure, where the middle term is
subject of both premisses, it is outside the extremes. His meaning
is simply that in his diagram the middle term comes above both
extremes in the second figure, and below both in the third, and
that in his ordinary formulation the middle term does not come
between the extremes in either figure; it is named before them
both in the second figure, after them both in the third. 'M belongs
to no N, and to all E' (second figure). 'Both IT and P belong to
all E' (third figure).
27"1. TiXELo!i ••• aX"Ila.TL. A. holds that the conclusion, in
the second and third figures, cannot be seen directly to follow
from the premisses, as it can in the first figure. Accordingly he
proves the validity of the valid moods in these figures by showing
that it follows from the validity of the valid moods in the first
figure. Sometimes the proof is by conversion, i.e. by inferring
from one of the premisses the truth of its converse, and thus
getting a first-figure syllogism which proves either the same
conclusion or one from which the original conclusion can be got
by conversion. Thus in "6---9 he shows the validity of Cesare as
follows: If No N is M and All E is M, No Eis N; for from No N
is M we can infer that NoM is N, and then we get the first-figu re
syllogism No M is N, All E is M, Therefore No Eis N.COMMENTARY
3 08
Sometimes the proof is by reductio ad impossibile, i.e. by showing
that if the conclusion were denied, by combining its opposite
with one of the premisses we should get a conclusion that con-
tradicts the other premiss. Thus in 3I4-I5 he indicates that
Cesare can be shown as follows to be valid (and Camestres
similarly) : If No N is M and All 3 is M, it follows that No 3 is N.
For suppose that some 3 is N. Then by the first figure we can
show that if no N is M and some 3 is N, it would follow that some
3 is not M. But ex hypothesi all 3 is M.
2-3. KaL Ka8oAou ••• OYTWV, i.e. both when the predicates of
both premisses are predicated universally of their subjects and
when they are not both so predicated. opwv is frequently used
thus brachylogically to refer to premisses.
8. ToilTo ••• 1Tponpov, 25b40-2632.
10. TO : T~ N. Proper punctuation makes it unnecessary to
adopt Waitz's reading, To/ 3 TO N.
14. waT' €aTaL ••• aUAAoYLalloS, i.e. so that Camestres re-
duces to the same argument as Cesare did in as-9, i.e. to Celarent.
14-15. (aTL SE ••• ciyovTas, cf. 3I n.
19-20. 0pOL ••• IlEaov o~aLa. I.e., all animals are substances, all
men are substances, and all men are animals. On the other hand,
all animals are substances, all numbers are substances, but no
numbers are animals. Thus in this figure AA proves nothing.
As AI. observes (81. 24-8), A. must not be supposed to hold
seriously that numbers are substances; he often takes his in-
stances rather carelessly, and here he simply uses for the sake of
example a Pythagorean tenet.
21-3. OPOL TOU U1TC1PXELV ••• AL8os. I.e., no animals are lines,
no men are lines, and in fact all men are animals. On the other
hand, no animals are lines, no stones are lines; but in fact no
stones are animals. Therefore in this figure EE proves nothing.
24.
WS EV a.pxn EL1ToIlEv, "3-5.
36. YLvETaL yap •.• aXTJIlClToS, i.e. in Ferio (26325-30).
bl _2 • KaL Et ••• Il~ 1TClVTL. This is not a new case, but an alterna-
tive formulation to cl To/ /LEv N 1TC1VTL TO M, To/ oE 3 nVL /L~ iJ1TapXH
("37).
5-6. OPOL ••• KOpC1~. I.e., some substances are not animals,
all ravens are animals; but in fact all ravens are substances.
On the other hand, some white things are not animals, all ravens
are animals; and in fact no ravens are white. Therefore in this
figure OA proves nothing.
6-8. OPOL TOU U1TC1PXELV ••• E1TLaTTJIlT)' I.e., some substances
are animals, no units are animals; but in fact all units are sub-stances. On the other hand, some substances are animals, no
sciences are animals; and in fact no sciences are substances.
Therefore in this figure lE proves nothing.
For.the treatment of units as substances cf. "19-20 n.
16-z3. OPOL ••• E<7To.L. That EO in the second figure proves
nothing cannot be shown in the way A. has adopted in other
cases, viz. by contrasted instances (cf. 26'2-9 n). He points
(b I 6) to an instance in which, no N being M, and some E not being
M, no E is N; no snow is black, some animals are not black, and
no animal is snow. But there cannot be a case in which all Eis
N, so long as the minor premiss is taken to mean that some Eis
not M and some is; for if no N is M and all Eis N, it would follow
that no E is M, whereas the original minor premiss is taken to
mean that some E is and some is not M. He therefore falls back
on pointing out that Some E is not M is true even when no E
is M, and on reminding us that No N is M, No E is M proves
nothing (as was shown in 320-3). The argument £K TOV ci.o,optaToV
(from the ambiguity of the particular proposition) has been
already used in 26bI4-20.
z6-8. OPOL ••• S€LKT€OV. I.e., all swans are white, some stones
are white; but in fact no stones are swans. Therefore AI in
the second figure does not warrant an affirmative conclusion.
That it does not warrant a negative conclusion is shown (as in
the previous case, b2 o- 3) by pointing out that Some E is M is true
even when all E is M, and that All N is M, All E is M proves
nothing.
3I-Z. OPOL TOU U1Ta.PX€LV ••• ).€ul(oY-)'L9o5-I(Opo.~. I.e., some
animals are not white, no ravens are white; and in fact all ravens
are animals. On the other hand, some stones are not white, no
ravens are white; but no ravens are stones. Thus OE in the
second figure proves nothing.
3z-4. €L S€ .•• KUKV05. I.e., some animals are white, all snow
is white; but in fact no snow is an animal. On the other hand,
some animals are white, all swans are white; and in fact all
swans are animals. Thus lA in the second figure proves nothing.
36-8. 0.),),' ouS' ••• ciSLOPLaTw5, 'nor does anything follow if a
middle term belongs to part of each of two extremes (Il), or
does not belong to part of each of them (00), or belongs to part
of one and does not belong to part of the other (10, 01), or does
not belong to either as a whole (00), or belongs without deter-
mination of quantity'. 7i I-'TfO£T£PCP 7Ta"Tt is not a new case, but an
alternative formulation to T'''' £KaT£PCP I-'~ inrapXH; cf. b l - 2 n. ;
so AI. 92. 33-94. 4· The awkwardness would be removed byCOMMENTARY
3 IO
omitting ~ jJ-~ iJ1TaPXH in b 37 with B', but this seems more likely
to be a mistake due to homoioteleuton.
Waitz reads in b 37 (with one late MS.) ~ jJ-7]8' ;-dpCfJ 1TavTl, which
he interprets as meaning .ry Tip ;TEPCfJ jJ-~ 1TaVTl, i.e. as expressing
alternatively what A. has already expressed by Tip SE jJ-~ (i.e. Tip S,£
TtV, jJ-~). the reference being to the combination 10 or 01. But
~ jJ-7]S' ;TEPCfJ 1TaVTl could not mean this.
38. 11 cUhOP L17T1al S, i.e. two premisses of indetenninate quantity
are in respect of invalidity like two particular premisses.
3S-9. OpOL SE ... li.lj1uxov. I.e., some animals are white and
some not, some men are white and some not, and in fact all men
are animals. On the other hand, some animals are white and
some not, some lifeless things are white and some not; but in
fact no lifeless thing is an animal. Thus in this figure Il, 01, 10,
or 00 proves nothing.
28 a 2. ws E~EX9TJ. in 27 3 3-5, 26-32.
6. et 11 EVU'II"o.PXEL ••• " TL9EVTaL WS U'II"09E17ELS. The plural Tl6EVTaL
is used carelessly, by attraction to the number of iJ1To6Et:1ELf.
CHAPTER 6
Assertoric syllogisms in the third figure
28 IO. If two predicates belong respectively to all and to none
of a given tenn, or both to all of it, or to none of it, I call this the
third figure, the common subject the middle tenn, the predicates
extreme tenns, the term farther from the middle term the major,
that nearer it the minor. The middle tenn is outside the extremes,
and last in position. There is no perfect syllogism in this figure,
but there can be a syllogism, whether or not both premisses are
universal.
3
17.
(A) Both premisses universal
AAI (Darapti) valid; this shown by conversion, reductio ad
impossibile, and ecthesis.
26. EAO (Fclapton) valid; this shown by conversion and by
reductio ad impossibile.
30. AE proves nothing; this shown by contrasted instances.
33. EE proves nothing; this shown by contrasted instances.
36. Thus two affirmative premisses prove an I proposition;
two negative premisses, nothing; a negative major and an affirma-
tive minor, an 0 proposition; an affirmative major and a
negative minor, nothing.1. 5. 27 b 3 8 - 6 . 28"33
3 11
(B) One premiss particular
(a) Two affirmative premisses give a conclusion. IAI (Disamis)
valid; this shown by con version.
11. AIl (Datisi) valid; this shown by conversion, reductio ad
impossibile, and ecthesis.
15. (b) Premisses differing in quality. (a) Affirmative premiss
universal. OAO (Bocardo) valid; this shown by reductio ad
impossibile and by ecthesis.
22. AO (All 5 is P, Some 5 is not R) proves nothing. If some 5
is not R and some is, we cannot find a case in which no R is P; but
we can show the invalidity of any conclusion by taking note of
the indefiniteness of the minor premiss.
JI. (P) Negative premiss universal. EIO (Ferison) valid; this
shown by conversion.
J6. lE proves nothing; this shown by contrasted instances.
J8. (c) Both premisses negative. OE proves nothing; this
shown by contrasted instances.
29"2. EO proves nothing; that tbis is so must be proved from
the indefiniteness of the minor premiss.
6.
(C) Both premisses particular
Il, 00, 10, 01 prove nothing; this shown by contrasted
instances.
11. It is clear then (I) what are the conditions of valid syllo-
gism in this figure; (2) that all syllogisms in this figure are im-
perfect; (3) that this figure gives no universal conclusion.
28"IJ-15.
fiEL~OV •••
OEaEL. For the meaning cf. 26b37-8 n.,
39 n.
2J. T~ EKOEaOa.L, i.e. by exposing to mental view a particular
instance of the class denoted by the middle term. A. uses lK£hcTLS
(I) as a technical term in this sense, (2) of the procedure of
setting out the words in an argument that are to serve as the
terms of a syllogism. Instances of both usages are given in our
Index. B. Einarson in A.].P. lvii (1936), 161-2, gives reasons for
thinking that A.'s usage of the word is adopted from 'the €K(hcTLS
of geometry, where the elements in the enunciation are repre-
sented by actual points, lines, and other corresponding elements
in a figure'.
28. 0 ya.p a.UTOS TP0""OS, i.e. as that in "19-22.
29. TijS PI "..poTa.aEws, i.e. the premiss 'R belongs to 5'.
JO. Ka.Oa."..EP E"..i TWV""POTEPOV, cf. 27"14-15, 38-bl, 28"22-3·
JI-J. OpOL ••• C1VOpI.l'ITOS. I.e., all men are animals, no menCOMME~TARY
3 12
are horses, but all horses are animals. On the other hand, all
men are animals, no men are lifeless, and no lifeless things are
animals. Therefore in this figure AE can prove nothing.
34-6. OPOL TOU U1Ta.PX€LV .•• a.",UXOV. I.e .. no lifeless things
are animals, no lifeless things are horses; and in fact all horses
are animals. On the other hand, no lifeless things are men, no
lifeless things are horses; but in fact no horses are men. Thus in
the third figure EE proves nothing.
bI4- I S. (erTL S' lL1TOSEL~ClL ••• 1TpOTEPOV, i.e. by reductio ad im-
possibile as in the case of Darapti (822-3) and Felapton ("29-30),
or by ecthesis as in the case of Darapti (a22--ti).
IS. 1TpOTEPOV should be read, instead of 7Tpo·dpwv; cf. 830, b28,
3Ib40, 3SbI7, 3682, etc.
19""20. Et ya.p •.. U1Ta.P~EL. The sense requires a comma after
Kat "TO P 7TaV"Tt "Tcf E, since this is part of the protasis.
20-1. SdKVUTClL •.• U1Ta.PXEL. This is the type of proof called
;KO~CTtS"
(a23--ti, bI4 ).
22-31. OTClV S' ••. aU).).0YLO,,1.0,.. That AO in the third figure
proves nothing cannot be shown by the method of contrasted
instances. We can show that it does not prove a negative, by
the example 'all animals are living beings, some animals are not
men; but all men are living beings'. But we cannot find an
example to show that the premisses do not prove an affirmative,
if Some S is not R is taken to imply Some S is R; for if all 5 is P
and some S is R, some R must be P, but we were trying to find
a case in which no R is P. We therefore fall back on the fact
that Some 5 is not R is true even when no 5 is R, and that All
S is P, No 5 is R has been shown in 830--3 to prove nothing.
28. EV TOL" 1TpOTEpOV, 26bI4-20, 27b20-3, 26-8.
36-8. 0pOL TOU U1Ta.PXELV ••• TO a.YPLOV. I.e., some wild things
are animals, no wild things are men; but in fact all men are
animals. On the other hand, some wild things are animals, no
wild things are sciences; and in fact no sciences are animals.
Thus lE in the third figure proves nothing.
39""29'6. 0pOL .•• SELKT€OV. That OE proves nothing is shown
by contrasted instances: some wild things are not animals, no
wild things are sciences; but no sciences are animals; on the
other hand, some wild things are not animals, no wild things are
men, and all men are animals.
That EO does not prove an affirmative conclusion is shown by
the fact that no white things are ravens, some white things are
not snow, but no snow is a raven. \-Ve cannot give an instance
to show that a negative conclusion is impossible (i.e. a case in3 1 3
which, no S being P, and some S not being R, All R is in fact P),
if Some S is not R is taken to imply that some S is R; for if all
R is P, and some S is R. some S must be P; but the case we were
trying to illustrate was that in which no S is P. We therefor.e
fall back on the fact that Some S is not R is true even when no
5 is R. and that if no S is P, and no S is R, nothing follows
(28"33-6). Ct. z6b14-Z0 n.
29'"7-8. 11 ;, Il EV •.. \nrapxn. These words could easily be
spared, since the case they state differs only verbally from what
follows, 6 fLfV nvl 0 Sf fL~ 1TaVTt. But elsewhere also (27a36-b2,
b 36--7) A. gives similar verbal variants, and the omission of the
words in question by B, C, and JI is probably due to homoio-
teleuton.
9-10. OPOL SE ••• twov-ciljlUXOV-XEUKOV. I.e., some white things
are animals and some not, some white things are men and some
not; and in fact all men are animals. On the other hand, some
white things are animals and some not, some white things are
lifeless and some not; but in fact no lifeless things are animals.
Thus n, m, 10, 00 in the third figure prove nothing.
CHAPTER 7
Common properties of the three figures
29"19. In all the figures, when there is no valid syllogism, (I)
if the premisses are alike in quality nothing follows; (2) if they are
unlike in quality, then if the negative premiss is universal, a
conclusion with the major term as subject and the minor as
predicate follows. E.g. if all or some B is A, and no C is B, by
converting the premisses we get the conclusion Some A is not C.
27. If an indefinite proposition be substituted for the parti-
cular proposition the same conclusion follows.
30. All imperfect syllogisms are completed by means of the
first figure, (I) ostensively or (z) by reductio ad impossibile. In
ostensive proof the argument is put into the first figure by con-
version of propositions. In reductio the syllogism got by making
the false supposition is in the first figure. E.g. if all C is A and is
B, some B must be A ; for if no B is A and all C is B, no C is
A ; but ex hypothesi all is.
bI. All syllogisms may be reduced to universal syllogisms in the
first figure. (1) Those in the second figure are completed by
syllogisms in the first figure-the universal ones by conversion of
the negati ve premiss, the particular ones by reductio ad impossibile.3I4
COMMENTARY
6. (2) Particular syllogisms in the first figure are valid by their
own nature, but can also be validated by reductio using the
second figure; e.g. if all B is A, and some C is B, some C is A ; for
if no C is A, and all B is A, no C will be B.
11. So too with a negative syllogism. If no B is A, and some
C is B, some C will not be A; for if all C is A, and no B is A, no
C will be B.
15. Now if all syllogisms in the second figure are reducible to
the universal syllogisms in the first figure, and all particular
syllogisms in the first are reducible to the second, particular
syllogisms in the first will be reducible to universal syllogisms in it.
19. (3) Syllogisms in the third figure, when the premisses are
universal, are directly reducible to universal syllogisms in the first
figure; when the premisses are particular, they are reducible to
particular syllogisms in the first figure, and thus indirectly to
universal syllogisms in that figure.
:z6. We have now described the syllogisms that prove an
affirmative or negative conclusion in each figure, and how those
in different figures are related.
:Z9·1~:Z7 . .t.ii~ov SE ••• au~~oyla""o5.
These generalizations
are correct, but A. has omitted to notice that OA in the second
figure and AO in the third give a conclusion with P as subject.
A.'s recognition of the fact that AE and lE in the first figure
yield the conclusion Some P is not S amounts to recognizing the
validity of Fesapo and Fresison in the fourth figure; but he does
not recognize the fourth as a separate figure. He similarly in
53a9-14 recognizes the validity of the other moods of the fourth
figure-Bramantip, Dimaris, Camenes. For an interesting study
of the development of the theory of the fourth figure from A.'s
hints cf. E. Thouverez in Arch. J. d. Gesch. d. Philos. xv (1902),
49-IIO; cf. also Maier, 2 a. 94-IOO.
:Z7-9. Sii~ov •.• axft .... aaw. In three of the moods which A.
has stated to yield a conclusion with the major term as subject
and the minor as predicate (lE in all three figures) the affirmative
premiss is particular. He here points out that an indefinite
premiss, i.e. one in which neither 'all' nor 'some' is attached to
the subject, will produce the same result as a particular premiss.
31-:Z. 11 ya.p SElKTlKW5 ••• 1rclVTE5. An argument is said to be
SHKTtK6S', ostensive, whcn the conclusion can be seen to follow
either directly from the premisses (in the first figure) or from
propositions that follow directly from the premisses (as when
an argument in the second or third figure is reduced to the first3 1 5
figure by conversion of a premiss). A reductio ad impossibile, on
the other hand, uses a proposition which does not follow from
the original premisses, viz. the opposite of the conclusion to be
proved.
A. says nothing of the proof by lK8HT'c; which he has often used,
because, being an appeal to our intuitive perception of certain
facts (cf., for instance, 28"22-6), not to reasoning, it is formally
less cogent. In any case it was used only as supplementing proof
by conversion, or by reductio ad impossibile, or by both.
b 4 . ot Il~v lCa86~ou ••• ci.vTLO'TpacjlEvTo~. The validity of Cesare
and Camestres has been so established in 27"5--9, 9-14.
5-6. TWV S' EV lliPEL ••• ci.1T"aywyil~. The validity of Baroco
has been so established in 27"36-b3. The validity of Festino was
established differently (27"32-6), viz. by reduction to Ferio; and
that establishment of it would not illustrate A.'s point here,
which is that all syllogisms may be reduced to universal syllogisms
in the first figure. The proof of the validity of Festino which he
has in mind must be the following: 'No P is M, Some 5 is M,
Therefore some 5 is not P. For if all 5 is P, we can have the
syllogism in Celarent (first figure) No P is M, All 5 is P,Therefore
no 5 is M, which contradicts the original minor premiss.'
18. ot lCaTa. IlEpO~1 sc. €v 'TcjJ Trpc!mp.
19-21. ot S' EV T~ TpLTltJ ••• O'U~~OyLO'IlWV. The main proof of
the validity of Darapti (28"17-22) was by reduction to Darii,
which would not illustrate A.'s present point, that all syllogisms
can be validated by universal syllogisms in the first figure. But
in 28 3 22-3 he said that Darapti can also be validated by reductio
ad impossibile, and that is what he has here in mind. All M is P,
All M is 5, Therefore some 5 is P. For if no 5 is P, we have the
syllogism No 5 is P, All M is 5, Therefore No M is p. which
contradicts the original major premiss.
Similarly Felapton was in 28"26--9 validated by reduction to
Ferio, but can be validated by reductio ad impossibile (ib. 29-30)
using a syllogism in Barbara.
21-2. o'Tav S' EV IlEPEL •.. O'xTJlla'TL. Disamis and Datisi were
validated by reduction to Darii (28b7-II, I I -1 4) , Ferison (ib.
33-5) by reduction to Ferio. But Bocardo (ib. 16-20) was validated
by reductio ad impossibile, using a syllogism in Barbara-which
would not illustrate A.'s point, that the non-universal syllogisms
in the third figure are validated by non-universal syllogisms in the
first figure. To illustrate this point he would have needed to
have in mind a different proof of Bocardo, viz. the following:
'Transpose the premisses Some M is not p. All M is5, and convertCOMMENTARY
the major by negation. Then we have All M is 5, Some not-P
is M, Therefore Some not-P is S. Therefore Some 5 is not-Po
Therefore Some 5 is not P.' But conversion by negation is not
a method he has hitherto allowed himself, so that AI. is right in
saying (1l6. 30-5) that A. has made a mistake. His general point,
however, is not affected-that ultimately all the moods in all
the figures are validated by the universal moods of the first
figure; for Bocardo is validated by reductio ad impossibile using
'
a syllogism in Barbara.
26-8. at I-L€V o~v . . . €T€PWV, A. has shown in chs. 4--6 the
position of syllogisms in each figure, with respect to validity or
invalidity, and in ch. 7 the position with regard to reduction of
syllogisms in one figure to syllogisms in another.
CHAPTER 8
Syllogisms with two apodeictic premisses
29b29' It is different for A to belong to B, to belong to it of
necessity, and to be capable of belonging to it. These three facts
will be proved by different syllogisms, proceeding respectively
from necessary facts, actual facts, and possibilities.
36. The premisses of apodeictic syllogisms are the same as
those of assertoric syllogisms except that 'of necessity' will be
added in the fonnulation of them. A negative premiss is conver-
tible on the same conditions, and 'being in a whole' and 'being
true of every instance' will be similarly defined.
3°"3. In other cases the apodeictic conclusion will be proved
by means of conversion, as the assertoric conclusion was; but in
the second and third figures, when the universal premiss is
affinnative and the particular premiss negative, the proof is not
the same; we must set out a part of the subject of the particular
premiss, to which the predicate of that premiss does not belong,
and apply the syllogism to this; if an E conclusion is necessarily
true of this, an 0 conclusion will be true of that subject. Each
of the two syllogisms is validated ill its own figure.
29b3I. Ta 5' ••• oAwc;, 'while others do not belong of necessity,
or belong at all'.
3°"2. TO TE yap O'TEpT]TLKOV WO'o.UTWC; cl.VTLO'T P€4» EL , i.e. is con-
vertible when universal, and not when particular (cf. 25'5-7,
12-13). Affirmative propositions also are convertible under the
same conditions in apodeietic as in assertoric syllogisms; but A.
mentions only negative propositions, because he is going to point1. 7. 29b26 - 8 . 3°"14
3 1 7
out (36b35-37a31) that these when in the strict sense problematic
are not convertible under the same conditions as when they are
assertoric or apodeictic.
2-3. KaL TO ~V 8~'l' ... a.'II"OliWC70IllEV, cf. 24b26-30.
3-9. ~v IlEV o~v ..• a.'II"OliU€lo;. I.e., in all the moods of the
second and third figures except AnOnOn in the second and OnAnOn
in the third a necessary conclusion from necessary premisses is
validated in the same way as an assertoric conclusion from
assertoric premisses, i.e. by reduction to the first figure. But this
method cannot be applied to AnOnOn and OnAnOn. Take AnOnOn.
'All B is necessarily A, Some C is necessarily not A, Therefore
some C is necessarily not B.' The assertoric syllogism in Baroco
was validated by reductio ad impossibile (27a36-b3), by supposing
the contradictory of the conclusion to be true. The contradictory
of Some C is necessarily not B is All C may be B. And this, when
combined with either of the original premisses, produces not a
simple syllogism with both premisses apodeictic, but a mixed
syllogism with one apodeictic and one problematic premiss. But
Aristotle cannot rely on such a syllogism, since he has not yet
examined the conditions of validity in mixed syllogisms.
9-14. ci~~' a.Va.YKT) ••• C7xTJllan. A. therefore falls back on
another method of validation of AnOnOn and OnAnOn. Take
AnOnOn. 'All B is necessarily A, Some C is necessarily not A,
Therefore some C is necessarily not B.' Take some species of C
(say D) which is necessarily not A. Then all B is necessarily A, All
D is necessarily not A, Therefore all D is necessarily not B (by
Camestres). Therefore some C is necessarily not B.
Again take OnAnOn. 'Some C is necessarily not A, All C is
necessarily B, Therefore some B is necessarily not A.' Take
a species of C (say D) which is necessarily not A. Then All D is
necessarily not A, All D is necessarily B, Therefore some B is
necessarily not A (by Felapton). £K8EfL£VOVS <p nv, EKaTEpov fL~
imapXH means 'setting out that part of the subject of the parti-
cular negative premiss, of which the respective predicate in each
of the two cases (AnOnOn and OnAnOn) is not true'.
Waitz has a different interpretation, with which we need not
concern ourselves, since it is plainly mistaken (cf. Maier, 2 a.
106 n.). AI. gives the true interpretation (12I. 15-122. 16). He
adds that this is a different kind of IK8£C1Ls from that used with
regard to assertoric syllogisms. There, he says, TO £KTL8£fL£VOV
was TL TWII alC18?JTWV KO.' fL~ 8£OfL£IIWII 8£~£ws (122. 19), whereas
here there is not an appeal to perception but TO £Kn8£fL£1I01l enters
into a new syllogism which validates the original one. He isCOMMENTARY
3 18
mistaken in describing the former kind of ecthesis as appealing
to a perceptible individual thing; the appeal was always to a
species of the genus in question. But he is right in pointing out
that the former use of ecthesis (e.g. in 28 a 12-r6) was non-syllogistic,
while the new use of it is syllogistic.
II-I2. Et s€ .•• TlVOs. This applies strictly only to the proof
which validates AnOnOn; there we prove that B is necessarily
untrue of all D (KaTa TOU EKTEOEIITO,) and infer that it is necessarily
untrue of some C (KaT' EKdllOV TLIIO,). In the proof which validates
OnAnOn, TO EKTEOEII (D) is middle term and nothing is proved of it.
The explanation is offered by Al., who says (122. 15-16) WOT' d
E'TT~ p-oplov TOU r ~ a(,~" vy,~" Ka~ E'TT' T'II()' TOU r Vy,~, a£,~£>
£OTa,; i.e. in the case of Bocardo the words we are commenting
on are used loosely to mean 'if the proof in which the subject
of the two premisses is D is correct, that in which the subject is
C is also correct'.
12-13. TO ya.p £KTE9€v ••• £O'TlV, 'for the term set out is identi-
cal with a part of the subject of the particular negative premiss'.
13-14. Y(VETa., S€ ••• O'X"fLa.Tl, i.e. the validation of AnOnOn
in the second figure and of OnAnOn in the third is done by syllo-
gisms in the second and third figure respectively.
CHAPTER 9
Syllogisms with one apodeictic and one assertoric premiss, in the
first figure
30"15. It sometimes happens that when one premiss is neces-
sary the conclusion is so, viz. if that be the major premiss.
(A) Both premisses universal
(a) Major premiss necessary. AnAAn, EnAEn valid.
23. (b) Minor premiss necessary. AAnAn invalid; this shown
by reductio ad impossibile and by an example.
32. EAnEn invalid; this shown in the same way.
33.
(B) One premiss particular
(a) If the universal premiss is necessary the conclusion is so;
(b) if the particular premiss is necessary the conclusion is not so.
37. (a) AnIIn valid.
b1 •
EnIOn valid.
(b) Alnln invalid, since the conclusion In cannot be validated
by reductio.
2.
5. ElnOn invalid; this shown by an instance.1. ·S. 30"II-g. 30"2S
Ch. 10 discusses combinations of an assertoric with an apodeictic
premiss in the second figure, ch. I I similar combinations in the
third figure. Though in ch. 9 there is no explicit limitation to the
first figure, it in fact discusses similar combinations in that figure.
Since the substitution of an apodeictic premiss for one of the
premisses of an assertoric syllogism will plainly not enable us
to get a conclusion when none was to be got before, the only
point to be discussed in these chapters is, which of the valid
combinations will, when this substitution is made, yield an
apodeictic conclusion. Thus in ch. 9 A. discusses only the moods
corresponding to Barbara, Celarent, Darii, and Ferio; in ch. 10
only those corresponding to Cesare, Camestres, Festino, and
Baroco; in ch. I I only those corresponding to Darapti, Felapton,
Datisi, Disamis, Ferison, and Bocardo.
In 30"15-23 A. maintains that when, and only when, the major
premiss is apodeictic and the minor assertoric, an apodeictic con-
clusion may follow. His view is based on treating the predicate of
a proposition of the form' B is necessarily A' as being 'necessarily
A' ; for if this is so, 'All B is necessarily A, All C is B' justifies the
conclusion All C is necessarily A; while, on the other hand, 'All B
is A, All C is necessarily B' contains more than is needed to prove
that all C is A, but not enough to prove that it is necessarily A.
Thus his view rests on a false analysis of the apodeictic proposition.
30"25-8. et ya.p ... ,hrapxeLv. The point to be proved is that
from All B is A, All C is necessarily B, it does not follow that all
C is necessarily A. If all C were necessarily A, says A., one could
deduce both by the first figure-from All C is necessarily A, Some
B is necessarily C (got by conversion of All C is necessarily B)-
and by the third-from All C is necessarily A, All C is necessarily
B-that some B is necessarily A; but this is rpEOOO> (since all
we know is that all B is A).
AI. rightly points out (I28. 31-129. 7) that this argument,
while resembling a reductio ad impossibile, is different from it.
A. does not assume the falsity of an original conclusion in order
to prove its validity, as he does in such a reductio. In order to
prove that a certain conclusion does not follow, he supposes that
it does, and shows that if it did, it would lead to knowledge which
certainly cannot be got from the original premisses. A. calls the
conclusion of this reductio-syllogism not impossible but rpEf;oo>
("27), by which he means that 'Some B is necessarily A', while
compatible with 'All B is A', cannot be inferred from it, nor
from it+' All C is necessarily B' ; i.e. it may be false though the
original premisses are true.COMMENTARY
3 20
Maier (z a. 110 n. I) criticizes AI. on the ground that his account
implies that the premiss All B is A is compatible with two con-
tradictory statements-Some B is necessarily A (A.'s .p€i)So<;)
and No B is necessarily A (which A. expressly states to be com-
patible with All B is A, in '27-8). But AI. is right; All B is A is
compatible with either statement, though all three are not com-
patible together.
40. TO ynp r UTrO TO B (aTe More strictly, part of r falls under
B ("38-{}).
bZ-S. Et SE ••• auXXOYl.O'flOLs. A. is dealing here with the com-
bination All B is A, Some C is necessarily B. AI.'s first interpreta-
tion of this difficult passage (133. 19-29) is: This combination gives
an assertoric (not an apodeictic) conclusion (OUK (UTal TO UVJ-L-
7T'paUJ-La dvuYKuLov), because nothing impossible results from this,
i.e. because by combining the conclusion Some C is A with either
of the premisses we cannot get a conclusion contradicting the
other premiss. This is obviously true, but the interpretation is
open to two objections: (I) that it is a very insufficient reason
(and one to which there is no parallel in A.) for justifying a con-
clusion; and (2) that it does not agree with the words Ka8c1.1T€p
ovS' £V TOL<; Ku86>.ov UW\).0YlUJ-L0L,. In "23-8 A. showed that the
conclusion from AAn cannot be An, because that would yield
a false (or rather, unwarranted) conclusion when combined with
one of the original premisses; and that bears no resemblance
to the present argument, as interpreted above.
AI., feeling these difficulties, puts forward a second interpreta-
tion (133. 29-134. zo) (his third and fourth suggestions, 134.21-31,
135. 6-15, while not without interest, are less satisfactory): The
conclusion from AIn cannot be apodeictic, because such a con-
clusion cannot be established by a reductio ad impossibile. An
attempt at such a reduction would say 'If it is not true that some
C is necessarily A, it is possible that no C should be A'. But from
this, combined with the original minor premiss Some C is neces-
sarily B, it only follows that it is possible that some B should not
be A (cf. 40bz-3), which does not contradict the original major
premiss. On the other hand (AI. supposes A. to mean us to
understand), if we deduce from our original premisses only that
some C is A, we can prove this by a reductio. For if no C is A,
and some C is necessarily B, we get Some B is not A (3281-4),
which contradicts the original premiss All B is A.
This is a type of argument for which there is a parallel, viz. in
36"19-25, where A. argues that a certain combination yields only
a problematic conclusion, because an assertoric conclusion cannot321
be established by a reductio. But (as Maier contends, 2 a. 112 n.)
the attempted reductio which A. had in mind is more likely to
have been that which combines It is contingent that no C should
be A with the original major premiss All B is A. From this
combination nothing, and therefore nothing impossible, follows.
This is more likely to have been A.'s meaning, since the invalidity
of AEc as premisses is put right in the forefront of his treat-
ment of combinations of an assertoric and a problematic premiss
in the second figure (37bI9-23), and may well have been in his
mind here.
Even this argument, however, is quite different from that used
in dealing (in 30a2S-8) with the corresponding universal syllogism.
Ollotv yap aovvaTov 01Jj.L7T{7TT£L must therefore be put within brackets,
instead of being preceded by a colon and followed by a comma.
It is a fair inference that A. held that where an apodeictic
consequence does follow it can be established by a reductio.
E.g., he would have validated the syllogism All B is necessarily
A, All C is B, Therefore all C is necessarily A, by the reductio
If some C were not necessarily A, then since all C is B, some B
would not be necessarily A ; which contradicts the original major
premiss.
5-6. 0110LWS SE •.• AEUKOV. I.e., Ern does not establish On,
as we can see from the fact that, while it might be the case that
no animals are in movement, and that some white things are
necessarily animals, it could not be true that some white things
are necessarily not in movement, but only that they are not in
movement.
CHAPTER 10
Syllogisms with one apodeictic and one assertoric premiss, in the
second figure
(A) Both premisses universal
(a) If the negative premiss is necessary the conclusion is so; (b)
if the affirmative premiss is necessary (AnE, EAn) the conclusion
is not so.
(a) EnAEn valid; this shown by conversion.
14. AEnEn valid; this shown by conversion.
18. (b) AnEEn invalid; this shown (a) by conversion.
24. (f3) by reductio.
31. ("I) by an example.
y322
COMMENTARY
31"1.
(B) One premiss particular
(a) When the negative premiss is universal and necessary the
conclusion is necessary; (b) when the affirmative premiss is
universal the conclusion is not necessary.
S. (a) EnIOn valid; this shown by conversion.
10. (b) AnOOn invalid; this shown by an example.
IS. AOnOn invalid; this shown by an example.
30b7-9. 'E1TL SE ••• a.vaYKaLov. This is true without exception
only when the premisses are universal (for AOn does not yield
an apodeictic conclusion (3I"IS-I7)), and in this paragraph A.
has in mind only the combinations of two universal premisses.
18-19. Et SE ••• a.vaYKaLov. This includes the cases AnE, EAn.
In b20- 40 A. discusses only the first case. He says nothing about
EAn, because it is easily converted into EAn in the first figure,
which has already been shown to give only an assertoric conclu-
sion ("32-3).
22-4. SESELKTaL ••• a.vaYKaLov, "32-3.
26-7. Il"SEVL ••• a.vaYK"S, 'necessarily belongs to none', not
'does not necessarily belong to any'.
32-3. TO O'ull1TEpaO'lla ••• a.vaYKaLov, the conclusion is not a
proposition necessary in itself, but only a necessary conclusion
from the premisses.
34. KaL aL 1TpoTaO'ELS 0llolwS Et~"cpeWO'av, sc. to those in b 20 -I.
3131-17. 'OllolwS S' ••• a.1T6SEL~L'. In '2-3, S-IO A. points out
that Festino with the major premiss apodeictic gives an apodeictic
conclusion. In '3-S, IO-IS, IS-17 he points out that Baroco (I)
with major premiss necessary, and (2) with minor premiss neces-
sary, gives an assertoric conclusion. He omits Festino with
minor premiss necessary-No P is M, Some 5 is necessarily M.
This is equivalent to No M is P, Some 5 is necessarily M, and he
has already pointed out that this yields only an assertoric con-
clusion (30bS-{i).
In the whole range of syllogisms dealt with in chs. 4-22 this
is the only valid syllogism, apart from some of those which are
validated by the 'complementary conversion' of problematic
propositions, that A. fails to mention.
14-15. OL yap aUToL ••• O'U~~OYLO'IlWV, cf. 30b33-8. If all men
are necessarily animals, and some white things are not animals,
then some white things are not men, but it does not follow that
they are necessarily not men.
17. SLa yap TWV aUTwv opwv ,; a.1T6SEL~L" cf. 30b33-8. If all323
men are in fact animals, and some white things are necessarily not
animals, it does not follow from the data that they are necessarily
not men.
CHAPTER 11
SyUogisms with one apodeictic and one assertoric premiss, in the
third figure
(A) Both premisses universal
(a) When both premisses are affirmative the conclusion is
necessary. (b) If the premisses differ in quality, (a) when the
negative premiss is necessary the conclusion is so; (f3) when the
affirmative premiss is necessary the conclusion is not so.
24. (a) AnAln valid; this shown by conversion.
31. AAnln valid; this shown by conversion.
33. (b) (a) EnAOn valid; this shown by conversion.
37. (f3) EAnOn invalid; cf. the rule stated for the first figure,
that if the negative premiss is not necessary the conclusion is
not so.
b 4 • Its invalidity also shown by an example.
31318.
(B) One premiss particular
(a) When both premisses are affirmative, (a) if the universal
premiss is necessary so is the conclusion. IAnln valid; this shown
by conversion.
19. AnIIn valid for the same reason.
20. (f3) If the particular premiss is necessary, the conclusion is
not so. Alnln invalid, as in the first figure.
27. Its invalidity also shown by an example.
31. InAln invalid; this shown by the same example.
33. (b) Premisses differing in quality. EnIOn valid.
37. OA"On, ElnO", OnAOn invalid.
40. Invalidity of OAnOn shown by an example.
3231. Invalidity of ElnOn shown by an example.
4. Invalidity of OnAOn shown by an example.
12.
31"31-3. 01-l0lWS SE ••• E~ ci~a.yKTJS. If all C is A and C is neces-
sarily B, then all C is necessarily B and some A is C. Therefore
some A is necessarily B. Therefore some B is necessarily A.
41-bl. TO SE r TLVL TWV B, sc. necessarily.
2-4' SESELKTaL yap •.. civaYKaLov. A. did not say this in so
many words in the discussion of mixed syllogisms in the first
figure (ch. 9). But he said (30315-17) that if the major premiss is
not apodeictic, the conclusion is not apodeictic. And in the firstCOMMENTARY
figure only the major premiss can be negative. Thus the former
statement includes the present one.
8-10. li Et ,.U; •.. TOUTloIV. Since the suggestion that every
animal is capable of being good might be rejected as fanciful, A.
substitutes another example. If no horse is in fact awake (or
'is in fact asleep'), and every horse is necessarily an animal, it
does not follow that some animal is necessarily not awake (or
'not asleep').
15-20. cj"rOSEl~lS 5' ... EUTlV. In bl6-19 IAnln is validated as
AA.nID was in 831-3. The premisses are Some C is A, All C is
necessarily B. Converting the major premiss and transposing the
premisses, we get All C is necessarily B, Some A is C, Therefore
some A is necessarily B. Therefore some B is necessarily A.
In bX9 - 20 AnIIn is validated as AnAln was in "24-30. The pre-
misses are All C is necessarily A, Some C is B. Converting the
minor premiss, we get All C is necessarily A, Some B is C, There-
fore some B is necessarily A.
324
2S~. OTE
5' ... a.Va.YKa.'lOV, 30"35-7, b2- 5 ·
31-3. OJ.lOlWS SE ••• a.Va.YKa.'lOV. If we use the same terms in
the same order we get A. saying 'It might be true that some
animals are necessarily awake, and that all animals are in fact
two-footed, and yet untrue that some two-footed things are
necessarily awake'. But, as AI. and P. observe, he is more likely
to have meant that it might be true that some animals are neces-
sarily two-footed, and that all animals are in fact awake, and yet
untrue that some waking things are necessarily two-footed.
38. li TO UTEPTlTlKOV Ka.Ta. j.lEpOS. sc. cil'aYKaLOI' n(Jfj, cf. 32a4-5·
39-40. Ta. j.lEv ya.p ••• EPOUjloEV, i.e. (x) that neither Some C is
not A, All C is necessarily B, nor No C is A, Some C is necessarily
B, yields an apodeictic conclusion follows for the same reason
for which No C is A, All C is necessarily B, does not yield one
(a37-bI0). (2) That Some C is necessarily not A, All C is B, does
not yield an apodeictic conclusion follows for the same reason for
which Some C is necessarily A, All C is B, does not yield one
(h3 I -3)·
40-1. OPOl S' ... j.lEUOV a.v9pwTl'os. I.e., it might be the case
that some men are not awake, and that all men are necessarily
animals, and yet not true that some animals are necessarily not
awake.
3284-5. OTa.V 5, ... j.lEUOV ~~ov. I.e., it might be true that
some animals are necessarily not two-footed, and that all
animals are in movement, and yet not true that some things that
are in movement are necessarily not two-footed.In giving instances of third-figure syllogisms, A. always names
the middle term last. Therefore we should read not Sl-rrow
fL/CTOV, which is the best supported reading, but fLECTOV ~(~IOV or
{4Jov fLECTOV, and of these the former (which is the reading of C)
is most in accordance with A.'s usual way of speaking (cf. 27820,
28"35. b 3 8, 31b41). The other readings must have originated from
Sl'TTOVII having been written above the line as a proposed emenda-
tion of ~4Jov.
CHAPTER 12
The modality of the premisses leading to assertoric or apodeictic
conclusions
32"6. Thus (I) an assertoric conclusion requires two assertoric
premisses; (2) an apodeictic conclusion can follow from an
apodeictic and an assertoric premiss; (3) in both cases there must
be one premiss of the same modality as the conclusion.
32"&-,. 4lavEpov oov . . . U1Ta.PXELV. inr&'PXHV is here (as often
elsewhere) used not to distinguish an affirmative from a negative
proposition. but an assertoric from an apodeictic. A. here says
that an assertoric proposition requires two assertoric premisses.
But in chs. 9-II he has shown that many combinations of an
assertoric with an apodeictic premiss yield an assertoric con-
clusion. The two statements can be reconciled by noticing that
when A. says an assertoric conclusion requires two assertoric
premisses, he means that this is the minimum support for an
assertoric conclusion. Now an apodeictic premiss says more than
an assertoric, and a problematic premiss says less; therefore an
assertoric and an apodeictic premiss can prove an assertoric
conclusion, but an assertoric and a problematic premiss cannot.
Cf. the indication in 29b30-2 that A. thinks of the possible as
including the actual, and the actual as including the necessary.
It should be noted, however, that A. has not proved what he
here describes as cpav£pov. He has proved (I) that an assertoric
conclusion can be drawn from two assertoric premisses, and from
an assertoric and an apodeictic premiss, and (2) that an apodeictic
conclusion in certain cases follows from an assertoric and an
apodeictic premiss; but he has not proved that an assertoric
conclusion requires that each premiss be at least assertoric (i.e.
be assertoric or apodeictic) ; and in chs. 16, 19, 22 he argues that
certain combinations of an apodeictic with a problematic con-
clusion yield an assertoric conclusion.COMMENTARY
CHAPTER 13
Preliminary discussion of the contingent
3ZaI6. We now proceed to discuss the premisses necessary for
a syllogism about the possible. By 'possible' I mean that which
is not necessary but the supposition of which involves nothing
impossible (the necessary being possible only in a secondary
sense).
[ZI. That this is the nature of the possible is clear from the
opposing negations and affirmations; 'it is not possible for it to
exist', 'it is incapable of existing', 'it necessarily does not exist'
are either identical or convertible statements; and so therefore
are their opposites; for in each case the opposite statements are
perfect alternatives.
z8. The possible, then, will be not necessary, and the not
necessary will be possible.]
z9. It follows that problematic propositions are convertible-
not the affirmative with the negative, but propositions affirmative
in form are convertible in respect of the opposition between the
two things that are said to be possible; i.e. 'it is capable of belong-
ing' into 'it is capable of not belonging', 'it is capable of belonging
to every instance' into 'it is capable of belonging to no instance'
and into 'it is capable of not belonging to every instance', 'it is
capable of belonging to some instance' into 'it is capable of not
belonging to some instance' ; and so on.
36. For since the contingent is not necessary, and that which is
not necessary is capable of not existing, if it is contingent for A to
belong to B it is also contingent for it not to belong.
bI. Such propositions are affirmative; for being contingent
corresponds to being.
4. 'Contingent' is used in two senses: (1) In one it means 'usual
but not necessary', or 'natural' ; in this sense it is contingent that
a man should be going grey, or should be either growing or
decaying (there is no continuous necessity here, since there is not
always a man, but when there is a man he is either of necessity
or usually doing these things).
10. (2) In another sense it is used of the indefinite, which is
capable of being thus and of being not thus (e.g. that an animal
should be walking, or that while it is walking there should be an
earthquake), or in general of that which is by chance.
13. In either of these cases of contingency' B may be A' is
convertible with' B may not be A': in the first case because1. 13. 32"16-29
3 2 7
necessity is lacking, in the second because there is not even a
tendency for either alternative to be realized more than the other.
18. There is no science or demonstration of indefinite combina-
tions, because the middle term is only casually connected with
the extremes; there is science and demonstration of natural com-
binations, and most arguments and inquiries are about such.
Of the former there can be inference, but we do not often look
for it.
23. These matters will be more fully explained later; we now
turn to discuss the conditions of inference from problematic pre-
misses. 'A is contingent for B' may mean (I) 'A is contingent for
that of which B is asserted' or 'A is true of that for which B is
contingent'. If B is contingent for C and A for B, we have two
problematic premisses; if A is contingent for that of which B is
true, a problematic and an assertoric premiss. We begin with
syllogisms with two similar premisses.
32aI6-bU. 1TEpt Si TOU iVSEXOI'EVOU ••. ~"1TEla9(ll. With this
passage should be compared 2S'37-b19 and the n. thereon.
I~21. ~EYW S' ••• ~EY0I'EV. In '18-20 A. gives his precise view
of 'TO iv8fXOP.fVOV. It is that which is not necessary, but would
involve no impossible consequence; and since that, and only
that, which is itself impossible involves impossible consequences,
this amounts to defining 'TO iv8fXOP.EVOV as that which is neither
necessary nor impossible. 'Necessary' and 'impossible' are not
contradictories but contraries; 'TO iv8fXOP.fVOV is the contingent,
which lies between them. It is only in a loose sense that the
necessary can be said iv8lXH18aL (a20-1)-in the sense that it is
not impossible.
21-9. on Si ••• Ev5EX0I'EVOV. Though this passage occurs in
all the MSS. and in AI. and P., it seems impossible to retain it in
the text. In ar8-20 A. has virtually defined 'TO iv8qoP.fVOV as
that which is neither impossible nor necessary, in "2I-8 it is
identified with the not impossible, and in '28-<) with the not
necessary. Becker (pp. II-I3) seems to be right in treating the
passage as an interpolation by a writer familiar with the doctrine
of De Int. 22"14-37. That passage contains several corruptions,
but with the necessary emendations it is found to identify 'TO
iv8fXOP.fVOV with the not impossible, i.e. to state the looser sense
of the term in which, as A. observes here in "20-1, even the neces-
sary is iv8fXOP.EVOV. But, since the complementary convertibility
of problematic propositions which is stated in 32'29-br implies
that the iv8fXOP.EVOV is not necessary, the interpolator introducesCOMMENTARY
3 2 8
the sentence in 82B---9 to lead up to it, but overshoots the mark by
completely identifying the lvS£xofL£VOV with the not necessary,
instead of with that which is neither necessary nor impossible.
29-35. UU\-L~ClLV£L 8£ ••• Ci~~wv. Since that which is contingent
is not necessary, it follows that (1) 'For all B, being A is con-
tingent' entails 'For all B, not being A is contingent' and 'For
some B, not being A is contingent', (2) 'For some B, being A is
contingent' entails 'For some B, not being A is contingent'.
(3) 'For all B, not being A is contingent' entails 'For all B, being
A is contingent' and 'For some B, being A is contingent', (4)
'For some B, not being A is contingent' entails 'For some B,
being A is contingent'.
b l - 3 . £tul 8' ••• 'lrpOT£pOV. I.e., just as 'B is not-A' is an
affirmative proposition, 'Bis capable-of-not-being A' is affirmative.
A. has already remarked in 2Sb21 that in this respect 'TO lIlSEX£'TCl'
'Tip ;C1'TLV OfLO{w~ 'Ta.'T'T€'TaL.
4-22. foLWPLU\-L£VWV 8£ ••• t"lT£LU9ClL. In 2Sb14-IS A. carelessly
identified 'TO b,S£xofLEVOV in the strict sense with 'TO W. l1ft 'TO 1fOA~
KClt 'Tip 1f£CPVKEVo.L. He here points out that 'TO lIlS£XOfL£l'OV in the
strict sense, that which is neither impossible nor necessary, occurs
in two forms, one in which one alternative is habitually realized,
the other only occasionally, and another in which there is no
prevailing tendency either way. The distinction has, as he points
out in b r 8-22, great importance for science, since that which is
habitual may become an object of scientific study while the
purely indeterminate cannot. But it should be noted that the
distinction plays no part in his general doctrine of the logic of
contingency, as it is developed in chs. r3-22. Apart from other
considerations the doctrine of compleme~tary conversion, which
is fundamental to his logic of the problematic syllogism, has no
application to a statemp.nt that something is true w. l1ft 'TO 1foM,
since' B is usually A' is not convertible with' B is usually not
A'. Becker (pp. 76-83) views the whole passage with suspicion,
though he admits that it may have an Aristotelian kernel. It
seems to me to be genuinely Aristotelian, but to be a note having
no organic connexion with the rest of chs. 13-22.
4. 1Tci.~LV ~£YW~EV can hardly mean 'let us repeat'; for, though
A. speaks in 2sbr4 of oao. 'Tip w. l1ft 'TO 7rOA~ Kat 'Tip 7r£CPVKEVo.L
MYE'TaL lvSEx£a(JaL, he says nothing there of 'TO o.0PLC1'TOV. 7raALII
MYWfL£V means 'let us go on to say'-a usage recognized in Bonitz,
lndf.x, SS9hI3--14. For the rp.adin~ UYWfLEV cf. 2Sb26 n.
13-15. ci.V'TLU'Tp£~£L ~EV o~v ••• iv8EXO~EVWV. The KO.' of the
Greek MSS. is puzzling. Al.'s best suggestion is that A. meansthat' B may be A' is convertible with' B may not be A' as well
as with' A may be B'; but probably rand Pacius are right in
omi tting the word.
I8-z3_ i1l'LUnll'''1 8€ ___ €1I'OI'EVOLS_ What A. means is this: If
all we know of the connexion between A and B, and between
Band C, is that B is capable of being A and that C is capable of
being B, then though we can infer that C is capable of being A,
the resulting probability of Cs being A is so small as not to be
worth establishing. On the other hand, if we know that B tends
to be A, and that C tends to be B, the conclusion 'c tends to be
A' will be important enough to be worth establishing. And since
in nature, according to A.'s view, most of the connexions we can
establish are statements of tendency or probability rather than
of strict necessity, most .\6YOL and aKNJ€L~ actually have premisses
and conclusions of this order.
A. postpones the discussion of the usual and the d6pLaTov to
an indefinite future (b z3 ). There is no passage of the A nalytics
that really fulfils the promise; but 43b32-6 and An. Post. 7S b33-6,
87b19-27, 96'8-19 touch on the subject.
z5-37- i1l'El 8€ - __ «AAoLs_ The passage is a difficult one, and
neither the statement with which it opens (b 2S - 7) nor the structure
of the first sentence can be approved; but correct punctuation
makes the passage at least coherent, and in view of the undisputed
tradition by which it is supported we should hardly be justified
in accepting Becker's excisions (pp. 36-7). A. starts with the
statement that (1) 'For B, being A is contingent' is ambiguous,
meaning either (z) 'For that to which B belongs, being A is
contingent' or (3) 'For that for which B is contingent, being A
is contingent'. He then (b z7 - 3o) supports this by the premisses
(a) that (4) Ka8' 00 TO B, TO A €V'MXETaL may mean either (2) or
(3) (because it is not clear whether lnrapXEL or €v8iXETaL is to be
understood after TO B), and (b) that (1) means the same as (4);
and (b 31 - 2 ) repeats his original statement as following from
these premisses.
In the remainder of the passage A. applies to the syllogism the
distinction thus drawn between two senses of Ka8' 00 TO B, TO
A €V8iX£TaL. If in the major premiss A is said to be contingent
for B, which is in the minor premiss said to be contingent for C,
we have two problematic premisses. If in the major premiss A
is said to be contingent for B, which is in the minor premiss said
to be true of C, we have a problematic and an assertoric premiss.
A. proposes to begin with syllogisms with two similar premisses,
Ka8a7T£p Ka~ €v TOt!;" a>..\OL~, i.e. as syllogisms with two assertoricCOMMENTARY
33 0
premisses (chs. 4-6) and those with two apodeictic premisses
(ch. 8) were treated before those with an apodeictic and an
assertoric premiss (chs. 9-II).
CHAPTER 14
Syllogisms in the first figure with two problematic premisses
(A) Both premisses universal
AeAeAe valid.
33"1. EeAeEe valid.
5. AeEeAe valid, by transition from Ee to Ae.
12. EeEeAe valid, by the same transition.
17. Thus if the minor premiss or both premisses are negative,
there is at best an imperfect syllogism.
32b38.
21.
(B) One or both premisses particular
(a) If the major premiss is universal there is a syllogism.
Aelele valid.
25. EeleOe valid.
27. (f3) If the universal premiss is affirmative, the particular
premiss negative, we get a conclusion by transition from Oe to le.
AeOele valid.
34. (b) If the major premiss is particular and the minor universal
(IeAe, OeEe, IeEe, OeAe) , or if (c) both premisses are particular
(Iele, OeOe, IeOe, Oele), there is no conclusion. For the middle
term may extend beyond the major term, and the minor term
may fall within the surplus extent; and if so, neither Ae, Ee, le,
nor Oe can be inferred.
b 3 • This may also be shown by contrasted instances. A pure or
a necessary conclusion cannot be drawn, because the negative
instance forbids an affirmative conclusion, and the affirmative
instance a negative conclusion. A problematic conclusion cannot
be drawn, because the major term sometimes necessarily belongs
and sometimes necessarily does not belong to the minor.
18. It is clear that when each of two problematic premisses is
universal, in the first figure, a conclusion always arises-perfect
when the premisses are affirmative, imperfect when they are
negative. 'Possible' must be understood as excluding what is
necessary-a point sometimes overlooked.
32°40-33"1. TO ya.p EVOEXEa9a.L ... EAEyoIlEv, i.e. we gave (in
32b25-32), as one of the meanings of 'A may belong to all B',33 1
A may belong to anything to which B may belong'. From this
it follows that if A may belong to all Band B to all C, A may
belong to all C.
33 3 3-5. TO yap KaO' o~ ••• EVSEXOI1EVWV. 'for the statement
that A is capable of not being true of that of which B is capable
of being true, implied that none of the things that possibly fall
under B is excluded from the statement'. p.~ £v8£xw6uL in "4 is
used loosely for £v8£xw6m p.~ imaPXHv.
5-'1. oTav SE ••• auAAoYLal1os. because premisses of the form
AE in the first figure prove nothing.
7-8. ciVTLaTpacpELaTJS SE ••• EvSEXEaOaL. i.e. when from 'B is
capable of belonging to no C' we infer' B is capable of belonging
to all C' : cf. 3ZaZ9-br.
8. YLvETaL 0 aUTOS oa1TEp 1TpOTEPOV. i.e. as in 32b38-40.
10. TO(lTO S' ELp"TaL 1TpOTEPOV. in 3Za29-bl.
ZI-3. 'Eav S' ••• TEAELOS. If T£AHor; be read, the statement will
not be correct; for in 3z7-34 A. goes on to point out that when the
particular premiss is negative and the universal premiss affirma-
tive, the latter being the major premiss, there is no T£AHor;
(TIJ)..)..0YLap.or;. There is no trace of TEAHor; in AI. (169. zo) or in
P.'s comment (r58. r3). and it is not a word they would have been
likely to omit to notice if they had had it in their text. Becker
(p. 75) seems to be right in wishing to omit it.
24-5. ToilTo SE ••. EvSEXEaOaL. \Vaitz reads TTUv-rL after £v8;-
XW6UL (following B's second thoughts), on the ground that it is
the remark in 32b25-32 rather than the definition of TO £v8£xw6uL
in 323r8-zo that is referred to. But the latter may equally well
be referred to, and the reading TTavTL no doubt owes its origip to
the fact that one of Al.'s two interpretations (r69. Z3-9) is that
£v8Exw6m is to be understood as if it were £v8£xw6uL TTUv-rL. Al..
however, thinks the definition of TO £V8iXW6UL in 32318-20 may
equally well be referred to (169. 30-2).
29. Tfi SE OEaEL 0110LW5 ExwaLv. i.e. 'but the universal premiss
is still the major premiss'.
29-30. orov ••• umipXELv. A. does not explicitly mention the
case in which the premisses are EcOc, which can be dealt with on
the same lines as the case mentioned, AcOc.
32. aVTLaTpacpELaTJS SE TTjS EV I1EPEL refers not to conversion in
the ordinary sense but to conversion from' B may not belong to
C' to 'B may belong to C' ; cf. 3Z3Z9-br.
33. TO aUTO ••• 1TPOTEPOV. i.e. as in "z4.
34. Ka06.1TEp ••• a.PXTi5. i.e. as AcEc gave the same conclusion
as AcAc ("5-12).
I332
COMMENTARY
34-8. 'Ea.v S' ••• auAAoy~a .... 6s. The first three £a.v T£ clauses
express alternatives falling under one main hypothesis; the
fourth expresses a new main alternative. Therefore there should
be a comma after diLoLOaX~iLovES' (a37).
The combinations referred to are IeAe, OeEe, IeEe, OeAe, IeIe,
OeOe, IeOe, OeIe. Since a proposition of the form 'For all B,
not being A is contingent' is convertible with 'For all B, being
A is contingent', and one of the form 'For some B, not being A
is contingent' with 'For some B, being A is contingent' (avn-
rrrp'q,ovaw at ICQ'Ta 'TO £v8'XEa8m Trpo'TaaELS', bz, cf. 3ZaZ9-b1), all
these combinations are reducible to the combinations 'For some
B, being A is contingent, For all C (or For some C), being B is
contingent'. Now since B may extend beyond A ("38-9), we may
suppose that C is the part of B which extends beyond A (Le. for
which being A is not contingent). Then no conclusion follows;
there is an undistributed middle.
b J -8. (TL SE • • • t .... a.TLov.
The examples given are ICO~VO~
mfv'Twv, i.e. they are to illustrate all the combinations of premisses
mentioned in '34-~ n. The reasoning therefore is as follows: The
premisses It is possible for some white things to be animals (or
not to be animals), It is possible for all (or no, or some) men
to be white (or for some men not to be white) might both be
true. But in fact it is not possible for any man not to be an animal.
Therefore a negative conclusion is impossible. On the other hand,
the same major premiss and the minor premiss It is possible for
all (or no, or some) garments to be white (or for some garments
not to be white) might both be true. But in fact it is not possible
for any garment to be an animal. Therefore an affirmative con-
clusion is impossible. Therefore no conclusion is possible.
11-13. 0 .... Ev ya.p ••• KaTacf!aTLKI\!' I.e., the possibility of an
affirmative conclusion is precluded by the fact that sometimes,
when the premisses are as supposed (i.e. the major premiss parti-
cular, in the first figure), the major term cannot be true of the
minor; and the possibility of a negative conclusion is precluded
by the fact that sometimes the major term cannot fail to be true
of the minor.
14-16. KaL 1TavTL TI\! Eaxa.T~ TO 1TPWTOV a.vaYK'l (se. v7TaPX£Lv),
Le. in some cases (e.g. every man must be an animal), KaL OUSEVL
EVS(XETa~ u1TapXE~v, i.e. in some cases (e.g. no garment can be an
animal).
16-17. TO ya.p a.vaYKaLov .•• £VSEX6 .... £VOV, cf. 32'19.
:n. 'lrA"V KaT'lyop~Kwv .... Ev TEA£LOC;. That AeAe yields a direct
conclusion has been shown in 3zb38-33ar.333
C7TEP11nKWV 8E Q.TEAtlS. That EcEc yields a conclusion indirectly
.
has been shown in a12-17.
23. KaTa TOV Elp11lL£vOV S~op~alLov, cf. 32"18-20.
£vtOTE Se Aav9b.vEL TO To~o(hov, i.e., the distinction between the
genuine €vS€X61-'€VOV (that which is neither impossible nor necessary)
and that which is €vS€X6/L€VOV only in the sense that it is not
impossible.
CHAPTER 15
Syllogisms in the first figure with one problematic and one assertoric
premiss
33b25.
(A) Both premisses universal
(a) When the major premiss is problematic <and (a) the minor
affirmative), the syllogism is perfect, and establishes contingency;
(b) when the minor is problematic, the syllogism is imperfect, and
those that are negative establish a propo~ition of the form 'A
does not belong to any C (or to all C) of necessity'.
33. (a) (a) AcAAc valid; perfect syllogism.
36. EcAEc valid; perfect syllogism.
3482. (b) When the minor premiss is problematic, a conclusion
can be proved indirectly by reductio ad impossibile. We first lay
it down that if when A is, B must be, when A is possible B must
be possible. For suppose that, though wheri A is, B must be, A
were possible and B impossible. If, then, that which was pos-
sible, when it was possible for it to be, might come into being,
while that which was impossible, when it was impossible for it to
be, could not come into being, but at the same time A were
possible and B impossible, A might come into being, and be,
without B.
u. We must take 'possible' and 'impossible' not only in
reference to being, but also in reference to being true and to
existing.
16. Further, 'if A is, B is' must not be understood as if A were
one single thing. Two conditions must be given, as in the pre-
misses of a syllogism. For if r is true of ..1, and..1 of Z, r must be
true of Z, and also if each of the premisses is capable of being
true, so is the conclusion. If, then, we make A stand for the
premisses, and B for the conclusion, not only is B necessary if A
is, but B is possible if A is.
25. It follows that if a false but not impossible assumption be
made, the conclusion will be false but not impossible. For sinceCOMMENTARY
334
it has been shown that when, if A is, B is, then if A is possible,
B is possible, and since A is assumed to be possible, B will be
possible; for if not, the same thing will be both possible and
impossible.
34. «(6) (a) Minor premiss a problematic affirmative.) In view
of all this, let A belong to all B, and B be contingent for all C.
Then A must be possible for all C (AAcAP valid). For let it not be
possible, and let B be supposed to belong to all C (which, though
it may be false, is not impossible). If then A is not possible for
all C, and B belongs to all C, A is not possible for all B (by a
third-figure syllogism). But A was assumed to be possible for all
B. A therefore must be possible for all C; for by assuming the
opposite, and a premiss which was false but not impossible, we
have got an impossible conclusion.
b[Z. We can also effect the reductio ad impossibile by a first-
figure syllogism.]
7. We must understand 'belonging to all of a subject' without
exclusive reference to the present; for it is of premisses without
such reference that we construct syllogisms. If we limit the
premiss to the present we get no syllogism; for (r) it might happen
that at a particular time everything that is in movement should
be a man; and being in movement is contingent for every horse;
but it is impossible for any horse to be a man;
14. (2) it might happen that at a particular time everything
that was in movement was an animal; and being in movement is
contingent for every man; but being an animal is not contingent,
but necessary, for every man.
19. EAcEp valid; this shown by reductio ad impossibile using
the third figure. What is proved is not a strictly problematic
proposition but' A does not necessarily belong to any C'.
31. We may also show by an example that the conclusion is not
strictly problematic;
37. and by another example that it is not always apodeictic.
Therefore it is of the form' A does not necessarily belong to any C'.
35"3. (b) (/3) Minor premiss a problematic negative. AEcAp
valid, by transition from Ec to Ac.
11. EEcEp valid, by transition from Ec to Ac.
zoo (Return to (a) (a) (/3) Major premiss problematic, minor
negative. AcE, EcE prove nothing; this shown by contrasted
instances.
z5. Thus when the minor premiss is problematic a conclusion
is always possible; sometimes directly, sometimes by transition
from Ec in the minor premiss to Ac.335
30.
(B) One premiss particular
(a) When the major premiss is universal, then (a) when the
minor is assertoric and affirmative there is a perfect syllogism
(proof as in the case of two universal premisses) (AeIIe, EelOe
valid).
35. (P) When the minor premiss is problematic there is an
imperfect syllogism-proved in some cases (Alelp, EleOP) by
reductio ad impossibile, while in some cases transition from the
problematic premiss to the complementary proposition is also
required,
b2. viz. when the minor is negative (AOelp, EOeO p).
8. (y) When the minor premiss is assertoric and negative
(AeO, EeO) nothing follows; this shown by contrasted instances.
II. (b) When the major premiss is particular (leA, IcE, OeA,
OeE, lAc, lEe, OAe, OEe), nothing follows; this shown by con-
trasted instances.
(C) Both premisses particular
When both premisses are particular nothing follows; this shown
by contrasted instances.
20. Thus when the major premiss is universal there is always a
syllogism; when the minor so, never.
33b25-33. 'Eo.v S' ... U1Ta.PXELV. A. lays down here four im-
portant generalizations: (I) that all the valid syllogisms (in the
first figure) which have a problematic major and an assertoric
minor are perfect, i.e. self-evidencing, not requiring a reductio ad
impossibile; (z) that they establish a possibility in the strict
sense (according to the definition of possibility in 32"18-20; (3)
that those which have an assertoric major and a problematic
minor are imperfect; and (4) that of these, those that establish
a negative establish only that a certain disconnexion is possible
in the loose sense. This distinction between a strict and a wider
use of the term 'possible' is explained at length in 34bI9-35a2;
'possible' in the strict sense means 'neither impossible nor
necessary', in the wider sense it means 'not impossible'.
All four generalizations are borne. out in A.'s treatment of the
various cases in the course of the chapter. But syllogisms with
an assertoric major and a problematic minor which prove an
affirmative (no less than those which prove a negative)-viz.
those with premisses AN (34"34-bZ ), AEc (3533-11), Ale (ib.
3S-bl), or AOe (3Sbz-8)-are validated by a reductio ad impossibile,
and A.'s arguments in 34bz7-37 and in 37"15-29 show that anyCOMMENTARY
syllogism so validated can only prove a possibility in the wider
sense of possibility. Becker (pp. 47---<)) therefore proposes to read
in 33bz9-31 a'n,A£LS" T£ 7TavT£S" ol (7tIAAoy,afLo~ Ka~ OU TOU ••• Jv8£xo-
fLtvOIJ, aAAa. TOU fL~ 19 avaYK1)S" iJ7TapXHv. But b 3 O- 3 ilia . . .
tJ'TTapXHV shows that A. has in mind here only conclusions stating a
negative possibility; he seems to have overlooked the point that
those which state a positive possibility similarly state a possibility
only in the wider sense.
A. does not state his reason either for saying that when the
major premiss is assertoric and the minor problematic, the
syllogism is imperfect, or for saying that the possibility established
is only possibility in the wider sense. But it is not difficult to
divine his reasons, For the first dictum his reason must, I think,
be that while' All B is capable of being A, All C is B' are premisses
that are already in the correct form of the first figure, 'All B is
A, All C is capable of being B' are premisses that in their present
form have no middle term. For the second dictum his reason
must be the following: For him €V8tX£Ta, in its strict sense is a
statement of genuine contingency; 'It is possible that all C should
be B' says that for all C it is neither impossible nor necessary
that it should be B. Now when all B is A, A may be (and usually
will be) a wider attribute than B, and if so, when Cs being B
is contingent, its being A may be not contingent but necessary.
The most, then, that could follow from the premisses is that it
is not impossible that all C should be A.
A.'s indirect proof that this follows is, as we 'shall see, not
convincing, He would have done better, it might seem, to say
simply that 'All B is A, For all C being B is contingent (neither
impossible nor necessary)' entail It is not impossible that all C
should be A. But that would have been open to the objection
that it is not in syllogistic form, having no single middle term.
And it is open to a less formal objection. All the existing B's
may be A, and it may be not impossible that all the Cs should be
B, and yet it may be impossible that all the Cs should have the
attribute A which all the existing B's have. This difficulty A.
tries to remove by his statement in 34b7-18 that to make the
conclusion 'It is not impossible that all C should be A' valid, the
premiss All B is A must be true not only of all the B's at a parti-
cular time. But this proviso is not strict enough. Even if all the
B's through all time have had, have, and will have the attribute
A, the premisses will not warrant the conclusion It is not im-
possible that all C should be A, unless A is an attribute which is
necessary to everything that is B, either as a precondition orI. IS. 34"1-24
337
as a necessary consequence of its being B. In other words, to
justify the conclusion we need as major premiss not All B is A,
but All B is necessarily A.
34"1-33. uO TL fLEv o~v ... 6.BuvaTov. This section is an ex-
cursus preparatory to the discussion of the combination AAc in
a34-b2. In that combination the premisses are All B is A, For
all C, B is contingent. In the reductio ad impossibile by which A.
establishes the conclusion It is possible that all C should be A,
he takes as minor premiss of the reductio-syllogism not the
original minor premiss, but All C is B, and justifies this on the
ground that this premiss is at worst false, not impossible, so that
if the resultant syllogism leads to an impossible conclusion, that
must be put down to the other premiss, i.e. to the premiss which
is the opposite of the original conclusion. He sees that this
procedure needs justification, and to provide this is the object
of the present section.
7. OUTWS £XOVTWV, i.e. so that, if A is, B must be.
12-15. BEL 8E •.• i~EL. A. has in "5-7 laid down the general
thesis that if, when A is, B must be, then when A is possible,
B must be possible. In 87-12 he has illustrated this by the type
of case in which 'possible' means 'capable of coming into being',
i.e. in which it refers to a potentiality for change. He now points
out that the thesis is equally true with regard to possibility as
it is asserted when we say 'it is possible that A should be truly
predicated of, and should belong to, B' (& Te? dAT)8EVEa8u, Kal
& Te? v1TC1PXELv)-where there is no question of change. It is
possibility in the latter sense that is involved in the application A.
makes of the general thesis to the case of the syllogism ("19-24).
The reference of Kal oaaxwS' c!AAWS' MynaL TO SuvaT6v (°14) is not
clear. AI. thinks it refers to TO WS' bri TO 7TAEtaTOV, TO d6pLaTov, and
TO €7T' iAaTTov (cf. 32b 4-22), or to the possibility which can be
asserted of that which is necessary (25"38), or to other kinds of
possibility recognized by the Megarian philosophers Diodorus and
Philo. None of these is very probable. 1vlaier's view (2 a. 155--6)
that the reference is to possibility 'on the ground of the syllogism'
(as exhibited in "19-24) can hardly be right, since this is surely
identical with that & Te? dAT)8EVEa(}aL Kal €V Te? lnrapXELv. More
likely the phrase is a mere generality and A. had no particular
other sense of possibility in mind.
1S-19. oIov ,hav ... aUAAoYLUfLoV, 'i.e., when the premisses
are so related as was prescribed in the doctrine of the simple
syllogism' (chs. 4--6).
22-4. wcnl'Ep o~v . . . 8uvaTov, i.e. we can now apply the
4085
zCOMMENTARY
general rule stated in 3 5- 7 to the special case in which A stands
for the premisses of a syllogism and B for its conclusion.
wa1TEp o~v EL TLe; 8EL" ••• aUf1~a.LvoL av is a brachylogy for
om-wS" ovv £X*" WG7TfOp t:L Tts" (lE{T} ... GUJL{3a{voL yap ay. The usage is
recognized in Bonitz, Index, 872b29-39.
z5-bz. TOUTOU SE ••• ciSUV(1TOV. A. has shown in "1-24 that
if a certain conclusion would be true if certain premisses were
true, it is capable of being true if the premisses are capable of
being true. He now (325-33) applies that principle in this way:
The introduction into an argument of a premiss which, though
unwarranted by the data, is not impossible, cannot produce an
impossible conclusion. The fact that an impossible conclusion
follows must be due to another premiss which is impossible.
And this principle is itself in 3 34_ b 2 applied to the establishment,
by reductio ad impossibile, of the validity of the inference 'If all
B is A, and all C may be B, all C may be A '. The reductio should
run 'For if not, some C is necessarily not A. But if we add to this
the premiss All C is B (which even if false is not impossible, since
we know that all C may be Bl. we get the conclusion Some B
is not A; which is impossible, since it contradicts the datum All
B is A. And the impossibility of the conclusion must be due not
to the premiss which though unwarranted is not impossible; the
other premiss (Some C is necessarily not A) must be impossible
and our original conclusion, All C may be A, true.'
The usually accepted reading in 338 £i ovv "TC\ JL€V A JL~ £vStxeraL
Ttf r makes A. commit the elementary blunder of treating No C
can be A as the contradictory of All C can be A ; of this we cannot
suppose A. guilty, so that n must be right in reading 7TaJn"{ before
Ttf r. Two difficulties remain. (1) In 338-40 A. says that Some C
cannot be A, All C is B yields the conclusion Some B cannot be
A, while in 3Ib37-9 he says that such premisses yield only the
conclusion Some B is not A. (2) In 34"40-1 he says 'it was assumed
that all B may be A', while what was in fact assumed in 334 was
that all B is A. To remove the first difficulty Becker supposes
(p. 56) that TO A OV 7TaVTL Ttf B fl-,SEXfOTaL ("39) means not Some B
cannot be A, but It follows that some B is not A ; and to remove
the second difficulty he excises £vSExw(}aL in 341. But (a) though
avayKT} is sometimes used to indicate not an apodeictic proposition
but merely that a certain conclusion follows, and though TO A
ov 7TavTL Ttf B £vStX£TaL lJ7Tapx*"v might perhaps mean 'it follows
tha t not all B is A', I do not think TO A ov 7TavTl Ttf B £vStX£TaL
can mean this; and (b) all the external evidence in 341 is in favour
of £vStxw(}aL. It is much more likely that A., forgetting the rule339
laid down in 3Ib37---9, draws the conclusion Some B cannot be
A, and that to complete the reductio he transforms (as he is
justified in doing) the 'All B is A' of a34 into the 'It is not im-
possible that all B should be A' of a4I (a proposition which
'All B is A' entails).
Both Becker (p. 53) and Tredennick charge A. with committing
the fallacy of saying 'since (I) Some C cannot be A and (2) All
C is B cannot both be true compatibly with (3) the datum All
B is A, and (2) is compatible with (3), (I) must be incompatible
with (3) and therefore false' ; whereas in fact (I) also when taken
alone is compatible with (3), as well as (2), and it is only the
combination of (I) and (2) that is incompatible with (3) ; so that
the reductio fails. The charge is not justified. A.'s argument is
really this: 'Suppose that All B is A and All C can be B are true.
(2) is plainly compatible with both of them together and we may
suppose a case in which it is true. Now (I) and (2) plainly entail
Some B cannot be A, which is false, since it contradicts one of
the data. But (2) is in the supposed case true, therefore (I) must
be false and All C can be A must be true.' The status of (I) and
that of (2) are in fact quite different; (2) is compatible with both
the data taken together, (I) with each separately but not with
both together.
25-']. TOUTOU liE ••• a.8UVa.TOV. A. knows well (ii. 2-4) that if
a premiss is false it does not follow that the conclusion will be
false, so that o/HiV8o, in a27 cannot mean 'false'. Both in a25 and in
a27 o/H,v8o, Ka, OUK &'8Vl'aTOI' means 'unwarranted by the data but not
incompatible with them'; for the usage cf. 37a22 and Poet. I460a22.
29-30. E"lrEL ya.p ... liuva.Tov, cf. as-IS.
37. Toiho ~E ",Eulios, i.e. unwarranted by the data; cf. a25-7 n.
b2-6. [EYXWPE~ liE ••• EYXWPE~V.] This argument claims to be a
reductio ad impossibile, but is in fact nothing of the sort. A
reductio justifies the drawing of a certain conclusion from certain
premisses by supposing the contradictory of the conclusion and
showing that this, with one of the premisses, would prove the
contradictory of the other premiss. But here the original con-
clusion (For all C, A is possible) is proved by a manipUlation of
the original premisses, and from its truth the falsity of its con-
tradictory is inferred. Becker seems to me justified in saying
(p. 57) that A. could not have made this mistake, and that it
must be the work of a rather stupid glossator. AI. and P. have
the passage, but we have found other instances of glosses which
had before the time of AI. found their way into the text; cf.
24bq-I8 n.COMMENTARY
34 0
7-18. AE'L SE ••• SLOpttOVTa.S. A. points out here that if the
combination of a problematic with a universal assertoric premiss
is to produce a problematic conclusion, the assertoric premiss
must state something permanently true of a class, not merely
true of the members it happens to contain at a particular time.
He proves his point by giving instances in which a problematic
conclusion (All S may be P) drawn from a combination which
offends against this rule is untrue because in fact (a) no S can be
P or (b) all S must be P. (a) It might be true that everything
that is moving (at a particular time) is a man, and that it is
possible for every horse to be moving; but no horse can be a man.
(b) It might be true that everything that is moving (at a particular
time) is an animal, and that it is possible that every man should
be moving; but the fact (not, as A. loosely says, the aVJ.L7TEpaaJ.La)
is necessary, that every man should be an animal.
There is a flaw in the reasoning in (b). The reductio in "36-b2
only justified the inferring, from the premisses AN, of the con-
clusion AP, 110t of the conclusion Ac; for A. shows in b27 - 37 and
37"15-29 that any reductio can establish only a problematic
proposition in which 'possible' = 'not impossible', not one in
which it = 'neither impossible nor necessary', while he here
assumes that what it establishes, when the truth of the assertoric
premiss is not limited to the present moment (b IO), is a problematic
proposition of the stricter sort. Becker (p. 58) infers that b l4- q
En EaTw ... ~c;.ov is a later addition. But it is not till further on
in the chapter (b 27 - 37 ) that A. makes the point that a reductio
can only validate a problematic proposition of the looser kind,
and he could easily have written the present section without
noticing the point. Becker's suspicion (ib.) of Sta ... av)J.oytaJ.L0,
(b8-II) seems equally unjustified.
19-31. naALv EaTIIl ••• cjlaaEIIlS. A. here explains the point
made without explanation at 33bz9-31, that arguments in the first
figure with an asscrtoric major and a problematic minor, when
they prove a negative possibility, do not prove a problematic
proposition as defined in 32"18-20 (Myw S' lVSEXEO"8a, Kat TO
£VSEX0J.L£VOV, 00 J.L-ry OVTO, avaYKa{ov, U.8EVTO, S' tmaPXEtV, ovS£v
EO"Tat Sta TOOT' dSvvaTov).
The premisses No B is A, For all C, being B is contingent, are
originally stated to justify the conclusion For all C, not being A
£VSEXHat (b I9 - 22 ). This A. proves by a reductio ad impossibile
(b 22- 7): 'For suppose instead that some C is necessarily A, and
that all C is B (which is unwarranted by the data (cf. "25-7 n.)
but not impossible, since it is one of our data that it is possible34 1
for all C to be E). It follows that some E is A (InAI in the third
figure, 3Ib31-3). But it is one of our data that no Eis A. And
since All C is E is at most false, not impossible, it must be our
other premiss (Some C is necessarily A) that has led to the
impossible result. It is therefore itself impossible, and the original
conclusion 'It is possible that no C should be A' is true.
'This argument', says A. (b 27 - 31 ), 'does not prove that for all
C, not being A is '"8£xojL£vo" according to the strict defmition of
JvS£XOjLEIIOII (i.e. that which we find in 32'18-20), viz. that for all
C, not being A is neither impossible nor necessary, but only that
for no C is being A necessary (i.e. that for all C, not being A is
not impossible) ; for that is the contradictory of the assumption
made in the reductio syllogism (for that was that for some C.
being A is necessary, and what is established by the reductio is
the contradictory of this).' In other words, the reductio has
proceeded as if being impossible were the only alternative to being
EIIS£XOjL£IIOll (whereas there is another alternative-that of being
necessary), and has established only that for all C, not being A
is '"SEXOjL£"O" in the loose sense in which what is necessary may
be said to be fl,8£XOjLEIIOII, cf. 32'20-1. But in the strict sense what
is '"SEXOjL£VO" is neither impossible nor necessary, and the reductio
has not established that for all C, not being A is '"8£xojL£"o" in
this sense. Becker's excision of b19-3582 (p. 59) is unjustified.
~3. Ka.96.1TEp 1TpOTEpOV, i.e. as in a36-7.
~6. "'EUSOUC; ya.p TE9ivTOC;, cf. a25-7 n.
3I-35a~. ETL SE ... opouc;. (I) Nothing that is thinking is a
raven, For every man, to be thinking is contingent. But it is not
contingent, but necessary, that no man should be a raven. On
the other hand, (2) Being in moverr.ent belongs to no science, For
every man it is contingent that science should belong to him.
But it is not necessary, but only contingent, that being in move-
ment should belong to no man.
The second example is (as A. himself sees-A7I'T'r£Oll pIAnoll TOU!>
0pou!>, 35a2) vitiated by the ambiguity of lJ7Tapx£'" (for which see
ch. 34). But take a better example such as AI. suggests (196.
8-11). (2a) Nothing that is at rest is walking, For every animal
to be at rest is contingent. But it is not necessary, but only
contingent, that no animal should be walking. Thus, since
premisses of the same form are in case (I) compatible with its
being necessary that no C should be A, and in case (2a) with its
being contingent that no C should be A, they cannot prove either,
but only that it is not necessary that any C should be A. This
establishes, as A. says, the same point which was made in b 27 - 3 I.34 2
COMMENTARY
35'5-6. aVTLUTpaq,ELu"1'i SE ... 'lTpOTEpOV. This is the process
already stated in 32"29-bI to be justified, that of inferring from
'For e, not being B is contingent' that for e, being B is contingent.
AEc is in fact reduced to AAc. Ka(U:rrEp Ell TOL, 7rpOTEpOII refers
to the validation of AcEcAc in 33"S-12.
Io-U. WU'lTEP 'lTpOTEPOV •.• 9Eun refers to the treatment of
AAcAp in 34a34-b2.
11-20. TOV aUTov SE TPO'ITOV . . . uuA""0YLUj.LO'i. EEc is here
similarly reduced to EAc, for which see 34bI9-3Sa2.
12. aj.Lq,oTEpWV TWV SLaUT"1j.LaTWV. A. not infrequently uses
8u1.aT'7fLa of a syllogistic premiss; the usage is probably connected
with a diagrammatic representation of the syllogism.
20-4. eav SE ... 'lTLTTa. A. here reverts to the case in which
the major premiss is problematic, the minor assertoric, and dis-
cusses the combinations omitted in 33b33-40, viz. those in which
the minor premiss is negative (both premisses being, as throughout
33b2S-3S"30, supposed to be universal), viz. AcE and EcE. He
offers no general proof of the invalidity of these moods, but shows
their invalidity by instances. The invalidity of AcE is shown by
the fact that (a) while it is contingent for all animals to be white,
and no snow is an animal, in fact all snow is necessarily white,
but (b) while it is contingent for all animals to be white, and no
pitch is an animal, in fact all pitch is necessarily not white. Thus
premisses of this form cannot entail either It is possible that no
e should be A or It is possible for all e to be A.
The invalidity of EcE is shown by the fact that (a) while it is
contingent for all animals not to be white, and no snow is an
animal, in fact all snow is necessarily white, but (b) while it is
contingent for all animals not to be white, and no pitch is an
animal, in fact all pitch is necessarily not white. Thus premisses
of this form cannot entail either It is possible that no e should be
A or It is possible for all e to be A.
20-1. eav SE ... lJ'ITapXELV, 'if the minor premiss is that B
belongs to no e, not that B is capable of belonging to no C.
28-30. 'lTA"v oTE j.LEv . . • Etp~Kaj.LEv. AAcAp (34"34-b2) and
EAcEp (34bI9-3Sa2) have been proved by reductio ad impossibile,
AEcAp (3Sa3-1I) and EEcEp (ib. 1I-20) by converting 'For all
e, not being B is contingent' into 'For all e, being B is contingent'
(rillnaTpa</>Et<TT}' T7j, 7rpOTaaEw,). Eg atlTwII therefore is not meant
to exclude the use of reductio, but only to exclude the comple-
mentary conversion of problematic propositions; it does not
amount to saying that the proofs are n'AHOt.
34-5. Ka9a'ITEp Kai. Ka9o,,"ou ... 'lTpOTEPOV, cf. 33b33-6, 36-40.343
4o-bZ. 1TXT]V Ol ~ ••• 1TpOTEpOV. What A. means is that
Alelp and IIeOp are proved by reductio, as were AAeAp(34"34-b6}
and EAeEp (34bI9-3I), and that AOelp and EOeOp are proved by
complementary conversion (reducing AOe to Ale, and EOe to
Ele) followed by reductio. The vUlgate reading omits Kat in bI,
but C has Kat, which is also conjectured by P.
Alelp may be validated by a reductio to EnIOn in the third
figure, EleOp by one to A nIl n in that figure.
bI. KaBa.1TEp EV TOl5 1TpOTEPOV, se. as in "3-20.
z-S. EUTaL SE ••• uuXXoYLuflo5. The combinations AOe, EOe,
which are here dealt with, are the two, out of the four enumerated
in &35-40, which need complementary conversion, so that Kat
in b 2 is puzzling. Waitz thinks that A. meant to say KaL (both)
o7'aIJ ... 7'() li1TapXlLIJ, ~ S' JIJ J-L£PlL ..• >"aJ-Lf3aIJIJ, KaL 'JTaIJ ~ Ka86>..ov
TTPO<; 7'0 J-LEi{oIJ aKpOIJ 7'0 J-L~ li1TaPXlLIJ, ~ S' JIJ J-L£PlL i] au7'~, but (for-
getting the original Kat) telescoped this into the form we have.
This is possible, but it seems preferable to omit Kat.
4. i1 flT] U1Ta.PXELV seems to be the work of the same interpolator
who has inserted the same words in b 23 and elsewhere. li1TapXlLIJ
is used of all assertoric, as opposed to problematic, propositions.
~. (lTav SE ••• uuXXoYLufl05. This formula, 'when the parti-
cular premiss is a negative assertoric', would strictly cover the
combinations AeO, EeO, OA<, ~Ee. But in b l I - I4 A. proceeds
to speak generally of the cases in which the major premiss is
particular and the minor universal, among which OAe and OEe
are of course included. We must therefore suppose him to be here
speaking only of AeO and EeO; i.e. we must suppose the condition
(iTaIJ Ka86>..ov fJ 7'0 TTPO<; 7'0 J-LEi{oIJ aKpOIJ ("35--6, cf. b 3) still to govern
the present passage.
erII. OpOL ••• ci1TOSEL€LV. Su], yap TOU ciSLOPLUTOU XTj1TTEOV TT]V
O,1TOSEL€LV is to be understood by reference to 26bI4-2I, where A.
applies the method of refutation Std 7'OV doptu7'oV to the combina-
tions AO and EO in the first figure. The combinations to be
examined here are For all B, being A is contingent, Some C is
not B, and For all B, not being A is contingent, Some C is not
B. A. is to prove that these yield no conclusion, Std 7'(VIJ OPWIJ, i.e.
by pointing to a case in which premisses of this form are com-
patible with its being in fact impossible that any C should not
be A, and a case in which they are compatible with its being
in fact impossible that any C should be A. Take, for example, the
proof that A<O yields no conclusion. (a) For all animals, being
white is contingent, and some snow is not an animal. But it is
impossible that any snow should not be white. (b) For all animals,COMMENTARY
being white is contingent, and some pitch is not an animal.
And it is impossible that any pitch should be white. Therefore
AeO does not justify the statement either of a negative or of an
affirmative possibility. But it occurs to A. that the three pro-
positions in (b) cannot all be true if Some C is not B is taken to
imply (as it usually does in ordinary speech) that some C is B.
He therefore points out that the form Some C is not B is dS~6p~­
ClTOV, assertible when no C is B as well as when some C is Band
some is not.
15-16. EtT' EvSEXEaOuL ••• EVUXXo.€. A. goes beyond the sub-
ject of the chapter to point out that not only when one premiss
is assertoric and the other problematic (£VaAA&~), but also when
both are problematic or both are assertoric, two particular
premisses prove nothing.
17-19. Q.1TOSEL€LS S' ••• L .... UTLOV. The same examples will
serve to show the invalidity of all the combinations referred to in
b I I- I4 and 14-17. Take, for example, leA. It might be the case
that for some white things, being animals is contingent, and that
all men are in fact white; and all men are necessarily animals. On
the other hand, it might be the case that for some white things,
being animals is contingent, and that all garments are in fact
white; but necessarily no garments are animals. Therefore pre-
misses of the form leA cannot prove either a negative or a positive
possibility.
Q.1TOSEL€LS S' ••• 1TPOTEPOV refers to 33a34-b8, which dealt with
the corresponding combinations with both premisses problematic,
and used the same examples.
20-2. CPUVEpOV O~V ••• OUSEVOS. A. here sums up the results
arrived at in a30-b14 with regard to combinations of one universal
and one particular premiss. The statement is not quite accurate,
for he has in b8-I1 pointed out that the combinations AeO, EeO
prove nothing.
344
CHAPTER 16
Syllogisms in the first figure with one problematic and one apodcictic
premiss
35b23' 'When one premiss is necessary, one problematic, the
same combinations will yield a syllogism, and it will be perfect
when the minor premiss is necessary (AeAn, EcAn, Aeln, Eeln);
when the premisses are affirmative (AeAn, Aeln, AnAe, Anle), the
conclusion will be problematic; but if they differ in quality, then
when the affirmative premiss is necessary (EeAn, Eeln, AnEe,345
AnOe) the conclusion will be problematic, but when the negative
premiss is necessary (EnAe, EnIe) both a problematic and an
assertoric conclusion can be drawn; the possibility stated in the
conclusion must be interpreted in the same way as in the previous
chapter. A conclusion of the form 'C is necessarily not A' can
never be drawn.
37.
(A) Both premisses universal
(a) If both premisses are affirmative the conclusion is not
apodeictic. AnA e gives the conclusion Ap by an im perfect syllogism.
36°2. AeAn gives the conclusion Ae by a perfect syllogism.
7. (b) Major premiss negative, minor affirmative. EnAe gives
the conclusion E by reductio ad impossibile.
IS. A fortiori it gives the conclusion Ep.
17. EeAn gives, by a perfect syllogism, the conclusion Ee, not
E; for Ee is the form of the major premiss, and no proof of E by
reductio ad impossibile is possible.
25. (c) Major premiss affirmative, minor negative. AnEe gives
the conclusion Ap by transition from Ee to Ae.
27. AeEn gives no conclusion; this shown by contrasted
instances.
:z8. (d) Both premisses negative. EeEn gives no conclusion;
this shown by contrasted instances.
32.
(B) One premiss particular
(a) (Major premiss universal.)
(a) Premisses differing in
quality. (i) (Universal) negative premiss necessary. EnIc gives
the conclusion 0, by reductio ad impossibile.
39. (ii) Particular affirmative premiss necessary. EeIn gives
only a problematic conclusion (Oe).
40. (f3) Both premisses affIrmative. When the universal pre-
miss is necessary (AnIe), there is only a problematic conclusion (Ip).
b 3 . (b) Minor premiss universal. (a) When the universal pre-
miss is problematic (I nEe, OnEe, InAe, OnN), nothing follows;
this shown by contrasted instances.
7. (f3) When the universal premiss is necessary (IcEn, OeEn,
leAn, OeAn), nothing follows; this shown by contrasted instances.
12.
(C) Both premisses particular
When both premisses are particular nothing follows; this shown
by contrasted instances.
19. Thus it makes no difference to the validity of a syllogism
whether the non-problematic premiss is assertoric or apodeictic,COMMENTARY
except that if the negative premiss is assertoric the conclusion is
problematic, while if the negative premiss is apodeictic both a
problematic and an assertoric conclusion follow.
In this chapter A. lays it down (35b23-36) that if, in the com-
binations of an assertoric with a problematic premiss discussed
in ch. IS, an apodeictic premiss be substituted for an assertoric,
the validity of the argument will not be affected, but, if anything,
only the nature of the conclusion. The combinations recognized
in ch. IS as valid are AeA, EeA, AAe, EAe, AEe, EEc, AeI, Eel,
Ale, Ele, AOe, EOe. Of the combinations got by substituting an
apodeictic for an assertoric premiss, Nln is omitted in the subse-
quent discussion, but what A. says of AeAn (3632-7) 'would
mutatis mutandis apply to it. AnOe and EnOe are omitted, but
are respectively reducible to Anle and Enle (for which v. 36a4o--b2,
&34---<)) by conversion from For some C, not being B is contingent
to For some C, being B is contingent (32a29-bI).
3sb32-4. TO S' EVSEXE0'9aL ••• 1TpOTEpOV, i.e. where a syllogism
is said (as in b30- 2) to prove both a problematic and an assertoric
conclusion, the former is not problematic in the strict sense
defined in 32318-20 (where All C admits of not being A means
It is neither impossible nor necessary that no C should be A), but
only in the wider sense stated in 33b30-I, 34b27--8 (where it means
It is not impossible that no C should be A). That is because the
conclusion is simply inferred a fortiori from the main conclusion
that no C is A (36315-17). Ct. 34bI9-3I n.
3631-2. TOV aUTov yap TP011'OV ••• 11'POTEPOV, i.e. the conclu-
sion Ap from AnN will be proved by a reductio ad impossibile, as
the conclusion from AAe was in 34a34-b2. The reductio of AnAeAp
will be in OnAO in the third figure.
7-17. Et SE ••. U1To.PXELV. A. shows here that from the pre-
misses EnAe in the first figure (I) E follows by a reductio ad
impossibile using EnIOn in the first figure (3 0bl - 2), and (2) Ep
follows a fortiori.
8-c). KaL TO I-LEv A ... T~ B. ABd have ES allaYK'I» after 'Tip B,
but AI. had not these words in his text, and their introduction
is almost certainly due to his using them in his interpretation
(208. II-I2). He introduces them by way of pointing out that
'TO A fL'I)O£lIL ElIo£x'a8w 'Tip B here means 'let it not be possible for
any B to be A', not 'let it be possible that no B should be A';
but that is made sufficiently clear by the words £a-rw 7TPW'TOll ~
aT£p'I)nK~ UlIUYKUtU in as. The combination fL'I)O£lIL ElIO'xw8uL
Et allaYK'I» would, I think, be unparalleled in A.347
10. avaYKTJ 8TJ •.. ulI'apxuv. This is not meant to be a neces-
sary proposition, but to express the necessary sequence of the
assertoric proposition No C is A from the premisses.
IG-15. KELa9w ya.p . . . apxij.... The words of which Becker
(p. 44) expresses suspicion are (as he points out) correct, though
unnecessary, and may be retained.
18. Kat TO ",Ev A ... ulI'apXEw. The difference must be noted
between TO A iv8Exla8w J.L7]8£1't T0 B ImaPXHv, 'let it be pos-
sible for A to belong to no B', and 334 £l TO J.L£V A J.L7]80" T0 B
iv81X!Tat lnrapXHV, 'if it is impossible for A to belong to any B'.
2G-4. aAA' Oll ••• ci8uvaTov. A. gives two reasons why the
conclusion from 'For all B, not being A is contingent, It is neces-
sary that all Cbe B' is 'For all C, not being A is contingent', not
'No C is A '. The first is that the major premiss is only problematic.
The second is that the conclusion No C is A could not be proved
by reductio ad impossibile, since (so the argument must continue)
if we assume its opposite Some C is A, and take with this the
original major premiss, we get the combination 'For all B, not
being A is contingent, Some C is A', from which we cannot infer
the contradictory of the original minor premiss, viz. It is possible
that some C should not be B. This follows from the general
principle stated in 37b19-22, that in the second figure an affirma-
tive assertoric and a negative problematic premiss prove nothing.
Thus in 323 nv{ must be right. The MSS. of AI. record nv, J.L~
as a variant (210.32), but AI.'s commentary (ib. 32-4) shows that
the variant he recognized was nvt. J.L7]8EV{, the reading he accepts
(2IO. 2I-30), is indefensible.
26. 8~a Tij ... aVTLaTpocpij ... , i.e. by the conversion of For all C,
not being B is contingent into For all C, being B is contingent;
cf.3 232 9- b r.
27. Ko.9all'Ep EV TOl", lI'pOTEpOV, i.e. as with the corresponding
mood (AEcAp) treated of in the last chapter (3S33-II).
28-31. ou8' ,hay ••• lI'LTTo.. It is implied that when both pre-
misses are negative and the minor is problematic (EnEc), a con-
clusion can be drawn, viz. by the complementary conversion of
EnEc into EnAc, which combination we have seen to be valid
('7-17)·
29-31. opo~ 8' ..• 1I'LTTo.. For all animals, being white, and
not being white, are contingent, it is necessary that no snow
should be an animal, and in fact it is necessary that all snow
should be white. On the other hand, for all animals, being white,
and not being white, are contingent, it is necessary that no pitch
should be an animal, but in fact it is necessary that no pitchCOMMENTARY
should be white. Thus AcEn and EcEn in the first figure prove
nothing.
32-bI2. Tov a.UTOV SE TPOll'OV ••• XLWV. A. now proceeds to
consider cases in which the premisses differ in quantity. b 3- 12
expressly considers those in which the minor premiss is universal,
so that 833-b2 must be concerned only with those in which the
major premiss is universal. Further, the statement in 833-4 must
be limited to the case in which it is the universal premiss that is a
negative apodeictic proposition.
When A. says (b 7- 12 ) that when the universal premiss is
apodeictic and the particular premiss problematic, nothing
follows, he seems to be condemning inter alia EnIc, EnOc, AnIc,
AnOe, which are valid; but he will be acquitted of this mistake
if we take the condition 'if the minor premiss is universal' to be
carried over from b 3 - 4 .
32. Tov a.UTOV SE TPOll'OV ••• C7u~~oyu"flwv. This follows from
the fact that if in a valid first-figure syllogism we substitute a
particular minor premiss for a universal one, we get a particular
conclusion in place of the original universal conclusion.
34-9' otov £t ••. ~V8EX£C79a.L. EnIcO is proved by a reductio in
EnAEn in the first figure. A. omits to add that OP follows a
fortiori (cf. 3Sb30-2). d"&YK'I) means 'it follows', as in "10 (where
see n.).
34. £t TO flEV A ... il1l'a.pX£LV. Cf. "18 n.
bl _2 • OUK ~C7Ta.L ••• C7U~~OYLC7flo", i.e. the conclusion will be
problematic.
2. a.1TOS£L~L" 8' ••• 1Tponpov. This must mean that EcInOe is
a perfect syllogism as was EcAnEe ("17-2S), and that AcIeIp is
proved by a reductio as was AnAeAp (3Sb38-36a2). The reductio
of AnIcIp will be effected in AnEnEn in the second figure.
S-7. C;pOL SE ••• tfla.TLOV. I.e. it is necessary that some white
things should and that others should not be animals; for all men,
being white, and not being white, are contingent; and in fact
all men are necessarily animals. On the other hand, it is necessary
that some white things should and that others should not be
animals; for all garments, being white, and not being white, are
contingent; but it is necessary that no garment be an animal.
Thus in the first figure InAc, InEe, OnAe, OnEc prove nothing.
8-12. C7T£PTJTLKOU flEV ••• XLWV. I.e. it is contingent that some
white things should be, and that they should not be, animals; it
is necessary that no raven be white; and every raven is necessarily
an animal. On the other hand, it is contingent that some white
things should be, and that they should not be, animals; it is349
necessary that no pitch be white; but necessarily no pitch is
an animal. Thus leEn, OeEn in the first figure prove nothing.
Again, it is contingent that some white things should be, and
that they should not be, animals; every swan is necessarily
white; and every swan is necessarily an animal. On the other
hand, it is contingent that some white things should be, and that
they should not be, animals; all snow is necessarily white; but
necessarily no snow is an animal. Thus leAn, OcAn in the first
figure prove nothing.
I:Z-x8. o{,S' OTa.V ... 0pOl.
A
(Major) Some white things are necessarily animals, some neces-
sarily not.
(Minor) Some men are necessarily white, some necessarily not.
(Minor) Some lifeless things are necessarily white, some
necessarily not.
B
(Major) For some white things, being animals is contingent; for
some white things, not being animals is contingent.
(Minor) For some men, being white is contingent; for some men,
not being white is contingent.
(Minor) For some lifeless things, being white is contingent; for
some lifeless things, not being white is contingent.
Combining a major from A with a minor from B or vice versa:
we can get true propositions illustrating all the possible combina-
tions of an apodeictic with a problematic proposition, both parti-
cular, in the first figure. That such premisses do not warrant a
negative conclusion is shown by the fact that all men are neces-
sarily animals; that they do not warrant an affirmative con-
clusion, by the fact that all lifeless things are necessarily not
animals.
19-24. cS>a.V(POV o~v ... ll1ra.px(lV. I.e. the valid combinations
of a problematic with an apodeictic premiss are the same, in
respect of quality and quantity, as the valid combinations of a
problematic with an assertoric (for which v. ch. IS). The only
difference is that where a negative premiss is assertoric (i.e. in
the combinations EAc, EEc, Elc, EOe) the conclusion is problema-
tic, and where a negative premiss is apodeictic (i.e. in the com-
binations EnAc, EnEc, Enlc, EnOc) both a problematic and an
assertoric conclusion follow. A. says 'the negative premiss', not
'a negative premiss', though in some of the combinations both35 0
COMMENTARY
premisses are negative. This is because in these cases the other
premiss, being problematic, is in truth no more negative than it is
affirmative, since For all C, not being B is contingent is convertible
with For all C, being B is contingent (32329-bl).
:Z4-5. SfjAOV SE .• . aXTJf-La.TwV. This sentence is quite indefen-
sible. A. has said in 33b2S-7 that in the first figure valid com-
binations of a problematic major and an assertoric minor yield
a perfect (i.e. self-evidencing) syllogism, and has pointed this
out in dealing with the several cases (AeA, EcA, AcI, Eel). In
3Sb23-6 he has said the same about the valid combinations of a
problematic major with an apodeictic minor, and has pointed
this out in dealing with the cases AcAn, EcAn, Ecln (Acln is not
expressly mentioned). He could not possibly have summed up
his results by saying that all the valid syllogisms are imperfect.
Some unintelligent scribe has lifted the sentence bodily from
39"1-3, his motive no doubt being to have at the end of the treat-
ment of the modal syllogism in the first figure a remark corre-
sponding to what A. says at the end of his treatment of modal
syllogism in the other two figures (3931-3, 4obIS-16).
CHAPTER 17
Syllogisms in the second figure with two problematic premisses
36b:z6. In the second figure, two problematic premisses prove
nothing. An assertoric and a problematic premiss prove nothing
when the affirmative premiss is assertoric; they do prove some-
thing when the negative, universal premiss is assertoric. So too
when there are an apodeictic and a problematic premiss. In these
cases, too, the conclusion states only possibility in the loose sense,
not contingency.
35. We must first show that a negative problematic proposition
is not convertible. If for all B not being A is contingent, it does
not follow that for all A not being B is contingent. For (I) sup-
pose this to be the case, then by complementary conversion it
follows that for all A being B is contingent. But this is false; for
if for all B being A is contingent, it does not follow that for all A
being B is contingent.
37 3 4. (2) It may be contingent for all B not to be A, and yet
necessary that some A be not B. It is contingent for every man
not to be white, but it is not contingent that no white thing should
be a man; for many white things cannot be men, and what is
necessary is not contingent.35 1
9. (3) Nor can the converse be proved by reductio ad impossibile.
Suppose we said 'let it be false that it is contingent for all A not
to be B; then it is not possible for no A to be B. Then some A
must necessarily be B, and therefore some B necessarily A. But
this is impossible.'
14. The reasoning is false. If it is not contingent for no A to
be B, it does not follow that some A is necessarily B. For we can
say 'it is not contingent that no A should be B', (a) if some A is
necessarily B, or (b) if some A is necessarily not B; for that which
necessarily does not belong to some A cannot be said to be
capable of not belonging to all A; just as that which necessarily
belongs to some A cannot be said to be capable of belonging to
all A.
20. Thus it is false to assume that since C is not contingent
for all D, there is necessarily some D to which it does not
belong; it may belong to all D 1.nd it may be because it
belongs necessarily to some, that we say it is not contingent for all.
Thus to being contingent for all, we must oppose not 'necessarily
belonging to some' but 'necessarily not belonging to some'. So
too with being capable of belonging to none.
29. Thus the attempted reductio does not lead to anything
impossible. So it is clear that the negative problematic proposi-
tion is not convertible.
32. Now assume that A is capable of belonging to no B, and to
all C (EeAe). We cannot form a syllogism (r) by conversion (as we
have seen) ; nor (2) by reductio ad impossibile. For nothing false
follows from the assumption that B is not capable of not belonging
to all C; for A might be capable both of belonging to all C and of
belonging to no C.
38. (3) If there were a conclusion, it must be problematic, since
neither premiss is assertoric. Now (a) if it is supposed to be
affirmative, we can show by examples that sometimes B is not
capable of belonging to C. (b) If it is supposed to be negative, we
can show that sometimes it is not contingent, but necessary, that
no C should be B.
b 3 . For (a) let A be white, B man, C horse. A is capable of
belonging to all C and to no B, but B is not capable of belonging
to C; for no horse is a man. (b) Nor is it capable of not belonging;
for it is necessary that no horse be a man, and the necessary is not
contingent. Therefore there is no syllogism.
10. Similarly if the minor premiss is negative (AcEe), or if the
premisses are alike in quality (AeAe, EeEe) , or if they differ in
quantity (Aele, AeOe, Eele, leAe, IeEe, OeAe, OeEe), or if both are35 2
COM~lE~T.\RY
particular or indefinite (Iele, leOe, Oele, OeOe) ; the same contrasted
instances will serve to show this.
16. Thus two problematic premisses prove nothing.
36b26-33. 'Ev 8£ T~ 8€UTEP,!:, ••• 1fPOTcl.I7EWV. These statements
are borne out by the detailed treatment in chs. 17-19, except
for the fact that IcE, OeE, ICE", OeE" prove nothing. These
are obviously condemned by their breach of the rule that in the
second figure the major premiss must be universal (to avoid
illicit major).
33-4. 8EL 8E ••• 1fPOTEPOV, i.e. the problematic conclusion
must be interpreted not as stating a possibility in the strict sense,
something that is neither impossible nor necessary (32"18-20), but
a possibility in the sense of something not impossible (33bz9-33,
34bz7-3I). This follows from the fact that problematic conclu-
sions in the second figure are validated by reductio ad impossibile;
for the reductio treats being impossible as if it were the only
alternative to being lJ,'oEX6fLfYOV, while in fact there is another
alternative, viz. being necessary.
37-37"3. KEL179w ya.p ••• I7TEpTJTLKOV. (I) For all B, being A is
contingent entails (2) For all B, not being A is contingent; (3)
For all A, not being B is contingent entails (4) For all A, being B
is contingent. Therefore if (2) entailed (3), (I) would entail (4),
which it plainly does not.
39-40. Kat at EvaVTLaL ••• &.VTLKEL .... EvaL. The precise meaning
of this is that Ee is inferrible from Ae and vice versa, and Oc
from le and vice versa, and Oe from Ae, and le from Ee. Ae is
not inferrible from Oe, nor Ee from le. Ct. 32329-35 n. Ae and
Ee are bav-daL; Ae and Oe, and again Ee and le, o.VTLKdfLEvaL.
le and Oe are probably reckoned among the £vav-duL, as I and 0
are in 59bIo-though in 63b23-30 they are included among the
o.VTLKEtfLEvaL (though only KaTd. TT/V ugw o.VTLKE{fLfYaL).
37 3 8-9. TO 8' &'vaYKa'Lov ••• Ev8EXO .... EVOV, cf. 32'18-20.
9-31. ' AAAa. .... "v ••• I7TEpTJTLKOV. The attempted proof, by
reductio ad impossibile, that if for all B, not being A is contingent,
then for all A, not being B is contingent (36b36--7) ends at
douvaTov (37"14), and A.'s refutation begins with OU yap. The
punctuation has been altered accordingly (Bekker and Waitz
have a full stop after TWV B and a colon after o.ouvaTov, in ,'14).
The attempt to prove by reductio ad impossibile that TO A
£vo£XETaL fL7)OEVt TcfJ B ima.pXHv en tails TO B £VO£XETaL fL7)OfYt Tip A
inni.pXHv goes as follows: Suppose the latter proposition false
('10). Then (X) TO B OUK £voiXETaL fL7)OEVL TcfJ A inrapXELv. ThenL 17. 36b26-37"37
353
(Y) it is necessary for B to belong to some A. Then (Z) it is
necessary for A to belong to some B. But ex hypothesi it is
possible for A to belong to no B. Therefore it must be possible
for B to belong to no A.
A.'s criticism in "14-31 is as follows: The step from (X) to (Y)
is unsound. 'It is necessary for B to belong to some A' is not
the only alternative to TO B JvoiX€Ta, f.1.7)O€II~ Tip A lnrapXHII. There
is also the alternative 'It is necessary for B not to belong to some
A'. Necessity, not only the necessity that some A be B, but
equally the necessity that some A be not B, is incompatible with
TO B JvoiX€Ta, f.1.7)O€lIt TC)J B inrapXHII. That is the strict meaning
of £VO£X€Tat-not 'not impossible' but 'neither impossible nor
necessary' (32"18-21). The proper inference, then, in place of
(Y), is 'Either it is necessary for B to belong to some A or it is
necessary for B not to belong to some A'. And from the second
alternative no impossible conclusion follows, so that the proof per
impossibile fails.
ZZ. 1TavTL yelp U1TclPXEL. The correct sense is given by n's
addition £l TUxO', 'there may be cases in which C belongs to all D'.
We should not read £l TVXOt, however, because it is missing both
in AI. (225. 31) and in P. (213. 27-8).
z8. flU TO E~ 6.vclYKTJ~ .•• 6.vclyKTJ~' Waitz's reading 0(; f.1.0vov
(so the MSS. Bdn) TO £g allaYK7)S" ..• ilia. Kat TO £g allaYK7)S" KT).
(BCdn) is supported by P. 214. 15-17, but not by AI. (226. 16-19,
27-30). The fuller reading seems to be an attempt to make things
easier. Not either alternative nor both, but the disjunction of the
two, is the proper inference from (X) (see "9-31 n.) ; but in answer
to the opponent's assumption of (Y) we must make the counter-
assumption It is necessary for B not to belong to some A; and
by pointing out this alternative we can defeat his argument.
34. EipTJTCU yelp ••• 1TpOTal7L!;, in 36b3s-37'31. ~ To,avT7) TTPO-
TacnS", i.e. such a premiss as For all B, not being A is contingent.
35-'7. 6.>'>" ouSE ••. U1Ta.PXELV. What A. says, according to the
traditional reading, is this: Nor again can the inference 'For all
B, not being A is contingent, For all C, being A is contingent,
Therefore for all C, not being B is contingent' be established by
a reductio ad impossibile. For if we assume that for all C, being
B is contingent, and reason as follows: 'For all B, not being A
is contingent, For all C, being B is contingent, Therefore for all
C, not being A is contingent', we get no false result, since our
conclusion is compatible with the original minor premiss.
There is a clear fallacy in this argument. It takes 'For all C.
being B is contingent' as the contradictory of 'For all C, not being
'1985
Aa354
COMMENTARY
B is contingent', in the same breath in which it points out that
'For all C, not being A is contingent' is compatible with 'For all
C, being A is contingent'. A. cannot really be supposed to have
reasoned like this; l\1aier's emendation (2 a. 179 n.) is justified.
The argument then runs: Suppose that we attempt to justify
the original conclusion 'For all C, not being B is contingent', by
assuming its opposite, 'For some C, not being B is not contingent',
and interpret this as meaning 'For some C, being B is necessary'
and combine with it the original premiss 'For all B, not being A
is contingent'. The only conclusion we could get is 'For some
C, not being A is contingent'. But so far is this from contradicting
the original minor premiss 'For all C, being A is contingent', that
the latter is compatible even with 'For all C, not being A is
con tingen t' .
AI. and P. have the traditional reading, and try in vain to
make sense of it. As Maier remarks, the corruption may be due
to a copyist, misled by "37, having thought that A. meant to
deduce as the conclusion of the reductio syllogism 'For all C,
not being A is contingent', and struck out the two jL~'S in order
to get a premiss that would lead to this conclusion. CL a similar
corruption in 2SbS.
b9-IO. TO S' a.va.YKa.iov ••• iVSEXOI1EVOV, cf. 32336.
1I. Ka.l ClV ••. O'TEP"1TLKOV, i.e. if the premisses are For all B,
being A is contingent, For all C, not being A is contingent.
12-13. SLa yap TWV a.UTWV opWV ••• a.1TOSEL~LS, i.e. we may use
the terms used in b 3- 10• For all men, being white, and not being
white, are contingent; for all horses, being white, and not being
white, are contingent; but it is necessary that no horse should be
a man.
15-16. a.El yap ••. CI.1TOSEL~LS, i.e. for all men, and for some
men, being white, and not being white, are contingent; for all
horses, and for some horses, being white, and not being white, are
contingent; but it is necessary that no horse should be a man.
CHAPTER 18
Syllogisms in the second figure with one problematic and one assertoric
premiss
37b19.
(A) Both premisses universal
(a) An assertoric affirmative and a problematic negative (AEc,
EcA) prove nothing; this shown by contrasted instances.
23. (b) Assertoric negative, problematic affirmative, EAcEp
valid, by conversion.355
29. AeEEp valid, by conversion.
29. (c) Two negative premisses give a problematic conclusion
(EEeEp and EeEEp), by transition from Ee to Ae.
35. (d) Two affirmative premisses (AAe, AeA) prove nothing;
this shown by contrasted instances.
(B) One premiss particular
(a) Premisses differing in quality. (a) When the affirmative
premiss is assertoric (AOe, OeA, lEe, Eel), nothing follows; this
shown by contrasted intances.
38"3. (fJ) When the negative premiss is assertoric (and is
universal, and is the major premiss) (Ele), OP follows by conversion.
4. (b) (a) When both premisses are negative and the assertoric
premiss is universal (and is the major premiss) (EOe), OP follows
by transition from Oe to le.
8. When (a) (y) the negative premiss, or (b) (fJ) one of two
negative premisses, is a particular assertoric (AeO, OAe, EeO,
OEe), nothing follows.
39·
10.
(C) Both premisses particular
When both premisses are particular nothing follows; this shown
by contrasted instances.
37bl~38. Et S· . . . av9pw1Tos. The combinations in which
one or both premisses are particular being dealt with in the next
paragraph, the present paragraph must be taken to refer to
combinations of two universal premisses (though there is an
incidental reference to the others in b zz ). It will be seen from the
summary above that all of these are dealt with. The generaliza-
tion that an affirmative assertoric and a negative problematic
premiss prove nothing (bIg---ZZ ) is true, whatever the quantity of
the premisses; but the statement that an affirmative problematic
and a negative assertoric give a conclusion (b z3 - 4 ) is true without
exception only when both premisses are universal.
22-3. ci1TOSU~~S S· ••• 8pwv. If for simplicity we confine our-
selves to the case in which both premisses are universal (for the
same argument applies to that in which one is particular), the
combinations to be proved invalid are All B is A, For all C, not
being A is contingent, and For all B, not being A is contingent,
All C is A. Let us take the first of these. The invalidity of the
combination can be shown by the use of the same terms that
were used in b 3- IO . It might be true that all men are white, and
that for all horses not being white is contingent; but it is not trueCOMMENTARY
35 6
either that for all horses being men is contingent, or that for all
horses not being men is contingent: they are necessarily not men.
Thus from premisses of this form neither an affirmative nor a
negative contingency follows.
z3-8. OT<1V S' ••• C7Xtl!L<1TOS. 'No B is A, For all C, being A is
contingent, Therefore it is not impossible that no C should be B'
is validated by conversion to 'No A is B, For all C, being A is
contingent, Therefore it is not impossible that no C should be B'
(34bI 9-3 I ).
z9. O!L0LWS SE ••• C7TEP"1TLICOV.
'For all B, being A is con-
tingent, No C is A' is converted into 'No A is C, For all B, being
A is contingent', from which it follows (34bI9-3I) that it is not
impossible that no B should be C; from which it follows that it is
not impossible that no C should be B. Maier argues (2 a. 180-1)
that A.'s admission of this mood is a mistake, on the ground that
(on A.'s principle, stated in 36b3S-37a3I) iv8(X£T<1' 'TO r JL1)8(v~ 'TijJ
B v-rraPXHv does not entail ivlUX(T<1' 'TO B JL1)O(V~ 'TijJ r v-rrapXHv. But
that principle applies (as the argument ill 36b3S-37a3I shows)
only when ivo(xoJL(Vov is used in its strict sense of 'neither im-
possible nor necessary', not when it is used in its loose sense of
'not impossible' (cf. 2Sa37-bI9 n.).
zc)-3S. ~a.v S' • • • axl1!L<1, i.e. EEc or EcE proves nothing
directly (as two negative premisses never do, in any figure), but
by the complementary conversion proper to problematic pro-
positions (32a29-bI) we can reduce EEc (to take that example)
to 'No A is B, For all C, being B is contingent', and then by
simple conversion of the major premiss get a first-figure argument
which is valid. -mu\w in b3S = 'as in b24-8'.
31. ~VSEX(ae<1L, sc. JL~ v-rrapXHV. B actually has these words,
but it is more likely that they were added in B by way of inter-
pretation than that they were accidentally omitted in the other
MSS.
3&-8. OpOL • • • a.v9pwlI'os. I.e. 'For every animal, being
healthy is contingent, Every man is healthy' is compatible with
its being necessary that every man should be an animal. On the
other hand, 'For every horse, being healthy is contingent, Every
man is healthy' is compatible with its being necessary that no
man should be a horse.
Again 'Every animal is healthy, For every man, being healthy
is contingent' is compatible with every man's being necessarily
an animal. On the other hand, 'Every horse is healthy, For
every man, being healthy is contingent' is compatible with its
being necessary that 1W man should be a horse.357
Thus AeA and AAe in the second figure prove nothing.
38al-2. TO~iTO li' .. .'1TpOTEpOV. This refers to the examples in
37b36-8. Take for instance Eel. For all animals, not being healthy
is contingent, some men are healthy, and every man is necessarily
an animal. On the other hand, for all horses not being healthy is
contingent, some men are healthy, but every man is necessarily
not a horse.
Again, take AOe. 'Every animal is healthy' and 'For some men,
not being healthy is contingent' are compatible with its being
necessary that every man should be an animal. On the other
hand, 'Every horse is healthy' and 'For some men, not being
healthy is contingent' are compatible with its being necessary
that no man should be a horse.
3-'7. ilTav li~ ... auAAoYLalloS. These two statements are too
widely expressed. The first would include AeO, Ele, IcE, OAe;
but in view of what A. says in a8-10 he is evidently thinking only
of the cases in which the negative premiss is a universal assertoric
proposition (which excludes AeO, OAe). Further, IcE, which
prima facie comes under this rule, and OeE, which prima facie
comes under the next, are in fact invalid because in the second
figure the major premiss must be universal, to avoid illicit major.
In both rules A. must be assuming the universal assertoric premiss
to be the major premiss.
3-4' ilTav 8~ ••• 1TpOTEpOV. 'No B is A, For some C, being A
is contingent, Therefore for some C, not being B is possible' is
validated by conversion to 'No A is B, For some C, being A is
contingent, Therefore for some C, not being B is possible'
(3Sa35-bI). Ka()a'TT€p £V 'TOL, 'TTPO'Tf;POV, i.e. as EAeEp in the second
figure was validated by conversion to EAeEp in the first (37b24-8).
6--;. clVna'Tpacpmos 8~ ••• 1TpOTEPOV, Le. as prescribed in
37 b3 2-3·
II-I2.
ci1T08EL~LS li' ... opwv. The reference is probably to the
proof by means of opo, in 37b36-8. Take e.g. lIe. Some animals
are healthy, for some men being healthy is contingent, and all
men are necessarily animals. On the other hand, some horses
are healthy, for some men being healthy is contingent, but
necessarily no men are horses. Therefore premisses of this form
cannot prove either a negative or an affirmative.COMMENTARY
CHAPTER 19
Syllogisms in the second figure with one problematic and one
apodeictic premiss
38a13.
(A) Both premisses universal
(a) Premisses differing in quality. (a) Negative premiss apodeic-
tic: problematic and assertoric conclusion. (fl) Affirmative pre-
miss apodeictic: no conclusion. (a) EnAcEp valid, by conversion.
EnAcE valid. by reductio ad impossibile.
25. AcEnEp and AcEnE similarly valid.
26. (fl) EcAn proves nothing; for (1) it may happen that C is
necessarily not B, as when A is white, B man. C swan. There is
therefore no problematic conclusion.
36. But neither is there (2) an apodeictic conclusion; for (i)
such a conclusion requires either two apodeictic premisses, or at'
least that the negative premiss be apodeictic. (ii) It is possible,
with these premisses, that C should be B. For C may fall under B,
and yet A may be contingent for all B, and necessary for C, as
when C is awake, B animal, A movement. Nor do the premisses
yield (3) a negative assertoric conclusion; nor (4) any of the
opposed affirmatives.
b 4 • AnEc similarly invalid.
6. (b) Both premisses negative. EnEcE and EnEcEp valid, by
conversion of En and transition from Ec to Ac.
12. EcEnE and EcEnEp similarly valid.
13. (c) Two affirmative premisses (AnAc, AcAn) cannot prove a
negative assertoric or apodeictic proposition, because neither
premiss is negative; nor a negative problematic proposition,
because it may happen that it is necessary that no C be B (this
shown by an instance); nor any affirmative, because it may
happen that it is necessary that no C be B.
(B) One premiss particular
(a) Premisses of different quality. (a) Negative premiss univer-
sal and apodeictic <being the major premiss). EnlcO and EnlcOp
valid, by conversion.
27. (fl) Affirmative premiss universal and apodeictic (AnOc.
OcAn): nothing follows, any more than when both premisses are
universal (AnEc, EcAn).
29. (b) Two affirmative premisses (Anlc, leAn, Acln. InAc):
nothing follows, any more than when both premisses are universal
(AnAc, AeAn).359
31. (c) Both premisses negative, apodeictic premiss universal
<being the major premiss). EnOcO and EnOcOp valid, by transi-
tion from Oc to le.
(C) Both premisses particular
Two particular premisses prove nothing; this shown by con-
trasted instances.
38. Thus (r) if the negative universal premiss is apodeictic,
both a problematic and an assertoric conclusion follow. (2) If
the affirmative universal premiss is apodeictic, nothing follows.
(3) The valid combinations of a problematic with an apodeictic
premiss correspond exactly to the valid combinations of a pro-
blematic with an assertoric premiss. (4) All the valid inferences are
imperfect, and are completed by means of the aforesaid figures.
35·
38313-16. 'Ea.v S' ••• ~aT(lL. Tlls fl~v aTEpT)TLKllS ..• U1rclPXEL is
true without exception only when both premisses are universal,
and it is such combinations alone that A. has in mind in the first
three paragraphs. rij~ ll£ Ka-ra<pa'TL~~ aUK £a-ra£ is true, whatever
the quantity of the premisses.
16-z5. KE(aOw ya.p ••• EvSEXEaO(lL. From Necessarily no B is
A, For all C, being A is contingent, we can infer (r) that it is
possible that no C should be B; for by converting the major
premiss and dropping the 'necessarily' we get the premisses
No A is B, For all C, being A is contingent, from which it follows
that for all C, not being B is possible (34bI9-3S32): (2) that no C is
B; for if we 'assume the opposite, we get the reductio ad impossibile
'Necessarily no B is A, Some C is B, Therefore necessarily some C
is not A (30br-2); but ex hypothesi for all C, being A is con-
tingent; therefore no C is B'. Becker's suspicions about the final
sentence (p. 46) are unjustified.
z5-6. TOV (lliTov S~ TP01rOV ••• aTEpT)TLKoV. From For all B,
being A is contingent, Necessarily no C is A, we can infer (r) that
for all C, not being B is possible; for by conversion the premisses
become Necessarily no A is C, For all B, being A is contingent,
from which it follows that for all B, not being C is possible
(36"7-17), and therefore that for all C, not being B is possible:
(2) that no C is B; for if we assume the opposite, we get the
redztctio 'For all B, being A is contingent, Some C is B, Therefore
for some C, being A is contingent (3S"3O--5); but ex hypothesi
necessarily no C is A; therefore no C is B'.
z9. aUfl~(l(vEL, not 'it follows', but 'it sometimes happens'.
35. TO yap E~ ciVclYKT)S .•. EVSEXOflEVOV, cf. 32336.360
COMMENTARY
3~7. TO yap ciVClYKClLOV ••• O'UVE~ClWEV. A. has proved in
30b18-40 that in the second figure an apodeictic conclusion does
not follow if the affirmative premiss is apodeictic and the negative
assertoric. A fortiori such a conclusion will not follow if the
affirmative premiss is apodeictic and the negative problematic.
38-b3. En SE .•. lllra.pxuv. A. offers here a second proof that
the premisses (r) For all B, not being A is contingent. (2) All C
is necessarily A, do not yield the conclusion Necessarily no C is
B. (r) is logically equivalent to (ra) For all B, being A is con-
tingent (for the general principle cf. 32"29-br ), and in "39-40 A.
substitutes (ra) for (I). But (ra) For all B, being A is contingent,
(2) All C is necessarily A, and C,) All C is B, may all be true, as
in the instance 'For all animals, being in movement is contingent,
every waking thing is necessarily in movement, and every waking
thing is an animal'.
b3- 4. ouSE S" ..• KClTa.c\lC10'Ewv. A. has shown that EeAn does
not prove Ee ("28-36) nor En )"36-b2) nor E (b2- 3). He now adds
that (for similar reasons) it does not prove any of the opposites
of these (i.e. either the contradictories le, In, I, or the contraries
Ac, An, A.).-Al. plainly read Ka'Taq,auEwlI (238. I), and the reading
q,au€wlI may be due to Al.'s (unnecessary) suggestion that KaTa-
q,au€wlI should be taken to mean q,au€wlI.
~12. O'TEpTJnKwv fLEV ••• O'XTl fLa. , i.e. by complementary con-
version of the minor premiss (32a29-br) and simple conversion of
the major we pass from All B is necessarily not A, For all C, not
being A is contingent, to All A is necessarily not B, For all C.
being A is contingent, from which it follows that no C is B, and
that for all C, not being B is possible (3637-17).
12-13. KQ.V d ... WUClUTWC;. A. is considering cases in which
both premisses are negative, so that at first sight it looks absurd
to say 'if it is the minor premiss that is negative'. But in the form
just considered (b8-r2) the minor premiss was no incurable
negative. Being problematic, it could be transformed into the
corresponding affirmative. A. now passes to the case in which
the minor premiss is incurably negative, i.e. is a negative apo-
deictic proposition: (r) 'For all B, not being A is contingent,
(2) All C is necessarily not A.' Since we cannot have the minor
premiss negative in the first figure, reduction to that figure must
proceed by a roundabout method: (2a) 'All A is necessarily not C,
(ra) For all B, being A is contingent (by complementary conver-
sion, 32a29-br), Therefore no B is C (36a7-17). Therefore no C is B.'
18-20. E~ Q.VUYK..,5 . . . a.VOPW1r05, i.e. there are cases in which,
when it is necessary that all B be A and contingent that all Cshould be A, or contingent that all B should be A and necessary
that all C be A, it is necessary (and therefore not contingent) that
no C be B. E.g. all swans are necessarily white, for all men being
white is contingent, but all men are necessarily not swans.
21. ou8£ yE •.. KaTa.pauEwv. Here, as in b4 , Al.'s reading (239.
36---9) is preferable.
21-2. E1TEl 8e8ELKTaL ••. u1Tapxov, i.e. in certain cases, such as
that just mentioned in b 19 - 20 •
24-35. '0IJ-0LWS 8' ••• 1TpOTEpOV. The first rule stated here
would prima facie include leEn, and the last rule (b31 - S) OeEn, but
these combinations are. in fact invalid because in the second figure
the major premiss must be universal, to avoid illicit major.
A. must be assuming the universal apodeictic premiss to be the
major premiss.
27. a.1To8EL~LS 8E .•. a.VTLUTPO.pijS. From Necessarily no B is
A, For some C, being A is contingent, (I) by converting the major
premiss we get the first-figure syllogism (36334---9) Necessarily no
A is B, For some C, being A is contingent, Therefore some C is
not B, and (2) from this conclusion we get For some C, not being
B is possible.
28-c). TOV aUTov ya.p TP01TOV .•. opwv, i.e. as in a30-bS. Take
for instance OeAn. For some men, not being white is contingent,
all swans are necessarily white, and necessarily no swans are
men. On the other hand, for some animals, not being in move-
ment is contingent, everything that is awake is necessarily in
movement, hut necessarily everything that is awake is an animal.
3~1. Kal ya.p .•• 1TpOTEPOV, i.e. as in b 13- 23 .
31-2. oTav 8E ••• u"lIJ-aLvouua, 'when both premisses are nega-
tive and that which asserts the non-belonging of an attribute to
a subject (not merely that its not belonging is contingent) is
universal and apodeictic (not assertoric), .
35. KaEla1TEp EV TOLS 1!'pOTEPOV, i.e. we may infer a statement of
possibility and one of fact, as with the combination dealt with in
b 25 - 7 (EnIe).
37. a.1To8EL~LS 8' ••• opwv. The reference is to the terms used
in a30-bS to show that EeAn and AnEe prove nothing. Take, for
instance, leIn. For some men being white is contingent; some
swans are necessarily white; but it is necessary that no swans
should be men. On the other hand, for some animals being in
movement is contingent; some waking things are necessarily in
movement; and it is necessary that all waking things should be
animals.
38-41. 4IavEpov oov ••• ou8e1ToTE. A. does not mean that allCOMMENTARY
combinations of a universal negative apodeictic premiss with any
problematic premiss yield a conclusion, but (1) that all valid
combinations containing such a premiss yield both a negative
problematic and a negative assertoric conclusion (for this v.
°16-26, b6-13, 25-7)' and (2) that no combination including a
universal affirmative apodeictic premiss yields a conclusion at all.
41-3931. KaL OTL ... auAAoYL0'l-L6s. i.e. the valid combinations
of a problematic with an apodeictic premiss correspond exactly
to those of a problematic with an assertoric premiss. The former
are EnAc, AcEn, EnEc, EcEn, Enlc, EnOc; the latter are EAc, AcE,
EEc, EcE, EIc, EOc (v. ch. 18).
39 3 3. SLcl TWV 1TpoeLP'll1ivwv aX'll'clTwV. AI. (242. 22-7) thinks
this means either 'by means of the first figure' or 'by means of the
aforesaid moods'. Both interpretations are impossible; Maier
therefore thinks (2 a. 176 n. 2) that the words are a corruption of
~ha TWV ;v Ttji 'TTPOf.LPTJfLlv~ axlifLaTL, i.e. in the first figure. But
EAcEp (37 b2 4- 8), NEEp (ib. 29), and EnNEp (38316-25) have
been reduced to EAcEp in the first figure, which was itself in
34bI9-3I reduced to InAI in the third figure; and EIcOp (3833-4)
has been reduced to EIcOp in the first figure, which was itself
in 35a35-bI reduced to AnIIn in the third figure. Thus S,a TWV
'TTPOf.LPTJfL1vwv aXTJfLaTWII is justified.
CHAPTER 20
Syllogisms in the third figure with two problematic premisses
39"4. In the third figure there can be an inference either with
both premisses problematic or with one. When both premisses
are problematic, and when one is problematic, one assertoric, the
conclusion is problematic. When one is problematic, one apodeic-
tic, if the latter is affirmative the conclusion is neither apodeictic
nor assertoric; if it is negative there may be an assertoric
conclusion; 'possible' in the conclusion must be understood as
= 'not impossible'.
14.
(A) Both premisses universal
AcAcIc valid, by conversion.
19. EcAcOc valid, by conversion.
z3. EcEcIc valid, by transition from Ec to Ac and conversion.
z8.
(B) One premiss particular
When one premiss is particular, the moods that are valid corre-
spond to the valid moods of pure syllogism in this figure. (a)
Both premisses affirmative. AcIcIc valid, by conversion.35. leAele similarly valid.
36. (b) A negative major and an affirmative minor give a
conclusion (EeleOe, OeAele), by conversion.
38. (c) Two negative premisses give a conclusion (EeOele,
OeEele), by complementary conversion.
(C) Both premisses particular
Nothing follows; this shown by contrasted instances.
bZ •
39 3 7-8. Ka.t OTa.V ••• u1TapXEw. For the justification of this
v. ch. 21.
S-II. OTav S' ••• 1TpOTEpOV. For the justification of this v.
ch. 22. Ka(M.7T€p •.. 7Tp6npov refers to 38'13-16 (the corresponding
combinations in the second figure).
II-IZ. XTJ1TTEOV SE ••• EVSEX0,""EVOV, i.e. the only sort of possi-
bility·that can be proved by any combination of a negative
apodeictic with a problematic premiss is possibility in the sense
in which 'possible' = 'not impossible' (cf. 33b29-33), not in the
strict sense in which it means 'neither impossible nor necessary',
(cf. 32'18-21). oj.Lo{w~ = 'as with the corresponding combinations
in the second figure'.
Z3-8. EL S' ... o.VTLC7Tp0<Plls. A. says here that premisses of
the form EeEe can be made to yield a conclusion 'by converting
the premisses', i.e. by complementary conversion (cf. 32a2g---b1).
By this means we pass from EeEe to AeAe, the combination
already seen in '14-19 to be valid.
In '27 Waitz reads, with n, £av j.L€TaATJq,Ofj TO £vSlxwOat j.L~
lmapxnv, assuming that j.L€TaATJq,Ofj means 'is changed'; and this
derives some support from Al.'s commentary (243. 23)-j.L€Ta-
ATJq,O€{UTJ, Si Tij, £AaTTovo, €{, T~V KaTaq,aTtK~V £vSqoj.L'VTJv-and
the corresponding remark in P. 229. 26. But the usual sense of
j.LfTaAaj.L{3aVHv in A. is 'to substitute' (cf. Bonitz, Index), and j.L~
is therefore not wanted.
ZS-3I. EL S' ... auXXoYLa,""os, i.e. 'the valid syllogisms in this
figure with two problematic premisses of different quantity
correspond to the valid syllogisms with two assertoric premisses
of different quantity'. Thus we have Aele, leAe, Eele, and OeAe
corresponding to Datisi, Disamis, Ferison, Bocardo. But in
addition, OW,~lg to the possibility of complementary conversion of
problematic premisses (32a2g---bI), A. allows EeOe and OeEe to be
valid (a38-b2). He says nothing of AeOe and IeEe, but these he
would regard as valid for the same reason.
36-8. 0,",,01WS SE ••• o.VTLO"Tpo<Plls, The validity of Eele wouldCOMMENTARY
be proved thus: By conversion of the minor premiss, 'For all
C, not being A is contingent, For some C, being B is contingent'
becomes 'For all C, not being A is contingent, For some B, being
C is contingent', from which it follows that for some B not being
A is contingent. The validity of OeAe would be proved thus: By
complementary conversion, followed by simple conversion, of the
major premiss, and by changing the order of the premisses, 'For
some C, not being A is contingent, For all C, being B is contingent'
becomes 'For all C, being B is contingent, For some A, being C
is contingent', from which it follows that for some A being B
is contingent, and therefore that for some B being A is contingent.
bl • Ka.9all'Ep tv TOLS lI'pOTEPOV, i.e. in the case of EeEe, OeAe
(&23-8, 36-8).
3-4' Ka.t yap ••• I'TJ5EVt UlI'apXELV, i.e. there are cases in which
A must belong to B, and cases in which it cannot, so that neither
a negative nor an affinnative problematic conclusion can follow
from premisses of this fonn.
4-6. OpOL TOU ulI'apXELv ••• flEUOV AEUKOV. I.e. it is possible
that some white things should be, and that some should not be,
animals, it is possible that some white things should be, and that
some should not be, men; and in fact every man is necessarily an
animal. On the other hand, it is possible that some white things
should be, and that some should not be, horses, it is possible that
some white things should be, and that some should not be, men;
but in fact it is necessary that no man should be a horse. Thus
Iele, IeOe, Oele, OeOe in the third figure prove nothing.
CHAPTER 21
Syllogisms in the third figure with one problematic and one assertoric
premiss
39 b 7. If one premiss is assertoric, one problematic, the con-
clusion is problematic. The same combinations are valid as were
named in the last chapter.
(A) Both premisses universal
(a) Both premisses affinnative: AAelp valid, by conversion.
16. AeAle valid, by conversion.
17. (b) Major premiss negative, minor affinnative: EAeOp,
10.23. (d) Both premisses negative <and minor premiss prob-
lematic>: a conclusion follows (EEcO p), by conversion.
(B) One premiss particular
26.
(a) Both premisses affirmative: a conclusion follows (Alclp,
AcHe, IAclc, IcAlp), by conversion.
27. (b) Universal negative and particular affirmative: a con-
clusion follows <except when the minor premiss is an assertoric
negative (IcE» (ElcOp, EcIOc, IEelc), by conversion.
3I. (c) Universal affirmative <assertoric minor> and particular
negative <problematic major>: OcAOp valid, by red1tctio ad im-
possibile.
40"I.
(C) Both premisses particular
Two particular premisses prove nothing.
39bXO. TOLc; 1TpOTEpOV. This refers to the treatment in ch. 20
of arguments in the third figure with two problematic premisses.
It is not, however, strictly true that the same combinations are
valid when one premiss is assertoric, one problematic, as when
both are problematic. In two respects the conditions are different.
A. (rightly) does not consider 'For all B, not being A is con-
tingent' convertible into 'For all A, not being B is contingent'
(36b3S-37"3I); and he does think it convertible into 'For all B,
being A is contingent' (32a29-bI). For these reasons the valid
combinations do not exactly correspond; while OcEc is valid (by
conversion to IcAc), neither OEc nor OcE is so.
I4-x6. &TE ya.p ••• EVS~XO""EVOV, I9-22 ytvETQL ya.p ••• EVS~­
XO .... EVOV, cf. ch. IS, especially 33b2S-3I.
22. TO CTTEP'lTLKOV. ABed have TO €vS£x6[.UVOV CTu.p7JnK6v. n
has TO CT7Ep7}nK6v, and both Al. (246. II-I6) and P. (231. 24-Q) have
this reading, and say that ivSEX6fLEVOV must be understood; their
comments are no doubt the reason why that word appears in
most of the MSS. The shorter reading prima facie covers the
combination AcE as well as AEc, and the words in the next line
~ Ku1 G.fLcfow A'Y]cfo8£{7J CT7£p'Y]nKa. prima facie cover the case EcE as
well as EEc; but AcE and EcE are invalidated by the fact that
in the third figure the minor premiss must be affirmative (to
avoid illicit major). AEc and EEc, on the other hand, can be
validated by complementary conversion of Ec into Ac. There is
therefore no doubt that the interpretation given by AI. and P.COMMENTARY
premiss are valid, but that when they are (i.e. when this premiss
is problematic) they can be validated by complementary con-
version of the minor premiss (SL' av-rwv f'~V T<VV K"Lf'EVWV OUK ;a-raL
OlJAAOYLUP.OS', avna-rpacpEvTwv S' ;U-raL, b23 - S).
25. Ka.8a.1TEP EV TOLS 1TpOTEPOV, i.e. by complementary con-
version AEe, EEe are reduced to the valid moods AAe, EAe, as
EeEe was reduced to AeAe in "26-8.
26-39' Et S' • • • U1Ta.PXE~V.
A. considers here premisses
differing in quantity. (I) If both premisses are affirmative, the
conclusion is validated by reduction to the first figure (b 27 - 3r ).
This covers Ale, leA, Ael, lAc. (2) So too if the universal premiss
is negative, the particular premiss affirmative (ib.). Prima facie
this covers Ele, IcE, Eel, lEe. But of these IcE (though A. does
not say so) is invalidated by the fact that in the third figure the
minor premiss must (to avoid illicit major) be affirmative (lEe
escapes this objection by complementary conversion of Ee).
(3) If the universal premiss is affirmative, the particular premiss
negative, the conclusion will be got (so A. says) by reductio ad
impossibile (b31 - 3 ). Prima facie this covers the cases AOe, OeA,
AeO, OAe. But the only case specifically mentioned is OeA
(b 33 --9), and it is this case A. has in view in saying that validation
is by reductio; for it is validated by a reductio in AnAAn (30"17-
23). AOe can in fact be validated by complementary conversion
of Oe. AeO is in fact invalid, since in the third figure the minor
premiss must be affirmative. A. says nothing of ON, which in
fact cannot be validated in any way.
A. says nothing of case (4), in which both premisses are nega-
tive. In fact EOe is reducible by complementary conversion to
the valid mood Ele. OeE and EeO are invalid because in the third
figure the minor premiss must be affirmative; OEe is invalid just
as is OAe above, to which it is equivalent by complementary
conversion.
30-1. waTE cjIa.VEPOV ••• auAAoYLalLoS. This follows from the
fact that in the first figure if one premiss is problematic the con-
clusion is so too (b r4- r6).
SESE~KTo.L 1TPOTEPOV, cf. 30"17-23.
40"2-3. a:II"OSEL~~S S' ••• 3pwv. The MSS., AI., and P. have i.v
TOrS' Ka(JOAOV, which would be a reference to the discussion of
moods with two universal premisses (39bro-2S); but A. did not
37. TOUTO yap
in fact condemn any of these, and could not, in the course of so
short a chapter, have forgotten that he had not. Al.'s supposition
(248. 33-7) that TOrS' Ka(JOAov means TOrS' SL' OAOV EVSEX0f'EVOLS',
premisses both of which are problematic, is quite unconvincing.Maier (za. zoz n. I) suspects the whole sentence; but it would
not be in A:s manner to dismiss these moods without giving
a reason. The most probable hypothesis is that A. wrote &
7TPO'Tf,POV, and that, the last word having been lost or become
illegible, a copyist wrote Ka(Jo>'ov, on the model of such passages
as 38bZ8-9, 40bII-IZ. ~V
7TPOT€POV will refer to 39bz-6; the
example given there will equally well serve A.'s purpose here.
TO',
TO',
CHAPTER 22
SyUogisms in the third figure with one problematic and one apodeictic
premiss
4°"4. If both premisses are affirmative, the conclusion is
problematic. When they differ in quality, if the affirmative
is apodeictic the conclusion is problematic; if the negative is
apodeictic, both a problematic and an assertoric conclusion, but
not an apodeictic one, can be drawn.
(A) Both premisses universal
(a) Both premisses affirmative: AnAcIp valid, by conversion.
II.
16.
IS.
valid,
25.
33.
valid,
35.
39·
AcAnIc valid, by conversion.
(b) Major premiss negative, minor affirmative: EcAnOc
by conversion.
EnAcO and EnAcOp valid, by conversion.
(c) Major premiss affirmative, minor negative: AnEcIp
by transition from Ec to Ac.
AcEn proves nothing; this shown by contrasted instances.
(B) One premiss particular
(a) Both premisses affirmative: a problematic conclusion fol-
lows (AnIcIp, IcAnIp, AcInIc, InAcIc), by conversion.
b2. (b) (Major premiss negative, minor affirmative.) (a) Affir-
mative premiss apodeictic: a problematic conclusion follows
(EcInOc, OcAnOp).
3. (13) Negative premiss apodeictic: an assertoric conclusion
follows (EnIcO, OnAcO).
S. (c) Major premiss affirmative, minor negative. (a) Negative
premiss problematic and universal: InEcIc valid, by conversion.
10. (13) Negative premiss apodeictic and universal: IcEn proves
nothing; this shown by contrasted instances, as for AcEn.
12. It is now clear that all the syllogisms in this figure are
imperfect, and are completed by means of the first figure.COMMENTARY
40aI5-16. OUTII,) yap ••. O'XTJf.LQTOS, cf. 36340_bz (Anlelp).
18-38. lI'c1~W EO'TW ..• Q.v9pwlI'os. Of combinations of pre-
misses (both universal) differing in quality, A. examines first
(&19-3Z) those with a negative major premiss, then (a33-8) those
with a negative minor. He does not discuss the combinations
of two negative premisses; but his treatment of them would have
corresponded to his treatment of those with an affirmative major
and a negative minor. EnEe is valid because it is reducible, by
complementary conversion of Ee, to EnAe; EeEn is invalid
because its minor premiss is incurably negative, and in the third
figure the minor must be affirmative to avoid illicit major.
:n-3. KQt yap ••• EVSEX0f.LEVOV, 'because the negative premiss
here, like the affirmative (minor) premiss in AnAe (aII-16) and the
affirmative (major) premiss in AeAn (a16-IS) is problematic'. The
yap clause, which gives the reason for what follows, not for what
goes before, is a good example of the 'anticipatory' use of yap.
Ct. Hdt. 4.79 'HfLl.v yap Ka-ray€Aan, ciJ EKVBat, OTt j3aKX€VOfL€V Ka~
~pla, 0 BEO, AafLj3avH' vvv oV-ro, 0 SaLfLwv Kat -rOV VfdTEPOV j3aa,}..Ea
AEAaj3'1JKE, and other instances cited in Denniston, The Greek
Particles, &}-70.
z3-5. OTE yap ••• €VSEX0I'EVOV. The combination in question,
EeAn, reduces, by conversion of the minor premiss, to Eeln in
the first figure, which was in 36a39-bz shown to yield only a
problematic conclusion.
3O-z. OTE S' ••• 1'1] lI1Tc1pXELV. The combination in question,
EnAe. reduces, by conversion of the minor premiss, to Enle in the
first figure, which was in 36a34-9 shown to yield an assertoric
conclusion, and a fortiori yields a conclusion of the form It is not
impossible that some 5 should not be P. avayK'1} here (33Z) only
means 'it follows' ; the conclusion is not apodeictic; cf. 36"10 n.
34-5. f1E-rQ~TJcp9ELO'T)S • • • lI'POTEPOV, i.e. by complementary
conversion of the minor premiss (cf. 3Zaz9-b1).
b z-8. KQt o-rQV ••• O'UI'1TL1TTELV. The first rule stated here (bZ- 3 )
prima facie includes InEe; but the rule in bS- IO also prima facie
includes it. Again, the rule in b 3 -S prima facie approves IcED,
which the rule in b1o- I I condemns; and in fact IcEn proves
nothing, since in the third figure the minor premiss cannot be
negative unless it is problematic and therefore convertible by
complementary conversion into an affirmative. Finally, AeOn,
which prima facie falls under the rule in b 3 -8, is invalid for the
same reason. Clearly, then, b Z- 3 , 3-8 are not meant to cover so
much as they appear to cover. Now in bS A. expressly passes to
the cases in which the major premiss is affirmative, the minor369
negative. All is made clear by realizing that in b 2 -8 A. has in
mind only the cases in which the major premiss is negative, the
minor affirmative; thus A. is not there thinking of the cases
InEe, IeEn, AeOn.
4-6. b ya.p CIoUTOS TP01l'OS ••• liVTWV. EnIeO is in fact validated
just as EnAeO was ( a2 5-3 2 ), by conversion; but OnAeO is validated
by reductio ad impossibile.
8-II. gTCIov SE ••• EC7TCIo'. Ku86Aov A'Yjrp8EJI is unnecessary, since
AnOe is valid, as well as InEe, and AeOn invalid, as well as IeEn.
But Ku8oAov A'Yjrp8b has the support of AI. and P., and of all the
1.
22. 40aI5-bI2
MSS.
II-I2. SuX9,;O'ETCIo' SE ••• gpwv, cf. 835-8. It is contingent that
some men should be asleep, no man can be a sleeping horse; but
every sleeping horse must be asleep. On the other hand, it is
contingent that some men should be asleep, no man can be a
waking horse; and in fact no waking horse can be asleep. There-
fore IeEn cannot prove either a negative or an affirmative con-
clusion.
CHAPTER 23
Every syllogism is in one of the three figures, and reducible to a
universal mood of the first figure
40bI7. We have seen that the syllogisms in all three figures are
reducible to the universal moods of the first figure; we have now
to show that every syllogism must be so reducible, by showing
that it is proved by one of the three figures.
23. Every proof must prove either an affirmative or a negative,
either universal or particular, either ostensively or from a hypo-
thesis (the latter including reductio ad impossibile). If we can
prove our point about ostensive proof, it will become clear also
about proof from an hypothesis.
30. If we have to prove A true, or untrue, of B, we must
assume something to be true of something. To assume A true of
B would be to beg the question. If we assume A true of C, but
not C true of anything, nor anything other than A true of C, nor
anything other than A true of A, there will be no inference;
nothing follows from the assumption of one thing about one other.
37. If in addition to 'C is A' we assume that A is true of some-
thing other than B, or something other than B of A, or something
other than B of C, there may be a syllogism, but it will not prove
A true of B ; nor if C be assumed true of something other than B,
and that of something else, and that of something else, without
establishing connexion with B.
4985
BbCOMMENTARY
4I"Z. For we stated that nothing can be proved of anything
else without taking a middle tenn related by way of predication
to each of the two. For a syllogism is from premisses, and a
syllogism relating to this tenn from premisses relating to this
tenn, and a syllogism connecting this term with that tenn from
premisses connecting this tenn with that; and you cannot get
premisses leading to a conclusion about B without affirming or
denying something of B, or premisses proving A of B if you do
not take something common to A and B but affinn or deny
separate things of each of them.
13. You can get something common to them, only by predi-
cating either A of C and C of B, or C of both, or both of C, and
these are the figures we have named; therefore every syllogism
must be in one of these figures. If more terms are used connecting
A with B, the figure will be the same.
ZI. Thus all ostensive inferences are in the aforesaid figures;
it follows that reductio ad impossibile will be so too. For all such
arguments consist of (a) a syllogism leading to a false conclusion,
and (b) a proof of the original conclusion by means of a hypothesis,
viz. by showing that something impossible follows from assuming
the contradictory of the original conclusion.
z6. E.g. we prove that the diagonal of a square is incommen-
surate with the side by showing that if the opposite be assumed
odd numbers must be equal to even numbers.
3Z. Thus reductio uses an ostensive syllogism to prove the false
conclusion; and we have seen that ostensive syllogisms must be
in one of the three figures; so that reductio is achieved by means of
the three figures.
37. So too with all arguments from an hypothesis; in all of
them there is a syllogism leading to the substituted conclusion,
and the original conchision is proved by means of a conceded
premiss or of some further hypothesis.
bI. Thus all proof must be by the three figures; and therefore
all must be reducible to the universal moods of the first figure.
37 0
40bI8-I9. 8LCl TWV • • • auAAoyw .... WV. In 29bl-25 A. has shown
that all the valid moods of the three figures can be reduced to the
universal moods of the first figure (Barbara, Celarent). Maier
(2 a. 217 n.) objects that it is only the moods of the pure syllogism
that were dealt with there, and that A. could not claim that all
the moods of the modal syllogism admit of such reduction; he
wishes to reject KuBoAolI here and in 41bS. But throughout the
treatment of the modal syllogism A. has consistently maintained37 1
that the modal syllogisms are subject to the same conditions,
mutatis mutandis, as the pure, and there can be no doubt that
he would claim that they, like pure syllogisms, are all reducible
to Barbara or Celarent. Both AI. and P. had Ka86Aov, and it
would be perverse to reject the word in face of their agreement
with the MSS.
25. ETL 11 IiEIKTLKwC; 11 ~~ U'lTOOEaEws. Cf. 29"31-2 n. A.
describes an argument as 19 tmo(J€G€w, when besides assuming the
premisses one supposes something else, in order to see what con-
clusion follows when it is combined with one or both of the
premisses. Reductio ad impossibile is a good instance of this. For
A.'s analysis of ordinary reasoning 19 tmo(J€G€w, (other than
reductio) cf. 50"16-28.
33-41"1. Et liE ... auAAoYLalloS. A. lays down (I) ("33-7) what
we must have in addition to 'C is A', in order to get a syllogism
at all. We must have another premiss containing either C or A.
He mentions the cases in which C is asserted or denied of some-
thing, or something of C, or something of A, but omits by
inadvertence the remaining case in which A is asserted or denied
of something). (2) (b37-41'2) he points out what we must have in
addi tion to 'C is A', to prove that B is A . We cannot prove this
if the other premiss is of the form 'D is A', 'A is D', 'C is D', or
'D is C.
41"2-4. aAws yap ••• KaTTJyopLaLS. A. has not made this
general statement before, but it is implied in the account he gives
in chs. 4--6 of the necessity of a middle term in each of the three
figures. TaL, KaT''1yop{aL, is to be explained by reference to '14-16.
22-b 3. aTL liE Kat OL EtS TO aSUvaTOV ••• aXTJllaTwv. For the
understanding of A.'s conception of reductio ad impossibile, the
present passage must be compared with 50"16-38. In both
passages reductio is compared with other forms of proof 19
tmo(J€G€w,. The general nature of such proof is that, desiring to
prove a certain proposition, we first extract from our opponent
the admission that if a certain other proposition can be proved,
the original proposition follows, and then we proceed to prove
the substituted proposition (TO J.L€TaAaJ.L{3av6J.L€vov, 41"39). The
substituted proposition is said to be proved syllogistically, the
other not syllogistically but Jg tmo(J€G€w,. Similarly reductio falls
into two parts. (I) Supposing the opposite of the proposition
which is to be proved, and combining with it a proposition known
to be true, we deduce syllogistic ally a conclusion known to be
untrue. (2) Then we infer, not syllogistically but Jg tmo8€G€w"
the truth of the proposition to be proved. That the tm68wL,COMMENTARY
37 2
referred to is not the supposition of the falsity of this proposition
(which i!3 made explicitly in part (I)) is shown (a) by the fact
that both in 41"32-4 and in 50a29-32 it is part (2) of the proof that
is said to be £~ Inro8£u£w" and (b) by the fact that in 50"32--8
reductio is said to differ from ordinary proof ;~ lnro8£u£w, in that
in it the Inr68wL, because of its obviousness need not be stated.
It is, in other words, of the nature of an axiom. A. nowhere makes
it perfectly clear how he ' .... ould have formulated this, but he
comes near to doing so when he says in 41'24 TO S' £~ apxij, E~
Inro8£u€w, 8£LKIIVOVULII, OTall aSvllaT611 n UVJ.Lf3a{vn Tij, alln.pau£w,
n8£{UYj" This comes near to formulating the hypothesis in the
form 'that from which an impossible conclusion follows cannot
be true'. But another element in the hypothesis is brought out
in An. Post. 77'22-5, where A. says that reductio assumes the law
of excluded middle; i.e. it assumes that if the contradictory of
the proposition to be proved is shown to be false, that proposition
must be true.
The above interpretation of the words TO S' E~ apxij, E~ Inro8£u£w,
8EtKYVDVULII is that of Maier (2 a. 238 n.). T. Gomperz in A .G.P.
xvi (1903), 274-5, and N. M. Thiel in Die Bedeutung des Wortes
Hypothesis bei Arist. 26-32 try, in vain as I think, to identify
the 1nr68£UL, referred to with the assumption of the contradictory
of the proposition to be proved.
26->]. otov on 6.uu ........ npo5 •.• TE9EIU".... The proof, as stated
by AI. in 260. 18-261. 19, is as follows: If the diagonal BC of a
square ABDC is commensurate with the side AB, the ratio of
BC to A B will be that of one number to another (by Euc. El.
10. 5, ed. Heiberg). Let the smil.llest numbers that are in this
ratio be e, j. These will be prime to each other (by Euc. 7. 22).
Then their squares i, k will also be prime to each other (by
Euc. 7. 27). But the square on the diagonal is twice the size of
the square on the side; i = 2k. Therefore i is even. But the half
of an even square number is itself even. Therefore i/2 is even.
Therefore k is even. But it is also odd, since i and k were prime
to each other and two even numbers cannot be prime to each
other. Thus either both i and k or one of them must be odd, and
at the same time both must be even. Thus if the diagonal were
commensurate with the side, certain odd numbers would be equal
to even numbers (or rather, at least one odd number must be
equal to an even number). The proof is to be found in Euc. 10,
App. 27 (ed. Heiberg and Menge).
30-1. TOUTO yap .•• auAAoylao.a9aL, cf. 29b7-II.
31-2. TO 8E~€o.l n . . . U1I'09"aLV, 'to prove an impossible result373
to follow from the original hypothesis', i.e. from the hypothesis
of the falsity of the proposition to be proved . .q Eg apX7]e; 1nr6(JH1te;
is to be distinguished from T6 Eg apX7]e; ("34), the proposition
originally taken as what is to be proved.
37-40. WO'CLUTWS S€ ... U'lT09EO'EWS. The interpretation of the
sentence has been confused by Waitz's assumption that fLETa-
AafLf3dv€,v is used in a sense which is explained in AI. 263. 26-36,
'taking a proposition in another sense than that in which it was
put forward', or (more strictly) 'substituting a proposition of the
form "A is B" for one of the form "If A is B, C is D" '. AI.
ascribes this sense not to A. but to o[ apxa'iot, the older Peri-
patetics, and it is (as Maier points out, 2 a. 250 n.) a Theophrastean,
not an Aristotelian, usage. According to regular Aristotelian
usage fLETaAafLf3a.VHV means 'to substitute' (cf. 48"9, 49 b3) , and
what A. is saying is this: In all proofs starting from an hypothesis,
the syllogism proceeds to the substituted proposition, while the
proposition originally put forward to be proved is established (I)
by an agreement between the speakers or (2) by some other
hypothesis. Let the proposition to be proved be 'A is B'. The
speaker who wants to prove this says to his opponent 'Will you
agree that if C is D, A is B?' (r) If the opponent agrees, the first
speaker proves syllogistically that C is D, and infers non-
syllogistically that A is B. (2) If the opponent does not agree,
the first speaker falls back on another hypothesis: 'Will you
agree that if E is F, then if C is D, A is B?', and proceeds to
establish syllogistically that E is F and that C is D, and non-
syllogistic ally that A is B. The procedure is familiar in Plato;
cf., for example, Meno, 86 e-87 c, Prot. 355 e. Shorey in 'L'vAAo-
i',afLOt Eg tJ7TofJla€we; in A: (A.J.P. x (1889),462) points out that
A. had the Meno rather specially in mind when he wrote the
A nalytics; cf. 67"21, 69324-<), An. Post. 71329.
bS. EtS TOUS EV TOUTlt' KCL96AOU O'UAAoy~O'""ous, cf. 4obr8-19 n.
CHAPTER 24
Quality and quantity of the premisses
4Ib6. Every syllogism must have an affirmative premiss and a
universal premiss; without the latter either there will be no
syllogism, or it will not prove the point at issue, or the question
will be begged. For let the point to be proved be that the pleasure
given by music is good. If we take as a premiss that pleasure is
good without adding 'all', there is no syllogism; if we specify one
particular pleasure, then if it is some other pleasure that isCOMMENTARY
374
specified, that is not to the point; if it is the pleasure given by
music, we are begging the question.
13. Or take a geometrical example. Suppose we want to prove
the angles at the base of an isosceles triangle equal. If we assume
the two angles of the semicircle to be equal, and again the two
angles of the same segment to be equal, and again that, when we
take the equal angles from the equal angles, the remainders are
equal, without making the corresponding universal assumptions,
we shall be begging the question.
22. Clearly then in every syllogism there must be a universal
premiss, and a universal conclusion requires all the premisses to
be universal, while a particular conclusion does not require this.
27. Further, either both premisses or one must be like the
conclusion, in respect of being affirmative or negative, and apodeic-
tic, assertoric, or problematic. The remaining qualifications of
premisses must be looked into.
32. It is clear, too, when there is and when there is not a
syllogism, when it is potential and when perfect, and that if there
is to be a syllogism the terms must be related in one of the
aforesaid ways.
41b6. "En TE ••• dvaL. A. offers no proof of this point; he
treats it as proved by the inductive examination of syllogisms
in chs. 4-22. The apparent exceptions, in which two negative
premisses, one or both of which are problematic, give a conclusion,
are not real exceptions. For a proposition of the form 'B admits
of not being A' is not a genuine negative (32br-3), and can be
combined with a negative to give a conclusion, by being comple-
mentarily converted into 'B admits of being A' (32329-br).
14. EV TO~~ 8Laypa.f-1fLaaw, 'in mathematical proofs'. For this
usage cf. Cat. 14 3 39, M ct. 998325.
15-22. ~aTwaav ••• Ad1TElT9aL. Subject to differences as to the
placing of the letters, the interpretation
given by AI. 268. 6-269. IS and that given
by P. 253. 28-254. 23 are substantially
the same, viz. the following: A circle
is described having as its centre the
meeting-point of the equal sides (A, E)
of the triangle, and passing through the
ends of the base. Then the whole angle
E +r (rryv Ar) = the whole angle Z +Ll
H7 ELl), they being 'angles of a semi-
circle'. And the angle r = the angle Ll, they being 'angles of a1. 24. 4I~22
375
segment'. But if equals are taken from equals, equals remain;
therefore the angle E = the angle Z.
Waitz criticizes this proof, on the ground that the angles
E+r. Z+Lt, r, Lt. being angles formed by a straight line and a
curve, are not likely to have been used in the proof of a proposition
so elementary as the pons asinorum. He therefore assumes a
different construction and proof. He assumes the upper ends of
the two diameters to be joined to the respective ends of the base.
Then the angle A + r = the angle B +Lt,
they being angles in a semicircle, and the
angle r = the angle Lt. they being angles
in the same segment. Therefore the angle
A = the angle B. He treats 'TaS" EZ in
b20 as an interpolation taking its origin
from the 'TaS" £g which was the original
reading of the MS. d. 'TaS" £g being itself
a corrupt reduplication of 'TO £g (dPx77S"),
which follows immediately.
Heiberg has pointed out (in Abh. ZUT Gesch. der Math. Wissen-
schaften, xviii (1904), 25-6) that mixed angles (contained by a
straight line and a curve), though in Euclid's Elements they occur
only in the propositions Ill. 16 and 31, fall within his conception
of an angle (I. def. 8 'E7Tl7TfiOOS" O£ ywvla £CTTLV ~ £V £7TL7T£O'fJ ovo
ypafLfLwv a.7T'TOfL£VWV lli-,)>'wv KaL fL~ £7T' dlfhlaS" KfiLfL£VWV 7TPOS"
lli-,)>.aS" 'TWV ypafLfLwv K>.tULS"; def. 9 ·O'Tav o£ at 7TfipdxovuaL TIJv
ywvlav ypafLfLaL d"JELaL cJJuw. dlfJU.ypafLfLoS" KaAfiL'TaL ~ ywvla). Further,
the angle of a segment is defined as ~ 7TEPLfiXOfL£VT) IJ7TO 'Tfi dlfhlaS"
KaL KVK>'OV 7TfipupfipfilaS" (Ill, def. 7), in distinction from the angle
in a segment (e.g. ~ £V ~fLLKVK>.l'fJ. An. Post. 94828, Met. 1051"27),
which (as in modern usage) is that subtended at the circum-
ference by the chord of the segment (Ill, def. 8). We must sup-
pose that A. uses the phrases 'TaS" 'TWV ~fLLKVK>.tWV (ywvlaS") b q and
TIJv TOU TfL-,)fLaToS" (ywvlav) b 1 8 in the Euclidean sense, as Al.'s
interpretation assumes. A. refers in one other passage to a mixed
angle-in Meteor. 375b24, where TIJv JUlCw ywvlav means the angle
between the line of vision and the rainbow. The use of mixed
angles had probably played a larger part in the pre-Euclidean
geometry with which A. was familiar, though comparatively
scanty traces of it remain in Euclid. The proposition stating the
equality of the mixed 'angles of a semicircle' occurs in ps.-Eucl.
CatoptTica, prop. 5.
A.'s use of letters in this passage is loose but characteristic.
A and B are used to denote radii (h15) ; for the use of single lettersCOMMENTARY
to denote lines cf. ,\1 eteor. 376'1I-24, hI, 4, De M em. 452hI9-20.
Ar, B.1 are used to denote the mixed angles respectively con-
tained by the radii A, B and the arc r.1 which they cut off.
rand .1 are used to denote the angles made by that arc with its
chord, and E and Z to denote the angles at the base of the triangle;
for the use of single letters to denote angles cf. An. Post. 94"29,
30, Meteor. 373"12, 13, 376a29·
24. KaL OGTWS Kcl.KELVWS. i.e. both when both the premisses are
universal and when only one is so.
27-30. Si}Aov SE •.• €VSEXO~V. A. gives no reason for this
generalization; he considers it to have been established induc-
tively by his review of syllogisms in chs. 4-22. The generalization
is not quite correct; for A. has admitted many cases in which an
assertoric conclusion follows from an apodeictic and a problematic
premiss (see chs. 16, 19, 22).
31. ~"lrLaKlhl,aa9aL SE ••• KaTT)yopLas. I.e. we must consider,
with regard to other predicates-e.g. 'true', 'false', 'probable',
'improbable', 'not necessary', 'not possible', 'impossible', 'true
for the most part' (cf. 43 b 33-{i)-whether, if a conclusion asserts
them, one of the premisses must do so.
33. KaL "lrOTE SuvaTos. owaTo, is used here to characterize the
syllogisms which are elsewhere called (hEAE'" A syllogism is
ovvaTo, if the conclusion is not directly obvious as following from
the premisses, but is capable of being elicited by some mani-
pulation of them.
CHAPTER 25
Number of the terms, premisses, and conclusions
41b36. Every proof requires three terms and no more; though
(I) there may be alternative middle terms which will connect two
extremes, or (2) each of the premisses may be established by a
prior syllogism, or one by induction, the other by syllogism. In
both these cases we have more than one syllogism.
42"6. What we cannot have is a single syllogism with more than
three terms. Suppose E to be inferred from premisses A, B, C, D.
One of these four must be related to another as whole to part. Let
A be so related to B. There must be some conclusion from them,
which will be either E, C or D, or something else.
14. (x) If E is inferred, the syllogism proceeds from A and B
alone. But then (a) if C and D are related as whole to part, there
will be a conclusion from them also, and this will be E, or A or B,
or something else. If it is (i) E or (ii) A or B, we shall have (i)377
alternative syllogisms, or {ii) a chain of syllogisms. If it is (iii)
something else, we shall have two unconnected syllogisms. (b) If
C and D are not so related as to form a syllogism, they have been
assumed to no purpose, unless it be for the purpose of induction
or of obscuring the issue, etc.
24. (2) If the conclusion from A and B is something other than
E, and (a) the conclusion from C and D is either A or B, or some-
thing else, (i) we have more than one syllogism, and (ii) none of
them proves E. If (b) nothing follows from C and D, they have
been assumed to no purpose and the syllogism we have does not
prove what it was supposed to prove.
30. Thus every proof must have three terms and only three.
32. It follows that it must have two premiss cs and only two
(for three terms make two premisses), unless a new premiss is
needed to complete a proof. Evidently then if the premisses
establishing the principal conclusion in a syllogistic argument are
not even in number, either the argument has not proceeded
syllogistically or it has assumed more than is necessary.
bI. Taking the premisses proper, then, every syllogism proceeds
from an even number of premisses and an odd number of terms;
the conclusions will be half as many as the premisses.
5. If the proof includes prosyllogisms or a chain of middle
terms, the terms will similarly be one more than the premisses
(whether the additional tcrm be introduced from outside or into
the middle of the chain), and the premisses will be equal in
number to the intervals; the premisses will not always be even
and the terms odd, but when the premisses are even the terms
will be odd, and vice versa; for with one term one premiss will
be added.
16. The conclusions will no longer be related as they were to
the terms or to the premisses; when one term is added, con-
clusions will be added one fewer than the previous terms. For the
new term will be inferentially linked with each of the previous
terms except the last; if D is added to A, B, C, there are two new
conclusions, that D is A and that D is B.
23. So too if the new term is introduced into the middle; there
is just one term with which it does not establish a connexion.
Thus the conclusions will be much more numerous than the terms
or the premisses.
4Ib36-40' d-il"ov Si . . . KwMu. This sentence contains a
difficult question of reading and of interpretation. In b 39 d and
the fust hand of B have AB Ka, Br, C and the second hand of ACOMMENTARY
(the original reading is illegible) have AB Ka, AT, and n, Al., and
P. have AB Ka, Ar Ka, Br. With that reading we must suppose
the whole sentence to set aside, as irrelevant to A.'s point (that
a syllogism has three terms and no more), the case in which alterna-
tive proofs of the same proposition are given. A. first sets aside
(b 38-<)) the case in which both premisses of each proof are different
from those of the other, as in All N is M (A), All P is N (B),
Therefore all P is M (E), and All 0 is M (r), All P is 0 (..::1), There-
fore all P is M (E). It then occurs to A. to suggest (in b39 ) that
there may be three alternative proofs each of which shares one
premiss with each of the other two proofs. Now here, the con-
clusion being identical, the extreme terms in each syllogism are
identical with the extreme terms in each of the other two
syllogisms; and, each syllogism having one premiss in common
with each of the other two syllogisms, the middle terms must
also be identical. The proofs must differ, then, only in the
arrangement of the terms; they will be proofs in the three figures,
using the same terms. AI. and P. adopt this interpretation.
Two difficulties at once present themselves. (1) If A and r
can each serve with the same premiss B to produce the saljIle
conclusion E, they must themselves have identical terms; and
if so, they cannot themselves combine as premisses of a third
syllogism. (2) If we avoid this difficulty by omitting the doubtful
words Ka, Ar (or Ka, Br), there still remains the objection that
two syllogisms containing the same terms differently arranged
would be no illustration of what A. is here conceding-the
possibility of the same conclusion being proved by the use of
d~fferent middle terms. To avoid this objection, Maier (2 a. 223 n.)
takes the passage quite differently. He reads SLa "TWV AB Ka, Br,
and supposes these words to refer not to alternative proofs but
to parts of a single proof, such as All N is M (A), All 0 is N (B),
All P is 0 (r), Therefore all P is M (E). The description of such
a sorites, however, as being SLa "TWV AB Ka, Br is unnatural; we
should rather expect Sui "TWV ABr, the premisses being named
continuously as in 42"9. Besides, it seems most unlikely that A.
could have coupled a reference to a single sorites with a reference
to two alternative syllogisms (b 38-<)); it is only in 4231 that he
comes to discuss the single chain of proof with more than one
middle term.
The great variety of readings points to early corruption. Now
in 42"1-2 A. goes on to the case in which each premiss of a syllogism
is supported by a pro syllogism ; and this makes it likely that he
has already referred to the case in which one of the premisses is so1. 25. 4235-30
379
supported. This points to the reading Iha 'TWII AB Kat Ar.1. A.
will then be saying in 41b37--9 'if we set aside as irrelevant (1)
the case in which E is proved by two proofs differing in both their
premisses and (2) that in which E is proved by two proofs sharing
one premiss; e.g. when All P is M is proved (a) from All N is M
and All P is N (A and B), and (b) from All N is M, All 0 is N,
and All P is 0 (A, r, and .1)'.
42 3 5. Kat TO r, i.e. the conclusion from A and B.
6-8. EL S' oov ••• ciSuvaTov, i.e. if anyone chooses to call a
syllogism supported by two prosyllogisms 'one syllogism', we
may admit that in that sense a single conclusion can follow from
more than two premisses; but it does not follow from them in
the same way as conclusion C follows from the premisses A, B,
i.e. directly.
erI2. OUKOUV civciYKTJ ••• opwv, i.e. to yield a conclusion, two
of the premisses must be so related that one of them states a
general rule and another brings a particular case under this rule.
This is A.'s first statement of the general principle that syllogism
proceeds by subsumption. That it does so is most clearly true
of the first figure, which alone A. regards as self-evident. 'TOU'TO yap
8ES£lK'Tat 7TponpolI is probably a reference to 40b30-41"20.
I~20. KaL lOt Il~v ••• uUIl~aLv40L. i.e. if C and D prove E, we
have not one but two syllogisms, A BE and C DE; if C and D
prove A or B, we have merely the case which has already been
admitted in "1-7 to occur without infringing the principle that
a syllogism has three and only three terms, viz. the case in which
a syllogism is preceded by one or two prosyllogisms proving one
or both of the premisses.
23-4. lOt IlTJ i1Taywyij5 ••• Xa.PLV, i.e. the propositions C, D
may have been introduced not as syllogistic premisses but (a) as
particular statements tending to justify A or B inductively, or
(b) to throw dust in the eyes of one's interlocutor by withdrawing
his attention from A and B, when these are insufficient to prove
E, or (c), as AI. suggests (279.4). to make the argument apparently
more imposing. Cf. Top. 155b20-4 allaYKaL'at SE A£YOIlTat (7Tp0'Taou<;)
SL' .1;11 0 ov'\'\oytuj.to<; yLIIE'Tat. at SE 7Tapa. 'TavTa<; '\aj.t{3alloj.tOJat
'TE'TTaPE<; Eluw' ~ yap E7Taywrii<; Xaptll 'TOU S06iillat 'TO Ka60'\ov, ~ El"
0YKOII 'TOU '\oyov, ~ 7TPO<; KpV.ptll 'TOU uvj.t7TEpauj.ta'To<;, ~ 7TPO<; 'TO
oarPEunpolI E Illat 'Tall ,\0YOV.
2~30. lOt 8~ IlTJ YLv40TaL ••• uuAAoYLullov. AI. noticed that this
point has been made already with regard to rand.1 ("22-4),
and therefore, to avoid repetition, suggested (280. 21-4) that AB
should be read for r.1. But in fact this sentence is no mereCOMMENTARY
repetition. In 314-24 A. was examining his first main alternative,
that the conclusion from A and B is E. Under this, he examines
various hypotheses as to the conclusion from rand LI, and the
last of these is that they have no conclusion. In "24-30 he is
examining his other main alternative, that the conclusion from A
and B is something other than E, and here also he has to examine,
in connexion with this hypothesis, the various hypotheses about
the conclusion from rand LI, and again the last of these is that
they have no conclusion.
32-S. TOUTOU 5' .. . 00UAAOYLO"fLWV. From the fact that there
are three and only three terms it follows that there are two and
only two premisses-unless we bring in a new premiss, by con-
verting one of the original premisses, to reduce the argument
from the second or third figure to the first (cf. 24b22--6, etc.). This
exception only 'proves the rule', for the syllogism then contains
only the original premiss which is retained and the new premiss
which is substituted for the other original premiss. The sense re-
quires ot ya.p Tpf/i~ ... 7rpoTa.a€L~ to be bracketed as parenthetical.
bS--6. lSTa.V 5e ... r6. Though Al.'s lemma has JL~ avVEXWV, his
commentary and quotations (283. 3, 284. 20, 29) show clearly that
he read avVEXWV, and this alone gives a good sense. If a subject
B is proved to possess an attribute A by means of two middle
terms C, D, this may be exhibited either by means of a syllogism
preceded by a prosyllogism, or as a sorites consisting of a con-
tinuous chain of terms: (I) C is A, D is C, Therefore D is A.
D is A, B is D, Therefore B is A. (2) B is D, D is C, C is A,
Therefore B is A. In either case the number of terms exceeds
by one the number of independent premisses; there are the four
terms A, C, D, B. and the three independent premisses C is A,
D is C, B is D.
8-10. 11 ya.p ... Oplo)v. In framing the sorites All B is D, All
D is C, All C is A, Therefore all B is A, we may have started with
All D is C, All C is A, Therefore all D is A, or with All B is D,
All D is C, Therefore All B is C, and then brought in the term B
in the first case, or A in the second, 'from outside'. Or again we
may have started with All B is C, All C is A, Therefore All B
is A, or ",ith All B is D, All D is A, Therefore all B is A, and
brought in the term D in the first case, or the term C in the
second, 'into the middle' (D between Band C, or C between D
and A). In any case, A.'s principle is right, that the number of
stretches from term to term, B-D, D-C, C-A, is one less than
the number of terms.
B. Einarson in A.J.F. lvii (1936), 158 gives reasons for believingthat the usage of TTap£p:rrl7rTHV in bg (as of EILTTl7rTHV and of
fIL{3&.>J..m(Jm) is borrowed from the language used in Greek mathe-
matics to express the insertion of a proportional mean in an interval.
15-16. c1Va.YKTJ 1TapaAAa.TTELV .•. YLVOIlEvTJS, i.e. the premisses
become odd and the terms even, when the same addition (i.e.
the addition of one) is made to both.
16-:t6. Ta SE al!ll1TEpa.allaTa •.. 1TpoTa.aEwv. The rule for the
simple syllogism was: one conclusion for two premisses (b4- S).
The rule for the sorites is: for each added term there are added
conclusions one fewer than the original terms. A. takes (I)
(b I9 - z3 ) the case in which we start from All B is A, All C is B,
Therefore all C is A, and add the term D, i.e. the premiss All
D is C. Then we do not get a new conclusion with C as predicate
(TTP0<; IL6vov TO £UXaTOV OV TTO'£' avILTTipauILa, b I9 - 20). But we get
a new conclusion with B as predicate (All D is B) and one with
A as predicate (All D is A). (Similarly if we add a further term
E, i.e. the premiss All E is D, we get three new conclusions-
All E is C, All E is B, All E is A (oILolwS' SE KaTT' TWV lliwv, b23 ).)
Again (b 23- S) suppose we start from All B is A, All C is B,
Therefore all C is A, and introduce a fourth term (2) between
B and A or (3) between C and B. In case (2) we have the premisses
All D is A, All B is D, All C is B, and we get a new conclusion
with A as predicate (All B is A) and one with D as predicate
(All C is D), but none with B as predicate. In case (3) we have
the premisses All B is A, All D is B, All C is D, and we get a
new conclusion ,,{ith A as predicate (All D is A) and one with
B as predicate (All C is B), but none with C as predicate.
Thus in a so rites 'the conclusions are much more numerous
than either the terms or the premisses' (b 2S --{i). The rule is:
z premisses, 3 terms, I conclusion,
3 premisses, 4 terms, 1+2 conclusions,
4 premisses, S terms, 1+2+3 conclusions,
and in general n premisses, n+ I terms, t n (n -I) conclusions.
TTOAV TTA£lw is, of course, correct only when n is greater than S.
CHAPTER 26
The kinds of proposition to be proved or disproved in each figure
42bZ7. Now that we know what syllogisms are about, and what
kind of thing can be proved, and in how many ways, in each
figure, it is clear what kinds of proposition are hard and what are
easy to prove; that which can be proved in more figures and in
more moods is the easier to prove.COMMENTARY
32. A is proved only in one mood, of the first figure; E in one
mood of the first and two of the second; I in one of the first and
three of the third; 0 in one of the first, two of the second, and
three of the third.
40. Thus A is the hardest to prove, the easiest to dispro:le. In
general, universals are easier to disprove than particulars. A is
disproved both by E and by 0, and 0 can be proved in all the
figures, E in two. E is disproved both by A and by I, and this can
be done in two figures. But 0 can be disproved only by A, I only
by E. Particulars are easier to p,'ove, since they can be proved both
in more figures and in more moods.
43"10. Further, universals can be disproved by particulars and
vice versa; but universals cannot be proved by particulars,
though particulars can by universals. It is clear that it is easier
to disprove than to prove.
16. We have shown, then, how every syllogism is produced,
how many terms and premisstOs it has, how the premisses are
related, what kinds of proposition can be proved in each figure,
and which can be proved in morc, which in fewer, figures.
42b27. 'E1I"El /)' ••• C"UXXOYLC"poOL, i.e. since we know what
syllogisms aim at doing, viz. at proving propositions of one of the
four forms All B is A, No B is A, Some B is A, Some B is not A.
32-3. TO POEV o~v Ka.Ta.CPa.TLKOV ••• poova.xw~, i.e. by Barbara
(2S b37-40 ).
34-5. TO /)E C"TEP"TLKOV ••• /)LXWS, i.e. by Celarent (2Sb4o--26a2),
or by Cesare (27 3 S-<)) or Camestres (ib. 9-14).
35--6. TO /)' EV POEpEL ••• EC"Xa.TOU, i.e. by Darii (26323-S), or by
Darapti (28 318-26), Disamis (28b7-II), or Datisi (ib. II-1S).
38-40' TO /)E C"TEP"TLKOV ••• TpLXWS, i.e. by Ferio (2632S-30), by
Festino (27332--6) or Baroco (ib. 36-b3). or by Felapton (28326-30),
Bocardo (28 b1S-21), or Ferison (ib. 31-5).
43"7. ~v, cf. 4 2b34.
CHAPTER 27
Rules for categorical syllogisms, applicable to all problems
43"20. We have now to say how we are to be well provided with
syllogisms to prove any given point, and how we are to find the
suitable premisses; for we must not only study how syllogisms
come into being, but also have the power of making them.
25. (I) Some things, such as Callias or any sensible particular,
are not predicable of anything universally, while other things arepredicable of them; (2) some are predicable of others but have
nothing prior predicable of them; (3) some are predicable of other
things while other things are also predicable of them, e.g. man of
Callias and animal of man.
32. Clearly sensible things are not predicated of anything else
except per accidens, as we say 'that white thing is Socrates'. We
shall show later, and we now assume, that there is also a limit in
the upper direction. Of things of the second class nothing can be
proved to be predicable, except by way of opinion; nor can
particulars be proved of anything. Things of the intermediate
class can be proved true of others, and others of them, and most
arguments and inquiries are about these.
hI. The way to get premisses about each thing is to assume
the thing itself, the definitions, the properties, the attributes
that accompany it and the subjects it accompanies, and the
attributes it cannot have. The things of which it cannot be an
attribute we need not point out, because a negative proposition
is convertible.
6. Among the attributes we must distinguish the elements in
the definition, the properties, and the accidents, and which of
these are merely plausibly and which are truly predicable; the
more such attributes we have at command, the sooner we shall
hit on a conclusion, and the truer they are, the better will be the
proof.
11. We must collect the attributes not of a particular instance,
but of the whole thing-not those of a particular man, but those
of every man; for a syllogism needs universal premisses. If the
term is not qualified by 'all' or 'some' we do not know whether the
premiss is uni versal.
16. For the same reason we must select things on which as a
whole the given thing follows. But we must not assume that the
thing itself follows as a whole, e.g. that every man is every
animal; that would be both useless and impossible. Only the
subject has 'all' attached to it.
22. When the subject whose attributes we have to assume is
included in something, we have not to mention separately among
its attributes those which accompany or do not accompany the
wider term (for they are already included; the attributes of
animal belong to man, and those that animal cannot have, man
cannot have); we must assume the thing's peculiar attributes;
for some are peculiar to the species.
29. Nor have we to name among the things on which a genus
follows those on which the species follows, for if animal followsCOMMENTARY
on man, it must follow on all the things on which man follows,
but these are more appropriate to the selection of data about man.
3z. We must assume also attributes that usually belong to the
subject, and things on which the subject usually follows; for a
conclusion usually true proceeds from premisses all or most of
which are usually true.
36. We must. not point out the attributes that belong to every-
thing; for nothing' can be inferred from these.
43aZ9-30. Ta. 8' atha. .•. KaT11YopELTaL. These are the highest
universals, the categories.
37. 'II'a.ALV epoullEV, An. Post.!. 19-22.
37-43' Ka.Ta. IlEV o~v TOUTWV ... TOUTWV. The effect of this is
that the 'highest terms' and the 'lowest terms' in question cannot
serve as middle terms in a first-figure syllogism, since there the
middle term is subject of one premiss and predicate of the other.
But the 'highest terms' can serve as major terms, and the 'lowest
terms' as minor terms. And further the 'highest terms' can serve
as middle terms in the second figure, and the 'lowest terms' as
middle terms in the third. It is noteworthy, however, that A.
never uses a proper name or a singular designation in his examples
of syllogism; the terms that figure in them are of the intermediate
class-universals that are not highest universals.
39. 'II'A1jv Et IlTJ KaTa S6~a.v. In view of what A. has said in
"29-30, it is clearly his opinion that no predication about any of
the categories can express knowledge. To say that substance
exists or that substance is one is no genuine predication, since
'existent' and 'one' are ambiguous words not conveying any
definite meaning. But there were people who thought that in
saying 'substance exists' or 'substance is one' they were making
true and important statements, and it is to this S6(a that A. is
referring. The people he has in view are those about whom he
frequently (e.g. in Met. 992b18-19) remarks that they did not
realize the ambiguity of 'existent' or 'one', viz. the Platonists.
bZ• Kal TOUS 0PLO'Il0US. The plural may be used (1) because A.
has to take account of the possibility of the term's being am-
biguous, or (2) because every problem involves two terms, the
subject and the predicate.
13-14. SLa. yap TWV Ka.ooAou . . . O'uAAoYLO'Il0S, i.e. syllogism
is impossible without a universal premiss; this has been shown
in ch. 24.
19. KaOa.'II'Ep Ka.l 'II'POTELvoIlEOa., 'which is also the form in which
we state our premisses'.25. ~t~'1'1fTaL ya.p (V lKElVOLS, 'for in assigning to things their
genera, we have assigned to them the attributes of the genera'.
26. lCat 8aa 11" U'lfa.PX~L, waauTws. This is true only if I-'~
Ima.PXEI be taken to mean 'necessarily do not belong'.
29-32. ouSE S" ... (KAoyl1s. This rule is complementary to
that stated in b22 -9. What it says is that in enumerating the
things of which a genus is predicable, we should not enumerate
the sub-species or individuals of which a species of the genus is
predicable, since it is self-evident that the genus is predicable of
them. We should enumerate only the species of which the genus
is immediately predicable.
3&-8. ETL Ta. 'lfClaLV E'lfollEva •.• S'1Aov. The reason for this
rule is stated in 44b2o-4 (where v. note) ; it is that if we select as
middle term an attribute which belongs to all things, and there-
fore both to our major and to our minor, we get two affirmative
premisses in the second figure, which prove nothing.
CHAPTER 28
Rules for categorical syllogisms, peculiar to different problems
43 b 39. If we want to prove a universal affirmative, we must
look for the subjects to which our predicate applies, and the
predicates that apply to our subject; if one of the former is
identical with one of the latter, our predicate must apply to our
subject.
43. If we want to prove a particular affirmative, we must look
for subjects to which both our terms apply.
44":£. If we want to prove a universal negative, we must look
for the attributes of our subject and those that cannot belong to
our predicate; or to those our subject cannot have and those that
belong to our predicate. We thus get an argument in the first or
second figure showing that our predicate cannot belong to our
subject.
9. If we want to prove a particular negative, we look for the
things of which the subject is predicable and the attributes the
predicate cannot have; if these classes overlap, a particular
negative follows.
I I . Let the attributes of A and E be respectively BI ... B n ,
Zl ... Z,., the things of which A and E are attributes r l . . .
HI ... H n , the attributes that A and E cannot have ..:11 .•. ..:1",
e l . . . 8,..
17. Then if any r (say r,.) is identical with a Z (say ZJ, (I)
4985
cc
rn'COMMENTARY
since all E is Zn. all E is rn. (2) since all rn is A and all E is rn.
all E is A.
19. If rn is identical with Hn. (I) since all rn is A. all Hn is A.
(2) since all Hn is A and is E. some Eis A.
21. If An is identical with Zn. (1) since no An is A. no Zn is A.
(2) since no Zn is A and all E is Zn. no Eis A.
25. If Bn is identical with en. (1) since no E is en. no E is Bn.
(2) since all A is Bn and no Eis Bn. no E is A.
28. If An is identical with Hn. (1) since no Lln is A. no Hn isA.
(2) since no Hn is A. and all Hn is E. some E is not A.
30. If Bn is identical with Hn. (1) since all Hn is E, all Bn is E.
(2) since all Bn is E and all A is Bn. all A is E, and therefore some
Eis A.
36. We must look for the first and most universal both of the
attributes of each of the two terms and of the things of which it is
an attribute. E.g. of the attributes of E we must look to KZ n
rather than to Zn only; of the things of which A is an attribute
we must look to KIn rather than to rn only. For if A belongs to
KZn it belongs both to Zn and to E; but if it does not belong to
KZn it may still belong to Zn. Similarly with the things of which
A is an attribute; if it belongs to Krn it must belong to rn. but
not vice versa.
b6. It is also clear that our inquiry must be conducted by means
of three terms and two premisses. and that all syllogisms are in
one of the three figures. For all E is shown to be A when rand Z
have been found to contain a common member. This is the
middle term and we get the first figure.
II. Some E is shown to be A when rn and Hn are the same;
then we get the third figure, with Hn as middle term.
12. No E is shown to be A. when An and Zn are the same; then
we get both the first and the second figure--":'the first because (a
negative proposition being convertible) no Zn is A. and all E is
Zn; the second because no A is An and all E is An.
16. Some E is shown not to be A when An and Hn are the same;
this is the third figure-No Hn is A, All Hn is E.
19. Clearly. then. (1) all syllogisms are in one or other of the
three figures; (2) we must not select attributes that belong to
everything. because no affirmative conclusion follows from con-
sidering the attributes of both terms, and a negative conclusion
follows only from considering an attribute that one has and the
other has not.
25. All other inquiries into the terms related to our given
terms are useless, e.g. (1) whether the attributes of each of the1. 28. 44az-II
two are the same, (2) whether the subjects of A and the attributes
E cannot have are the same, or (3) what attributes neither can
have. In case (I) we get a second-figure argument with two
affirmative premisses; in case (2) a first-figure argument with a
negative minor premiss; in case (3) a first- or second-figure
argument with two negative premisses; in no case is there a
syllogism.
38. We must discover which terms are the same, not which are
different or contrary; (I) because what we want is an identical
middle term; (2) because when we can get a syllogism by finding
contrary or incompatible attributes, such syllogisms are reducible
to the aforesaid types.
45 3 4. Suppose Bn and Zn contrary or incompatible. Then we
can infer that no E is A, but not directly from the facts named,
but in the way previously described. Bn will belong to all A and
to no E; so that Bn must be the same as some 8. [If Bn and Rn
are incompatible attributes, we can infer that some E is not A,
by t!le second figure, for all A is B n , and no E is Bn; so that Bn
must be the same as some en (which is the same thing as Bn and
Rn's being incompatible).J
17. Thus nothing follows directly from these data, but if Bn
and Zn are contrary, Bn must be identical with some 8 and that
gives rise to a syllogism. Those who study the matter in this way
follow a wrong course because they fail to notice the identity of
the B's and the 8's.
44"2-4. ~ ~EV ••• 1rap£LvaL. The full reading which I have
adopted (following the best MSS.) is much preferable to that
of AI. (preferred by Waitz), which has 0 for ~ in 3 2 and omits
El, Ta br6f-LElla, 0 8J 8Et f-L~ il7rapxnll. AI.'s reading is barely intel-
ligible, and its origin is easily to be explained by haplography.
7-8. YLvETaL ya.p ••• ~EO'~. The second alternative ("4-6)
clearly produces a syllogism in Camestres. The first alternative
("2-4) at first sight produces a second-figure syllogism (Cesare)
rather than one in the first figure. But A. has already observed
that it is not necessary to select things of which the major or minor
term is not predicable; it is enough to select things that are not
predicable of it, because a universal negative proposition is con-
vertible (43 b S-6). Thus he thinks of the data No P is M, All S
is M, as immediately reducible to No M is P, All S is M, which
produces a syllogism in the first figure (Celarent).
9-II. Ec1V 6£ . . . U1ra.PXELV. Similarly here A. thinks of the
data ~o P is M, All M is S, as reduced at once to No M is P,COMMENTARY
All M is S, yielding a syllogism in the third figure (Felapton);
for of course he does not recognize our fourth figure, to which the
original data conform.
11-35. Ma).).ov S' . . ..... EPOS. A.'s meaning can be easily
followed if we formulate his data ("12-17): All A is B1 ... B n ,
All r 1 • • • rn is A, No A is.1 1 • • . .1 n , All E is Zl ... Zn' All
H1 ... Hn is E, No E is e 1 ... en; each of the letters B, r,.1,
Z, H, e stands for a whole group of terms. In "17-35 A. shows
that a conclusion with E as subject and A as predicate follows
if any of the following pairs has a common member-r and Z,
rand H . .1 and Z, Band e . .1 and H, Band H. In b 25- 37 he
shows that nothing .follows from the possession of a common
member by the remaining pairs-B and Z, rand e, .1 and e.
Et SE TO r Ka.l TO H Ta.llTC1v (819-20) must be interpreted in the
light of the more careful phrase El 'Tav'T6 'Tt €aTI 'TWII r 'TIll, 'TWII Z
(°17); and 50 with the corresponding phrases in 821-2, 25, 28,
30-1, b2 6-R, 29-30, 34-5.
17. Et .... EV o~V ••• Z. The sense requires us to read iaTl for
22. EK 1rp0<7U).).OyL<7 .... 0U. The prosyllogism is No .1 n is A (since
No A is .1 n is convertible, "23). All Zn is .1 n • Therefore no Zn
is A ; the syllogism is No Zn is A, All E is Zn, Therefore no E is A.
31. a.VTE<7Tpa. .... ..-.EVOS E<7Ta.~ <7u).).oy~<7 .... os. The syllogism is
called all'Tfi<7'Tpap.p.b>o<; because (the fourth figure not being recog-
nized) the data are not such as to lead directly to a conclusion
with E as subject and A as predicate; our conclusion must be
converted. ,
34-5. TLvl S' ....... EPOS. I have adopted B's reading, which was
that of AI. (306. 16) and of P. (287. 10). all'TIaTpE</JElII means 'to be
convertible', and the universal is convertible into a particular
(All A is E into Some E is A). not vice versa. CL 31"27 SId 'TO
all'TI(T'Tp/<PElV 'TO Ka06>.ou 'Tip Ka'Ta IL/P0<;' and ib. 31-2, 51"4, 52b8-9,
67 b37.
36-b5. 41a.vEpov o~v ••• EyXWpEL. The primary method of proof
-that in Barbara ("17-19)-consists in finding a subject (Tn) of
which our major term (A) is predicable, which is identical with an
attribute (Zn) of our minor term (E). A. now recommends the
person who is trying to prove that all E is A to take the highest or
widest subject of which A is necessarily true (Kr n • i.e. the Ka06>'ou
which rn falls under), and the highest attribute which necessarily
belongs to E (KZn' the Ka06>.ou which Zn falls under). We have
then these data-All E is Zn. All Zn is KZn . All rn is Krn. All
Krn is A; whereas, before we took account of KZn , Krn, whatwe knew was simply that all E is Zn and all rn is A. The brevity
of A.'s account makes it difficult to see why he recommends this
course; but the following interpretation may be offered con-
jecturally. If we find that KZ n is identical with rn' or with
Krn, then all KZn is A and (since all Zn is KZn and all E is Zn)
it follows that all Zn is A and that all E is A; and All KZn is A
contains implicitly the statements All Zn is A, All E is A,
without being contained by them. It is thus the most pregnant
of the three statements and the one that expresses the truth most
exactly. since (when all three are true) it must be on the generic
character K Zn and not on the specific character Zn or on the
more specific character E that being A depends. If, on the other
hand, we find that we cannot say All K Zn is A. we can still fall
back on the question 'Is all Zn A ?', and if it is, we shall have
found an alternative answer to our search for a middle term
between E and A. Thus the method has two advantages: (I) it
gives us two possible middle terms, and (2) if KZ n is a true middle
term it is a better one to have than ZD' since it states more
exactly the condition on which being A depends. This is what
A. conveys in b I - 3. The next sentence repeats the point, stating
it, however, with reference to Kr n instead of K Zn' If Kr n
necessarily has the attribute A, then rn (which is a species of
Kr n) necessarily has it, and 'KT n is A' is more strictly true. since
it is not qua a particular species of KTn but qua a species of Kr n
that rn is A. If. on the other hand. Kr n is not necessarily A.
we may fall back on a species of it, and fmd that that is neces-
sarily A.
The upshot of the paragraph is that where there is a series of
middle terms between E and A. the preferable one to treat as
the middle term is that which stands nearest to A in generality.
It is more correct to say All Krn is A, All E is Krn, Therefore
all E is A, than to say All rn is A, All E is rn. Therefore all
Eis A.
AI. takes aUTO in b 3 to be E, and is able to extract a good sense
from b 3- S on the assumption. But the structure of the paragraph
makes it clear that b 3- S is meant to elucidate 840-bI (TaU SE A ...
P.01l01l). as bl - 3 is meant to elucidate 839-40 (TaU /-LEII E ... P.01l01l).
b8-I9. 5ELKVUTal ya.p ... T~ H. b8-IO answers to &17-19, b II - U
to &19-21, b I4 - IS to "2I-S. bI6-19 to 828-30. b IS - I 6 gives a new
proof that if Ll n and Zn are identical. no E is A, viz.: If all Zn
is LlD' (I) since all E is ZD' all E is Lln' (2) since no A is Ll n and
all E is Ll n• no E is A.
20-4. KaL on ••. I'TJ u1TIipX€lv. Both AI. and P. interpretCOMMENTARY
= dP.q,OTtpO~~, both major and minor term. But it is
hardly possible that A. should have used '7Tf1a~V so; we must
suppose him to mean what he says, that the attributes that are
common to all things. i.e. such terms as QV or EV, which stand
above the categories, should never. in the search for a syllogism.
be mentioned among the attributes of the extreme terms. Sup-
pose J1 is such a term. We cannot then get an affirmative con-
clusion. since All A is M. All E is M, proves nothing (as was
shown in 27818-20). and we could get a negative conclusion only
by making the false assumption that M is untrue of all or some
A or E.
That this. and not the interpretation given by AI. and P., is
correct is confirmed by the fact that the point they make, that
no use can be made of any attribute that belongs to both major
and minor term. is made as a new point just below. in b26-7:
zcr36. ~L I1EV ya.p ... au~.) .. oYLallos. (1) If Bn is identical with
Zo. then since all A is Bn. all A is Zo' But from All A is Zn and
All E is Zo nothing follows. (2) If rn is identical with en, then
since all rn is A, all en is A. But from All en is A and No E is
en nothing follows. (3) If ..1 n is identical with en. then (a) since
no ..1 n is A, no en is A ; but from No en is A and No E is en
nothing follows; (b) Since no A is ..1 0 ' no A is en; but from No
A is en and No E is en nothing follows.
38. l1.ip .. ov liE ••• ~1"TTEOV, i.c. KaL on >"7JTTTtOV EaTLV oTTo,a Tav-rct
Ean. on, though our only ancient evidence for it is the lemma of
AI. (which. as often. has on Ka{ instead of the more correct KaL on),
is plainly required by the sense.
45·4~. otov EL ••• e. A. here points out that from the data
'All A is Bn. All E is Zn' (the permanent assumptions stated in
44"12-15), 'Bn is contrary to (or incompatible with) Zn', expressed
in that form, we cannot infer that no E is A (since there is no
middle term entering into subject-predicate relations with A
and with E). But, he adds, we can get the conclusion No E is A
if we rewrite the reasoning thus: If no Zn is B n , (1) since all Eis
Zn' no E is Bn. (2) since all A is Bn and no E is B n , no Eis A. In
fact. he continues, since no E is Bn. Bn must be one of the e's
(the attributes no E can have)-which suggests an alternative
way of reaching the conclusion No E is A, viz. that which has
been given in 44"25-7.
crI6. 1Ta.~LV EL ••• ll1Ta.PXELV. A. (if the section be A.'s) now
turns to consider thc case in which Bn (a predicate of A) and Ho
(a subject of which E is predicable) are incompatible. Then. he
says, it can be inferred that some E is not A, and this, as in the
'7Tf1aW as39 I
case dealt with in "4-9, is done by a syllogism ill the second
figure.
At this point a difficult question of reading arises. In "12 n,
the first hand of B, and Al. (315. 23) read T~ S£ E oUS£v{.
ACd, the second hand of B, and P. (294. 23-4) read T~ S£ H
ouS£v{. probably as a result of Al.'s having offered this reading
conjecturally (316. 6). Waitz instead reads T~ S£ E ou TLV{ (in
the sense of TLV~ ~V; for the form cf. 24&19, 26b32, 59blO, 63b26-7).
With n's reading the reasoning will be: if no H n is B n , (1) since
all H n is E, no E is B n, (2) since all A is Bn and no E is B n, some
E is not A (second figure). With Al.'s conjecture the reasoning
will be: If no Hn is B n, (I) since all A is B n, no Hn is A (second
figure), (2) since no Hn is A and all Hn is E, some E is not A.
With Waitz's conjecture the reasoning will be: If no Hn is B n•
(I) since all Hn is E, some E is not B n, (2) since all A is Bn and
some E is not B n , some E is not A (second figure).
n's reading is clearly at fault in two respects; the inference that
no E is Bn involves an illicit minor, and the appropriate inference
from All A is Bn and No E is Bn is not Some E is not A, but No
E is A. Either of the conjectures avoids these errors.
But now comes a further difficulty. The clause, as emended
in either way, will not support the conclusion WUT' o.vo.yK7} TO B
Tau-rOIl TLVL £lvaL TWV e (the same as one of the attributes E cannot
have). With either reading all that follows is that Bn is an attri-
bute which some E does not possess. Al. recognizes the difficulty,
and points out (316. 18-20) that what really follows is not that Bn
is identical with one of the e's, but that H n is identical with one
of the J's. A. has in 44"28-30 and bl6--17 pointed out that this is
the assumption from which it follows that some E is not A.
On the other hand, the unemended reading in 45"12, if what it says
were true, would justify the conclusion that Bn is identical with
one of the e's.
Thus each of the three readings would involve A. in an elemen-
tary error with which it is difficult to credit him. Now it must
be noted that the next paragraph makes no reference to the
assumption that Bn and Hn are incompatible; it refers only to
the assumption that Bn and Zn are incompatible, which was
dealt with in "4-9. I conclude that "9-16 are not the work of A.,
but of a later writer who suffered from excess of zeal and lack of
logic.39 2
COMMENTARY
CHAPTER 29
Rules for reductio ad impossibile, hypothetical syllogisms, and
modal syllogisms
45·z3. Like syUogism, reductio ad impossibile is effected by
means of the consequents and antecedents of the two terms. The
same things that are proved in the one way are proved in the
other, by the use of the same terms.
z8. If you want to prove that No E is A, suppose some E to be
A ; then, since All A is B and Some E is A, Some E is B; but ex
hypothesi none was. So too we can prove that Some E is A, or the
other reI a tions ~etween E and A. Reductio is always effected by
means of the consequents and antecedents of the given terms.
36. If we have proved by reductio that No E is A, we can by
the use of the same terms prove it ostensively; and if we have
proved it ostensively, we can by the use of the same terms prove
it by reductio.
b 4 . In every case we find a middle term, which will occur in the
conclusion of the reductio syllogism; and by taking the opposite
of this conclusion as one premiss, and retaining one of the original
premisses, we prove the same main conclusion ostensively. The
ostensive proof differs from the reductio in that in it both premisses
are true, while in the reductio one is false.
IZ. These facts will become clearer when we treat of reductio;
but it is already clear that for both kinds of proof we have to look
to the same terms. In other proofs from an hypothesis the terms
of the substituted proposition have to be scrutinized in the same
way as the terms of an ostensive proof. The varieties of proof
from an hypothesis have still to be studied.
ZI. Some of the conclusions of ostensive proof can be reached in
another way; universal propositions by the scrutiny appropriate
to particular propositions, with the addition of an hypothesis. If
the r and the H were the same, and E were assumed to be true
only of the H's, all E would be A ; if the Ll and the H were the
same, and E were predicated only of the H's, no E would be A.
z8. Again, apodeictic and problematic propositions are to be
proved by the same terms, in the same arrangement, as assertoric
conclusions; but in the case of problematic propositions we must
assume also attributes that do not belong, but are capable of
belonging, to certain SUbjects.
36. It is clear, then, not only that all proofs can be conducted
in this way, but also that there is no other. For every proof has393
been shown to be in one of the three figures, and these can only
be effected by means of the consequents and antecedents of the
given tenns. Thus no other tenn can enter into any proof.
The object of this chapter is to show (I) that the same con-
clusions can be proved by reductio ad impossibile as can be
proved ostensively; (2) that for a proof by reductio, no less than
for an ostensive one, what we must try to find is an antecedent
or consequent of our major tenn which is identical with an
antecedent or consequent of our minor (i.e. wc must use the
method described in ch. 28). Incidentally A. remarks (I) that
the same scrutiny of antecedents and consequents is necessary
for arguments from an hypothesis--i.e. where, wanting to prove
that B is A, we assume that B is A if D is C, and then set
ourselves to prove that D is C-with the proviso that in this case
it is the antecedents and consequents of D and C, not of B and A,
that we scrutinize (4SbIS-I9) ; (2) that identities which, according
to the method described in ch. 28, yield a particular conclusion,
will with the help of a certain hypothesis yield a universal con-
clusion (ib. 21--8) ; and (3) that the same scrutiny is applicable to
modal as to pure syllogisms (ib. 28-3S).
45"27. KC1L 8 Iha. TOU ciSUVo.TOU, KilL S€lKTLKW5. In his treat-
ment of the moods of syllogism, A. has generally used reductio ad
impossibile as an alternative proof of something that can be
proved ostensively. But there were two exceptions to this. The
moods Baroco (27 a 36- b 3) and Bocardo (28 bIS-20) were proved by
reductio, without any ostensive proof being given (though yet
another mode of proof of Bocardo is suggested in 28b2o-I). But
broadly speaking A.'s statement is true, that the same premisses
will give the same conclusion by an ostensive proof and by a
reductio.
28-33. otov OTL TO A ... U'TTijPXEV. A. shows here how the con-
clusion (a) of a syllogism in Camestres (All A is B, No E is B,
Therefore no E is A) and (b) of a syllogism in Darapti (All H is A,
All H is E. Therefore some E is A) can be proved by reductio, from
the same premisses as are used in the ostensive syllogism.
b 4-8. 0",,0(W5 SE ... opwv. A. now passes from the particular
cases dealt with in a28- b 3 to point out that by the use of the same
middle tenn we can always construct (a) a reductio and (b) an
ostensive syllogism to prove the same conclusion. (a) The way to
construct a reductio is to find 'a tenn other than the two terms
which are our subject-matter' (i.e. which we wish to connect or dis-
connect) 'and common to them' (i.e. entering into true predicativeCOMMENTARY
relations with them), 'which will become a tenn in the conclu-
sion of the syllogism leading to the false conclusion'. E.g. if
we want to prove that some C is not A, we can do this if we can
find a tenn B such that no B is A and some C is B. Then by
taking one of these premisses (No B is A) and combining with it
the supposition that all C is A, we can get the conclusion No
C is B. Knowing this to be false, we can infer that the merely
supposed premiss was false and that Some C is not A is true.
(b) To get an ostensive syllogism, we have only to return to the
original datum whose opposite was the conclusion of the reductio
syllogism (aVTurTparp£LUTJC; -ramlc; -rTjc; 7TPO-ra.U£wc;, b6) (i.e. to assume
that some C is B), and combine with it the other original datum
(No B is A), and we get an ostensive syllogism in Ferio proving
that some C is not A.
u-13. T a.ilTa. ILEV o~v ... }..£YWILEV, i.e. in ii. 14.
15-19. Ev SE TOL5 a.}"hO~5 ••• im~}..£"'Ew5. Arguments KCtTC1
p.£-ra.AT/rfLV are those in which the possession of an attribute by
a tenn is proved by proving its possession of a substituted attri-
bute (-r6 fLfi-raAa/43avop.EvoV, 41a39). Arguments Ka-rd. 7ToLoT7/-ra, says
AI. (324. 19-325. 24), are those that proceed a7T6 -rou (1) p.ii.>..>..OV
Kat (2) t;-rTOV Kat (3) OP.OLOV, all of which 'accompany quality'.
(1) may be illustrated thus: Suppose we wish to prove that happi-
ness does not consist in being rich. We argue thus: 'If something
that would be thought more sufficient to produce happiness than
wealth is not sufficient, neither is that which would be thought
less sufficient. Health, which seems more sufficient than wealth,
is not sufficient. Therefore wealth is not.' And we prove that
health is not sufficient by saying 'No vicious person is happy,
Some vicious people are healthy, Therefore some healthy people
are not happy'. A corresponding proof might be given in mode
(2). (3) May be illustrated thus: 'If noble birth, being equally
desirable with wealth, is good, so is wealth. Noble birth, being
equally desirable with wealth, is good' (which we prove by saying
'Everything desirable is good, Noble birth is desirable, Therefore
noble birth is good'), 'Therefore wealth is good.'
Arguments Ka-rd. 7ToLoT7/-ra are thus one variety of arguments
KaTd. P.E-rrf>..T/rfLV, since a substituted tenn is introduced. In all
such arguments, says A. (if the text be sound), the UKirfLc;, i.e. the
search for subjects and predicates of the major and minor term,
and for attributes incompatible with the major or minor term
(43 b39-44·q), takes place with regard not to the terms of the
proposition we want to prove, but to the terms of the proposition
substituted for it (as something to be proved as a means to proving
394395
it). The reason for this is that, whereas the logical connexion
between the new term and that for which it is substituted is
established 0,' 0fL0>'oy{a!; if nvo!; ill?)!; lJTTO (}€a€w!;, the substituted
proposition is established by syllogism (41338-bl).
Maier (2 a. 282-4) argues that the expressions KaTa fLtETIi>.'T/rp,v,
KaTa 7TOWT'T/Ta are quite unknown in A.'s writings, and that orov .••
7To,oT1)Ta is an interpolation by a Peripatetic familiar with
Theophrastus' theory of the hypothetical syllogism, in which, as
we may learn from AI., these expressions were technical terms
(for a full account of Theophrastus' theory see Maier, 2 a. 263-87).
But since A. here uses the phrase £v TOr!; fL€TaAafL{3allofLlllo,!;, it
can hardly be said that he could not have used the phrase KaT a
fL€Tll.>.?)rp"" and it would be rash to eject 0[011 . . • 7To,oT1)Ta in face
of the unanimous testimony of the MSS., AI., and P.
19-20. E1rl<7KEljla.<79a.l !iE •.• U1r09E<7EWS. A. nowhere discusses
this topic in general, but reductio ad impossibile is examined in
n. 11-14.
22-8. (<7n!iE ••• Em~~E1rTEoV. I.e. the assumption that rn
(a subject of the major term A) is identical with Rn (a subject of
the minor term E), which in 44"19-21 proved that some E is A,
will, if we add the hypothesis that only Rn's are E, justify the
conclusion All E is A. ({I) All rn is A, All Rn is rn' Therefore
all Rn is A; (2) All Rn is A, All E is H n, Therefore all E is A.)
And the assumption that ..::tn (an attribute incompatible with A)
is identical with Rn' which in 44328-30 proved that Some E is
not A, will, if we add the hypothesis that only Rn's are E, justify
the conclusion No E is A. ({I) No..::t n is A, All Rn is ..::t n, Therefore
no Rn is A; (2) i\o Rn is A, All E is Rn' Therefore no E is A.)
Therefore it is useful to examine whether only the Rn's are E,
in addition to the connexions of terms mentioned in 44"12-35
£7TL{3>.E7TT/oII, 4Sb28).
28-31. TOV a.UTOV SE TP01rOV ••• au~~oYla .... os. This refers
to the method prescribed in ch. 28, i.e. to the use of the terms
designated A-EJ in 44"12-17.
32-4. SE!iUKTa.l yap .•. au~~oYlallos. This was shown in the
(Kal OVTW!;
chapters on syllogisms with at least one problematic premiss
(chs. 14-22).
34. 0lloLWS !iE ... Ka.T"lYOpu;Jv, i.e. propositions asserting that
it is OVllaToII, OU OvvaToII, OUK £II0EXOfLEIIOII, dOt/llaTov, OUK dOt/llaToII, OUK
dllaYKa,oll, d>''T/e£!;, OUK d>''T/e£!;, that E is A (De Int. 22aII-13).
Such propositions are to be established, says A., 0fLO{W!;, i.e. by
the same scrutiny of the antecedents and consequents of E and
A, and of the terms incompatible with E or A (43 b39-44a3S).COMMENTARY
CHAPTER 30
Rules proper to the several sciences and arts
4683. The method described is to be followed in the establish-
ment of all propositions, whether in philosophy or in any science;
we must scrutinize the consequents and antecedents of our two
terms, we must have an abundance of these, and we must pro-
ceed by way of three terms; if we want to establish the truth we
must scrutinize the antecedents and consequents really connected
with our subject and our predicate, while for dialectical syllogisms
we must have premisses that command general assent.
10. We have described the nature of the starting-points and
how to hunt for them, to save ourselves from looking at all that
can be said about the given terms, and limit ourselves to what is
appropriate to the proof of A, E, I, or 0 propositions.
17. Most of the suitable premisses state attributes peculiar to
the science in question; therefore it is the task of experience to
supply the premisses suitable to each subject. E.g. it was only
when the phenomena of the stars had been sufficiently collected
that astronomical proofs were discovered; if we have the facts we
can readily exhibit the proofs. If the facts were fully discovered
by our research we should be able to prove whatever was provable,
and, when proof was impossible, to make this plain.
28. This, then, is our general account of the selection of pre-
misses; we have discussed it more in detail in our work on
dialectic.
46"5. 1rEpt ElCaTEpov, i.e. about the subject and the predicate
between which we wish to establish a connexion.
8. EIC TWV lCaT' ciXft9Elav ~ha.YEypa.I'I'EVWV u1TIipXnv, i.e. from the
attributes and subjects (Ta tJ7TapxoV"Ta Kai ors- tJ7TaPXH, as) which
have been catalogued as really belonging to the subject or
predicate of the conclusion.
16. lCa9' ElCaaTOv ••• OVTWV. The infmitive is explained by the
fact that Sd is carried on in A.'s thought from "4 and all.
19. XEyw /)' orov T~V a.aTpoXoyLIC~v J-LEV EJ-L1rELplaV (sc. 8€, 7Tapa-
Soilva, Tas-) TllS a.aTpoXoYLlCllS E1rLaTTtJ-LTJS.
29-30. SL' a.lCpLI3ELas ..• /)LaXEICTLICTtV, i.e. in the Topics. parti-
cularly in I. 14. It is, of course, only the selection of premisses
of dialectical reasoning that is discussed in the Topics; the nature
of the premisses of scientific reasoning is discussed in the Posterior
Analytics.397
CHAPTER 31
Division
46"31. The method of division is but a small part of the method
we have described. Division is a sort of weak syllogism; for it
begs the point at issue, and only proves a more general predicate.
But in the first place those who used division failed to notice this,
and proceeded on the assumption that it is possible to prove the
essence of a thing, not realizing what it is possible to prove by
division, or that it is possible to effect proof in the way we have
described.
39. In proof, the middle term must always be less general than
the major term; division attempts the opposite-it assumes the
universal as a middle term. E.g. it assumes that every animal is
either mortal or immortal. Then it assumes that man is an
animal. What follows is that man is either mortal or immortal.
but the method of division takes for granted that he is mortal.
which is what had to be proved.
b u . Again, it assumes that a mortal animal must either have
feet or not have them, and that man is a mortal animal; from
which it concludes not (as it should) that man is an animal with
or without feet, but that he is one with feet.
:zoo Thus throughout they take the universal term as middle
term, and the subject and the differentiae as extremes. They
never give a clear proof that man is so-and-so; they ignore the
resources of proof that are at their disposal. Their method cannot
be used either to refute a statement, or to establish a property,
accident, or genus, or to decide between contradictory proposi-
tions, e.g. whether the diagonal of a square is or is not com-
mensurate with the side.
:Z9. For if we assume that every line is either commensurate or
incommensurate, and that the diagonal is a line, it follows that it
must be either commensurate or incommensurate; but if we infer
that it is incommensurate, we beg the question. The method is
useful, therefore, neither for every inquiry nor for those in which
it is thought most useful.
A. resumes his criticism of Platonic S,alpHnc; as a method of
proof, in An. Post. ii. 5. In An. Post. 96b25-97b6 he discusses
the part which division may play in the establishment of
definitions.
Maier (2 b. 77 n. 2) thinks that this chapter sits rather looselyCOMMENTARY
between two other sections of the book (chs. 27-30 on the mode
of discovery of arguments and chs. 32-45 on the analysis of them).
He claims that A. states in 46334-9, b22- 5 that definitions are not
demonstrable and that this presupposes the proof in A n. Post.
ii. 5-7 that this is so. 46b22-5 does not in fact say that definitions
are not demonstrable, but only that the method of division does
not demonstrate them; but 334-7 seems to imply that A. thinks
definitions not to be demonstrable, and Maier may be right in
inferring ch. 31 to be later than the proof of this fact in A n. Post.
ii. He is, however, wrong in thinking that the chapter has little
connexion with what precedes; it is natural that A., after ex-
pounding his own method of argument (the syllogism), should
comment on what he.regarded as Plato's rival method (division).
46"31-2. ~OTL S' ••• tSELV. The tone of the chapter shows that
/-UKPOV n fLOPLOV Ean means 'is only a small part'. ~ SLd 'TWV yEvWV
SLatpwL, is the reaching of definitions by dichotomy preached and
practised in Plato's Sophistes (219 a-237 a) and Politims (258 b-
267 c).
34. au>'>'oyLtETa.~ S' ••• a.VW9EV, i.e. what the Platonic method
of division does prove is that the subject possesses an attribute
higher in the scale of extension than the attribute to be proved.
37-8. t:JaT' OUTE ••• Etpt]Ka.I'EV. This sentence yields the best
sense if we read 0 'TL in "37 with AI. and P. For SLaLpoufLEVOL we
should read SLaLp0tJfLEVOtJ" with the MS. n, or SLaLpotJfLEVOL,. The
MSS. of P. vary between SLaLpotJfLEvOtJ, and SLaLpotJfLEVOL" and in AI.
335. II the best MS. corrected SLaLpotJfLEVTJ' (probably a corruption
of SLaLpotJfLEVOL, by itacism) into SLaLpotJfL€vOtJ,. The variants are best
explained QY supposing SLaLpotJfLEVOL, to have been the original
reading.
S TL EVSEXETa.L au>'>'0yLaa.a6a.L lha.LPOUI'EVOLS. What it is possible
to prove is, as A. proceeds to explain, a disjunctive proposition,
not the simple proposition which the partisans of divisior.
think they prove by it. oVrw, W, Elp~KafLEv refers to A.'s own
method, described in chs. 4-30.
3~2. EV I'EV o~v Ta.LS ci1ToSEL~EaLv ••• a.KPWV. In Barbara,
the only mood in which a universal affirmative (such as a
definition must be) can be proved, the major term must be at
least as wide as the middle term, and is normally wider.
b 2 2-4. TE>'OS SE ••• EtVa.L. A.'s meaning is expressed more
fully in A n. Post. 91 b24- 7 'Tt ydp KWAUEL 'TOV'TO ciA7JIU, fL~V 'TO 7Tav
dvaL Ka'Td 'TOV civfJpomotJ, fL~ fLEVTOL 'TO 'Tt Ean fL7JS£ 'TO 'Tt ~v ECvaL
S7JAOVV; En 'Tt KwAUEL ~ 7Tpoalh'ivat n ~ cirpEAE'iv ~ {l7TEp{3E{37JKEvaL
'Tii, ouata,;399
36-7. oiiT' EV ot5 •.. 1TPE1TEW, i.e. in the finding of definitions,
the use to which Plato had in the Sophistes and Politicus put the
method of division. Cf. "35-7.
CHAPTER 32
Rules for the choice of premisses, middle term, and figure
46b40' Our inquiry will be completed by showing how syllogisms
can be reduced to the afore-mentioned figures, and that will
confirm the results we have obtained.
47aIO. First we must extract the two premisses of the syllo-
gism (which are its larger elements and therefore easier to extract),
see which is the major and which the minor, and supply the miss-
ing premiss, if any. For sometimes the minor premiss is omitted,
and sometimes the minor premisses are stated but the major
premisses are not given, irrelevant propositions being introduced.
18. So we must eliminate what is superfluous and add what is
necessary, till we get to the two premisses. Sometimes the defect
is obvious; sometimes it escapes notice because something follows
from what is posited.
24. E.g., suppose we assume that substance is not destroyed by
the destruction of what is not substance, and that by the destruc-
tion of elements that which consists of them is destroyed. It
follows that a part of a substance must be a substance; but only
because of certain unexpressed premisses.
28. Again, suppose that if a man exists an animal exists, and
if an animal exists a substance exists. It follows that if a man
exists a substance exists; but this is not a syllogism, since the
premisses are not related as we have described.
31. There is necessity here, but not syllogism. So we must not,
if something follows from certain data, attempt to reduce the
argument directly. We must find the premisses, analyse them
into their terms, and put as middle term that which occurs in
both premisses.
40. If the middle term occurs both as predicate and as subject,
or is predicated of one term and has another denied of it, we have
the first figure. If it is predicated of one term and denied of the
other, we have the second figure. If the extreme terms are
both predicated, or one is predicated and one denied, of it, we
have the third figure. Similarly if the premisses are not both
universal.
b 7 • Thus any argument in which the same term is not mentioned
twice is not a syllogism, since there is no middle term. Since we4 00
COMMENTARY
know what kinds of premiss can be dealt with in each figure, we
have only to refer each problem to its proper figure. When it can
be dealt with in more than one figure, we shall recognize the figure
by the position of the middle term.
47'2-5. ~d yap ••• 'll'p09EaLS, £l yap 'T~II 'T£ yl"£a," nUll uvAAo-
Y'Uf'clJII 8£WPOLf'£1I points back to chs. 2-26; KO.' 'TOU £VplUKHII iXO'f'£1I
OVllo.f'''' to chs. 27-30; in O£ 'Tou. Y£Y£\IT}f'IlIov. allaAvo'f'£1I £l. 'Ta
7TPOHPTJf'IlIo. UX~f'o.Ta forward to chs. 32-45, especially to chs.
32-3, 42, 44. It is to this process of analysis of arguments into
the regular forms (the moods of the three figures) that the name
'Ta allaAunKd (A.'s own name for the Prior and Posterior Analytics)
refers. The use of the word a"o.'\v£," implies that the student has
before him an argument expressed with no regard to logical form,
which he then proceeds to 'break up' into its propositions, and
these into their terms. This use of allaAVHII may be compared
with the use of it by mathematical writers, of the process of dis-
covering the premisses from which a predetermined conclusion
can be derived. Ct. B. Einarson in A.J.P. lvii (1936),36---9.
There is a second use of allo.'\vHII (probably derived from that
found here) in which it stands for the reduction of a syllogism
in one figure to another figure. Instances of both usages are given
in our Index.
12. j.LE(tW SE ••• 6,V, i.e. the premisses are larger components
of the syllogism than the terms.
16-17. il TauTas ••• 'll'apaAE('II'ouaw. At first sight it looks as
if 'To.VTo.. meant 'both the premisses', and ~h' cLII o.o-ro., 7TEpaLIIO\I'To.,
the prior syllogisms by which they are proved; but a reference
to these would be irrelevant, since the manner of putting forward
a syllogism is not vitiated by the fact that the premisses are not
themselves proved. 'To.VTas- must refer to the minor premisses,
and 0,' Jjll o.OTa' 7T£po.lIlO\I'To., to the major premisses by which they
are 'completed', i.e. supplemented. So AI. 342. 15-18.
4o- b 5. 'Eav j.LEV o~v ••• EaxaTov. Ko.TTJyopfi in bI (bis), 3 is used
in the sense of 'accuses', sc. accuses a subject of possessing itself,
the predicate, i.e. 'is predicated', and Ko.TTJyopiiTa' in b I (as in
An. Post. 73 b I7) in the corresponding sense of 'is accused', sc.
of possessing an attribute. In 47 b 4, 5 Ko.'TTJyopij'To., is used in its
usual sense 'is predicated'. a7Tapvij'To., in b2 , 3, 4 is passive.
b S-6. olhw yap ••• j.LEaov, cf. 25b32-5, 26b34-8, 28"10-14.I. 32. 4782 - 33. 47 h 37
4 0 1
CHAPTER 33
Error of supposing that what is true of a subject in one respect is
true of it without qualification
47hIS' Sometimes we are deceived by similarity in the position
of the terms. Thus we might suppose that if A is asserted of B,
and B of C, this constitutes a syllogism; but that is not so. It is
true that Aristomenes as an object of thought always exists, and
that Aristomenes is Aristomenes who can be thought about; but
Aristomenes is mortal. The major premiss is not universal, as it
should have been; for not every Aristomenes who can be thought
about is eternal, since the actual Aristomenes is mortal.
29. Again, Miccalus is musical Miccalus, and musical Miccalus
might perish to-morrow, but it would not follow that Miccalus
would perish to-morrow; the major premiss is not universally true,
and unless it is, there is no syllogism.
38. This error arises through ignoring a small distinction-
that between 'this belongs to that' and 'this belongs to all of that'.
47b16. wcnTEp ELp1JTa.l 1I'pOTEpOV, in a3I-5.
17. 1I'a.pa. T~V b~Ol6'"1Ta. T115 TWV 8pwv 9EC7EW5, 'because the
arrangement of the terms resembles that of the terms of a syllo-
gism'.
21-9. iaTW ya.p •••• APlC7TO~€vOU5. elf{ €un Otavo7]Tds- ~Pt(]7'O­
ILEv7]S-·
o ~PWTOIL"''7}S- €(]7'~ Ota-
v07JTds- ~P/,(1TOILEv7]S- .
. '. 0 ~Pt(]7'OILfv7]S- ;(]7'tV
, ,
an.
This looks like a syllogism without being one. A. hardly does
justice to the nature of the fallacy. He treats its source as lying
in the fact that the first proposition cannot be rewritten as
1Tas- 0 Otavo7]Td, ~Pt(]7'OIL'V7], dEL ;(]7'tV, which it would have to be,
to make a valid syllogism in Barbara. But there is a deeper source
than this; for the statement that an Aristomenes can always be
thought of cannot be properly rewritten even as 'some Aristo-
menes that can be thought of exists for ever'.
The Aristomenes referred to is probably the Aristomenes who
is named as a trustee in A.'s will (D. L. v. 1. I2)-presumably a
member of the Lyceum.
29-37. 1I'a.AW iaTw . . . auXXoYla~o5. ' Miccalus is musical
Miccalus; and it may be true that musical Miccalus will perish
408S
DdCOMMENTARY
to-morrow, i.e. that this complex of substance and attribute will
be dissolved to-morrow by Miccalus' ceasing to be musical; but
it does not follow that Miccalus will perish to-morrow. A. treats
this (b 34- S) as a second example of confusion due to an indefinite
premiss being treated as if it were universal. But this argument
cannot be brought under that description. The argument he
criticizes is: Musical Miccalus will perish to-morrow, Miccalus
is musical Miccalus, Therefore Miccalus will perish to-morrow.
What is wrong with the argument is not that an indefinite major
premiss is treated as if it were universal, but that a premiss which
states something of a composite whole is treated as if the predicate
were true of every element in the whole. The confusion involved
is that between complex and element, not that between individual
and universal.
The name Miccalus is an unusual one; only two persons of the
name are recognized in Pauly-Wissowa. If the reference is to
any particular bearer of the name, it may be to the Miccalus who
was in 323 B.C. sent by Alexander the Great to Phoenicia and
Syria to secure colonists to settle on the Persian Gulf (Arrian,
An. 7. 19. S). We do not know anything of his being musical.
4 02
CHAPTER 34
Error due to confusion between abstract and concrete terms
47b40' Error often arises from not setting out the terms cor-
rectly. It is true that it is not possible for any disease to be
characterized by health, and that every man is characterized by
disease. It might seem to follow that no man can be characterized
by health. But if we substitute the things characterized for the
characteristics, there is no syllogism. For it is not true that it is
impossible for that which is ill to be well; and if we do not assume
this there is no syllogism, except one leading to a problematic
conclusion-'it is possible that no man should be well'.
4S"IS. The same fallacy may be illustrated by a second-figure
syllogism,
IS. and by a third-figure syllogism.
z4. In all these cases the error arises from the setting out of the
terms; the things characterized must be substituted for the
characteristics, and then the error disappears.
Chs. 34-41 contain a series of rules for the correct setting out of
the premisses of a syllogism. To this chs. 42-6 form an appendix.4882-15. otov Et .•• uyiEIa.V. From the true premisses Healthi-
ness cannot belong to any disease, Disease belongs to every man,
it might seem to follow that healthiness cannot belong to any
man; for there would seem to be a syllogism of the mood recog-
nized in 30"17-23 (EnAEn in the first figure). But the conclusion
is evidently not true, and the error has arisen from setting out
our terms wrongly. If we substitute the adjectives 'ill' and 'well'
for the abstract nouns, we see that the argument falls to the
ground, since the major premiss Nothing that is ill can ever be
well, which is needed to support the conclusion, is simply not
true. Yet (013-15) without that premiss we can get a conclusion,
only it will be a problematic one. For from the true premisses
It is possible tbat nothing that is ill should ever be well, It
is possible that every man should be ill, it follows that it is
possible that no man should ever be well; for this argument
belongs to a type recognized in 3301-5 as valid (EcAcEc in the
first figure).
15-18. miAw ... voaov. Here again we have a syllogism which
seems to have true premisses and a false conclusion: It is im-
possible that healthiness should belong to any disease, It is
possible that healthiness should belong to every man, Therefore
it is impossible that disease should belong to any man. But if
we substitute the concrete terms for the abstract, we find that
the major premiss needed to support the conclusion, viz. It is
impossible that any sick man should become well, is simply not
true.
According to the doctrine of 38"16-25 premisses of the form
which A. cites would justify only the conclusions It is possible
that disease should belong to no man, and Disease does not
belong to any man. Tredennick suggests voaoS' (sc. lnra.PXn) for
voaov (sc. £vo£xera.1 117Ta.PXHv) in a18. But the evidence for voaov
is very strong, and A. has probably made this slip.
18-23. EV SE T((J Tpinl' aX"f'an . . . a.AA"AOIS. While in the
first and second figures it was an apodeictic conclusion (viz. in
the first-figure example ("2-8) No man can be well, in the second-
figure example (016-18) No man can be ill) that was vitiated by
a wrong choice of terms, in the third figure it is a problematic
conclusion that is so vitiated. The argument contemplated is
such an argument as Healthiness may belong to every man,
Disease may belong to every man, Therefore healthiness may
belong to some disease. The premisses are true and the conclusion
false; and (821-3) this is superficially in disagreement with the
principle recognized in 39814-19, that Every C may be A, EveryCOMMENTARY
C may be E, justifies the conclusion Some E may be A. But the
substitution of adjectives for the abstract nouns clears up the
difficulty; for from 'For every man, being well is contingent, For
every man, being ill is contingent' it does follow that for something
that is ill, being well is contingent.
CHAPTER 35
Expressions for which there is no one word
48"2.9. We must not always try to express the terms by a
noun; there are often combinations of words to which no noun is
equivalent, and such arguments are difficult to reduce to syllo-
gistic form. Sometimes such an attempt may lead to the error of
thinking that immediate propositions can be proved by syllogism.
Having an'gles equal to two right angles belongs to the isosceles
triangle because it belongs to the triangle, but it belongs to the
triangle by its own nature. That the triangle has this property is
provable, but (it might seem) not by means of a middle term.
But this is a mistake; for the middle term is not always to be
sought in the form of a 'this'; it may be only expressible by a
phrase.
48a31-«). EVLOTE 8E •.. AEXOEVTOS. There may be a proposition
which is evidently provable, but for the proof of which there is
no easily recognizable middle term (as there is when we can say
Every E is an A, Every C is a E). In such cases it is easy to fall
into the error of supposing that the terms of a proposition may
have no middle term and yet the proposition may be provable.
We can say Every triangle has its angles equal to two right
angles, Every isosceles triangle is a triangle, Therefore every
isosceles triangle has its angles equal to two right angles. But
we cannot find a name X such that we can say Every X has angles
equal to two right angles, Every triangle is an X. It might seem
therefore that the proposition Every triangle has its angles equal
to two right angles is provable though there is no middle term
between its terms. But in fact it has a middle term; only this
is not a word but a phrase. The phrase A. has in mind would be
'Figure which has its angles equal to the angles about a point',
i.e. to the angles made by one straight line standing on another;
for in M et. 1051a24 he says SLa Tf. Suo dp8a, TO Tpf.ywvov; C5TL al 71'EP'
p.f.av aTLYp.~v ywvf.aL taaL Suo dp8a~,. El ovv aV'T/KTo ~ 71'apa ~v
71':\ Eupav, lSoVTL av 'iv Ev8", Si):\ov SLa Tf.. The figure implied iswhere CE is parallel to BA. Then LABC = LECD, and
LCAB = LACE, and therefore LABC + LCAB+ LBCA =
LECD+ LACE + LBCA = two right angles.
36--;. waT' OUK iaT<u ••• OVTOS. This is the apparent conclu-
sion from the facts stated in "35~. The triangle has its angles
equal to two right angles in virtue of itself; i.e. there is no wider
class of figures to which the attribute belongs directly, and there-
fore to triangle indirectly. It might seem therefore that though
the proposition 'The angles of a triangle are equal to two right
angles' is provable, it is not by means of a middle term. In fact
it is provable by means of a middle term, but only by that stated
in the previous note, which is a property peculiar to the triangle.
CHAPTER 36
The nominative and the obliqtte cases
48"40. We must not assume that the major term's belonging
to the middle term, or the latter's belonging to the minor, implies
that the one will be predicated of the other, or that the two pairs
of terms are similarly related. 'To belong' has as many senses as
those which 'to be' has, and in which the assertion that a thing is
can be said to be true.
b 4 . E.g. let A be 'that there is one science', and B be 'contra-
ries'. A belongs to B not in the sense that contraries are one
science, but in the sense that it is true to say that there is one
science of them.
10. It sometimes happens that the major term is stated of the
middle term, but not the middle term of the minor. If wisdom is
knowledge, and the good is the object of wisdom, it follows that
the good is an object of knowledge; the good is not knowledge,
but wisdom is.
14. Sometimes the middle term is stated of the minor but the
major is not stated of the middle term. If of everything that is a
quale or a contrary there is knowledge, and the good is a quale
and a contrary, it follows that of the good there is knowledge; theCOMMENTARY
good is not knowledge, nor is that which is a quale or a contrary,
but the good is a quale and a contrary.
20. Sometimes neither is the major term sta ted of the middle
term nor the middle of the minor, while the major (a) mayor (b)
may not be stated of the minor. (b) If of that of which there is
knowledge there is a genus, and of the good there is knowledge,
of the good there is a genus. None of the terms is stated of any
other. (a) On the other hand, if that of which there is knowledge
is a genus, and of the good there is knowledge, the good is a genus.
The major term is stated of the minor, but the major is not stated
of the middle nor the middle of the minor.
27. So too with negative statements. 'A does not belong to B'
does not always mean' B is not A'; it may mean 'of B (or for B)
there is no A'; e.g. 'of a becoming there is no becoming, but of
pleasure there is a becoming, therefore pleasure is not a becoming'.
Or 'of laughter there is a sign, of a sign there is no sign, therefore
laughter is not a sign'. Similarly in other cases in which the
negative answer to a problem is reached by means of the fact that
the genus is related in a special way to the terms of the problem.
3S. Again, 'opportunity is not the right time; for to God belongs
opportunity, but no right time, since to God nothing is advan-
tageous'. The tenns are right time, opportunity, God; but the
premiss must be understood according to the case of the noun.
For the terms ought always to be stated in the nominative, but
the premisses should be selected with reference to the case of each
term-the dative, as with 'equal', the genitive, as with 'double',
the accusative, as with 'hits' or 'sees', or the nominative, as in 'the
man is an animal'.
In this chapter A. points out that the word t),ITIJ.PXHIl, 'to belong',
which he has used to express the relation of the terms in a pro-
position, is a very general word, which ma.y stand for 'be pre-
dicable of' or for various other relations. Thus (to take his first
example) in the statement 'TWIl WUVTtWIl lOT< /-LLa E-man]/-LTf, he
treats as what is predicated 'that there is one science'; but
the sentence does not say 'contraries are one science', but 'of
contraries there is one science' (48b4---9).
A. says in 48b39-49aS that in reducing an argument to syllo-
gistic form we must pick out the two things between which the
argument establishes a connexion, and the third thing, which
serves to connect them. The names of these three things, in the
nominative case, are the terms. But his emphasis undoubtedly
falls on the second half of the sentence (4981-S). While these arethe three things we are arguing about, we must not suppose that
the relations between them are always relations of predicability;
we must take account of the cases of the nouns and recognize
that these are capable of expressing a great variety of relations,
and that the nature of the relations in the premisses dictates the
nature of the relation in the conclusion. A. never evolved a theory
of these relational arguments (of which A = B, B ~ C, Therefore
A = C may serve as a typical example), but the chapter shows
that he is alive to their existence and to the difficulties involved
in the treatment of them.
48"40. T~ &Kp~, i.e. to the minor term.
b:z- 3 . aAA' ouaXW5 •. ' • TOUTO, 'in as many senses as those in
which" B is A" and "it is true to say that B is A" are used'.
7-8. oUX tJUTE ••• E1I'LUT" .... TJV. It seems impossible to defend
the traditional reading, and AI. says simply ov yap luny Tj
TTp6TauLS Myovua 'Ta. €YUYT{a fLla€UTiy €mun)fL7J' (361. IS). P. has
the traditional reading, but has difficulty in interpreting it.
UVTCVV, at any rate, seems to be clearly an intruder from b8.
12. TOU S' aya90\l EUTLV ,; uo~(a. €muT~fL7J (which most of the
MSS. add after uoq,tU) , though AI. had it in hjs text and tries hard
to defend it, is plainly an intruder, and one that might easily
have crept into the text. We have the authority of one old and
good MS. (d) for rejecting it.
13-14. TO .... EV Si) aya90v OUK EUTLV E1I'LUT" .... TJ. A.'s point being
that the middle term is not predicated as an attribute of the minor
term, he ought to have said here TO fL€V S~ aya80v OVK €un uoq,{a.
But lman)fL7J is well supported (AI. 362. 19-21, P. 336. 23-8), and
the slip is a natural one.
20. EUTL SE ....
Bekker and Waitz have €un S€ OTE fL~TE, but
if OTE were read grammar would require it to be followed by OVTE.
KUT7)yope'iu8uL or Myeu8aL is to be understood.
24-']. Et S' ... AEYETClL. KUT' a,\).~'\wv S' ov MYETaL means 'the
major is not predicable of the middle term, nor the middle term
of the minor'. A. makes a mistake here. The major term is
predicated not only of the minor but also of the middle term
('that of which there is knowledge is a genus'). A. has carelessly
treated not 'that of which there is knowledge' but 'knowledge' as
if it were the term that occurs in the major premiss.
33-5. 0 .... OL1&I5 SE ••• yEv05. This refers to arguments in the
second figure (like the two arguments in Cesare in b3o-Z, 3Z-3)
in which 'the problem is cancelled', i.e. the proposed proposition
is negated, or in other words a negative conclusion is reached, on
the strength of the special relation (a relation involving the use
"TE.408
COMME0lTARY
of an oblique case) in which the genus, i.e. the middle term
(which in the second figure is the predicate in both premisses),
stands to the extreme terms. aUTO (sc. TO 1Tpof3A7]p..a) is used care-
lessly for the terms of the proposed proposition.
41. Tel.') KAtlUEl') TWV OVOfLa.TWV, i.e. their nominatives. Ct.
Soph. El. I73b40 iXOIlTWV 87]Aelas ~ appevos KAfjaw (cf. 182"18).
49"2-5. ~ yel.p ••• 1TPClTa.ULV, 'for one of the two things may
appear in the dative, as when the other is said to be equal to it,
or in the genitive, as when the other is said to be the double of
it, or in the accusative, as when the other is said to hit it or see
it, or in the nominative, as when a man is said to be an animal-
or in whatever other way the word may be declined in accordance
with the premiss in which it occurs'.
CHAPTER 37
The various kinds of attribution
490.6. That this belongs to that, or that this is true of that, ha:;
a variety of meanings corresponding to the diversity of the
categories; further, the predicates in this or that category may
be predicated of the subject either in a particular respect or
absolutely, and either simply or compounded; so too in the case
of negation. This demands further inquiry.
49"6-8. To S' IJ1Ta.PXElV ••• SlTIP"VTa.l, i.e. in saying' A belongs
to B' we may mean that A is the kind of substance B is, a quality
B has, a relation B is in, etc.
8. Ka.L Ta.UTa.') ~ 1Tn ~ (mAw,), i.e. in saying 'A belongs to B'
we mean that A belongs to B in some respect, or without quali-
fication.
ETl ~ (mAn,) ~ UUfL1TE1TAEYfLivas. e.g. (to take A1.'s examples)
we may say 'Socrates is a man' or 'Socrates is white', or we may
say 'Socrates is a white man'; we may say 'Socrates is talking' or
'Socrates is sitting', or we may say 'Socrates i:; sitting talking'.
9-10. E1TlUKE1TTiov SE .•• ~iATlov. This probably refers to all
the matter:; dealt with in this chapter. The words do not amount
to a promise; they merely say that these questions demand
further study.CHAPTER 38
The difference between proving that a thing can be known, and
proving that it can be known to be so-and-so
49"JJ. A word that is repeated in the premisses should be
attached to the major, not to the middle, term. E.g., if we want
to prove that 'of justice there is knowledge that it is good', 'that
it is good' must be added to the major term. The correct analysis
is: Of the good there is knowledge that it is good, Justice is good,
Therefore of justice there is knowledge that it is good. If we say
'Of the good, that it 'is good, there is knowledge', it would be
false and silly to go on to say 'Justice is good, that it is good'.
22. Similarly if we wanted to prove that the healthy is know-
able qua good, or the goat-stag knowable qua non-existent, or man
perishable qua sensible object.
27. The setting out of the terms is not the same when what is
proved is something simple and when it is qualified by some
attribute or condition, e.g. when the good is proved to be know-
able and when it is proved to be capable of being known to be
good. In the former case we put as middle term 'existing thing';
in the latter, 'that which is some particular thing'. Let A be
knowledge that it is some particular thing, B some particular
thing, C good. Then we can predicate A of B; for of some particular
thing there is knowledge that it is that particular thing. And we
can predicate B of C ; for the good is some particular thing. There-
fore of the gt>od there is knowledge that it is good. If 'existing
thing' were made middle term we should not have been able to
infer that of the good there is knowledge that it is good, but only
that there is knowledge that it exists.
49"11-22. To S' E1I'aVaSL1r~ouf1EvOV
uuvnov. 'That the
good is good can be known' is in itself as proper an expression as
'The good can be known to be good', and A. does not deny this.
What he points out is that only the latter form is available as a
premiss to prove that justice can be known to be good. To treat
the former expression as a premiss would involve having as the
other premiss the absurd statement 'Justice is that the good is
good'.
14. i1 ciya8ov. A. is here anticipating. The whole argument
in "12-22 deals with the question in which term of the syllogism
(to prove that there is knowledge of the goodness of justice) 'that
it is good' must be included. 'There is knowledge of justice in so
n4 10
COMMENTARY
far as it is good' is a different proposition, belonging to the type
dealt with in 822-S. But here also A.'s point is sound. If we want
to prove that justice in so far as it is good is knowable, we must
put our premisses in the form \\That is good is knowable in so far
as it is good, Justice is good. For if we begin by saying The good
in so far as it is good is knowable, we cannot go on to say Justice
is good in so far as it is good. This, if not .p£v8o" is at least OV
O"Vv£7"6v (822).
18. ,; yelp 8tKatOauVT) 01TEP ciya96v. 'for justice is exactly what
good is'. It would be stricter to say ~ ya.p 8LKaLocn1V7] 07T£P aya96v
7"£ (cf. b 7-8), 'justict: is identical with one kind of good', 'justice
is a species of the genus good'.
23. il TpayEXa~os
QV. sc. £7TtCT7"T}7"6v £O'n. Bekker, with the
second hand of B and of d, inserts 8o~a0'7"6v before
AI. and P.
interpret the clause as meaning 'the goat-stag is an object of
opinion qua not existing', but this is because they thought A.
could not have meant to say that a thing can be known qua not
existing; it is clear that P. did not read 8o~a0'7"6v (P. 34S. 16--18).
But in fact A. would not have hesitated to say 'the goat-stag qua
not existing can be known', sc. not to exist.-The 7"paylAacpo, was
'a fantastic animal, represented on Eastern carpets and the like'
(L. and S.); cf. De Int. 16 8 16, An. Post. 92b7, Phys. 208"30, Ar.
Ran. 937, PI. Rep. 488 a.
25. 1TpOS T~ a.KP'l'. to the major, not to the middle tenn.
27-b2. OuX'; aUT11 ••. opous. The point A. makes here is that
a more d«:;terminate middle tenn is needed to prove a subject's
possession of a more detenninate attribute.
37-8. Kal1Tpos T~ QKP'l' ..• EXiX9T). 'and if "existent", simply,
had been included in the fonnulation of the major tenn'; cf.
n . . "
n.
"2S-{i·
bI. EV TOLS EV .... Epn auXXoyta .... oLS. i.e. cITav 7"68£
avAAoYLu8fi (328).
7"£
~
7rfj ~ 7TW,
CHAPTER 39
5ztbstitution of equivalent expressions
49 b 3. We should be prepared to substitute synonymous ex-
pressions, word for word, phrase for phrase, word for phrase or
vice versa, and should prefer a word to a phrase. If 'the suppos-
able is not the genus of the opinabJe' and 'the opinable is not
identical with a certain kind of supposable' mean the same, we
should put the supposable and the opinable as our terms, instead
of using the phrase named.4 I I
A. makes here two points with regard to the reduction of
arguments to syllogistic form. (I) The argument as originally
stated may use more than three terms, but two of those which
are used may be different ways of saying the same thing; in such
a case we must not hesitate to substitute one word for another,
one phrase for another, or a word for a phrase or a phrase for a
word, provided the meaning is identical. (2) The EK8£at~, the
exhibition of the argument in syllogistic form, is easier if words
be substituted for phrases. This is, of course, not inconsistent
with ch. 35, which pointed out that it is not always possible to
find a single word for each of the terms of a syllogism.
49 b 6-c). otoy Et ••• 9ET~OY. A. sometimes uses 8o~d~(tV and
iJ1TOAafLf3dvHv without distinction, but strictly iJ1TOAafLf3dvHv im-
plies a higher degree of conviction than 8o~a.~nv, something like
taking for granted. Al. is no doubt right in supposing that A.
means to express a preference for the phrase 'TO 8o~aa'Tov OUK
<aTtv 01T(P V1TOA7J1TT6v 'Tt as compared with 'TO V1TOA7J1T'TOV OUK EaT'
y£vo~
'TOU
8o~aaTou.
CHAPTER 40
The difference between proving that B is A and proving that B is
the A
49bIO. Since 'pleasure is good' and 'pleasure is the good' are
different, we must state our terms accordingly; if we are to prove
the latter, 'the good' is the major term; if the former, 'good' is so.
CHAPTER 41
The difference between' A belongs to all of that to which B belongs'
and 'A belongs to all of that to all of which B belongs'. The 'setting
out' of terms is merely illustrative
49bI4' It is not the same to say 'to all that to which B belongs,
A belongs' and 'to all that, to all of which B belongs, A belongs'.
If 'beautiful' belongs to something white, it is true to say 'beauti-
ful belongs to white', but not 'beautiful belongs to all that is
white',
20. Thus if A belongs to B but not to all B, then whether B
belongs to all C or merely to C, it does not follow that A belongs
to C, still less that it belongs to all C.
22. But if A belongs to everything of which B is truly stated,COMMENTARY
A will be true of all of that, of all of which B is stated; while if
A is said (without quantification) of that, of all of which B is
said, B may belong to C and yet A not belong to all C, or to any C.
27. Thus if we take the three terms, it is clear that' A is said
of that of which B is said, universally' means' A is said of all the
things of which B is said'; and if B is said of all of C, so is A;
if not, not.
33. We must not suppose that something paradoxical results
from isolating the terms; for we do not use the assumption that
each term stands for an individual thing; it is like the geometer's
assumption that a line is a foot long when it is not-which he
does not use as a premiss. Only two premisses related as whole
and part can form the basis of proof. Our exhibition of terms
is akin to the appeal to sense-perception; neither our examples
nor the geometer's figures are necessary to the proof, as the pre-
misses are.
412
49bX4-3Z. OUK ~an ••. 1T<1VTOS. A.'s object here is to point
out that the premiss which must be universal, in a first-figure
syllogism, is the major. This will yield a universal or a particular
conclusion according as the minor is universal or particular; a
particular major will yield no conclusion, whether the minor be
universal or particular.
Maier points out (2 a. 265 n. 2) that this section forms the
starting-point of Theophrastus' theory about syllogisms KUTa.
lTpOaATJeftv. Cf. AI. 378. 12-379. 11.
In b26 AI. (377- 25--6) and P. (351. 8-10) interpret as if there
were a comma before KUTa. lTalJT()" taking these words with
>.iyt:Tat b 25 . But that would make A. say that if the major premiss
is universal, yet no conclusion need follow (~ OAw, fL~ tmapxuv).
He is really saying that if A is only said to be true of that, of all
of which B is said to be true, B may be true of C (not of all of
that), and yet A may not be true of all C, or may be true of no C.
Waitz correctly removed Bekker's comma before KUTa. IT<1VTO,.
In b 2 8 also Waitz did rightly in removing Bekker's comma
before lTUVTC),. The whole point is that the phrase KU()' 00 TO B
K<1Ta. lTUVTCl, TO A MYfTUt is ambiguous until we know whether KUTc1
lTaVTo, goes with what precedes or with what follows. What A.
says is that we have a suitable major premiss only if A is said
to be true of all that of which B is said, not if A is merely asserted
of that, of all of which B is asserted.
33-50'3. OU SEL .•• auXAoy~af!os. fKT{8fa8at and fK()fat, are
used in two distinct senses by A. (I) Sometimes they are used ofthe process of exhibiting the validity of a fonn of syllogism by
isolating in imagination particular cases (28"23, b14, 3°"9, Il, 12,
b 31 ). (2) S6metimes they are used of the process of picking out
the three tenns of a syllogism and affixing to them the letters
A, B, r (48"1,25,29,49, b6, 57"35)· Al. (379· 14), P. (352.3-7), and
Maier (2 a. 320 n.) think this is what is referred to here. In favour
of this interpretation is the fact that such an EK8e(lt> nol' OpWI' is,
broadly speaking, the subject which engages A. in chs. 32-45.
But it is open to certain objections. One is that it is difficult to
see what absurdity or paradox (n o.TOl'OI', 49 b33-4) could be sup-
posed to attach to this procedure. Another (which none of these
interpreters tries to meet) is that it affords no explanation of the
words ouS~1' yap l'poaxpwj.L£8a Tcji T6S£ n £fl'aL.
Waitz gives the other interpretation, taking A.'s point to be
that the selection of premisses which are in fact incorrect should
not be thought to justify objection to the method, since the
premisses are only illustrative and the validity of a fonn of syllo-
gism does not depend on the truth of the premisses we choose to
illustrate it. To this Maier objects that there has been no refer-
ence in the context to the use of examples, so that the remark
would be irrelevant. This interpretation, however, comes nearer
to doing justice to the words ouS£v yap l'pouX'pwj.L£8a Tcji T6S£ n
£Il'aL, since this might be interpreted to mean 'for we make no
use of the assumption that the particular fact is as stated in our
example'. But that is evidently rather a loose interpretation of
these words.
There is one passage that seems to solve the difficulty-Soph.
El. 178b36-179a8 Kat on Ean n, Tp{TO, o.l'8pWl'O, l'ap' aUrol' Kat TOl"
Ka8' EKaUTol" TO yap o.l'8pWl'O, Kat a.1Tal' TO KOLI'OI' ou T6Se n, aAAa
TOL6I'S£ n ~ 1TOaOI' ~ 1Tp6, n ~ TeOlI TOWV-rWI' n aTJj.La{l'n. OJ.Lo{w, SE KaL
(1T' TOU Kop{aKo, Ka, Kop{aKo, j.LoVaLK6" 1T6upol' TaUTOI' ~ EUpOI'; TO
j.LEII yap T 6 S £ n TO SE TOL6I1S£ GTJj.La{l'n, WUT' OUK EUTLII aUTO
f.K8Ea8aL· ou TO £KTl8ea(laL SE 1T0t€' Tal' Tp{TOI' o.I'8PW1T01', ilia
TO 01T£P T6S£ n dllaL CTVYXWP£'". OU yap EaTaL T6S£ n (fllaL 01T£P
KaMla, KaL 01T£P o.1'8PW1T6, f.aTLI'. ouS' £t n, TO f.Kn8Ej.L£I'OI' j.L~
01T£P T6S£ n dllaL MyOL aAA' CJ1T£P 1TOL61', OUSEI' Sw{an° (UTaL yap TO
1Tapa TOU, 1ToAAol" Ell n, 0[011 TO o.Il8pW1T0<;. Here the EK8wL<; of man
from individual men, and the £K8€aL<; of 'musical' from Coriscus,
is distinguished from the admission that 'man' or 'musical' is
a T6S£ n, and we are told that it is the latter and not the former
that gives rise to paradoxical conclusions. The same point is
put more briefly in Met. 1078"17-21.
Here, then, A. is saying that no one is to suppose thatCOMMENTARY
paradoxical consequences arise from the isolation of the terms
in a syllogism as if they stood for separable entities. Wc make
no use of the assumption that each term isolated is a ToSe n,
an individual thing.
With this usage of EKTLfJEa8at may be connected the passages in
which A. refers to the EK8eat!; of the One from the Many by the
Platonists (Met. 992b10, 10°3"10, 1086b 10, 1°9°"17).
37-50"1. o),.ws ya.p •.. aU),.),.0YLafloS, cf. 42"9-12 n.
50"Z. TOV flav9a.vOVT' ci.),.EYOVTlS. The received text has TOV
fLav8a.vovTa AEYOVTf!;, and Waitz interprets this as meaning TOV
fLav8a.vovra T<IJ EKT[8w8at Ka~ T<IJ ala8a.vw()at xpija()at AEYOVTf!;. This
is clearly unsatisfactory. it is not the learner but the teacher who
uses TO EKT[8w()at, and even if we take the reference to be simply
to TO ala8a.vw8at the grammar is very difficult; Phys. 189h32
<pafL~v yap y[yvw()at
Eg
c:i,uov aAAO Ka~
Eg
£TEPOV fTfPOV ~ Ta a.1TAa
MYOVTE<; ~ Ta aVYKE[p.Eva,
which Waitz cites, is no true parallel.
I have ventured to write TOV fLav8a.VOVT' dMyoVTE<;, 'in the interests
of the learner'. A. is not averse to the occasional use of a poetical
word; cf. for instance M et. 1090"36 Ta AEYOfLEVa . . . aa[VH-rTJV
,pVX'lV'
Pacius' 1TPO!; TOV fLav8a.VOVTU MyoVTf<; is probably con-
jectural.
CHAPTER 42
A nalysis of composite syllogisms
5°"5. We must recognize that not all the conclusions in one
argument are in the same figure, and must make our analysis
accordingly. Since not every type of proposition can be proved
in each figure, the conclusion will show the figure in which the
syllogism is to be sought.
5°"5-7. M~ ),.av9avETw ••• li),.),.ou. av,uoytap.o!; is here used of
an extended argument in which more than one syllogism occurs.
A. points out that in such an argument some of the conclusions
may have been reached in one figure, some in another, and that
the reduction to syllogistic form must take account of this.
~. (11"(1 8' ... TETaYflEva. All four kinds of proposition can
be proved in the first figure, only negative propositions in the
second, only particular propositions in the third.CHAPTER 43
In discussing definitio1%S, we must attend to the precise point at issue
50aII. When an argument has succeeded in establishing or
refuting one element in a definition, for brevity's sake that element
and not the whole definition should be treated as a term in the
syllogism.
50aII-15. Tous TE 1TPOS oP~O'l'ov ••• 9ET€OV. AI. and P. take
the reference to be to arguments aimed at refuting a definition.
But the reference is more general-to arguments directed towards
either establishing or refuting an element in the definition of
a tenn. For this use of 7Tp6; Waitz quotes parallels in 29"23,
40b39, 41"5-<), 39, etc. The object of OETEO)) is (Toiho) 7TPO; 0
S'ED.EKTtU. TOV; 7TPO; OP'UfLO)) TW)) A6yw)) is an accusativus pendens,
such as is not infrequent at the beginning of a sentence; cf.
52"29-30 n. and Kiihner, Gr. Gramm., § 412 . 3.
CHAPTER 44
Hypothetical arguments are not reducible to the figures
50&16. We should not try to reduce arguments ex hypothesi to
syllogistic form; for the conclusions have not been proved by
syllogism, they have been agreed as the result of a prior agree-
ment. Suppose one assumes that if there are contraries that are
not realizations of a single potentiality, there is not a single
science of such contraries, and then were to prove that not every
potentiality is capable of contrary realizations (e.g. health and
sickness are not; for then the same thing could be at the same
time healthy and sick). Then that there is not a single potentiality
of each pair of contraries has been proved, but that there is no
science of them has not been proved. The opponent must admit it,
but as a result of previous agreement, not of syllogism. Only the
other part of the argument should be reduced to syllogistic form.
29. So too with arguments ad impossibile. The reductio ad
impossibile should be reduced, but the remainder of the argument,
depending on a previous agreement, should not. Such arguments
differ from other arguments from an hypothesis, in that in the
latter there must be previous agreement (e.g. that if there has
been shown to be one faculty of contraries, there is one science of
contraries), while in the latter owing to the obviousness of the
falsity there need not be formal agreement-e.g. when we assumeCOMMENTARY
the diagonal commensurate with the side and prove that if it is,
odds must be equal to evens.
39. There are many other arguments ex hypothesi. Their
varieties we shall discuss later; we now only point out that and
why they cannot be reduced to the figures of syllogism.
50"16. TOU~ i~ ll1ro9£a£w~ aUAAoYLa~ou~, cf. 41"37-40 n., 45b15-
19 n.
19"""28. otOY £l ..• u1To9£aL~. Maier (2 a. 252) takes the imo9wt,
to be that if there is a single potentiality that does not admit
of contrary realizations, there is no science that deals with a pair
of contraries. But the point at issue is (as in 48b4~) not whether
all sciences are sciences of contraries, but whether every pair of
contraries is the object of a single science. The whole argument
then is this:
(A) If health and sickness were realizations of a single poten-
tiality, the same thing could be at the same time well
and ill, The same thing cannot be at the same time well
and ill, Therefore health and sickness are not realizations
of a single potentiality.
(B) Health and sickness are not realizations of a single poten-
tiality, Health and sickness are contraries, Therefore not
all pairs of contraries are realizations of a single poten-
tiality.
(C) If not all pairs of contraries are realizations of a single
potentiality, not all contraries are subjects of a single
science, Not all contraries are realizations of a single poten-
tiality, Therefore not all contraries are subjects of a single
science.
A. makes no comment on (A) ; the point he makes is that while
(B) is 'presumably' a syllogism, (C) is not. 'Presumably', i.e.,
he assumes it to be a syllogism, though he does not trouble to
verify this by reducing the argument to syllogistic fonn.
In 821 OUK [aTt 7Ta,ua ouvaj.Lt)' T(VV £vavTlwv is written loosely
instead of the more correct aUK [aTt j.Lla 1TCf,VTWV 'T(VV £vaVT{WV
ouvaj.Lt)' (&23)'
29-38. ·O~O(W~
SE ••• cipT(OL~. The nature of a reductio ad
impossibile (on which cf. 41"22-63 n.) is as follows: If we want to
prove that if all P is M and some 5 is not M, it follows that some
5 is not P, we say 'Suppose all 5 to be P. Then (A) All P is M,
All 5 is P, Therefore all 5 is M. But (B) it is known that some
5 is not M, and since All 5 is M is deduced in correct syllogistic
form from All P is M and All 5 is P, and All P is M is known tobe true, it follows that All 5 is P is false. Therefore Some 5 is
not P:
A. points out that the part of the proof labelled (A) is syllogistic
but the rest is not; it rests upon an hypothesis. But the proof
differs from other arguments from an hypothesis, in that while
in them the hypothesis (e.g. that if there are contraries that are
not realizations of a single potentiality, there are contraries that
are not objects of a single science) is not so obvious that it need
not be stated, in reductio ad impossibile 'TO "'£uSo~ is .pav£pov, i.e.
it is obvious that we cannot maintain both that all 5 is M (the
conclusion of (A)) and that some 5 is not M (our original minor
premiss). Similarly, in the case which A. takes ("37-8), if it can
be shown that the commensurability of the diagonal of a square
with the side would entail that a certain odd number is equal to
a certain even number (for the proof cf. 41"26-7 n.), the entailed
proposition is so obviously absurd that we need not state its
opposite as an explicit assumption.
40j,2. TivES IlEV o~v .•• EpoullEv. This promise is nowhere ful-
filled in A:s extant works.
CHAPTER 45
Resolution of syllogisms in one figure into another
SObS. When a conclusion can be proved in more than one figure,
one syllogism can be reduced to the other. (A) A negative syllo-
gism in the first figure can be reduced to the second; and an
argument in the second to the first, but only in certain cases.
9. (a) Reduction to the second figure (a) of Celarent,
13. (~) of Ferio.
17. (b) Of syllogisms in the second figure those that are uni-
versal can be reduced to the first, but of the two particular
syllogisms only one can.
19. Reduction to the first figure (a) of Cesare,
21. (~) of Cam est res,
25. (y) of Festino,
30. Baroco is irreducible.
33. (B) Not all syllogisms in the third figure are reducible to the
first, but all those in the first are reducible to the third.
35. (a) Reduction (a) of Darii,
38. (~) of Ferio.
51"1. (b) Of syllogisms in the third figure, all can be converted
into the first, except that in which the negative premiss is par-
ticular. Reduction (a) of Darapti,
~98s
Ee7. (fj) of Datisi,
8. (y) of Disamis,
12. (8) of Felapton,
IS. (E) of Ferison.
18. Bocardo cannot be reduced.
22. Thus for syllogisms in the first and third figure to be
reduced to each other, the minor premiss in each figure must be
converted.
26. (C) (a) Of syllogisms in the second figure, one is and one is
not reducible to the third. Reduction of Festino.
31. Baroco cannot be reduced.
34. (b) Reduction from the third figure to the second, (a) of
Felapton and (fj) of Ferison.
37. Bocardo cannot be reduced.
40. Thus the same syllogisms in the second and third figures
are irreducible to the third and second as were irreducible to the
first, and these are the only syllogisms that are validated in the
first figure by reductio ad impossibile.
50b31-2. OUT£" ya.p •.• uUXXOYL.<TIlOS. The universal affirma-
tive premiss cannot be simply converted, and if it could, and we
tried to reduce Baroco to the first figure by converting its major
premiss simply, we should be committing an illicit major.
34. o~ S' iv T~ TrPWTltI 1Ta.VT£"S, i.e. all the moods of the first
figure which have such a conclusion as the third figure can prove,
i.e. a particular conclusion.
5Ia22. Ta. UX"Ila.Ta., i.e. the first and third figures.
26-33. TWV S' ... Ka.9oXou. Of the moods of the second figure,
only two could possibly be reduced to the third figure, since only
two have a particular conclusion. Of these, Festino is reducible;
Baroco is not, since we cannot get a universal proposition by
converting either premiss (the major premiss being convertible
only per accidens, the minor not at all).
34-5' Ka.L O~ iK TOU Tphou ••• UTEpTJTlKOV. Of the moods of
the third figure, only three could possibly be reduced to the
second, since only three have a negative conclusion. Of these
Felapton and Ferison are reducible, Bocardo is not.
4~b2. c1Ja.VEPOV o~v ••• 1TEpa.[VOVTa.L, i.e. (I) in considering
conversion from the second figure to the third and vice versa, we
find the same moods to be inconvertible as were inconvertible
to the first figure, viz. Baroco and Bocardo; (2) these are the same
moods which could be reduced to the first figure only by reductio
ad impossibile (27 a 36- b3, 28bI5-20).CHAPTER 46
Resolution of arguments involving the expressions 'is not A' and
'is not-A'
SlbS. In the establishment or refutation of a proposition it is
important to determine whether 'not to be so-and-so' and 'to be
not-so-and-so' have the same or different meanings, They do not
mean the same, and the negative of 'is white' is not 'is not-white',
but 'is not white',
10. The reason is as follows: (A) The relation of 'can walk' to
'can not-walk', or of 'knows the good' to 'knows the not-goad',
is similar to that of 'is white' to 'is not-white', For 'knows the
good' means the same as 'is cognisant of the good', and 'can walk'
as 'is capable of walking'; and therefore 'cannot walk' the same
as 'is not capable of walking', If then 'is not capable of walking'
means the same as 'is capable of not-walking', 'capable of walk-
ing' and 'not capable of walking' will be predicable at the same
time of the same person (for the same person is capable of walking
and of not walking) ; but an assertion and its opposite cannot be
predicable of the same thing at the same time.
u. Thus, as 'not to know the good' and 'to know the not-
good' are different, so are 'to be not-good' and 'not to be good'.
For if of four proportional terms two are different, the other two
must be different.
:zS. (B) Nor are 'to be not-equal' and 'not to be equal' the same;
for there is a kind of subject implied in that which is not-equal,
viz. the unequal, while there is none implied in that which merely
is not equal. Hence not everything is either equal or unequal, but
everything either is or is not equal.
:z8. Again, 'is a not-white log' and 'is not a white log' are not
convertible. For if a thing is a not-white log, it is a log; but that
which is not a white log need not be a log.
31. Thus it is clear that 'is not-good' is not the negation of 'is
good'. If, then, of any statement either the predicate 'affirma-
tion' or the predicate 'negation' is true, and this is not a negation,
it must be a sort of affirmation, and therefore must have a
negation of its own, which is 'is not not-good',
36. The four statements may be arranged thus:
'Is good' (A)
'Is not good' (B)
'Is not not-good' (D)
'Is not-good' (C).
Of everything either A or B is true, and of nothing are both true;
so too with C and D. Of everything of which C is true, B is trueCOMMENTARY
(since a thing cannot be both good and not-good, or a white log
and a not-white log). But e is not always true of that of which B
is true; for that which is not a log will not be a not-white log.
S~a6. Therefore conversely, of everything of which A is true,
D is true; for either e or D must be true of it, and e cannot be.
But A is not true of everything of which D is true; for of that
which is not a log we cannot say that it is a white log. Further,
A and e cannot be true of the same thing, and Band D can.
IS. Privative terms are in the same relation to affirmative
terms, e.g. equal (A), not equal (B), unequal (e), not unequal (D).
18. In the case of a number of things some of which have an
attribute while others have not, the negation would be true as in
the case above; we can say 'not all things are white' or 'not
everything is white' ; but we cannot say 'everything is not-white'
or 'all things are not-white'. Similarly the negation of 'every
animal is white' is not 'every animal is not-white', but 'not every
animal is white'.
24. Since 'is not-white' and 'is not white' are different, the one
an affirmation, the other a negation, the mode of proving each is
different. The mode of proving that everything of a certain kind
is white and that of proving that it is not-white are the same, viz.
by an affirmative mood of the first figure. That every man is
musical, or that every man is unmusical, is to be proved by
assuming that every animal is musical, or is un musical. That no
man is musical is to be proved by anyone of three negative moods.
39. When A and B are so related that they cannot belong to
the same subject and one or other must belong to every subject,
and rand LI are similarly related, and r implies A and not vice
versa, (1) B will imply LI, and (2) not vice versa; (3) A and LI are
compatible, and (4) Band r are not.
b4. For (1) since of everything either r or J is true, and of that
of which B is true, r must be untrue (since r implies A), LI must
be true of it.
8. (3) Since A does not imply r, and of everything either r or
LI is true, A and LI may be true of the same thing.
10. (4) Band r cannot be true of the same thing, since r
implies A.
12. (2) LI does not imply B, since LI and A can be true of the
same thing.
q. Even in such an arrangement of terms we may be deceived
through not taking the opposites rightly. Suppose the conditions
stated in a39-b2 fulfilled. Then it may seem to follow that LI
implies B, which is false. For let Z be taken to be the negation
4204 21
of A and B, and e that of rand..:1. Then of everything either
A or Z is true, and also either r or e. And ex hypothesi r implies
A. Therefore Z implies 8. Again, since of everything either Z
or B, and either e or..:1, is true, and Z implies e, ..:1 will imply B.
Thus if r implies A, Ll implies B. But this is false; for the
implication was the other way about.
z9. The reason of the error is that it is not true that of every-
thing either A or Z is true (or that either Z or B is true of it) ; for
Z is not the negation of A. The negation of good is not 'neither
good nor not-good' but 'not good'. So too with rand Ll; we have
erroneously taken each tenn to have two contradictories.
The programme stated in 32. 47"2-5, £L .•. TOU, Y£Y£IIT}Il-£1I0V,
(sc. t7t1AAOYWIl-0U,) aVaAUOLIl-£V £L, Ta. TTpOnpT}ll-£vu UXr/Il-UTU, T£>'O, all £XOL
~ 19 apxfj, TTPO!JWL" has, as A. says in 51°3-5, been fulfilled in chs.
32-45. Ch. 46 is an appendix without any close connexion with
what precedes. But, as Maier observes (2 a. 324 n. I), this need not
make us suspect its genuineness, for we have already had in
chs. 32-45 a series of loosely connected notes. Maier thinks
(2 b. 364 n.) that the chapter fonns the transition from A n. Pr. I
to the De I nterpretatione. He holds that the recognition of the
axioms of contradiction and excluded middle (51°20-2,32-3) pre-
supposes the discussion of them in the Metaphysics (though in
a more general way they are already recognized in A n. Pr. I and 2,
Cat., and Top.)-reftection on the axioms having cleared up for
A. the meaning and place of negation in judgement, and ch. 46
being the fruit of this insight. At the same time he considers the
chapter to be earlier than the De I nterpretatione, on the grounds
that once A. had undertaken (in the De Interpretatione) a separate
work on the theory of the judgement, it would have been in-
appropriate to introduce one part of the theory into the discussion
of the theory of syllogism, and that the discussion in De Int. 10
presupposes that in the present chapter.
These views cannot be said to be very convincing. It seems to
me that A. might at any time in his career have fonnulated the
axioms of contradiction and excluded middle as he does here,
since they had already been recognized by Plato; and though
De Int. 19°31 has a reference (which may well have been added
by an editor) to the present chapter, the De I nterpretatione as
a whole seems to be an earlier work than the Prior A nalytics,
since its theory of judgement stands in the line of development
from Sophistes 261 e ft. to the Prior Analytics (cf. T. Case in
Enc. BritY ii. 511-12). Maier's view (A.C.P. xiii (1900), 23-72)422
COMMENTARY
that the De Interpretatione is the latest of all A.'s works and was
left unfinished is most improbable, and may be held to have·
been superseded by Jaeger's conclusions as to the trend of A.'s
later thought.
A. first tries to prove the difference between the statement
'A is not B' and the statement 'A is not-B', using an argument
from analogy drawn from the assumption that' A is B' is related
to 'A is not-B' as 'A can walk' is related to • A can not-walk',
and as 'A knows the good' to 'A knows the not-good' (SlbIO-13).
This in turn he supports by pointing out that the propositions
'A knows the good', 'A can walk' can equally well be expressed
with an explicit use of the copula 'is'-' A is cognizant of the
good', 'A is capable of walking'; and that their opposites can
equally well be expressed in the form • A is not cognizant of the
good', 'A is not capable of walking' (b I3 - 1 6). He then points out
that if . A is not capable of walking' meant the same as • A is
capable of not walking', then, since he who is capable of not
walking is also capable of walking, it would be true to say of the
same person at the same time that he is not capable of walking
and that he is capable of it; which cannot be true. A similar
impossible result follows if we suppose' A does not know the good'
to mean the same as 'A knows the not-good' (bI 6-22). He con-
cludes that, since the relation of 'A is B' to 'A is not-B' was
assumed to be the same as that of 'A knows the good' to • A knows
the not-good' -sc. and therefore that of (i) 'A is not B' to (ii)
'A is not-B' the same as that of (iii) 'A does not know the good' to
(iv) 'A knows the not-good'-and since (iii) and (iv) have been seen
to mean different things, (i) and (ii) mean different things (b 22- S).
The argument is ingenious, but fallacious. 'A is B' is related to
'A is not-B' not as 'A can walk' to • A can not-walk', or as 'A
knows the good' to 'A knows the not-good', but as • A is capable
of walking' to • A is not-capable of walking', or as . A is cognizant
of the good' to 'A is not-cognizant of the good', and thus the
argument from analogy fails.
It is not till b 2S that A. comes to the real ground of distinction
between the two statements. He points out here that being
not-equal presupposes a definite nature, that of the unequal, i.e.
presupposes as its subject a quantitative thing unequal to some
other quantitative thing, while not being equal has no such pre-
supposition. In b28-32 he supports his argument by a further
analogy; he argues that (I) 'A is not good' is to (2) 'A is not-
good' as (3) • A is not a white log' is to (4) • A is a not-white log',
and that just as (3) can be true when (4) is not, (I) can be truewhen (2) is not.
The analogy is not a perfect one, but A:s main
point is right. Whatever may be said of the form' A is not-B',
which is really an invention of logicians, it is the case that such
predications as 'is unequal', 'is immoral' (which is the kind of
thing A. has in mind-note his identification of p.~ '0011 with a.1I~001l
in b2S -S) do imply a certain kind of underlying nature in the subject
(im6lCHTat n, b 2 6), while 'is not equal', 'is not moral' do not.
52"15-17. 'OfLOLW$ 6' ... d. A. means that what he has said
in 51 b36-S2a14 of the relations of the expressions 'X is white',
'X is not white', 'X is not-white', 'X is not not-white' can equally
be said if we substitute a privative term like 'unequal' for an
expression like 'not-white'. OVIC '0011, OVIC alllUolI here stand not
for Eunll OVIC rOOIl, Eonll OVIC all~ooll, but for OVIC Eunll '0011, OVIC
EcrrtV avtO'ov.
18-24. Ka.t ~1Tt 1ToAAwv SE ••• AEUKOV. A. now passes from the
singular propositions he has dealt with in Sl bS-S2 a q to proposi-
tions about a class some members of which have and others have
not a certain attribute, and says (a) that the fact that 'not all
so-and-so's are white' may be true when 'all so-and-so's are not-
white' is untrue is analogous (op.otw>, "19) to the fact that 'X is
not a white log' may be true when 'X is a not-white log' is
untrue (84-5); and (b) that the fact that the contradictory of
'every animal is white' is not 'every animal is not-white' but 'not
every animal is white' is analogous (op.otw>, 322) to the fact that
the contradictory of 'X is white' is not 'X is not-white' but
'X is not white' (SlbS-IO).
29-30. aAAa. TO fLEv ••• TP01TO$. TOiJ P.(II (which n reads) would
be easier, but Waitz points out that A. oft-en has a similar
anacolouthon; instances in An. Pr. may be seen in 47b13, 50811 n.,
b S. 'With regard to its being true to say ... the same method of
proof applies:
34-5. Et 6" ... fL" fLOU<TlKOV EtVa.l. It is necessary to read
EOTa~, not ;OT~II. 'If it is to be true', i.e. if we are trying to prove
it to be true, Al:s words (412. 33) a {3ov>'6p.(8a S(,ga~ on 71'a>
a1l8pw71'0> leT>.. point to the reading ;OTa~.
38. Ka.Ta. TOU$ EtpT)fLEVOU$ TP01TOU$ TPEi$, i.e. Celarent ( 2S b4 D-
26"2), Cesare (27"5---9), Camestres (ib. 9-14).
39-bI3. 'A1TAW$ 6' . . . ~11T6.PXEW. In Slb36-S2"14 A. has
pointed out that (A) 'X is good'. (B) 'X is not good', (C) 'X is
not-good', (D) 'X is not not-good' are so related that (I) of any
X, either A or B is true, (2) of no X can both A and B be true,
(3) of any X, either C or D is true, (4) of no X are both C and
D true, (5) C entails B, (6) B does not entail C, (7) A entails D,COMMENTARY
(8) D does not entail A, (9) of no X are both A and C true, (10)
of some X's both Band D are true. He here generalizes with
regard to any four propositions A, B, C, D so related that condi-
tions (I) to (4) are fulfilled, i.e. such that A and B are contra-
dictory and C and D are contradictory. But he adds two further
conditions-not, as above, that C entails B and is not entailed
by it, but that C entails A and is not entailed by it. Given these six
conditions, he deduces four consequences: (I) B implies D (b2,
proved b4 -8) , (2) D does not imply B (b2- 3 , proved bI2- 13 ), (3) A
and D are compatible (b3 , proved b8-IO), (4) Band C are not
compatible (b 4 , proved b1 o- I2 ). The proof of (2) is left to the
end because (3) is used in proving it.
bS. 1TuALV (1TlL T~ A TO r OUK cl.VTLC7TplCPlL. 76 A 70 r has
better MS. authority, but (as Waitz points out) it is A.'s usage,
when the original sentence is 70 r 76 A inrapxn, to make 76 r
the subject of o.v-ncrrplq,n. Ct. 31"32, 51"4, 67b3~, 68"22, b2 6.
P. (382. 17) had 70 A 76 r.
14-34' IUf1~a.(VlL S' ... lt17LV. A. here points out that if we
make a certain error in our choice of terms as contradictories,
it may seem to follow from the data assumed in a39-b2 (viz. (I)
that A and Bare contradictories, (2) that rand Ll are contra-
dictories, (3) that r entails A) that Ll entails B, which we saw
in bI2- 13 to be untrue.
The error which leads to this is that of assuming that, if we
put Z = 'neither A nor B', and suppose it to be the contra-
dictory both of A and of B, and put e = 'neither r nor Ll', and
suppose it to be the contradictory both of r and of Ll, we shall go
on to reason as follows: Everything is either A or Z, Everything
is either r or e, All r is A, Therefore (I) all Z is e. Everything is
either Z or B, Everything is either e or Ll, All Z is e ((I) above),
Therefore (2) all Ll is B. The cause of the error, A. points out in
b 29 - 33 , is the assumption that A and Z (= 'neither A nor B'),
and again B and Z, are contradictories. The contradictory of
'good' is not 'neither good nor not-good', but 'not good'. And the
same error has been made about rand Ll. For each of the four
original terms we have assumed two contradictories (for A, B
and Z; for B, A and Z; for r, Ll and e; for Ll, rand 8);
but one term has only one contradictory.
7.7. TO(lTO ya.p 'L17f1lV, since we proved in b4-8 that if one member
of one pair of contradictories entails one member of another pair,
the other member of the second pair entails the other member of
the first.
7.8-c). cl.vu1Ta.ALV yup ••• cl.KoAou91lI7LS, cf. b 4 -8.