
What is said to be COMPLETE is:

[1]     one outside which not even one part is found

        The complete time of each thing is the one outside which
        there is no time to be found.

[2]     that which, as regards virtue or the good, cannot be surpassed
        relative to its kind

        A doctor is complete or a flute-player when they lack nothing as regards
        the form of their own proper virtue since virtue is a sort of completion. 
        For each thing is complete and each substance is complete when,
        as regards the form of its proper virtue, it lacks no part of its proper magnitude.

[3]     things that have attained their end, this being something excellent

        For things are said to be complete in virtue of having attained their end.
        And the end -- that is, the for-the-sake-of-which -- is a last thing.

This, then, is the number of ways in which things said to be intrinsically complete
are said to be such, some because with respect to goodness they are deficient
in nothing and can neither be exceeded nor can anything of theirs be found
outside them, others generally in that they can not be exceeded in their
several kinds and have nothing of theirs outside them.

The rest are said to be complete because with reference to these
because they either produce or possess something of the relevant sort.
or are fitted for it, or in some way or other are said to be complete
with reference to the things that are said to be such in the primary way.
