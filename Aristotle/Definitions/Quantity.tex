
Something is said to be a QUANTITY:

[1a]    when it is divisible into components,
        each of which is naturally a one and a this something

[1b]    when it is a plurality if it is countable,
        and it is potentially divisible into non-continuous parts
        
        A limited plurality is a number.

[1c]    when it is a magnitude if it is measurable,
        and it is potentially divisible into continuous parts
        
        Continuous in one dimension is a length, in two a breadth, in three a depth. 
        limited length a line, limited breadth a surface, limited depth a body.

[2]     intrinsically

[2a]    in virtue of their substance

[2b]    attributes and states of such substances

        For example, much and little, long and short, broad and narrow, deep and shallow.

[2c]    both the great and the small and the greater and the smaller

        Either when said to be such intrinsically or in relation to each other,
        however these names are often transferred to others.

[3]     coincidentally

[3a]    in virtue of there being a certain quantity to which they belong

[3b]    in which a movement and a time are, certain quantities and continuous
        because of the divisibility of the things of which they are attributes.
        since the amount of the movement being a quantity, the change too is a quantity
