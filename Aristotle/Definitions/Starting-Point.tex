
Something is said to be a STARTING-POINT if it is:

[1]     a point in the thing which we would move first

        Of a line or of a road there is this starting-point from here,
        and another from the contrary direction.

[2]     the one from which each thing would best come to be

        Even in learning we must sometime begin not with what is primary,
        that is the starting-point of the thing, but from
        the point from which it is easiest to learn.

[3]     the component from which a thing first comes to be

        As of a ship the keel does and of a house the foundation,
        whereas of animals some take it that the heart does,
        some the brain, and others some other thing.

[4]     the non-component from which a thing first comes to be,
        from which movement and change naturally first begin.

        A child  comes to be from its mother and father,
        and a fight from abusive language.

[5]     the one in accord with whose deliberative choice what is moved is moved

        The rulers in cities, dynasties, and kingships are said to be archai.
        as are crafts, especially architectonic ones.

[6]     the one from which a thing can first be known

        The hypotheses are the starting-points of demonstrations.

Things are also said to be causes in an equal number of ways,
since all causes are starting points.
The first thing from which a thing is or comes to be, or is known,
but of these some are components and some are external.

[7]     starting-points:
[7a]    the nature of a thing
[7b]    the element of a thing
[7c]    thought
[7d]    deliberative choice
[7e]    substance
[7f]    for-the-sake-of-which

For of many things the starting point both of knowledge and movement is
the good and the noble.
