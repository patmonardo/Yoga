Something is said to HAVE in many ways:

[1]     to treat it in accord with the possessor's own nature or
        in accord with its own impulse

        A fever is said to have a hold on a human being and 
        tyrants to have a hold of their cities, and
        people to have the clothes they wear.

[2]     what a thing is present in, as in something receptive of it,
        is said to have it

        The bronze has the form of the statue, and
        the body has the disease.

In another way HOLD:

[3]     what encompasses a thing is said to hold what it encompasses

        For what is in what encompasses is said to be held by it.
        The vessel holds liquids, and the city human beings, and the ship sailors.
        And it is this way that whole has its parts.

[4]     what hinders a thing from moving or acting in accord with its own impulse
        is said to hold it

        The pillars hold the weight lying on them, and,
        as the poets say "Atlas holds up the heavens" on the supposition
        that it would fall to the earth otherwise.

Also, things are said to be IN something in ways that are similar to and
correspond to those in which they are said to have them.
