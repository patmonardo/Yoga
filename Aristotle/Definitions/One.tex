
Things are said to be ONE either coincidentally or intrinsically:

[1]     coincidentally

        Things coincide either in one substance or
        a part of each with one and the same thing.

        Coriscus, the musical, and musical Coriscus are one,
        and the musical, the just, and musical and just Coriscus are one.
        For all these are said to be one coincidentally
        because they coincide in one substance, the musical and Coriscus,
        Similarly, the musical Coriscus is in a way one with Coriscus,
        because one of the parts in the account coincides with the other,
        for example the musical with Coriscus. And musical Coriscus is
        in a way one with just Coriscus, because a part of each coincides
        with one and the same thing.

        The case is similar if the coincident is said of the genus or
        of the names of something universal. For example, a human and
        a musical human are one and the same. For it is either because
        the musical coincides with the human, who is one substance,
        or because each of the two coincide with some particular thing,
        such as Coriscus. Except they do not both belong to him in the same way,
        but human presumably does so as genus and included in the substance,
        whereas musical does so as a state or attribute of the substance.

[2]     intrinsically

[2a]    in continuity

        When its movement is intrinsically one and cannot be otherwise.
        A movement is one when it is indivisible and indivisible with respect to time.
        Things are intrinsically continuous when they are one but not by contact.

[2b-i]  in form

        When the underlying subject is undifferentiated in form.
        It is undifferentiated when its form is perceptually indivisible.
        The relevant underlying subject is either the first or the last
        relative to the end. For wine is said to be one and water is said to be one
        insofar as they are indivisible in form. Juices and oil are said to be one
        because the ultimate underlying subject of all of them is one.

[2b-ii] in genus
            
        Whose genus is one although distinguished by opposite differentia.
        And all these are said to be one when because the genus of their
        underlying subject is one. For example, dog, human, horse are one something,
        because all are animals and indeed in cases quite similar to when the 
        matter is the same.

[2c]    in account

        When the account that states the essence is indivisible from the account
        that makes the thing clear, for intrinsically every account is divisible.
        What has increased in size and is diminishing in size is one
        because its account is one.
        When the understanding that understands the essence of certain things is 
        indivisible, and cannot separate them in time or place or account,
        these things are most of all one and of those things those that are
        substances are most of all one.
        For those things that do not admit of division are universally said to be one,
        insofar as something is human and it does not admit of division, are universally
        said to be one.

[3] to be one

        to be a sort of starting-point of number.


Also, it is evident that things will be said to be MANY in ways opposite to those
in which they are said to be one.