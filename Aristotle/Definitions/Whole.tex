
Something is said to be a WHOLE if:

[1]     none of the parts of which it is said to be by nature a whole
        is absent from it

[2]     what encompasses the things it encompasses in such a way 
        that they are one, and this in two ways:

[2a]    as each being one

        For what is universal, or in general what is taken as such 
        when we say "as a whole" is universal as encompassing
        many things by being predicated of each of them and by all of them,
        each one of them being one thing, as human, horse, god are,
        because each are living things.
        
[2b]    as together composing one

        What is continuous and limited is a whole when it is one thing
        composed of many things, especially if they present in it potentially,
        but, failing this, if they are present in it actually.
        Of these themselves, those which are by nature such a sort are
        wholes to a higher degree than those that are so by craft,
        as we also said in the case of what is one,
        wholeness being in fact a sort of oneness.

[3]     as quantities

        The position of the parts makes no difference are said to be alls,
        those to which it does are said to be wholes. Those that admit of both
        are said to be both wholes and alls. These are things whose nature remains
        the same when the parts change while their shape does not remain the same.
        Such as wax or cloak, for they are said to be both wholes and alls,
        since they have both characteristics.


Things to which ALL is applied to one thing are sad to be EVERY,
"every" being applied to them as divided up -- "all this number" but
"every one of these units".
