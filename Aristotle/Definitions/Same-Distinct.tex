
Things are said to be the SAME coincidentally or intrinsically:

[1]     some things are said to be the SAME coincidentally
[1a]    when they coincide with the same thing
[1b]    when one coincides with the other

        That is why these are not all said universally,
        since it is not true to say that every man is the musical.
        For universals belong intrinsically whereas coincidents do not belong intrinsically.
        But in the case of particulars they are said of them unconditionally,
        since Socrates and musical Socrates seem to be the same.

[2]     some things are said to be the SAME intrinsically
[2a]    when things are said to be one
[2b]    when matter is one either in form or in number
[2c]    when form is one

        It is evident that sameness is a sort of oneness of the being,
        either of more than one thing or as one thing when it is treated as
        more than one thing.

[3]     some things are said to be DISTINCT
[3a]    when either their forms or their matter or
        the account of their substance are more then one
[3b]    generally in ways opposite opposite to the SAME

[4]     some things are said to be DIFFERENT
[4a]    when they are distinct, though they are the same something
        only not in number but either in form or in kind or by analogy
[4b]    when they are distinct in kind
[4c]    when they are contraries
[4d]    when they have distinctness in their substance

[5]     some things are said to be SIMILAR
[5a]    whose attributes are the same in every respect
[5b]    whose attributes are more the same than distinct
[5c]    whose quality is one
[5d]    whose attributes are shared with something else
        a majority of those contraries with respect to which
        alteration is possible or with more control

[6]     things are said to be DISSIMILAR
[6a]    in generally in ways opposite to the SIMILAR
