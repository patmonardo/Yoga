
Something is said to be an ELEMENT if:

[1]     the primary component from which a thing is composed
        and indivisible in kind into other kinds

        The elements of a voiced sound are the things from which
        the voiced sound is composed and the ultimate things
        into which it is divided, while they are no longer divided
        into voiced sounds distinct in form, rather if they are divided,
        their parts are of the same form. A part of water is water but
        a part of a syllable is not a syllable. Similarly, elements of bodies
        mean the ultimate things that bodies are divided into, while
        they are not divided into other things differing in form.

[2]     primary demonstrations that are components of other demonstrations
        are said to be elements of demonstration
        
        Primary deductions consisting of three terms proceeding through one middle.

[3]     what is small, simple, indivisible

        By metaphorical transference, people call an element when,
        being one and small, it is useful for many things. That is why
        most universal things are elements and that the one and the point
        are thought to be starting points. Since the genera are 
        universal and indivisible (for there is no account of them),
        some say the genera are elements, and more so than the differentia,
        because the genus is more universal. For where the differentia belongs,
        the genus follows along with it, but where the genus belongs,
        the differentia does not always do so.

The primary component of each thing is an element of it.
