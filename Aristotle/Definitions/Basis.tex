
Something is said to be that IN VIRTUE OF WHICH:

[1]     the form and the substance of each thing.

        that in virtue of which man is good is good itself.

[2]     the primary thing in which something naturally comes to be.

        color in a surface.

The primary way, then, that something is said to be that in virtue of which
is by being the form, and in a secondary way by being the matter of each thing
and the first underlying subject of each.
In general, that in virtue of which belongs to things in the same
number of ways as the cause.

[3]     as cause: in virtue of what has he come as

        for the sake of what has he come.

[4]     as cause: in virtue of what has he deduced wrongly as

        what is the cause of his deducing wrongly.

[5]     with reference to position.

        at which he stands based or along which he walks,
        since all these signify position or place.

So INTRINSICALLY must also be said of things in many ways:

[1]     the essence of each things is what it is intrinsically

        Callias is intrinsically Callias and the essence of Callias.

[2]     anything that is present in the what-it-is

        Callias is intrinsically an animal since animal is present in the account.

[3]     the primary recipient

        A surface is intrinsically white, and a human is intrinsically alive,
        since the soul, which is part of a human, is the primary recipient of life.

[4]     what has no other cause

        For a human has many causes, the animal, the two-footed,
        nonetheless a human is intrinsically a human.

[5]     attributes of a subject that belong to it alone
        and insofar as they belong to it merely because of itself,
        considered as what is separate by itself
