Something is said to LACK something:

[1]     if it does not have one of the attributes that it naturally has,
        even if this thing itself does not naturally have it

        A plant is said to lack eyes.

[2]     if what it or its genus naturally has,
        it itself does not have

        A human who is blind and a mole lack sight in different ways,
        the one in contrast to its genus, the other intrinsically.

[3]     if it does not have what is naturally for it to have
        when it is natural for it to have it

        for blindness is a sort of lack, but a being is not
        blind at every age but only when it does not have sight
        when it is natural for it to have it. Similarly,
        if it does not have sight in that in which,
        and in virtue of which, and in relation to which,
        and in the way in which, is natural.

[4]     a thing forcefully taken away is said to be lacked

Also, things are said to be lacks in as many ways as there are
ways as saying things with negative affix. 
Further something is said to lack something if it has little of it.
For example, pitless fruit. This is a case of having it minimally.
Further if it does not have it easily or well.
Further if it does not have it at all.
That is why not everybody is good or bad, or just or unjust,
but there is also an intermediate state.
