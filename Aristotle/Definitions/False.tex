
What is said to be FALSE:

[1]     what is false as a thing, and this in two ways:

[1a]    because it is not combined or cannot be combined

        The diagonal's being said to be commensurable with the side or
        of you being seated, since of these the former is always false and
        the latter sometimes false.

[1b]    though the things in question are, their nature is
        either to appear not to be such as they are or
        to be things that are not
                
        Things are said to be false in these ways then --
        either because they themselves are not or
        because the appearance they gave rise to are
        of something that is not.

[2]     what is false as an account is of things that are not

[2a]    every account is false of something other than what it is true of

        The account of a circle is false of a triangle.

[2b]    every account is in a way one and in another many

        Each thing has one account, its essence, and many, 
        since both itself and it with an attribute are in a way the same.
        For example, Socrates and musical Socrates.

[2c]    a false account is unconditionally not the account of anything

        That is why Antithenes was naive in thinking that nothing can be put
        fairly into words except by the account that properly belonged to it, one to one.
        From which it follows that there is no such thing as contradicting a falsehood
        and almost no such thing as speaking a falsehood either. 
        But it is possible to put a given thing into words
        not only by means of the account of itself
        but also by means of the account of something else.
        This may be done altogether falsely indeed but also in a way truly.

[3]     what is false as a false human being

        A false human being is one who readily and by deliberate choice
        gives false account, not because of something else but because of itself,
        and who produces such accounts in other people,
        just as we say things are false when they produce a false appearance in us.
        That is why the argument given in the Hippias, that the
        same human being is both true and false is misleading.
        For it assumes that someone is false if he is capable of falsity,
        that is, the one who knows and is wise, and, furthermore,
        that someone who voluntarily does base actions is better.
        This false assumption is due to the induction, since someone
        who limps voluntarily is better than one who does so involuntarily --
        if by limping is meant pretending to limp, since if someone were
        really lame voluntarily, he would presumably be worse, as in the
        case of moral character.
