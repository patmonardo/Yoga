ANALYTICA PRIORA
BOOK II
CHAPTER 1
More than one conclusion can sometimes be drawn from the same
premisses
52b38. We have now discussed (I) the number of the figures,
the nature and variety of the premisses, and the conditions of
inference, (2) the points to be looked to in destructive and con-
structive proof, and how to investigate the problem in each kind
of inquiry, (3) how to get the proper starting-points.
53a3. Universal syllogisms and particular affirmative syllogisms
yield more than one conclusion, since the main conclusion is
convertible; particular negative syllogisms prove only the main
conclusion, since this is not convertible.
IS. The facts about (I) universal syllogisms may be also stated
in this way: in the first figure the major term must be true of
everything that falls under the middle or the minor term.
25. In the second figure, what follows from the syllogism (in
Cesare) is only that the major term is untrue of everything that
falls under the minor; it is also untrue of everything that falls
under the middle term, but this is not established by the syllogism.
34. (2) In particular syllogisms in the first figure the major is
not necessarily true of everything that falls under the minor. It
is necessarily true of everything that falls under the middle term,
but this is not established by the syllogism.
40. So too in the other figures. The major term is not neces-
sarily true of everything that falls under the minor; it is true of
everything that falls under the middle term, but this is not estab-
lished by the syllogism, just as it was not in the case of universal
syllogisms.
52b38-c). 'Ev 1ToaoL~ ... au>'>'oYLallo~, cf. 1. 4-26.
40-53a2. ETL S' •.. IlE90Sov, cf. 1. 27-31.
53a2-3. ETL SE ••• a.pxa~, cf. 1. 32-46.
3- b 3. E1Tt:L S' ... TOlhwv. In this passage A. considers the prob-
lem, what conclusions, besides the primary conclusion, a syllo-
gism can be held to prove implicitly. He first (A) ("3-14) considers
conclusions that follow by conversion of the primary conclusion.
Such conclusions follow from A, E, or I conclusions, but not from
an 0 conclusion, since this alone is not convertible either simply
or per accidens. (B) He considers secondly (aI5-b3) conclusionsCOMMENTARY
derivable from the original syllogism, with regard to terms which
can be subsumed either under the middle or under the minor term
(the latter expressed by V1TO TO croIL1TlpaalLa, "q). A. considers first
(I) syllogisms in which the conclusion is universal, (a) in the first
figure. If we have the syllogism All C is A, All B is C, Therefore
all B is A, then if all D is B, it is implicitly proved that all D is
A ("21-2). And if all E is C, it follows that all E is A ("22-4).
Similar reasoning applies to an original syllogism of the form
No C is A, All B is C, Therefore no B is A ("24). (b) In the second
figure. If we have the syllogism No B is A, All C is A, Therefore
no C is B, then if all D is C, it is implicitly proved that no D is
B ("25-8). If all E is A, it follows that no E is B, but this does not
follow from the original syllogism. That syllogism proved that no
C (and therefore implicitly that no D) is B; but it assumed (that
no B is A, or in other words) that no A is B, and it is from this+
All E is A that it follows that no E is B ("29-34).
A. next considers (2) syllogisms in which the conclusion is
particular, and as before he takes first (a) syllogisms in the first
figure. While in "I9-24 B was the minor and C the middle term,
he here takes B as middle term and C as minor. Here a term sub-
sumable under C cannot be inferred to be A, or not to be A (un-
distributed middle). A term subsumable under B can be inferred
to be (or not to be) A, but not as a result of the original syllogism
(but as a result of the original major premiss All B is A (or No B
is A)+the new premiss All D is B) ("34-40).
The commentators make A.'s criticism in "29-34 turn on the
fact that the major premiss of Cesare (No B is A) needs to be
converted, in order to yield by the dictum de omni et nullo the
conclusion that no Eis B. But this consideration does not apply
to the syllogisms dealt with in "34-40. Take a syllogism in Darii-
All B is A, Some C is B, Therefore some C is A. Then if all D is
B, it follows from the original major premiss+All D is B,
without any conversion, that all D is A. And there was no
explicit reference in the case of Cesare ("29-34) to the necessity of
conversion. I conclude that A.'s point was not that, but that the
conclusion No E is B followed not from the original syllogism, but
from its major premiss.
Finally (b), A. says ("4o- b 3) that in the case of syllogisms with
particular conclusions in the second or third figure, subsumption
of a new term under the minor term yields no conclusion (un-
distributed middle), but subsumption under the middle term
yields a conclusion-one, however, that does not follow from the
original syllogism (but from its major premiss), as in the case ofII. 1. 53"7-12
syllogisms with a universal conclusion, so that we should either
not reckon such secondary conclusions as following from the
universal syllogisms, or reckon them (loosely) as following from
the particular syllogisms as well (WaT' ~ ouS' £K£I: £aTa, ~ Ka, br,
'ToVrWJI). I take the point of these last words to be that A. has now
realized that he was speaking loosely in treating (in "zl-4) the
conclusions reached by subsumption under the middle term of a
syllogism in Barbara or Celarent as secondary conclusions from
that syllogism; they, like other conclusions by subsumption under
the middle term, are conclusions not from the original syllogism,
but from its major premiss, i.e. by parity of reasoning.
A. omits to point out that from Camestres, Baroco, Disamis,
and Bocardo, by subsumption of a new term under the middle
term, no conclusion relating the new term to the major term can
be drawn.
,. TJ SE aTEpT)TLKT), i.e. the particular negative.
8-c). TO SE aUll'll'Epaalla ••• EaTLv. This should not be tacked
on to the previous sentence. It is a general statement designed to
support the thesis that certain combinations of premisses estab-
lish more than one conclusion ("4--6), viz. the statement that a
single conclusion is the statement of one predicate about one
subject, so that e.g., the conclusion Some A is B, reached by
conversion from the original conclusion All B is A or Some B is A,
is different from the original conclusion (&1O-1Z).
9-12. c,ae' ot IlEv o.).).OL aU).).0YLaIlOL • • • EIl'll'poa8£v. In
pointing out that the conclusion of a syllogism in Barbara,
Celarent, or Darii may be converted, A. is in fact recognizing the
validity of syllogisms in Bramantip, Camenes, and Dimaris. But
he never treats these as independent moods of syllogism; they are
for him just syllogisms followed by conversion of the conclusion.
(In pointing out that conclusions in A, E, or I are convertible,
he does not limit his statement to conclusions in the first figure;
he is in fact recognizing that the conclusions of Cesare, Camestres,
Darapti, Disamis, and Datisi may be converted. But here con-
version gives no new result. Take for instance Cesare-No P is
M, All S is M, Therefore no S is P. The conclusion No P is S can
be got, without conversion, by altering the order of the premisses
and getting a syllogism in Camestres.)
In 29"19-z9 A. pointed out that if we have the premisses (a) No
C is B, All B is A, or (b) No C is B, Some B is A, we can, by
converting the premisses, get No B is C, Some A is B, Therefore
some A is not C. I.e., he recognizes the validity of Fesapo and
Fresison.COMMENTARY
Thus A. recognizes the validity of all the moods of the fourth
figure, but treats them as an appendix to his account of the first
figure.
CHAPTER 2
True conclusions from false premisses, in the first figure
53 b 4' The premisses may be both true, both false, or one true
and one false. True premisses cannot give a false conclusion;
false premisses may give a true conclusion, but only of the fact,
not of the reason.
11. True premisses cannot give a false conclusion. For if B is
necessarily the case if A is, then if B is not the case A is not. If,
then, A is true, B must be true, or else A would be both true and
false.
16. If we represent the datum by the single symbol A, it must
not be thought that anything follows from a single fact; there
must be three terms, and two stretches or premisses. A stands for
two premisses taken together.
(A) Both premisses universal
26.
We may get a true conclusion (a) when both premisses are
false, (b) when the minor is wholly false, (c) when either is partly
false.
Combinations of fact
30. (a) No B is A.
No C is B.
All C is A.
35.
All B is A.
No C is B.
No C is A.
54"1. Some B is not A.
Some C is not B.
All Cis A.
Some B is A.
Some C is not B.
No C is A.
:.
:.
:.
:.
Inference
All B is A.
All C is B.
All C is A.
No B is A.
All Cis B.
No C is A.
All B is A.
All C is B.
All C is A.
No B is A.
All C is B.
No C is A.
Wholly false.
11
True.
Wholly false.
11
True.
Partly false.
11
True.
Partly false.
11
True.
2. A wholly false major and a true minor will not give a true
concl usion :
6.
No
is
All B is A. Wholly false.} Im-
B A.} Impossible.
11.
All C is B.
AIlC is A.
All
is
All C is B.
No C is A.
B A.} Impossible.
All C is B. True.
:. All C is A . "
.bl
e.
POSSI
No B .is A. Wholly false.} Im-
.bl
All C IS B. True
:. No C is A.
11
POSSI e.18. (cl (a) A partly false major and a true minor can give a
true conclusion:
Combinations of fact
Some B is A.
All C is B.
All C is A.
23.
Some B is A.
All C is B.
No C is A.
Inference
All B is A.
All C is B.
:. All C is A.
No B is A.
All C is B.
:. No C is A.
Partly false.
True.
" false.
Partly
True.
28. (b) A true major and a wholly false minor can give a true
conclusion.
35·
All B is A.
No C is B.
All C is A.
No B is A.
No C is B.
No C is A.
All B is A.
All C is B.
:. All C is A.
No B is A.
All C is B.
:. No C is A.
True.
Wholly false.
True.
True.
Wholly false.
True.
b2. (cl (ft) A true major and a partly false minor can give
a true conclusion.
9·
All Bis A.
Some C is B.
All C is A.
No B is A.
Some C is B.
No C is A.
All B is A.
All C is B.
:. All C is A.
No B is A.
All C is B.
:. No C is A.
True.
Partly false.
True.
True.
Partly false.
True.
(B) One premiss particular
17. (al A wholly false major and a true minor. (b) a partly false
major and a true minor. (cl a true major and a false minor. (d)
two false premisses. can give a true conclusion:
No B is A.
Some C is B.
Some C is A.
All B is A.
27·
Some C is B.
Some C is not A.
35. (b) Some B is A.
Some C is B.
Some C is A.
55" 2 . Some B is A.
Some C is B.
Some C is not A.
4· (c) All B is A.
No C is B.
Some C is A.
21. (a)
:.
:.
:.
:.
:.
All B is A. Wholly false.
Some C is B. True.
Some C is A.
" false.
No B is A. Wholly
Some C is B. True.
Some C is not A. True.
All B is A. Partly false.
Some C is B. True.
Some C is A.
" false.
No B is A. Partly
Some C is B. True.
Some C is not A. True.
All B is A. True.
Some C is B. False.
Some C is A. True.COMMENTARY
43 0
Combinations of fact
10.
No B is A.
NoC is B.
Some C is not A.
19. (4) Some B is A.
NoC is B.
Some C is A.
Some B is A.
26.
No G is B.
Some C is not A.
No B is A.
28.
NoC is B.
Some C is A.
All B is A.
36.
No C is B.
Some C is not A.
:.
:.
:.
:.
:.
Inference
No B is A. True.
Some C is B. False.
Some C is not A. True.
All B is A. Partly false.
Some C is B. False.
Some C is A. True.
No B is A. Partly false.
Some C is B. False.
Some C is not A. True.
All B is A. Wholly false.
Some C is B. False.
Some C is A. True.
No B is A. Wholly false.
Some C is B. False.
Some C is not A. True.
53blo. SL' ~v S' a.LTLa.V ••• AEX9";O'ETa.L, i.e. in S7"4o-b17.
23-4' TO o~v A ... O'UAAT)ct»9ElO'a.L, i.e. the A mentioned in b I2_
14 (the whole datum from which inference proceeds), not the A
mentioned in
b 2I - 2
(the major term).
~TUXEV. (moTlpa~ (for (moTlpa) is
rather an extraordinary example of attraction, but has parallels
in A., e.g. An. Post. 79 b41, SO"14, 81"9.
28-30' Ec1V1TEP OAT)V ••• 01TOTEPa.O'OUV. All B is A is 'wholly
false' when no B is A, and No B is A 'wholly false' when all B is
A (S4"4--6). All B is A and No B is A are 'partly false' when
some B is A and some is not. Cf. S6 1 S- b 3 n.
54"7-15. liv Sf] ... r. The phrases ~ TO AB, ~ TO Br in "S, 12
are abbreviations of '" 7Tp6Taa,~ £cp' V K€i:Ta, TO AB (TO BT). Similar
instances are to be found in An. Post. 94"31, Phys. 21S bS, 9, etc.
8-9. Ka.l1Ta.vTL ••• A, 'i.e. that all B is A'.
11-14. OI'OLIIlS S' ••• ~O'Ta.L. A. begins the sentence meaning to
say 'similarly if A belongs to all B, etc., the conclusion cannot be
true' (cf. "9), but by inadvertence says 'the conclusion will be
false', which makes the ouS' in all incorrect; but the anacoluthon
is a very natural one.
13. Ka.LI'T)S€Vl;e TO B, TO A, 'i.e. that no B is A'.
31-2. otov 00'a. ••• a.AAT)Aa., e.g. when B and C are species of
A, neither included in the other.
38. otov TOlS E~ a.AAou Y€vous ••• Y€vos, 'e.g. when A is a
genus, and B and C are species of a different genus'.
b5-6. otov TO YEVOS ..• SLa.ct»Op~, 'e.g. when A is a genus, B a
species within it, and C a differentia of it' (confined to the genus
but not to the species).
27. Ta.UTT)S S' OUX 01TOTEpa.S43 I
II-I2. OlOY TO yevos ..• ~ha.cpopq., 'e.g. when A is a genus, B
a species of a different genus, and C a differentia of the second
genus' (confined to that genus but not to the species).
558:13-14. otOY TO yevos .•• ErliEcn, 'e.g. when A is a genus, B
a species of another genus, and C an accident of the various species
of A' (not confined to A, and never predicable of B).
IS. AEUK~ SE nyL In order to correspond with aI2 Tep SJ r nv,
p.~ lmapXE'V and with 817 T6 A TW' Tep r otlX i)1Tap~E', this should
read AEVKep SE T'V' ov, and this should perhaps be read; but it has
no MS. support, and in S6aI4, an exactly similar passage, AWKep SJ
"'V, ov has very little. It is probable that A. wrote AEVKep St nvi,
since he usually understands a proposition of the form Some S is
P as meaning Some S is P and some is not.
CHAPTER 3
True conclusions from false premisses, in the second figure
SSb3 • False premisses can yield true conclusions: (a) when both
are wholly false, (b) when both are partly false, (c) when one is
true and one false.
10.
(A) Both premisses universal
Combination of facts
(a) No B is A.
All C is A.
No C is B.
14.
All B is A.
NoCis A.
No C is B.
16.(c) All B is A.
All C is A.
No C is B.
All B is A.
All C is A.
No C is B.
Some B is A.
23·
All C is A.
No C is B.
All B is A.
3 0 •
Some C is A.
No C is B.
Some B is A.
3 1 •
No C is A.
No C is B.
Inference
All B is A.
No C is A.
:. No C is B.
No B is A.
All C is A.
:. No C is B.
All B is A.
No C is A.
:. No C is B.
No B is A.
All C is A.
:. No C is B.
No B is A.
All C is A.
:. No C is B.
All B is A.
No C is A.
:. No C is B.
All B is A.
No C is A.
:. No C is B.
Wholly false.
"
True.
Wholly false.
True. "
True.
Wholly false.
True.
Wholly false.
True.
" false.
Partly
True.
"
True.
Partly false.
True.
Partly false.
True.COMMENTARY
43 2
Combinations of facts
38. (b) Some B is A.
Some C is A.
NoC is B.
56&3. Some B is A.
Some C is A.
NoC is B.
5·
Inference
All B is A.
NoC is A.
:. No C is B.
No B is A.
All C is A.
:. No C is B.
Partly false.
True.
Partly false.
True.
(B) One premiss particular
(c) All B is A.
Some C is A.
Some C is not
n.
No B is A.
Some C is not
Some C is not
18.
No B is A.
No C is A.
Some C is not
All B is A.
All C is A.
Some C is not
32. (a) All B is A.
All C is A.
Some C is not
37.
No B is A.
All C is A.
Some C is not
B.
A.
B.
B.
B.
B.
B.
No B is A. Wholly false.
Some C is A. True.
:. Some C is not B. True.
All B is A. Wholly false.
Some C is not A. True.
:. Some C is not B.
No B is A. True.
Some C is A. False.
:. Some C is not B. True.
All B is A. True.
Some C is not A. False.
:. Some C is not B. True.
No B is A. Wholly false.
Some C is A (se. and some not.) False.
:. Some C is not B. True.
All B is A. Wholly false.
Some C is not A. False.
:. Some C is not B. True.
55b3-IO. 'Ev SE T~ ~icr~ crx,,~aTL ••• cruAAoYLcrj.!.WV. The vul-
gate text of this sentence purports to name six possibilities. But
of these the sixth (£; ~ fLEV oA'Y} .p£v8~, ~ 8' £7T[ Tt dA'Y}e~,) is not
mentioned in the detailed treatment which follows. nor anywhere
in chs. 2-4 except in 2. 55"r9-28. It is to be noted too that the
phrase £7T{ Tt dA'Y}e~, does not occur anywhere else in these chapters.
and that the distinction between a premiss which is £7T{ Tt .p£v8~,
and one which is £7T{ Tt dA'Y}e~, is a distinction without a difference.
since each must mean an A or E proposition asserted when the
corresponding I or 0 proposition would be true. Waitz is justified.
therefore. in excising the two clauses he excises. But the whole
structure of the latter part of the sentence. I(at £; dfL<PoTEpa, ...
dA'Y}e~, b 7 -9 is open to suspicion. In all the corresponding sentences
in chs. 2-4 (53b26-30. 54 bq-2I. S6b4-9) all the alternatives are
expressed by participial clauses. Further, the phrase a7TAw,
d>..'Y}e~, b 7 does not occur elsewhere in chs. 2-4. Thus the words433
from KaL El al-'<poTEpal to E7TL 'TL a.\1}O~, betray themselves as a gloss,
meant to fill supposed gaps in the enumeration in b 4- 7.
If we retain o.\1}' in b6, the words al-'<poT(pCJJv ... .\al-'/3avol-'(vCJJv
cover the cases mentioned in b lO- I 6, the words (7T/. 'TL iKaTEpa,
('each partly false') those in b 3 8-S6"4, and the words Tij, I-'Ev
a.\1}OoiJ, ... 'TLO€I-'EV7], those in bI6-23 and in 56"5-18, but those in
55b23-38 and in 56"18-32 are not covercd. By excising o.\1}' we
get an enumeration which covers all the cases mentioncd down
to 56"32. o.\1}' must be a gloss, probably traceablc to the same
scribe who had inserted it in 54b20.
The enumeration stillleavcs out (as do the €l clauses) the cases
mentioncd in 56"32-b3, in which the minor premiss, being parti-
cular, is simply 'false' and cscapes the disjunction 'wholly or
partly falsc', which is applicablc only to univcrsal propositions.
The chapter is made easier to follow if we remember that in this
figure A always stands for the middle, B for the major, r for the
minor term.
18-19. oloy TO Y£YOC; ••• E'{SEaLY, cf. 54"61-2 n.
20. EBY O~y ~T)~9ii, sc. TO ,.pov.
56"14. ~EUKit» Si TLYl. Strict logic would require .\WK.p 8E 'TLVL OV,
to correspond to T.p 8E r 'TLVL I-'~ tmapXELv, "13. But A. often uses
Some S is P as standing for Some S is P and some is not. Ct.
55"15 n.
27-8. otOY TO Y£voC; ••• ~iL(l+op~, i.e. when B is a species of A,
and C a differentia of A (confined to A but not to B).
35. Tit» SE r nyl U1rc1PXELY. Here, as in "IS, 'TLVL Imapxnv stands
for TtVL I-'EV tmapXELv TtVL 8' av, which is untrue because it contra-
dicts TO A ... T.p r o.\<p tmapXEtV, "33-4.
CHAPTER 4
True conclusions from false premisses, in the third figure
S6b4' False premisses can give a true conclusion: (a) when both
premisses are wholly false, (b) when both are partly false, (c) when
one is true and one wholly false, (d) when one is partly false and
one true.
(A) Both premisses universal
9·
Combination of facts
(a) No C is A.
No C is B.
Some B is A.
4985
Infl!1'ence
All C is A. Wholly false.
All C is B.
"
:. Some B is A. True.
Ff434
Combination of facts
All C is A.
NoC is B.
Some B is not A.
:zoo (b) Some C is A.
Some C is B.
Some B is A.
Some C is not A.
26.
Some C is B.
Some B is not A.
33. (c) All C is A.
All C is B.
Some B is not A.
No C is A.
40·
No C is B.
Some B is not A.
57- I • NoC is A.
All C is B.
Some B is A.
All C is A.
8.
No C is B.
Some B is A.
9. (d) Some C is A.
All C is B.
Some B is A.
All C is A.
IS·
Some C is B.
Some B is A.
Some C is A.
18.
All C is B.
Some B is not A.
No C is A.
23·
Some C is B.
Some B is not A.
COMMENTARY
14·
:.
:.
:.
:.
:.
:.
:.
:.
:.
:.
:.
Inference
No C is A. Wholly false.
All C is B.
Some B is not A. " True.
All C is A. Partly false.
All C is B.
"
Some B is A. True.
No C is A. Partly false.
All C is B.
Some B is not A. " True.
NoC is A. Wholly false.
All C is B. True.
Some B is not A. True.
No C is A. True.
All C is B. Wholly false.
Some B is not A. True.
All C is A. Wholly false.
AllC is B. True.
Some B is A. True.
All C is A. True.
All C is B. Wholly false.
Some B is A. True.
AllC is A. Partly false.
All C is B. True.
Some B is A. True.
All C is A. True.
AllCis B. Partly false.
Some B is A. True.
NoC is A. Partly false.
All C is B. True.
Some B is not A. True.
No C is A. True.
All C is B. Partly false.
Some B is not A. True.
(B) Both premisses particular
29. Here too the same combinations of two false premisses, or
of a true and a false premiss, can yield a true conclusion.
36. Thus if the conclusion is false, one or both premisses must
be false; but if the conclusion is true, neither both premisses nor
even one need be true. Even if neither is true the conclusion may
be true, but its truth is not necessitated by the premisses.
40. The reason is that when two things are so related that if
one exists the other must, if the second does not exist neither will
the first, but if the second exists the first need not; while on the
other hand the existence of one thing caIUlot be necessitated both435
by the existence and by the non-existence of another, e.g. B's
being large both by A's being and by its not being white.
b6. For when if A is white B must be large, and if B is large C
cannot be white, then if A is white C cannot be white. Now when
one thing entails another, the non-existence of the second entails
the non-existence of the first, so that B's not being large would
necessitate A's not being white; and if A's not being white
necessitated B's being large, B's not being large would necessitate
B's being large; which is impossible.
The reasoning in this chapter will be more easily followed if we
remember that in this figure A stands for the major, B for the
minor, for the middle term.
56b7-8. Kat o.vclwa}..w ••• wpOTclaE~~. ava1Ta'\Lv is meant to dis-
tinguish the case in which the major premiss is wholly false and
the minor true from that in which the major is true and the minor
wholly false (h6), and that in which the major is true and the
mioor partly false from that in which the major is partly false and
the minor true (b 7). KQ~ oaQXwS" a.:\'\wS" EYXWPEL J.LETa'\afkiv 'Td!>
1TpO'TauHS" is probably meant to cover the distinction between the
case in which both premisses are affirmative and that in which one
is negative, and between that in which both are universal and that
in which one is particular.
40-57"1. ofLo[w~ SE .•. ch/tuxov. The argument in b 33 - 40 was:
• "All C is A, All C is B, Some B is not A" may all be true; for in
fact all swans are animals, all swans are white, and some white
things are not animals. But if we assume falsely that no C is A and
truly that all C is B, we get the true conclusion Some B is not A:
A. now says that the same terms will suffice to show that a true
conclusion can be got from a true major and a false minor pre-
miss. Waitz is no doubt right in bracketing as corrupt J.LE,\av-
KVKvo,-at/Jvxov, which are not in fact ot av'To~ OPOL as those used just
before (or anywhere else in the chapter). But even without these
words all is not well; for if we take a true major and a false minor,
and say All swans are animals, No swans are white, we can prove
nothing, since the minor premiss in the third figure must be
affirmative. A. probably had in mind the argument No swans are
lifeless, All swans are black, Therefore some black things are not
lifeless-where, if not the terms, at least the order of ideas is much
the same as in b 33- 40 . But this does not justify the words J.LE,\av-
KUKvo,-at/Jvxov; for A. would have said at/JvxoV-J.LE'\av-KVKvo!>.
57"Z3-5. Wcl}..LV EwEt ••• UWclpXnv. A. evidently supposes him-
self to have proved by an example that No C is A. Some C is B,
rCOMMENTARY
Some B is not A are compatible, but has not in fact done so. He
may be thinking of the proof, by an example, that Some C is not
A, Some C is B, Some B is not A are compatible (S6b27-30)' The
reference cannot be, as Waitz supposes, to S481-2.
33-5. OUOEV yap .•. iK8EOW, i.e. whether in fact no 5 is P or
only some 5 is P, in either case the same proposition All 5 is P
will serve as an instance of a false premiss, which yet with another
premiss may yield a true conclusion. The point is sound, but is
irrelevant to what A. has just been saying in 829-33. He has been
pointing out that the same instance will serve to show the possi-
bility of true inference from false premisses when the premisses
differ in quantity as when both are universal. What he should be
pointing out now, therefore, is not that the difference in the state
of the facts between I-'7]O£Vt InrctpxoVTO, and TLVt inrctpxovTO, does not
affect the validity of the example, but that the difference between
the false assumption 1TaVTt inrctpX£Lv and the false assumption 'TtVt
inrctpX£Lv does not affect the validity of the example. If the fact is
that Ol~O£Vt inrctpXH, both the assumption 1TavTt InrctPX£LV and the
assumption nvt InrctPX£LV may serve to illustrate the possibility of
reaching a true conclusion from false premisses.
36-bI7. <l>avEpov o~v ••• TpLWV. This section does not refer,
like the rest of the chapter, specially to the third figure. It dis-
cusses the general question of the possibility of reaching true
conclusions from false premisses. The main thesis is that in such
a case the conclusion does not follow of necessity (140). This is of
course an ambiguous statement. It might mean that the truth of
the conclusion does not follow by syllogistic necessity; but if A.
meant this he would be completely contradicting himself. What
he means is that in such a case the premisses cannot state the
ground on which the fact stated in the conclusion really rests,
since the same fact cannot be a necessary consequence both of
another fact and of the opposite of that other (b 3- 4).
4o-bI7. aLTLOV 0' . . . TpLWV. A. has said in "36-40 (1) that
false premisses can logically entail a true conclusion, and (2) that
the state of affairs asserted in such premisses cannot in fact
necessitate the state of affairs asserted in the conclusion. He first
(140_b 3) justifies the first point, and then justifies the second, in
the following way. An identical fact cannot be necessitated both
by a certain other fact and by the opposite of it (b 3--<i). For if
A's being white necessitates B's being large, and B's being large
necessitates Cs not being white, A's being white necessitates Cs
not being white (b~). Now if one fact necessitates another, the
opposite of the latter necessitates the opposite of the former437
(b 9- I I ). Let it be the case that A's being white necessitates B's
being large. Then B's not being large will necessitate A's not being
white. Now if we suppose that A's not being white (as well as A's
being white) necessitates B's being large, we shall have a situation
like that described in b6---9. H's not being large will necessitate A's
not being white; A's not being white will necessitate B's being
large; therefore B's not being large will necessitate B's being
large. But this is absurd: therefore we must have been wrong in
supposing that A's not being white, as well as its being white,
necessitates B's being large. The point is the same as W<lS made
briefly in 53 bj-ro, that while false premisses may necessitate a
true conclusion, they cannot state the reason for it, i.e. the facts
on which its truth rests.
b lo• TO 1rpWTOV. The subject of J.L~ €lvat must be the state of
affairs asserted in the first proposition (8aT€pOV of b9) ; but through-
out b6-r7 A, B, r stand not for propositions but for subject-
terms. I have therefore read TO TTpW-rOV. For the substitution of a
for TTPWTOV in MSS. cl. M et. I047bzz, and many instances in the MSS.
17. WS 5Lc1 TpLWV. In b6--<} A. pointed out that' A 's being white
necessitates B's being large' and 'B's being large necessitates Cs
not being white' give the conclusion' A 's being white necessitates
Cs not being white'. In b9- I7 he has used only two subject-terms,
A and B, not three, and has pointed out that similarly 'B's not
being large necessitates A's not oeing white' and' A's not being
white necessitates B's being large' yield the conclusion' B's not
being large necessitates B's being large'. Maier (za. 26r n.) thinks
that W!i SUI TptWV is spurious, because the word to be supplied,
according to A.'s terminology, must be opwv (cf. TTaaa aTT()Sngt!i
laTat Std. TptWV opwv (4rb36), Std. TptwV (65"19), TptWV OVTWV £KaaTov
CTVJ.LTT€paaJ.La y€yOV€ (SSa33)-the three terms of an ordinary syllo-
gism being in each case referred to), while in fact in b6--<} six terms
(he does not say what these are) are used. He considers that the
word to be supplied is probably inro8€a€wv, and that the phrase
was used by a Peripatetic or Stoic copyist familiar with the phrase
Std TptWV imo8f:TtKO!i uv,uoytaJ.L0!i (a syllogism with two hypothetical
premisses and a hypothetical conclusion)-perhaps the same
interpolator who has been at work in 45bI6-r7 and in S8b9. He
may be right, but I see no particular difficulty in the phrase W!i
Sld TptWV if we suppose A. to have only the subject-terms in
view, which are in fact the only terms to which he has assigned
letters. W'i Std TptWV will then mean 'we shall have a situation like
that described in b6--<}, but with the two terms A, B, instead of the
three terms A, B, C.COMMENTARY
CHAPTER 5
Reciprocal proof applied to first-figure syllogisms
S7bI8. Reciprocal proof consists in proving one premiss of our
original syllogism from the conclusion and the converse of the
other premiss.
21. If we have proved that C is A because B is A and C is B,
to prove reciprocally is to prove that B is A because C is A and B
is C, or that C is B because A is Band C is A. There is no other
method of reciprocal proof; for if we take a middle term distinct
from C and A there is no circle, and if we take as premisses both
the old premisses we shall simply get the same syllogism.
32. Where the original premisses are inconvertible one of our
new premisses will be unproved; for it cannot be proved from the
original premisses. But if all three terms are convertible, we can
prove all the propositions from one another. Suppose we have
shown (I) that All B is A and All C is B entail All C is A, (2) that
All C is A and All B is C entail All B is A, (3) that All A is Band
All C is A entail All C is B. Then we have still to prove that all
B is C and that all A is B, which are the only unproved premisses
we have used. We prove (4) that all A is B by assuming that all
C is B and all A is C, and (5) that all B is C by assuming that all
A is C and all B is A.
58-6. In both these syllogisms we have assumed one unproved
premiss, that all A is C. If we can prove this, we shall have proved
all six propositions from each other. Now if (6) we take the pre-
misses All B is C and All A is B, both the premisses have been
proved, and it follows that all A is C.
u. Thus it is only when the original premisses are convertible
thdt we can effect reciprocal proof; in other cases we simply assume
one of our new premisses without proof. And even when the
terms are convertible we use to prove a proposition what was
previously proved from the proposition. All B is C and All A is
B are proved from All A is C, and it is proved from them.
21.
26.
Syllogism
No B is A.
All C is B.
:. No C is A.
Reciprocal proo]
No C is A.
All B is C.
:. No B is A.
All of that, none of which is A, is B.
No C is A.
:. All C is B.n. s. S7hr8 -
Syllogism
36.
All B is A.
Some C is B.
:. Some C is A.
tlz.
6. No B is A.
S8a3 2
439
Reciprocal pToof
The universal premiss cannot be proved reci-
procally, nor can anything be proved from
the other two propositions, since these are
both particular.
All A is B.
Some C is A.
:. Some C is B.
The universal premiss cannot be proved.
Some C is B.
:. Some C is not A.
Some of that, some of which is not A, is B.
Some C is not A.
:. Some C is B.
57bI8-~O. To SE KUKA<tJ .•. AO~-m1v, The construction would
be easier if we had Aa{3£Lv in h 20 , or if the second TaU in hI9 were
omitted; but either emendation is open to the objection that it
involves A. in identifying TO o£lKvua8at (passive) with TO aup.7T£-
pavaa8at (middle). The traditional text is possible: 'Circular and
reciprocal proof means proof achieved by means of the original
conclusion and by converting one of the premisses simply and
inferring the other premiss:
~4' Ka.L TO A T~ B. The sense and the parallel passage h 25 - 7
show that these words should be omitted.
~5-6. i1 EL [on] ... ll1l'a.pxov. on must be rejected as ungram-
matical.
58al4-I5. EV SE TaLC; aAAo~c; ••• EL1I'O!,EV refers to 57h32-5,
where A. pointed out that when the terms are not simply con-
vertible, the circular proof can be effected only by assuming
something that is unprovable, viz. the converse of one of the
original premisses. He omits to point out that even when the
terms are coextensive, the converse of an A proposition cannot be
inferred from that proposition, though its truth may be known
independently.
22. €aTW TO I1EV B . . . u1I'apXELv, 'let it be the case that B
belongs to all C. Waitz is justified in reading lJ7TapXHV, with all
the best MSS. Cf. 830 (where it is read by all the MSS.) and L. and
S. S.v. £lp.l, A. VI. b.
~5. €aTW
must be read, not EaTaL.
EL S' ••• Tii' r U1I'a.PXELV. Of the valid moods of syllo-
gism, there are nine that have a negative premiss and a negative
conclusion, and in the case of these it is impossible to prove the
affirmative premiss in the way A. adopts in other cases, viz. from
26-3~.COMMENTARY
44 0
the conclusion of the original syllogism and the converse of the
other premiss; for an affirmative cannot be proved from two
negatives. Of these nine moods. in three-Baroco. Felapton. and
Bocardo-it is impossible by any means to effect reciprocal proof
of the affirmative premiss; for this is universal. while one or both
of the other propositions are particular. For four of the remaining
moods A. adopts a new method of proof-for Celarent (SS326-32).
Ferio (b 7- 12 ). Festino (b 33-S). Ferison (S9a24~). He says (SSb I3 -
IS) that Cesare and Camestres cannot be similarly treated. but in
fact they can. The affinity of the six proofs can be best seen if we
call the minor. middle. and major terms S. M. P in each case.
Celarent
No M is P.
A1l5isM.
:. No 5 is P.
All of that, none of which is P, is M.
No 5 is P.
:. All S is M.
Cesare
All of that. none of which is P, is M.
No5 is P.
No P is M.
All S is M.
:. No 5 is P. :.AIl5isM.
All PisM.
No5 is M.
:. No 5 is P. All of that, none of which is 5, is M.
(No 5 is P, :.) No P is 5.
:. All P is M.
No M is P.
Some 5 is M.
:. Some 5 is not P. Some of that, some of which is not P, is M.
Some S is not P.
:. Some 5 is M.
No P is M.
Some 5 is M.
:. Some 5 is not P. Some of that, some of which is not P, is M.
Some S is not P.
:. Some S is M.
No M is P.
Some M is 5.
:. Some S is not P. Some of that, some of which is not P, is M.
Some 5 is not P.
:. Some 5 is M.
:. Some 111 is 5.
Camestres
Ferio
Festino
Ferison
All the reciprocal proofs fall into one or other of two forms: If
no X is Y. all X is Z. No X is Y. Therefore all X is Z. or If some
X is not Y. some X is Z. Some X is not Y. Therefore some X is Z.
The 'conversion' of 'No M is ])' into 'All of that. none of whichII. S. S8 b 6-8
is P, is M' strikes one at first sight as a very odd kind of conver-
sion. But on a closer view we see that what A. is doing is to make
a further, arbitrary, assumption, viz. that M and P, besides being
mutually exclusive, are exhaustive alternatives; i.e. that they are
contradictories. And this is no more arbitrary than the assump-
tion A. makes in the other reciprocal proofs he offers in chs. 5-7,
viz. that All B is A can be converted into All A is B. Throughout
these chapters the proofs that are offered are not offered as proofs
that can be effected on the basis of the original data alone, but
simply as a mental gymnastic.
b6-n. (t s( . . . 1I'pOTaULY, cf. "26-32 n.
7. S,' 8 KaL 1I'pOTEpOY (A£X81'1, in a38-b2.
7-10. T~Y S' €Y !,€Pu ••• U1I'a.PX(lY. The vulgate reading, which
has little MS. support and involves a use of TTpOCT),:Tjr/M which is
foreign to A. and belongs to Theophrastus, is no doubt a later
rewriting of the original. P. (who of course was familiar with the
Theophrastean terminology) describes the curious 'conversion' as
TTPOCT),:Tjr/M (418. 28), and it may be his comment that gave rise to
the insertion of the spurious words into the text. Their absence
from the original text is confirmed by the remark of an anonymous
commentator (r89. 43 Brandis), inroypa</m oov ~fLLV (l~o~ iT(pov
TTpoTaCTfwv, OTTfP 0 e(OcppaCTTo~ KaAfL KaTc1 TTpOCTATJ""V. A. uses
TTPOCTAafL{3a.VfLV quite differently, of ordinary conversion (b27 , 2885,
42"34, 59"12, 22). On the later theory see Maier's learned note,
2a. 265 n. 2.
S. WU1I'(P Ka.1I'L TWY Ka8oAou, cf. 826-32.
CHAPTER 6
Reciprocal proof applied to second-figure syllogisms
5SbI3. The affirmative premiss cannot be established by a
reciprocal proof, because the propositions by which we should seek
to establish it are not both affirmative (the original conclusion
being in this figure always negative); the negative premiss can be
established.
IS.
u.
Syllogism
All B is A.
No C is A.
:. NoC is B.
No B is A.
All C is A .
. '. No C is B.
Reciprocal proof
All A is B.
NoC is B.
:. No C is A.
No C is B.
All A isC.
:. No A is B, :. No B is A.COMMENTARY
27. When one premiss is particular, the universal premiss can-
not be proved reciprocally. The particular premiss can, when the
universal premiss is affirmative:
Syllogism
All B is A.
Some C is not A.
:. Some C is not B.
33.
No B is A.
Some C is A.
:. Some C is not B.
Reciprocal proof
All A is B.
Some C is not B.
:. Some C is not A.
Some of that, some of which is not B, is A.
Some C is not B.
:. Some C is A.
S8bzo. T~ S€ r IlTJS€VL, omitted by ABI Cl N, is no doubt a
(correct) gloss; the words can easily be supplied in thought.
There is a similar ellipse in 5988.
z7. 'lfpoa~TJf9(LO'T)S S' £T£pas EaTaL, i.e. by adding the premiss
If no A is B, no B is A ; cf. 59"12-13.
z9. SLa. TTtv aUTTtv ahLav. The reference is to a38-b2.
33-8. €L S' ••• U'lfo.pX€LV, cf. &26-32 n.
3S-6. aUIl~aLvEL ya.p ••• cl.'lfOfaTLK";v, i.e. (the original syllo-
gism being No B is A, Some C is A, Therefore some C is not B),
if we take as new premisses No A is B, Some C is not B, we shall
have two negative premisses; and even if the first of these could
be altered into an affirmative form we should still have one
negative premiss, and therefore cannot prove what we want to
prove, that some C is A.
37. WS Kal ('lfl TWV Ka9o~ou. A. has not in fact used this method
to prove premisses of the universal moods of the second figure
(though he might have done; cf. 826-32 n.) ; he is thinking of the
use of it to prove the minor premiss of Celarent in the first figure
(826-3 2).
CHAPTER 7
Reciprocal proof applied to third-figure syllogisms
S8b39' Since a universal conclusion requires two universal pre-
misses, but the original conclusion is always in this figure parti-
cular, when both premisses are universal neither can be proved
reciprocally, and when one is universal it cannot be so proved.
59"3. When one premiss is particular, reciprocal proof is some-
times possible:
Syllogism
All C is A.
Some C is B.
:. Some B is A.
Reciprocal proof
All A isC.
Some B is A.
:. Some B is C, :. Some C is B.H. 6. S8 bzo-7. 5934I
Syllogism
15.
Some C is A.
All C is B.
:. Some B is A.
18.
Some C is not A.
All C is B.
:. Some B is not A.
24.
No C is A,
Some C is B.
:. Some B is not A.
443
Reciprocal proof
Some B is A.
All B is C.
:. Some C is A.
Some B is not A.
All B is C.
:. Some C is not A.
Some of that, some of which is not A, is C.
Some B is not A.
:. Some B is C.
(:. Some C is B).
[32. Thus (r) reciprocal proof of syllogisms in the first figure is
effected in the first figure when the original conclusion is affirma-
tive, in the third when it is negative; (2) that of syllogisms in the
second figure is effected both in the second and in the first figure
when the original conclusion is universal, both in the second and
in the third when it is particular; (3) that of syllogisms in the
third figure is effected in that figure; (4) when the premisses of
syllogisms in the second or third figur~ are proved by syllogisms
not in these figures respectively, these arguments are either not
reciprocal or not perfect.]
59324-31. 'ha.v I)' ••• uuAAOYLUfLoS, cf. 58326-32 n.
32-41. cS>a.VEPOV o~v •.• cl.TEAELS. The statement that in the
first figure, when the conclusion is affirmative, reciprocal proof is
effected in the first figure refers to the cases in which the original
conclusion is affirmative; and the statement is correct, since the
proof of both premisses of Barbara and of the minor premiss of
Darii were in the first figure. The statement that in the first
figure, when the conclusion is negative, reciprocal proof is effected
in the third figure refers to Celarent and Ferio; and the statement
is erroneous, since (r) it overlooks the fact that the proof of the
major premiss of Celarent was in the first figure (58"22-6), and (2)
it treats the proof of the minor premisses of Celarent and Ferio
(ib. 26-32, b 7- r2) as being in the third figure. The statement that
in the second figure, when the syllogism is universal, reciprocal
proof is effected in the first or second figure refers to the cases in
which the original conclusion is universal; and the statement is
correct, since the proof of the minor premiss of Carnestres was in
the second figure and that of the major premiss of Cesare in the
first. The statement that in the second figure, when the syllogism
is particular, reciprocal proof is in the second or third figure refers
to Baroco and Festino, and erroneously treats the proof of theCOMMENTARY
444
minor premiss of Festino (S8b33-8) as being in the third figure.
The statement that all the reciprocal proofs applied to the third
figure are in that figure (I) overlooks the fact that the proof of the
minor premiss of Datisi (S9a6-II) was in the first figure and (2)
treats the proof of the minor premiss of Ferison ("24-9) as being
in the third figure. Thus two types of error are involved: (a) the
errors with regard to the major premiss of Celarent and the minor
premiss of Datisi, and (b) the treatment of the reciprocal proofs of
the minor premisses of Celarent, Ferio, Festino, and Ferison as
being in the third figure. Take one case which will serve for all-
that of Celarent. Here we have No B is A, All C is B, Therefore
no C is A. A. converts the major premiss into All that, none of
which is A, is B (in other words If no X is A, all X is B), adds the
original conclusion No C is A, and infers that all C is B. P. (417.
22-{)) describes this as being a proof in the third figure, and an
anonymous scholiast (19°"17-27 Brandis) gives the reason, viz.
that the major premiss has a single subject with two predicates,
as the two premisses of a third-figure syllogism have. But this is
a most superficial analogy, since the relation between the protasis
and the apodosis of a hypothetical statement is quite different
from that between the premisses of a syllogism. The affinities of
the argument are with a first-figure syllogism, and it is easily
turned into one. The doctrine that there are three kinds of hypo-
thetical syllogism answering to the three figures is one of which
there is no trace in A.
The final statement (S9a39-41), that reciprocal proofs applied
to the second or third figure, if not effected in the same figure,
either are not Ka'Td riJv KVI(>"o/ Of Lt LV or are imperfect, at first sight
conflicts with the previous statement that all reciprocal proofs
applied to the third fIgure are effected in that figure. But the
statements can be reconciled by noting that all the normal con-
versions of syllogisms in these figures, viz. those of Camestres,
BarocQ, Disamis, and Bocardo (S8 b 18-zz, 2']-33, 59"15-18, 18-23),
are carried out in the original figure, while those that are not in
the original figure either involve the abnormal conversion men-
tioned in our last paragraph (ou 7Tapd riJv KVI(>"o/ Of Lt LV) (viz. those
of Festino and Ferison, S8b33-8, 59324-31) or are imperfect, invol-
ving a conversion of the conclusion of the new syllogism (viz.
those of Cesare and Datisi, S8 b2Z-7, 59"6-14).
The errors pointed out in (a) above might be a mere oversight,
but that pointed out in (b) is a serious one which A. is most
unlikely to have fallen into; and there can be little doubt that the
paragraph is a gloss.445
11. 7. 59 3 3 2 -4 1
CHAPTER 8
Conversion of first-figure syllogisms
59bI. Conversion is proving, by assuming the opposite of the
conclusion, the opposite of one of the premisses; for if the con-
clusion be denied and one premiss remains, the other must be
denied.
6. We may assume either (a) the contrary or (b) the contra-
dictory of the conclusion. A and 0, I and E are contradictories;
A and E, I and 0, contraries.
(A) Universal syllogisms
11.
(a)
All B is A.
AllC is B.
:. All C is A.
All B is A.
No C is A.
:. No C is B.
No C is A.
AIlC is B.
:. Some B is not A.
The contrary of the major premiss cannot be proved, since the
proof will be in the third figure.
20. So too if the syllogism is negative.
No B is A.
All C is B.
:. No C is A.
No B is A.
All C is A.
:. No C is B.
All C is A.
All C is B.
:. Some B is A.
25. (b) Here the reciprocal syllogisms will only prove the cod-
tradictories of the premisses, since one of their premisses will be
particular.
All B is A.
All C is B.
:. AllC is A.
All B is A.
Some C is not A.
:. Some C is not B.
Some C is not A.
All C is B.
:. Some B is not A.
3:1. So too if the syllogism is negative.
No B is A.
All C is B.
:. No C is A.
37.
No B is A.
Some C is A.
:. Some C is not B.
Some C is A.
All C is B.
:. Some B is A.
(B) Particular syllogisms
(a) If we assume the contradictory of the conclusion, both
premisses can be refuted;
(b) if the subcontrary, neither.
60"1. (a)
All B is A.
Some C is B.
:. Some C is A.
All B is A.
No C is A.
:. No C is B.
NoC is A.
Some C is B.
:. Some B is not A.446
5. (b)
COMMENTARY
All B is A.
Some C is B.
:. Some C is A.
All B isA.
Some C is not A.
Some C is B.
Some C is not A.
:. Some C is not B,
Nothing follows.
which does not disprove
Some C is B.
11. So too with a syllogism in Ferio. Both premisses can be dis-
proved by assuming the contradictory of the conclusion, neither
by assuming the subcontrary.
A. tells us in chs. 8-10 how the moods of the three figures can be
converted, but he does not tell us the point of the proceeding.
Conversion is defined as the construction of a new syllogism having
as premisses one of the original premisses and the opposite of the
original conclusion, and as conclusion the opposite of the other
premiss. Now when the original syllogism is in the second or
third figure and the converse syllogism in the first, the latter may
be regarded as an important confirmation of the former. For A.
always regards a first-figure syllogism as more directly proving its
conclusion than one in the second or third figure, so that if by a
first-figure syllogism we can prove that if the conclusion of the
originaI syllogism is untrue, one of its premisses must have been
untrue, we confirm the original syllogism. But in these chapters
A. also considers the conversion of a first-figure syllogism into a
second- or third-figure syllogism, that of a second-figure syllogism
into a third-figure syllogism, and that of a third-figure syllogism
into a second-figure syllogism; and such conversion can add
nothing to the conclusiveness of the original syllogism. What then
is the point of such conversion? It is stated in Top. 163829-36,
where pr.actice in the conversion of syllogisms is commended 1TpO)
YUfLvacrLav Ka, fL€>'erTjv T<7JV TOLOUTWV >'6ywv, i.e. to give the student
of logic experience in the use of the syllogism. But conversion of
syllogisms has this special importance for A., that it is identical
with the syllogistic part of reductio ad impossibile, which is a really
important method of inference; v. 61 a 18-33.
59},2-3. TO aKpov ••• T~ TEAEUTC1L~, the major term, the minor
term.
10. ou TLVL is used here in the sense of the more usual nv, OV
(i.e. an 0 proposition) ; cf. 63b26.
15-16. ou yap ••• <TXTJI'C1T05, cf. 29"16-18.
3«)-60"1. ou yap ••• ciVC1LpELV. In the case of original syllo-
gisms with two universal premisses (b II- 36) there were instances
(b 13- 20, 23-4) in which, though the conclusion of the converse
syllogism lacked universality ('?>'>'EL1TOVTO), b 4 0) , it disproved an1I. 8. 59bZ - 9. 60"Z7
447
original premiss (since a particular conclusion is contradictory to
an original universal premiss); but when one of the original
premisses is particular, the subcontrary of the original conclusion
will not prove even the contradictory, let alone the contrary, of
either of the original premisses. For (60"S-Il) if we combine it
with the universal original premiss we can only infer the sub-
contrary of the particular original premiss; and if we combine it
with that premiss we have two particular premisses and therefore
no conclusion.
CHAPTER 9
Conversion of second-figure syllogisms
(A) Universal syllogisms
The contrary of the major premiss cannot be proved, whether
we assume the contradictory or the contrary of the conclusion;
for the syllogism will be in the third figure, which cannot prove a
universal. (a) The contrary of the minor premiss can be proved
by assuming the contrary of the conclusion; (b) the contradictory
by assuming the contradictory.
All B is A.
No C is A.
:. No C is B.
(b)
All B is A.
No C is A.
No C is B.
ZI. (a)
z6.
All B is A.
All C is B.
:. All C is A.
All B is A.
Some C is B.
:. Some C is A.
No C is A.
All C is B.
:. Some B is not A.
No C is A.
Some C is B.
:. Some B is not A.
3X. So too with Cesare.
3Z.
(B) Particular syllogisms
(a) If the sub contrary of the conclusion be assumed, neither
premiss can be disproved; (b) if the contradictory, both can.
(a) No B is A.
Some C is A.
:. Some C is not B.
b r . (b) No B is A.
Some C is A.
:. Some C is not B.
No B is A.
Some C is A.
Some C is B.
Some C is B.
:. Some C is not A,
Nothing follows.
which does not disprove
Some C is A.
Some C is A.
No B is A.
All C is B.
All C is B.
:. Some B is A.
:. No C is A.
bS. So too with a syllogism in Baroco.
6oax8. Ka.8o).,ou S' ... UU).,).,0Y1Uf.L-0S, cf. 29&r6-r8.
z7. TJ f.L-EV AB ... ciVTlKELf.L-EVIIIS, i.e. the contradictory of theCOMMENTARY
major premiss will be proved, as
conclusion was assumed ('24--{j);
premiss will be proved, not the
proved when the contrary of the
34. Ka.9a.1rEP oilS' EV T~ 1rpWTt~
it was when the contrary of the
the contradictory of the minor
contrary, which was what was
conclusion was assumed ('22-4).
axt]~an, cf. 59 b39--{jO"I, 60'5-14.
CHAPTER 10
Conversion of third-figure syllogisms
6o b 6. (a) When we assume the subcontrary of the conclusion,
neither premiss can be disproved; (b) when the contradictory,
both can.
(A) Affirmative syllogisms
Some B is not A.
All C is B.
Nothing follows.
All C is A.
AIlC is B.
:. Some B is A.
(a)
Some B is not A.
AIlC is A.
Nothing follows.
14. So too if one premiss is particular; (a) if the subcontrary is
taken, either both premisses or the major premiss will be particu-
lar, and neither in the first nor in the second figure does this give
a conclusion; (b) if the contradictory is taken, both premisses can
be disproved.
zoo
22.
(b)
All C is A.
All C is B.
:. Some B is A.
No B is A.
All C is B.
:. No C is A.
No B is A.
AllC is A.
:. No C is B.
SO too if one premiss is particular.
All C is A.
Some C is B.
:. Some B is A.
25·
(a)
33. (b)
No C is
All C is
:. Some B
No C is
All C is
:. Some B
No B is A.
Some C is B.
:. Some C is not A.
(B) Negative syllogisms
Some B is A.
A.
B.
is not A.
A.
B.
is not A. :.
All C is B.
Nothing follows.
All B is A.
All C is B.
All C is A.
No B is A.
All C is A.
:. No C is B.
Some B is A.
No C is A.
Nothing follows.
All B is A.
No C is A.
:. No C is B.
37. So too if one premiss is particular.
(b)
No C is A.
Some C is B.
:. Some B is not A.
61&1. (a)
No C is A.
Some C is B.
:. Some B is not A.
All B is A.
Some C is B.
:. Some C is A.
Some B is A.
Some C is B.
Nothing follows.
All B is A.
No C is A.
:. No C is B.
Some B is A.
No C is A.
Nothing follows.449
We see, then, (I) how the conclusion in each figure must be
converted in order to give a new CDnclusion, (2) when the contrary
and when the contradictory of an original premiss is proved, (3)
that when the original syllogism is in the first figure, the minor
premiss is disproved by a syllogism in the second, the major by one
in the third, (4) that when the original syllogism is in the second
figure, the minor premiss is disproved by one in the first, the major
by one in the third, (5) that when the original syllogism is in the
third figure, the major premiss is disproved by one in the first,
the minor by one in the second.
s.
6o bl']-18. oihw 8' .•. .uu'l', cf. 26817-21, 27 b4--8, 28-39.
11)-20. EClV S' ••• cifJ4lonpa.l. Waitz is no doubt right in sug-
gesting that the reading a.VTurrp'</>CJJVTa, is due to a copyist who
punctuated after instead of before ai 7Tpo-rama,. Throughout chs.
8-10 the movement is from the opposite of the conclusion.
28. OIITW ya.p ••• UUAAOYlUjlO'i, cf. 28&26-30, bI5 - 2X , 31-5.
31. OlJlC ~v ••• r, cf. 26&30--6.
32. ollK ~v ••. UUAAOYlUjlO'i, cf. 27b6-8.
CHAPTER 11
'Reductio ad impossibile' in the first figure
61817. Reductio ad impossibile takes place when the contra-
dictory of the conclusion is assumed and another premiss is added.
It takes place in all the figures; for it is like conversion except
that conversion takes place when a syllogism has been formed and
both its premisses have been expressly assumed, while reductio
takes place when the opposite of the conclusion of the reductio
syllogism has not been previously agreed to but is obviously
true.
26. The terms, and the way we take them, are the same; e.g. if
all B is A, the middle term being C, then if we assume Some B is
not A (or No B is A) and AllC is A (which is true), Some B will not
be C (or no B will be C). But this is impossible, so thattheassump-
tion must be false and its opposite true. So too in the other
figures; wherever conversion is possible, so is reductio.
34. E, I, and 0 propositions can be proved by reductio in any
figure; A propositions only in the second and third. For to get a
syllogism in the first figure we must add to Some B is not A (or
No B is A) either All A is C or All D is B.
498s
G gCOMMENTARY
45 0
40. Propositions to
be proved
Remark
Reductio
All B is A.
All A is C.
}
Nothing follows.
Some B is not A. .
Some B is not A. }
All D is B.
7·
Nothing follows.
No B is A.
All D is B.
:. No D is A. } If the conclusion is false,
this only shows that
Some B is A.
All A is C.
No B is A. } Nothing follows.
Thus an A proposition cannot be proved by reductio in the first
figure.
10. Some
B is A .
No B is A.
}
All (or Some) C is B.
:. No C is A (or Some C is not A).
All A is C.
IS·
If the conclusion is false,
some B must be A.
}
No B is A.
Nothing follows.
17. The assumption some B is not A also leads to no conclusion.
Thus it is the contradictory of the conclusion that must be assumed.
All A is C.
Some B is A.
:. Some B is C.
19. No B is A.
22.
23·
24·
3 0 .
}
No A isC.
}
Some B is A.
:. Some B is not C.
Some B is A.
}
All C is B (or No C is B).
All A is C.
All B is A.
:. All B is C.
All B is A.
All C is B.
:. All C is A.
}
}
If the conclusion is false,
no B can be A.
If the conclusion is false,
no B can be A.
Nothing follows.
If the conclusion is false,
this only shows that
some B is not A.
If the conclusion is false,
this only shows that
some B is not A.
Thus it is the contradictory of the conclusion that must be assumed.
33. Not all B is A.
36 .
All A is C.
All B is A.
:. All B is C.
All B is A.
All C is B.
:. All C is A.
}
}
If the conclusion is false,
it follows that some B
is not A.
If the conclusion is false,
it follows tha t some B
is not A.II. 11. 61"27-31
37·
38 •
Reductio
No A isC.
All B is A.
:. No B is C.
All B is A.
No C is B.
45 1
Remark
} If the conclusion is false,
it follows that some B
is not A.
} Nothing follows.
(I) If the conclusion is
39·
All A is C.
Some B is A.
:. Some B is C.
false, this proves too
much, viz. that no B
IS A, which IS not
true; (2) the con-
clusion is not in fact
false.
So too if we were trying to prove that some B is not A (which
= Not all B is A).
6Z a ll. Thus it is always the contradictory of the proposition to
be proved, that we must assume. This is doubly fitting; (I) if we
show that the contradictory of a proposition is false, the proposi-
tion must be true, and (2) if our opponent does not allow the truth
of the proposition, it is reasonable to make the supposition that
the contradictory is true. The contrary is not fitting in either
respect.
Chapters 11-13 deal with reductio ad impossibile, in the three
figures. It is defined as an argument in which 'the contradictory
of the conclusion is assumed and another premiss is added to it'
(61 3 18-21); and in this respect it is like conversion of syllogisms
("21-2). But it is said to differ from conversion in that 'conversion
takes place when a syllogism has been formed and both its pre-
misses have been expressly assumed, while reductio takes place
when the opposite' (i.e. the opposite of the conclusion of the
reductio syllogism) 'has not been previously agreed upon but is
obviously true' (322-5). This is equivalent to saying that pre-
viously to the reductio syllogism no ostensive syllogism has been
formed, so that when A. describes the reductio as assuming the
contradictory of the conclusion, this must mean 'the contra-
dictory of the conclusion we wish to prove'.
What reductio has in common with conversion is that it is an
indirect proof of a proposition, by supposing the contradictory to
be true and showing that from it and a proposition known to be
true there follows a proposition known or assumed to be false.
61"27-31. ala v EL • • • aVTLICEL .... EVOV. A. here leaves it an open
question whether it is the contradictory (p.~ 7TaVT{, 328) or theCOMMENTARY
452
contrary (J.L"1S£vt. ibid.) of the proposition to be proved that is
to be assumed as the basis of the reductio syllogism. But in the
course of the chapter he shows that the assumption of the contrary
of an A or E proposition (bl - Io• 24-33). or the subcontrary of an I
or 0 proposition (b I7 - 1 8, 3~2·IO), fails to disprove the A, E, I,
or 0 proposition.
b 7-8. ouS' o!'a.v ••• U'II'clPXELV. This is a repetition of what A.
has already said in "4o-- b I. The sentence would read more natu-
rally if we had WU7T£P ooS'.
6~a4-'1. in ••• T~ B. The received text, £Tt OV 'II'apa. n}v V1ro-
9£C1W uvJ.L/3atv£L TO ciSvvaTOv, gives the wrong sense 'further, the
impossible conclusion is not the result of the assumption'. The
sense required is rather that 'the assumption leads to nothing
impossible; for if it did, it would have to be false (since a false
conclusion cannot follow from true premisses), but it is in fact
true; for some B is in fact A'. A. as usual treats Some B is not
A as naturally implying Some B is A. The reading I have adopted
receives support from n.'s ovS£' and from P.'s paraphrase ovSo,
aT01I'OV £7T£TaL.
13. a.~twl-La., 'assumption'. This sense is to be distinguished
from a second sense, in which it means 'axiom'. Examples of
both senses are given in our Index.
IS. fl.TJ T£8'1O'LV is used in the sense of 'does not admit'. and has as
its understood subject the person one is trying to convince. Cf.
Met. r063bro J.L"19& Tt9£,VT£, ciVaLpOV<1L TO SLaM),£<19aL.
19. 8clnpov (se. TO 7TaVTt) ... 8clTEPOV (sc. TO J.L"1S£vt) answers to
n}v KaTa.paaLv •.• n}v ci7TOtf>onLV in "r6.
CHAPTER 12
'Reductio ad impossibile' in the second figure
Thus all forms of proposition except A can be proved by
reductio in the first figure. In the second figure all four fonns can
be proved.
h"~o.
Proposition to
be pruved
:l3. All B is A.
28.
Reductio
All C is A.
Rem4rk
}
Some B is not A.
:. Some B is not C.
All C is A.
")
No B is A.
:. No B is C.
~
j
If the conclusion is false,
all B must be A.
If the conclusion is false,
it only follows that
some B is A.453
Propositions to
be proved
]2. Some B is A.
36•
37.
No B is A.
40. Some B is not A.
b2 •
Reductio
All C is A.
No B is A.
}
:. No B isC.
Some B is not A.
Remark
If the conclusion is false,
some B must be A.
Cf. remark in 61b17-18.
NoCis A.
}
Some B is A.
:. Some B is not C. If the conclusion is false,
no B can be A.
NoCis A.
AU B is A.
:. No B is C. If the conclusion is false,
it follows that some B
is not A.
}
Thus all four kinds of proposition can be proved in this
figure.
62"3:.z-3. on Sf ... {,.rra.pXEL TO A. Here, in "40, and in bI4 the
best MSS. and P. read on, while Bekker and Waitz with little MS.
authority read 07"£. There can be no doubt that on is right; the
construction is elliptical-'with regard to the proposition that',
'if we want to prove the proposition that'. Ct. PI. erat. 384 c 3,
Prot. 330 e 7. Phaedo IIS d 2, Laws 688 b 6.
36-7. TQlIT' EC7TQL • • • ax"flQTOS. 'Tav-r' should obviously be
read. for the vulgate 'Taih'. So too in bIO, 23. The reference is to
6IbI7-I8.
40.
on S' oll 'II'QVTL, cf. "32-3 n.
CHAPTER 13
'Reductio ad impossibile' in the third figure
6:.z b S. All four kinds of proposition can be proved in this figure.
Propositions to
be proved
All B is A.
8.
11.
Some B is A.
Reductio
Remark
SQme B is not A.}
All B is C.
:. Some C is not A. If the conclusion is false,
all B must be A.
No B is A.
}
All B is C.
:. Some C is not A. If the conclusion is false,
this only shows that
some B is A.
No B is A.
}
Some B is C.
:. Some C is not A. If the conclusion is false,
some B must be A.COMMENTARY
454
Propositions to
be proved
14. No B is A.
18.
19. Some B is not A.
Z3.
Reductio
Some B is A.
All B is C.
:. Some C is A.
All B is A.
All B is C.
:. Some C is A.
All B is A.
All B isC.
:. Some C is A.
Some B is A.
Remark
} n
}
}
the conclusion is false,
no B can be A.
If the conclusion is false,
this only shows that
some B is not A.
If the conclusion is false,
some B must not be A.
CL remark in 61 b3tr-62a8.
Z5. Thus (r) in all cases of reductio what we must suppose true
is the contradictory of the proposition to be proved; (2) an affinna-
tive proposition can in a sense be proved in the second figure, and
a universal proposition in the third.
6z blcr-I1.
The reference is to 6r br-
8, 62828-32. For the reading cf. 62836-7 n.
14. on 8'. Cf. 832-3 n.
18. OU 8£LKVUTaL TO rrpoTE9Ev. For the proof of this cf. the
corresponding passage on the first figure, 6rb24-33.
23-4' TauT' £aTaL •.• rrpOnPTJI-lEVwV. The reference is to the
corresponding passage on the first figure, 6rb39--62a8. For the
reading cf. 62'36-7 n.
z6-8. 8il),ov 8£ • • . Ka9o),ou, i.e. an affirmative conclusion,
which cannot be proved ostensively in the second figur~, can be
proved by a reductio in that figure (62a23-8, 32--6) ; and a universal
conclusion, which cannot be proved ostensively in the third figure,
can be proved by a reductio in it (62bS-8, II-r4).
TauT' EaTaL ••• rrpoTEpov.
CHAPTER 14
The relations between ostensive proof and 'reductio ad impossibile'
6ZbZ9. Reductio differs from ostensive proof by supposing what
it wants to disprove, and deducing a conclusion admittedly false,
while ostensive proof proceeds from admitted premisses. Or
rather, both take two admitted propositions, but ostensive proof
takes admitted propositions which form its premisses, while
reductio takes one of the premisses of the ostensive proof and the
contradictory of the conclusion. The conclusion of ostensive
proof need not be known before, nor assumed to be true or to be
false; the conclusion of a reductio syllogism must be already knownII. 13. 6Z b 1o-z8
455
to be false. It matters not whether the main conclusion to be
proved is affirmative or negative; the method is the same.
38. Everything that can be proved ostensively can be proved
by reductio, and vice versa, by the use of the same terms. (A)
When the reductio is in the first figure, the ostensive proof is in
the second when it is negative, in the third when it is affirmative.
(B) When the reductio is in the second figure, the ostensive proof
is in the first. (C) When the reductio is in the third figure, the
ostensive proof is in the first when affirmative, in the second when
negative.
Data
Reductio
(A) First figure
63"7. All A is C.
Ostensive proof
Second figure
All A is C.
:. if some B is A, some
B isC.
But No B is C.
:. No B is A. All A is C.
No B isC.
:. No B is A.
All A is C.
Some B is not C. All A is C.
:. if all B is A, all Bis C.
But some B is not C.
:. Some B is not A. All A isC.
Some B is not C.
:. Some B is not A.
No A isC.
All B is C. No A isC.
:. if some B is A, some
B is not C.
But all B is C.
:. No B is A. No A isC.
All B is C.
:. No B is A.
No A isC.
Some B is C. No A isC.
:. if all B is A,no B isC.
But some B is C.
:. Some B is not A. No A isC.
Some B isC.
:. Some B is not A.
18. All C is A.
All C is B. If no B is A, then since
all C is B, no C is A.
But all C is A.
:. Some B is A. ~3. All C is A.
Some C is B.
No B isC.
16.
If no B is A, then since
all C is B, no C is A.
Third figure
AllCis A.
All C is B.
:. Some B is A.
All C is A.
Some C is B.
:. Some B is A.
But all C is A.
:. Some B is A.
Some C is A.
All C is B.
If no B is A, then since
allC is B, noC is A.
But some C is A.
:. Some B is A.
Some C is A.
All C is B.
:. Some B is A.45 6
COMMENTARY
Data
25· All C is A.
All B is C.
29· All C is A.
Some B is'C.
p. No C is A.
All B is C.
:.
:.
:.
:.
:.
:.
35·
NoC is A.
Some B isC.
:.
:.
40· All C is A.
All B is C.
b 3· All C is A.
Some B isC.
5· No A is C.
All BisC.
8. No A isC.
Some B is C.
Reduc/io
Ostensive proof
(B) Second figure
Firs/figure
All C is A.
All C is A.
if some B is not A,
All B is C.
some B is not C.
:. All B is A.
But all B is C.
All B is A.
All C is A.
All C is A.
Some B isC.
if no Bis A, no B isC.
But some B is C.
:. Some B is A.
Some B is A.
No C is A.
No C is A.
All B is C.
if some B is A, some
B is not C.
:. No B is A.
But all B is C.
No B is A.
No C is A.
NoC IS A.
ifallBisA,noBisC.
Some B is C.
But some B is C.
:. Some B is not A.
Some B is not A.
(C) Third figure
If some B is not A,
then since all B is C,
some C is not A.
But all C is A.
:. All B is A.
If no B is A, then since
some B is C, some C
is not A.
But all C is A.
:. Some B is A.
Firs/figure
All C is A.
All B is C.
:. All B is A.
All C is A.
Some B is C.
:. Some B is A.
Second figure
:.
:.
:.
:.
All B is C.
if some B is A, some
A isC.
But no A is C.
No B is A.
Some B is C.
if all B is A, some A
isC.
But no A is C.
Some B is not A.
No A is C.
All B is C.
:. No B is A.
No A isC.
Some B is C.
:. Some B is not A.
12. Thus any proposition proved by a reductio can be proved
ostensively, by the use of the same teITIls; and vice versa. If we457
take the contradictory of the conclusion of the ostensive syllogism
we get the same new syllogism which was indicated in dealing
with conversion of syllogisms; and we already know the figures
in which these new syllogisms must be.
62b32-3. Aal1~a.vouu~ IlEV o~v ••. <>1l0AoyoUI1€vas.
j.£~v ovv
introduces a correction. The usage is common in dialogue (Den-
niston, The Greek Particles, 475-8), rare in continuous speech (ib.
478--9) ; for Aristotelian instances cf. Rhet. I399"I5, 23·
36-7. ~9a SE •.• EO"1"LV. Cf. A n. Post. 87"I4 OTUV p.~v oov !J TO
uvj.£'TT'pauj.£a YVWPLj.£WTfPOV OTL OUK lUTLV, .;, £ls TO dSvvaTOV ylv£TaL
d'TT6S£LgLS.
41-63"7. <hay I1EV yap ••• I1€U't'. There are also negative osten-
sive syllogisms in the third figure answering to reductio syllogisms
in the first, ostensive syllogisms in the third answering to reductio
syllogisms in the second, and negative ostensive syllogisms in the
first answering to reductio syllogisms in the third. E.g.,
Data
NoC is A.
All C is B.
Reductio
If all B is A, then since all C is B,
all C is A.
But no C is A.
:. Some B is not A.
Ostensive syllogism
NoC is A.
All C is B.
:. Some B is not A.
But A.'s statement here is a correct summary of the correspon-
dences he gives in this chapter, which are presumably not meant
to be exhaustive.
41-63"1. <> uuAAOYLUI10S .•• TO ciAT)9€s, the reductio . .. the
ostensive proof.
63"7. EO"1"W yap SESELYI1€vOV, sc. by reductio.
bU-13. 41avEpov o~v .•• ciSUVa.TOU. KU( SHKTLKWS means 'os-
tensively as well as by reductio', so that Kat Sui TOU dSvvcholJ is
superfluous; indeed, it makes the next sentence pointless.
16-17. YLvovTaL yap ••• civnuTpocj>f)s, i.e. the reductio syllogism
is related to the ostensive syllogism exactly as the converse syllo-
gisms discussed in chs. 8-IO were related to the original syllogisms.
CHAPTER 15
Reasoning from a pair of opposite premisses
63b22. The following discussion will show in what figures it is
possible to reason from opposite premisses. Of the four verbal
oppositions, that between I and 0 is only verbally an opposition,COMMENTARY
458
that between A and E is contrariety, and those between A and
and between E and I are true oppositions.
31. There cannot be such a syllogism in the first figure-not an
affirmative syllogism because such a syllogism must have two
affirmative premisses; not a negative syllogism because opposite
premisses must have the same subject and the same predicate,
but in this figure what is subject of one premiss is predicate of the
other.
40. In the second figure there may be both contradictory and
contrary premisses. If we assume that all knowledge is good and
that none is, it follows that no knowledge is knowledge.
64 3 4. If we assume that all knowledge is good and that no
medical knowledge is so, it follows that one kind of knowledge is
not knowledge.
7. If no knowledge is supposition and all medical knowledge is
so, it follows that one kind of knowledge is not knowledge.
12. Similarly if the minor premiss is particular.
IS. Thus self-contradictory conclusions can be reached pro-
vided that the extreme terms are either the same or related as
whole to part.
20. In the third figure there cannot be an affirmative syllogism
with opposite premisses, for the reason given above; there may
be a negative syllogism, with or without both premisses universal.
If no medical skill is knowledge and all medical skill is knowledge,
it follows that a particular knowledge is not knowledge.
27. So too if the affirmative premiss is particular; if no medical
skill is knowledge and a particular piece of medical skill is know-
ledge, a particular knowledge is not knowledge. When the premisses
are both universal, they are contrary; when one is particular,
contradictory.
33. Such mere assumption of opposite premisses is not likely
to go unnoticed. But it is possible to infer one of the premisses by
syllogism from admissions made by the adversary, or to get it in
the manner described in the Topics.
37. There being three ways of opposing affirmations, and the
order of the premisses being reversible, there are six possible
combinations of opposite premisses, e.g. in the second figure AE,
EA, AO, El; and similarly a variety of combinations in the third
figure. So it is clear what combinations of opposite premisses are
possible, and in what figures.
b 7 . We can get a true conclusion from false premisses, but not
from opposite premisses. Since the premisses are opposed in
quality and the terms of the one are either identical with, or
°459
related as a whole to part to, those of the other, the conclusion
must be contrary to the fact-of the type 'if 5 is good it is not
good'.
13. It is clear too that in paralogisms we can get a conclusion
of which the apodosis contradicts the protasis, e.g. that if a certain
number is odd it is not odd; for if we take contradictory premisses
we naturally get a self-contradictory conclusion.
17. A self-contradictory conclusion of the type 'that which is
not good is good' cannot be reached by a single syllogism unless
there is an explicit self-contradiction in one premiss, the premisses
being of the type 'every animal is white and not white, man is an
animal'.
21. Otherwise we must assume one proposition and prove the
opposite one; or one may establish the contrary propositions by
different syllogisms.
25. This is the only way of taking our premisses so that the
premisses taken are truly opposite.
63b26. T~ ou nVL = T~ TLVl. oU. Cf. 59bIO.
64"21-2. lha. TTJV ELP'lflEV'lv QLTLQV ••• aXtlflQTO'i, cf. 63 b33-5.
23-30. EaTw ya.p ••• E1TLaTtlflTJv, A. here seems to treat the
premisses. All A is B, No A is C ("23-7) and the premisses Some
A is B, No A is C ("27-30) as yielding the conclusion Some C is
not B, which they do not do. But since B and C stand for the
same thing, knowledge, these premisses may be rewritten re-
spectivelyas No A is B, All A is C and as No A is B, Some A
is C, each of which combinations does yield the conclusion Some
C is not B.
36-7. ian liE ... AQ~ELV. The methods of obtaining one's pre-
misses in such a way as to convince an incautious opponent, so
that he does not see what he is being led up to, are described at
length in Top. viii. I. But they reduce themselves to two main
methods-the inferring of the premisses by syllogism and by
induction (155 b35--{i).
37-8. E1TEt liE .•. TpEL'i, i.e. AE, AO, lE-not 10, since an I
proposition are only verbally opposed
proposition and an
(63 b2 7-8).
38- b 3. E~QXW'i aUfl~Q'VEL ••• 0POU'i' Of the six possible com-
binations AE, AO, lE, EA, OA, El, A. evidently intends to
enumerate in b I - 3 the four possible in the second figure-AE, EA,
AO, El. TO A ... f.L~ 1TaVTt gives us AE, EA, AO; Kat 1T(f'\,v TOVTO
a.vnaTp.:.pa, KaTd TOU, opov, must mean 'or we can make the uni-
versal premiss negative and the particular premiss affirmative'(EI).
°COMMENTARY
The combinations possible in the third figure (1)3-4) are of course
EA,OA,El.
b8. KCl9all'Ep ELP"TClL lI'pOTEPOV, in chs. 2-4.
9-13. cl.El yap ••• flEp05. A. has shown in 63b40-64"3I how, by
taking two premisses opposite in quality, with the same predicate
and with subjects identical or related as genus to species (second
figure), or with the same subject and with predicates identical or
related as genus to species (third figure), we can get a conclusion
of the form No A is A (illustrated here by £l £crnv dya8ov, p.~ £rVC11
dya8ov) or Some A is not A.
13-15. S"Mv SE ••• lI'EPLTTOV. A paralogism is defined in Top.
101"13-15 as an argument that proceeds from assumptions appro-
priate to the science in question but untrue. This A. aptly illus-
trates here by referring to the proof (for which v. 41826-7 n.) that
if the diagonal of a square were commensurate with the side, it
would follow that odds are equal to evens, i.e. that what is odd is
not odd.
15-16. EK yap TWV cl.VTLKELflEVWV ••• O'UAAOYLO'flo5, 'since, as we
saw in bg-- I3 , an inference from premisses opposite to one another
must be contrary to the fact'.
17-25. SEL SE ••• O'UAAOYLO'flwv. A. now turns to quite a
different kind of inference, in which the conclusion is not negative
but affirmative-not No A is A or Some A is not A, but All (or
Something) that is not A is A. He puts forward three ways in
which such a conclusion may be reached. (I) (1)20-1) It may be
reached by one syllogism, only if one premiss asserts contraries of
a certain subject; e.g. Every animal is white and not white, Man
is an animal, Therefore man is white and not white (from which
it follows that Something that is not white is white). (2) (b2I- 4)
A more plausible way of reaching a similar conclusion is, not to
assume in a single proposition that a single subject has opposite
attributes, but to assume that it has one and prove that it (or
some of it) has the other, e.g. to assume that all knowledge is
supposition, and then to reason 'No medical skill is supposition,
All medical skill is knowledge, Therefore some knowledge is not
supposition'. (3) (1)2S) We may establish the opposite propositions
by two separate syllogisms.
24. WO'1I'EP ot EAEYXOL yLVOVTClL. Anyone familiar with Plato's
dialogues will recognize the kind of argument referred to, as one
of the commonest types used by Socrates in refuting the theories
of others (particularly proposed definitions).
25-'7. WO'TE S' ••• lI'pOTEpOV. It is not clear whether this is
meant to sum up what has been said in b IS - 2S of the methods ofobtaining a conclusion of the form 'Not-A is A', or to sum up the
main results of the chapter as to the methods of obtaining a
conclusion of the form 'A is not A'. The latter is the more prob-
able, especially in view of the similarity of the language to that in
63b22-8 .. WOT' £tvaL o,aVT{a KaT' aA7}8€Lav Ta ElATJp.p.iva means then
'so that the premisses of a single syllogism are genuinely opposed
to one another'. How, and how alone, this can be done, has been
stated in 63b40-64b6.
:z6. OGK iO'TlV, sc. >.a{kiv, which is easily supplied from the
previous Ta ElA."p.p.€va.
CHAPTER 16
Fallacy of 'Petitio principii'
64 b :z8.
Petitio principii falls within the class of failure to prove
the thesis to be proved; but this may happen if one does not
syllogize at all, or uses premisses no better known than the
conclusion, or logically posterior to it. None of these constitutes
petitio principii.
34. Some things are self-evident; some we know by means of
these things. It is petitio principii when one tries to prove by
means of itself what is not self-evident. One may do this (a) by
assuming straight off the point at issue, or (b) by proving it
by other things that are naturally proved by it, e.g. proposition A
by B, and B byC, when C is naturally proved by A (as when people
think they are proving the lines they draw to be parallel, by means
of assumptions that cannot be proved unless the lines are parallel).
65"7. People who do this are really saying 'this is so, if it is so';
but at that rate everything is self-evident; which is impossible.
10. (i) If it is equally unclear that C is A and that B is A, and
we assume the latter in order to prove the former, that in itself
is not a petitio principii, though it is a failure to prove. But if B
is identical with C, or plainly convertible with it, or included in
its essence, we have a petitio principii. For if B and C were con-
vertible one could equally well prove from 'c is A' and 'C is B'
that B is A (if we do not, it is the failure to convert 'C is B', and
not the mood we are using, that prevents us) ; and if one did this,
one would be doing what we have described above, effecting a
reciprocal proof by altering the order of the three terms.
19. (ii) Similarly if, to prove that C is A, one assumed that
C is B (this being as little known as that C is A), that would be
a failure to prove, but not necessarily a petitio principii. But ifCOMMENTARY
A and B are the same either by being convertible or by A's being
necessarily true of B, one commits a petitio principii.
26. Petitio principii, then, is proving by means of itself what
is not self-evident, and this is (a) failing to prove, (b) when con-
clusion and premiss are equally unclear either (ii above) because
the predicates asserted of a single subject are the same or (i above)
because the subjects of which a single predicate is asserted are the
same. In the second and third figure there may be petitio principii
of both the types indicated by (i) and (ii). This can happen in an
affirmative syllogism in the third and first figures. When the
syllogism is negative there is petitio principii when the predicates
denied of a single su bj ect are the same; the two premisses are not
each capable of committing the petitio (so too in the second figure),
because the terms of the negative premiss are not interchangeable.
35. In scientific proofs petitio principii assumes true proposi-
tions; in dialectical proofs generally accepted propositions.
64b29' TOlITO SE uUf.L~a£vEL 1To~~axw5. bnuvfLfJa{YH, which ap-
pears in all the early MSS. except n, is not found elsewhere in any
work earlier than ps.-A. Rhet. ad Al. (1426 3 6), and the lm- would
have no point here.
3 I. KaL EL SUl. TWV UUT£PWV TO 1TpOTEpOV refers to logical priority
and posteriority. A. thinks of one fact as being prior to another
when it is the reason or cause of the other; cf. An. Post. 71b22,
where 7TpO'r£pWY and alT{wy TaU aVfL7TEpuafLaTO, are almost synony-
mous.
36-7. f.LT] TO SL' aUTOU YVWUTOV ••• £1TLXELPTI SELKVUVaL is, in
Aristotelian idiom, equivalent to TO fL~ SL' aVTOV YYWG'TOY . • .
lmXHpfj OHKyVyaL. Cf. M et. 1068 3 28 fL€TafJ€fJA'Y}KO, laTaL . . . El, fL~
T~Y TvxouO'ay aA, Rhet. 1364b37 0 7TUYTE, aipOVYTaL (KU>">"L6y EaTL) TOU
fL~ 0
7TUYTE,.
65'4-7. 01TEP 1TOLOUULV ••• 1Tapa~~"~wv. P. has a particular
explanation of this (454. 5-7) fJOVAOYTaL yap 7TapaAA~AOv, dBE{a,
am) TOV fL£O''Y}fLfJPLYOV KVKAOV KaTaypu.paL OWaTOY (01'), KaL AafLfJa-
YOVO'L O"YJfLEI:OY W, El7TEl:y 7TPOO'7TL7T'TOY 7TEpL TO E7T{7TEOOY EKdyov, KaL
oilTw<; EKfJuAAovaL Ta, EvBE{a,. But we do not know what authority
he had for this interpretation; the reference may be to any pro-
posed manner of drawing a parallel to a given line (which involves
proving two lines to be parallel) which assumed anything that
cannot be known unless the lines are known to be parallel.
Euclid's first proof that two lines are parallel (1. 27) assumes only
that if a side of a triangle be produced, the exterior angle is
greater than either of the interior and opposite angles (1. 16), but1TapaAA~AovS' UV/-L1TL1T'THV .
£l 'TO 'Tplywvov
£X'" 1TA£LOVS' ap()os OV£LV it seems that some geometer known to A.
assumed, for the proof of I. 27, that the angles of a triangle =
two right angles (I. 32), which involves a circulus in probando; and
it is probably to this that 'ToLaiha
1TapaAA~Awv refers. As
from 66"13-15 ofov 'Tch'
a ...
Heiberg suggests (A bh. zur Gesch. d. M ath. W issenschaften, xviii.
19), it may have been this defect in earlier text-books that led
Euclid to state the axiom of parallels (fifth postulate) and to
place I. 16 before the proof that the angles of a triangle = two
right angles. For a full discussion of the subject cf. Heath,
M athematics in Aristotle, 27-30.
X0-25. Et oov •.. SftAOV. A. here points out the two ways in
which petitio principii. may arise in a first-figure syllogism. Let
the syllogism be All B is A, All C is B, Therefore all C is A. (I)
("10-19) There is petitio principii if (a) we assume All B is A when
this is as unclear as All C is A, and (b) B is (i) identical with C (i.e.
if they are two names for the same thing), or (ii) manifestly con-
vertible with C (as a species is with a differentia peculiar to it) or
(iii) B is included in the essential nature of C (as a generic character
is included in the essence of a species). If Band C are convertible
(this covers cases (i) and (ii)) and we say All B is A, All C is B,
Therefore all C is A. we are guilty of petitio principii; for (316-17)
if we converted All C is B we could equally well prove All B is A
by means of the other two propositions-All C is A, All B is C,
Therefore all B is A.
In al5 the received text has IJ1Td.pX£L. IJ1Td.pX"'V is A.'s word for
the relation of any predicate to its sUbject. and lJ1Td.PX£L is therefore
too wide here. A closer connexion between subject and predicate
is clearly intended. and this is rightly expressed by £VV1Td.PXH, 'or
if B inheres as an element in the essence of C. P. consistently
uses £VV1Td.PX£LV in his commentary on the passage (451. 18,454. 21,
23,455. 17)· The same meaning is conveyed by 'Tifj £1Tw()aL 'Tifj B 'TO
A ('by A's necessarily accompanying B') in 322. An early copyist
has assimilated £VV1Td.pX'" here to IJ1TaPX£L in 316. For confusion in
the MSS. between the two words cf. An. Post. 73337-8 n., 38,
84313, 19,
20.
The general principle is that when one premiss connects identi-
calor quasi-identical terms, the other premiss commits a petitio
principii; it is the nature of a genuine inference that neither of the
premisses should be a tautology, that each should contribute
something to the proof.
VUV SE TOUTO KWhUEl, aAA' OUX ;, TPC)1I'OS (317) is difficult. P.
(455. 2) is probably right in interpreting 'TOV'TO as 'TO J.L~ o.v'Turrplc/mv.COMMENTARY
'If he does not prove 'B is A' from 'C is A' (sc. and 'C is B'), it
is his failure to convert 'C is B', not the mood he is using, that
prevents his doing so'. Not the mood; for the mood Barbara,
which he uses when he argues 'B is A, C is B, Therefore C is A',
has been seen in 57b35-58"15 to permit of the proof of each of its
premisses from the other premiss and the conclusion, if the terms
are convertible and are converted.
If (A. continues in "18-19) we do thus prove All C is A from All
B is A and All C is B, and All B is A from All C is A and All B is
C (got by converting All C is B), we shall just be doing the useless
thing described above ("1-4)-ringing the changes on three terms
and proving two out of three propositions, each from the two
others, which amounts to proving a thing by means of itself.
(2) &19-25. Similarly" we shall have a petitio principii if (a) we
assume All C is B when this is no clearer than All C is A, and (b)
(i) A and B are convertible or (ii) A belongs to the essence of B.
(i) here corresponds to (i) and (ii) above, (ii) to (iii) above.
Thus where either premiss relates quasi-identical terms, the
assumption of the other commits a petitio principii.
ZOo 01)1I"W TO ~~ apXllS, sc. atTt'wBa{ Eu-n, or at-rE'i-rat.
24. ELp"Tat, in 64b34-8.
26-35. Et o(;v ••• O'u~~oyl.O"\lous. A. here considers petitio
principii in the second and third figures, and in negative moods of
the first figure. He begins by summarizing the two ways in which
petitio has been described as arising in affirmative moods of the
first figure-7) -rip -rau-rcl -rip av-rip 7) -rip -rav-roll -rOt~ au-roL~ inrdpX£tIl.
-rav-roll -rOt~ au-rot~ refers to case (I) ("10--19), in which an identical
term A is predicated of quasi-identical terms B, C in a premiss
and in the conclusion, -rau-rcl -rip au-rip to case (2) ("19-25). in which
quasi-identical terms B, A are predicated of an identical term C
in a premiss and in the conclusion. A study of the paradigms of
the three figures
Fir si figure
B is (or is not) A.
C is B.
:. C is (or is not) A.
Second figure
A is (or is not) B.
C is not (or is) B.
:. C is not A.
Third figure
B is (or is not) A.
B isC.
:. C is (or is not) A.
shows that (I) can occur in affirmativ.e and negative moods of the
first figure (Barbara, Celarent) and of the third (Disamis, Bocardo),
and (2) in affirmative moods of the first (Barbara, Darn), and in
moods of the second (which are of course negative) (Camestres,
Baroco). It is at first sight puzzling to find A. saying that both
(I) and (2) occur in the second and third figures; for (I) seems notn.
16. 65820--1
to occur in the second, nor (2) in the third. But in Cesare (No A is
B, All C is B, Therefore no C is A), in ~ying No A is B we are
virtually saying No B is A, and therefore in the major premiss and
the conclusion may be denying the identical term A of quasi-
identical terms (case I)). And in Datisi (All B is A, Some B is C,
Therefore some C is A), in saying Some B is C we are virtually
saying Some C is B, and therefore in the minor premiss and the
conclusion may be asserting quasi-identical terms of the identical
term C (case (2)).
Having pointed out the distinction between case (I) and case
(2), A. proceeds to point out that the affirmative form of each can
only occur in the first and third figures (since there are no affirma-
tive moods in the second figure). He designates the negative
forms of both kinds of petitio by the phrase (J.rav Ta aVTa a1TO TOV
QVToii. We might have expected him to distinguish from this the
case oTav TaVTov a1To TWV aVTwv, but the distinction is unnecessary,
since in denying an identical term of two quasi-identical ones we
are (since universal negative propositions are simply convertible)
virtually denying them of it. Finally, he points out that in
negative syllogisms the two premisses are not alike capable of
committing a petitio. Since the terms of a negative premiss can-
not be quasi-identical, it is only in the negative premiss that a
petitio can be committed.
27. OTQV, sc. TOVTO yCvrJTat.
CHAPTERS 17, 18
FaUacy of false cause
65"38. The objection 'that is not what the falsity depends on'
arises in the case of reductio ad impossibile, when one attacks the
main proposition established by the reductio. For if one does not
deny this proposition one does not say 'that is not what the falsity
depends on', but 'one of the premisses must have been false'; nor
does the charge arise in cases of ostensive proof, since such a proof
does not use as a premiss the counter-thesis which the opponent is
maintaining.
b4. Further, when one has disproved a proposition ostensively
no one can say 'the conclusion does not depend on the supposi-
tion' ; we can say this only when, the supposition being removed,
the conclusion none the less follows from the remaining premisses,
which cannot happen in ostensive proof, since there if the premiss
is removed the syllogism disappears.
9. The charge arises, then, in relation to reductio, i.e. when the
4985
H hCOMMENTARY
supposition is so related to the impossible conclusion that the
latter follows whether the former is made or not.'
13. (1) The most obvious case is that in which there is no
syllogistic nexus between the supposition and the impossible con-
clu~ion; e.g. if one tries to prove that the diagonal of the square
is incommensurate with the side by applying Zeno's argument and
showing that if the diagonal were commensurate with the side
motion would be impossible.
21. (2) A second case is that in which the impossible conclusion
is syllogistically connected with the assumption, but the im-
possibility does not depend on the assumption. (a) Suppose that
B is assumed to be A, to be B, and .1 to be
but in fact .1 is
not B. If when we cut out A the other premisses remain, the
premiss' B is A' is not the cause of the falsity.
28. (b) Suppose that B is assumed to be A, A to beE, andEto
be Z, but in fact A is not Z. Here too the impossibility remains
when the premiss' B is A' has been cut out.
32. For a reductio to be sound, the impossibility must be con-
nected with the terms of the original assumption' B is A'; (a)
with its predicate when the movement is downward (for if it is .1's
being A that is impossible, the elimination of A removes the false
conclusion), (b) with its subject when the movement is upward
(if it is B's being Z that is impossible, the elimination of B removes
the impossible conclusion). So too if the syllogisms are negative.
66"1. Thus when the impossibility is not connected with the
original terms, the falsity of the conclusion is not due to the
original assumption. But (3) even when it is so connected, may
not the falsity of the conclusion fail to be due to the assumption?
If we had assumed that K (not B) is A, and is K, and .1 is
the impossible conclusion '.1 is A' may remain (and similarly if
we had taken terms in the upward direction). Therefore the
impossible conclusion does not depend on the assumption that
B is A.
8. No; the charge of false cause does not arise when the sub-
stitution of a different assumption leads equally to the impossible
conclusion, but only when, the original assumption being elimi-
nated, the remaining premisses yield the same impossible conclu-
sion. There is nothing absurd in supposing that the same false
conclusion may result from different false premisses; parallels will
meet if either the interior angle is greater than the exterior, or a
triangle has angles whose sum is greater than two right angles.
16. A false conclusion depends on the first false assumption
on which it is based. Every syllogism depends either on its two
r
r,
r
r,premisses or on more than two. If a false conclusion depends on
two, one or both must be false; if on more-e.g. r on A and B,
and these on .1 and E, and Z. and H, respectively-one of the
premisses of the prosyllogisms must be false, and the conclusion
and its falsity must depend on it and its falsity.
65 8 38- b 3. To SE IlTJ wapo. TOUTO •.• a.vTicl)11a~v. A. makes two
points here about the incidence of the objection 'that is not the
cause of the falsity'. Suppose that someone wishes to maintain
the thesis No C is A, on the strength of the data No B is A, All C
is B. (I) He may use a reductio ad impossibile: 'If some C is A,
then since all C is B, some B will be A. But in fact no B is A.
Therefore Some C is A must be false, and No C is A true.' Now,
A. maintains, a casual hearer, hearing the conclusion drawn that
some B is A, and knowing that no B is A, will simply say 'one of
your premisses must have been wrong' (b l - 3). Only a second
disputant, interested in contradicting the thesis which was being
proved by the reductio, i.e. in maintaining that some C is A ("40-
b I ), will make the objection 'Some B is A is no doubt false, but
not because Some C is A is false'. (2) The first disputant may
infer ostensively: 'No B is A, All C is B, Therefore no C is A',
and this gives no scope for the objection ov 7Tapa. TOVTO, because
the ostensive proof, unlike the reductio, does not use as a premiss
the proposition Some C is A, which the second disputant is main-
taining in opposition to the first (b 3 -6).
b l - 3 . Tn EtS TO a.SuvaTOV ••• ~V Tn SE~KVOUan, Sc. (iTTOO~~H (cf.
62b29)·
8-4). a.va~pE9EiOTlS yap • • . auXXoy~allos. ~ (Na" means, as
usual in A. (cf. b l4 , 66 8 2, 8) the assumption, and 0 7TPO, TaVr1)"
<TtI,uOYW'fLOS- is 'the syllogism related to it', i.e. based on it.
15-16. 8WEP ELp"Ta~ •.. T OW~KOLS, i.e. in Soph. El. 167b21-36
(cf. 168b22-S, 181"31-S).
17-19. otov Et ••• a.SuvaTov. Heath thinks that this 'may point
to some genuine attempt to prove the incommensurability of the
diagonal by means of a real "infinite regression" of Zeno's type'
(Mathematics in Aristotle, 30-3). But it is equally possible that
the example A. takes is purely imaginary.
18-19. TOV Z~vwvos Xoyov ••• K~vELa8a~. For the argument
cf. Phys. 233"21-3, 239bS-240"18, 26384-1 I.
24-8. olov Et ••• uwo9Ea~v. If we assume that B is A, r is B,
and.1 is r, and if not only '.1 is A' but also '.1 is B' is false, the
cause of the falsity of '.1 is A' is to be found not in the falsity
of' B is A' but in that of T is B' or in that of '.1 is F'.COMMENTARY
66'5' TO l.,liUya.TOY, sc. that ~ is A.
5-6. bI'OLw~ lit ••• 8pous, i.e. if we had assumed that B is
~, and ~ is E, and E is Z, the impossible conclusion 'B is Z'
might remain.
7. TOUTOU, the assumption that B is A.
8-15. ~ TO I'TJ OYTO~ ••• liUELY; For';; introducing the answer
to a suggestion cf. An. Post. 99"2, Soph. El. 177b25, 178"31.
13-15. otOY Ta~ 'll'a.pa.).).';).ou~ ••• liUELy. As Heiberg (AM.
zur Gesch. d. M ath. W issenschaften. xviii. 18-19) remarks. the first
conditional clause refers to the proposition which appears as
Euc. i. 28 ('if a straight line falling on two straight lines makes the
exterior angle equal to the interior and opposite angle on the
same side of the straight line ... the straight lines will be parallel').
while the second refers to Euc. i. 27 ('if a straight line falling on
two straight lines makes the alternate angles equal. the straight
lines will be parallel'), since only in some pre-Euclidean proof
of this proposition, not in the proof of i. 28. can the sum of the
angles of a triangle have played a part. Cf. 65 8 4-7 n.
16-24. '0 liE -¥EUlitiS ••• -¥EGlio~. Chapter 18 continues the
treatment of the subject dealt with in the previous chapter, viz.
the importance of finding the premiss that is really responsible
for the falsity of a conclusion; if the premisses that immediately
precede the conclusion have themselves been derived from prior
premisses, at least one of the latter must be false.
19-20. i€ a.)."eWy ••• <1u).).0ywI'6~ refers back to 53bII-25.
CHAPTERS 19, 20
Devices to be used against an opponent in argument
66 2S_ To guard against having a point proved against us we
should, when the arguer sets forth his argument without stating
his conclusions. guard against admitting premisses containing the
same terms, because without a middle term syllogism is impossible.
How we ought to look out for the middle tenn is clear. because we
know what kind of conclusion can be proved in each figure. We
shall not be caught napping because we know how we are sustain-
ing our own side of the argument.
33- In attack we should try to conceal what in defence we should
guard against. (I) We should not immediately draw the conclu-
sions of our prosyllogisms. (2) We should ask the opponent to
admit not adjacent premisses but those that have no common
tenn. (3) If the syllogism has one middle tenn only. we should
start with it and thus escape the respondent's notice.
an. 17. 66"5-20. 66bI7
b 4 • Since we know what relations of the tenns make a
syllogism possible, it is also clear under what conditions refutation
is possible. If we say Yes to everything, or No to one question
and Yes to another, refutation is possible. For from such admis-
sions a syllogism can be made, and if its conclusion is opposite to
our thesis we shall have been refuted.
11. If we say No to everything, we cannot be refuted; for there
cannot be a syllogism with both premisses negative, and therefore
there cannot be a refutation; for if there is a refutation there must
be a syllogism, though the converse is not true. So too if we make
no universal admission.
66"25-32 may be compared with the treatment of the same
subject in Top. viii. 4. and 66"33-b3 with Top. viii. 1-3.
66-27-8. hruSTprEp tUI1EV ••• YLV£T(U, cf. 40b3Q-4I"20.
2CT"32. W!; SE S£i ... Myov. To take two examples given by
Pacius, (I) if the respondent is defending a negative thesis, he need
not hesitate to admit two propositions which have the same
predicate, since the second figure cannot prove an affinnative
conclusion. (2) If he is defending a particular negative thesis
(Some 5 is not P), he should decline to admit propositions of the
fonn All M is P, All 5 is M, since these will involve the conclusion
All 5 is P. We shall not be caught napping because we know
the lines on which we are conducting our defence ('IT(V, inrixop.t"v
TbV '\6yov). imixwp.£v, 'how we are to defend our thesis', would
perhaps be more natural, and would be an easy emendation.
37. al1£ua. here has the unusual but quite proper sense 'pro-
positions that have no middle tenn in common'. This reading,
as Waitz observes, is supported by P.'s phrase cWvvapn7Tov, t"lva,
Ta., 1TpOTaau, (460.
28).
bl - 3 . Ka.V SL' (VD!; • • • n1TOKPWOI1EVOV.
A. has in mind an
argument in the first figure. If we want to make the argument
as clear as possible we shall either begin with the major and say
'A belongs to B, B belongs to C, Therefore A belongs to C', or
with the minor and say 'C is B, B is A, Therefore C is A '. There-
fore if we want to make the argument as obscure as possible we
shall avoid these methods of statement and say either' B belongs
to C, A belongs to B, Therefore A belongs to C', or 'B is A,
C is B, Therefore C is A'.
4-17. 'E1T£t S' ... uuAAoywl1ou. Chapter 20 is really continuous
with that which precedes. A. returns to the subject dealt with
in the first paragraph of the latter, viz. how to avoid making
admissions that will enable an opponent to refute our thesis. An47 0
COMMENTARY
elenchus is a syllogism proving the contradictory of a thesis that
has been maintained (b II ). Therefore if the maintainer of the
thesis makes no affirmative admission, or if he makes no universal
admission, he cannot be refuted, because a syllogism must have
at least one affirmative and one universal premiss, as was main-
tained in i. 24.
66b~IO. Et TO KEillEVOV • . • aUIl1TEpa.alla.n. £vavT{ov is used
here not in the strict sense of 'contrary', but in the wider sense
of 'opposite'. A thesis is refuted by a syllogism which proves
either its contrary or its contradictory.
12-13. ou ya.p .•• OVTI.IV, cf. 41b6.
14-15. Et Il£v ya.p ••• EAEYXOV. The precise point of this is
not clear. A. may only mean that every refutation is a syllogism
but not vice versa, since a refutation presupposes the maintenance
of a thesis by an opponent. Or he may mean that there is not
always, answering to a syllogism in a certain figure, a refutation
in the same figure, since, while the second figure can prove a
negative, it cannot prove an affirmative, and, while the third
figure can prove a particular proposition, it cannot prove the
opposite universal proposition.
15-17. waa.llTl&l5 SE ••• auAAoy~allou, cf. 4Ib6-z7.
CHAPTER 21
How ignorance of a conclusion can coexist with knowledge of the
premisses
66 b I8. As we may err in the setting out of our terms, so may
we in our thought about them. (I) If the same predicate belongs
immediately to more than one subject, we may know it belongs
to one and think it does not belong to the other. Let both Band
C be A, and D be both Band C. If one thinks that all B is A and
all D is B, and that no C is A and all D is C, one will both know
and fail to know that D is A.
26. (2) If A belongs to B, B to C, and C to D, and someone
supposes that all B is A and no C is A, he will both know that all
D is A and think it is not.
30. Does he not claim, then, in case (2) that what he knows he
does not think? He knows in a sense that A belongs to C through
the middle term B, knowing the particular fact by virtue of his
universal knowledge, so that what in a sense he knows, he main-
tains that he does not even think; which is impossible.
34. In case (I) he cannot think that all B is A and no C is A,11.
66 b g-17
471
and that all D is B and all D is C. To do so, he must be having
wholly or partly contrary major premisses. For if he supposes
that everything that is B is A, and knows that D is B, he knows
that D is A. And again if he thinks that nothing that is C is A,
he thinks that no member of a class (C), one member of which
(D) is B, is A. And to think that everything that is 1] has a
certain attribute, and that a particular thing that is B has it not,
is wholly or partly self-contrary.
67-5' We cannot think thus, but we may think one premiss
about each of the middle terms, or one about one and both about
the other, e.g. that all B is A and all D is B, and that no C is A.
8. Then our error is like that which arises about particular
things in the following case. If all B is A and all C is B, all C will
be A. If then one knows that all that is B i,: A, one knows that
C is A. But one may not know that C exists, e.g. if A is 'having
angles equal to two right angles', B triangle, and C a sensible
triangle. If one knows that every triangle has angles equal to two
right angles but does not think that C exists, one will both know
and not know the same thing. For 'knowing that every triangle
has this property' is ambiguous; it may mean having the universal
knowledge, or having knowledge about each particular instance.
It is in the first sense that one knows that C has the property,
and in the second sense that one fails to know it, so that one is
not in two contrary states of mind about C.
21. This is like the doctrine of the M eno that learning is
recollecting. We do not know the particular fact beforehand; we
acquire the knowledge at the same moment as we are led on to the
conclusion, and this is like an act of recognition. There are things
we know instantaneously, e.g. we know that a figure has angles
equal to two right angles, once we know it is a triangle.
7.7. By universal knowledge we apprehend the particulars,
without knowing them by the kind of knowledge appropriate to
them, so that we may be mistaken about them, but not with an
error contrary to our knowledge; we have the universal knowledge,
we err as regards the particular knowledge.
30. So too in case (1). Our error with regard to the middle
term C is not contrary to our knowledge in respect of the syllo-
gism; nor is our thought about the two middle terms self-contrary.
33. Indeed, there is nothing to prevent a man's knowing that
all B is A and all C is B, and yet thinking that C is not A (e.g.
knowing that every mule is barren and that this is a mule, and
thinking that this animal is pregnant) ; for he does not know that
C is A unless he surveys the two premisses together.
20.472
COMMENJ ARY
37. A fortiori a man may err if he knows the major premiss
and not the minor, which is the position when our knowledge is
merely general. We know no sensible thing when it has passed
out of vur perception, except in the sense that we have the
universal knowledge and possess the knowledge appropriate to the
particular, without exercising it.
b3. For 'knowing' has three senses-universal, particular, and
actualized-and there are three corresponding kinds of error.
Thus there is nothing to prevent our knowing and being in error
about the same thing, only not so that one is contrary to the
other. This is what happens where one knows both premisses and
has not studied them before. When a man thinks the mule is
pregnant he has not the actual knowledge that it is barren, nor is
his error contrary to the knowledge he has; for the error contrary
to the universal knowledge would be a belief reached by syllogism.
12. A man who thinks (a) that to be good is to be evil is think-
ing (b) that the same thing is being good and being evil. Let
being good be A, being evil B, being good C. He who thinks that
B is the same as C will think that C is B and B is A, and therefore
also that C is A. For just as, if B had been true of that of which C
is true, and A true of that of which B is true, A would have been
true of that of which C is true, so too one who believed the first two
of these things would believe the third. Or again, just as, if C
is the same as B, and B as A, C is the same as A, so too with the
believing of these propositions.
22. Thus a man must be thinking (b) if he is thinking (a). But
presumably the premiss, that a man can think being good to be
being evil, is false; a man can only think that per accide'ns (as
may happen in many ways). But the question demands better
treatment.
A.'s object in this chapter is to discuss various cases in which
it seems at first sight as if a man were at the same time knowing
a certain proposition and thinking its opposite-which would be
a breach of the law of contradiction, since he would then be
characterized by opposite conditions at the same time. In every
case, A, maintains, he is not knowing that B is A and thinking
that B is not A, in such a way that the knowing is opposite to and
incompatible with the thinking.
Maier (ii. a. 434 n. 3) may be right in considering ch. 21 a later
addition, especially in view of the close parallelism between
67"8-26 and An. Post. 71"17-30. Certainly the chapter has no
close connexion with what precedes or with what follows.11. 2I-2
473
A. considers first (66h20-6) a case in which an attribute A
belongs directly both to B and to C, and both B and C belong
to all D. Then if some one knows (b2?'; in h24 A. says 'thinks',
and the chapter is somewhat marred by a failure to distinguish
knowledge from true opinion) that all B is A and all D is B,
and thinks that no C is A and all D is C, he will be both knowing
and failing to know an identical subject D in respect to its
relation to an identical attribute A. The question is whether
this is possible.
A. turns next (b26-34) to a case in which not two syllogisms
but one sorites is involved. If all B is A, and all C is B, and all
D is C, and one judged that all B is A but also that no C is A,
one would at the same time know (A. again fails to distinguish
knowledge from true opinion) that all D is A (because all B is A,
all C is B, and all D is C) and judge that no D is A (because one
would be judging that no C is A and that all D is C). The intro-
duction of D here is unnecessary (it is probably due to the presence
of a fourth term D in the case previously considered) ; the question
is whether one can at the same time judge that B is A and C is B,
and that C is not A. Is not one who claims that he can do this
claiming that he can know what he does not even think? Cer-
tainly he knows in a sense that C is A, because this is involved
in the knowledge that all B is A and all C is B. But it is plainly
impossible that one should know what he does not even judge
to be true.
A. now (b34-67a8) returns to the first case. One cannot, he
says, at the same time judge that all B is A and all D is B, and
that no C is A and all D is C. For then our major premisses must
be 'contrary absolutely or in part', i.e. 'contrary or contradictory'
(cf. 0),,1) .p€vfn7~, €7rl n .p€v87}~ in 54al-4). A. does not stop to ask
which they are. In fact the major premisses (All B is A, No C
is A) are only (by implication) contradictory, since No C is A,
coupled with All D is C and All D is B, implies only that some B
is not A, not that no B is A.
But, A. continues (67°5-8), while we cannot be believing all four
premisses, we may be believing one premiss from each pair,
or even both from one pair, and one from the other; e.g. we may
be judging that all B is A and all D is B, and that no C is A.
So long as we do not also judge that all D is C and therefore that
no D is A, no difficulty arises.
The error here, says A. ("8-21), is like that which arises when
we know a major premiss All B is A, but through failure to
recognize that a particular thing C is B, fail to recognize that it is474
COMMENTARY
A ; i.e. the type of error already referred to in 66 b26-34. In both
cases the thinker grasps a major premiss but through ignorance
of the appropriate minor fails to draw the appropriate conclusion.
If all C is in fact B, in knowing that all B is A one knows by
implication that all C is A, but one need not know it explicitly,
and therefore the knowledge that all B is A can coexist with
ignorance of Cs being A, and even with the belief that no C is
A, without involving us in admitting that a man may be in two
opposite states of mind at once.
This reminds A. ("21-3°) of a famous argument on the subject
of implicit knowledge, viz. the argument in the M eno (81 b--S6 b)
where a boy who does not know geometry is led to see the truth
of a geometrical proposition as involved in certain simple facts
which he does know, and Plato concludes that learning is merely
remembering something known in a previous existence. A. does
not draw Plato's conclusion; no previous actual knowledge, he
says, but only implicit knowledge, is required; that being given,
mere confrontation with a particular case enables us to draw the
particular conclusion.
A. now recurs ("30-3) to the case stated in 66 b20-6, where two
terms are in fact connected independently by means of two middle
tem1S. Here, he says, no more than in the case where only one
middle tenn is involved, is the error into which we may fall
contrary to or incompatible with the knowledge we possess. The
erroneous belief that no C is A (~ Ka-ra -ro Jdaov d1Ta-r~) is not
incompatible with knowledge of the syllogism All B is A, All
D is B, Therefore all D is A (°31-2); nor a fortiori is belief that
no C is A incompatible with knowledge that all B is A ("32-3).
A. now (°33-7) takes a further step. Hitherto (66b34-67°S) he
has maintained that we cannot at the same time judge that all
B is A and all D is B, and that no C is A and all D is C, because
that would involve us in thinking both that all D is A and that
no D is A. But, he now points out, it is quite possible to know
both premisses of a syllogism and believe the opposite of the con-
clusion, if only we fail to see the premisses in their connexion;
and a fortiori possible to believe the opposite of the conclusion if
we only know one of the premisses (a37--{;)).
A. has already distinguished between ~ Ka86>..ov £1Ttcrn7f4YJ,
knowledge of a universal truth, and ~ Ka8' £Kaa-rov (°18, 20), ~ -rwv
Ka-ra f4'pO, ('23), or ~ olKEla ('27), knowledge of the corresponding
particular truths. He now adds a third kind, ~ -rip £V£PYEI.V. This
further distinction is to be explained by the reference in a39-bl
to the case in which we have already had perceptual awareness11. 21. 66br8-r9
475
of a particular but it has passed out of our ken. Then, says A.,
we have ~ olKda E'TI"Lcrn}J-LTJ as well as ~ Ka86>.ov, but not ~ -rtfJ
lv£py£Lv; i.e. we have a potential awareness that the particular
thing has the attribute in question, but not actual awareness of
this; that comes only when perception or memory confronts us
anew with a particular instance. Thus we may know that all
mules are barren, and even have known this to be true of certain
particular mules, and yet may suppose (as a result of incorrect
observation) a particular mule to be in foal. Such a belief (bIe-Il)
is not contrary to and incompatible with the knowledge we have.
Contrariety would arise only if we had a syllogism leading to the
belief that this mule is in foal. A., however, expresses himself
loosely; for belief in such a syllogism would be incompatible not
with belief in the major premiss (~ Ka86>.ov, bIll of the true
syllogism but with belief in that whole syllogism. Belief in both
the true and the false syllogism would be the position already
described in 66 b 24--8 as impossible.
From considering whether two opposite judgements can be
made at the same time by the same person, A. passes (67bI2-26)
to consider whether a self-contradictory judgement, such as
'goodness is badness', can be made. He reduces the second case
to the first, by pointing out that if anyone judges that goodness
is the same as badness, he is judging both that goodness is badness
and that badness is goodness, and therefore, by a syllogism in
which the minor term is identical with the major, that goodness is
goodness, and thus being himself in incompatible states. The fact
is, he points out, that no one can judge that goodness is badness,
El J-L?J Ka-ra. CTIIJ-Lf3Ef3TJK6, (b23-S). By this A. must mean, if he is
speaking strictly, that it is possible to judge, not that that which
is in itself good may per accidens be bad, but that that which is
in itself goodness may in a certain connexion be badness. But
whether this is really possible, he adds, is a question which needs
further consideration.
The upshot of the whole matter is that in neither of the cases
stated in 66 b20--6, 26-34 can there be such a coexistence of error
with knowledge, or of false with true opinion, as would involve
our being in precisely contrary and incompatible states of mind
with regard to one and the same proposition.
66bI8-19. Ka9a.1I"Ep EV T11 9Ean ••• ci1TIlTw!'E9a. The reference
is to errors in reasoning due to not formulating our syllogism
correctly-the errors discussed in i. 32-44; cf. in particular
47bIS-I7 a:rra-ra.u8aL .•. 7Tapa. -r?Jv oJ-Lot07TJ-ra -rii, -rwv opwv 8£UEW,
(where confusion about the quantity of the terms is in question)COMMENTARY
and similar phrases ib. 40-4802, 49&27-8, bIo-II , 50&II-I3. Error
EV Tfj BEan TWV opwv is in general that which arises because the
propositions we use in argument cannot be formulated in one of
the valid moods of syllogism. The kind of error A. is now to
examine is rather loosely described as KaTa -rTJv &roA7)rf'v, It is
error not due to incorrect reasoning, but to belief in a false pro-
position. The general problem is, in what conditions belief in a
false proposition can coexist with knowledge of true premisses
which entail its falsity, without involving the thinker's being
in two opposites states at once.
26. Ta. EK T11S aUTt;S UU<7TOlXlaS, i.e. terms related as super-
ordinates and subordinates.
32. TTI Ka8b)'ou, sc. E7T£aT~P.TI' cf. 67818.
try°12. a.YVOELV TO r OT! ~<7TLV. {moM.{3o, . . . av TL~ IL~ Elva,
TO r (314-15) is used as if it expressed the same situation, and EaV
ElSwP.EV OT' Tplywvov (325) as if it expressed the opposite. Thus A.
does not distinguish between (I) not knowing that the particular
figure exists, (2) thinking it does not exist, (3) not knowing that
the middle term is predicable of it. He fails to distinguish two
situations, (a) that in which the particular figure in question is
not being perceived, and we have no opinion about it (expressed
by (I)), (b) that in which it is being perceived but not recognized
to be a triangle (expressed by (3)). The loose expression (2) is
due to A.'s having called the minor term alaB7)Tov Tptywvov
instead of alaB7)TOV axfip.a. Thus thinking that the particular
figure is not a triangle (one variety of situation (b)) comes to be
expressed as 'thinking that the particular sensible triangle does
not exist'.
17. Suo op8aLS, sc. €Xn Ta~ ywvta~ Zaa~.
23. af-La TTI E1TaywYTI, 'simultaneously with our being led on
to the conclusion'. For this sense cf. An. Post. 71"20 OTL S~ -rOOf
TO EV TijJ ~P.'KVKAr.q, Tptywvov EaT'V, ap.a ETTayop.Evo~ lyvwp,aEv (cf.
Top. IIIb38). There is no reference to induction; the reasoning
involved is deductive.
27. TTI ••• Ka8b)'ou, sc. br'aT~p.TI' cf. 66b32 n.
29. a.1TaT(1u8aL SE Tt,V KaTa. foLEpOS. The MSS. have -rfj, but
T~V must be right-'fall into the particular error'. Ct. An. Post.
74"6 a7TaTwp.EBa SE Tal.!'77jv -rTJv a7TaT~v.
b2 . Tii> Ka86)'ou, sc. E7T[aTaaBa" cf. 66b32 n.
23. TOUTO, i.e. that a man can think the same thing to be the
essence of good and the essence of evil. TO TTPWTOV, i.e. that a man
can think the essence of good to be the essence of evil (b12).477
CHAPTER 22
Rules for the use of convertible terms and of alternative terms, and for
the comparison of desirable and undesirable objects
67b27. (A) (a) When the extreme terms are convertible, the
middle term must be convertible with each of them. For if A is
true of C because B is A and C is B, then if All C is A is convert-
ible, (a) All C is B, All A is C, and therefore all A is B, and (P)
All A is C, All B is A, and therefore all B is C.
32. (b) If no C is A because no B is A and all C is B, then (a)
if No B is A is convertible, all C is B, no A is B, and therefore
no A is C; (P) if All C is B is convertible, No B is A is convertible;
(y) if No C is A, as well as All C is B, is convertible, No B is A is
convertible. This is the only one of the three conversions which
starts by assuming the converse of the conclusion, as in the case
of the affirmative syllogism.
68a3. (B) (a) If A and B are convertible, and so are C and D,
and everything must be either A or C, everything must be either
B or D. For since what is A is B and what is C is D, and every-
thing is either A or C and not both, everything must be either B
or D and not both; two syllogisms are combined in the proof.
11. (b) If everything is either A or B, and either C or D, and
not both, then if A and C are convertible, so are Band D. For if
any D is not B, it must be A, and therefore c. Therefore it must
be both C and D; which is impossible. E.g. if 'ungenerated' and
'imperishable' are convertible, so are 'generated' and 'perishable'.
16. (C) (a) When all B is A, and all C is A, and nothing else
is A, and all C is B, A and B must be convertible; for since A is
predicated only of B and C, and B is predicated both of itself and
of C, B is predicable of everything that is A, except A itself.
21. (b) When all C is A and is B, and C is convertible with B,
all B must be A, because all C is A and all B is C.
25. (1) When of two opposites A is more desirable than B, and
D similarly is more desirable than C, then if A +C is more desir-
able than B+D, A is more desirable than D. For A is just as
much to be desired as B is to be avoided ; and C is just as much to
be avoided as D is to be desired. If then (a) A and D were equally
to be desired, Band C would be equally to be avoided. And there-
fore A +C would be just as much to be desired as B+D. Since
they are more to be desired than B+D, A is not just as much to
be desired as D.
33. But if (b) D were more desirable than A, B would be less toCOMMENTARY
be avoided than C, the less to be avoided being the opposite of the
less to be desired. But a greater good+a lesser evil are more
desirable than a lesser good + a greater evil; therefore B -+- D would
be more desirable than A +C. But it is not. Therefore A is more
desirable than D, and C less to be avoided than B.
39- If then every lover in virtue of his love would prefer that
his beloved should be willing to grant a favour (A) and yet not
grant it (C), rather than that he should grant it (D) and yet not
be willing to grant it (B), A is preferable to D. In love, therefore,
to receive affection is preferable to being granted sexual inter-
course, and the former rather than the latter is the object of love.
And if it is the object of love, it is its end. Therefore sexual
intercourse is either not an end or an end only with a view to
receiving affection. And so with all other desires and arts.
The first part of this chapter (67b27-{i8a3) discusses a question
similar to that discussed in chs. 5-7, viz. reciprocal proof. But the
questions are not the same. In those chapters A. was discussing
the possibility of proving one of the premisses of an original
syllogism by assuming the conclusion and the converse of the
other premiss; and original syllogisms in all three figures were
considered. Here he discusses the possibility of proving the
converse of one of the propositions of an original syllogism by
assuming a second and the converse of the third, or the converses
of both the others; and only original syllogisms in the first figure
are considered.
The rest of the chapter adds a series of detached rules dealing
with relations of equivalence, alternativeness, predicability, or
preferability, between terms. The last section (68a25-b7) is dia-
lectical in nature and closely resembles the discussion in Top.
iii. 1-4.
6Jb 27 -8. ~OTC!.V S' ... a.~cpw. This applies only to syllogisms in
Barbara (b28-32). A. says (1Tt TOU JL~ Ima.PXHII waav-rw, (b32 ), but
this means only that conversion is possible also with syllogisms
in Celarent; only in one of the three cases discussed in b34 -{i8a r
does the conversion assume the converse of the conclusion, as
in the case of Barbara.
32-6833. Kal E1T1 TOU ~" U1Ta.PXELV •• _ (J"uAAoYL(J"~OU. If we
start as A. does with a syllogism of the form NoB is A, All C is
B, Therefore no C is A, only three conversions are possible: (r)
All C is B, No A is B, Therefore no A is C; (2) All B is C, No C is
A, Therefore no A is B; (3) All B is C, No A is C, Therefore no
A is B. b34-{i refers to the first of these conversions. b37 -8 is479
difficult. The vulgate reading, Kat cl T~ B TO r aVTLUTp'q,n, Kat
T~ A aVTLCTTpEq,n, gives the invalid inference All B is C, No B is
A, Therefore no A is C. We must read either (a) Kat lit T~ B TO
r aVTLUTp'q,n, Kat TO A aVTLUTp'q,n (or av-nUTp,.pn) , or (b) Kat d
TO B T~ r aVTLUTp'q,n, Kat T~ A aVTLUTp'q,n (or aVTLUTpt.pH), either
of which readings gives the valid inference (2) above. a38--68'r is
also difficult. The vulgate reading Kat lit TO r ,"poe; TO A aVTL-
CTTp'q,n gives the invalid inference All C is B, No A is C, Therefore
no A is B. In elucidating this conversion, A. explicitly assumes
not All C is B, but its converse (c;, ya.p TO B, TO r). The passage
is cured by inserting Ka{ in b 3 8; we then get the valid inference
(3) above. The reading thus obtained shows that TO r must be
the subject also of the protasis in b 37 , and confirms reading (a)
above against reading ·(b).
On this interpretation, the statement in 68"1-3 must be taken
to mean that only the last of the three conversions starts by
converting the conclusion, as both the conversions of the affirma-
tive syllogism did, in a28-32.
68a3-16. n6.~w Et ••• ciSUvaTov. A. here states two rules. If
we describe as alternatives two terms one or other of which must
be true of everything, and both of which cannot be true of any-
thing, the two rules are as follows: (I) If A and B are convertible,
and C and D are convertible, then if A and C are alternative,
Band D are alternative ("3-8); (2) If A and B are alternative,
and C and D are alternative, then if A and C are convertible,
'B and D are convertible (all-I6). A. has varied his symbols by
making B and C change places. If we adopt a single symbolism
for both rules, we may formulate them thus: If A and Bare
convertible, and A and C are alternative, then (la) if C and D
are convertible, Band D are alternative; (2a) if Band Dare
alternative, C and D are convertible; so that the second rule is
the con verse of the first.
Between the two rules the MSS. place an example ('8-Il):
If the ungenerated is imperishable and vice versa, the generated
must be perishable and vice versa. But, as P. saw (469. 14-17),
this illustrates rule (2), not rule (I), for the argument is plainly
this: <Since 'generated' and 'ungenerated' are alternatives, and
so are 'perishable' and 'imperishable'), if 'ungenerated' and
'imperishable' are convertible, so are 'generated' and 'perishable'.
Pacius has the example in its right place, after the second rule,
and since he makes no comment on this we may assume that it
stood so in the text he used.
It remains doubtful whether ouo ya.p av'\'\0YLUP.Ot UVyK€LVTaLCOMMENTARY
(al0) should come after aj-La in a8, as Pacius takes it, or after
d,8olla'TolI in 816, as P. (469. 18-470. 3) takes it. On the first hypo-
thesis the two arguments naturally suggested hy "6-8 are (1)
Since all A is E and all C is D, and everything is A or C, every-
thing is E or D, (z) Since all A is E and all C is D, and nothing
is both A and C, nothing is both E and D. But the second of
these arguments is clearly a bad one, and the arguments intended
must rather be Since A is convertible with E, and C with D,
(1) What must be A or C must be E or D, Everything must be
A or C, Therefore everything must be E or D, (z) What cannot
be both A and C cannot be both E and D, Nothing can be both
A and C, Therefore nothing can be both E and D.
On the second hypothesis the two arguments are presumably
those stated in "14-15: (1) Since A and E are alternative, any
D that is not E must be A, (z) Since A and C are convertible,
any D that is A must be C-which it cannot be, since C and D
are alternative; thus all D must be E.
On the whole it seems best to place the words where Pacius
places them, and adopt the second interpretation suggested on
that hypothesis.
16-21. "OTav 8E • : • A. The situation contemplated here is that
in which E is the only existing species of a genus A which is
notionally wider than E, and C is similarly the only subspecies
of the species E. Then, though A is predicable of C as well as
of E, it is not wider than but coextensive with E, and E will be
predicable of everything of which A is predicable, except A
itself ("Zo-l). It is not predicable of A, because a species is not
predicable of its genus (Cat. zbZ1 ). This is not because a genus is
wider than any of its species; for in the present case it is not
wider. It is because 'TO €l8oS' 'TOV rlllovS' j-Lfi.>J.OII o?Jcr[a (Cat. zb zz ), so
that in predicating the species of the genus you would be reversing
the natural order of predication, as you are when you say 'this
white thing is a log' instead of 'this log is white'. The latter is
true predication, the former predication only in a qualified sense
(An. Post. 83"1-18).
21-5. 1Tc1AIV ~lTav • . • B. This section states a point which
is very simple in itself, but interesting because it deals with
the precise situation that arises in the inductive syllogism
(b 15- Z4). The point is that when all C is A, and all C is E, and
C is convertible with E, then all E is A.
39-41. et 8" ... 1'1 TO xapL~€a9aL. With EAOL'TO we must 'under-
stand' j-Lfi.>J.OI'.
b6-J. teal ya.p ••• OUTlI15, i.e. in any system of desires, and inparticular in the pursuit of any art, there is a supreme object of
desire to which the other objects pf desire are related as means
to end. Ct. Eth. Nic. i. 1.
CHAPTER 23
Induction
68 b 8. The relations of terms in respect of convertibility and of
preferability are now clear. We next proceed to show that not
only dialectical and demonstrative arguments proceed by way of
the three figures, but also rhetorical arguments and indeed any
attempt to produce conviction. For all conviction is produced
either by syllogism or by induction.
IS. Induction, i.e. the syllogism arising from induction, con-
sists of proving the major term of the middle term by means of
the minor. Let A be 'long-lived', B 'gall-less', C the particular
long-lived animals (e.g. man, the horse, the mule). Then all C is
A, and all C is B, therefore if C is convertible with B, all B must
be A, as we have proved before. C must be the sum of all the
particulars; for induction requires that.
30. Such a syllogism establishe!? the unmediable premiss; for
where there is a middle term between two terms, syllogism con-
nects them by means of the middle term; where there is not, it
connects them by induction. Induction is in a sense opposed to
syllogism; the latter connects major with minor by means of the
middle term, the former connects major with middle by means of
the minor. Syllogism by way of the middle term is prior and more
intelligible by nature, syllogism by induction is more obvious to us.
In considering the origin of the use of brayW'Y'i as a technical
term, we must take account of the various passages in which
A. uses bra.ynv with a logical significance. We must note (I) a
group of passages in which £1Ta.ynv is used in the passive with a
personal subject. In A n. Post. 7Ia20 we have ~'n S£ 'T6S€ 'TO £11
'TtjJ ~P.LICV0.'-o/ 'TptywII611 £(nw, ap.a £1Tay6p.€1I0, £yvwPw€v. That
£1Tay6p.£lIo, is passive is indicated by the occurrence in the same
passage (ib. 24) of the words 1Tptll S' £1Tax(JijvaL ~ Aa{3€LV UVAAOYLUP.OIl
'Tp61TOII p.lv TLva LUW, </>a'TlolI £1TtO"'Tau(JaL, 'Tp61TOII S' IDoII OU. Again
in An. Post. 8IbS we have £1Tax8ijvaL o£ p.~ £X0v-ra, aLu87]uw
dSvva'ToII.
P. interprets £1Tay6p.€1I0, in 71"21 as 1Tpo{1{3illwv athtjJ Ka'Ta 'T~V
aL{1(J7]ULV (17. 12, cf. 18. 13). But (a) in the other two passages
£1Ta.Y€{1(JaL clearly refers to an inferential process, and (b) in the
4985
I
iCOMMENTARY
usage of £tray£tll in other authors it never seems to mean 'to lead
up to, to confront with, facts', while if we take £1Tayw8u~ to mean
'to be led on to a conclusion', it plainly falls under sense I. 10
recognized by L. and 5., 'in instruction or argument, lead on',
and has affinities with sense I. 3, 'lead on by persuasion, influence'.
(2) With this use is connected the use of £1Tay£t1l without an
object-A n. Post. 91 hIS WU1T(£P ouo' 0 £1TaywlI a1Too.dKllVULII (cf.
ib. 33). 92"37 w, 0 £1TaywlI o~a TOlII Ku8' £KUa-rU O~'\WII OIlTWII, Top.
108 b l I OU yap p40LC)1I £a-rLII £1Tay£t1l j.L~ £lOOTa, Ta OJ.LOLU, 15684 £1Tayov-ru
a1To TOlII Ku8' £KUa-rOIl £1TL TO Ku8o'\ov, 157 8 34 £1TaY0ll"TO, £1TL 1TO'\'\OlII,
Soph. El., 174"34 £1TUYUYOIlTa TO Ku8o'\ov 1To'\'\aKL, OUK £pWT1]r£OIl
a'\'\'
O£OOj.L£II<p XfY'7a-r£OIl, Rhet. 13S6hS allaYK7J <~> UV'\'\OYL,Oj.L£1I01l
~ £1TaYOIlTa O£LKIIVIIUL OTLOVII. The passages cited under (I) definitely
w,
envisage two persons, of whom one leads the other on to a con-
clusion. In the passages cited under (2) there is no definite
reference to a second person, but there is an implicit reference
to a background of persons to be convinced. This usage is related
to the first as £1Tay£t1l in the sense of 'march against' is related to
£1TayulI in the sense of 'lead on (trans.) against' (both found under
L. and S. I. zb.
(3) In one passage we find £1TayulI TO Ku8o,\ov--Top. IOSblO
rfi Ku8'
£KUa-rU £1TL TOlII OJ.LO[WII £1TUYWyfj TO Ku8o'\ov a~LOVj.L£1I £1Tay£tll.
(In Soph. El. 174 3 34, cited under (2), it is possible that TO Ku8o'\ov
should be taken as governed by £1TUYUYOIlTa as well as by £pwT1]-
T£OIl.) This should probably be regarded as a development from
usage (z)-from 'infer (abs.) inductively' to 'infer the universal
inductively'.
(4) In Top. IS9"IS we find £1TUyUy£tll Tall '\0YOIl, a usage which
plainly has affinities with usages (I), (z), (3).
(5) There is a usage of £1Tayw8uL (middle) which has often been
thought to be the origin of the technical meaning of £1TUyWyt) ,
viz. its usage in the sense of citing, adducing, with such. words
as j.LapTvpu" j.LUPTVPLU, £LKOIlU, (L. and S. 11. 3). A. has £1Tay£u8u~
1TOL'7T~1I (M et. 995"S), and £1TUy0j.L£1I0L KUL Tall ·OJ.L'7POIl (Part. An.
673"15), but apparently never uses the word of the citation of
individual examples to prove a general conclusion. There is,
however, a trace of this usage in A.'s use of £1TUKTLKO" £1TUKTLKOl,.
In A n. Post. nb 33 £1TUKTLK~ 1TPOTUUL, and in Phys. 2IO b S £1TUKTLKOl>
UK01TOVULII the reference is to the examination of individual
instances rather than to the drawing of a universal conclusion.
The same may be true of the famous reference to Socrates as
having introduced £1TUKTLKOL ,\OYOL (Met. I07Sb28); for in fact
Socrates adduced individual examples much more often to refute11.23
a general proposition than he used them inductively, to establish
such a proposition.
Of the passages in which the word £1TaywyrJ itself occurs, many
give no defmite clue to the precise shade of meaning intended;
but many do give such a clue. In most passages £1TaywyrJ clearly
means not the citation of individual instances but the advance
from them to a universal; and this has affinities with senses (I),
(2), (3), (4) of £7TclYHII, not with sense (s). E.g. Top. 105"13 £7Taywyi]
~ am) TWII Ka8' £Kacrra £7T' TO Ka86Aov £</>oSoS', An. Post. 8 1 b l ~
£7Taywyi] £K TWII KaTa p.£po" A n. Pr. 68blS £7TaywY'l £crr, ... TO S,d
TOV £TEPOV 8clT£POII aKpoII T0 p.£uo/ uv).).oyLuau8a,. But occasionally
£7TayWJn7 seems to mean 'adducing of instances' (corresponding to
sense (5) of £7TclyuII)-Top. l08 b lO Tfi Ka8' £Kacrra £7T' TWII OP.OLWII
£7Taywyfj TO Ka86Aov atlovp.£II £7TclYHII, Soph. El. 174336 S,a ,"," Tij,
£7Taywyij, p.IIldall, Cat. 13 b 37 SijAoII Tfj Ka8' £Kacrroll £7Taywyfj, Met.
1048"35 SijAoII S' £7T' TWII Ka8' £Kacrra Tfi £7Taywyfj 0 {3ovAop.£8a MYHII.
(The use of £7TaywyrJ in 67323 corresponds exactly to that of
£7Tayop.£IIo>in A n. Post. 71"21. Here, as in Top. lllb 38, a deductive,
not an inductive, process is referred to.)
The first of these two usages of £7TaywyrJ has its parallels in
other authors (L. and S. sense S a), and has an affinity with the
use of the word in the sense of 'allurement, enticement' (L. and S.
sense 4 a). The second usage seems not to occur in other authors.
Plato's usage of £7TclYUII throws no great light on that of A.
The most relevant passages are Polit. 278 a £7TclYHII aVTov, £7T'
Ta P.~7TW y''Y''wuK6p.£IIa (usage (I) of £7TclYUII) , and Hipp. Maj.
289 b, Laws 823 a, Rep. 364 c, Prot. 347 e, Lys. 215 c (usage 5).
£7TaywY'l occurs in Plato only in the sense of 'incantation' (Rep.
364 c, Laws 933 d), which is akin to usage (I) of £7TclYHII rather than
to usage (5).
It is by a conflation of these two ideas, that of an advance in
thought (without any necessary implication that it is an advance
from particular to universal) and that of an adducing of particular
instances (without any necessary implication of the drawing of
a positive conclusion), that the technical sense of £7TaywyrJ as
used by A. was developed. A.'s choice of a word whose main
meaning is just 'leading on', as his technical name for induction, is
probably influenced by his view that induction is m8allwT£poII
than deduction (Top. 105"16).
A. refers rather loosely in the first paragraph to three kinds of
argument-demonstrative and dialectical argument on the one
hand, rhetorical on the other. His view of the relations between
the three would, if he were writing more carefully, be stated asCOMMENTARY
follows: The object of demonstration is to reach knowledge. or
science; and to this end (a) its premisses must be known. and (b)
its procedure must be strictly convincing; and this implies that
it must be in one of the three figures of syllogism-preferably in
the first. which alone is for A. self-evidencing. The object of
dialectic and of rhetoric alike is to produce conviction (1Tt<Tns-);
and therefore (a) their premisses need not be true; it is enough if
they are ;Jl8o~0,. likely to win acceptance; and (b) their method
need not be the strict syllogistic one. Many of their arguments
are quite regular syllogistic ones. formally just like those used in
demonstration. But many others are in forms that are likely to
produce conviction. but can be logically justified only if they can
be reduced to syllogistic form; and it is this that A. proposes to
do in chs. 23-7. Thus these chapters form a natural appendix to
the treatment of syllogism in 1. I-I!. 22.
The distinction between dialectical and rhetorical arguments
is logically unimportant. They are of the same logical type; but
when used in ordinary conversation or the debates of the schools
A. calls them dialectical. when used in set speeches he calls them
rhetorical.
Conviction. says A. (1)13-14). is always produced either by
syllogism or by induction; and this statement is echoed in many
other passages. But besides these there are processes akin to
syllogism (t:llCos- and C1T)lu'ioJl. ch. 27) or to induction (1Tap&ilfl'Yl-'a.
ch. 24). And with them he discusses reduction (ch. 2S) and
objection (ch. 26). which are less directly connected with his theme
---discusses them because he wants to refer to all the kinds of
argument known to him.
Induction and 'the syllogism from induction' (i.e. the syllogism
we get when we cast an inductive argument into syllogistic form)
'infer that the major term is predicable of the middle term. by
means of the minor term' (b IS - I7 ). The statement is paradoxical;
it is to be explained by noticing that the terms are named with
reference to the position they would occupy in a demonstrative
syllogism (which is the ideal type of syllogism). A. bases his
example of the inductive syllogism on a theory earlier held. that
the absence of a gall-bladder is the cause of long life in animals
(Pari. An. 677"30 8&0 lCat xap'/t:rraTa "l,-ovu, TWJI ripxaUvJl 0,
t/>aulCoJITt:s- atnoJl flJla, TOU 1TAEUv {~JI X,P6J10Jl TO I-'~ EXf'JI XoA~JI).
A. had his doubts about the completeness of this explanation;
in An. Post. 99 b 4-7 he suggests that it may be true for quadrupeds
but that the long life of birds is due to their dry constitution or to
some third cause. The theory serves, however, to illustrate hisn. 23. 68 b20-J
point. In the demonstrative syllogism, that which explains
facts by their actual grounds or causes, the absence of a gall-
bladder is the middle term that connects long life with the
animal species that possess long life. Thus the inductive syllogism
which aims at showing not why certain animal species are long-
lived but that all gall-less animals are long-lived, is said to prove
the major term true of the middle term (not, of course, its own
middle but that of the demonstrat~ve syllogism) by means of the
minor (not its own minor but that of the demonstrative syllogism).
Now if instead of reasoning demonstratively 'All B is A, All C is
B, Therefore all C is A', we try to prove from All C is A, All C
is B, that all B is A, we commit a fallacy, from which we can
save ourselves only if in addition we know that all B is C (b 23 fl
oVv aVTLcrrp'</>fL 'T() r Tip B Kat /L~ inrffYTf{V(L TO /L'UOV, i.e. if B, the
p.(uov of the demonstrative syllogism, is not wider than C).
68b~O. Ecf c{I SE r TO Ka8' EKaUTO\l ...,aKP oj3 LO\l. In b 27 -9 A.
says that, to make the inference valid, r must consist of all
the particulars. Critics have pointed out that in order to prove
that all gall-less animals are long-lived it is not neeessary to
know that all long-lived animals fall within one or another of the
species examined, but only that all gall-less animals do. Accord-
ingly Grote (Arist. 3 187 n. b) proposed to read axo).ov for /Lo.KpO-
fJwv, and M. Consbruch (Arch. f. Gesch. d. Phil. v (1892), 310)
proposed to omit /Lo.KpofJLoV. Grote's emendation is not probable.
Consbruch's is more attractive, since fl4KpOfJwv might easily be
a gloss; and it derives some support from P.'s paraphrase, which
says (473. 16-17) simply TO r orOV Kopo.e KO.' ouo. TOLo.iho.. Myn OVV
on <> Kopo.e KO.' <> [).o.</>o, axo).a /LaKpOfJLQ. fluw. But P.'s change of
instances shows that he is paraphrasing very freely, and therefore
that his words do not throw much light on the reading. The
argument would be clearer if /Lo.Kp6fJwv, which is the major term A,
were not introduced into the statement of what r stands for.
But the vulgate reading offers no real difficulty. In saying ,,</>'
c;, Sf r TO Ko.O' [Ko.crrov /Lo.Kp6fJwv, A. does not say that r stands for
all /Lo.Kp6fJLa, but only that it stands for the particular /L;"Kp6fJLa
in question, those from whose being /Lo.KpofJLa it is inferred that all
ax0).o. are /Lo.KpOfJLa.
U-3. T~ Si) r ... T~ r. The structure of the whole passage
b 2I - 7 shows that in the present sentence A. must be stating the
data All C is A, All C is B, and in the next sentence adding
the further datum that' All C is B' is convertible, and drawing the
conclusion All B is A. Clearly, then, he must not, in this sentence,
state the first premiss in a form which already implies that allCOMMENTARY
B is A, so that 'Tra.v yap Td O:XOAov p.aKp6{3tov cannot be right; we
must read r for O:xoAov. Finding p.aKp6{3tov (which is what A
stands for) substituted by A. for A in b 22 , an early copyist has
rashly substituted axoAov for r; but r survives (though deleted)
in n after O:xoAov, and Pacius has the correct reading. Instead of
the colon before and the comma aftir miv . . . p.aKp6{3tov printed
in the editions, we must put brackets round these words.
Tredennick may be right in suggesting the omission of 'Tra.v ...
p.aKp6{3tov, but I hesitate to adopt the suggestion in the absence
of any evidence in the MSS.
24-9. lWiEIICT(U yap ••• miVTIalV. A. has shown in a21-4 that
if all C is A, and all C is B, and C (Td O:KPOV of b 2 6, i.e. the term
which would be minor term in the corresponding demonstrative
syllogism All B is A, All C is B, Therefore all C is A) is convertible
with B (8cfTEpOV aVTwv of b 2 6), A will be true of all B (TctJ av·rt-
u-rPErPOVTt of b2 6, the term convertible with C). But of course to
require that C must be convertible with B is to require that C
must contain all the things that in fact possess the attribute B.
26. TO aKpov, i.e. C, the minor term of the apodeictic syllogism.
In b 34 , 35 'Td O:KPOV is A, the major term of both syllogisms.
27-8. OE~ Of ••• aUYKELfJ-£VOV, 'we must presume C to be the
class consisting of all the particular species of gall-less animals'.
For vo,,'v with double accusative cf. L. and S. S.v. VOEW I. 4.
It may seem surprising that A. should thus restrict induction
(as he does, though less deliberately, in 69"17 and in An. Post.
92"38) to its least interesting and important kind; and it is
certain that In many other passages he means by it something
quite different, the intuitive induction by which (for instance)
we proceed from seeing that a single instance of a certain geo-
metrical figure has a certain attribute to seeing that every
instance must have it. It is certain too that in biology, from
which he takes his example here, nothing can be done by the
mere use of perfect induction; imperfect induction is what really
operates, and only probable results can be obtained. The present
chapter must be regarded as a tour de force in which A. tries at
all costs to bring induction into the form of syllogism; and only
perfect induction can be so treated. It should be noted too that
he does not profess to be describing a proof starting from observa-
tion of particular instances. He knows well that he could not
observe all the instances, e.g., of man, past, present, and future.
The advance from seeing that this man, that man, etc., are both
gall-less and long-lived has taken place before the induction here
described takes place, and has taken place by a different method(imperfect induction). What he is describing is a process in
which we assume that all men, all horses, all mules are gall-less
and long-lived and infer that all gall-less animals are long-lived.
And while he could not think it possible to exhaust in observation
all men, all horses, all mules, believing as he does in a limited
number of fixed animal species he might well think it possible
to exhaust all the classes of gall-less animals and find that they
were all long-lived. The induction he is describing is not one
from individuals to their species but from spec;:ies to their genus.
This is so in certain other passages dealing with induction (e.g.
Top. IOSaI3~I6, Met. I048a3S~b4), but in others induction from
individual instances is contemplated (e.g. Top. I03b3-6, IOS b2S-9,
Rhet. 1398a32~bI9). In describing induction as proceeding from
'TO Ka8' €Kaa'TOV to 'TO Ka86Aov he includes both passage from indi-
viduals to their species and passage from species to their genus.
30-1. "Ean S' ... trpOTaaEw5. i.e. such a syllogism establishes
the proposition which cannot be the conclusion of a demonstrative
syllogism but is its major premiss, neither needing to be nor
capable of being mediated by demonstration.
36---7. ';ILLV S' ... Etra.ywy115, i.e. induction, starting as it does
not from general principles which may be difficult to grasp but
from facts that are neater to sense, is more immediately con-
vincing. Nothing could be more obvious than the sequence of
the conclusion of a demonstration from its premisses, but the
difficulty in grasping its premisses may make us more doubtful
of the truth of its conclusion than we are of the truth of a con-
clusion reached from facts open to sense.
CHAPTER 24-
Argument from an example
68b38. It is example when the major term is shown to belong
to the middle term by means of a term like the minor term. We
must know beforehand both that the middle term is true of the
minor, and that the major term is true of the term like the minor.
Let A be evil, B aggressive war on neighbours, C that of Athens
against Thebes, D that of Thebes against Phocis. If we want to
show that C is A, we must first know that B is A; and this we
learn from observing that e.g. D is A. Then we have the syllogism
'B is A, C is B, Therefore C is A'.
69 a 7. That C is B,that D is B, and that D is A, is obvious;
that B is A is proved. by means of D. More than one term like C
may be used to prove that B is A.COMMENTARY
13. Example, then, is inference from part to part, when both
fall under the same class and one is well known. Induction
reasons from all the particulars and does not apply the conclusion
to a new particular; example does so apply it and does not reason
from all the particulars.
The description of 1Tapd,8£t'Yf.£a in the' first sentence of the
chapter would be very obscure if that sentence stood alone.
But the remainder of the chapter makes it clear that by 1Tapa8£t'Yf.£a
A. means a combination of two inferences. If we know that two
particular things C ('TI) Tpl-rOV) and D (Td of.£o~ov TcjJ TP'Tcp) both have
the attribute E (Td ,."£aov), and that D also has the attribute A
(Td aKpov or 7TpWTOV) (6987-10), we can reason as follows: (I) D is
A, D is E, Therefore E is A, (2) E is A, C is E, Therefore C is A.
The two characteristics by which A. distinguishes example from
induction (69"16-19) both imply that it is not scientific but purely
dialectical or rhetorical in character; in its first part it argues
from one instance, or from several, not from all, and in doing so
commits an obvious fallacy of illicit minor; and to its first part,
in which a generalization is reached, it adds (in its second part)
an application to a particular instance. Its real interest is not,
like that of science, in generalization, but in inducing a particular
belief, e.g. that a particular aggressive war will be dangerous to
the country that wages it.
68b38. TO a.KPOV, Le. the major term (A); so in 69&13, 17. Td
aKpov ib. 18 is the minor term (C).
6982: e,,~aiou, 1I'pO, 41wKEi,. This refers to the Third Sacred
War, in 356-346, referred to also in Pol. 13°4812. The argument
is one such as Demosthenes might have used in opposing the
Spartan attempt in 353 to induce Athens to attack Thebes in
the hope of recovering Oropus (cf. Dem. :Y7T£p TWV M£'Ya'\o1To'\~TWV)
12-13. " 1I'ian, ••. a.KPOV. Waitz argues that if Td aKpov here
meant the major term, i.e. if the proposition referred to were that
the major term belongs to the middle term, A. would have said
~ 1T,crn, 'Y'VO~TO TOV aKpov 1Tpd, Td f.££aov. That is undoubtedly A.'s
general usage, the term introduced by 1TPO, being the subject of
the proposition referred to; cf. 26817, 27"26, 28°17, b S, 40b 39 , 41°1,
45 b5, 5884. Waitz supposes therefore that A. means the proof
that the middle term belongs to the minor. But there is no proof
of this; it is assumed as self-evident (68b39-40, 69 87-8). A. must
mean the proof connecting the middle term (as subject) with the
major (as predicate); cf. 817-18.
17. HiEiKVUEV, i.e. 'shows, as we saw in ch. 23'.CHAPTER 25
Reduction of one problem to another
64J zo. Reduction occurs (1) when it is clear that the major
term belongs to the middle term, and less clear that the middle
term belongs to the minor, but that is as likely as, or more likely
than, the conclusion to be accepted; or (2) if the terms inter-
mediate between the minor and the middle term are few; in any
of these cases we get nearer to knowledge.
z4. (1) Let A be 'capable of being taught', B 'knowledge',
'justice'. B is clearly A ; if T is B' is as credible as, or more
credible than, T is A' we come nearer to knowing that r is A,
by having taken in the premiss 'B is A'.
z9. (2) Let.1 stand for being squared, E for rectilinear figure,
Z for circle. If there is only one intermediate between E and Z,
in that the circle along with certain lunes is equal to a rectilinear
figure, we shall be nearer to knowledge.
34. When neither of these conditions is fulfilled, that is not
reduction; and when it is self-evident that r is B, that is not
reduction, but knowledge.
a
r
a:rrIlYIIlYTJ (simpliciter) is to be distinguished from the more
familiar a1Taywyry £,!> 'T6 aovva'Tov, but has something in common
\vith it. In both cases, wishing to prove a certain proposition
and not being able to do so directly, we approach the proof of it
indirectly. In reductio ad impossibile that happens in this way:
having certain premisses from which we cannot prove what we
want to prove, by a first-figure syllogism (which alone is for
A. self-evidencing), we ask instead what we could deduce if the
proposition were not true, and find we can deduce something
incompatible with one of the premisses. In red'uctio (simpliciter)
it happens in this way: we turn away to another proposition
which looks at least as likely to be accepted by the person with
whom we are arguing (op.o[wc; 1TLU'TOV ~ p.iiM.ov 'ToO uvp.1T£paup.a'Toc;,
821) or likely to be proved with the use of fewer middle terms
(av o'\'ya -n 'Ta p./ua, a22), and point out that if it be admitted, the
other certainly follows. If our object is merely success in argu-
ment and if our adversary concedes the substituted proposition,
that is enough. If our object is knowledge, or if our opponent
refuses to admit the substituted proposition, we proceed to try to
prove the latter.
This type of argument might be said to be semi-demonstrative,49 0
COMMENTARY
semi-dialectical, inasmuch as it has a major premiss which is
known, and a minor premiss which for the moment is only
admitted. It plays a large part in the dialectical discussions of
the Topics (e.g. 159b8-23, 160'II-14). But it also plays a large
part in scientific discovery. It was well recognized in Greek
mathematics; cf. Procl. in Eucl. 212. 24 (Friedlein) ~ 8£ d7Taywr1J
fLerd{3aa{'i £anv d7T' a'\'\ou 7Tpo{3'\~p.aTO'i ~ (i£wp~p.aTO'i £7T' a'\'\o, ov
yvwa(Uv-ro'i ~ 7TOPLU81v-r0'i Kat TO 7TpOKdp.£vov £GTaL KaTorpavl<;. In
fact it may be said to be the method of mathematical discovery,
as distinct from mathematical proof.
It is in form a perfect syllogism, but inasmuch as an essential
feature of it is that the minor premiss is not yet known, it belongs
properly not to the main theory of syllogism (to which it is
indifferent whether the premisses are known or not), but to the
appendix (chs. 23-7) of which this chapter forms part. Maier
(ii a. 453 n. 2) suggests that it may be a later addition to this
appendix, and that perhaps its more proper place would be
between chs. 21 and 22. But it seems to go pretty well in its
present place, along with the discussion of the other special types
of argument-induction, example, objection, and enthymeme.
The method is described clearly by Plato (who does not use
the word d7Tayw~, but describes the method as that of proof £~
tJ7To81u£w'i) in Meno 86 e-87 c. It is from there that A. takes his
example, 'virtue is teachable if it is knowledge'; and Plato also
anticipated A. ('30-4) in taking an example from mathematics.
69'21-2. 0f-L0LW5 S£ ••• aUf-LlTEpaaf-LaTo5. The premiss will be
no use unless it is more likely to be admitted than the conclusion.
I suppose A. means that it must be a proposition which no one
would be less likely to admit, and some would be more likely to
admit, than the conclusion.
28-c}. SLC1. TO lTpOaELAT]cj>evaL ••• ima:T11f-LT]v. The MSS. have
Ar; but 7Tpou'\op.f3dvHv is used regularly of the introduction of a
premiss (28'S, 29"16, 42'34, etc.), and A. could not well say 'we
get nearer to knowing that C is A by having brought in the know-
ledge that C is A'. Nor can it be 'the knowledge that C is B';
for this is only believed, not known ('21-2). It must be the know-
ledge that B is A ; by recogn;zing this fact, which we had not
recognized before, we get nearer to knowing that C is A, since
we have grasped the connexion of A with one of the middle terms
which connect it with C.
30-4. oIov £t ••. t:tSevaL. If we are trying to show that the
circle can be squared, we simplify our problem by stating a premiss
which can easily be proved, viz. that any rectilinear figure cann.
25. 69"21-34
49 1
be squared. We then have on our hands a slightly smaller task
(though still a big enough one!), viz. that of linking the subject
'circle' and the predicate 'equal to a discoverable rectilinear
figure', by means of the middle term 'equal, along with a certain
set of lunes' (i.e. figures bounded by two arcs of circles), to a dis-
coverable rectilinear figure'.
This attempt to square the circle is mentioned thrice elsewhere
in A.-in Soph. El. 171b1Z 'Td ydp .pwooypac/nifLa'Ta OUK £P'(TT.Ka. •..
OUO' y' £L 'T[ £(TTL .pwOoypac!>TJfLa 7T£pl cl)4Nc;, ofov 'TO 'I7T7ToKpa.'TOUC; ~ 0
'T£'TpaywvLofL0C; 0 otd 'TWV fLTJV[OKWV, ib. 17z"2 ofov 0 'T£'Tpaywv.ofL0C; 0
fL£v SLd 'TWV fLTJV{OKWV OUK £PWTLKOC;, and Phys. 185"14 o'fLa 0' OUO£
Au£LV o'7TaV'Ta 7TPOrn/KH, cl'\A' Ti oua £K 'TWV clpxwv 'T'C; £mOHK\.·vc;
o£
o.d
.p£uO£'TaL, oua
fL~, ov, ofov 'TOV 'T£'Tpaywv.ufLOv 'TOV fL£V
'TWV
'TfLTJfLa.'TWV Y£WfL£-rp.KOV OLa,\vua.. There has been much discussion
as to the details of the attempt. The text of Soph. El. 17lbl5
implies that it was different from the attempt of Hippocrates of
Chios; but there is enough evidence, in the commentators on the
Physics, that it was Hippocrates that attempted a solution by
means of lunes, and Diels is probably right in holding Ti 0 'T£'Tpa-
YWVLOfL0C; 0 oLd 'TWV fLTJV{UKWV to be a (correct) gloss, borrowed
from 172"2, on 'TO 'I7T7ToKpa.'Touc;.
I have discussed the details at length in my notes on Phys.
185"16, and there is a still fuller discussion in Heath, Hist. of
Gk. Math. i. 183-200, and Mathematics in Aristotle, 33-6. Re-
ferences to modern literature are given in Diels, Vors.5 i. 396; to
these may be added H. Milhaud in A.G.P. xvi (19°3),371-5.
CHAPTER 26
Objection
69"37. Objection is a premiss opposite to a premiss put forward
by an opponent. It differs from a premiss in that it may be par-
ticular, while a premiss cannot, at least in universal syllogisms.
An objection can be brought (a) in two ways and (b) in two figures;
(a) because it may be either universal or particular, (b) because
it is opposite to our opponent's premiss, and opposites can be
proved in the first or third figure, and in these alone.
bS. When the original premiss is that all B is A, we may object
by a proof in the first figure that no B is A, or by a proof in the
third figure that some B is not A. E.g., let the opponent's premiss
be that contraries are objects of a single science; we may reply (i)
'opposites are not objects of a single science, and contraries areCOMMENTARY
opposites', or (ii) 'the knowable and the unknowable are not
objects of a single science, but they are contraries'.
15. So too if the original premiss is negative, e.g. that con-
traries are not objects of a single science, we reply (i) 'all opposites
are objects of a single science, and contraries are opposites', or
(ii) 'the healthy and the diseased are objects of a single science,
and they are contraries'.
19. In general, (i) if the objector is trying to prove a universal
proposition, he must frame his opposition with reference to the
term which includes the subject of his opponent's premiss; if he
says contraries are not objects of a single science, the objector
replies 'opposites are'. Such an objection will be in the first
figure, the term which includes the original subject being our
middle term.
24. (ii) If the objector is trying to prove a particular proposi-
tion, he must take a term included in the opponent's subject, and
say e.g. 'the knowable and the unknowable are not objects of a
single science'. Such an objection will be in the third figure, the
term which is included in the original subject being the middle
term.
28. For premisses from which it is possible to infer the opposite
of the opponent's premiss are the premisses from which objections
must be drawn. That is why objections can only be made in these
two figures; for in these alone can opposite conclusions be drawn,
the second figure being incapable of proving an affirmative.
32. Besides, an objection in the second figure would need
further proof. If we refuse to admit that A belongs to B, because
C does not belong to A, this needs proof; but the minor premiss of
an objection should be self-evident.
38. The other kinds of objection, those based on consideration
of things contrary or of something like the thing, or on common
opinion, require examination; so does the question whether there
can be a particular objection in the first figure, or a negative one
in the second.
49 2
This chapter suffers from compression and haste. Objection is
defined as 'a premiss opposite to a premiss' (for Evav-ria in 6I) a 37
must be used in its wider sense of 'opposite', in which it includes
contradictories as well as contraries). The statement that
EVU"Taat, is a premiss opposed to a premiss is to be taken seriously;
EVUrraa(Jat is 'to get into the way' of one's opponent, to block him
by denying one of his premisses, instead of waiting till he has
framed his syllogism and then offering a counter-syllogism (Rhet.11.26
493
1402831, 1403"26, 1418b S). In Top. 160839-bl0 A. contrasts lvcrram!;
with aVTl.uv.uoy~afLO!;, to the advantage of the former; it has the
merit of pointing out the TTpWTOV "'~fjSO!; on which the opponent's
contemplated argument would rest (Soph. El. 179b23, cf. Top.
160b3 6).
But lvcrrau,~ is not merely the stating of one proposition in
opposition to another. It involves a process of argument; and
the proposition it opposes, while it is described throughout as
a premiss, is itself thought of as having been established by a
syllogism. For it is only on this assumption that we can explain
the reason A. gives for saying that objections can only be carried
out in the first and third figures, viz. that only in these can
opposites be proved, or in other words that the second figure
cannot prove affirmative propositions (b3- S, 29-32). A. must
mean that tvcrrau,~ is the disproving of a premiss (which the
opponent might otheIWise use for further argument) by a proof
in the same figure in which that premiss was proved.
A. places three arbitrary restrictions on the use of tvcrrau.!;.
(1) He restricts it to the refutation of universal premisses, on the
ground that only such occur in the original syllogism, or at least
in syllogisms proving a universal ("39-bl). This restriction is
from the standpoint of forrnallogic unjustifiable, but less so from
the standpoint of a logic of science, since syllogisms universal
throughout are scientifically more important than those that
have one premiss particular. (2) He insists, as we have seen, that
the objection must be carried out in the same figure in which the
original syllogism was couched, and that for this reason it cannot
be in the second figure. But he should equally, on this basis, have
excluded the third figure. This can prove conclusions in I and
in 0, but these form no real contradiction. (3) While he is justi-
fied, on the assumption that the second figure is excluded, in
limiting to the first figure the proof of the contrary of a universal
proposition, he is unjustified in limiting to the third figure, and
to the moods Felapton and Darapti, the proof of its contradictory
(b S- 19)·
Removing all these limitations, he should have recognized that
an A proposition can be refuted in any figure (by Celarent or
Ferio; Cesare, Camestres, Festino, or Baroco; Felapton, Bocardo,
or Ferison); an E proposition in the first or third figure (by
Barbara or Darii; Darapti, Disamis, or Datisi) ; an I proposition
in the first or second figure (by Celarent, Cesare, or Camestres) ;
an proposition in the first (by Barbara).
If we allow A. to use the third figure while inconsistently
°COMMENTARY
494
rejecting the second, his choice of moods-Celarent to prove the
contrary of an A proposition (b9- I2 ), Felapton to prove its con-
tradictory (b I2- IS ), Barbara to prove the contrary of an E pro-
position (b IS - q ), Darapti to prove its contradictory (b q - I8)-is
natural enough; only Celarent will prove the contrary of an A
proposition, only Barbara that of an E proposition; Felapton
is preferred to Ferio, Bocardo, and Ferison, and Darapti to Darii,
Disamis, and Datisi, because they have none but universal
premisses.
The general principles A. lays down for ElIO'TaaL, (b I9 - 2 8) are
that to prove a universal proposition a superordinate of the sub-
ject should be chosen as middle term, and that to prove a parti-
cular proposition a subordinate of the subject should be chosen.
This agrees with his choice of moods; for in Celarent and Barbara
the minor premiss is All 5 is M, and in Felapton and Darapti it
is All MisS.
Maier (2 a. 4SS-6) considers that A. places a fourth restriction
on ElIa'raaL,-that an objection must deny the major premiss from
which the opponent has deduced the TTpo-raaL, we are attacking,
so that the opposed syllogisms must be (to take the case in which
we prove the contrary of our opponent's proposition) of the form
All M is P, All 5 is M, Therefore all 5 is P-No M is P, All
5 is M, Therefore no 5 is P. He interprets a.lIayK1) TTPO, 'TO Ka8oAou
'TWlI TTponLlIo/dllWlI 'T~1I a.V7't,paaLII ElTTELII (b 2 o- I ) as meaning 'he must
take as his premiss the opposite of the universal proposition from
which as a major premiss the opposed TTpO'TaaL, was derived'. If
the article in 'TO Ka8oAov is to be stressed, this interpretation must
be accepted; for if A. is thinking of 5 as having only one super-
ordinate, the opposed syllogisms must be related as shown
above. It is, however, quite unnecessary to ascribe this further
restriction to A. What the words in question mean is 'he must
frame his contradiction with a view to the universal (i:e. some
universal) predicable of the things put forward by the opponent'
(i.e. of the subject of his TTp0'TaaL,). For A. goes on to say 'e.g., if
the opponent claims that no contraries are objects of a single
science, he should reply that opposites (the genus which includes
both contraries and contradictories) are'-without suggesting that
the opponent has said 'No opposites are objects of a single science,
and therefore no contraries are'. In fact an ElIO'TaaL, would be
much more plausible if it did not start by a flat contradiction
of the opponent's original premiss, but introduced a new middle
term; and A. can hardly have failed to see this. This interpreta-
tion is confirmed by what A. says about the attempt to proveII. 26. 69b21-37
a particular 'objecting' proposition (b 24- S). There the objector
must frame his objection 'with reference to that, relatively to
which the original subject was universal' (i.e. to a (not the)
subordinate of the subject, as in the fonner case to a super-
ordinate of it).
Maier argues (ii. a. 471-4) that the treatment of £vu-raUt, here
presupposes the treatment in Rhet. 2. 15. He thinks, in particular,
that the vague introductory definition of £vu-raut> , as 'a premiss
opposite to a premiss', is due to the fact that in the Rhetoric
£vcrraut, not involving a counter-syllogism is recognized as well
as the kind (which alone is treated in the present chapter) which
does involve one. But his argument to show that the present
chapter is later than the context in which it is found is not con-
vincing, though his conclusion may be in fact true. The kind of
£vu-raut, dealt with in the present chapter turns out to be a per-
fectly normal syllogism; its only pecularity is that it is a syllogism
used for a particular purpose, that of refuting a premiss which
one's opponent wishes to use. And in this respect, that it is
a particular application of syllogism, it is akin to the other
processes dealt with in this appendix to A n. Pr. II (chs_ 23-7).
69b2I-Z. otov EL ... I-4Lav. The sense requires the placing of a
comma before 7TavTwv, not after it as in Bekker and Waitz; cf. b 1 6.
z4-5. 1TpOS 0 ... 1TpoTacns. 7TPO, 0 = 7TPO, TOVTO 7TPO, 0, 'the
objector must direct himself to the tenn by reference to which
the subject of his opponent's premiss is universal'.
31. lha. ya.p TOU I-4(UOU ••• KaTa4>q.TlI(WS. cf. 28"7-9.
3Z-7. ETI S( .•. (unv. This further reason given for objection
not being possible in the second figure is obscure. It is not clear,
at first sight, whether in b 34 aVrce means A or B, nor whether
TOVTO means (la) 'that A is not C' or (Ib) 'that B is not C' or
(2a) 'that .. B is not A" follows from .. A is not C .. " or (2b)
'that" B is not A" follows from" B is not C" '. Interpretations
la and Ib would involve A. in the view that negative proposi-
tions cannot be self-evident, but this interpretation is ruled
out by three considerations. (I) A. definitely lays it down in
A n. Post. i. 15 that negative propositions can be self-evident. (2)
He has already used negative premisses, as of course he must do,
for the £vu-raUt, in the first or third figure to an affinnative
proposition (b S- IS ). (3) He says in b 36 that the reason why an
£va-raut, in the second figure is less satisfactory than one in the
first or third is that the other premiss should be obvious, i.e. that
if we state the £vuTaUt, briefly, by stating one premiss, it should
be clear what the 'understood' premiss is. Thus interpretationCOMMENTARY
za or zb must be right. Of the two, 2a is preferable. For if
to All B is A we object No A is C, it is, owing to the change both
of subject and of predicate, by no means clear what other premiss
is to be supplied, while if we object No B is C, it is clear that the
missing premiss must be All A is C.
.
36-7. S~o Ka.l ••• Ea~w. Cook Wilson argued (in Trans. of the
Oxford Philol. Soc. 1883-4, 45-6) that this points to an earlier
form of the doctrine of enthymeme than that which is usual in
the Prior A nalytics and the Rhetoric; that A. recognized at this
early stage an analogy between Evcrra(n~ and the argument from
signs, in that while Evcrra(n~ opposes a particular statement to a
universal and a universal statement to a particular, O"TJIL£~oV sup-
ports a universal statement by a particular and a particular
statement by a universal.
Wilson cannot be said to have established his point. The
present sentence does not refer to any general analogy between
Evcrraa~~ and O"TJIL£Zov, but only to the fact that because of ob-
scurity the second figure is unsuitable for both purposes.
The sentence is unintelligible in its traditional position. It
might be suggested that it was originally written in the margin,
and was meant to come after KaTarj>anKw~ in b 3 I. The fact that
the second figure is essentially negative is in effect the reason
given in 70"35-7 for the invalidity of proof by signs in that figure.
. But even so the sentence can hardly be by A. For A. does not
in fact hold that the second figure alone is unsuitable for O"TJIL£ZOV.
He mentions in the next chapter O"TJIL£Za in all three figures
(70&11-28). It is true that he describes O"TJIL£'ia in the second figure
as always refutable (because of undistributed middle) ("34-7), but
he also describes those in the third figure as refutable because,
though they prove something, they do not prove what they claim
to prove (because of illicit minor) ("30-4). Ch. 27 in fact draws a
much sharper line between O"TJIL£'ia in the first figure (T£KIL~p,a)
and those in the other two, than it does between those in the
third and those in the second figure. Susemihl seems to be right
in regarding the sentence as the work of a copyist who read
ch. 27 carelessly and overstressed the condemnation of the second
figure O"TJIL£ZoV in 70"34-7. There is no trace of the sentence in P.
38-7oaz. 'E1rWItE1rTEOV SE ... ~a.!3E'iV. In Rhet. ii. 25 A. recog-
nizes four kinds of EvcrraaL~: (I) &.rj>' £avrov. If the opponent's
statement is that love is good, we reply either (a) universally by
saying that all want is bad, or (b) particularly by saying that
incestuous love is bad. (2) alTO 'rov 'vaVTLov. If the opponent's
statement is that a good man does good to all his friends, we reply497
'a bad man does not do evil to all his friends'. (3) a1To TOV ofLolov. If
the statement attacked is that people who have been badly
treated always hate those who have so treated them, we reply
that people who have been well treated do not always love those
who have so treated them. (4) a, Kplu~L~ a' d.1T0 TWV yvwPLfLwV av8pwv.
If the statement attacked is that we should always be lenient
to those who are drunk, we reply 'then Pittacus is not worthy
of praise; for if he were he would not have inflicted greater
penalties on the man who does wrong when drunk'.
Here the first kind agrees exactly with that described in the
present chapter; the other three kinds (which answer to EK TOV
EvaVTlov Kat TaU OfLolov Kat TOV KaT<l 06~av here), not being sus-
ceptible of simple syllogistic treatment, are not suitable for dis-
cussion in the Prior A nalytics.
The second half of the sentence raises the question whether it
is not possible to prove a particular 'objecting' statement in'the
first figure, or a negative one in the second. But even to suggest
this is to undermine the whole teaching of the chapter.
From the irrelevance of the first part of the sentence and the
improbability of the second, Cook Wilson (in Gott. Gel. A nzeiger,
1880, Bd. 1,469-74), followed by Maier (ii a. 460 n. 2), has inferred
that the sentence is a later addition by someone familiar with the
teaching of Rhet. ii. 25. This conclusion would be justified if the
Prior Analytics were a work prepared for publication. But
probably none of A.'s extant works was so prepared, and in an
'acroamatic' work the sentence is not impossible as a note to
remind the writer himself that the whole chapter needs further
consideration. Similar notes are to be found in 35"2, 41b31, 4Sb19,
49 8 9, 67 b26 .
We need not concern ourselves with the wider sense in which
the word £vu'rauL~ is used in the Topics, covering any attempt to
interfere with an opponent's carrying through his argument. Cf.
for instance 16Ial-15, where four kinds are named, of which the
first (av£A6vTa 1Tap' 0 ylv£TaL TO rff£v80~, disproving the premiss on
which the false conclusion of our opponent depends) includes
£vrrrauL<; as described in the, present chapter, but also £VUTaUL>
against an inductive argument. But it may be noted that the
great majority of the EvrrrauH<; in the Topics belong to the second
of the two types discussed in this chapter-refutation of a pro-
position by pointing to a negative instance (II48zo, IISbI4,
1I7a18, lZ3b17, 27, 34, lz4b3Z, lz581, I28 b6, 156a34, 157 bz ). For the
discussion of £vuTauL<; in the wider sense reference may be made
to Maier, ii. a. 462-74.
4985
KkCOMMENTARY
CHAPTER 27
I nference from signs
70"la. An enthymeme is a syllogism starting from probabilities
or signs. A probability is a generally approved proposition, some-
thing known to happen, or to be, for the most part thus and thus.
6. A sign is a demonstrative premiss that is necessary or gene-
rally approved; anything such that when it exists another thing
exists, or when it has happened the other has happened before or
after, is a sign of that other thing's existing or having happened.
II. A sign may be taken in three ways, corresponding to the
position of the middle term in the three figures. First figure, This
woman is pregnant; for she has milk. Third figure, The wise are
good; for Pittacus is good. Second figure, This woman is preg-
nant; for she is sallow.
24. If we add the missing premiss, each of these is converted
from a sign into a syllogism. The syllogism in the first figure is
irrefutable if it is true; for it is universal. That in the third figure
is refutable even if the conclusion is true; for it is not universal,
and does not prove the point at issue. That in the second figure is
in any case refutable; for terms so related never yield a conclusion.
Any sign may lead to a true conclusion; but they have the differ-
ences we have stated.
bl. We may either call all such symptoms signs, and those of
them that are genuine middle terms evidences (for an evidence
is something that gives knowledge), or call the arguments from
extreme ternlS signs and those from the middle term evidences;
for that which is proved by the first figure is most generally
accepted and most true.
7. It is possible to infer character from bodily constitution, if
(I) it be granted that natural affections change the body and the
soul together (a man by learning music has presumably undergone
some change in his soul; but that is not a natural affection; we
mean such things as fits of anger and desires) ; if (2) it be granted
that there is a one-one relation between sign and thing signified;
and if (3) we can discover the affection and the sign proper to each
species.
14. For if there is an affection that belongs specially to some
infima species, e.g. courage to lions, there must be a bodily sign of
it; let this be the possession of large extremities. This may belong
to other species also, though not to them as wholes; for a sign is
proper to a species in the sense that it is characteristic of the
whole of it, not in the sense that it is peculiar to it.II.27
499
22. If then (1) we can collect sllch signs in the case of animals
which have each one special affection, with its proper sign, we
shall be able to infer character from physical constitution.
26. But if (z) the species has two characteristics, e.g. if the lion
is both brave and generous. how are we to know which sign is the
sign of which characteristic? Perhaps if both characteristics
belong to some other species but not to the whole of it, and if
those other animals in which one of the two characteristics is
found possess one of the signs, then in the lion also that sign will
be the sign of that characteristic.
32. To infer character from physical constitution is possible
because in the first-figure argument the middle term we use is
convertible with the major. but wider than the minor; e.g. if B
(larger extremities) belongs to C (the lion) and also to other
species, and A (courage) always accompanies B. and accompanies
nothing else (otherwise there would not be a single sign correlative
with each affection).
The subject of this chapter is the enthymeme. The enthymeme
is discussed in many passages of the Rhetoric, and it is impossible
to extract from them a completely consistent theory of its nature.
Its general character is that of being a rhetorical syllogism (Rhet.
13S6b4). This, however, tells us nothing directly about its real
nature; it only tells us that it is the kind of syllogism that orators
tend to use. But inasmuch as the object of oratory is not know-
ledge but the producing of conviction, to say that enthymeme is
a rhetorical syllogism is to tell us that it lacks something that a
scientific demonstration has. It may fail short of a demonstra-
tion, however. in anyone of several ways. It may be syllogisti-
cally invalid (as the second- or third-figure arguments from signs
in fact are, 70330-7). It may proceed from a premiss that states
not a necessary or invariable fact but only a probability (as the
argument £~ £lKClTWV does, ib. 3-7). It may be syllogistically
correct and start from premisses that are strictly true. but these
may not give the reason for the fact stated in the conclusion, but
only a symptom from which it can be inferred (as in the first-
figure argument from signs (ib. 13-16).
A.'s fullest list of types of enthymeme (Rhet. 14ozb13) describes
them as based on four different things-£LKo" 1Tapa.8£tYfLa,
T£KfL~PtoV, aTJfL£LOV. But elsewhere 1Tapa.8EtYfLa is made co-ordinate
with £VOU,.L7]fLa, and is said to be a rhetorical induction. as enthy-
meme is a rhetorical syllogism (13S6b4--Q). Thus the list is reduced
to three, and since T£KfLrJPtov is really one species of aTJfL£LOVCOMMENTARY
(7 0br --6), the list is reduced finally to two-the enthymeme
J~ "lKOTWII and the enthymeme JK 07Jp,€lwll. "lKO, is described here
as 7TPOTUU', £1I00~O, (7°84) ; in Rhet. r357834-br it is described more
carefully--ro fL£1I ya.p £lKO, Junll W, J7T~ TO 7TO>.U Y'"OfL€JIO", oux
ci'TTAW, O£ KalJd7T£p op{~ollTa{ TW£" dAAa TO 7Tf;P~ Ta Jllo"xop,€lIa row,
£XHII, oVrw, £XOII 7TPO, JK£tIlO 7TPO, a £lKO, 01, TO KalJoAov 7TPO, TO
KaTa fL£PO'. I.e., an £lKO, is the major premiss in an argument of
the form 'B as a rule is A, C is B, Therefore C is probably A'.
The description of £lKO, in the present chapter (70"3-7) is per-
functory, because the real interest of the chapter is in 07JfL"toll.
07JfL"wII is described as a 7TpoTau" d7TooHKnlO} ~ dllaYKa{a ~ £1I00~O,
(87). The general nature of the 7TpOTau" is alike in the two cases;
it states a connexion between a relatively easily perceived
characteristic and a less easily perceived one simultaneous,
previous, or subsequent to it (as-ro). The distinction expressed
by ~ dllaYKala ~ £1I00~O, is that later pointed out between the
T"KfL~PWII or sure symptom and the kind of 07JfL£wII which is an
unsure symptom. The distinction is'indicated formally by saying
that a T£KfL~PWII gives rise to a syllogism in the first figure-e.g.
('All women with milk are pregnant), This woman ·has milk,
Therefore she is pregnant', while a 07JfL£toll of the weaker kind
gives rise to a syllogism in the third figure-e.g. 'Pittacus is
good, (Pittacus is wise,) Therefore the wise are good' -or in the
second-e.g. (,Pregnant women are sallow,) This woman is
sallow, Therefore she is pregnant'. The first-figure syllogism is
unassailable, if its premisses are true, for its premisses warrant
the universal conclusion which it draws (829-30). The third-figure
syllogism is assailable even if its conclusion is true; for the
premisses do not warrant the universal conclusion which it draws
(a30-4). The second-figure syllogism is completely invalid because
two affirmative premis$es in that figure warrant no conclusion at
all (a34-7).
In modern books on formal logic the enthymeme is usually
described as a syllogism with one preI?iss or the conclusion
omitted; A.· notes (8r9-2o) that an obvious premiss is often
omitted in speech, but this forms no part of his definition of the
enthymeme, being a purely superficial characteristic.
On A.'s treatment of the enthymeme in general (taking account
of the passages in the Rhetoric) cf. Maier, ii a. 474-501.
70810. 'Ev9ul'T)l'a. SE .•• <7TIf1€~wv. These words should stand
at the beginning of the chapter, which in its traditional form
begins with strange abruptness; the variation in the MSS.
between O£ and fL£1I OVII may point to the sentence's having got
500501
out of place and to varying attempts having been made to fit
it in. If the words are moved to 2 a, the chapter about £Y8vp.7Jp.a
begins just as those about £1Tayw;n) , 1Tapcfof:'yp.a, a1Tayw;n1, and
EYfTTams do, with a summary definition.
7-8. OTJI141LOV SE ..• ;vSo~os. Strictly only a necessary premiss
can be suitable for a place in a demonstration, and Maier there-
fore brackets ayaYKata as a gloss on a1ToSHKnK1J. But avaYKala is
well supported, and a1TooHKnK1J may once in a way be used in a
wider sense, the sense of uv'\'>'O')"fTT'K1/; cf. Soph. El. 167bg £Y 'TOrS
PTJ'T0p'KOrS al Ka'Ta 'T6 CT7JP.f:LOY a1ToSd~HS £K 'TWY ;1TOP.£YWY f:laCy
(which is apparently meant to include all arguments from CT7JP.Era,
not merely those from Tf:KP.1/pW), De Gen. et Corr. 333b24 ~ opluau8a,
~ lnr08£u8a, ~ a1TOOf:L~a" ~ aKp,{3ws ~ p.a).aKws, Met. I02SbI3
a1TOSf:'KYVOVU'Y ~ ayaYKaWTf:pOY ~ p.aAaKW-rf:pOY.
bI-S' "H Si) ... C7)(tll1(lTOS. 'T6 p.£UOY is the term which occupies
a genuinely intermediate position, i.e. the middle term in the
first figure, which is the subject of the major premiss and the
predicate of the minor. 'Ta. aKpa are the middle terms in the other
two figures, which are either predicated of both the other terms
or subjects to them both.
']-38. To SE CPUULOYVWl10VELV ••• OTJIlELOV. 'T6 cpvu'O')'Vwp.oYt:LV is
offered by A. as an illustration of the enthymeme £K CT7Jp.f:iwy.
The passage becomes intelligible only if we realize something that
A. never expressly says, viz. that what he means by 'T6 CPVULO')'YW-
P.OYEry is the inferring of mental characteristics in men from the
presence in them of physical characteristics which in some other
kind or kinds of animal go constantly with those mental character-
istics. This is most plainly involved in A.'s statement in b32 -8
of the conditions on which the possibility of 'TO CPvu,O')'ywP.OYf:ry
depends. Our inference that this is what he means by 'T6 cpvuw-
YVWP.OYELy is confirmed by certain passages in the Physiognomonica,
which, though not by A., is probably Peripatetic in origin and
serves to throw light on his meaning. The following passages are
significant: 805&18 ol p.~ O~Y 1TPO')'EYEV7JP.£YO' CPVU'O')'VWp.oYES Ka'Ta 'TpErS
0,
'Tp61TOVS E1TEXElP7Juav cpvu,O')'YWP.OYE'iY, EKafTTOS Ka8' Eva.
P.Ey yap £K
'TWY YEYWY 'TWY ~ c{JWY cpvu,oyvwP.OYOVU" n8£p.EYO' Ka8' EKafTTOY Y£YOS
ElS6s n 'c{Jov Ka~ Sufyol.(lY oia E1TE'Ta, 'Tip 'TowVr<p uwp.an, El'Ta 'T6Y
oP.OWY 'ToVr<p 'T6 uwp.a EX0Y'Ta Ka~ ~Y ifroxYIY op.olay tI1TEM.p.{3ayoy
(so Wachsmuth). 807&29 OV yap o'\oy 'T6 yIyos 'TWY ay8pwW'wY
cpvuwyywP.OYOVP.EY, aM& nya 'TWY 0, 'Tip Y£YH. 8IO a I I oua SE 1Tp6s 'T6
cpvu,oyywp.ovijua, UVyc.OE'iy app.M'TH a 1T 6 'T W Y 'c{J W Y, £Y 'Tfi 'TWY
CT7Jp.Elwy £K'\Oyfi p7J81/UETa'.
The preliminary assumptions A. makes are (I) that natural (as5 0 2
COMMENTARY
opposed to acquired) mental phenomena (7Ta8~JLa'Ta, KtV1)O"€Lo;,
mi.8TJ)' such as fits of anger or desire, and the tendencies to them,
such as bravery or generosity, are accompanied by a physical
alteration or characteristic (7ob7-n); (2) that there is a one-one
correspondence between each such 7TC180s and its bodily accompani-
ment (ib. 12); (3) that we can find (by an induction by simple
enumeration) the special mi.80s and the special O"TJJL(/iov of each
animal species (ib. 12-13). Now, though these have been described
as rSta to the species they characterize, this does not prevent
their being found in certain individuals of other species, and in
particular of the human species; and, the correspondence of O"TJ-
JL£iov to 7TC1.80S being assumed to be a one-one correspondence,
we shall be entitled to infer the presence of the m5.80s in any human
being in whom we find the O"TJJL£iov (ib. 13-26). Let 51 be a species
of which all the members (or all but exceptional members) have
the mental characteristic M l , and the physical characteristic PI.
Not only can we, if we are satisfied that PI is the sign of M l ,
infer that any individual of another species 52 (say the human)
that has PI has M l . \Ve can also reason back from the species
only some of whose members have PI to that all of whose mem-
bers have it. If the members of 51 have two mental characteristics
Ml and M 2 , and two physical characteristics PI and P 2 , how are
we to know which P is the sign of which M? We can do so if we
find that some members of 52 have (for instance) Ml but not
M t , and PI but not P 2 (ib. 26-3 2 ).
Thus the possibility of inferring the mental characteristics of
men from the presence of physical characteristics which are in
some other species uniformly associated with those character-
istics depends on our having a first-figure syllogism in which the
major premiss is simply convertible and the minor is not, e.g.
All animals with big extremities are bra.ve, All lions have big
extremities, Therefore all lions are brave. The major premiss
must be simply convertible, or else we should not have any
physical symptom the absence of which would surely indicate
lack of courage; the minor premiss must not be simply convertible,
or else we should have nothing from whose presence in men we
could infer their courage (ib. 32--8).
19. on o?ou ... [1nl.9os]. If we read 77(180s, we must suppose that
this word, which in b lO, 13, 15, 24 stands for a mental characteristic
(in contrast with O"TJJL£'iov) , here stands for a physical one. It would
be pointiess to bring in a reference to the mental characteristic
here, where A. is only trying to explain the sense in which the
O"TJJL£iov can be called ,Stov. There is no trace of 7TCi.80s in P.
