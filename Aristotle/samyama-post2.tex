POSTERIOR ANALYTICS
BOOK II
CHAPTER 1
There are four types of inquiry
89bZ3. The objects of inquiry are just as many as the objects
of knowledge; they are (1) the that, (2) the why, (3) whether the
thing exists, (4) what it is. The question whether a thing is this
or that (e.g. whether the sun does or does not suffer eclipse)
comes under (1), as is shown by the facts that we cease from
inquiring when we find that the sun does suffer eclipse, and do not
begin to inquire if we already know that it does. When we know
(1) the that, we seek (2) the why.
31. Sometimes, on the other hand, we ask (3) whether the
thing (e.g. a centaur, or a god) is, simply, not is thus or thus
qualified, and when we know that it is, inquire (4) what it is.
In the first Book A. has considered demonstration both as
proving the existence of certain facts and as giving the reason
for them. In the second Book he is,to consider demonstration as
leading up to definition. By way of connecting the subjects of
the two Books, he now starts with an enumeration of all possible
subjects of inquiry, naming first the two that have been con-
sidered in the first Book-the question 'why' and the preliminary
question of the 'that'-and going on to the two to be considered
in the second Book, the question what a certain thing is, with the
preliminary question whether the thing exists.
It is probable that A. meant primarily by the four phrases "TC)
on, TO S,on, £t £un, Tt EaT, the following four questions: (1)
whether a certain subject has a certain attribute, (2) why it has
1985
Rr610
COMMENTARY
it, (3) whether a certain subject exists, (4) what it is: and the
examples given in this chapter conform to these distinctions.
The typical example of (I) is 'whether the sun suffers eclipse',
of (2) 'why it does', of (3) 'whether a god exists', of (4) 'what
a god is'. But the phrases cm £crn and £l £(T'n do not in themselves
suggest the distinction between the possession of an attribute by
a subject and the existence of a subject, and the phrase Tt £cm does
not suggest that only the definition of a subject is in question.
Naturally enough, then, the distinctions become blurred in the
next chapter. In 89b38-90"S the distinction formerly conveyed by
the phrases on £CT'HV and £l £CT7W is conveyed by the phrases £l
£CT'HV br, "L€pov!: (= £l €CT'H Tt, 9°"3), whether a subject is qualified
in this or that particular way, i.e. whether it has a certain
attribute) and £l £unv Q.7TAW!: (whether a certain subject exists at
all). Further, even £l £unv Q.7TAw!: comes to be used so widely in
90"4-5 as to include the inquiry whether night, which is surely
an attribute rather than a subject (i.e. a substance), exists.
Again, the question Tt €un, which was originally limited to the
problem of defining subjects, is extended to include the problem
of defining such an attribute as eclipse (9°015). It has always to
be remembered that A. is making his vocabulary as he goes, and
has not succeeded in making it as clear-cut as might be wisHed.
8c) bZ s. t:L<; a.pL9jlov 9EVTIE<;. This curious phrase should probably
be taken (as it is by P., E., Zabarella, and Pacius) to mean 'intro-
ducing a plurality of terms', i.e. ascribing a particular attribute
to the subject, as against a proposition which says that a certain
subject exists. Waitz takes the phrase to mean 'stating more than
one possibility'. But that is not part of the essence of the inquiry
as to the on.
CHAPTER 2
They are all concerned with a middle term
When we inquire whether a thing is thus or thus
qualified, or whether a thing exists, we are asking whether there
is a middle term; when we know that a thing is thus or thus
qualified, or that a thing exists, i.e. the answer to the particular
or to the general question, and go on to ask why it is thus or
thus qualified, or what it is, we are asking what the middle term
is. By the 'that' or particular question I mean a question like
'does the moon suffer eclipse?', i.e. 'is it qualified in a particular
way?'; by the general question a question like 'does the moon
exist?' or 'does night exist?'
90'5. Thus in all inquiries we are asking whether there is
89b36.IT.
I.
Bgh25
6II
a middle term, or what it is; for the cause is the middle term, and
we are always seeking the cause. 'Does the moon suffer eclipse?'
means 'Is there a cause of this?' If we know there is, we ask
what it is. For the cause of the existence of a thing's substantial
nature, or of an intrinsic or incidental property of it, is the
middle term.
14. In all such cases the what and the why are the same. What
is eclipse? Privation of light from the moon by the interposition
of the earth. Why does eclipse happen? Because the light fails
when the earth is interposed. What is harmony? An arithmetical
ratio between a high and a low note. Why does the high note
harmonize with the low? Because the ratio between them is
expressible in number!;.
24. That our search is for the middle term is shown by cases
in which the middle term is perceptible. If we have not perceived
it we inquire whether the fact (e.g. eclipse) exists; if we were
on the moon, we should not have inquired either whether or why
eclipse exists; it would have been at once obvious. For from
perceiving the particular facts, that the earth was interposed and
that the moon was eclipsed, one would have grasped the universal
connexion.
31. Thus to know the what is the same as knowing the why,
i.e. why a thing exists, or why it has a certain attribute.
There are two perplexing statements in this chapter. One is
the statement that when we are asking whether a certain con-
nexion of subject and attribute exists (Tt; on) or whether a certain
thing exists (El £un), we are inquiring whether there is a I'..£uov,
and that this inquiry precedes the inquiry what the I'.iuov is
(89b37-90aI). The other is the statement that in all four of the
inquiries enumerated in 89b24-S we are asking either whether
there is a Iduov or what it is (90"S--6). By p..luov A. means not any
and every term that might serve to establish a conclusion (as
a symptom may establish the existence of that of which it is
a symptom), but the actual ground in reality of the fact to be
explained (9°"6-7). His meaning therefore must be that, since
everything that exists must have a cause, to inquire whether
a certain connexion of subject and attribute, or a certain thing,
exists is implicitly to inquire whether something that is its cause
exists. This is intelligible enough when the inquiry is whether,
or why, a certain complex of subject and attribute, or-of subject
and event, exists (on £un or 8,d Tt £un). It is also intelligible
when the inquiry is whether a certain attribute (or event) existsCOMMENTARY
(t:l £an applied to an attribute or event) or what it is (T{ Jan
applied to an attribute or event). For since an attribute can
exist only in a subject, t:l £an here reduces itself to on £an,
and A. holds that T{ JaT' reduces itself to Sui T{ £an, i.e. that the
proper definition of an attribute is a causal definition explaining
why the attribute inheres in its subject. But how can t:l £an or
T{ Jan applied to a substance be supposed to be concerned with
a middle term? A substance does not inhere in anything; there
are no two terms between which a middle term is to be found.
A. gives no example of what he means by the piaov in such a
case, and in this chapter the application of the questions t:l £an
and T{ Jan to substances is overshadowed by its application to
attributes and events, which is amply illustrated (90&15-23). He
does not seem to have thought out the implications of his view
where it is the t:l £an or the T{ Jan of a substance that is in ques-
tion, and the only clue we have to his meaning is his statement
that by piaov he means ainov. As regards the t:l £an of substances,
then, he will be saying that since they, no less than attributes,
must have a sufficient ground of their being, to inquire whether
a certain substance exists is by implication to inquire whether
something that is its cause exists. As regards the T{ Jan of sub-
stances he will be saying that to inquire what a certain substance
is, is to inquire what its cause is; i.e. that its definition, no less
than that of an attribute, should be causal, that a substance
should be defined by reference either to a final or to an efficient
cause. This is the doctrine laid down in Met. 1041a26-Kat S,d
T{ TaS{, ofov 7TAMJo, Kat AlBo" olK{a JaT{v; ,pavt:pov TO{VVV on {7)Tt:' TO
ainov' TOlYrO S' JUTt TO T{ ~v t:rva" WS- t:l7Tt:'V AOY'KWS-, (; J7T' Jv{wv /L€V
Ja7L T{VOS- lvt:Ka, ofov Laws- br' olK{as- 7) KAlV7)S-, br' Jv{wv S~ T{ JK{V7)at:
TTPWTOV' aZnov yap Kat TOlYrO' (cf. I041b4-9, 1043814-21). But it
cannot be said that A. remains faithful to this view; the definitions
he offers of substances far more often proceed per genus et differ en-
tiam without any mention of a cause.
The general upshot is that the questions t:l £U7L and T{ Jan,
which in ch. I referred to substances, have in ch. 2 come to refer
so much more to attributes and events that the former reference
has almost receded from A.'s mind, though traces of it still
remam.
89 b39. 11 TO OT! •.• (1'11').(;)0;. TO J7Tt l"povs- further characterizes
TO on, making it plain that this refers to the question whether
a certain subject has a certain particular attribute (e.g. whether
the moon suffers eclipse, 9°"3). TO am\ws- further characterizes
TO t:l £anv, indicating that this refers to the question whether
6I261 3
a certain subject (e.g. the moon, as) or a certain attribute (e.g. the
deprivation of light which we call night, ib.) exists at all.
90'3-4' Et ya.p .•• TL, 'whether the subject has or has not some
particular attribute'.
5. ;; vu~. The mention of night here, where we should expect
only substances to be in A.'s mind, is surprising, but the words
are sufficiently vouched for by P. 338. 13 and E. 20. 1I-18. Cf.
the introductory n. on ch. 1.
10. ;; TOU Ill] a.1TAw~. Both sense and grammar require us to
read TOO for the TO of the MSS., as Bonitz points out in Ar.
Stud. iv. 28 n.
11. ;; KClTIl aUIlJ3IEJ3"Ko~. This can hardly refer to pure acci-
dents, for with these A. holds that science has nothing to do.
Zabarella is probably right in thinking that the reference is to
attributes which result from the operation of one thing on another,
while TWV Ka(j' aV-ro refers to attributes springing simply from the
essential nature of the thing that has them.
13-14. TO SE Tt ••• Il'" £K>"~'.p,V, an attribute of moon or sun;
laOTT)Ta dvwOrq-ra, alternative attributes of a pair of triangles;
€v piaif' ~ p.~, being in the centre of the universe or not (the ques-
tion discussed in De Caelo 293a1S-b1S), alternative attributes one
of which must belong to the earth.
18-23. TL EaTL • • • Myo~; The Pythagoreans had discovered
the dependence of consonance on the ratios between the lengths of
vibrating strings-that of the octave on the ratio 1: 2, of the fifth
on the ratio 2 : 3, of the fourth on the ratio 3: 4; see Zeller-Mondolfo,
ii. 454-5.
29-30. Kat yap ... €y€V€TO, 'and so, since it would have been
also clear that the moon is now in eclipse, the universal rule would
have become clear from the particular fact'. The yap clause is
anticipatory; cf. Denniston, Greek Particles, 69-70.
33. OTL Suo op9aL, that the subject (the triangle, cf. "13) has
angles equal to two right angles.
CHAPTER 3
There is nothing that can be both demonstrated and defined
90'35. We must now discuss how a definition is proved, and
how reduced to demonstration, what definition is and what things
are definable. First we state some difficulties. It may be asked
whether it is possible to know the same thing, in the same respect,
by definition and by demonstration.
b 3 . (A) <Not everything that can be demonstrated can beCOMMENTARY
defined.) (I) Definition is of the what, and the what is universal
and affirmative; but some syllogisms are negative and some are
particular.
7. (2) Not even all affirmative facts proved in the first figure
are objects of definition. The reason for this discrepancy is that
to know a demonstrable fact is to have a demonstration of it,
so that if demonstration of such facts is possible, there cannot be
also definition of them, since if there were, one could know the
fact by having the definition, without the demonstration.
13. (3) The point may be made by induction. We have never
come to recognize the existence of a property, whether intrinsic or
incidental, by defining it.
16. (4) Definition is the making known of an essence, but
such things are not essences.
18. (B) Can everything that can be defined be demonstrated?
(1) We may argue as before, that to know something that is
demonstrable is to have demonstration of it; but if everything
that is definable were demonstrable, we should by defining it
know it without demonstrating it.
24. (2) The starting-points of demonstration are definitions, and
there cannot be demonstration of the starting-points of demonstra-
tion; either there will be an infinite regress of starting-points or
the starting-points are definitions that are indemonstrable.
28. (C) Can some things be both defined and demonstrated?
No, for (I) definition is of essence; but the demonstrative sciences
assume the essence of their objects.
33. (2) Every demonstration proves something of something,
but in definition one thing is not predicated of another-neither
genus of differentia nor vice versa.
38. (3) What a thing is, and that a connexion of subject and
attribute exists, are different things; and different things demand
different demonstrations, unless one demonstration is a part of
the other (as the fact that the isosceles triangle has angles equal
to two right angles is part of the fact that every triangle has
this property) ; but these two things are not part and whole.
91a7. Thus not everything that is definable is demonstrable,
nor vice versa; nor is anything at all both definable and demon-
strable. Thus definition and demonstration are not the same,
nor is one a part of the other; for if they were, their objects would
be similarly related.
9O a 37. 8La.1TOPtlUa.VTES 1TPWTOV 1TEpt a.UTIllY. The fact that the
chapter (as also chs. 4-7) is aporematic implies that it is dialecti-615
11.3.9 0 "37-9 1 "9
cal, using sometimes arguments that A. could not have thought
really convincing.
bI. OI.KELOTa.TTJ TWV EXOI1EVIIIV )..oywv, 'most appropriate to the
discussions that are to follow', not 'to those that have preceded'.
For the meaning cf. Bonitz, Index, 306"48-58.
7-17. ElT a oUSE ••• oUuLaL. A.'s point here is that while
demonstration is of facts such as that every triangle has its angles
equal to two right angles, or in general that a certain subject has
a certain property, definition is of the essence of a subject. In
b I4 - I 6 it is assumed that both Ta Ka()' av-ro inrapxovTa and Ta
G1Jf£j3ej3'YJKC1Ta are objects of demonstration, so that the distinction
is not between properties and accidents, but (as in a rr ) between
properties following simply from the essential nature of their
subject and those that follow upon interaction between the subject
and something else; for accidents cannot be demonstrated.
10. TO ci1roSELKTOV, though rather poorly supported by MSS.
here, is confinned by b2I and is undoubtedly the right reading.
16. Ta. yE TOLa[iTa, i.e. Ta Ka()' alho l'nrapXOVTa Kat Ta G1Jf£j3ej3'YJ-
KOTa, such as that the angles of a triangle are equal to two right
angles (b8-{)).
19. TI SaL; I have accepted B's reading, as being more likely
to have been corrupted than Tl S'. For Tl Sal cf. Denniston, Greek
Particles, 262-4. The colloquial phrase is particularly appropriate
in a dialectical passage like the present one.
z5. SESELKTaL 1TpOTEpOV, in 72bI8-25 and 84a29-b2.
34-8. EV SE T~ bPLUI1~ ••• E1TI1TESOV. A. takes optUf£oS here as
being not a sentence such as av()pClJ7TOS Eun ~<pov Sl7TOliV, but simply
a phrase such as ~<pov S{7TOliV, put forward as the equivalent of
av()pw7ToS. In such a phrase the elements are not related by way
of assertion or denial, but by way of qualification or restriction
of the genus by the addition of the differentia.
9Ia8--c). OUTE 0)..1115 ••• EXELV. The MSS. have WaTE for oine.
But (I) we can hardly imagine A. to reason so badly as to say
'(a) not everything that is definable is demonstrable, (b) not
everything that is demonstrable is definable, therefore (c) nothing
is both definable and demonstrable'. And (2) in the course of the
chapter (b), (a), and (c) have been proved separately in 90b3-19,
19-27, 28-{)1"6, (c) not being deduced from (a) and (b). Therefore
we must read oine, which Pacius already reacl. Whether he had
any authority for the reading we do not know. Hayduck's
grounds for suspecting the whole sentence (Obs. Crit. in aliquot
locos A rist. 14-15) are insufficient.616
COMMENTARY
CHAPTER 4
It cannot be demonstrated that a certain phrase is the definition
of a certain term
9IaI2. We must now reconsider the question whether definition
can be demonstrated. Syllogism proves one term true of another
by means of a middle term; now a definition states what is both
(I) peculiar and (2) essential to that whose definition it is. But
then (I) the three terms must be reciprocally predicable of each
other. For if A is peculiar to C, A must be peculiar to B, and
B to C.
18. And (2) if A is essential to the whole of B, and B to the
whole of C, A must be essential to C; but unless we make both
assumptions the conclusion will not follow; i.e. if A is essential
to B but B is not essential to everything of which it is predicated.
Therefore both premises must express the essence of their subjects.
And so the essence of the subject will be expressed in the middle
term before it is expressed in the definition we are trying to prove.
26. In general, if we want to prove what man is, let C be man,
and A the proposed definition. If a conclusion is to follow, A
must be predicated of the whole of a middle term B, which will
itself express the essence of man, so that one is assuming what
one ought to prove.
33. We must concentrate our attention on the two premises,
and on direct connexions at that; for that is what best brings out
our point. Those who prove a definition by reliance on the
convertibility of two terms beg the question. If one claims that
soul is that which is the cause of its own life, and that this is a
self-moving number, one is necessarily begging the question in
saying that the soul is essentially a self-moving number, in the
sense of being identical with this.
bI. For if A is a consequent of Band B of C, it does not follow
that A is the essence of C (it may only be true of C) ; nor does this
follow if A is that of which B is a species, and is predicated of
all B. Every instance of being a man is an instance of being an
animal, as every man is an animal; but not so as to be identical
with it. Unless one takes both the premises as stating the
essence of their subjects, one cannot infer that the major term is
the essence of the minor; but if one does take them so, one has
already assumed what the definition of C is.
9IaI2. T aUTo. .... EV O~V • • • S~"lI'op"a811). This does not mean
that A. has come to the end of the aporematic part of his dis-61 7
cussion of definition; his positive treatment of the question begins
with ch. 8. What he says is 'so much for these doubts' ; there are
more to .come, in chs. 4-7.
13-14. Ko.8a.1TEP vuv ••• U1TE8ETO, i.e. in ch. 3.
16. To.UTo. S' cl.Vo.YKTj cl.YTLaTpEcJluv, 'terms so related must be
reciprocally predicable'. The phrase is rather vague, but A.'s
meaning is made clear by the reason given for the statement,
which follows in "16-18: 'Since the definitory formula is to be
proved to be peculiar to the term defined, all three terms used
in the syllogism must be coextensive. For, definition being a
universal affirmative statement, the proof of it must be in
Barbara: All B is A, All C is B, Therefore all C is A. Now if B
were wider than the extreme terms, which are ex hypothesi
coextensive, the major premiss would be untrue; and if it were
narrower than they are, the minor would be untrue. Therefore
it must be equal in extent to them.'
23. ~TJ Ko.8' oawv • . . EaTLY, 'but B is not included in the
essence of everything to which it belongs'. The phrase would be
easier if we supposed a second "TO B, after the comma, to have
fallen out.
24-5. EaTo.L apo. ••• EaTLV. The comma read by the editors
after "TOU r must be removed.
26. E1Tl TOU ~Eaou . . . dyo.L. bd, because it is at the stage
represented by the middle term (i.e. by the premiss which
predicates this of the minor) that we first find the "Tt £un (of the
minor), before we reach the conclusion.
30-1. TOUTO S' ... av8pw1To~. The reading is doubtful. All the
external evidence is in favour of "ToVrov, and "TOVTOV would natur-
ally refer to B; then the words would mean 'and there will be
another definitory formula intermediate between C and B' (as
B is, between C and A), 'and this new formula too will state the
essence of C (man),. I.e. A.'s argument will be intended to
show that an infinite regress is involved in the attempt to prove
a dt'finition. Then in "33-5 A. would go on to say 'but we should
study the matter in the case where there are but two premises,
and no prosyllogism'. But there are difficulties in this interpreta-
tion. (a) A. does not, on this interpretation, show that the
original middle term B must be a definition of C, which would be
the proper preliminary to showing that the new middle term
(say, D) must be a definition of C. (b) He gives no reason why 'C
is B' must be supported by a prosyllogism. (c) He uses none of
the phrases by which he usually points to an infinite regress (e.g.
Els TO a.7THPOI' {3aoLELTaL). He simply says that the proposed proof618
COMMENTARY
begs the question, and he points not to D and the further terms
of an infinite series in justification of the charge, but simply says
("31-Z) that in assuming All C is B (i.e. is definable as B) the
person who is trying to prove the definition of C as A is assuming
the correctness of another definition of C.
It seems probable, then, that there is no reference to an infinite
regress. In that case, 'TOO'TOU must refer to A, and the meaning
must be 'and there will be another definitory formula than A
intermediate between C and A (i.e. B), and this will state the
essence of man'. But, Ka'TC~ 'TOU B being the emphatic words in
the previous clause, it is practically certain that 'TOIJ'TOU would
necessarily refer to B and not to A. This being so, it is better to
adopt Bonitz's conjecture 'TOU'TO (Arist. Stud. iv. Z3), which is
read by one of the best MSS. of Anonymus.
31-2. KQL yap TO B ... QV9pw1Toe;. Bonitz (Arist. Stud. iv. 23)
is almost certainly right in reading EVTaL for €V'T'; cf. aZ4, z6, 30, b 9.
37- b I , otov E'( TLe; ••• QV. The definition of soul as o.pdJ/LoC;
mhoc; av'Tov KLVWV was put forward by Xenocrates (Plut. Mor.
10lZ D). A. refers to it in De An. 404bz9, 408b3z, without naming
its author.
b 3 . aAA' cl.ATJgee; ....... ovov. If we keep the reading of most of
the MSS. (o.'\,\' o.'\T)(}€c; 7}V €i7r€iv EVTaL /Lovov) , we must put EVTaL
in inverted commas and interpret the clause as meaning 0.'\'\'
o.'\T)(}€c; 7}V €im:iv 'ev'TaL 'T<fJ r'TO A' /L6vov, 'it was only true to predi-
cate A of C, not to assume their identity. But n (confirmed by
E. 6z. z5 o.'\'\a /LOVOV EV'TaL av'Tou o.'\T)(}wc; Ka'TT)yopoO/L€Vov) gives what
is probably the right reading. Of the emendations Mure's ap-
pears to be the best.
9-10. 1TPOTEPOV ~aTQL • • • B. The grammar of the sentence
is best corrected by treating 'TO B as a (correct) gloss.
CHAPTER 5
It cannot be shown by division that a certain phrase is the
definition of a certain term
91b12. Nor does the method of definition by division syllogize.
The conclusion nowhere follows from the premises, any more than
does that of an induction. For (r) we must not put the conclusion
as a question nor must it arise by mere concession; it must arise
from the premises, even if the respondent does not admit it.
Is man an animal or a lifeless thing? The definer assumes that
man is an animal; he has not proved it. Again, every animal is61 9
either terrestrial or aquatic; he assumes that man is terrestrial.
(2) He assumes that man is the whole thus produced, terrestrial
animal; it makes no difference whether the stages be many or
few. (Indeed, those who use the method do not prove by
syllogism even what might be proved.) For the whole fonnula
proposed may be true of man but not indicate his essence. (3)
There is no guarantee against adding or omitting something or
passing over some element in the being of the thing defined.
~8. These defects are disregarded; but they may be obviated
by taking none but elements in the essence, maintaining con-
secutiveness in division, and omitting nothing. This result is
necessarily secured if nothing is omitted in the division; for then
we reach without more ado a class needing no further division.
3~. But there is no syllogism in this; if this process gives
knowledge, it gives it in another way, just as induction does.
For as, in the case of conclusions reached without the middle
terms, if the reason er says 'this being so, this follows', one can
ask 'why?', so too here we can say 'why?' at the addition of
each fresh detenninant. The definer can say, and (as he thinks)
show by his division, that every animal is 'either mortal or
immortal'. But this whole phrase is not a definition, and even
if it were proved by the process of division, definition is still not
a conclusion of syllogism.
9IbIZ-13. 'AAAa. I-LTJv,
ElpT]TaL. The Platonic method of
definition by division (illustrated in the Sophistes and Politicus)
has already been discussed 'in that part of our analysis of argu-
ment which concerns the figures of syllogism', i.e. in A n. Pr.
i.31. The value of division as a preliminary to definition is brought
out in 96b27-97b6.
16. 0~8E T~ 800vaL EtvaL, 'nor must it depend on the respon-
dent's conceding it'.
18. Eh' EAa~E ~~ov, i.e. then, when the respondent answers
'animal', the questioner assumes that man is an animal.
~o-I. KaL TO EtvaL ••• TOOTO. Bekker's and Waitz's comma be-
fore TO oAov is better away; TO oAov is the whole formed by ~cfov
Tl'E~6v (cf. TO Tl'av, b 2S ). The point made here is a fresh one (made
more clearly in b 24 -6). Even if the assumption that man is an
animal and is two-footed is true, what guarantee have we that
man is just this complex, 'two-footed animal', i.e. that this is
his essence?
~3-4' cl.CTUAAOYUTTOS I-LEV o~v ••• CTUAAOYLae~VaL. The process
of division is liable not only to assume that the subject hasCOMMENTARY
attributes that it cannot be proved to have, but also to assume
that it has attributes that it could be proved to have.
!-'£v ouv 'nay rather', introducing a stronger point against the
method A. is criticizing than that introduced before. 'The
speaker objects to his own words, virtually carrying on a dialogue
with himself' (Denniston, The Greek Particles, 478).
26. Il-" Il-£VTOL •.• STJ).ouv, 'the definitory formula may not
succeed in showing what the thing is, or what it was to be the
thing'; no real distinction is meant to be drawn between the two
phrases.
26---]. £TL TL KwM€L ••• oU<TLa .. ; The process of division may (r)
introduce attributes that are properties or accidents of the subject,
not part of its essence. It may (2) fail to state the final differentia
of the subject. Or (3) it may pass over an intermediate differentia.
E.g. substance is divisible into animate and inanimate, and
animate substance into rational and irrational. If then we define
man as rational substance, we shall have omitted an intennediate
differentia.
30. ahouflEvov TO 1I"pc~hov, 'postulating the next differentia at
each stage'.
30-2. TOUTO S' ... dvaL. Waitz omits £l ... J)')'£l'7TH (as well
as the second TOtrrO 8' avuYKuLOV), on the ground that these words
are a mere repetition of the previous sentence; but there seems
to be just enough of novelty in the clause to make it not pointless.
On the other hand, the repetition of TOtrrO 8' avaYKaLov is highly
suspicious; it may so easily have arisen from the words having
been first omitted, then inserted in the margin, and then drawn
into the text at two different points. Besides, they would have
to mean two quite different things. The first TOtrrO 8' avaYKaiov
would mean 'and this result is necessarily achieved', the second
'and this condition must be fulfilled'. The second TOtrrO S'
avaYKaiov might be saved if we read (with A and d) TOtrrO 8'
avaYKaiov aTo!-'ov ij87) £lva~, 'and the result so produced must
necessarily be a formula needing no further differentiation'. But
the balance of probability is in favour of the reading I have
adopted.
32. aTo/J-ov yap iiSTJ SEL EtvaL. The sense would not be seriously
altered if we adopted B's original reading Ei8H (for ij87)) ; but the
idiomatic ij87J is rather the more likely. aTo!-'ov must be taken in
a special sense. The correct definitory formula will not be indi-
visible, unless the tenn to be defined happens to be an infima
species; but it will be unsuitable for further division, since a
further division would only yield too narrow a fonnula.
62011. 5.
gl b 26-9 28 4
621
92&3-4. 0 S€ TOLOUTO~ ••• opLal1o~. Bonitz's conjecture of
(7lJ'\'\0I'LUfLO, for opwp.O, (Arist. Stud. iv. 27) gives a good sense, but
does not seem to be required, and has no support in the MS.
evidence.
CHAPTER 6
Attempts to prove the definition of a term by assuming the definition
either of definition or of the contrary term beg the question
9Za6. Is it possible to demonstrate the definition on the basis
of an hypothesis, assuming that the definition is the complex
composed of the elements in the essence and peculiar to the
subject, and going on to say 'these are the only elements in
the essence, and the complex composed by them is peculiar to
the subject'? For then it seems to follow that this is the essence
of the subject.
9. No; for (I) here again the essence has been assumed, since
proof must be through a middle term. (2) As we do not in a
syllogism assume as a premiss the definition of syllogism (since
the premises must be related as whole and part), so the definition
of definition must not be assumed in the syllogism which is to
prove a definition. These assumptions must lie outside the pre-
misses. To anyone who doubts whether we have effected a
syllogism we must say 'yes, that is what a syllogism is' ; and to
anyone who says we have not proved a definition we must say
'yes; that is what definition meant'. Hence we must have
already syllogized without including in our premises a definition
of syllogism or of definition.
zoo Again, suppose that one reasons from a hypothesis. E.g.
'To be evil is to be divisible; for a thing to be contrary is to be
contrary to its contrary; good is contrary to evil, and the indivisible
to the divisible. Therefore to be good is to be indivisible.' Here too
one assumes the essence in trying to prove it. 'But not the same
essence', you say. Granted, but that does not remove the objec-
tion. No doubt in demonstration, too, we assume one thing to
be predicable of another thing, but the term we assume to be true
of the minor is not the major, nor identical in definition and
correlative to it.
:1.7. Both to one who tries to prove a definition by division and
to one who reasons in the way just described, we put the same
difficulty: Why should man be 'two-footed terrestrial animal' and
not animal and terrestrial and two-footed? The premises do not
show that the formula is a unity; the characteristics might simplyCOMMENTARY
belong to the same subject just as the same man may be musical
and grammatical.
622
9Z86-c). 'A 'A.. 'A.. , clpa
EKELV'l,l'
In this proposed proof of a
definition the assumption is first laid down, as a major premiss,
that the definition of a given subject must (1) be composed of
the elements in its essence, and (2) be peculiar to the subject.
It is then stated, as a minor premiss, that (I) such-and-such
characteristics alone are elements in the essence, and (2) the whole
so constituted is peculiar to the subject. Then it is inferred that
the whole in question is the definition of the subject. (The method
of proof is that which A. himself puts forward in Top. 15387-22
as the method of proving a definition; and that which he criticizes
in 320-33 is that which he puts forward in 153"24-b24; Maier
(2 b. 78 n. 3) infers that the present chapter must be later than
that part of the Topics. This is very likely true, but Cherniss
(Aristotle's Criticism of Plato and the Academy, i. 34 n. 28) shows
that the inference is unsound; the Topics puts these methods
forward not as methods of demonstrating a definition, but as
dialectical arguments by which an opponent may be induced
to accept one.)
This analysis shows that Pacius is right in reading ,01OV after
Janv in 88. Cf. the application of '01OS' to the definition in 91 "IS,
Top. IOIbI9-23, 140833-4.
9-19. 11 'lT6.'A..LV •• ,TL. On this proposed proof A. makes two
criticisms: (1) (89-10) that the proof really begs the question that
the proposed complex of elements is the definition of the subject,
whereas it ought to prove this by a middle term. It begs the
question in the minor premiss; for if 'definition' just means
'formula composed of elements in the essence, and peculiar to
the subject' (which is what the major premiss says), then when
we say in the minor premiss' ABC is the formula composed of
elements in the essence of the subject and peculiar to it', we are
begging the question that A BC is the definition of the subject.
(2) ("II-I9) that just as the definition of syllogism is not the major
premiss of any particular syllogism, the definition of definition
should not be made the major premiss of any syllogism aimed at
establishing a definition. He is making a similar point to that
which he makes when he insists that neither of the most general
axioms-the laws of contradiction and of excluded middle-
which are presupposed by all syllogisms, should be made the
major premiss of any particular syllogism (77"10-12, 88836-b3).
He is drawing in fact the very important distinction between premises
 from which we reason and principles according to
which we reason.
9. lTa.).lV, because A. has made the same point in chs. 4 and 5
passim.
II-19. ETl ~I71TEP • • • n. The premises of a syllogism should
be related as whole and part, i.e. (in the first figure, the only
perfect figure) the major premiss should state a rule and the
minor premiss bring a particular type of case under this rule, the
subject of the major premiss being also the predicate of the minor.
But if the major premiss states the general nature of syllogism
and the minor states particular facts, the minor is not related
to the major as part to whole, since it has no common term with
it. The facts on which the conclusion is based will be all contained
in the minor premiss, and the major will be otiose. The true
place of the definition of syllogism is not among the premises
of a particular syllogism, nor that of the definition of definition
among the premises by which a particular definition is proved
(if it can be proved) ; but when we have syllogized and someone
doubts whether we have, we may say 'yes; that is what a syllo-
gism is', and when we have proved a definition, and this is
challenged, we may say 'yes; that is what definition is'-but we
must first have syllogized, or (in particular) proved our definition,
before we appeal to the definition of syllogism or of definition.
14-16. K(J.L lTPOS TOV cl.p.cJ!la(3T}TouvT(J. ... aU).).0Ylap.os. Bonitz
(Arist. Stud. iv. 29) points out that, with the received punctuation
(El avAAEAoytaTat ~ fL~ TOUTO, a7TavTav), TOUTO is not in its idiomatic
position.
18. 11 TO Tl ,;V EtV(J.l. The argument requires the reading TO,
not TOU, and this is confirmed by T. 47.17-19, P. 356. 4--6, E. 85. 1I.
700-']. Kliv E~ iJ1To8EaEWS . . . civnaTpEcJ!El. The use of the
T07TOS" a7T<~ TOU ivavTlov is discussed in Top. I 53B26-bZ4. It was one
of the grounds on which Eudoxus based his identification of
the good with pleasure (Eth. Nic. 1172b18-20). The description of
evil as divisible and of good as indivisible, also, is Academic;
it was one of Speusippus' grounds for denying that pleasure is
good. He described the good as 'taov, and pleasure (and pain) as
fL£r,OV Kat iAaTTov (Eth. Nic. II73·15-17 Myovat O£ TO fL£V ayaOov
wp[aOat T~V O£ ~oo~v aOptaTOv dvat OTL O'X£TaL TO fLaAAov KaL TO
7]TTOV, II53 b4--6 ws" yap ~7T£vam7Tos" .rAv£v, ov avfL{3a[vEt ~ Avats",
Wa7TEp TO fL£Z'ov T0 iAaTTOVt KaL T0 'taCfJ ivavT{ov· OV yap ClV tfoalTJ
07T£P KaKOV TL dvat T~V ~ooV7}v), and identified the taov with the
aOLalp£Tov (aaXtaTov yap ad Kat £vo£tO£S" TO taov, frag. 4. 53, ed.
Lang) , and the aoptaTOv (i.e. the fL£Z'oV Kat .rAaTTov) with theCOMMENTARY
imperfect (Met. 1092313). On this whole question cf. Cherniss,
Ar.'s Cr1'ticism of Plato and the Academy, i. 36-8.
ZI. TO S' EVa.VT(~ •.• EtVa.L, 'and to be one of two contraries
is to be the contrary of the other'. Bonitz's emendations (Arist.
Stud. i. 8 n. 2, iv. 23-4) are required by the argument.
z4-7. Ka.t yap ••• o.VTLO'TpECPEL, 'for here too (cf. 39 n.) he
assumes the definition in his proof; but he assumes it in order to
prove the definition. You say "Yes, but a different definition".
I reply, "Granted, but that does not remove the objection, for
in demonstration also one assumes indeed that this is true of
that, but the term one assumes to be true of the minor term is
not the very term one is proving to be true of it, nor a term which
has the same definition as this, i.e. which is correlative with it".'
In "25 Bekker and Waitz have £T£pOV jLEVTOt £CTTW, but the
proper punctuation is already found in Pacius. £CTTW is the
idiomatic way of saying 'granted'; cf. Top. 176323 aTTOKptTEOV O·
lTT~ jL£V TWV OOKOWrWV TO 'ECTTW' AEyoVTa.
The point of A.'s answer comes in Kat (sc. Ka~ (}; for the grammar
cf. H.A 494"17, Part. An. 694"7, Met. 990"4, Pol. 1317"4) avn-
CTTPErPK Good and evil are correlative, and in assuming the
definition of evil one is really assuming the definition of good.
z7-33' 1rpOS o.~OTEpOUS ••• ypa.f1f1a.TLKOS. A.'s charge is that
the processes of definition he is attacking, though they can build
up a complex of attributes each of which is true of the subject,
cannot show that these form a real unity which is the very
essence of the subject; the complex may be only a series of
accidentally associated attributes (as 'grammatical' and 'musical'
are when both are found in a single man). The difficulty is that
which A. points out at length in M et. Z. 12 and attempts to solve
in H. 6 by arguing that the genus is the potentiality of which the
species are the actualizations. It is clear how the difficulty
applies to definition by division; it is not so clear how it applies
to definitions by hypothesis such as have been considered in
320-7. But the answer becomes clear if we look at Top. 153"23-
b 24 , where A. describes a method of discovering the genus and the
successive differentiae of a term by studying those of its contrary.
30. t~ov 1rEtov .•• S(1rouv. T. 47. 9, P. 357. 24, and E. 87· 34
preserve the proper order ~tPov TT£~OV otTToliv---working from
general to particular (cf. Top. 103327, 133"3, b8). In the final
clause again, where the MSS. read ~tP0v Ka~ TT£~6v, and Bonitz
(Arist. Stud. iv. 32-3) reads ~tP0v StTTOliV Ka~ TT£~6v, P. 357. 22 and
E. 88. 1 seem to have the proper reading ~tP0v Ka~ TT£~OV Ka~
SlTTOliV .H. 6. 92"21-3°
CHAPTER 7
Neither definition and syllogism nor their objects are the same;
definition proves nothing; knowledge of essence cannot be got either
by definition or by demonstration
9:Z"34. How then is one who defines to show the essence? (I)
He will not prove it as following from admitted facts (for that
would be demonstration), nor as one proves a general conclusion
by induction from particulars; for induction proves a connexion
or disconnexion of subject and attribute, not a definition. What
way is left? Obviously he will not prove the definition by appeal
to sense-perception.
b4 • (2) How can he prove the essence? He who knows what
a thing is must know that it is; for no one knows what that which
is not is (one may know what a phrase, or a word like 'goat-deer',
means, but one cannot know what a goat-deer is). But (a) if he
is to prove what a thing is and that it is, how can he do so by
the same argument? For definition proves one thtng, and so does
demonstration; but what man is and that man is are two things.
u. (b) We maintain that any connexion of a subject with an
attribute must be proved by demonstration, unless the attribute
is the essence of the subject; and to be is not the essence of any-
thing, being not being a genus. Therefore it must be demonstra-
tion that shows that a thing is. The sciences actually do this; the
geometer assumes what triangle means, but proves that it exists.
What then will the person who defines be showing, except what
the triangle is? Then while knowing by definition what it is, he
will not know that it is; which is impossible.
19. (c) It is clear, if we consider the methods of definition now
in use, that those who define do not prove existence. Even if
there is a line equidistant from the centre, why does that which has
been thus defined exist? and why is this the circle? One might
just as well call it the definition of mountain-copper. For
definitions do not show either that the thing mentioned in the
definitory formula can exist, or that it is that of which they claim
to be definitions; it is always possible to ask why.
:z6. If then definition must be either of what a thing is or of
what a word means, and if it is not the former, it must be simply
a phrase meaning the same as a word. But that is paradoxical;
for (a) there would then be definitions of things that are not
essences nor even realities; (b) all phrases would be definitions;
for to any phrase you could assign a name; we should all be
408.
5 sCOMMENTARY
626
talking definitions, and the Iliad would be a definition; (c) no
demonstration can prove that this word means this; and therefore
definitions cannot show this.
35. Thus (I) definition and syllogism are not the same; (2)
their objects are not the same; (3) definition proves nothing;
(4) essence cannot be known either by definition or by demons-
tration.
This is a dialectical chapter, written by A. apparently to clear
his own mind on a question the answer to which was not yet clear
to him. The chapter begins with various arguments to show that
a definition cannot be proved. (I) (9213S-b3) A person aiming at
establishing a definition uses neither deduction nor induction,
which A. here as elsewhere (A n. Pr. 68br3-r4, E.N. II39b26--8)
takes to be the only methods of proof. (2) One who knows what
a thing is must know that it exists. But (a) (b 4 - rr ) definition
has a single task, and it is its business to show what things are,
and therefore not its business to show that things exist. (b)
(b r2- 1 8) To show that things exist is the business of demonstra-
tion, and therefore not of definition. (c) (b I9- 2S ) It can be seen
by an induction from the modes of definition actually in use that
they do not prove the existence of anything corresponding to
the definitory formula, nor that the latter is identical with the
thing to be defined.
Concluding from these arguments that definition cannot prove
the existence of anything, A. now infers (b26-34) that it must
simply declare the meaning of a word, and points out that this
interpretation of it is equally open to objection. Finally, he sums
up the results of his consideration of definition up to this point.
92b3S-:6 OVT£ opLafLos . . . opLafLos refers to ch. 3, b37 -8 TTPOS S~
TOm-OLS ... yvwvaL to chs. 4-7.
9Zb7. Tpay~Aa~os. cf. An. Pr. 49"23 n.
8-c). a.AAa fLt,v . . . SEL~EL; Waitz reads ilia fL-fJ El SE {gEL T{
~crTL, KaL OTL EaTL; Kat TTWS Tip am-ip AOYIf S£~EL; His fL~ is a mis-
print. In reading KaL TTWS he is following the strongest MS.
tradition. This reading involves him in putting a comma before
Kat fYrL EaTL and in treating this as a question. But in the absence
of apa it is difficult to treat it as a question; and Bekker's reading,
which I have followed, has very fair evidence behind it.
12-15. EtTa Kat ••• ECTTLV. The sense requires us to read 0 TL
EaTLV, not OTL EaTLV. 'Everything that a thing is (Le. its possession
of all the attributes it has) except its essence is shown by demon-
stration. Now existence is not the essence of anything (beingnot being a genus). Therefore there must be demonstration that
a thing exists.' For OV yap ylvo, TO QV cf. M et. 998b22-7.
16. ()TL 8' EUTI, 8ELKVUULY. Mure remarks that 'triangle is for
the geometer naturally a subject and not an attribute; and in
that case on S' £un should mean not "that it exists", but "that
it has some attribute", e.g. equality to two right angles. It is
tempting to read £aTt Tt.' But that would destroy A.'s argument,
which is about existential propositions alln is to the effect that
since it is the business of demonstration to prove existence, it
cannot be the business of definition to do so. A.'s present way
of speaking of Tptywvov as one of the attributes whose existence
geometry proves, not one of the subjects whose existence it
assumes, agrees with what he says in 71814 and what his language
suggests in 76835 and in 93b31-2.
17. TL oov ••• TPLYWVOV; The vulgate reading Tt oov S£t~H <>
<>Pt~ofL£vo, Tt £aTtV; ~ TO Tptywvov; gives no good sense. P. 36I.
18-20 <> yovv <>Pt~OfL£VO' Kat TOV <>ptUfLOV a1TOOtOO1k Tt apa S£t~H; ~
1TavTw, 1Tap{UrYJUt T{ £aTt Tp{yWVOV KaOo Tp{yWVOV, E. 98. 13-14
<> O1Jv OptU/LO, OHKVV, TO Tp{yWVOV T{ AOt1TOV od~H ~ Tt £UTt; and An.
559. 24-5 Tt OOV O£{KVVUtV 0 OPt~ofL£vo, Kat TO Tt £anv U1TOOtS01),
nvo,; 1) TO T{ £unv £K£'iVO (; <>p{~£Tat; point to the reading I have
adopted. Tt . •. ij = Tt o.,uo ij, cL PI. Cri. 53 e and K iihner,
Cr. Cramm. ii. 2.304 n. 4.
21. 6.~~a 8La TL E<TTL TO apLugev; It is necessary to accent £un,
if this clause is to mean anything different from that which
immediately follows. The first clause answers to b 23 - 4 oun .
OPOt, the second to b 24 oun . . . OptUfLol.
24. ci.~~' 6.eL E~e<TTL ~EYELY TO 8La TL, as in b21 or in 9Ib37-9.
28-9. 1TPWTOV flEV yap ••• E~TJ. OVUtWV cannot here mean 'sub-
stances', for there would be nothing paradoxical in saying that
things that are not substances can be defined. It must mean
'definable essences'.
32-3. ETI ou8EflLa ••• av. The best supported reading omits
a1ToOH~t,. But the ellipse seems impossible here; a1TooHg" or
£1TLUTTJf.LTJ is needed to balance optUf.L0{ (b 34). The reading of d
U7TOO£~t, £[£V av points to the original reading having been
a1Toongt, a1Tood~H£V av. In most of the MSS. a1ToOH~t, disappeared
by haplography, and in some £muTTJf.LTJ was inserted to take its
place.
34. ou8' OL apLuflot ••• 1Tpou8TJ~oiJuLY, 'and so, by analogy,
definitions do not, in addition to telling us the nature of a thing,
prove that a word means so-and-so'.628
COMMENTARY
CHAPTER 8
The essmce of a thing that has a cause distinct from itself cannot be
demonstrated, but can become known by the help of demonstration
93"1. We must reconsider the questions what definition is and
whether there can be demonstration and definition of the essence.
To know what a thing is, is to know the cause of its being; the
reason is that there is a cause, either identical with the thing or
different from it, and if different, either demonstrable or indemon-
strable; so if it is different and demonstrable, it must be a middle
term and the proof must be in the first figure, since its conclusion
is to be universal and affirmative.
9. One way of using a first-figure syllogism is the previously
criticized method, which amounts to proving one definition by
means of another; for the middle term to establish an essential
predicate must be an essential predicate, a~d the middle term
to establish an attribute peculiar to the subject must be another
such attribute. Thus the definer will prove one, and will not prove
another, of the definitions of the same subject.
14. That method is not demonstration; it is dialectical
syllogism. But let us see how demonstration can be used. As we
cannot know the reason for a fact before we know the fact, we
cannot know what a thing is before knowing that it is. That a thing
is, we know sometimes per accidens, sometimes by knowing part of
its nature-e.g. that eclipse is a deprivation of light. When our
knowledge of existence is accidental, it is not real knowledge and
does not help us towards knowing what the thing is. But where
we have part of the thing's nature we proceed as follows: Let
eclipse be A, the moon C, interposition of the earth B. To ask
whether the moon suffers eclipse is to ask whether B exists, and
this is the same as to ask whether there is an explanation of A;
if an explanation exists, we say A exists.
35. When we have got down to immediate premises, we know
both the fact (that A belongs to C) and the reason; otherwise
we know the fact but not the reason. If B' is the attribute of
producing no shadow when nothing obviously intervenes, then
if B' belongs to C, and A to B', we know that the moon suffers
eclipse but not why, and that eclipse exists but not what it is.
The question why is the question what B, the real reason for
A, is-whether it is interposition or something else; and this
real reason is the definition of the major term--eclipse is blocking
by the earth.
b7. Or again, what is thunder? The extinction of fire in a11. 8
62 9
cloud. Why does it thunder? Because the fire is quenched in
the cloud. If C is cloud, A thunder, B extinction of fire, B belongs
to C and A to B, and B is the definition of A. If there is a further
middle term explaining B, that will be one of the remaining
definitions of thunder.
IS. Thus there is no syllogism or demonstration proving the
essence, yet the essence of a thing, provided the thing has a cause
other than itself, becomes clear by the help of syllogism and
demonstration.
A. begins the chapter by intimating (93"1-3) that he has reached
the end of the a:1TOplat which have occupied chs. 3-7, and that he
is going to sift what is sound from what is unsound in the argu-
ments he has put forward, and to give a positive account of what
definition is, and try to show whether there is any way in which
essence can be demonstrated and defined. The clue he offers is
a reminder of what he has already said in 90"14-23, that to know
what a thing is, is the same as knowing why it is (93"3-4). The
cause of a thing's being may be either identical with or different
from it ("5-6). This is no doubt a reference to the distinction
between substance, on the one hand, and properties and events
on the other. A substance is the cause of its own being, and there
is no room for demons,tration here; you just apprehend its nature
directly or fail to do so (cf. 93b21-5, 9489-10). But a property or
an event has an arTtO~ other than itself. There are two types of
case which A. does not here distinguish. There are permanent
properties which have a ground (not a cause) in more fundamental
attributes of their subjects (as with geometrical properties, "33-5).
And there are events which have a cause in other events that
happen to their subjects (as with eclipse, b 3- 7, or thunder, b 7- 12).
Further ("6) some events, while they have causes, cannot be
demonstrated to follow from their causes; A. is no doubt referring
to Ta EVS£x0j.LO'a ruWS" Exnv, of which we at least cannot ascertain
what the causes are. But ("6-9) where a thing has a cause other
than itself and proof is possible, the cause must occur as the
middle term, and (since what is being proved is a universal
connexion of a certain subject and a certain attribute) the proof
must be in the first figure.
One attempt to reach definition by an argument in the first
figure is that which A. has recently criticized (0 vvv E~T)Taaj.L€VoS",
al0), viz. the attempt (discussed in 9Ia14-bII) to make a syllogism
with a definition as its conclusion. In such a syllogism the middle
term must necessarily be both essential and peculiar to the subjectCOMMENTARY
(938II-r2, cf. 9r"r5-r6), and therefore the minor premiss must
itself be a definition of the subject, so that the definer proves one
and does not prove another of the definitions of the subject
(93arz-r3), in fact proves one by means of another (as A. has
already pointed out in 9r"25-32, b9- U ). Such an attempt cannot
be a demonstration (93"r4-r5, cf. 9rbro). It is only a dialectical
inference of the essence (93"15). It is this because, while syllogisti-
cally correct, it is, as A. maintains (91a31-2, 36-7, b9- I I ), a petitio
principii. In attempting to prove a statement saying what the
essence of the subject is, it uses a premiss which already claims
to say this.
A. now begins (93"15) to show how demonstration may be used
to reach a definition. He takes up the hint given in a3-4, that to
know what a thing-i.e. a property or an event (for he has in
effect, in a5-9, limited his present problem to this)-is, is to
know why it is. Just as we cannot know why a thing is the case
without knowing that it is the case, we cannot know what a
thing is without knowing that it is (ar6-20). In fact, when we
are dealing not with a substance but with a property or event,
whose esse is inesse in subjecto, to discover its existence is the
same thing as discovering the fact that it belongs or happens
to some subject, and to discover its essence is the same thing
as to discover why it belongs to that subject. Now when a fact
is discovered by direct observation or by inference from a mere
symptom or concomitant, it is known before the reason for it is
known; but sometimes a fact is discovered to exist only because,
and therefore precisely when, the reason for it is discovered to
exist; what never happens is that wp- should know why a fact
exists before knowing that it exists (aI6-20).
Our knowledge that a thing exists may (I) be accidental ("21),
t.e. we may have no direct knowledge of its existence, but have
inferred it to exist because we know something else to exist of
which we believe it to be a concomitant. Or (2) ("21-4) it may be
accompanied by some knowledge of the nature of the thing-of
its genus (e.g. that eclipse is a loss of light) or of some other
element in its essence. In case (I) our knowledge that it exists
gets us nowhere towards knowing what is its essence; for in fact
we do not really know that it exists (a24-7).
It is difficult at first sight to see how we could infer the existence
of something from that of something else without having some
knowledge of the nature of that whose existence we infer; but it
is possible to suggest one way in which it might happen. If we
hear some one whom we trust say 'that is a so-and-so', we infer theII. 8
existence of a so-and-so but may have no notion of its nature. It
is doubtful, however, whether A. saw the difficulty, and whether,
if he had, he would have solved it in this way.
A. turns (827) to case (2), that in which we have some inkling
of the nature of the thing in question, as well as knowledge that
it exists, e.g. when we know that eclipse exists and is a loss of
light. This sets us on the way to explanation of why the moon
suffers eclipse. At this point A.'s account takes a curious turn.
He represents the question whether the moon suffers eclipse as
being solved not, as we might expect, by direct observation or
by inference from a symptom, but by asking and answering the
question whether interposition of the earth between the sun and
the moon-which would (if the moon has no light of its own) both
prove and explain the existence of lunar eclipse-exists. He takes
in fact the case previously (in 817-18) treated as exceptional, that
in which the fact and the reason are discovered together. He adds
that we really know the reason only when we have inferred the
existence of the fact in question through a series of immediate
premises (835-6) ; i.e. (if N be the fact to be explained) through
a series of premises of the form' A (a directly observed fact)
directly causes B, B directly causes C ... M directly causes N'.
But, as though he realized that this is unlikely to happen, he
turns to the more usual case, in which our premises are not
immediate. We may reason thus: 'Failure to produce a shadow,
though there is nothing between us and the moon to account for
this, presupposes eclipse, The moon suffers such failure, Therefore
the moon must be suffering eclipse'. Here our minor premiss is
not immediate, since the moon in fact fails to produce a shadow
only because it is eclipsed; and we have discovered the eclipse
of the moon without explaining it (a36-b3). Having discovered
it so, we then turn to ask which of a vanety of causes which might
explain it exists, and we are satisfied only when we have answered
this question. Thus the normal order of events is this: we begin
by knowing that there is such a thing as eclipse, and that this
means some sort of loss of light. We first ask if there is any
evidence that the moon suffers eclipse and find that there is, viz.
the moon's inability to produce a shadow, at a time when there
are no clouds between us and it. Later we find that there is an
explanation of lunar eclipse, viz. the earth's coming between the
moon and the sun.
The conclusion that A. draws (b 15 - 20) is that while there is no
syllogism with a definition as its conclusion (the conclusion drawn
being not that eclipse is so-and-so but that the moon suffersCOMMENTARY
eclipse), yet a regrouping of the contents of the syllogism yields
the definition 'lunar eclipse is loss of light by the moon in con-
sequence of the earth's interposition between it and the sun'.
9306. Kliv
aAAo, 11 a:rrOS(LKTOV 11 a.V[11TOSUKTOV. This does
not mean that the cause may, or may not, be demonstrated, in
the sense of occurring in the conclusion of a demonstration. What
A. means is that the cause may, or may not, be one from which
the property to be defined may be proved to follow.
4)-16. (t~ ~Ev S~ Tp01TO~ ••• a.pxi1~. Pacius takes the -rp67TO<;
referred to to be that which A. expounds briefly in a3~ and fully
in al6- bI4. But this interpretation will not do. A. would not
admit that the syllogism he contemplates in b3- S ('That which is
blocked from the sun by the earth's interposition loses its light.
The moon is so blocked, Therefore the moon loses its light') is
not a demonstration but a dialectical syllogism (014-15). Pacius
has to interpret A.'s words by saying that while it is a demonstra-
tion as proving that the moon suffers eclipse, it is a dialectical
argument if considered as proving the definition of eclipse. But
A. in fact offers no syllogism proving that 'eclipse is so-and-so';
the moon is the only minor term he contemplates.
Again, the brief mention of a method in ·3~ by no means
amounts to an £~l-raa,<; (aIO) of it. The parallels I have pointed out
above (pp. 629-30) show that 91a14-bII is the passage referred to.
Pacius has been misled, not unnaturally, by supposing vvv to refer
to what immediately precedes. But it need not do this; cf. Plato
Rep. 414 b, referring to 382 a, 389 b, and 6II b referring to 435 b ff.
Pacius interprets av O£ -rp/mov £vO£x£-raL (015) to mean 'how the
dialectical syllogism can be constructed' ; on our interpretation it
means 'how demonstration can be used to aid us in getting a
definition' .
10. TO SL' aAAou TOU TL EO'TL SELKVUO'Q(U. The meaning is made
much clearer by reading -rou for the MS. reading -r6, and the
corruption is one which was very likely to occur.
24. KaL +UXTJv, OTL aUTO aUTO KLVOUV, a reference to Plato's
doctrine in Phaedr. 245 C-246 a, Laws 895 e-896 a; cf. 9Ia37-bI.
34. TOU iXELV Suo bpQc1~, i.e. of the triangle's having angles
equal to two right angles.
b I2• KaL (O'TL y( ••• aKpou. y£ lends emphasis: 'and B is, you
see, a definition of the major term A'.
20. IV TOL~ SLa1ToPTJ~aO'Lv. Ch. 2 showed that definition of
something that has a cause distinct from itself is not possible
without demonstration, ch. 3 that a definition cannot itself be
demonstrated.
nH. 8. 93"6-9. 93h27
CHAPTER 9
What essences can and what cannot be made known by demonstration
93h:u. Some things have a cause other than themselves;
others have not. Therefore of essences some are immediate and
are first principles, and both their existence and their definition
must be assumed or made known in some other way, as the
mathematician does with the unit. Of those which have a middle
term, a cause of their being which is distinct from their own
nature, we may make the essence plain by a demonstration,
though we do not demonstrate it.
93h2I. "Eun 5i ... iUTLv. By the things that have a cause
other than themselves A. means, broadly speaking, properties
and accidents; by those that have not, substances, the cause of
whose being lies simply in their form. But it is to be noted that
he reckons with the latter certain entities which are not sub-
stances but exist only as attributes of subjects, viz. those which
a particular science considers as if they had independent existence,
and treats as its own subjects, e.g. the unit (h2S). ·T(1 ydp fLaO~fLaTa
7r£pl. £ZS7j £UTt.,· ouydp KaO' InrOKHfL£"OV TL.,oS"· £l ydp KaO' InrOKHfL£"oV
TL.,oS" Td Y£WfL£TP'Ka. £UT''', aAA' oux y£ KaO' V7rOKHfL£"OV (79"7-10).
23-4' a Kill dVIlL ••• 1TOLijUIlL. Of apxa[ generally A. says in
E.N. I09Bh3 at fL~ £7raywyfi O£wp0VvTa, (where experience of more
than one instance is needed before we seize the general principle),
at S' ala8~aH (where the perception of a single instance is enough
to reveal the general principle), at S' £OwpiiJ TL.,{ (where the apxa{
are moral principles), Kat llia, S' lliwS". But we can be rather
more definite. The existence of substances, A. would say, is dis-
covered by perception; that of the quasi-substances mentioned
in the last note by abstraction from the data of perception. The
definitions of substances and quasi-substances are discovered by
the method described in ch. 13 (here alluded to in the words llio.,
Tp07rO., c!>a.,£pd 7rodjaa,), which is not demonstration but requires
a direct intuition of the genus the subject belongs to and of the
successive differentiae involved in its nature. Both kinds of apxa{
-the InrOO£aHS" (assumptions of existence) and the op,ap.o{ (for
the distinction cL 72aI8-24)-should then be laid down as assump-
tions (v7roO£aOa, o£i).
25-7. TWV 8' iXOVTWV jUuov ••• a.1To8€LKVUVTIl~. TO T{ £UT' must
be 'understood' as the object of o7jAwaa,. Kat cJw ... ouaw.. is
explanatory of TW., £XOYTw., fL£ao.,. WU1T£P £i7rOfL£" refers to ch. 8.
Waitz's 0' (instead of 0") is a misprint.
nCOMMENTARY
CHAPTER 10
The types of definition
93b~9. (r) One kind of definition is an account of what a word
or phrase means. When we know a thing answering to this exists,
we inquire why it exists; but it is difficult to get the reason for the
existence of things we do not know to exist, or know only per
accidens to exist. (Unless an account is one merely by being
linked together-as the Iliad is-it must be one by predicating
one thing of another in a way which is not merely accidental.)
38. (2) A second kind of definition makes known why a thing
exists. (1) points out but does not prove; (2) is a sort of demonstra-
tion of the essence, differing from demonstration in the arrange-
ment of the terms. When we are saying why it thunders we say
'it thunders because the fire is being quenched in the clouds';
when we are defining thunder we say 'the sound of fire being
quenched in clouds'. (There is of course also a definition of thunder
as 'noise in clouds'. which is the conclusion of the demonstration
of the essence.)
94"9. (3) The definition of unmediable terms is an in demon-
strable statement of their essence.
11. Thus definition may be ((3) above) an indemonstrable
account of essence, ((2) above)-a syllogism of essence, differing
in grammatical form from demonstration, or ((1) above) the con-
clusion of a demonstration of essence. It is now clear (a) in what
sense there is demonstration of essence. (b) in the case of what
terms this is possible, (c) in how many senses 'definition' is used,
(d) in what sense it proves essence. (e) for what terms it is possible,
(f) how it is related to demonstration, (g) in what sense there can
be demonstration and definition of the same thing.
The first two paragraphs of this chapter fall into four parts
which seem at first sight to describe four kinds of definition-
93b29-37, 38-<)4a7. 94 a7-<). 9-10; and T. SI. 3-26 and P. 397. 23-8
interpret the passage so. As against this we have A.'s definite
statement in 94aII-14 (and in 7Sh31-2) that there are just three
kinds; P. attempts to get over this by saying that a nominal
definition, such as is described in the first part of the chapter, is
not a genuine definition.
Let us for brevity's sake refer to the supposed four kinds as the
first, second. third, and fourth kind. In 93h38-<) the second kind
is distinguished from the first by the fact that it shows why the11.
10
thing defined exists; and this is just how the second kind is
distinguished from the third--the second says, for instance,
'thunder is a noise in clouds caused by the quenching of fire',
the third says simply 'thunder is a noise in clouds'. In fact,
there could be no better example of a nominal definition than this
latter definition of thunder. In answer to this it might be said
that while a nominal definiti0n is identical in form with a defini-
tion of the third kind. they differ in their significance, the one
being a definition of the meaning of a word, without any implica-
tion that a corresponding thing exists, the other a definition of
the nature of a thing which we know to exist. But this, it seems,
is not A.'s way of looking at the matter. In 72"18-24 definition is
distinguished from t'm6fhat> as containing no implication of the
existence of the deftniendum; and in 76a32-6 this distinction is
again drawn.
Further. A.'s statement that a definition of the first kind can
originate a search for the cause of the deftniendum (93b32) is a
recapitulation of what he has said in the previous chapter (a2I-b 7).
and the definition of thunder which occurs in this chapter as an
example of the third kind of definition (94"7-8) occurs in that
chapter as an example of the kind of definition we start from in
the search for the cause of the deftniendum (93"22-3).
It seems clear, then. that the 'third kind' of definition is
identical with the first. Further, it seems a mistake to say that
A. ever recognizes nominal definition by that name. The mis-
take starts from the supposition that in 93b30 '\6yo> ET£PO> ovo-
fLaTf . .o8TJ> is offered as an alternative to '\6yo> TOU T{ aTJfLa[VH TO
ovofLa. But why ETEPO>? For if '\6yo> oVOfLaTw8TJ> means nominal
definition, that is just the same thing as '\6yo> TOU T{ GTJfLa{VH TO
ovofLa. Besides, ovofLaTwSTJ> means 'of the nature of a name', and
a nominal definition is not in the least of the nature of a name.
'\6yo> ETEPO> ovofLaTwSTJ> is, we must conclude (and the form of
the sentence is at least equally compatible with this interpreta-
tion), alternative not to '\6yo> TOU T{ GTJIJ.a{VH TO ovofLa but to TO
ovofLa, and means 'or another noun-like expression'. Definitions
of such expressions (e.g. of EMh/.a ypafLfL~, £7T{7TESO> £m</>avHa,
afLf3'\ELa ywv[a) are found at the beginning of Euclid, and were
very likely found at the beginning of the Euclid of A.'s day, the
Elements of Theudius.
As we have seen in ch. 8, it is, according to A.'s doctrine, things
that have no cause of their being, other than themselves, i.e.
substances, that are the subjects of indemonstrable definition.
Thus definitions of the first kind are non-causal definitions ofCOMMENTARY
attributes or events, those of the second kind causal definitions
of the same. The sentence at 94"7-9 does not describe a third
kind; having referred to the causal definition of thunder ("5), A.
reminds the reader that there can also be a non-causal definition
of it. There are only three kinds, and the 'fourth kind' is really
a third kind, definition of substances. The three reappear in
reverse order in 94311-14.
93b3I. otov TL OTIlLaLvn . . . TP£Ywvov. The vulgate reading
olov TO T{ aTJfLa{VH T{ Eanv fI Tp{yWVOv seems impossible. P.'s
interpretation in 372. 17-.18 olov 7Tapr.crr~ T{ aTJfLa{VH TO ovofLa TOU
TP'Ywvov Ka(Jo Tp{ywvov seems to show that he read olov T{ aTJfLa{VH
Tp{ywvov (or Tp{ywvov fJ Tp{ywvov). T{ E(rnV has come in through a
copyist's eye catching these words in the next line.
Since the kind of definition described in the present passage
and in 94 3 7-9 is distinguished from the definition of immediate
terms (9439-10) (i.e. of the subjects of a science, whose definition
is not arrived at by the help of a demonstration assigning a cause
to them, but is simply assumed), Tp{ywvov is evidently here thought
of not as a subject of geometry but as a predicate which attaches
to certain figures. A. more often treats it as a SUbject, a quasi-
substance, but the treatment of it as an attribute is found else-
where, in 71al4, 76a33-{), and 92bIS-I6.
3Z-3. xaX'E1rov S' ... E<7T~V, 'it is difficult to advance from a
non-causal to a causal definition, unless besides having the non-
causal definition we know that the thing definitely exists'.
34. ~rp"Ta~ 1rpOT~pOV, 824-7.
36. ;, IJ-Ev auvS~afJ4l, Wa1r€P ..; 'IXL(i.~, cf. 92b32.
36-7. ;, SE . . . aUIJ-~£~"KO'i. A definition is a genuine pre-
dication, stating one predicate of one SUbject, and not doing so
KaTa. uVfLfJ4JTJK6~, i.e. not treating as grammatical subject what
is the metaphysical predicate and vice versa (cf. 8 I b23-9, 83aI-23).
94"6-7. KaL wSLIJ-Ev ... 0pwlJ-0'i. As Mure remarks, 'Demon-
stration, like a line, is continuous because its premises are parts
which are conterminous (as linked by middle terms), and there
is a movement from premises to conclusion. Definition resembles
rather the indivisible simplicity of a point'.
9. T~IV a.uawv. For the explanation cf. 93b2I-S.
U. 1rTWan, 'in grammatical form', another way of saying what
A. expresses in "2 by Tfj O/UH, 'in the arrangement of the terms'.CHAPTER 11
Each of four types of cause can function as middle term
94":10. We think we know a fact when we know its cause.
There are four causes-the essence, the conditions that necessitate
a consequent, the efficient cause, the final cause; and in every
case the cause can appear as middle term in a syllogism that
explains the effect.
:14. For (I) the conditions that necessitate a consequent must
be at least two, linked by a single middle term. We can exhibit
the matter thus: Let A be right angle, B half of two right angles,
C the angle in a semicircle. Then B is the cause of Cs being A ;
for B = A, and C = B. B is identical with the essence of A,
since it is what the definition of A points to.
35. (2) The essence, too, has previously been shown to function
as middle term.
36. (3) Why were the Athenians made war on by the Medes?
The efficient cause was that they had raided Sardis. Let A be
war, B unprovoked raiding, C the Athenians. Then B belongs
to C, and A to B. Thus the efficient cause, also, functions as
middle term.
bS. (4) So too when the cause is a final cause. Why does a man
walk? In order to be well. Why does a house exist? In order
that one's possessions may be safe. Health is the final cause of
the one, safety of the other. Let walking after dinner be C,
descent of food into the stomach B, health A. Then let B attach
to C, and A to B; the reason why A, the final cause, attaches to
C is B, which is as it were the definition of A. But why does B
attach to C? Because A is definable as B. The matter will be
clearer if we transpose the definitions. The order of becoming here
is the opposite of the order in efficient causation; there the middle
term happens first, here the minor happens first, the final cause
last.
:17. The same thing may exist for an end and as the result of
necessity-e.g. the passage of light through a lantern; that which
is fine-grained necessarily passes through pores that are wider
than its grains, and also it happens in order to save us from
stumbling. If things can be from both causes, can they also
happen from both? Does it thunder both because when fire is
quenched there must be a hissing noise and (if the Pythagoreans
are right) as a means to alarming the inhabitants of Tartarus?
34. There are many such cases, especially in natural processesCOMMENTARY
and products; for nature in one sense acts for an end, nature in
another sense acts from necessity. Necessity itself is twofold;
one operating according to natural impulse, the other contrary to
it (e.g. both the upward and the downward movement of stones
are necessary, in different senses).
95 8 3- Of the products of thought, some (e.g. a house or a statue)
never come into being by chance or of necessity, but only for
an end; others (e.g. health or safety) may also result from chance.
It is, properly speaking, in contingent affairs, when the course of
events leading to the result's being good is not due to chance,
that things take place for an end--either by nature or by art.
No chance event takes place for an end.
This chapter is one of the most difficult in A.; its doctrine is
unsatisfactory, and its form betrays clearly that it has not been
carefully worked over by A. but is a series of jottings for further
consideration. The connexion of the chapter with what precedes
is plain enough. As early as ch. 2 he has said (9°35) uup.{1aiIlH apa
Ell cbaaatS" TaLS" ~7)-n)UWt ~7)TELII ~ El £an P.(UOII ~ Tt Eun T6 p.£aoll.
76 P.£II yap aLnoll T6 p.£aoll, Ell a:rraat o£ Totho ~7)TELTat, and in
chs.8 and 10 he has shown that the scientific definition of any of
the terms of a science except the primary subjects of the science
is a causal definition; but he has not considered the different
kinds of cause, and how each can play its part in detinition. He
now sets himself to consider this question. In the first paragraph
he sets himself to show that in the explanation of a result by any
one of four types of cause, the cause plays the part of ('is exhibited
through', 94323) the middle term. Three of the causes named in
·21-3 are familiar to students of A.-the formal, efficient, and
final cause. The place usually occupied in his doctrine by the
material cause is here occupied by T6 T'IIWII OIlTWII allaYK7) Tour'
Elllat. This pretty clearly refers to the definition of syllogism as
given in A n. Pr. 24b18-20, and the reference to the syllogism is
made explicit in 94324-7. He is clearly, then, referring to the
relation of ground to consequent. The ground of the conclusion
of a syllogism is the two premises taken together, but in order to
make his account of this sort of aLnoll fit into his general formula
that the aLnoll functions as middle term in the proof of that
whose aLnoll it is, he represents this aLnoll as being the middle
term-the middle term, we must understand, as related in a
certain way to the major and in a certain way to the minor.
In Phys. 195316-19 the premises are described as being the
E~ QV or material cause of the conclusion, alongside of other moreII.lI
typical examples of the material cause (Ta p.£v yap crrOtX£La TWV
OlI).).a{3Wv Ka~ .;, V},71 TWV UKroacrrWV Ka~ TO 7TUP Ka~ Ta TOUliha TWV
uwp.aTWV Ka~ Ta p./PTJ TOU o},OV KO.' a: inrO(NU£L, TOU uvp.7T£paup.aTo<;) ,
sc. as being a quasi-material which is reshaped in the conclusion;
cf. M et. IOI3bI7-2I. Both T. and P. take A. to be referring in the
present passage to the material cause, and to select the relation
of premises to conclusion simply as an ~xatnple of the relation
of material cause to effect. But even if the premises may by
a metaphor be said (as in Phys. 195&16-19) to be an example of
the material cause, it is inconceivable that if A. had here meant
the material cause in general, he should not have illustrated it
by some literal example of the material cause. Besides, the
material cause could not be described as TO T{VWV OVTWV o.vaYKTJ
Toih' £Ivat. It does not necessitate that whose cause it is; it is
only required to make this possible. Although in Phys. 195al6-19
A. includes the premises of a syllogism as examples of the material
cause, he corrects this in 200"15-30 by pointing out that their
relation to the conclusion is the converse of the relation of a
material cause to that whose cause it is. The premises necessitate
and are not necessitated by the conclusion; the material cause is
necessitated by and does not necessitate that whose ahwv it is.
Nor could the material cause be described as identical with the
formal canse (94"34-5). It may be added that both the word
v},71 and the notion for which it stands are entirely absent from
the Organon. It could hardly be otherwise; v},71 is ayvwuTo<; Ka8'
aVn)v (M et. 1036"9); it does not occur as a term in any of our
ordinary judgements (as apart from metaphysical judgements),
and it is with judgements and the inferences that include them
that logic is concerned. The term inroK£{P.£VOV, indeed. occurs in
the Organon. but then it is used not as equivalent to iJ},71, but as
standing either for an individual thing or for a whole class of
individu;:tl things; the analysis of the individual thing into matter
and form belongs not to logic but to physics (as A. understands
physics) and to metaphysics. and it is in the Metaphysics and the
physical works that the word v},71 is at home.
A., tllen, is not putting forward his usual four causes. It may be
that this chapter belongs to an early stage at which he had not
reached the doctrine of the four causes. Or it may be that,
realizing that he could not work the material cause into his
thesis that the cause is the middle term, he deliberately substi-
tutes for it a type of a,nov which will suit his thesis, namely, the
ground of a conclusion as the ahwv of the conclusion. Unlike
efficient and final causation, in both of which there is temporalCOMMENTARY
difference between cause and effect (b 23-{i), in this kind of necessi-
tation there is no temporal succession; ground and consequent
are eternal and simultaneous. And since mathematics is the
region in which such necessitation is most clearly evident, A.
naturally takes his example from that sphere ("28-34).
The four causes here named, then, are formal cause, ground
(T[VWV OVTWV avaYK7J TOV-r' €tva,), efficient cause, final cause. But
A:s discussion does not treat these as all mutually exclusive.
He definitely says that the ground is the same as the fonnal
cause ("34-5). Further, he has already told us (in clls. 8, 10) that
the middle tenn in a syllogism wl:ich at the same time proves
and explains the existence of a consequence is an element in the
definition of the consequence, i.e. in its formal cause (the general
fonn of the definition of a consequential attribute being' A is
a B caused in C by the presence of D'). It is not that the middle
tenn in a demonstration is sometimes the fonnal cause of the
major tenn, sometimes its ground, sometimes its efficient cause,
sometimes its final cause. It is always its fonnal cause (or
definition), or rather an element in its fonnal cause; but this
element is in some cases an eternal ground of the consequent
(viz. when the consequence is itself an eternal fact), in some cases
an efficient or a final cause (when the consequence is an event);
the doctrine is identical with that which is briefly stated in Met.
1041"27-30, q,avfpdv TO[VVV on ~7JT€'i: Td ai'nov· TOV-rO S' £aTt Td Tt
1}v €tva" w<; €l1T£'iV AOY'KW<;, 0 £1T' £v[wv ,.dv £an T[VO<; £V€Ka, oCov
i'aw<; £1T' olK[a<; 7) KAlV7J<;, £1T' £v[wv S£ T[ £K[V7Jae 1TpWTOV· ai'nov yap
Kat TOV-rO. Cf. ib. 1044"36 T[ S' W<; TO efSo<;; Td T[ 1}v etva,. T[ S' w<;
00 £vfKa; Td T£Ao<;. i'aw<; S£ TaV-ra aJi4w Td ath-a. In chs. 8 and 10
(e.g. 93b3-12, 38--<)4"7) the doctrine was illustrated by cases in
which the element-in-the-definition which serves as middle term
of the corresponding demonstration was in fact an efficient cause.
Lunar eclipse is defined as 'loss of light by the moon owing to
the interposition of the earth', thunder as 'noise in clouds due
to the quenching of fire in them'. In this chapter A. attempts to
show that in other cases the element-in-the-definition which serves
as middle term of the corresponding demonstration is an eternal
ground, and that in yet others it is a final c:ause.
The case of the eternal ground is illustrated by the proof of the
proposition that the angle in a semicircle is a right angle ("28-34).
The proof A. has in mind is quite different from Euclid's proof
(El. iii. 31). It is only hinted at here, but is made clearer by Met.
1051"27 £V -i],.UKVKAlCf dp81] Ka80Aov S,a T[; £av taa, Tpe'i<;, 7j Tf {Jaa,<;
SVO Kat -i] £K ,.daov £maTa8e'iaa dp8fJ, lS6vn Sf/Aov Tep £K€LVO (i.e.lI.n
on 000 dp9al TO Tp{yWVOV (1051-24), that the angles of a triangle
equal two right angles) £lS6n. From 0, the centre of the circle,
()-
NL-----~~----~
OQ perpendicular to the diameter NP is drawn to meet the
circumference, and NQ, PQ are joined. Then, NOQ and POQ
being isosceles triangles, LOQN = ONQ, and LOQP = OPQ.
Therefore OQN +OQP ~= NQP) = ONQ+OPQ, and therefore
= half of the sum of the angles of NQP, i.e. of two right angles,
and therefore = one right angle. (Then, using the theorem that
angles in the same segment of a circle are equal (Euc. iii. 21), A.
must have inferred that any angle in a semicircle is a right
angle.) In this argument, NQP's being the half of two right
angles is the ground of its being one right angle, or rather the
causa cognoscendi of this. (This is equally true of the proof inter-
polated in the part of Euclid after iii. 31, and quoted in Heath,
M athematics i1~ Aristotle, 72; but A. probably had in mind in the
present passage the proof which he clearly uses in the Metaphysics.)
But A.'s comment 'this, the ground, is the same as the essence
of the attribute demonstrated, because this is what its definition
points to' (334-5) is a puzzling statement. Reasoning by analogy
(it would appear) from the fact that, e.g., thunder may fairly be
defined as 'noise in clouds due to the quenching of fire in them',
A. seems to cont~mplate some such definition of the rightness of
the angle in a semicircle as 'its being right in consequence of being
the half of two right angles'; and for this little can be said. The
analogy between the efficient cause of an event and the causa
cognoscendi of an eternal consequent breaks down; the one can
fairly be included in the definition of the event, the other cannot
be included in the definition of the consequent.
Two comments may be made on A.'s identification of the
ground of a mathematical consequent with the definition of the
consequent. (I) The definition of 'right angle' in Euclid (and
probably in the earlier Elements known to A.) is: a-rav £v9£'ta 'TT'
£v9£'tav erra9£'iaa T<1S- N)£~ijs- ywvlas- taas- lli1}Aa,s- TTo'fi, dp9~ lKaT£pa
TWV tawv ywv~v 'err, (El. i, Def. 10). Thus the right angle is
defined as half of the sum of a certain pair of angles, and it is not
unnatural that A. should have treated this as equivalent to
4~S
TtCOMMENTARY
defining it as the half of two right angles. (2) While it is not
defensible to define the rightness of the angle in a semicircle as
its being right by being the half of two right angles, there is
more to be said for a similar doctrine applied to a geometrical
problem, instead of a geometrical theorem. The squaring of a
rectangle can with some reason be defined as 'the squaring of it
by finding a mean proportional between the sides' (De An.
4 13"13- 20 ).
A. offers no separate proof that the formal cause of definition
functions as middle tenn. He merely remarks ("35-6) that that
has been shown before, i.e. in chs. 8 and 10, where he has shown
that the cause of an attribute, which is used as middle tenn in
an inference proving that and explaining why a subject has the
attribute, is also an element in the full definition (i.e. in the fonnal
cause) of the attribute.
With regard to the efficient cause (a36- b 8) A. makes no attempt
to identify it with the fonnal cause, or part of it. He merely points
out that where efficient causation is involved, the event, in con-
sequence of whose happening to a subject another event happens
to that subject, functions as middle tenn between that subject
and the later event. The syllogism, in the instance he gives,
would be: Those who have invaded the country of another people
are made war on in return, The Athenians have invaded the
country of the Medes, Therefore the Athenians are made war on
by the Medes.
With regard to the final cause (b8-23) A. similarly argues that
it too can function as the middle tenn of a syllogism explaining
the event whose final cause it is. He begins by pointing out
(b8-I2) that where a final cause is involved, the proper answer
to the question 'why?' takes the fonn 'in order that .. .'. He
implies that such an explanation can be put into syllogistic fonn,
with the final cause as middle term; but this is in fact impossible.
If we are to keep the major and minor tenns he seems to envisage
in the example he takes, i.e. 'given to walking after dinner' and
'this man', the best argument we can make out of this is: Those
who wish to be healthy walk after dinner, This man wishes to be
healthy, Therefore this man walks after dinner. And here it is
not 'health' but 'desirous of being healthy' that is the middle
tenn. If, on the other hand, we say 'Walking after dinner pro-
duces health, This man desires health, Therefore this man walks
after dinner', we abandon all attempt at syllogistic fonn. A. is
in fact mistaken in his use of the notion of final cause. It is never
the so-called final cause that is really operative, but the desirelI.II
of an object; and this desire operates as an efficient cause, being
what corresponds, in the case of purposive action, to a mechanical
or chemical cause in physical action.
Up to this point A. has tried to show how an efficient cause may
function as middle term (·36-b8) and how a final cause may do so
(b8-IZ). He now (b I2- zo) sets himself to show that an efficient
cause and a final cause may as it were play into each other's hands,
by pointing out that between a purposive action (such as walking
after dinner) and the ultimate result aimed at (e.g. health) there
may intervene an event which as efficient cause serves to explain
the occurrence of the ultimate result, and may in turn be teleo-
logically explained by the result which is its final cause. He offers
first the following quasi-syllogism: Health (A) attaches to the
descent of food into the stomach (B), Descent of food into the
stomach attaches to walking after dinner (C), Therefore health
attaches to walking after dinner. A. can hardly be acquitted of
failing to notice the ambiguity in the word ImaPXHII. In his
ordinary formulation of syllogism it stands for the relation of
predicate to subject, but here for that of effect to cause; and
'A is caused by B, B is caused by C, Therefore A is caused by C',
while it is a sound argument, is not a syllogism.
A. adds (b I9- zo) that here B is 'as it were' a definition of A,
i.e. that just as lunar eclipse may be defined by means of its
efficient cause as 'failure of light in the moon owing to the inter-
position of the earth' (ch. 8), so health may be defined as 'good
condition of the body due to the descent of food into the stomach'.
This is only 'as it were' a definition of health, since it states not
the whole set of conditions on which health depends, but only
the condition relating to the behaviour of food.
'But instead of asking why A attaches to C' (A. continues in
b zo--3 ) 'we may ask why B attaches to C; and the answer is
"because that is what being in health is-being in a condition
in which food descends into the stomach." But we must transpose
the definitions, and so everything will become plainer.' It may
seem surprising that A. should attempt to explain by reference to
the health produced by food's descent into the stomach (sc. and
the digestion of it there) the sequence of the descent of food upon
a walk after dinner-a sequence which seems to be sufficiently
explained on the lines of efficient causation. And in particular,
it is by no means easy to see what syllogism or quasi-syllogism
he has in mind; the commentators are much puzzled by the
passage and have not been very successful in dealing with it. We
shall be helped towards understanding the passage if we takeCOMMENTARY
note of the very strong teleological element in A.'s biology
(especially in the De Partibus Animalium), and consider in parti-
cular the following passages: Phys. 200"15 £an 8i 'n) avaYKawv £V
,..
R...!.
",..
\ J..'
,
I
\
'T£ 'TOLS" J.Lav"J.LaaL KaL £V 'TOLS" Ka'Ta ,/,vaLV YLYVOJ.Lt!.VOLS" 'TPOTTOV nva
TTapaTT).:TJaLwS"· iTT£~ yap 'T6 £u8u 'T08L ianv, al'aYKTj 'T6 'TPLYWVWV BOo
op8a'ts" raaS" £X£LV' ill' OUK iTT'" 'ToVro, EK£'1I0' a,U' £, Y£ 'ToVro J.L~
£CT'TLV, o~t 'T6 ro8u £CT'TLV. EV 8i 'TO'S" YLyvOJ.L£I'OLS" £v£Ka 'TOV avaTTa.\Lv,
£l 'T6 'Tl>"oS" £a'TaL ~ £an, Ka~ 'T6 £J.LTTpoa8£v £a'TaL ~ £CT'TLV' £l
8t J.L~, WCT7T£P EK£' J.L~ 0l''T0S" 'TOV CTVJ.LTT£paaJ.La'ToS" ~ ap~ OUK £CT'TaL, Ka~
£v'Tav8a 'T6 'TtAoS" Ka~ 'T6 o~ £v£Ka. Part. An. 639b26 avaYKTj 8£
'ToLav8£ 'T~V v>"Tjv il7TcJ.p,aL, d £a'TaL olK{a ~ a>..>.o 'Tt 'T£>"0S"'
Ka~ ywlaBaL 'Tt!. Ka~ K'VT/BiivaL 8£, 'T08£ TTPW'TOV, £l'Ta 'T08£, Ka~ 'ToVrov
8~ 'T6V 'TPOTTOV E.p£'iiS" J.LIXPL 'TOV 'T£>"ovS" Kal o~ £v£Ka y{v£'TaL £KaCT'TOV
Ka~ £CT'TLV. diaav-rwS" 8£ Ka~ iv 'TOLS" .pva£L YLV0J.L£VOLS". ill' 0 'TPClTToS" ri]S"
aTT08£~£wS" Ka~ ri]S" avcJ.YKT/S" ;'T£poS", iTT{ 'Tt!. ri]S" .pvaLKiiS" Ka~ 'TWV
B£WPTj'TLKWV E7TI.o7TJJ.LWV, ~ yap ap~ 'TO LS" J.L£v 'T6 OV, 'T 0 , S" 8 £ 'T 6
iaoJ.L£vov· £TT£~ yap 'Tolov8£ £CT'T~V ~ v,,{na ~ 0 avBpwTToS", avcJ.YKT/
'Tos' £fval .ry y£vlaBaL, ill' OUK ETT£~ 'ToS' £CT'TLV ~ Y£YOV£I', CK£LVO i,
u.vcJ.YKT/S" £CT'TLV ~ £CT'TaL.
In the light of these passages we can see that A.'s meaning
must be that instead of the quasi-syllogism (I) (couched in terms
of efficient causation) 'since descent of food into the stomach
produces health, and walking after dinner produces such descent,
walking after dinner produces health' (b lI - 20), we can have the
quasi-syllogism (2) (couched in terms of final causation) 'since
health presupposes descent of food into the stomach, therefore
if walking after dinner is to produce health it must produce
such descent'.
In 8££ J.La'Ta.\aJ.Lf3cJ.v£Iv 'TO vS" >..OyovS", >"0yovS" might mean 'reason-
ings', but the word has occurred in b I9 in the sense of definition,
and it is better to take it so here. A.'s point is this: In the quasi-
syllogism (I) above, we infer that walking after dinner produces
health because it produces what is 'as it were' the definition of
health. Now transpose the definition; instead of defining health
as a good condition of body caused by descent of food into the
stomach, define descent of food into the stomach as movement
of food necessitated as a precondition of health, and we shall see
that in the quasi-syllogism (2) we are inferring that if walking
after dinner is to produce that by reference to which descent of
food into the stomach is defined (viz. health), it must produce
descent of food into the stomach.
The order of becoming in final causation, A. continues (b 23 -{i),
is the opposite of that in efficient causation. In the latter then.II
middle term must come first; in the former, C, the minor term,
must come first, and the final cause last. Here the type of
quasi-syllogism hinted at in h 2 o- 1 is correctly characterized. C,
the minor term (walking after dinner), happens first; A, the final
cause and middle term (health), happens last; and B, the major
term (descent of food into the stomach), happens between the
two. But what does A. mean by saying that in efficient causation
the middle term must come first? In the last syllogism used to
illustrate efficient causation (in h18-20) not the middle term B
(descent of food) but the minor term C (walking after dinner)
happens first. A. is now thinking not of that syllogism but of
the main syllogism used to illustrate efficient causation (in
"36- h 8). There the minor term (the Athenians) was not an event
but a set of substances; A. therefore does not bring it into the
time reckoning, and in saying that the middle term happens first
means only that it happens before the major term.
A. has incidentally given an example of something that
happens both with a view to an end and as a result of necessity.
viz. the descent of food into the stomach, which is produced by
walking after dinner and is a means adopted by nature for the
production of health. He now (h27--9S"3) points out in general
terms the possibility of such double causation of a single event.
He illustrates this (1) by the passage of light through the pores
of a lantern. This may occur both because a fine-grained sub-
stance (light) must be capable of passing through pores which
are wider than its grains (A. adopts, as good enough in a mere
illustration of a general principle, Gorgias' theory, which is not
his own, of the propagation of light (cf. 88"14-16 n.)), and because
nature desires to provide a means that will save us from stumbling
in the dark. A. illustrates the situation (2) by the case of thunder.
This may occur both because the quenching of fire is bound to
produce noise and-A. again uses for illustrative purposes a view
he does not believe in-to terrorize the inhabitants of Tartarus.
Such double causation is to be found particularly in the case
of combinations that nature brings into existence from time to
time or has permanently established (TolS' Ka'Tcl .pOCTlV O'VVl.Q'Ta-
P.iVOLS' Kal avV€<TTWOW, h3S ). Natural causation is probably meant
to be distinguished from mathematical necessitation, which never
has purpose associated with it, and from the purposive action of
men, which is never necessitated. A study of various passages
in the De Parlibus Animalium (6S8 b2-7, 663b22-664"II, 679"2S-30)
shows that A. considers the necessary causation to be the primary
causation in such cases, and the utilization for an end to be a sortCOMMENTARY
of afterthought on nature's part (1I"wS' St riiS' avaYKaLaS' 4>vu£wS'
tXOVC17)S' TOLS' v-rro.PXOUULV £~ avaYK7JS' ~ KaTa TOV A6yov 4>VULS' EV£Ko.
TOU KaTaK£XP7JTaL, Uywp.£v, 663b22-4).
Incidentally (94 b 37-95 3 3) A. distinguishes the natural necessity
of which he has been speaking from a form of necessity which is
against nature; this is illustrated by the difference between the
downward movement of stones which A. believes to be natural
to them and the upward movement which may be impressed upon
them by the action of another body-a difference which plays a
large part in his dynamics (cf. my edition of the Physics, pp. 26-
33)·
From natural products and natural phenomena A. turns (95"3-6)
to consider things that are normally produced by purposive
action; some of these, he says, are never produced by chance or
by natural necessity, but only by purposive action; others may
be produced either by purposive action or by chance-e.g. health
or safety. This point is considered more at length in M et. 1034"9-
21, where the reason for the difference is thus stated: aiTLov Sf
OTL TWV p.£v ~ vA7J ~ a.pxouua TfiS' y£v£u£wS' tv To/ 1I"OLELV Kat YLYVEu()aL
TL TWV d1l"o T'XVT}S', tv fJ !maPXEL TL p.£POS' TOU 1I"payp.aT0>-4 p.Ev
TOLaVr7J taTLV oia KLvELu()aL of aOTfiS' ~ S' OV, Kat TaVr7]S' ~ P.EV WS,
oia 'rE ~ SE dSvvaToS'. 'Chance production is identical in kind with
the second half of the process of artistic production. The first
half, the V07JULS', is here entirely absent. The process starts with
the unintended production of the first stage in the making, which
in artistic production is intended. This may be produced by
external agency, as when an unskilled person happens to rub
a patient just in the way in which a doctor would have rubbed
him ex aTte, and thus originates the curative process. Or again,
it may depend on the initiative resident in living tissue; the sick
body may itself originate the healing process' (Aristotle's M eta-
physics, ed. Ross, i, p. cxxi).
Zabarella makes 95 8 3-6 the basis of a distinction between what
he calls the non-conjectural arts, like architecture and sculpture,
which produce results that nothing else can produce, and produce
them with fair certainty, and conjectural arts, like medicine or
rhetoric, which merely contribute to the production of their
results (nature, in the case of medicine, or the state of mind of
one's hearers, in that of rhetoric, being the other contributing
cause)-so that, on the one hand, these arts may easily fail to
produce the results they aim at, and on the other the causes
which commonly are merely contributory may produce the
results without the operation of art.Finally ('6---9), A. points out that teleology is to be found,
properly speaking, in these circumstances: (1) EV OUOL, lv8'X£TaL
Kat ~8£ Kat aAAw" i.e. when physical circumstances alone do not
determine which of two or more events shall follow, when (2)
the result produced is a good one, and (3) the result produced is
not the result of chance. He adds that the teleology may be
either the (unconscious) teleology of nature or the (conscious)
teleology of art. Thus, as in M et. 1032'12-13, A. is working on the
assumption that events are produced by nature, by art (or, more
generally, by action following on thought), or by chance. The
production of good results by nature, and their production by art,
are coupled together as being teleological. With the present
rather crude account should be compared the more elaborate
theory of chance and of necessity which A. develops in the
Physics (cf. my edition, 38-44).
It is only by exercising a measure of goodwill that we can con-
sider as syllogisms some of the 'syllogisms' put forward by A.
in this chapter. But after all he does not use the word 'syllogism'
here. What he says is that any of the four causes named can serve
as p.'uov between the subject and the attribute, whose connexion
is to be explained. He had the conception, as his account of
the practical syllogism shows (E.N. II44831-3 ot yap uvAAOYLUP.Ot
TWV 7TpaKTwv apxTJv iXOVT', £lULV '£7T£t8-ry TOLov8£ TO T'>'O, Kat TO
apLuTov'). of quasi-syllogisms in which the relations between terms,
from which the conclusion follows, are other than those of subject
and predicate; i.e. of something akin to the 'relational inference'
recognized by modern logic, in distinction from the syllogism.
94b8. "Ouwv 5' a.tTLOV TO iVEKa. T(VOo;. The editors write iV£Kcl
TLVO" but the sense requires iV£Ka TtvO, as in b12 (cf. TO Tlvo,
iV£Ka, • 23).
32-4. WO""ITEP Et ••• ~OI3WVTa.L. 'as for instance if it thunders
both because when the fire is quenched there must be a hissing
noise, and (if things are as the Pythagoreans say) to intimidate
the inhabitants of Tartarus'. It seems necessary to insert on,
and this derives support from T. 52. 26 Kat ~ {3POVrt] , 8LOTL T£
a7Toa{3£vvvp.lvov KT>' .• and E. 153. II 8La Tt {3pOV'T~; on 7TVp a7TO-
a{3£vvvp.£VOV KT>'.
9Sa~. I1cl.ALuTa. SE ... TExVn. The received punctuation (oTav
p.~ a7To TVXTJ> ~ y'v£UL>
WGT£ TO T'>'O, aya80v iV£Kcl TOV yivETaL Kat
11 c/>vun 11 TIXVT/) is wrong; the comma after must be omitted, and
one must be introduced after aya8ov. Further, TIAo, must be
understood in the sense of result, not of end.
n,
nCOMMENTARY
CHAPTER 12
The inference of past and future events
95"10. Similar effects, whether present, past, or future, have
similar causes, correspondingly present, past, or future. This
is obviously true in the case of the formal cause or definition
(e.g. of eclipse, or of ice), which is always compresent with that
whose cause it is.
22. But experience seems to show that there are also causes
distinct in time from their effects. Is this really so?
27. Though here the earlier event is the cause, we must reason
from the later. Whether we specify the interval between the
events or not, we cannot say 'since this has happened, this later
event must have happened'; for during the interval this would
be untrue.
35. And we cannot say 'since this has happened, this other
will happen'. For the middle term must be coeval with the major;
and here again the statement would be untrue during the interval.
hI. We must inquire what the bond is that secures that event
succeeds event. So much is clear, that an event cannot be con-
tiguous with the completion of another event. For the completion
of one cannot be contiguous with the completion of another,
since completions of events are indivisible limits, and therefore,
like points, cannot be contiguous; and similarly an event cannot
be contiguous with the completion of an event, any more than
a line can with a point; for an event is divisible (containing an
infinity of completed events), and the completion of an event is
indivisible.
13. Here, as in other inferences, the middle and the major term
must be immediately related. The manner of inference is: Since
C has happened, A must have happened previously; If D has
happened, C must have happened previously; Therefore since
D has happened, A must have happened previously. But in
thus taking middle terms shall we ever reach an immediate
premiss, or will there (owing to the infinite divisibility of time)
always be further middle terms, one completed event not being
contiguous with another? At all events, we must start from an
immediate connexion, that which is nearest to the present.
25. So too with the future. The manner of inference is: If
D is to be, C must first be; If C is to be, A must first be; Therefore
if D is to be, A must first be. Here again, subdivision is possible
ad infinitum; yet we must get an immediate proposition as
starting-point.n.
I2
31. Inference from past to earlier past illustrated.
35. Inference from future to earlier future illustrated.
38. We sometimes see a cycle of events taking place; and this
arises from the principle that when both premises are convertible
the conclusion is convertible.
96-8. Probable conclusions must have probable premises; for
if the premises were both universal, so would be the conclusion.
17. Therefore there must be immediate probable premises,
as well as immediate universal premises.
A. starts this chapter by pointing out that if some existing
thing A is the cause (i.e. the adequate and commensurate cause)
of some existing thing B, A is also the cause of B's coming to
be when it is coming to be, was the cause of its having come to
be if it has come to be, and will be the cause of its coming to be
if it comes to be in the future. He considers first (814-24) causes
simultaneous with their effects, i.e. fonnal causes which are an
element in the definition of that whose causes they are, as 'inter-
position of the earth' is an element in the definition of lunar
eclipse as 'loss of light owing to the interposition of the earth'
(cf. ch. 8), or as 'total absence of heat' is an element in the
definition of ice as 'water solidified owing to total absence of
heat'.
It is to be noted that, while in such cases the causes referred
to are elements in the fonnal cause (or definition) of that whose
cause they are, they are at the same time its efficient cause; for
fonnal and efficient causes are, as we have seen (ch. II, intro-
ductory note), not mutually exclusive. What A. is considering
in this paragraph is in fact efficient causes which he considers
to be simultaneous with their effects.
From these cases A. proceeds (a24-b37) to consider causes that
precede their effects in time; and here we must take him to be
referring to the general run of material and efficient causes. He
starts by asking whether in the time-continuum an event past,
future, or present can have as cause another event previous to it,
as experience seems to show (U)(rTU,P DOKEL ~IL'V' 825). He assumes
provisionally an affinnative answer to this metaphysical question,
and proceeds to state a logical doctrine, viz. that of two past
events, and therefore also of two events still being enacted, or
of two future events, we can only infer the occurrence of the
earlier from that of the later (though even here the earlier is of
course the originative source of the later (·28~)). (A) He con-
siders first the case of inference from one past event to another.COMMENTARY
We cannot say 'since event A has taken place, a later event B
must have taken place' -either after a definite interval, or without
determining the interval ("31-4). The reason is that in the
interval (A. assumes that there is an interval, and tries to show
this later, in b 3- 12) it is untrue to say that the later event has
taken place; so that it can never be true to say, simply on the
ground that event A has taken place, that event B must have
taken place ("34-5). So too we cannot infer, simply on the ground
that an earlier future event will take place, that a later future
event must take place ("35--6).
(B) A. now turns to the question of inference from a past to
a future event ("36). We cannot say 'since A has taken place,
B will take place'. For (I) the middle term must be coeval with
the major, past if it is past, future if it is future, taking place if it
is taking place, existing if it is existing. A. says more than he
means here; for what he says would exclude the inference of a
past event from a present one, no less than that of a future from
a past one. He passes to a better argument: (2) We cannot say
'since A has existed, B will exist after a certain definite interval',
nor even 'since A has existed, B will sooner or later exist'; for
whether we define the interval or not, in the interval it will not
be true that B exists; and if A has not caused B to exist within
the interval, we cannot, simply on the ground that A has existed,
say that B ever will exist.
From the logical question as to the inferability of one event
from another, A. now turns (b1 ) to the metaphysical question
what the bond is that secures the occurrence of one event after
the completion of another. The discussion gives no clue to A.'s
answer, and we must suppose that he hoped by attacking the
question indirectly, as he does in b 3- 37 , to work round to an
answer, but was disappointed in this hope. He lays it down that
since the completion of a change is an indivisible limit, neither
a process of change nor a completion of change can be contiguous
to a completion of change (b 3- S). He refers us (bIO- 12 ), for a
fuller statement, to the Physics. The considerations he puts for-
ward belong properly to .pUGtK1/ i1Tta-r-r/JLTJ, and for a fuller dis-
cussion of them we must indeed look to the Physics, especially to
the discussion of time in iv. 10-14 and of the continuous in vi.
In Phys. 227"6 he defInes the contiguous (ixoJL€lIoy) as 0 ay i.p€~ij<;
~II a'TT77}Tat. I.e. two things that are contiguous must (I) be succes-
sive, having no third thing of the same kind between them (226b34-
22786), and (2) must be in contact, i.e. having their extremes
together (226b23); lines being in contact if they meet at a point,H.
I2
planes if they meet at a line, solids if they meet at a plane, periods
of time or events in time if they meet at a moment. Now the
completion of a change is indivisible and has no extremes (since
it occurs at a moment, as A. proves in 23S b3O-236a 7), just as a
point has not. It follows that two completions of change cannot
be contiguous (9S b4-6). Nor can a process of change be contiguous
to the completion of a previous change, any more than a line can
be contiguous to a point (b6---9) ; for as a line contains an infinity
of points, a process of change contains an infinity of completions
of change (h9-ro)-a thesis which is proved in 236b32-237ar7.
From his assumption that there is an interval between two
events in a causal chain (a34, br), and from his description of them
as merely successive (b I3 ), it seems that A. considers himself
to have proved that they are not continuous or even contiguous.
But this assumption rests on an ambiguity in the words y£yovoS",
y£vojL£vov, Y£Y£VTJjL£vov (which he treats as equivalent). He has
shown that two completions of change cannot be contiguous,
any more than two points, and that a process of change cannot
be contiguous to a completion of change, any more than a line
can be to a point. But he has not shown that two past processes
of change cannot be contiguous, one beginning at the moment at
which the other ends.
In inference from effect to cause (A. continues, br4 ), as in all
scientific inference (Kullv TOUTOLS", bIS ), there must be an immediate
connexion between our middle term and our major, the event
we infer from and the event we infer from it (b I4 - IS ). Wherever
possible we must break up an inference of the form 'Since D
has happened, A must have happened' into two inferences of the
form 'Since D has happened, C must have happened', 'Since C
has happened, A must have happened'-C being the cause (the
causa cognoscendi) of our inference that A has happened (b r6-2I).
But in view of the point we have proved, that no completion of
change is contiguous with a previous one, the question arises
whether we can ever reach two completions of change C and A
which are immediately connected (b22- 4). However this may be,
A. replies, we must, if inference is to be possible, start from an
immediate connexion, and from the first of these, reckoning back
from the present.
A. does not say how it is that, in spite of the infinite divisibility
of time, we can arrive at a pair of events immediately connected.
But the answer may be gathered from the hint he has given
when he spoke of becoming as successive (b I3 ). Events, as he has
tried to show, cannot be contiguous, but they can be successive;COMMENTARY
there may be a causal train of events ACD such that there is no
effect of A between A and C, and no effect of C between C and
D, though there is a lapse of time between each pair; and then
we can have the two immediate premises 'C presupposes A,
D presupposes C', from which we can infer that D presupposes A.
So too with the inferring of one future event from another
(b 25 --8); we can infer the existence of an earlier from that of a
later future event. But there is a difference. Speaking of past
events we could say 'since C has happened' (b1 6); speaking of
future events we can only say 'if C is to happen' (b 29 ).
Finally, A. illustrates by actual examples (~1T~ TWJI £PYWJI, b 32 )
inference from a past event to an earlier past event (b 32- 5), and
from a future imagined event to an earlier future event (b35-7).
To the main discussion in the chapter, A. adds two further
points: (1) (b38---96a7) he remarks that certain cycles of events
can be observed in nature, such as the wetting of the ground,
the rising of vapour, the formation of cloud, the falling of rain,
the wetting of the ground .... He asks himself the question how
this can happen. His example contains four terms, but the
problem can be stated more simply with three terms. The
problem then is: If C entails Band B entails A, under what
conditions will A entail C? He refers to his previous discussion
of circular reasoning. In A n. Pr. ii. 5 he has shown that if we
start with the syllogism All B is A, All C is B, Therefore all C
is A, we can prove the major premiss from the conclusion and
the converse of the minor premiss, and the minor premiss from
the conclusion and the converse of the major premiss. And in
A n. Post. 73"6-20 he has pointed out that any of the six proposi-
tions All B is A, All C is A, All B is C, All C is B, All A is B,
All A is C can be proved by taking a suitable pair out of the other
five. This supplies him with his answer to the present problem.
A will entail C if the middle term is convertible with each of the
extreme terms; for then we can say B entails C, A entails B,
Therefore A entails C. (2) (9638-19) he points out that, since the
conclusion from two universal premises (in the first figure) is
a universal proposition, the premises of a conclusion which only
states something to happen for the most part must themselves
(i.e. both or one of them) be of the same nature. He concludes
that if inference of this nature is to be possible, there must be
immediate propositions stating something to happen for the most
part.
95828-9. a.PX~ SE ••• yryOVOTa.. This is best interpreted (as
by P. 388. 4--8, 13-16, and E. 164. 34-165. 3) as a parentheticalreminder that even if we infer the earlier event from the later,
the earlier is the originating source of the later. YEyovoTa stands
for 7TpoyeyovoTa.
b3-5' T) Sil Aoy ••• o.Tofla. ycyovoS" (or ycvoJLcvov) here means
not a past process of change; for that could not be said to be
indivisible. It means the completion of a past change, of which
A. remarks in Phys. z36aS-7 that it takes place at a moment, i.e.
is indivisible in respect of time.
18. a EOTLY cipXTJ TOU XpOyou. The now is the starting-point
of time in the sense that it is the point from which both past and
future time are reckoned; cf. Phys. ZI9bII TO Oe vilv TOV XpOVOV
opt{n, -n 7TPOTCPOV Ka~ VCTTCPOV, ZZO·4 Ka~ aVIICX1}S" TC O~ 0 XPOIIOS" TtP
viill, Ka~ Otzlp7JTat KaTd. TO viiv, and for A.'s whole doctrine of the
relation between time and the now cf. zI8-6-zzo·z6, z33b33-Z34b9.
%4. wCTTn,p iAiX9'1' in b 3 -ti.
%4-5. ciAA' o.p~aaea£ yE ••• 1I"pWTOU. A.'s language in blS and
31 shows that the reading a7T' aJL£aou is right. Ka~ a7To TOV IIijy
7TPWTOU is ambiguous. It may mean (I) that we must start from
the present, i.e. must work back from a recently past event to
one in the more remote past. Or more probably (so P. 394. 14,
An. S77. z4) (z) the whole phrase a7T' aJL£aov Ka~ a7To TOV villi 7TPWTOV
may mean 'from a connexion that is immediate and is the first
of the series, reckoning back from the present'.
34. Et1l"EP Kat otK£a y(yOYEY. The sense requires this reading,
which is confirmed by E. 176. 19. The writer of the archetype of
our MSS. has been misled by MOouS" ycyoII{IIat and OfJL£AtOll
yeyov£IIat.
96"1. EY TO~S 1I"pWTOLS, i.e. in 73"6-zo (cf. An. Pr. ii. 5).
18. cipxat o.f1E(70L, a(7a. oaa is in apposition to apxat.
CHAPTER 13
The use of division (a) for the finding of definitions
~%O.
We have shown how the essence of a thing is set out
in the terms of a syllogism, and in what sense there is or is not
demonstration or definition of essence. Let us state how the
elements in a definition are to be searched for. Of the attributes
of a subject, some extend beyond it but not beyond its genus.
'Being', no doubt, extends beyond the genus to which 'three'
belongs; but 'odd' extends beyond 'three' but not beyond its
genus.
3%' Such elements we must take till we get a collection ofCOMMENTARY
attributes of which each extends, but all together do not extend,
beyond the subject; that must be the essence of the subject.
hI. We have shown previously that the elements in the 'what'
of a thing are true of it universally, and that universal attributes
of a thing are necessary to it; and attributes taken in the above
manner are elements in the 'what' ; therefore they are necessary
to their subjects.
6. That they are the essence of their subjects is shown as
follows: If this collection of attributes were not the essence of
the subject, it would extend beyond the subject; but it does not.
For we may define the essence of a thing as the last predicate
predicable in the 'what' of the individual instances.
15. In studying a genus one must (I) divide it into its primary
inftmae species, (2) get the definitions of these, (3) get the category
to which the genus belongs, (4) study the special properties in
the light of the common attributes.
:n. For the properties of the things compounded out of the
primary inftmae species will follow from the definitions, because
definition and what is simple is the source of everything, and the
properties belong only to the simple species per se, to the complex
species conseq uen tially.
z5. The method of division according to differentiae is useful
in the following way, and in this alone, for inferring the 'what' of
a thing. (I) It might, no doubt, seem to be taking everything
for granted; but it does make a difference which attribute we
take before another. If every successive species, as we pass
from wide to narrow, contains a generic and a differential element,
we must base on division our assumption of attributes.
35. (2) It is the only safeguard against omitting anything that
belongs to the essence. If we divide a genus not by the primary
alternatives but by alternatives that come lower, not the whole
genus will fall into this division (not every animal, but only every
winged animal, is whole-winged or split-winged). If we divide
gradually we avoid the risk of omitting anything.
97"6. (3) The method is not open to the objection that one who
is defining by division must know everything. Some thinkers
say we cannot know the difference between one thing and others
without knowing each of these, and that we cannot know each
of these without knowing its difference from the original thing;
for two things are or are not the same according as they are or
are not differentiated. But in fact (a) many differences attach,
but not per se, to things identical in kind.
14. And (b) when we take opposites and say 'everything fallsH.13
655
here or here', and assume that the given thing falls in a particular
one of the divisions, and know this one, we need not know all the
other things of which the differentiae are predicated. If one
reaches by this method a class not further differentiated, one has
the definition; and the statement that the given thing must fall
within the division, if the alternatives are exhaustive, is not an
assumption.
z3. To establish a definition by division we must (1) take
essential attributes, (2) arrange them properly, (3) make sure
that we have got them all. (I) is secured by the possibility of
establishing such attributes by the topic of 'genus'.
z8. (2) is secured by taking the first attribute, i.e. that which
is presupposed by all the others; then the first of the remaining
attributes; and so on.
3S' (3) is secured by taking the differentiation that applies to
the whole genus, assuming that one of the opposed differentiae
belongs to the subject, and taking subsequent differentiae till
we reach a species not further differentiable, or rather one which
(including the last differentia) is identical with the complex term
to be defined. Thus there is nothing superfluous, since every
attribute named is essential to the subject; and nothing missing,
since we have the genus and all the differentiae.
b7. In our search we must look first at things exactly like, and
ask what they have in common; then at other things like in
genus to the first set, and in species like one another but unlike
the first set. When we have got what is common to each set, we
ask what they all have in common, till we reach a single definition
which will be the definition of the thing. If we finish with two
or more definitions, clearly what we are inquiring about is not
one thing but more than one.
IS. E.g. we find that certain proud men have in common
resentment of insult, and others have in common indifference to
fortune. If these two qualities have nothing in common, there
are two distinct kinds of pride. But every definition is universal.
z8. It is easier to define the particular than the universal, and
therefore we must pass from the former to the latter; for am-
biguities more easily escape notice in the case of universals than
in that of infimae species. As in demonstrations syllogistic
validity is essential, clearness is essential in definitions; and this
is atta:-ined if we define separately the meaning of a term as
applied in a single genus (e.g. 'like' not in general but in colours
or in shapes, or 'sharp' in sound), and only then pass to the
general meaning, guarding thus against ambiguity. To avoidCOMMENTARY
reasoning in metaphors, we must avoid defining in metaphors
and defining metaphorical terms.
In this chapter A. returns to the subject of definition. In
chs. 3-7 he has considered it aporematically and pointed out
apparent objections to the possibility of ever establishing a defini-
tion of anything. In chs. 8-10 he has pointed out the difference
between the nominal definition, whether of a subject or of an
attribute, and the causal definition of an attribute, and has
shown that, while we cannot demonstrate the definition of an
attribute, we can frame a demonstration which may be recast
into the form of a definition. He has also intimated (93bz1-4) that
a non-causal definition must either be taken for granted or made
known by some method other than demonstration. This method
he now proceeds to expound. In 96az4-b14 he points out that the
definition of a species must consist of those essential attributes
of the species which singly extend beyond it but collectively do
not. In b 15- Z5 he points out that a knowledge of the definitions
of the simplest species of a genus may enable us to deduce the
properties of the more complex species. In bZ5-97b6 he points
out how the method of division, which, considered as an all-
sufficient method, he has criticized in ch. 5, may be used as a
check on the correctness of the application of his own inductive
method. In 97b7-Z9 he points out the importance of defining
species before we define the genus to which they belong.
96 a 2o-2. nws ~Ev o~v ••• 1fpOT€pov. The reference is to chs. 8
and 9. TTW, TO Tt (fTTW ~l, TOV, opov, aTTootooTaL ('is qistributed
among the terms') refers to the doctrine stated in ch. 8 ~bout the
definition of attributes, like eclipse. In the demonstrat~on which
enables us to reach a complete causal definition of an ~ttribute,
the subject which owns the attribute appears as minor term, the
attribute as major term, the cause as middle term; 'the moon
suffers eclipse because it suffers the interposition of the earth.'
28-9. WC71I'€P TO QV ••• o.pL914 is an illustration of the kind of
(TT!. TTMov imapx~w which A. does not mean, i.e. extension not
merely beyond the species but beyond the genus; this is merely
preliminary to his illustration of the kind of (TT!. TTMov imapXHv
he does mean (aZ9-3z).
36--7. TO 1fPWTOV ••• o.pL9~v, i.e. three is primary both in the
sense that it is not a product of two numbers and in the sense
that it is not a sum of two numbers; for in Greek mathematics
I is not a number, but apxTI apL8fLov. Ct. Heath, Mathematics in
Aristotle, 83-4.bl-5. ill'd S, ... Ta.UTa.. The MSS. have in b2 on a.vai"caia p.b.
With this reading 'Ta. Ka86).ov S€ a.vaYKaia spoils the logic of the
passage, since without it we have the syllogism 'Elements in
the "what" are necessary, The attributes we have ascribed to the
number three are elements in its "what", Therefore they are
necessary to it'; 'Ta. Ka86).ov S€ a.vaYKaia contributes nothing to
the proof. The ancient commentators saw this, and say that Sl
must be interpreted as if it were yap. Then we get a prosyllogism
to support the major premiss above: 'Universal attributes are
necessary. (Elements in the "what" are universal,> Therefore
elements in the "what" are necessary.' Sl cannot be interpreted
as yap; but we might read yap for Sl. This, however, would not
cure the sentence; for it is not true that -rfi 'Tpw.S~ •.. ).ap.{3av6p.6'a
has been proved previously (iv 'Toi, avw b2 ). What the structure of
the sentence requires is (I) two general principles that have been
proved already, distinguished by p.lv and Sl, and (z) the applica-
tion of these to the case in hand. The sentence can be cured only
by reading Ka86).ov for a.vaYKaia in b z and supposing the eye of
the writer of the archetype to have been caught by a.vaYKaia in
the line below. We then get: '(I) We have proved (a) that elements
in the "what" are universal, (b) that universal elements are
necessary. (z) The attributes we have ascribed to the number
three are elements in its "what". Therefore (3) these elements
are necessary to the number three.'
The reference in £v 'Toi, avw is to 73"34-7, b zS -8.
1:2. ~lI'L TOLS a.TO .... OLS. The 'Tai, of the MSS. is due to a mechani-
cal repetition of the 'Tai, in bIO. E. 189. I7 has'Toi,.
iaXa.TOS TOLa.UTll Ka.TT\YOpLa.. The form luXa'To, as nom. sing.
fern. is unusual, but occurs in Arat. 6z5. 6z8.
15-25. XPTt SE . . . €1C€LVa.. Most of the commentators hold
that while in aZ4-bI4 A. describes the inductive method of 'hunt-
ing' the definition of an infima species, he here describes its use
in hunting the definition of a subaltern genus, i.e. of a class inter-
mediate between the categories (bIg-ZO) and the infima species.
They take A. to be describing the obtaining of such a definition
inductively, by first dividing the genus into its infima species
(b IS - I7 ), then obtaining inductively the definitions of the infimae
species (b I7 - I9 ), then discovering the category to which the genus
belongs (b I9- z0), and finally discovering the differentiae proper
to the genus (i.e. characterizing the whole of it) by noting those
common to the species (bzo-I ); the last step being justified by
the remark that the attributes of the genus composed of certain
infimae species follow from the definitions of the species, and
uuCOMMENTARY
belong to the genus because they belong directly to the species
(bZI- S). There are great difficulties in this interpretation. (I) The
interpretation put upon Ta iS~a 7Ta(JTJ (J£.wp£.'v S~ TWV /cO~VWV
7TPWTWV (bzo-I) is clearly impossible. The words suggest much
rather the deducing of the peculiar consequential attributes of
different species (7Ta(JTJ suggests these rather than differentiae)
from certain attributes common to all the species. (z) The inter-
pretation of TO', UVVTL(J£ldvo~, £/C TWV (hop.wv (bZI ) as meaning the
genera, and of TOr, a7TAor, (b Z3 ) as meaning the species, while not
impossible, is very unlikely; A. would be much more likely to
call the genus simple and the species complex (cf. IoobZ n.).
uvp.{3a[voVTa, like 7T(i(JTJ, suggests properties rather than differentiae,
and the contrast A. expresses is one between uvp.{3a[voVTa and
0PU7P.O[, not between the opLap.a, of a genus and the opwp.o[ of
its species. It might be objected that a reference to the deduction
of properties would be out of place in a chapter that is concerned
only with the problem of definition; the answer is that while the
chapter as a whole is concerned with definition, this particular
section concerns itself with the question what method of approach
to the problem of definition is the best prelude to the scientific
study of a subject-genus (bIS)-which study will of course aim
(on A.'s principles) at deducing the properties of the genus from
its definition. (3) the immediately following section on the utility
of division (bZS-97b6) is relevant to the defining of infimae species
(CLV(JPW7TO" 96b34), not of genera.
Maier (2 a. 404 n. z) takes TOr, aVVTL(J£p.'VOL, £/C TWV (hop.wv (bZI)
to mean the individuals, the avv(J£'Ta~ ooa[aL, composed of the
infima species + matter ; but this again is unlikely.
Pacius provides the correct interpretation. He supposes Ta
CLTop.a TijJ £iSn Ta 7TpWTa (bI6) to mean not the infimae species of
the genus, in general, but the primary infimae species. His sug-
gestion is that A. has in mind the fact that in certain genera
some species are definitely simpler than others, and is advocating
the study of the definitions of these as an element in the study
of a whole genus-in the attempt to deduce the properties of the
other species from the primary attributes common to the primary
and the complex species (Ta rSLa 7Ta(JTJ (J£.wp£'iv SLa TWV /cOLVWV
7TPWTWV, bzo- I ). A.'s examples agree with this view. Of the
infimae species of number (i.e. the cardinal numbers) he names
only Z and 3, precisely the two that are designated as 7TpwTa
in 83S-b I. Of the species of line he takes the two simplest, the
straight line (that out of which all crooked lines may be said to
be compounded (avvTL(J£p.'VOLS, bZI )) and the circle, which A.11. 13. g6bz6-grII
doubtless thought of as the prototype of all curved lines. Of the
species of angle he names only the right angle, by reference to
which the acute and the obtuse angle are defined. His idea would
then be, for instance, that by studying the definition of the
number two and that of the number three. we shall be able to
deduce the properties of the number six as following from the
definitions of its two factors. A better example for his purpose
would be the triangle, which is the simplest of rectilinear figures,
and from whose definition the properties of all other rectilinear
figures are derived.
26. €iP"lTa.L EV TOl5 1TpOTEPOV, i.e. in ch. 5 and in A n. Pr. i. 31.
32-5- Et ya.p ••. a.tTELaila.L. This sentence is difficult. In b28-
30 A. has pointed out the objection to the Platonic method of
definition by division which he has stated at length in ch. 5-that
it has at each stage to take for granted which of two alternative
differentiae belongs to the subject. In b3 O- 2 he points out that
division is nevertheless useful as securing that the elements in
a definition are stated in proper order, passing continuously from
general to particular. In b32 - 5, though the sentence is introduced
by yap, he seems to be harking back to the objection stated in
b28-30, and the commentators interpret him so; yet he can
scarcely be so inconsequent as this. We must give a different
turn to the meaning of the sentence, by interpreting it as follows:
'if everything consists of a generic and a differential element,
and "animal, tame", as well as containing two such elements, is
a unity, and out of this and a further differentia man (or whatever
else is the resultant unity) is formed, to get a correct definition
we must assume its elements not higgledy-piggledy (Wa7T£p all
£~ £g apxfj. £AaI-L{3all£
all£v Tfj. (Uatp£C1Ew., b29 ) but on the basis
of division: The stress in fact is on Ot£AOI-LfIlOIl, not on a~T£LC10at.
97'6-11. OUSEV SE ••• TOUTOU. T. 58.4, P. 405. 27, E. 202. I7
refer the implicit objection (,you cannot define by the help of
division without knowing all existing things') to Speusippus. An.
584. I7 does the same, and quotes Eudemus as his authority. The
objection may be interpreted in either of two ways. Let A be
the thing we wish to define, and B, C, D the things it is to be
distinguished from. The argument may be (I) 'We cannot know
the differences between A and B, C, D without first knowing
B, C, D; but we cannot know B, C, D without first knowing the
differences between them and A', so that there is a problem like
that of the hen and the egg. Or (2) it may be 'We cannot know
the differences between A and B, C, D without knowing B, C, D;
and we cannot know A without knowing its differences from
n.660
COMMENTARY
B, C, D; therefore we cannot know A without knowing B, C, D.'
The first interpretation has the advantage that it makes EKClOTOV
throughout refer to B, C, D, while the other makes it refer to
B, C, D in b 9 and to A in b10. On the other hand, the second
interpretation relates the argument more closely to the thesis
mentioned in bfr.7, that you cannot know one thing without
knowing everything else.
P. and E. interpret Speusippus' argument as a sceptical attack
on the possibility of definition and of division; but Zeller
(ii. a 4 • 996 11. 2) remarks truly that an eristic attack of this kind
is not in keeping with what we know about Speusippus. His point
seems rather to have been an insistence on the unity of knowledge
and the necessity for a wide knowledge of facts as a basis of
theory. As Cherniss remarks (Ar.'s Criticism of Plato and the
Academy, i. 60), 'for Plato ... the independent existence of the
ideas furnished a goal for the search conducted by means of
"division" which Speusippus no longer had, once he had aban-
doned those entities. Consequently, the essential nature of any
one concept must for him exist solely in its relations of likeness
and difference to every other concept, relations which, while for
the believer in ideas they could be simply necessary implications
of absolute essences, must with the loss of the ideas come to con-
stitute the essential nature of each thing. The principle of Of'0,oTTJ~'
the relations expressed by TaVrov and ETEPOJl, changed then from an
heuristic method to the content of existence itself.' Cf. the whole
passage ib. 59-63 for the difference between the attitudes of Plato,
Speusippll6, and A. to the process of division.
11-14. oil ycip ••• a.uTci, i.e. there are many separable acci-
dents which belong to some members of a species and not to
others, while leaving their definable essence the same.
::&::&. Eill'EP CKELVOU S,a.+Opcl CaT,.' The sense demands not ECTTa,
but £CTT', which seems to have been read by P. (408. 20) and E.
(207. 19): 'if the differentiation is a differentiation of the genus
in question, not of a subordinate genus'.
::&6-8. caT' S€ ••• Ka.Ta.aKEuclaa.,. A. has shown that a defini-
tion cannot be scientifically proved to be correct (chs. 4,7), which
follows from the fact that the connexion between a term and its
definition is immediate. But just as an accident can be estab-
lished by a dialectical syllogism (cf. Top. ii, iii) , so can a definition,
and this can be done Iha TOU YEJlOIIS", i.e. by using the 7011'0' proper
to the establishment of the genus to which the subject belongs
(for which see Top. iv) ; for the differentiae are to be established
by the same T01l'0' as the genus (Top. 101bI7-19).661
31-9. 'l'OU 8E 'l'iAlu'I'au,u • • • 'l'ou...o. The first clause is mis-
leading, since it suggests that in defining any species we must
reach a complex of genus and differentiae that is not further
differentiable. This would be untrue; for it is only if the species
is an infima species that this condition must be fulfilled. The
second clause supplies the necessary correction.
b l _%. 'll"o.vTa ya.p ••• TOUT"'V. '7faVTCl TOVrWV seems to be used,
as E. 212. 32-3 says, in the sense of £KaG'TOV TOVrwv, as we say 'all
of these'. The lexicons and grammars, so far as I know, quote
no parallels to this.
3-4. yivos
OUV ••• 'll"poC7).a~~av6IUvov, i.e. we may treat
as the genus to which the species belongs either the widest genus,
with which we started, or the genus next above the species, got
by combining the widest genus with the subsequently discovered
differentiae.
-rIO. aUTo'Ls .uv TaUTo.. The sense requires aUTO'S", which is
presupposed by E.'s '7fpoS" ci.U7JAa (2I3. 32).
15-%5' olov ).€y", ••• lU)'aAo"'uXia.s. A.'s classical description
of JUilaAor/roXla is in E.N. II23&34-II25"35. He does not there dis-
tinguish two types; but the features of his account which repel
modern sympathies correspond roughly to TO fL~ ci.v'X£C18a, u{3p,Co-
JUVO', and those which attract us to TO ci.S,a,popo, (lva, (VTVXOWr(S"
,
Ka, , aTVXOVVT€S".
.uv
-
17-18. olov Et 'A).K'~'o.S"S ••• (. ALas. This is a nice example
of Fitzgerald's Canon {Wo Fitzgerald, A Selection from the Nic.
Eth. of A. 163-4}, which lays it down that it is A.'s general prac-
tice to use the article before proper names only when they are
names of characters in a book. 0 ~xwUroS" Ka~ 0 Aiels means
'Homer's Achilles and Ajax'. Ct. I. Bywater, Cont. to the Textual
Criticism of A.'s Nic. Eth. 52, and my edition of the Metaphysics,
i, pp. xxxix-xli.
%~. atEt 5' ... ci+OpLC7aS. This goes closely with what has
gone before. Every definition applies universally to its subject;
therefore a definition that applies only to some fL£yaAor/roxol is
not the definition of fL€yaAor/roxia.
28-39. p~6v TE ••• IUTa+OPGo'Ls. In b7-27 A. has shown the
advantage of working from particular instances upwards, in our
search for definition, viz. that it enables us to detect ambiguities
in the word we are seeking to define. Here he makes a similar
point by saying it is easier to work from the definition of the
species {TO Ka8' £Kacrrov, b 2 8} to that of the genus, rather than
vice versa.
33. 5,,, TWV Kat' iKaC7TOY Et)."~~,,,V. In view of b12 we should662
COMMENTARY
read dATJJLJLEVWV, which seems to have been read by E. (220. 33.
221. Il, 222. 14, 18, 25, 36, 223. 13, 21, 22). In the MSS. the com-
moner word replaced the rarer.
34-5. olov TO ollo~ov ... aX';Ila.a~. 'Like' does not mean the
same when applied to colours and when applied to figures
(99 aIl - I S)·
CHAPTER 14
The use of division (b) for the orderly discussion of problems
98al. In order to formulate the propositions to be proved, we
must pick out the divisions of our subject-matter, and do it in
this way: we must assume the genus common to the various
subjects (e.g. animal), and discover which of the attributes belong
to the whole genus. Then we must discover which attributes
belong to the whole of a species immediately below the genus
(e.g. bird), and so on. Thus if A is animal, B the attributes com-
mon to every animal, C, D, E, the species of animal, we know why
B belongs to D, viz. through A. So too with the connexion of
C or E with B. And so too with the attributes proper to classes
lower than A.
13. We must pick out not only common nouns like 'animal'
but also any common attributes such as 'horned', and ask (I)
what subjects have this attribute, and (2) what other attributes
accompany this one. Then the subjects in (I) will have the
attributes in (2) because these subjects are horned.
~O. Another method of selection is by analogy. There is no
one name for a cuttle-fish's pounce, a fish's spine, and an animal's
bone, but they have common properties which imply the posses-
sion of a common nature.
Zabarella maintains that this chapter is concerned with advice
not as to the solution of 7Tpoj3A-ryJLaTa (with which chs. 15-18 are
concerned), but as to their proper formulation; his reason being
that if you say ("g-Il) 'C is B because A ,is Band C is an A',
you are not giving a scientific demonstration because in your
minor premiss and your conclusion the predicate is wider than the
subject. You have not solved the real problem, viz. why B belongs
to A, but have only reduced the improper question why C is B
to the proper form 'why is AB?'
This interpretation might seem to be an ultra-refinement; but
it is justified by A.'s words, 7TpO'i TO £xnv Ta 7Tpoj3A-ryJLaTa. The
object in view is not that of solving the problems, but that ofhaving them in their truly scientific form. What he is doing
in this chapter is to advise the scientific inquirer to have in his
mind a 'Porphyry's tree' of the genera and species included in
his subject-matter, and to discover the widest class, of the whole
of which a certain attribute can be predicated-this widest class
then serving to mediate the attribution of the attribute to classes
included in the widest class. He further points out that some-
times (°1-12) ordinary language furnishes us with a common name
for the subject to which the attribute strictly belongs, sometimes
(aI3-19) it has only a phrase like 'having horns', and sometimes
(a20-3) where several subjects have an attribute in common, we
cannot descry and name the common nature on which this depends
but can only divine its presence. The chapter expresses, though
in very few words, a just sense of the extent to which language
helps us, and of the point at which it fails us, in our search for the
universals on which the possession of common properties depends.
1. During points out in Aristotle's De Partibus A nimalium:
Critical and Literary Commentaries, 109-14, that Aristotle's four
main discussions of the problem of classification-Top. vi. 6,
An. Post. ii. 14, Met. Z. 12, and De Part. An. i. 2-4-Show a
gradual advance from the Platonic method of a priori dichotomy
to one based on empirical study of the facts.
98·I-z. npos SE TO iXE~V .•• EKXEyE~V. In "I Mynv, and in °2
S,a>.t!ynv, is the reading with most MS. support. But A. seems
nowhere else to use S,aMynv, while he often uses £KMy£w (e.g. in
the similar passage An. Pr. 43bII); and £KMynv derives some sup-
port from a20. Further, £KMY£LV . . . oVrw S£ £KMynv would be
an Aristotelian turn of phrase. I therefore read £KMY£LV in both
places, with Bekker.
I. Tel.S TE a.vaTo~as Kal Tas S~a~pE<1US. A. does not elsewhere
use aVaTOJL-rJ or avaT'JLV£LV metaphorically, and Plato does not use
the words at all. But A. once (Met. 1038'28), and Plato once,
(Polit. 261") use TOJL-rJ of logical division, and that is probably
what is meant here, there being no real distinction between
aVaTOJLa~ and S,a,pt!a£L~. Mure suggests that avaTOJL-rJ means 'that
analysis of a subject, for the purpose of eliciting its properties,
which would precede the process of division exhibiting the true
generic character in virtue of which the subject possesses those
properties'. But if A. had meant this, he would probably have
devoted some words to explaining the distinction between the
two things.
T. 59. 15-16, 25-{), P. 417.6-17, E. 224. 21-5 suppose the reference
to be to literal dissection (in which sense A. uses aVaT'JLVEW andCOMMENTARY
elsewhere). But such a reference would not be natural
in a purely logical treatise; it would apply only to biological
problems, not to problems in general, and it is ruled out by the
fact that the words which follow describe a purely logical pro-
cedure.
IZ. E1ri TWV Ka.TW. n's reading KclTW is clearly preferable to
lliwv, which has crept in by repetition from the previous clause.
I~I7. olov TOLs KEpa.Ta. EXaU(TL ••• ElVa.L. In Part. An. 663b31-
664"3 A. explains the fact that animals with horns have no front
teeth in the upper jaw (that is what [.L~ dl4cfJOOVT' ~tva, means;
cf. H.A. 5°1"12-13) as due to the 'law of organic equivalents'
(Ogle, Part. An. ii. 9 n. 9), later formulated by Goethe in the words
'Nature must save in one part in order to spend in another.' In
Part. An. 674"22-bI5 he explains the fact that horned animals
have a third stomach (iXtvos-) by the principle of compensation.
Because they have horns they have not front teeth in both jaws;
and because of this, nature gives them an alternative aid to
digestion.
ava:ro[.L..J
CHAPTER 15
One middle term will often explain several properties
9B"Z4. (1) Problems are'identical in virtue of having the same
middle term. In some cases the causes are the same only in
genus, viz. those that operate in different subjects or in different
ways, and then the problems are the same in genus but different
in species.
29. (2) Other problems differ only in that the middle term of
one falls below that of the other in the causal chain; e.g. why does
the Nile rise in the second half of the month? Because this half
is the stormier. But why is it the stormier? Because the moon
is waning. The stormy weather falls below the waning of the
moon in the causal chain.
In the previous chapter A. has shown that problems of the
form 'why is CB?', 'why is DB?', 'why is E B?' may be reduced
to one by finding a genus A of which C, D, and E are species,
and the whole of which has the attribute B. Here various
problems have a common predicate. In the present chapter he
points out that problems with different predicates (and sometimes
with different subjects) may meet through being soluble (1) by
means of the same middle term, or (2) by means of middle terms
of which one is 'under' the other. (1) ("24--9) dvn1T~pluTa.q,S" (defined665
thus by SimPl. Phys. 1350. 3I-o'VTL7I'~p{CTTauL~ OE iCTTLV o.rav i,w-
80VP.EVOV TLVO~ UWp.aTO~ ~7I'O uwp.aTo~ dVTa.>.>.a~ YEV7JTaL TWV T07l'WV,
Ka~ TO P.& i,w8~uav £V TclJ TOU i,w8'T/8EVTO~ crrfi TOmtl. TO OE i,w8'T/8&
TO 71'pO(T~XE~ i,w8fj Ka~ iK~rVO TO iX6p.~ov, o.rav 71'>'~{ova
£w~ av TO
£UXaTOV £V TclJ TO-rr<tl yiV7JTaL TOU 1Tpcln-OV i'We1}(TaVTo~) might be used
to explain the flight of projectiles (Phys. 215"15, 266bz7-267aI9),
the action of heat and cold on each other (M eteoy. 348bz-349ag),
the mutual succession of rain and drought (ib. 360b3Q-361"3),
the onset of sleep (De Somno 4s7a33-b2, 458825-8); cf. also
Probl. 867b31-3, 909822-6, 962 81-4, 963"5-12. In certain cases, A.
adds (<)88Zs-9), as in that of dvaK>'aaL~ (and the remark would no
n,
doubt apply also to dVTL7I'~pUrTaaL~), the middle term, and there-
fore the problem, is only generically identical, while specifically
different. (2) (829-34) (a) Why does the rising of the Nile (A)
accompany the second half of the month (D)? Because the Nile's
rising (A) accompanies stormy weather (B), and stormy weather
(B) accompanies the second half of the month (D). (b) Why
does stormy weather (B) accompany the second half of the
month (D)? Because stormy weather (B) accompanies a waning
moon (G), and a waning moon (G) accompanies the second half
of the month (D).
9B829-30. Ta. s •••• 1fPOj3>-'1I'a.TIIIV. Ta oi answers to Ta P.Ev in
824, and we therefore expect A. to mention a second type of case
in which two problems 'are the same'. He actually mentions a
type of case in which two problems differ. But the carelessness
is natural enough, since in fact the two problems are partly the
same, partly different.
It will be seen from the formulation given above that the
middle term used in solving the first problem (B) is in the chain
of predication 'above' that used in solving the second (G), i.e.
predicable of it (TO B ~apX~L TclJ r, A. would say). But when A.
says (829-30) TclJ TO P.EUOV V7I'O TO £upov p.iaov ~rvaL he is probably
thinking of the p.iuov of the first problem as falling below that of
the second. ~o TO £upOV p.Eaov means not 'below the other
middle term in the chain of predication' but 'below it in the chain
of causation'; a waning moon produces stormy weather.
32. b I'E£S. This form, which n has here, is apparently the
only form of the nominative singular that occurs in A. (G.A.
777b23) or in Plato (Grat. 409 c 5, Tim. 39 c 3).666
COMMENTARY
CHAPTER 16
Where there is an attribute commensurate with a certain subject,
there must be a cause commensurate with the attribute
98a35. Must the cause be present when the effect is (since if the
supposed cause is not present, the cause must be something else) ;
and must the effect be present when the cause is?
b4. If each entails the other, each can be used to prove the
existence of the other. If the effect necessarily accompanies the
cause, and the cause the subject, the effect necessarily accom-
panies the subject. And if the effect accompanies the subject,
and the cause the effect, the cause accompanies the subject.
16. But since two things cannot be causes of each other (for
the cause is prior to the effect; e.g. the interposition of the earth
is the cause of lunar eclipse and not vice versa), then since proof
by means of the cause is proof of the reasoned fact, and proof
by means of the effect is proof of the brute fact, one who uses the
latter knows that the cause is present but not why it is. That
eclipse is not the cause of the interposition of the earth, but vice
versa, is shown by the fact that the latter is included in the
definition of the former, so that evidently the former is known
through the latter and not viCe versa.
25. Or can there be more than one cause of the same thing?
If the same thing can be asserted immediately of more than one
thing, e.g. A of B and of C, and B of D, and C of E, then A will
belong to D and E, and the respective causes will be B and C.
Thus when the cause is present the effect must be, but when the
effect is present a cause of it but not every cause of it must be
present.
32. No: since a problem is always universal, the cause must
be a whole and the effect commensurately universal. E.g. the
shedding of leaves is assigned to a certain whole, and if there are
species of this, it is assigned to these universally, to plants or to
plants of a certain kind, and therefore the middle term and the
effect must be coextensive. If trees shed their leaves because of
the congealing of the sap, then if a tree sheds its leaves there must
be congealing, and if there is congealing (sc. in a tree) the leaves
must be shed.
98a35-b4. nEpL S' ah(ou . . . cl>UAAOppOEL. This passage is re-
duced to order by treating Wa7T€P d ... av-rwv as parenthetical,
and the rest of the sentence as asking two questions, Does effectentail cause? and Does cause entail effect? If both these things
are true, it follows that the existence of each can be proved from
the existence of the other (b4- S).
bl 6-ZI. ~L 5E ••• 01i. Bonitz (Ar. Stud. ii, iii, 79) is right in
pointing out that this is one sentence, with a colon or dash (not,
as in the editions, a full stop) before (L in bI9 . The parenthesis
ends with £KA£{7T£w (b I9 ), not with ainov (bI7).
17. TO ya.p atTLov •.. atTLov. 7TpOT£pOI' means 'prior in nature',
not 'prior in time'; for A. holds that there are causes that are
simultaneous with their effects; cf. 9S'I4-24.
22-3. EV ya.p T~ My~ ... f1€C7~, cf. 93 b 3-7.
25-31. "H Ev5EXETa~ ••• OU f1EVTO~ .".av. A. raises here the pro-
blem whether there can be plurality of causes, and tentatively
answers it in the affirmative. Ka~ yap EL (b 2S ) does not mean 'for
even if': it means 'yes, and if', as in examples from dialogue
quoted in Denniston, The Greek Particles, 10<)-10. The content of
b 2S - 3I , summarized, is 'Can there be more than one cause of
one effect? Yes, and if the same predicate can be affirmed im-
mediately of more than one subject, this must be so.'
32-8. 11 €t ci€t .•• cjIUAAOPPO€'iv. This is A.'s real answer to the
question whether there can be plurality of causes. A 'problem',
i.e. a proposition such as science seeks to establish, is always
universal, in the sense explained in i. 4, viz. that the predicate
is true of the subject KaTa 7TaI'TO" Ka8' aUTO, and fI aUTO (in virtue
of the subject's being precisely what it is). It follows that the
premises must be universal; the cause, which is the subject of
the major premiss, must be oAol' n, the whole and sole cause of
the effect, which must in turn attach to it Ka8oAov (b32 - 3). E.g.
if we ask what is the cause of deciduousness, we imply that there
is a class of things the whole of which, and nothing but which,
suffers this effect, and therefore that there is a cause which ex-
plains the suffering of this effect by this whole class and by noth-
ing else, and must therefore be coextensive with the effect (b 3S -6).
Thus a system of propositions such as is suggested in '26--9 cannot
form a scientific demonstration. A cannot be a commensurately
universal predicate of Band r, but only of something that in-
cludes them both, say Z; and this will not be a commensurately
universal predicate of.1 and E, but only of that which includes
them both, say H; the demonstration will be 'All Z and nothing
else is A, All H and nothing else is Z, Therefore all H and nothing
else is A' ; and we shall have proved not only that but also pre-
cisely why all H and nothing else is A.668
COMMENTARY
CHAPTERS 17, 18
Differmt causes may produce the same effect, but not in things
specifically the same
99"1. Can there be more than one cause of the occurrence of
an attribute in all the subjects in which it occurs? If there is
scientific proof, there cannot; if the proof is from a sign or per
accidens, there can. We may connect the attribute with the
subject by means of a concomitant of either; but that is not
regarded as scientific. If we argue otherwise than from a con-
comitant, the middle term will correspond to the major: (a) If
the major is ambiguous, so is the middle term. (b) If the major
is a generic property asserted of one of the species to which it
belongs, so is the middle term.
8. Example of (b).
11. Example of (a).
15. (c) If the major term is one by analogy, so is the middle
term.
16. The effect is wider than each of the things of which it can
be asserted, but coextensive with all together; and so is the
middle term. The middle term is the definition of the major
(which is why the sciences depend on definition).
z5. The middle term next to the major is its definition. For
there will be a middle term next to the particular su bjects, assigning
a certain characteristic to them, and a middle connecting this
with the major.
30. Schematic account. Suppose A to belong to B, and B to
belong to all the species of D but extend beyond each of them.
Then B will be universal in relation to the several species of D
(for an attribute with which a subject is not convertible may be
universal to it. though only one with which the subject as a whole
is convertible is a primary universal to it). and the cause of their
being A. So A must be wider than B; else A might as well be
the cause of the species of D being B.
37. If now all the species of E have the attribute A, there will
be a term C which connects them with it. Thus there may be
more than one term explaining th~ occurrence of the same
attribute, but not its occurrence in subjects specifically the same.
b7. If we do not come forthwith to immediate propositions-
if there are consecutive middle terms-there will be consecutive
causes. Which of these is the cause of the particular subject's
having the major as an attribute? Clearly the cause nearest toI1. I7, I8
669
the subject. If you have four terms D, C, B, A (reading from
minor to major), C is the cause of D's having B, and therefore of
its having A; B is the cause of Cs having A and of its own
having A.
The question raised and answered in this chapter is the same
that has been raised and answered in 98bZ5-38, and it would seem
that the two passages are alternative drafts, of which the second
is the fuller and more complete. A. answers, as in 98b3z-8, that
where there is a genuine demonstration of an attribute A as
following from an element B in the nature of a subject C, only
one cause can appear as middle term, viz. that which is the
definition of the attribute; his meaning may be seen by reference to
ch. 8, where he shows that, for example, the term 'interposition of
the earth', which serves to explain the moon's suffering eclipse,
becomes an element in the definition of lunar eclipse. He admits,
however, that there are arguments in which the subject's posses-
sion of a single attribute may be proved by means of different
middle terms. An obvious case is proof Ka7"cl C17]po(rOV (99 a 3); A
may have several consequences, and any of these may be used
to prove Cs possession of A (though of course it does not explain
it); cf. 93 3 37·_ b 3 and An. Pr. ii. 27. Another case is proof Ka7"a
UVpof3(f3TJKOS; both the attribute and the subject may be considered
Ka7"a UVpof3(f3TJKOS: (°4-5) ; C may be shown to possess A because it
possesses an inseparable concomitant of A, or because an in-
separable concomitant of C entails A, and of course a variety of
concomitants may be thus used. ov po~v OOK(' (A. continues)
1Tpof3).~poa7"a (Cva, (,these, however, are not thought to be scientific
treatments of the problem'). (l oE p.~, opoo'ws ;~(' 7"0 poEClOV. (l oE
po~ is taken by the commentators to mean (l OE po~ OV OOK('
1Tpof3).~poa7"a (Cva" 'if such treatments of the problem are admitted' ;
and what follows in a6-I6 is taken to offer various types of argu-
ment Ka7"cl UVpof3"f3TJKOS. But if so, the logic of the passage would
require them to be arguments in which a single effect is proved
to exist by the use of more than one middle term. What A. asserts,
however, is that in the three cases he discusses (a7, 7-8, I5-I6) the
middle term used has precisely the kind of unity that the effect
proved has. I infer that the three cases are not put forward as
cases of proof Karcl UVpof3(f3TJKOS, and that (l oE po~ means 'if we
study not Ka7"cl ClVpof3(f3TJKOS: the o~ atnov or the c{I atnov'.
The three cases, then, are cases which might seem to show that
there can be more than one cause of the same effect, but do not
really do so. They are as follows: (a) We may be considering notCOMMENTARY
one effect but two effects called by the same name, or (b) (ws- El'
'YEVH, "7) the major may be predicable of a whole genus, and we
may be asking why it is predicable of various species of the genus.
Case (b) is illustrated first (B8-II). All proportions between
quantities are convertible alternando (i.e. if a is to b as c is to d,
a is to c as b is to d). If we ask not why all proportions between
quantities are convertible, but why proportions between lines,
and again why proportions between numbers, are convertible (a
procedure which in 74"17-25 A. describes as having been followed
by the earlier mathematicians), there is a misfit between subject
and predicate. There is a single reason why all proportions are
convertible, consisting in the attribute, common to all quantities,
of bearing definite ratios to quantities of the same kind (V EXOV
aV~TJatv TotavBl, aIO). But if we ask why proportions between lines
are convertible, we shall use a middle term following from the
nature of lines, and if we ask why proportions between numbers are
convertible, a middle term following from the nature of numbers.
A. now (all) turns to case (a). Similarity between colours is not
the same thing as similarity between figures; they are two things
with a single name; and it is only to be expected that the middle
term used to prove that two colours are similar will be different
from that used to prove that two figures are similar; and if the
two middle terms are called by the same name, that also will be
a case of ambiguity.
Finally (c.) (ar5-r6), when two effects are analogous, i.e. when
they are neither two quite different things called by the same
name, nor yet two species of the same genus, but something
between the tw<r-when the resemblance between two things is
one of function or relation, not of inherent nature or structure
(bone, for example, playing the same part in animals that fish-
spine does in fishes, 98"20-3), there will naturally be two causes
which also are related by analogy. (For oneness by analogy as
something more than unity of name and less than unity of nature
cf. Met. ror6b3I-I017a3, E.N. 1096b25-8.)
A consequential attribute, A. continues (BI8), is wider than
each species of its proper subject but equal to all together.
Having external angles equal to four right angles, which has as its
proper subject 'all rectilinear figures', is wider than triangle or
square but coextensive with all rectilinear figures taken together
(for these are just those that have that attribute), and so is the
middle term by which the attribute is proved. In fact the middle
term is the definition of the major (for A.'s proof of this as regards
the middle term by which a physical effect is explained, cf. ch. 8,H. 17, 18
and for his attempt to show that the same is true of the middle
term in a mathematical proof cf. 94"24-35); and that is why all
the sciences depend on definitions-viz. since they have to use
the definitions of their major terms as middle terms to connect
their major terms with their minor terms ("21-3). (For the part
played by definitions among the apxal of science cf. 72814-24.)
To the mathematical example A. adp.s a biological one.
Deciduousness extends beyond the vine or the fig-tree, but is co-
extensive with all the species of deciduous trees taken together.
He adds the further point, that in this case two middle terms
intervene between the vine or fig-tree and deciduousness. The
vine and fig-tree shed their leaves because they are both of a
certain class, sc. broad-leaved (98b4), but there is a middle term
between 'broad-leaved' and 'deciduous', viz. 'having the sap con-
gealed at the junction of the leaf-stalk with the stem'. The latter
is the 'first middle term', counting from the attribute to be ex-
plained, and is its definition; the former is the 'first in the other
direction', counting from the particular subjects (99325-8). Thus
there are two syllogisms: (I) All trees in which the sap is con-
gealed, etc., are deciduous, All broad-leaved trees have their sap
congealed, etc., Therefore all broad-leaved trees are deciduous. (2)
All broad-leaved trees are deciduous, The vine is (or the vine,
the fig-tree, etc., are) broad-leaved, Therefore the vine is (or
the vine, the fig-tree, etc., are) deciduous. In syllogism (I) all
the propositions are genuine scientific propositions and their
terms are convertible. In syllogism (2) the minor premiss and the
conclusion, in either of their forms, are not scientific universals;
for the vine is not the only broad-leaved tree, and 'the vine, the
fig-tree, etc.', are not one species but an aggregate of species;
but if we enumerate all the species of broad-leaved trees both
the minor premiss and the conclusion will be convertible.
A. now ("30) proposes to exhibit in schematic form (J1TL 'TWV
UX"IJLCLTwv) the correspondence of cause and effect. But he
actually gives a formula which seems to fit quite a different type
of case, viz. that previously outlined in 98b25-3I. He envisages
two syllogisms, parallel, not consecutive like the two in 99"23--9.
(I) All B is A, All the species of Dare B, Therefore all the species
of Dare A. (2) All C is A, All the species of E are C, Therefore
all the species of E are A. Thus he omits altogether the single
definitory middle term which he insisted on above. He is taking
jar granted two syllogisms which connect Band C respectively
with A through a middle term definitory of A, and is drawing
attention to the later stage only.COMMENTARY
The general upshot of the chapter is that, to explain the occur-
rence of an attribute, wherever it occurs, there must be a single
middle term 'next' the attribute, which is the defmition of the
attribute and therefore coextensive with it; there may also be
alternative middle terms connecting different subjects with the
definitory middle term and therefore with the attribute to be
explained (825-8). Thus in a sense there is and in a sense there is
not plurality of causes.
99"13-14. Even I'EV yap •.• ywvlns. This is Euclid's defmition of
similality (El. vi, def. 1). As Heiberg remarks (Abh.zurGesch.d.
M ath. Wissensch. xviii. 9), A.'s tentative taw, may indicate that
the definition had not found its way into the text-books of his
time.
19-20. orOV TO TETTnpaw ••• taov. cf. 85b38-86aI n.
20-1. oan yap ••• i~w. 'for all the subjects taken together are
ex hypothesi identical with all the figures whose external angles
equal four right angles'. This must be printed as parenthetical.
~9. ~v Tn aUVo.ljIEL TOU a1l'EPl'nToS. P. 430. 9 says TO aTTlpp.a
means TO aKpov TOU oxavov (presumably =' channel for sap, akin
to 0X€T(l!>-a usage of OXavov not mentioned in L. and S.), Ka8'
o avva7TT€Ta~ TcjJ q,tJAA<p. aTT/pp.a a€ ).lY€Ta~ TO aKpov at« TO EYK€ta8a~
EV aUTcjJ TiJv aTT€pp.aTtK~V dpxJ/v Ka, otJvap.~v, E, 1j, q,tJ€Ta~ TO q,tJAAov.
E. 248. 16 says 0 yap OTTO, o&ro, a.p.a p.€V Tp/q,€t TO q,tJ).).ov o~a TOU
oxavov KaL 8ill€tv TTO~€t, ap.a KaL TcjJ o/vop<p aUTO TTpoaKoAA~.
30. WSE Q.1I'oSwaEL. 'the thing will work out thus'; cf. the in-
transitive use of d7Too~oova~ in Meteor. 363&11, H.A. 585b32, 586"2,
G.A. 722"8, Met. 1057&8.
32-5. TO I'EV S~ B •.• 1I'npEKTElVEL. B will be Ka8o).ov, predicable
KaTa TTavro, and Ka8' alho of each of the D's, but TTPWTOV Ka8o).ov,
i.e. predicable also fJ aUTO (to use the language of i. 4) only of
D as a whole.
33. TOUTO yap AEYW Ka.90Aou ~ I'~ a.VTLaTpEc!>EL. <p (instead of
the usual reading 0) is required (1) by parallelism with the next
clause, and (2) by the fact that when A. wishes to say 'the pro-
position "B is A" is convertible', he says TO B aVTWTp/q,€t TcjJ A,
not vice versa. Cf. Cat. 2b 2I, An. Pr. 31"31, 5134, 52b8, 67 b 37. The
first hand of B seems to have had the right reading. So also
E. 251. 7 TTPO, o.
35-6. KnL 1I'npEKTElVEL ••• ~1I'L 1rAEOV TOU B E1I'EKTElVELV. In 836
the MSS. and P. have TTap€KT£LV€~V, but this is difficult to accept,
because in 835 TTap£KT€LV€t must mean 'are coextensive'. Zabarella
says that in "35 some MSS. have KaL p.~ TTap£KT£Lv€t, and takes this
to mean 'and do not extend beyond'. But that does not give theI!. I7. 99"I3-I8. 99bII
right sense; there is no question of the subspecies of D collectively
extending beyond B-the point is that B does not extend beyond
them. Besides, the natural meaning of TTapEKTE{vELV is 'to be co-
extensive' (L. and S., sense iii). It is TTapEKTElvELV in 836 that is
difficult; L. and S. quote no other example of the sense 'extend
beyond'. To avoid interpreting the word differently in the two
lines, Mure supposes that TOUTO yap • • • oE aVTtuTpEt/>EL (&33-5)
should be read as a parenthesis, and Ka, TTapEKTElvEL coupled with
Ka86>.ov av EtTJ TOL • .J (a33). But this gives an unnatural sentence;
and we should then expect 7TapEKTEtvEL SE. The passage is
best cured by reading ETTEKTELVELV (or VrrEPEKTE{VELV) in &36; ETTEKTEl-
VELV E7T' 7TAEov occurs in 96a24. The corruption is clearly one that
might easily have occurred.
3e,....,. 8€L clpCl ••• €K€LVOU; This is a very careless inference.
A. recognizes causes coextensive with their effects (i.e. the causes
which are definitions of their effects (cf. 98b32-8)); and clearly
as between two coextensive events priority of date would suffice
to establish which alone could be the cause of the other.
bZ• otov [TO A] ... A. Hayduck's emendations will be found in
his Obs. Crit. in aliquos locos Arist. IS. TO A seems to me more
likely to have come in by intrusion from the previous line.
n~~' clpCl, as Bonitz's Index says, has the force enunciati
modeste vel dubitanter affirmantis.
7-8. Et 8€ ••• Tn ClLTLCl 1T~"LW. This starts a topic distinct from
that discussed in '30-b7 (though broached in &25--{)) , and connected
with what follows, which should never have been treated as a
separate chapter. The sentence has been connected with what
precedes by some editor who thought TO ClTOfLOV meant TO aTOfLOv
ElOo., and connected it in thought with TaL. aUToL. To/ EiSEL (b4).
But El. TO aTOfLOv means 'to the immediate proposition', and the
clause means 'if the oLCil7T7}fL,a between the subject and the effect
to be explained cannot be bridged by two immediate propositions'.
11. TO €YYUTClTCl should be read, instead of Ta Eyyu-raTa, which
is a natural corruption. EYYVTaTa is the superlative of the adverb;
cf. To/ EyyvTaTa, 9886.
CHAPTER 19
How we come by the apprehension of first principles
99bIS. We have described what syllogism and demonstration
(or demonstrative science) are and how they are produced; we
have now to consider how the first principles come to be known
and what is the faculty that knows them.
4985
xxCOMMENTARY
20. We have said that demonstrative science is impossible
without knowledge of the first principles. The questions arise (1)
whether these are objects of science, as the conclusions from them
are, or of some other faculty, and (2) whether such faculty comes
into being or is present from the start without being recognized.
26. (2) It would be strange if we possessed knowledge superior
to demonstration without knowing it. On the other hand, we
cannot acquire it, any more than demonstration, without pre-
existing knowledge. So we can neither possess it all along, nor
acquire it unless we already have some faculty of knowledge. It
follows that we must start with some faculty, but not one
superior to that by which we know first principles and that by
which we know the conclusions from them.
34. Such a faculty all animals have-an innate faculty of dis-
cernment, viz. perception. And in some animals perceptions
persist. There is no knowledge outside the moment of perception,
for animals in which perceptions do not persist, or about things
about which they do not persist; but in some animals, when they
have perceived, there is a power of retention. And from many
such acts of retention there arises in some animals the forming of
a conception.
100"3. Thus from perception arises memory, and from repeated
memory of the same thing experience. And from experience-
i.e. when the whole universal has come to rest in the soul-the
one distinct from the many and identical in all its instances-
there comes the beginning of art and science-of art if the
concern is with becoming, of science if with what is.
10. Thus these states of knowledge are neither innate in a
determinate form, nor come from more cognitive states of mind,
but from perception; as when after a rout one man makes a
stand and then another, till the rally goes right back to where
the rout started. The soul is so constituted as to be capable of
this.
14. To be more precise: when an infima species has made
a stand, the earliest universal is present in the soul (for while
what we perceive is an individual, the faculty of perception
is of the universal-of man, not of the man Callias); again a
stand is made among these, till we reach the unanalysable con-
cepts, the true universals-we pass from 'such and such a kind
of animal' to 'animal', and from 'animal' to something higher.
Clearly, then, it is by induction that we come to know the first
principles; for that is how perception, also, implants the universal
in
11S.11. 19
bS. (1) Now (a) of the thinking states by which we grasp truth
some (science and intuitive reason) are always true, while others
(e.g. opinion and calculation) admit of falsity, and no state is
superior to science except intuitive reason; and (b) the first prin-
ciples are more knowable than the conclusions from them, and
all science involves the drawing of conclusions. From (b) it
follows that it is not science that grasps the first principles; and
then from (a) it follows that it must be intuitive reason that does
so. This follows also from the fact that demonstration cannot be
the source of demonstration, and therefore science cannot be
the source of science; if, then, intuitive reason is the only neces-
sarily true state other than science, it must be the source of
science. It apprehends the first principle, and science as a whole
grasps the whole subject of study.
The apxat, with the knowledge of which this chapter is con-
cerned, are the premises from which science or demonstration
starts, and these have been classified in 72"14-24. They include
(1) a~£C.ofLaTa or Ko£va~ apxal. These in turn include (a) principles
which apply to everything that is, i.e. the law of contradiction
and that of excluded middle; and (b) principles valid of every-
thing in a particular category, such as the principle (common to
all quantities) that the whole is greater than the part and equal
to the sum of its parts. (a) and (b) are not distinguished in 72"14-
24 but are distinguished elsewhere. Secondly (2) there are BECTE£S'
or r8£a£ apxal, which in turn are subdivided into (a) OP£CTfLOL,
nominal definitions of all the terms used in the given science, and
(b) lJ1ToBECTE£S', assumptions of the existence of things corresponding
to the primary terms of the given science.
All of these are propositions, while the process described in
99b3S-1oobS seems to be concerned with the formation of universal
concepts (cf. the examples avOpc.v1ToS', ~cpov in l00 bl-3). It would
not be difficult to argue that the formation of general concepts
and the grasping of universal propositions are inseparably inter-
woven. But A. makes no attempt to show that the two pro-
cesses are so interwoven; and he could hardly have dispensed
with some argument to this effect if he had meant to say that they
are so interwoven. Rather he seems to describe the two processes
as distinct, and alike only in being inductive. 8~Aov 8~ ~T£ ~fL'V
Ta 1TpCna t1Taywyfj YVWPL~ELV avaYKaLOV' Ka~ yap Ka~ ~ arCT~(1£S'
OVTW TO Ka86Aov tfLr.O£EL (l00b3 ).
The passage describing the advance from apprehension of the
particular to that of the universal should be compared with Met.
XX2COMMENTARY
980127--981"12, where the fonnation of universal judgements is
definitely referred to (TO fL£1' yap £XHV lmO).T}tPtV OTt Ka».tf!- KafLVOVTt
'"Iv~)1 ,",I' vOO'ov T081 avV1)VE')IKI; Ka1 I:wKpauL Ka1 Ka8' £KaO"TOI' oVrw
7TO).).OrS, £fL7THptac; £O'Ttl'· TO 8' OTt 7TaO't TOrS TOtO r0'81; KaT' I; l80s £V
d~opt0'81;rO't, Ka.fLVOVO'L '"Iv81 ,",V VOO'OI', O'VV~VI;YKI;I', orOV TOrS ~).I;YfLa­
Tw8wtv 1} XO).W8WL [1}) 7TVP£'T'TOVO'L KauO'/fJ, T£XVT}S, 981 "7-12), while
much of what A. says is equally applicable to the fonnation of
general concepts.
99bI9. 'I1'poa'l1'Op"UaUL 'I1'pWTOV. This refers to b22- 34 below. Of
the questions raised in b22 -6 the last, 7TOTI;POI' OUK £vovO'a, at £,HS
Eyytl'ol'TaL 1} £vovO'aL ).I;).1/8aO'tl', is discussed in b2 6-IOO b S ; the answers
to the other questions are given in IOObS-q.
21. ELP'lTaL 'I1'pOTEPOV, in i. 2.
24. 11 ov is clearly superfluous, and there is no trace of it in
P. 433. 8-12 or in E. 260. 28-30.
30. WU'I1'EP Kat ('11'1 TijS ci1roliE(~EWS ~AEYO~EV, in 7 rar-I I.
39. aLu80~oLs seems to be a necessary emendation of al0'8avo-
cf. An. 600. 10.
100"2-3. TOL~ ~EV ••• ~ovij~. Presumably A. thinks this true
only of man. But in M et. 980b2r-s he draws a distinction among
fL£I'OtS;
the animals lower than man. Some do not advance beyond
memory, and even these can be ~poI'LfLa; but those that have
hearing as well go beyond this and are capable of learning from
experience.
4-S. EK liE ~v"~1J~ ••• E~1rELp(a. On A.'s conception of memory
I may be allowed to quote from my edition of the Metaphysics
(i. u6-q). 'It is not easy to see what Aristotle wants to say
about £fL7THpta, the connecting link between memory and art or
science. Animals have a little of it; on the other hand it involves
thought (98r"6). In principle it seems not to differ from memory.
If you have many memories of the same object you will have
£fL7THpta; those animals, then, which have good memories will
occasionally have it, and men will constantly have it. After
having described it, however, as produced by many memories
of the same object, Aristotle proceeds to describe it as embracing
a memory about Callias and a memory about Socrates. These
are not the same object, but only instances of the same universal;
say, 'phlegmatic persons suffering from fever'. An animal, or a
man possessing only £fL7THpta, acts on such memories, and is
unconsciously affected by the identical element in the different
objects. But in man a new activity sometimes occurs, which
never occurs in the lower animals. A man may grasp the universal
of which Calli as and Socrates are instances, and may give to athird patient the remedy which helped them, knowing that he is
doing so because the third patient shares their general character.
This is art or science-for here these two are not distinguished
by Aristotle.
'What is revived by memory has previously been experienced
as a unit. Experience, on the other hand, is a coagulation of
memories; what is active in present consciousness in virtue of
experience has not been experienced together. Therefore (a) as
embodying the data of unconsciously selected awarenesses it fore-
shadows a universal; but (b) as not conscious of what in the past
is relevant, and why, it is not aware of it as universal. I.e.
experience is a stage in which there has appeared ability to inter-
pret the present in the light of the past, but an ability which
cannot account for itself; when it accounts for itself it becomes
art:
6-1. 11 €K 1TC1VTDS ••• IjIuxft. The passage contains a remini-
scence of PI. Phaedo 96 b 0 S' iYKErpaAO<; icmv .) 'Td.<; aia8~aEl<;
1TapEXwv ..• iK 'TOVTWV oE ytYVOt'TO fL~fL7] Kal oo~a, iK oE fL~fL7]<; Kal
00'7]<; Aa{3oVC17J<; 'TO ~PEfLE'V, Ka'Td. 'Taii'Ta ytyvEa8at i1TlC1'T~fL7]v.
7. TOU EVDs 1TClpa Ta 1TO~~c1, not 'existing apart from the
many' (for it is iv a1TaatV iKEtvOt<;), but 'distinct from the many'.
13. €WS €1Ti cipXTJv ~M€v. It has been much debated whether
apX17 here means 'rule' (or 'discipline') or 'beginning'. I doubt
whether the words can mean 'returns to a state of discipline',
though inT' apxTJv ~A8EV could well have meant that. P. seems to
be right in thinking (436. 23-9) that the meaning is 'until the
process of rallying reaches the point at which the rout began';
Zabarella accepts this interpretation, which derives support from
a comparison with M eteoT. 341 b 2 8 (about meteors) id.v fLEV 1TAEOV 'TO
iJ1TEKKavfLa -n Ka'Td. 'TO fLTJKO<; ~ 'TO 1TAU'TO<;, o'Tav fLf.V ofov d1Toa1Tlv87]pt'rJ
afLa KatofLEvoV ('Toii'To oE ytyvE'Tat Otd. 'TO napEK1Tvpoua8at, Ka'Td. fLtKPd.
fLEV, in' apxTJv OE), at~ KaAEtTat, where i1T' apX17v seems to mean
'continuously with that from which the process of taking fire
began'.
14. 0 6' €~€X8'1 I'€V 1Tc1~Cl' refers to "6-7. 1TaAat can refer to
a passage not much previous to that in which it occurs, e.g. P hys.
254"16 referring to 252"5-32, Pol. 1262b29 referring to "24, 1282"15
referring to 1281a39-bII. L. and S. recognize 'just now' as a
legitimate sense of 1TUAat.
IS. TWV ci6,ClCPOPWV, i.e. of the not further differentiable species,
the infimae species; cf. 97"37 'TOU oE nAEV'Tatov fL7]KETI Elvat otarpopuv.
I~bI. KCli ya.p Clta86.v€TCl' • • • KCl~~lOU civ8pW1TOU.
These
words serve to explain how it is that the 'standing still' of anCOMMENTARY
individual thing before the memory is at the same time the llrst
grasping of a universal; this is made easier to understand by the
fact that even at an earlier stage----that of perception (KaL yap
ala8av£'TaL)-the awareness of an individual is at the same time
awareness of a universal present in the individual; we perceive an
individual thing, but what we perceive in it is a set of qualities
each of which can belong to other individual things.
bZ • €WS QV ••• Ka96Aou. The reaching of 'Ta ap.£pij is described
as the culmination of the process, so that 'Ta ap.£pij cannot mean
universals in general, but only the widest universals, the cate-
gories, which alone cannot be resolved into the elements of genus
and differentia; and 'Ta Ka86Aov must be used as synonymous with
'Ta ap.£pij, i.e. as standing for the universals par excellence, the
most universal universals. For ap.£pij in this sense cf. Met. 10141>6
o8£v £A1jAvlh 'Ta p.cfAw-ra Ka86Aov aTOLx£ta £[vaL, on tKaaTOV aVruJV ;v
QV KaL G.1TAoilv £V 1TOAAOtS- inrapX£L •.. £1T£L oov 'Ta KaAOop.£va y£VT)
(i.e. the highest y£V1]) Ka86Aov KaL aSLaLp£'Ta (ov yap iaTL A6yos-
aVrwv) , aTOL)(£ta 'Ta y£vY} MyovaL nv£s-, KaL p.a).).ov 1) rTJv 8w/>opav
on Ka86Aov p.aJ..AOV 'TO y£vos-, 1023 b22 in 'Ta £V 'To/ A6y<fJ 'To/ SY}>'oilV'TL
tKaa'TOV, KaL 'TaiITa p. 6 p L a 'Toil OAOV' SLO 'TO y"'os- 'Toil £ ,SovS" KaL
p.£pos- MY£'TaL, 1084b14 a>'A' aSLaLp£'TOV Kat 'TO Ka86Aov. In Met.
994b21 'Ta (iTop.a is used of the highest universals.
16-17. " SE 'lTaua •.• 'lTpa.Yl'a, i.e. science as a whole grasps
its objects with the same certainty with which intuitive reason
grasps the first principles.
