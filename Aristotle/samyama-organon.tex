ANALYTICA PRIORA

BOOK I

CHAPTER 1

Subject and scope of the Analytics.

Definitions of fundamental terms

Our first task is to state our subject, which is demonstration;
next we must define certain terms.

A premiss is an affirmative or negative statement of
something about something.

A universal statement is one which says that
something belongs to every, or to no, so-and-so;

a particular statement says that
something belongs to some, or does not belong to some, so-and-so;

an indefinite statement says
that something belongs to so-and-so,
without specifying whether it is to all or to some.

A demonstrative premiss differs from a dialectical one, in
that the former is the assumption of one of two contradictories,
while the latter asks which of the two the opponent admits; but
this makes no difference to the conclusion's being drawn, since in
either case something is assumed to belong, or not to belong, to
something.

Thus a syllogistic premiss is just the affirmation or denial
of something about something; a demonstrative premiss must in
addition be true, and derived from the original assumptions; a
dialectical premiss is, when one is inquiring, the ~sking of a pair
of contradictories, and when one is inferring, the assumption of
what is apparent and probable.

A term is that into which a premiss is analysed (a subject or predicate),
 'is' or 'is not' being tacked on to the terms.

A syllogism is a form of speech in which,
certain things being laid down,
something follows of necessity from them, because of them,
without any further term being needed to justify the conclusion.

A perfect syllogism is one that needs nothing other than the
premisses to make the conclusion evident;
an imperfect syllogism needs one or more other statements
which are necessitated by the given terms
but have not been assumed by way of premisses.

For B to be in A as in a whole is
the same as for A to be predicated of all B.
A is predicated of all B
when there is no B of which A will not be stated;
'predicated of no B' has a corresponding meaning.

