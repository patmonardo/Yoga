SK 22.

Prakriti mahat tata ahamkara tasmat ca sodasaka gana
sodasakat tasmad api pancabhyah panca bhutani

From nature evolves the Great Principle;
from this evolves the I Principle;
from this evolves the set of sixteen;
from the five of this set of sixteen evolves the five elements.

SK 23.

buddhi adhyavasaya dharma jnana viraga aisvarya
etad rupam sattvikam tasmat viparyastam tamasam

Buddhi is self-certainty.
Virtue, knowledge, dispassion, and power are its manifestations
when the sattva attribute abounds.
And the reverse of these, when the tamasa attribute abounds.

YS II.44

svadhyaya ista-devata-samprayoga

YS II.45

samadhi-siddhi îsvara-pranidhana

SK 24.

ahamkara abhimana tasmat pravartate dvividha sarga eva
ekadasaka gana ca tanmatra pancaka

Ahamkara is self-assertion;
from that proceeds a two-fold evolution
the set of eleven and the five-fold primary elements.

YS II.46

sthira-sukha asana

YS II.47

prayatna-saithilya-ananta-samapattibhya

YS II.48

tato dvandva-anabhighata

SK 25.

vaikrita ahamkara ekadasaka sattvika pravartate
tanmatra bhutade sa tamasa taijasa ubhayam

The set of eleven sattvic indriyas proceeds
from the Vaikriti form of I-Principle;
the set of five tamasic tanmatras proceed
from the Bhutadi form of I-Principle.
From the Taijasa form of I-Principle proceed both of them.

SK 26.

buddhi indriyani akhyani caksu srotram ghrana rasana tvak
karma indriyani ahu vak pani pada payu upastha

Organs of knowledge are called
the Eye, the Ear, the Nose, the Tongue, the Skin.
Organs of action are called
the Speech, the Hand, the Feet, the excretory organ and the organ of generation.

YS II.49

tasmin sati svasa-prasvasayo gati-viccheda pranayama

YS II.50

bahya-abhyantara-stambha-vritti desa-kala-sankhyabhi-paridrsta dirgha-suksma

YS II.51

bahya-abhyantara-visaya-aksepi caturtha

SK 27.

atra mana ubhayatmaka sankalpaka ca sadharmaya indriya
nanatva bahya bheda ca guna-parinama-visesa

Of these (sense organs), the Mind possesses the nature of both.
It is the deliberating principle, and is also called a sense organ
since it posseses properties common to the sense organs.
Its multifariousness and also its external diversities are owing
to special modifications of the Attributes.

SK 28.

pancana rupadisu alocanamatra isyate
pancana vritti vacana adana viharana utsarga ca ananda

The function of the five in respect to form and the rest,
is considered to be mere observation.
Speech, manipulation, locomotion, excretion and gratification
are the functions of the other five.

YS II.52

tata ksiyate prakasa-avarana

YS II.53

dharanasu ca yogyata manasa

YS II.54

sva-visaya-asamprayoge cittasya svarupa-anukara ive indriyana pratyahara

YS II.55

tata parama vasyata indriyana

YS III.1

desa-bandha cittasya dharana

YS III.2

tatra pratyaya-eka-tanata dhyana

YS III.3

tad evartha-matra-nirbhasa svarupa-sunya iva samadhi

YS III.4

traya ekatra samyama

YS III.5

taj-jaya prajna-aloka

YS III.6

tasya bhumisu viniyoga

YS III.7

traya antar-anga purvebhya

YS III.8

tad api bahir-anga nirbijasya

SK 29.

trayasya svalaksanya vritti sa esa asamanya bhavati
samanya karana-vritti pranadhyah-vayavah-panca

Of the three internal instruments,
their own characteristics are their functions;
this is peculiar to each.
The common modification of the instruments is the five airs
such as prana and the rest.

YS III.9

vyutthana-nirodha-samskarayor abhibhava-pradur-bhavau
nirodha-kshana-cittanvayo nirodha-parinama

YS III.10

tasya prasanta-vahita samskarat

Manas, the understanding, first the faculty for
the cognition of the general (of rules)

a. The concept as such

SK 30.

drste catustayasya tu yugapat vrttih tasya kramasasca nirdista
tatha api adrste trayasya vrttih tat purvika

Of all the four, the functions are said to be simultaneous and also successive
with regard to the seen objects;
with regard to the unseen objects
the functions of the three are preceeded by that.

YS III.11

sarva-arthata-ekagratayo kshaya-udayau cittasya samadhi-parinama

The logical form of all judgments consists of
the objective unity of the apperception of the concepts contained therein

Ahamkara, the determinative power of judgment,
second the faculty for the subsumption of the particular under the general

b. Judgment

SK 31.

sva sva vrittim pratipadyante paraspara akuta hetuka
purushartha eva hetu na kenacit karanam karyate

Instruments enter into their respective functions
being incited by mutual impulse.
The purpose of the Spirit is the sole motive
(for the activities of the instruments).
By none whatsoever is an instrument made to act.

YS III.12

tata puna-shanta-udita tulya-pratyaya cittasya-ekagrata-parinama

Buddhi, reason, third the faculty for the determination of
the particular from the general (for the derivation from principles)

c. Syllogism

SK 32.

karana trayodasavidha tad aharana dharana prakasakara
tasya karya ca dasadha aharya dharya prakasya ca

Instruments are of thirteen kinds performing the functions of
seizing, sustaining and illuminating.
Its objects are of ten kinds, vis
the seized, the sustained, and the illumined.

YS III.13

etena bhuta-indriyesu dharma-laksana-vastha-parinama vyakhyata

a. Mechanism

SK 33.

antahkarana trividha bahya dasadha trayasya visayakhya
bahya sampratakala abhyantara karana trikala

The internal instruments are three-fold.
The external are ten-fold;
they are called the objects of the three (internal instruments).
The external instruments function at the present time and
the internal instruments function at all the three times.

YS III.14

shanta-udita-vyapadesya-dharma-anupati dharmi

b. Chemism

SK 34.

Tesa panca buddhi visesa-avisesa-visaya
Vak sabda-visaya-bhavati sesani tu pancavisayani

Of these, the five organs of knowledge have as their objects
both the specific and the general.
Speech has speech as its object;
the rest have all the five as its object.

YS III.15

krama-anyatva parinama-anyatve hetu

c. Teleology

SK 35.

yasmat buddhi santahkarana sarvam visayam avagahate
tasmat trividham karana dvari sesani dvarani

Since buddhi along with the other internal instruments
comprehends all of the objects these three instruments are
like the warders while the rest are like the gates.

YS III.16

parinama-traya-samyama atita-nagata-jnanam

SK 36.

ete manas ahamkara pradipa-kalpa paraspara vilaksana
guna-visesa krstna prakasya purushasya-artha buddha prayacchati

These (external instruments along with manas and ahamkara)
which are characteristic-wise different from one another
and are different modifications of the attributes
and which resemble a lamp (in action)
illuminating all (their respective objects)
present them to the Buddhi for the purpose of the Spirit,

YS III.17

sabda-artha-pratyayana itaretara-adhyasat sankaras
tat-pravibhaga-samyama sarva-bhuta-ruta-jnana

YS III.18

samskara-saksat-karana purva-jati-jnana

YS III.19

pratyayasya para-citta-jnana

YS III.20

na ca tat salambana tasya-avisayi-bhutatva

SK 37.

yasmat buddhi purusasya upabho-gam sarva pratisadhyati
sa eva ca suksma pradhana-purusantara visinasti

Because it is the Buddhi that accomplishes the experiences
with regard to all objects to the Purusha.
It is that again that discriminates the subtle difference
between the Pradhana and the Purusha.

YS III.21

kaya-rupa-samyama tad-grahya-sakti-stambhe
caksu-prakasa-asamprayoge antardhana

YS III.22

etena sabda-adi-antardhana ukta


III.35
hrdaya citta-samvit

III.36
sattva-purushayor atyanta-asankirnayoh pratyaya-avisheso bhogah pararthatvat svartha-samyamat purusha-jnanam


III.47
kaya-akasa sambandha samyamat laghu-tula-samapattes ca akasagamanam


III.44
bahir akalpita vrittir mahavideha tatah prakasa-avarana-ksayah
