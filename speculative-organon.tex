The Speculative Organon

Rather than try to start another "From Kant to Hegel" thread,
Perhaps, I could ask for a Kant to Hegel thread with a twist.
The twist would be to request, hey,
why not throw Aristotle into the mix?
I think the thread should emphasise speculative frenzy
and so I could add another twist to the thread and mention
there may have been a long lost Vedic Organon and the
Plato-Aristotle thread may have been based on an early
orginal Sankhya-Yoga Vedic Organon.

How's that for degree of acceptable speculation.
I am not even going to provide references to start off!

The Western Organon

One question I would pose is who amongst the fab four,
(Kant-Fichte-Schelling-Hegel) would be the better interpreters
of Aristotle. I would posit that Fichte's 1804 Science of Knowing
is the best interpretation of Aristotle.
Kant is obviously "an Organist" but his Organon is still
based on a division of Logic and Metaphysics that are inadequate
and highly Aristotelian in origin.
Hegel is obviously "an Organist" but I think his Organon is complete
and more compatible with the long lost Vedic Organon.

The Vedic Organon

I am rambling about a long lost Vedic Organon
and an 1804 text from Fichte that nobody reads.
In that text, however, I state that Fichte is abandoning
his original early philosophy and replacing its basis with
Schelling's System of Identity.

One text that I base this on is the "Rupture" text translated
by Vater and Wood. This text is important in that it covers
a crucial period, the years prior to 1804 unfortunately.
If they had covered the 1804 SoK perhaps the tale would be
very different.
