
of the concept in general

What the nature of the concept is cannot be given right away,
not any more than can the concept of any other subject matter.
It might perhaps seem that, in order to state the concept of a subject matter,
the logical element can be presupposed, and that this element would not
therefore be preceded by anything else, or be something deduced,
just as in geometry logical propositions, when they occur applied to
magnitudes and employed in that science, are premised in the form of axioms,
underived and underivable determinations of cognition.
Now the concept is to be regarded indeed, not just as
a subjective presupposition but as absolute foundation;
but it cannot be the latter except to the extent that
it has made itself into one.
Anything abstractly immediate is indeed a first;
but, as an abstraction, it is rather something mediated, the foundation of which,
if it is to be grasped in its truth, must therefore first be sought.
And this foundation will indeed be something immediate,
but an immediate which has made itself such by the sublation of mediation.

From this aspect the concept is at first to be regarded simply
as the third to being and essence, to the immediate and to reflection.
Being and essence are therefore the moments of its becoming;
but the concept is their foundation and truth as the identity
into which they have sunk and in which they are contained.
They are contained in it because the concept is their result,
but no longer as being and essence;
these are determinations which they have only in so far as
they have not yet returned into the identity which is their unity.
Hence the objective logic, which treats of being and essence,
constitutes in truth the genetic exposition of the concept.

More precisely, substance already is real essence,
or essence in so far as it is united with being
and has stepped into actuality.
Consequently, the concept has substance
for its immediate presupposition;
substance is implicitly what the concept is explicitly.
The dialectical movement of substance
through causality and reciprocal affection
is thus the immediate genesis of the concept
by virtue of which its becoming is displayed.
But the meaning of its becoming,
like that of all becoming,
is that it is the reflection of something
which passes over into its ground,
and that the at first apparent other
into which this something has passed over
constitutes the truth of the latter.

Thus the concept is the truth of substance,
and since necessity is the determining relational mode of substance,
freedom reveals itself to be the truth of necessity and
the relational mode of the concept.
The necessary forward course of determination characteristic of sub-
stance is the positing of that which is in and for itself.
The concept is now this absolute unity of being and reflection whereby being-in-and-for-itself
only is by being equally reflection or positedness, and positedness only is by
being equally in-and-for-itself.
– This abstract result is elucidated by the
exposition of its concrete genesis which contains the nature of the concept
but had to precede its treatment.
We must briefly sum up here, therefore, the main moments of this exposition
(which has been treated in detail in Book Two of the Objective Logic).

Substance is the absolute, the actual in-and-for-itself:

in itself, because it is the simple identity of possibility and actuality;
absolute, because it is the essence containing all actuality and possibility within itself;
for itself, because it is this identity as absolute power or absolutely self-referring negativity.

The movement of substantiality posited by these moments consists in the following stages:

1. Substance, as absolute power or self-referring negativity, differentiates
itself into a relation in which what are at first only simple moments are
substances and original presuppositions. – Their specific relation is that
of a passive substance, of the originariness of the simple in-itself which,
powerless to posit itself, is only originary positedness, and of an active
substance, the self-referring negativity which has as such posited itself as
an other and refers to it. This other is precisely the passive substance
which the active substance, as originative power, has presupposed for itself
as its condition. – This presupposing is to be understood in the sense
that the movement of substance is at first in the form of one moment of
its concept, that of the in-itself – that the determinateness of one of the
substances standing in relation is itself also the determinateness of this
relation.
2. The other moment is the being-for-itself or the power positing itself as
self-referring negativity and thereby again sublating what it presupposes. –
The active substance is cause; it acts; this means that it is now a positing,
just as before it was a presupposing, that (a) power is also given the reflective
shine of power, positedness also the reflective shine of positedness. What
in the presupposition was the originary becomes in causality, by virtue
of the reference to an other, what it is in itself. The cause brings about
an effect. But it does so in another substance and it is now power with
reference to an other; it thus appears as cause but is cause only by virtue
of this appearing. – (b) 2 The effect enters the passive substance and
by virtue of it the latter now also appears as positedness, but is passive
substance only in this positedness.
3. But there is more still present here than just this appearance, namely:
(a) the cause acts upon the passive substance, alters its determination;
but this determination is its positedness, for otherwise there is noth-
ing else to alter; the other determination which it obtains is however
that of causality; the passive substance thus comes to be cause, power,
and activity;
(b) the effect is posited in it by the cause; but that which is posited by
the cause is the cause itself which, in acting, is identical with itself;
it is this cause that posits itself in the place of the passive substance.

Similarly, with respect to the active substance:

(a) the action is the translation of the cause into the effect,
into its other, the positedness;
(b) the cause reveals itself in the effect as what it is;
the effect is identical with the cause, is not an other;
in acting the cause thus reveals the positedness to be
that which it (the cause) essentially is.
Each side, therefore, in accordance with how it refers to the other both
as identical with it and as the negative of it, becomes the opposite of
itself, but, in becoming this opposite, the other, and therefore also each,
remains identical with itself.
But both, the identical and the negative
reference, are one and the same; substance is self-identical only in its
opposite and this constitutes the absolute identity of the two substances
posited as two. It is by its act that active substance is manifested as cause
or originary substantiality, that is, by positing itself as the opposite of
itself, a positing which is at the same time the sublating of its presupposed
otherness, of passive substance. Contrariwise, it is by being acted upon that
the positedness is manifested as positedness, the negative as negative, and
consequently the passive substance as self-referring negativity, and in this
other the cause simply rejoins itself.

Through this positing, therefore, what is presupposed,
that is, the implicit originariness, becomes explicit;
but this being, which is now in and for itself, is only by virtue of a positing which
is equally the sublation of what is presupposed, or because the absolute
substance has returned to itself only out of, and in, its positedness and for
that reason is absolute.
Hence this reciprocal action is appearance that
again sublates itself – the revelation that the reflective shine of causality, in
which the cause is as cause, is just that, that it is reflective shine.
This infinite immanent reflection – that the being-in-and-for-itself is
only such by being a positedness – is the consummation of substance.
But this consummation is no longer the substance itself
but is something higher, the concept, the subject.
The transition of the relation of substantiality occurs through
its own immanent necessity and is nothing more than the manifestation
of itself, that the concept is its truth, and that freedom is the truth of
necessity.

Earlier, in Book Two of the Objective Logic,
I have already called attention to the fact that
the philosophy that assumes its position at the standpoint of
substance and stops there is the system of Spinoza.
I have also indicated there the defect of this system,
both with respect to form and matter.
Something else, however, is the refutation of it.
Elsewhere, in connection with the refutation of a philosophical system,
I have also remarked quite in general that we must get over the distorted
idea that that system has to be represented as if thoroughly false, and as
if the true system stood to the false as only opposed to it. It is on the basis
of the context within which the system of Spinoza is presented here that
we can see its true standpoint and ask whether the system is true or false.
The relation of substantiality was generated by the nature of essence; this
relation and also its exposition as an expanded totality in the form of system
is, therefore, a necessary standpoint at which the absolute positions itself.
Such a standpoint, therefore, is not to be regarded as just an opinion, an
individual’s subjective, arbitrary way of representing and thinking, as an
aberration of speculation; on the contrary, speculation necessarily runs into
it and, to this extent, the system is perfectly true. – But it is not the highest
standpoint. By itself alone, therefore, the system cannot be regarded as false,
as either requiring or being capable of refutation. This alone is rather to
be considered false in it: that it would be the highest standpoint. It also
follows that the true system cannot be related to it as just its opposite, for
as so opposed it would itself be one-sided. Rather, as the higher system, it
must contain it within as its subordinate.
Further, any refutation would have to come not from outside, that is, not
proceed from assumptions lying outside the system and irrelevant to it. The
system need only refuse to recognize those assumptions; the defect is such
only for one who starts from such needs and requirements as are based on
them. For this reason it has been said that there cannot be any refutation of
Spinozism for anyone who does not presuppose a commitment to freedom
and the independence of a self-conscious subject. 4 Besides, a standpoint
so lofty and inherently so rich as that of the relation of substance does not
ignore those assumptions but even contains them: one of the attributes of
the Spinozistic substance is thought. The system knows how to resolve and
assimilate the determinations in which these assumptions conflict with it,
so that they re-emerge in it, but duly modified accordingly. The nerve,
therefore, of any external refutation consists solely in obstinately clinging
to the opposite categories of these assumptions, for example, to the absolute
self-subsistence of the thinking individual as against the form of thought
which in the absolute substance is posited as identical with extension.
Effective refutation must infiltrate the opponent’s stronghold and meet
him on his own ground; there is no point in attacking him outside his
territory and claiming jurisdiction where he is not. The only possible
refutation of Spinozism can only consist, therefore, in first acknowledging
its standpoint as essential and necessary and then raising it to a higher
standpoint on the strength of its own resources. The relation of substantiality,
considered simply on its own, leads to its opposite: it passes over into the
concept. The exposition in the preceding Book of substance as leading to
the concept is, therefore, the one and only true refutation of Spinozism. It is
the unveiling of substance, and this is the genesis of the concept the principal
moments of which we have documented above. – The unity of substance is
its relation of necessity. But this unity is thus only inner necessity. By positing
itself through the moment of absolute negativity, it becomes manifested
or posited identity, and also, therefore, the freedom which is the identity
of the concept. This concept, the totality resulting from the relation of
reciprocity, is the unity of the two substances that stand in that relation,
but in such a way now that the two belong to freedom: they no longer
possess their identity blindly, that is to say, internally; on the contrary, the
substances now explicitly have the determination that they are essentially
reflective shine or moments of reflection, and for that reason that each has
immediately rejoined its other or its positedness, that each contains this
positedness in itself and in its other, therefore, is posited simply and solely
as identical with itself.

In the concept, therefore, the kingdom of freedom is disclosed.
The concept is free because the identity that exists in and for itself
and constitutes the necessity of substance exists
at the same time as sublated or as positedness,
and this positedness, as self-referring, is that very identity.
Vanished is the obscurity which the causally related substances have for each other, for the
originariness of their self-subsistence that makes them causes has passed
over into positedness and has thereby become self-transparently clear;
the “originary fact” is “originary” because it is a “self-causing fact,”
and this is the substance that has been let go freely into the concept.
The direct result for the concept is the following more detailed determination.
Because being which is in and for itself is immediately a positedness,
the concept is in its simple self-reference an absolute determinateness which,
by referring only to itself, is however no less immediately simple identity.
But this self-reference of the determinateness in which the latter rejoins itself
is just as much the negation of determinateness, 8 and thus the concept, as
this equality with itself, is the universal. But this identity equally has the
determination of negativity; it is a negation or determinateness that refers
to itself and as such the concept is the singular. 9 Each, the universal and the
singular, is a totality; each contains the determination of the other within
it and therefore the two are just as absolutely one totality as their oneness
is the diremption of its self into the free reflective shine of this duality.
And this is a duality which in the differentiation of singular and universal
appears to be perfect opposition, but an opposition which is so much of a
reflective shine that, in that the one is conceptualized and said, immediately
the other is therein conceptualized and said.
The determinateness does not go past itself,
even excludes the possibility of going past itself.
In this sense, because it precludes reference to anything else besides itself,
it ceases to be a determinateness
and becomes a universal. It negates its own determinateness, i.e. itself as negation.
In precluding reference to anything besides itself,
the universal regains negativity.
It is just itself and nothing else.
In this sense, it is a universe by itself, a singular.
The foregoing is to be regarded as the concept of the concept.
To some it
may seem to depart from the common understanding of “concept,” and
they might require that we indicate how our result fits with other ways of
representing or defining it.
But, for one thing, this cannot be an issue
of proof based on the authority of ordinary understanding. In the science
of the concept, the content and determination of the latter can be proven
solely on the basis of an immanent deduction which contains its genesis,
and such a deduction lies behind us. And also, whereas the concept of the
concept as deduced here should in principle be recognized in whatever else
is otherwise adduced as such a concept, it is not as easy to ascertain what
others have said about its nature. For in general they do not bother at all
enquiring about it but presuppose that everyone already understands what
the concept means when speaking of it. Of late especially one may indeed
believe that it is not worth pursuing any such enquiry because, just as it
was for a while the fashion to say all things bad about the imagination,
then about memory, it became in philosophy the habit some time ago, and
is still the habit now, to heap every kind of defamation on the concept, to
hold it in contempt – the concept which is the highest form of thought –
while the incomprehensible and the non-comprehended are regarded as the
pinnacle of both science and morality.

I confine myself to one remark which may contribute to the compre-
hension of the concept here developed and facilitate one’s way into it. The
concept, when it has progressed to a concrete existence which is itself free,
is none other than the “I” or pure self-consciousness. True, I have con-
cepts, that is, determinate concepts; but the “I” is the pure concept itself,
the concept that has come into determinate existence. It is fair to suppose,
therefore, when we think of the fundamental determinations which con-
stitute the nature of the “I,” that we are referring to something familiar,
that is, a commonplace of ordinary thinking. But the “I” is in the first place
purely self-referring unity, and is this not immediately but by abstracting
from all determinateness and content and withdrawing into the freedom
of unrestricted equality with itself. As such it is universality, a unity that
is unity with itself only by virtue of its negative relating, which appears as
abstraction, and because of it contains all determinateness within itself as
dissolved. In second place, the “I” is just as immediately self-referring nega-
tivity, singularity, absolute determinateness that stands opposed to anything
other and excludes it – individual personality. This absolute universality
which is just as immediately absolute singularization – a being-in-and-
for-itself which is absolute positedness and being-in-and-for-itself only by
virtue of its unity with the positedness – this universality constitutes the
nature of the “I” and of the concept; neither the one nor the other can
be comprehended unless these two just given moments are grasped at the
same time, both in their abstraction and in their perfect unity.
When I say of the understanding that I have it, according to ordinary
ways of speaking, what is being understood by it is a faculty or a property
that stands in relation to my I in the same way as the property of a thing
stands related to that thing – as to an indeterminate substrate which is not
the true ground or the determining factor of the property. In this view, I
have concepts, and I have the concept, just as I also have a coat, complexion,
and other external properties. – Kant went beyond this external relation
of the understanding, as the faculty of concepts and of the concept, to
the “I.” It is one of the profoundest and truest insights to be found in
the Critique of Reason that the unity which constitutes the essence of the
concept is recognized as the original synthetic unity of apperception, the
unity of the “I think,” or of self-consciousness. 10 – This proposition is
all that there is to the so-called transcendental deduction of the categories
which, from the beginning, has however been regarded as the most difficult
piece of Kantian philosophy – no doubt only because it demands that we
should transcend the mere representation of the relation of the “I” and the
understanding, or of the concepts, to a thing and its properties or accidents,
and advance to the thought of it. – The object, says Kant in the Critique of
Pure Reason (2nd edn, p. 137), is that, in the concept of which the manifold
of a given intuition is unified. But every unification of representations
requires a unity of consciousness in the synthesis of them. Consequently, this
unity of consciousness is alone that which constitutes the reference of the
representations to an object, hence their objective validity, and that on
which even the possibility of the understanding rests. Kant distinguishes this
objective unity from the subjective unity of consciousness by which the
“I” becomes conscious of a manifold, whether simultaneously or successively
depending on empirical conditions. 11 In contrast to this subjective unity,
the principles of the objective determination of representations are only to
be derived from the principle of the transcendental unity of apperception. It
is by virtue of the categories, which are these objective determinations, that
the manifold of given representations is so determined as to be brought
to the unity of consciousness. 12 – On this explanation, the unity of the
concept is that by virtue of which something is not the determination of
mere feeling, is not intuition or even mere representation, but an object, and
this objective unity is the unity of the “I” with itself. – In point of fact,
the conceptual comprehension of a subject matter consists in nothing else
than in the “I” making it its own, in pervading it and bringing it into its
own form, that is, into a universality which is immediately determinateness,
or into a determinateness which is immediately universality. As intuited
or also as represented, the subject matter is still something external, alien.
When it is conceptualized, the being-in-and-for-itself that it has in intuition
and representation is transformed into a positedness; in thinking it, the “I”
pervades it. But it is only in thought that it is in and for itself; as it
is in intuition or representation, it is appearance. Thought sublates the
immediacy with which it first comes before us and in this way transforms
it into a positedness; but this, its positedness, is its being-in-and-for-itself or
its objectivity. This is an objectivity which the subject matter consequently
attains in the concept, and this concept is the unity of self-consciousness into
which that subject matter has been assumed; consequently its objectivity
or the concept is itself none other than the nature of self-consciousness,
has no other moments or determinations than the “I” itself.
Accordingly, we find in a fundamental principle of Kantian philosophy
the justification for turning to the nature of the “I” in order to learn
what the concept is. But conversely, it is necessary to this end that we have
grasped the concept of the “I” as stated. If we cling to the mere representation
of the “I” as we commonly entertain it, then the “I” is only the simple thing
also known as the soul, a thing in which the concept inheres as a possession
or a property. This representation, which does not bother to comprehend
either the “I” or the concept, is of little use in facilitating or advancing the
conceptual comprehension of the concept.

The position of Kant just cited contains two other points which concern
the concept and necessitate some further comments.
First of all, preceding
the stage of understanding are the stages of feeling and of intuition. It is
an essential proposition of Kant’s Transcendental Philosophy that concepts
without intuition are empty, and that they have validity only as references
connecting the manifold given by intuition.
Second, the concept is given
as the objective element of cognition, consequently as the truth. Yet it is
taken to be something merely subjective, and we are not allowed to extract
reality from it, 15 for by reality objectivity is to be understood, since reality
is contrasted with subjectivity.
Moreover, the concept and anything logical are declared to be something
merely formal which, since it abstracts from content, does not contain truth.

Now, in the first place, as regards the relation of the understanding or concept
to the stages presupposed by it, the determination of the form of these stages
depends on which science is being considered. In our science, which is pure
Logic, they are being and essence. In Psychology, the stages preceding the
understanding are feeling and intuition, and then representation generally.
In the Phenomenology of Spirit, which is the doctrine of consciousness,
the ascent to the understanding is made through the stages of sensuous
consciousness and then of perception. Kant places ahead of it only feeling
and intuition. But, for a start, he himself betrays the incompleteness of
this progression of stages by appending to the Transcendental Logic or the
Doctrine of the Understanding a treatise on the concepts of reflection – a
sphere lying between intuition and understanding, or being and concept. 16
And if we consider the substance itself of these stages, it must first be
said that such shapes as intuition, representation, and the like, belong to
the self-conscious spirit which, as such, does not fall within the scope of
logical science. Of course, the pure determinations of being, essence, and
the concept, also constitute the substrate and the inner sustaining structure
of the forms of spirit; spirit, as intuiting as well as sensuous consciousness, is
in the form of immediate being, just as spirit as representational and also
perceptual consciousness has risen from being to the stage of essence or
reflection. But these concrete shapes are of as little interest to the science
of logic as are the concrete forms that logical determinations assume in
nature. These last would be space and time, then space and time as assuming
a content, as inorganic and then organic nature. Similarly, the concept is also
not to be considered here as the act of the self-conscious understanding,
not as subjective understanding, but as the concept in and for itself which
constitutes a stage of nature as well as of spirit. Life, or organic nature, is
the stage of nature where the concept comes on the scene, but as a blind
concept that does not comprehend itself, that is, is not thought; only as
self-aware and as thought does it belongs to spirit. Its logical form, however,
is independent of such shapes, whether unspiritual or spiritual. This is a
point which was already duly adumbrated in the Introduction, 17 and one
that one must be clear about before undertaking Logic, not when one is
already in it.

But, in second place, how the forms that precede the concept might
ever be shaped depends on how the concept is thought in relation to them.
This relation, as assumed in ordinary psychology as well as in Kant’s
Transcendental Philosophy, 18 is that the empirical material, the manifold
of intuition and representation, is at first just there by itself, and that the
understanding then comes into it, brings unity to it, and raises it through
abstraction to the form of universality. The understanding is in this way
an inherently empty form which, on the one hand, obtains reality only by
virtue of that given content, and, on the other, abstracts from it, that is
to say, discards it as something useless, but useless only for the concept.
In both operations, the concept is not the one which is independent, is
not what is essential and true about that presupposed material; rather, this
material is the reality in and for itself, a reality that cannot be extracted
from the concept.
Now it must certainly be conceded that the concept is as such not yet
complete, that it must rather be raised to the idea which alone is the unity
of the concept and reality; and this is a result which will have to emerge
in what follows from the nature of the concept itself.
For the reality that
the concept gives itself cannot be picked up as it were from the outside
but must be derived from the concept itself in accordance with scientific
requirements. But the truth is that it is not the material given by intuition
and representation which must be validated as the real in contrast to the
concept. “It is only a concept,” people are wont to say, contrasting the
concept, as superior to it, not only with the idea, but with sensuous, spatial
and temporal, palpable existence. For this reason the abstract is then held to
be of less significance than the concrete, because so much of this palpable
material has been removed from it. In this view, to abstract means to
select from a concrete material this or that mark, but only for our subjective
purposes, without in any way detracting from the value and the status of
the many other properties and features that are left out; on the contrary,
by retaining them as reality, but yonder on the other side, still as fully
valid as ever. It is only because of its incapacity that the understanding
thus does not draw from this wealth and is forced rather to make do with
the impoverished abstraction. But now, to regard the given material of
intuition and the manifold of representation as the real, in contrast to what
is thought and the concept, is precisely the view that must be given up as
condition of philosophizing, and that religion, moreover, presupposes as
having already been given up. How could there be any need of religion, how
could religion have any meaning, if the fleeting and superficial appearance
of the sensuous and the singular were still regarded as the truth? But it
is philosophy that yields the conceptually comprehended 19 insight into the
status of the reality of sensuous being. Philosophy assumes indeed that the
stages of feeling, intuition, sense consciousness, and so forth, are prior to
the understanding, for they are the conditions of the genesis of the latter,
but they are conditions only in the sense that the concept results from
their dialectic and their nothingness and not because it is conditioned by
their reality. Abstractive thought, therefore, is not to be regarded as the
mere discarding of a sensuous material which does not suffer in the process
any impairment of reality; it is rather the sublation and reduction of that
material as mere appearance to the essential, which is manifested only in
the concept. Of course, if what is to be taken up into the concept from the
concrete appearance is intended to serve only as a mark or sign, then it may
well be anything at all, any mere sensuous singular determination of the
subject matter will do, selected from the others because of some external
interest but of like kind and nature as the rest.
In this conjunction, the prevailing fundamental misunderstanding is
that the natural principle, or the starting point in the natural development
or the history of an individual in the process of self-formation, is regarded
as the truth and conceptually the first. Intuition or being are no doubt first
in the order of nature, or are the condition for the concept, but they
are not for all that the unconditioned in and for itself; on the contrary,
in the concept their reality is sublated and, consequently, so is also the
reflective shine that they had of being the conditioning reality. If it is not
the truth which is at issue but only narration, as it is the case in pictorial and
phenomenal thinking, then we might as well stay with the story that we
begin with feelings and intuitions, and that the understanding then extracts
a universal or an abstraction from their manifold, for which purpose it quite
understandably needs a substrate for these feelings and intuitions which,
in the process of abstraction, retains for representation the same complete
reality with which it first presented itself. But philosophy ought not to
be a narrative of what happens, but a cognition of what is true in what
happens, in order further to comprehend on the basis of this truth what in
the narrative appears as a mere happening.
If on the superficial view of what the concept is all manifoldness falls
outside it, and only the form of abstract universality or of empty reflective
identity stays with it, we can at once call attention to the fact that any
statement or definition of a concept expressly requires, besides the genus
which in fact is already itself more than just abstract universality, also a
specific determinateness. And it does not take much thoughtful reflection on
the implication of this requirement to see that differentiation is an equally
essential moment of the concept. Kant introduced this line of reflection
with the very important thought that there are synthetic judgments
a priori. His original synthesis of apperception is one of the most pro-
found principles for speculative development; it contains the beginning of
a true apprehension of the nature of the concept and is fully opposed to any
empty identity or abstract universality which is not internally a synthesis. –
The further development, however, did not live up to this beginning. The
term itself, “synthesis,” easily conjures up again the picture of an external
unity, of a mere combination of terms that are intrinsically separate. Then,
again, the Kantian philosophy has never got over the psychological reflex
of the concept and has once more reverted to the claim that the concept is
permanently conditioned by the manifold of intuition. It has declared the
content of the cognitions of the understanding, and of experience, to be
phenomenal, not because of the finitude of the categories as such but, on the
ground of a psychological idealism, because they are only determinations
derived from self-consciousness. Here accordingly we have again the suppo-
sition that apart from the manifoldness of intuition the concept is without
content, empty, despite the fact that the concept is said to be a synthesis a
priori; as such, it surely contains determinateness and differentiation within
itself. And because this determinateness is the determinateness of the con-
cept, and hence the absolute determinateness, singularity, the concept is the
ground and the source of all finite determinateness and manifoldness.
The formal position that the concept never abandons as understanding
is completed in Kant’s exposition of what reason is. One should expect that
in reason, which is the highest stage of thought, the concept would lose
the conditionality with which it still appears at the stage of understanding
and would attain perfect truth. But this expectation is disappointed. For
Kant defines the relation of reason to the categories as merely dialectical.
Indeed, he even takes the result of this dialectic to be simply and solely
an infinite nothingness, the result being that the synthesis is again lost, lost
also to the infinite unity of reason, and lost with it is whatever beginning
there was of a speculative, truly infinite, concept; reason becomes the well-
known, totally formal, merely regulative unity of the systematic employment
of the understanding. It is declared an abuse when Logic, which is supposed
to be a mere canon of judgment, is considered instead as an organon for
the production of objective insights. The concepts of reason, in which
we would have expected a higher power and a deeper content, no longer
possess anything constitutive as still do the categories; they are mere ideas
which we are of course are quite at liberty to use, provided that by these
intelligible entities in which all truth was to be revealed we mean nothing
more than hypotheses to which it would be the height of arbitrariness and
recklessness to ascribe absolute truth, for they – cannot be found in any
experience. 21 – Would anyone have ever thought that philosophy would
deny truth to intelligible entities on the ground that they lack the spatial
and temporal material of the senses?
Directly connected with this is the issue of how to view the concept
and the character of logic generally, the issue namely of the relation of
the concept and its science to truth itself. This is an issue on which the
Kantian philosophy holds the same position as is commonly taken. We
cited earlier from Kant’s deduction of the categories to the effect that,
according to it, the object in which the manifold of intuition is unified is
this unity only by virtue of the unity of self-consciousness. 22 The objectivity of
thought is here, therefore, specifically defined: it is an identity of concept
and thing which is the truth. In the same way it is also commonly accepted
that, as thought appropriates a given subject matter, this subject matter
thereby undergoes an alteration and is made from something sensuous
into something thought. But nothing is changed in this alteration in so
far as the essentiality of the object goes; on the contrary, it is accepted that
the object is in its truth only in its concept, whereas in the immediacy in
which it is given it is only appearance and accidentality; that the cognition
conceptualizing the subject matter is a cognition of it as it is in and for itself,
and the concept is its very objectivity. But, on the other hand, it is also
equally claimed that we cannot know things as they are in and for themselves
and that truth is inaccessible to rational cognition; that the aforesaid truth
that would consist in the unity of the object and the concept is in fact
only appearance, again on the ground now that the content is only the
manifold of intuition. But we have just remarked, regarding this point,
that it is precisely in the concept that the manifold is sublated inasmuch
as it pertains to intuition as opposed to the concept, and that through the
concept the subject matter is reduced to its non-contingent essentiality;
the latter does enter into appearance, and this is why appearance is not
something merely essenceless but is the manifestation of essence. When
this manifestation of essence is set free, then we have the concept. – These
propositions that we are now recalling are not dogmatic assertions, for
they are results that obtained on their own out of the whole development
of essence. The present position to which this development has led is that
the form of the absolute which is higher than being and essence is the
concept. Viewed from this side, the concept has subjugated the spheres of
being and essence to which, from other starting points, feeling, intuition,
and representation, which appeared to be its antecedent conditions, also
belong; it has demonstrated itself to be their unconditional foundation. But
this is one side alone. There is a second side left to which this third book of
the Logic is devoted, namely the demonstration of how the concept forms
within and from itself the reality that has vanished in it. It is conceded, in
other words, that the cognition that does not go past the concept, purely
as concept, is still incomplete, that it has only arrived at abstract truth. But
its incompleteness does not lie in its lack of that alleged reality as would be
given in feeling and intuition, but in the fact that the concept has yet to give
to itself its own reality, one that it generates out of itself. The demonstrated
absoluteness of the concept as against the material of experience and, more
exactly, the categorial and the reflective determinations of it, consists in
this, that as this material appears outside and before the concept, it has no
truth but that it has it only in its ideality or in its identity with the concept.
The derivation of the real from the concept, if “derivation” is what we
want to call it, consists at first essentially in this, that the concept in its
formal abstraction reveals itself to be incomplete and through a dialectic
immanently grounded in it passes over into reality: it passes over into it,
however, as into something which it generates out of itself, not as if it
were falling back again onto a ready-made reality which it finds opposite
it, or as if it were taking refuge, because it sought for something better
but found none, into something that has already been proven to be the
unessential element of appearance. – It will always be a source of wonder
how the Kantian philosophy did acknowledge that the relation of thought
to sensuous existence (the relation at which it stopped) is only a relation of
mere appearance, and also well recognized in the idea in general a higher
unity of those two terms, even gave expression to it, as for example in the
idea of an intuitive understanding, and yet stopped short at that relative
relation and at the claim that the concept remains utterly separate from
reality – thus asserting as truth what it declared to be finite cognition, and
explaining away as extravagant and illegitimate figments of thought what
it recognized as truth and had specifically defined as such.
Since it is logic above all and not science generally whose relation to
truth is the issue here, it must be further conceded that logic as the formal
science cannot also contain, nor should contain, the kind of reality which
is the content of the other parts of philosophy, of the sciences of nature
and of spirit. These concrete sciences do attain to a more real form of the
idea than logic does, but not because they have turned back to the reality
which consciousness abandoned as it rose above the appearance of it to
science, or because they have again resorted to the use of such forms as
are the categories and the determinations of reflection, the finitude and
untruth of which were demonstrated in the logic. The logic rather exhibits
the rise of the idea up to the level from which it becomes the creator
of nature and passes over into the form of a concrete immediacy whose
concept, however, again shatters this shape also in order to realize itself as
concrete spirit. These sciences, just as they had the logic as their prototype,
hold on to its logical principle or the concept as in them their formative
factor. As contrasted with them, the logic is of course the formal science,
yet the science of the absolute form which is implicit totality and contains
the pure idea of truth itself. This absolute form has in it a content or reality
of its own; the concept, since it is not a trivial, empty identity, obtains its
differentiated determinations in the moment of negativity or of absolute
determining; and the content is only these determinations of the absolute
form and nothing else – a content posited by the form itself and therefore
adequate to it. – This form is for this reason of quite another nature than
logical form is ordinarily taken to be. It is truth already on its own account,
because this content is adequate to its form or this reality to its concept,
and it is pure truth, because the determinations of the content do not yet
have the form of an absolute otherness or of absolute immediacy. – When
Kant in the Critique of Pure Reason (p. 83), 23 in connection with logic
comes to discuss the old and famous question: What is truth?, he starts by
passing off as a triviality the nominal definition that it is the agreement of
cognition with its subject matter – a definition which is of great, indeed of
supreme value. If we recall this definition together with the fundamental
thesis of transcendental idealism, namely that rational cognition is incapable
of comprehending things in themselves, 24 that reality lies absolutely outside
the concept, it is then at once evident that such a reason, one which is
incapable of setting itself in agreement with its subject matter, and the things
in themselves, such as are not in agreement with the rational concept – a
concept that does not agree with reality and a reality that does not agree
with the concept – that these are untrue conceptions. If Kant had measured
the idea of an intuitive understanding against that first definition of truth,
he would have treated that idea which expresses the required agreement,
not as a figment of thought but rather as truth.
“What we would want to know,” Kant proceeds to say, “is a universal
and certain criterion of truth of any cognition whatever, one that would be
valid for all cognitions without distinction of their subject matters; but since
any such criterion would abstract from all content of cognition (the reference
to its object), and truth has to do precisely with this content, it would be quite
impossible, even absurd, to ask for a mark of the truth of this content of
cognitions.” 25 – Here we have, clearly expressed, the ordinary conception
of the formal function of logic which gives to the adduced argument the air
of convincing. But first of all it is to be noted what usually happens to this
kind of formal argumentation: it forgets as it speaks that on which it is based
and of which it speaks. It would be absurd, it says, to ask for a criterion
of the truth of the content of cognition. But according to the definition on
which it is based it is not the content that constitutes the truth, but the
agreement of it with the concept. Such a content as is here spoken of, one
without the concept, is something void of concept and therefore void of
essence; of course, we cannot ask of such a content for a criterion of truth,
but for the opposite reason, namely, not because it cannot be the required
agreement on account of its being void of concept, but because it cannot
be anything more than just another truthless opinion. – Let us leave aside
any talk about content, which is the cause of the confusion here – the
confusion in which formalism invariably falls, and which is responsible for
making it say, every time it tries to explain itself, the opposite of what it
wants to say – and let us just stay with the abstract view that the logic
is only formal, that it abstracts from all content. What we then have is a
one-sided cognition which is not supposed to contain any subject matter,
an empty form void of determination which is therefore just as little an
agreement (for it necessarily takes two for an agreement) as it is truth. –
In the a priori synthesis of the concept, Kant did have a higher principle in
which it was possible to recognize a duality and therefore what is required
for truth; but the material of the senses, the manifoldness of intuition, was
too strong for him to be able to wrest himself away from it and turn to a
consideration of the concept and the categories in and for themselves, and
to a speculative form of philosophizing.

Since logic is the science of the absolute form, this formal discipline,
in order to be true, must have a content in it which is adequate to its form;
all the more so, because logical form is pure form and
logical truth, accordingly, the pure truth itself.

This formal discipline must therefore be thought of as
inherently much richer in determinations and content,
and also of infinitely greater efficacy over the concrete,
than it is normally taken to be.

The laws of logic by themselves
(extraneous elements aside, such as applied logic and
the rest of the psychological and anthropologicalmaterial)
are commonly restricted, apart from the law of contradiction,
to a few meager propositions concerning the conversion of judgments
and the forms of inference.

And the forms, too, that come up in this context, as well
as their further specifications, are only taken up historically as it were, not
subjected to criticism to see whether they are in and for themselves true.
For example, the form of the positive judgment is accepted as something
perfectly correct in itself, and whether the judgment is true is made to
depend solely on the content. No thought is given to investigating whether
this form of judgment is a form of truth in and for itself; whether the
proposition it enunciates, “the individual is a universal,” is not inherently
dialectical. It is at once assumed that the judgment is capable of possessing
truth on its own account, and that every proposition expressed in a positive
judgment is true, even though it is patently evident that the judgment lacks
what is required by the definition of truth, namely the agreement of the
concept with its subject matter; for if the predicate, which here is the
universal, is taken as the concept, and the subject, which is the singular,
as the subject matter, then the concept does not agree with it. But if the
abstract universal which is the predicate does not yet amount to a concept
(for surely there is more that belongs to it); or if the subject, for its part,
still is not much more than a grammatical one, how should the judgment
possibly contain truth seeing that its concept and the intended object do
not agree, as also that the concept is missing and indeed the object as well? –
This rather is then where the impossible and the absurd lie, in the attempt
to grasp the truth in such forms as are the positive judgment or a judgment
in general. Just as the Kantian philosophy did not consider the categories
in and for themselves, but declared them to be finite determinations unfit
to hold the truth, on the only inappropriate ground that they are subjective
forms of self-consciousness, still less did it subject to criticism the forms
of the concepts that make up the content of ordinary logic. What it did,
rather, is to pick a portion of them, namely the functions of judgments,
for the determination of categories, and simply accepted them as valid
presuppositions. Even if there were nothing more to the forms of logic
than these formal functions of judgment, for that reason alone they would
already be worthwhile investigating to see how far, by themselves, they
correspond to the truth. A logic that does not perform this task can at most
claim the value of a natural description of the phenomena of thought as
they simply occur. It is an infinite merit of Aristotle, one that must fill us
with the highest admiration for the power of his genius, that he was the
first to undertake this description. But it is necessary to go further and
determine both the systematic connection of these forms and their value.
