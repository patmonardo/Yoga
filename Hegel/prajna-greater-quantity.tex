SECTION II

Magnitude (Quantity)

The difference between quantity and quality has been indicated.
Quality is the first, immediate determinateness.
Quantity is the determinateness that
has become indifferent to being;
a limit which is just as much no limit;
being-for-itself which is absolutely
identical with being-for-another:
the repulsion of the many ones
which is immediate non-repulsion,
their continuity.

Because that which exists for itself is now so posited
that it does not exclude its other
but rather affirmatively continues in it,
it is then otherness, inasmuch as
existence surfaces again on this continuity
and its determinateness is at the same time
no longer simple self-reference,
no longer the immediate determinateness
of the existent something,
but is posited as repelling itself from itself,
as referring to itself in the determinateness
rather of an other existence
(a being which exists for itself);
and since they are at the same time indifferent limits,
reflected into themselves and unconnected,
determinateness is as such outside itself,
an absolute externality and a something just as external;
such a limit, the indifference of the limit as limit
and the indifference of the something to the limit,
constitutes the quantitative determinateness of the something.

In the first place, we have to distinguish pure quantity
from quantity as determinate, from quantum.
First, pure quantity is real being-for-itself
turned back into itself, with as yet no determinateness in it:
a compact, infinite unity which continues itself into itself.

Second, this quantity proceeds to determinateness,
and this is posited in it as a determinateness
that at the same time is none, is only external.
Quantity becomes quantum.
Quantum is indifferent determinateness,
that is, one that transcends itself, negates itself;
as this otherness of otherness,
it lapses into infinite progress.
Infinite quantum, however, is
sublated indifferent determinateness:
it is the restoration of quality.

Third, quantum in qualitative form is quantitative ratio.
Quantum transcends itself only in general;
in the ratio, however, it transcends itself into its otherness,
in such a way that this otherness in which it has its determination
is at the same time posited, is another quantum.
With this we have quantum as turned back into itself
and referring to itself as into its otherness.

At the foundation of this relation there still
lies the externality of quantum;
it is indifferent quanta that
relate themselves to each other,
that is, they have the reference
that mutually connects them
in this being-outside-itself.
The ratio is, therefore, only
a formal unity of quality and quantity,
and dialectic is its transition into
their absolute unity, in measure.

CHAPTER 1

Quantity

A. PURE QUANTITY

Quantity is sublated being-for-itself.
The repelling one that behaved only
negatively towards the excluded one,
now that it has gone over in connection with it,
behaves towards the other as identical to itself
and has therefore lost its determination;
being-for-itself has passed over into attraction.
The absolute obduracy of the one has melted away
into this unity which, however, as containing the one, is
at the same time determined by the repulsion residing in it;
as unity of the self-externality, it is unity with itself.
Attraction is in this way the moment of continuity in quantity.

Continuity is therefore simple, self-same reference to itself
unbroken by any limit or exclusion;
not, however, immediate unity but the unity of ones
which have existence for themselves.
Still contained in it is the outside-one-another of plurality,
though at the same time as something without distinctions, unbroken.
Plurality is posited in continuity as it implicitly is in itself;
the many are each what the others are,
each is like the other,
and the plurality is, consequently,
simple and undifferentiated equality.
Continuity is this moment of self-equality
of the outsideness-of-one-another,
the self-continuation of the different ones
into the ones from which they are distinguished.

In continuity, therefore, magnitude immediately possesses
the moment of discreteness, repulsion as now a moment in quantity.
Steady continuity is self-equality,
but of many that do not become exclusive;
it is repulsion that first expands self-equality to continuity.
Hence discreteness is, for its part, a discreteness of confluents,
of ones that do not have the void to connect them,
not the negative, but their own steady advance
and, in the many, do not interrupt this self-equality.

Quantity is the unity of these moments,
of continuity and discreteness.
At first, however, it is this continuity
in the form of one of them, of continuity,
as a result of the dialectic of the being-for-itself
which has collapsed into the form of self-equal immediacy.
Quantity is as such this simple result
in so far as the being-for-itself has
not yet developed its moments
and has not posited them within it.
Quantity contains these moments at first
as being-for-itself posited in its truth.
It was the determination of being-in-itself
to be self-sublating self-reference,
a perpetual coming-out-of-itself.
But what is repelled is itself;
repulsion is thus a creative flowing away from itself.
On account of the sameness of what is repelled,
this discerning is unbroken continuity;
and on account of the coming-out-of-itself,
this continuity is at the same time,
without being broken off, a plurality,
a plurality which persists just as
immediately in its equality with itself.

B. CONTINUOUS AND DISCRETE MAGNITUDE

1. Quantity contains the two moments of continuity and discreteness.
It is to be posited in both, in each as its determination.
It is already from the start the immediate unity of the two,
that is, quantity is itself posited at first
only in one of the two determinations, that of continuity,
and as such is continuous magnitude.

Or continuity is indeed one of the moments of quantity
which is brought to completion only with the other, discreteness.
But quantity is concrete unity only in so far as
it is the unity of distinct moments.
These are to be taken, therefore, also as distinct,
without however resolving them again
into attraction and repulsion
but, rather, as they truly are,
each remaining in its unity with the other,
that is, remaining the whole.
Continuity is only the compact unity
holding together as unity of the discrete;
posited as such, it is no longer only moment
but the whole quantity: continuous magnitude.

2. Immediate quantity is continuous magnitude.
Quantity, however, is not as such an immediate;
immediacy is a determinateness,
the sublated being of which is precisely quantity.
Quantity is to be posited, therefore,
in the determinateness immanent to it,
and this is the one.
Quantity is discrete magnitude.

Discreteness is, like continuity, a moment of quantity,
but is itself also the whole quantity
just because it is a moment in it, the whole,
and therefore as distinct moment does not
diverge from its unity with the other moment.
Quantity is the outsideness-of-one-another as such,
and continuous magnitude is this outsideness-of-one-another
onwardly positing itself without negation
as an internally self-same connectedness.
On the other hand, discrete magnitude is this
outsideness-of-one-another as discontinuous, as broken off.
With this aggregate of ones, however,
the aggregate of atom and void,
repulsion in general, is not thereby reinstated.
Because discrete magnitude is quantity,
its discreteness is itself continuous.
Such a continuity in the discrete consists
in the ones being the same as one another,
or in that they have the same unity.
Discrete magnitude is therefore
the one-outside-the-other of
the many ones as of a same;
not the many ones in general,
but posited rather as the many of a unity.

C. THE LIMITING OF QUANTITY

Discrete magnitude has,
first, the one for its principle
and, second, is a plurality of ones;
third, it is essentially continuous,
it is the one as at the same time
sublated, as unity,
the self-continuing as such
in the discreteness of the ones.
Consequently, it is posited as one magnitude,
and the “one” is its determinateness;
a “one” which, in this posited and determinate existence,
excludes, is a limit to the unity.
Discrete magnitude as such is not supposed
to be immediately limited,
but, when distinguished from continuous magnitude,
it is an existence and a something,
the determinateness of which,
and in it also the first negation and limit,
is the “one.”

This limit, besides referring to the unity
and being the moment of negation in it,
is also, as one, self-referred;
thus it is enclosing, encompassing limit.
The limit here is not at first distinct
from the something of its existence,
but, as one, is essentially this negative point itself.
But the being which is here limited is essentially as continuity,
and in virtue of this continuity it transcends the limit,
transcends this one, and is indifferent to it.
Real, discrete quantity is thus one quantity, or quantum:
quantity as an existence and a something.

Since the one which is a limit encompasses
within it the many ones of discrete quantity,
it posits them equally as sublated in it;
it is a limit to continuity simply as such
and, consequently, the distinction between
continuous and discrete magnitude is here indifferent;
or, more precisely, it is a limit to
the continuity of the one
just as much as of the other;
in this, both pass over into being quanta.

CHAPTER 2

Quantum

Quantum, which in the first instance is quantity
with a determinateness or limit in general,
in its complete determinateness is number.
Second, quantum divides first into extensive quantum,
in which limit is the limitation of
a determinately existent plurality;
and then, inasmuch as the existence of
this plurality passes over into being-for-itself,
into intensive quantum or degree.
This last is for-itself but also,
as indifferent limit, equally outside itself.
It thus has its determinateness in an other.
Third, as this posited contradiction of
being determined simply in itself
yet having its determinateness outside itself
and pointing outside itself for it,
quantum, as thus posited outside itself within itself,
passes over into quantitative infinity.

CHAPTER 3

Ratio or the quantitative relation

The infinity of quantum has been determined up to
the point where it is the negative beyond of quantum,
a beyond which quantum, however, has within it.
This beyond is the qualitative moment in general.
The infinite quantum, as the unity of the two moments,
of the quantitative and the qualitative determinateness,
is in the first instance ratio.

In ratio, quantum no longer has
a merely indifferent determinateness
but is qualitatively determined as
simply referring to its beyond.
It continues in its beyond, and this beyond is
at first just an other quantum.
Essentially, however, the two do not refer
to each other as external quanta
but each rather possesses its determinateness
in this reference to the other.
In this, in their otherness, they have
thus returned into themselves;
what each is, that it is in its other;
the other constitutes the determinateness of each.
The quantum's self-transcendence does not now mean, therefore,
that quantum has simply changed either into some other
or into its abstract other, into its abstract beyond,
but that there, in the other, it has attained its determinateness;
in its other, which is an other quantum, it finds itself.
The quality of quantum, its conceptual determinateness,
is its externality as such,
and in ratio quantum is now posited as having
its determinateness in this externality, in another quantum;
as being in its beyond what it is.
It is quanta that stand to each other in the connection
that has now come on the scene.
This connection is itself also a magnitude;
quantum is not only in relation,
but is itself posited as relation;
it is a quantum as such that has
that qualitative determinateness in itself.
So, as relation (as ratio),
quantum gives expression to itself as self-enclosed totality
and to its indifference to limit by containing
the externality of its being-determined in itself:
in this externality it is only referred back to itself
and is thus infinite within.

Ratio in general is:

1. direct ratio.
In this, the qualitative moment does not
yet emerge explicitly as such;
in no other way except still as quantum is quantum posited
as having its determinateness in its externality.
In itself, however, the quantitative relation is
the contradiction of externality and self-reference,
the persistence of quanta and their negation.
Such a contradiction next sublates itself:

2. first inasmuch as in indirect or inverse ratio
the negation of each of the quanta is
as such co-posited in the alteration of the other,
and the variability of the direct ratio is itself posited;

3. but in the ratio of powers, the unity,
which in its difference refers back to itself,
proves to be a simple self-production of the quantum;
this qualitative moment itself,
finally posited in a simple determination
and as identical with the quantum, becomes measure.

About the nature of the following ratios,
much was anticipated in the preceding remarks
concerning the infinity of quantity, that is,
the qualitative moment in it;
it only remains to analyze, therefore,
the abstract concept of this ratio.

A. THE DIRECT RATIO

1. In the ratio which, as immediate, is direct,
the determinateness of each quantum lies in
the reciprocal determinateness of the other.
There is only one determinateness or limit of both
one which is itself a quantum,
namely the exponent of the ratio.

2. The exponent is some quantum or other;
however, in referring itself to itself
in the otherness which it has within it,
it is only a qualitatively determined quantum,
for its difference, its beyond and otherness, is in it.
This difference in the quantum is the difference of unit and amount;
the unit, which is the being-determined-for-itself;
the amount, which is the indifferent fluctuation of determinateness,
the external indifference of quantum.
Unit and amount were at first the moments of quantum;
now, in the ratio, in quantum as realized so far,
each of its moments appears as a quantum on its own
and as determinations of the existence of the quantum,
as delimitations against the otherwise
external, indifferent determinateness of magnitude.

The exponent is this difference as simple determinateness,
that is, it has the meaning of both
determinations immediately in it.
First, it is a quantum and thus an amount.
If the one side of the ratio which is taken as
unit is expressed in a numerical one,
and has only the value of one,
then the other, the amount, is
the quantum of the exponent itself.
Second, it is simple determinateness as
the qualitative moment of the sides of the ratio.
When the quantum of the one side is determined,
the other is also determined by the exponent
and it is a matter of total indifference
how the first is determined;
it no longer has any meaning
as a quantum determined for itself
but can just as well be any other
quantum without thereby altering
the determinateness of the ratio,
which rests solely on the exponent.
The one side which is taken as unit always
remains unit however great it becomes,
and the other, however great it too thereby becomes,
must remain the same amount of that unit.

3. Accordingly, the two truly constitute only one quantum;
the one side has only the value of unit
with respect to the other, not of an amount;
and the other only that of amount.
According to their conceptual determinateness,
therefore, they are themselves not complete quanta.
But this incompleteness is in them a negation,
and it is so not because of their general variability,
according to which one of them (any of the two)
can assume all possible magnitudes,
but because they are so determined that, as one is altered,
the other is increased or decreased in corresponding measure.
This means that, as indicated, only one of them,
the unit, is altered as quantum;
the other side, the amount, remains the same quantum of units,
and the first side too retains the value of a unit,
however much it is altered as quantum.
Each side is thus only one of the two moments of quantum,
and the self-subsistence which is their proper characteristic
is in principle negated;
in this qualitative combination they are to be
posited as negative with respect to each other.

The exponent ought to be the complete quantum,
since the determinations of both sides come together in it;
but in fact, even as quotient the exponent
only has the value of amount, or of unit.
There is nothing available for determining
which of the two sides of the relation would have to
be taken as the unit or as the amount;
if one side, quantum B,
is measured against quantum A as unit,
then the quotient C is the amount of such units;
but if A is itself taken as amount,
the quotient C is the unit which is required by
the amount A for the quantum B.
As exponent, therefore, this quotient is
not posited for what it ought to be,
namely the determinant of the ratio,
or the ratio's qualitative unity.
It is posited as such only to
the extent that its value is that
of the unity of the two moments,
of unit and amount.
And since these two sides, as quanta,
are indeed present as they should be
in the explicated quantum, in the ratio,
but at the same time have the value,
which is specific to them as the sides of the ratio,
of being incomplete quanta and of counting
only as one of those qualitative moments,
they are to be posited with this negation qualifying them.
Thus there arises a more real ratio,
one more in accordance with its definition,
one in which the exponent has the meaning
of the product of the sides.
In this determinateness,
it is the inverse ratio.

B. THE INVERSE RATIO

1. The ratio as now before us is the sublated direct ratio.
It was an immediate relation and therefore not yet truly determinate;
henceforth, the newly introduced determinateness gives
the exponent the value of a product,
the unity of unit and amount.
In immediacy, as we have just seen,
it was possible for the exponent to be
indifferently taken as unit or amount.
Moreover, it also was only a quantum in general
and therefore an amount by choice.
One side was the unit,
and this was to be taken as a numerical one
with respect to which the other side would be a fixed amount
and at the same time the exponent.
The quality of the latter, therefore,
was only that this quantum is taken as fixed,
or rather that the constant only
has the meaning of quantum.

Now in the inverse ratio,
the exponent is as quantum likewise immediate,
a quantum or other which is assumed as fixed.
But to the one of the other quantum in the ratio,
this quantum is not a fixed amount; the ratio,
previously taken as fixed, is now posited instead as alterable;
if another quantum is taken as
the unit of the one side,
the other side now no longer remains
the same amount of units of the first side.
In direct ratio, this unit
is only the common element of both sides;
as such, it continues into the other side, the amount;
the amount itself, or the exponent,
is by itself indifferent to the unit.

But as the determinateness of the ratio now is,
the amount as such alters relative to the unit
with respect to which it makes up the other side of the ratio;
whenever another quantum is taken as the unit,
that amount alters.
Consequently, although the exponent is
still only an immediate quantum
and only arbitrarily assumed as fixed,
it does not remain fixed in the side of the ratio:
rather this side, and with it
the direct ratio of the sides, is alterable.
In the ratio now before us,
the exponent as the determining quantum is
thus posited as negative towards itself
as a quantum of the ratio,
and hence as qualitative, as limit;
the result is that the qualitative moment distinctly comes
to the fore for itself as against the quantitative moment.
In the direct ratio, the alteration of the two sides is
only the one alteration of the quantum
from which the unity which is the common element is taken;
by as much, therefore, as the one side is
increased or decreased, so also is the other;
the ratio itself is indifferent to this alteration
and the alteration external to the ratio.
In the indirect ratio, on the contrary,
although still arbitrary according to
the moment of quantitative indifference,
the alteration is contained within the ratio,
and its arbitrary quantitative extension is
limited by the negative determinateness
of the exponent as by a limit.

2. We must now consider this qualitative nature
of the indirect ratio more closely, as it is realized,
and sort out the entanglement of the affirmative
and the negative moments that are contained in it.
Quantum is posited as being quantum qualitatively,
that is, as self-determining,
as displaying its limit within it.
Accordingly, first, it is an immediate magnitude
as simple determinateness;
it is the whole as existent, an affirmative quantum.
But, second, this immediate determinateness is
at the same time limit;
for that purpose it is distinguished into two quanta
which are at first the other of each other;
but as the qualitative determinateness of these quanta,
and as a determinateness which is moreover complete,
quantum is the unity of the unit and the amount,
a product of which the two are the factors.
Thus on the one hand the exponent of their ratio
is in them identical with itself
and is their affirmative moment by which they are quanta;
on the other hand, as the negation posited in them,
it is in them the unity according to which each,
at first an immediate and limited quantum in general,
is at the same time so limited as to be
only implicitly in itself identical with its other.
Third, the exponent as the simple determinateness is
the negative unity of this differentiation of it into the two quanta,
and the limit of their reciprocal limiting.

In accordance with these determinations,
the two moments limit themselves inside the exponent
and each is the negative of the other,
for the exponent is their determinate unity;
the one moment becomes as many
times smaller as the other becomes greater;
each possesses a magnitude of its own
to the extent that this magnitude is in it that of their other,
that is, is the magnitude that the other lacks.
The magnitude of each in this way
continues into the other negatively;
how much it is in amount,
that much it sublates in the other as amount
and is what it is only through this negation
or limit which is posited in it by the other.
In this way, each contains the other as well,
and is proportioned to it,
for each is supposed to be only that quantum
which the other is not;
the magnitude of the other is
indispensable to the value of each,
and therefore inseparable from it.

This continuity of each in the other
constitutes the moment of unity
through which the two magnitudes stand in relation;
the moment of the one determinateness,
of the simple limit which is the exponent.
This unity, the whole, constitutes the in-itself
of each from which their given magnitude is distinct:
this is the magnitude according to which each is
only to the extent that it takes from the other
a part of their common in-itself which is the whole.
But each can take from the other only as much
as will make it equal to this in-itself;
it has its maximum in the exponent
which, in accordance with the stated second determination,
is the limit of the reciprocal delimitation.
And since each is a moment of the ratio only
to the extent that it limits the other
and is thereby limited by it,
it loses this, its determination,
by making itself equal to its in-itself;
in this loss, the other magnitude will not only
become a zero, but itself vanishes,
for what it ought to be is not just a quantum
but what it is as such a quantum,
namely only the moment of a ratio.
Thus each side is the contradiction
between the determination as its in-itself,
that is, as the unity of the whole which is the exponent,
and the determination as the moment of a ratio;
this contradiction is infinity again,
in a new form peculiar to it.

The exponent is the limit of the sides of its ratio,
within which limit the sides increase and decrease
proportionately to each other;
but they cannot become equal to this exponent
because of the latter's affirmative determinateness as quantum.
Thus, as the limit of their reciprocal limiting,
the exponent is (a) their beyond which they
infinitely approximate but can never attain.
This infinity in which the sides approximate their beyond is
the bad infinity of the infinite progression;
it is an infinity which is itself finite,
that finds its restriction in its opposite,
in the finitude of each side
and of the exponent itself,
and for this reason is only approximation.
But (b) the bad infinite is equally posited here
as what it is in truth, namely
the negative moment in general,
in accordance with which the exponent
is the simple limit as against the distinct quanta of the ratio:
it is the in-itself to which, as the absolutely alterable,
the finitude of the quanta is referred,
but which, as the quanta's negation,
remains absolutely different from them.
This infinite, which the quanta can only approximate,
is then equally found affirmatively present on their side:
the simple quantum of the exponent.
In it is attained the beyond with which
the sides of the ratio are burdened;
it is in itself the unity of the two or, consequently,
in itself the other side of each side;
for each side has only as much value
as the other does not have;
its whole determinateness thus rests in the other,
and this, their being-in-itself, is as affirmative infinity
simply the exponent.

3. With this, however, we have
the transition of the inverse ratio
into a determination other than
the one it had at first.
This consisted in the fact that
the quantum is immediate
but at the same time so connected to
an other that the greater it is,
the smaller is the other,
that it is what it is by virtue of
negatively relating to the other;
also, a third magnitude is the
common restriction on this alteration
in magnitude that the two quanta undergo.
This reciprocal alteration,
as contrasted to the fixed qualitative limit,
is here their distinctive property;
they have the determination of
sum up are not only that this infinite beyond is
at the same time some present finite quantum or other,
but that its fixity
(which makes it with respect to the quantitative moment
the infinite beyond that it is,
and which is the qualitative moment of
being only as abstract self-reference)
has developed itself as a mediation
of itself with itself in its other,
the finite moments of the relation.
The general point is that the whole is as
such the limit of the reciprocal limiting of the two terms,
and that the negation of the negation
(and consequently infinity, the affirmative self-relation)
is therefore posited.
The more particular point is that, as product,
the exponent already implicitly is already
the unity of unit and amount,
whereas each of these two terms is
only one of two moments,
and for this reason the exponent encloses them in itself
and in them it implicitly refers itself to itself.
But in the inverse ratio the difference has developed
into the externality of quantitative being,
and the qualitative being is not only something fixed,
nor does it simply enclose the two moments
of the ratio immediately in it,
but in the externally existent otherness it rejoins itself.
It is this determination that stands out as a result
in the moments we have seen.
The exponent, namely, is found to be
the implicit being whose moments are realized in quanta
and in their generalized alterability.
The indifference of the magnitudes of these moments in the course of their
alteration displays itself as an infinite progression,
the basis of which is that in their indifference
their determinateness is to have their value
each in the value of the other.
Thus, (a) according to the affirmative side of their quantum,
the determinateness of the moments is that each is in itself
the whole of the exponent;
equally (b) they have the magnitude of the
exponent for their negative moment, for their reciprocal limiting;
their limit is that of the exponent.
The fact that such moments do not have any other immanent limit,
any fixed immediacy, is posited in the infinite progression
of their existence and in their limitation,
in the negation of every particular value.
This is, accordingly, the negation of the externality
of the exponent which is displayed in them, and the exponent
(itself equally a quantum as such and also expanded into quanta)
is thereby posited as preserving itself
in the negation of the indifferent subsistence of the moments,
as rejoining itself, and thus as the determining factor
in this movement of self-surpassing.

The ratio is hereby determined as the ratio of powers.

C. THE RATIO OF POWERS

1. Quantum, in positing itself as self-identical in its otherness
and in determining its own movement of self-surpassing,
has come to be a being-for-itself.
As such a qualitative totality, in positing itself as developed,
it has for its moments the conceptual determinations of number:
the unit and the amount.
This last, amount, is in the inverse ratio still an aggregate
which is not determined as amount by the unit itself but from elsewhere,
by a third determinate aggregate;
but now it is posited as determined only by the unit.
This is the case in the ratio of powers where the unit,
which in it is amount, is at the same time
the amount as against itself as unit.
The otherness, the amount of units, is the unit itself.
The power is an aggregate of units,
each of which is this aggregate itself.
The quantum, as indifferent determinateness, changes;
but inasmuch as the alteration is the raising to a power,
the otherness of the quantum is determined purely by itself.
The quantum is thus posited in the power as having returned into itself;
it is immediately itself and also its otherness.

The exponent of this ratio is no longer an immediate quantum,
as in the direct ratio and also in the inverse ratio.
In the ratio of powers, the exponent
is of an entirely qualitative nature;
it is this simple determinateness:
that the amount is the unit itself,
that the quantum is self-identical in its otherness.
And the side of its quantitative nature is to be found in this:
that the limit or negation is not an immediate existent,
but that existence is posited rather
as continuing in its otherness.
For the truth of quality is precisely to be quantity,
or immediate determinateness as sublated.

2. The ratio of powers appears at first as
an external alteration to which a given quantum is subjected;
but it has a closer connection with the concept of quantum,
namely, that in the existence into which the quantum
has developed in the ratio of powers,
quantum has attained that concept,
has realized its concept to the fullest.
The ratio of powers is the display of
what the quantum is implicitly in itself;
it expresses its determinateness of quantum
or the quality by which it is distinguished from another.
Quantum is indifferent determinateness posited as sublated,
that is to say, determinateness as limit,
one which is just as much no determinateness,
which continues in its otherness and in it,
therefore, remains identical with itself.
Thus is quantum posited in the ratio of powers:
its otherness, the surpassing of itself in another quantum,
as determined through the quantum itself.

If we compare the progressive realization
of quantum in the preceding ratios,
we find that quantum's quality of being
the difference of itself from itself is simply this:
that it is a ratio.
As the direct ratio, quantum is this posited difference
only in the first instance or immediately,
so that the self-reference which it has as exponent,
in contrast to its differences,
counts only as the fixity of an amount of the unit.
In the inverse ratio, as negatively determined,
quantum is a relating of itself to itself
(to itself as to its negation
in which, however, it has its value);
as an affirmative self-reference,
it is an exponent which, as quantum,
is only implicitly in itself the determinant of its moments.
But in the ratio of powers quantum is present
in the difference as a difference of itself from itself.
This externality of determinateness is
the quality of quantum and is thus posited,
in conformity to the concept of quantum,
as quantum's own determining,
as its reference to itself, its quality.

3. By being thus posited as it is in conformity to its concept,
quantum has passed over into another determination;
or, as we can also say, its determination is
now also as the determinateness,
the in-itself also as existence.
It is quantum in so far as the externality
or the indifference of its determining
(as we say, it is that which can be increased or decreased)
is simply accepted and immediately posited;
it has become the other of itself, namely quality,
in so far as that same externality is now posited
as mediated by quantum itself
and hence as a moment of quantum,
so that in that very externality
quantum refers itself to itself,
is being as quality.
At first quantity as such thus appears
in opposition to quality;
but quantity is itself a quality,
self-referring determinateness as such,
distinct from the determinateness which is its other,
from quality as such.
Except that quantity is not only a quality,
but the truth of quality itself is quantity,
and quality has demonstrated itself as passing over into it.
Quantity, in its truth, is instead the externality
which has returned into itself,
which is no longer indifferent.
Thus is quantity quality itself,
in such a way that outside this determination
quality as such would yet not be anything at all.

For the totality to be posited,
a double transition is required,
not only the transition of
one determinateness into the other,
but equally the transition of
this other into the first,
its going back into it.
Through the first transition,
the identity of the two is present at first only in itself:
quality is contained in quantity,
but the latter still is only a one-sided determinateness.

Conversely, that quantity is equally contained in quality,
that it is equally also only as sublated,
this results in the second transition,
the going back into the first determinateness.
This remark regarding the necessity of the double transition
is everywhere of great importance for scientific method.

Quantum is henceforth no longer an indifferent
or external determination but is sublated as such,
and it is a quality and that by virtue of which
anything is what it is;
the truth of quantum is to be measure.
