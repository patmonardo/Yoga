A. ABSOLUTE GROUND

a. Form and essence

The determination of reflection,
inasmuch as this determination returns into ground,
is a first immediate existence in general
from which the beginning is made.
But existence still has only the meaning of positedness
and essentially presupposes a ground,
in the sense that it does not really posit a ground;
that the positing is a sublating of itself;
that it is rather the immediate that is posited,
and the ground the non-posited.
As we have seen, this presupposing is the positing
that rebounds on that which posits;
as sublated determinate being, the ground is not an indeterminate
but is rather essence determined through itself,
but determined as indeterminate or as sublated positedness.
It is essence that in its negativity is identical with itself.

The determinateness of essence as ground is thus twofold:
it is the determinateness of the ground and of the grounded.
It is, first, essence as ground,
essence determined to be essence
as against positedness, as non-positedness.
Second, it is that which is grounded,
the immediate that, however, is not anything in and for itself:
is positedness as positedness.
Consequently, this positedness is equally identical with itself,
but in an identity which is that of the negative with itself.
The self-identical negative and the self-identical positive
are now one and the same identity.
For the ground is the self-identity
of the positive or even also of positedness;
the grounded is positedness as positedness,
but this its reflection-into-itself is the identity of the ground.
This simple identity, therefore, is not itself ground,
for the ground is essence posited as
the non-posited as against positedness.
As the unity of this determinate identity (the ground)
and of the negative identity (the grounded),
it is essence in general distinct from its mediation.

For one thing, this mediation,
compared with the preceding reflections
from which it derives,
is not pure reflection,
which is undistinguished from essence
and still does not have the negative in it,
consequently also does not as yet contain
the self-subsistence of the determinations.
These have their subsistence, rather,
in the ground understood as sublated reflection.
And it is also not the determining reflection
whose determinations have essential self-subsistence,
for that reflection has foundered, has sunk to the ground,
and in the unity of the latter
the determinations are only posited determinations.
This mediation of the ground is thus
the unity of pure reflection and determining reflection;
their determinations or that which is posited has self-subsistence,
and conversely the self-subsistence of
the determinations is a posited subsistence.
Since this subsistence of the determinations is
itself posited or has determinateness,
the determinations are consequently distinguished
from their simple identity,
and they constitute the form as against essence.

Essence has a form and determinations of this form.
Only as ground does it have a fixed immediacy or is substrate.
Essence as such is one with its reflection,
inseparable from its movement.
It is not essence, therefore, through which
this movement runs its reflective course;
nor is essence that from which the movement begins,
as from a starting point.
It is this circumstance that above all makes
the exposition of reflection especially difficult,
for strictly speaking one cannot say
that essence returns into itself,
that essence shines in itself,
for essence is neither before its movement nor in the movement:
this movement has no substrate on which it runs its course.
A term of reference arises in the ground only following upon the
moment of sublated reflection.
But essence as the referred-to term is determinate essence,
and by virtue of this positedness it has form as essence.
The determinations of form, on the contrary,
are now determinations in the essence;
the latter lies at their foundation
as an indeterminate which in its determination
is indifferent to them;
in it, they are reflected into themselves.
The determinations of reflection should have
their subsistence in them and be self-subsistent.
But their self-subsistence is their dissolution,
which they thus have in an other;
but this dissolution is itself this self-identity
or the ground of the subsistence that they give to themselves.

Everything determinate belongs in general to form;
it is a form determination inasmuch as it is something posited
and hence distinguished from that of which it is the form.
As quality, determinateness is one with its substrate, being;
being is the immediate determinate,
not yet distinct from its determinateness
or, in this determinateness,
still unreflected into itself,
just as the determinateness is, therefore,
an existent determinateness,
not yet one that is posited.
Moreover, the form determinations of essence are,
in their more specific determinateness,
the previously considered moments of reflections:
identity and difference, the latter as both
diversity and opposition.
But also the ground-connection belongs among
these form determinations of essence,
because through it, though itself the
sublated determination of reflection,
essence is at the same time as posited.
By contrast, the identity that has the ground immanent in it
does not pertain to form, because positedness,
as sublated and as such (as ground and grounded),
is one reflection, and this reflection constitutes
essence as simple substrate which is the subsistence of form.
But in ground this subsistence is posited,
or this essence is itself essentially as determinate and, consequently,
is in turn also the moment of the ground-connection and form.
This is the absolute reciprocal connecting reference of form and essence:
essence is the simple unity of ground and grounded
but, in this unity, is itself determined, or is a negative,
and it distinguishes itself as substrate from form,
but at the same time it thereby becomes itself
ground and moment of form.

Form is therefore the completed whole of reflection;
it also contains this determination of reflection, that it is sublated;
just like reflection, therefore, it is one unity of its determining,
and it is also referred to its sublatedness,
to another that is not itself form but in which the form is.
As essential self-referring negativity,
in contrast with that simple negative,
form is positing and determining;
simple essence, on the contrary, is indeterminate and inert
substrate in which the determinations of form have their subsistence
or their reflection into themselves.
External reflection normally halts at
this distinction of essence and form;
the distinction is necessary,
but the distinguishing itself of the two is their unity,
just as this unity of ground is essence repelling itself from itself
and making itself into positedness.
Form is absolute negativity itself
or the negative absolute self-identity
by virtue of which essence is indeed not being but essence.
This identity, taken abstractly, is essence as against form,
just as negativity, taken abstractly as positedness,
is the one determination of form.
But this determination has shown itself to be in truth
the whole self-referring negativity
which within, as this identity, thus is simple essence.
Consequently, form has essence in its own identity,
just as essence has absolute form in its negative nature.
One cannot therefore ask, how form comes to essence,
for form is only the internal reflective shining of essence,
its own reflection inhabiting it.
Form equally is, within it,
the reflection turning back into itself
or the identical essence;
in its determining, form makes the determination
into positedness as positedness.

Form, therefore, does not determine essence,
as if it were truly presupposed, separate from essence,
for it would then be the unessential,
constantly foundering determination of reflection;
here it rather is itself the ground of its sublating
or the identical reference of its determinations.
That the form determines the essence means, therefore,
that in its distinguishing form sublates this very distinguishing
and is the self-identity that essence is
as the subsistence of the determinations;
form is the contradiction of being sublated in its positedness
and yet having subsistence in this sublatedness;
it is accordingly ground as essence which
is self-identical in being determined or negated.

These distinctions, of form and of essence,
are therefore only moments of the simple reference of form itself.
But they must be examined and fixed more closely.
Determining form refers itself to itself as sublated positedness;
it thereby refers itself to its identity as to another.
It posits itself as sublated;
it therefore presupposes its identity;
according to this moment, essence is the indeterminate
to which form is an other.
It is not the essence which is absolute reflection within,
but essence determined as formless identity:
it is matter.

b. Form and matter

Essence becomes matter in that its reflection is
determined as relating itself
to essence as to the formless indeterminate.
Matter, therefore, is the simple identity,
void of distinction, that essence is,
with the determination that it is the other of form.
Hence it is the proper base or substrate of form,
since it constitutes the immanent reflection
of the determinations of form,
or the self-subsistent term,
to which such determinations refer
as to their positive subsistence.

If abstraction is made from every determination,
from every form of a something, matter is what is left over.
Matter is the absolutely abstract.
(One cannot see, feel, etc. matter;
what one sees or feels is a determinate matter,
that is, a unity of matter and form.)
This abstraction from which matter derives is not, however,
an external removal and sublation of form;
it is rather the form itself which, as we have just seen,
reduces itself by virtue of itself to this simple identity.

Further, form presupposes a matter to which it refers.
But for this reason the two do not find themselves
confronting each other externally and accidentally;
neither matter nor form derives from itself, is a se,
or, in other words, is eternal.
Matter is indifferent with respect to form,
but this indifference is the determinateness
of self-identity to which
form returns as to its substrate.
Form presupposes matter for the very reason
that it posits itself as a sublated,
hence refers to this, its identity,
as to something other.
Contrariwise, form is presupposed by matter;
for matter is not simple essence,
which immediately is itself absolute reflection,
but is essence determined as something positive,
that is to say, which only is as sublated negation.
But, on the other hand, since form posits itself
as matter only in sublating itself,
hence in presupposing matter,
matter is also determined as groundless subsistence.
Equally so, matter is not determined as the ground of form;
but rather, inasmuch as matter posits itself as
the abstract identity of the sublated determination of form,
it is not that identity as ground,
and form is therefore groundless with respect to it.
Form and matter are consequently alike determined as
not to be posited each by the other,
each not to be the ground of the other.
Matter is rather the identity of
the ground and the grounded,
as the substrate that stands over
against this reference of form.
This determination of indifference that
the two have in common is
the determination of matter as such
and also constitutes their reciprocal reference.
The determination of form, that it is
the connection of the two as distinct,
equally is also the other moment of
the relating of the two to each other.
Matter, determined as indifferent,
is the passive as contrasted to form,
which is determined as the active.
This latter, as self-referring negative,
is inherently contradiction, self-dissolving,
self-repelling, and self-determining.
It refers to matter, and it is posited to refer to this matter,
which is its subsistence, as to another.
Matter is posited, on the contrary,
as referring only to itself
and as indifferent to the other;
but, implicitly, it does refer to the form,
for it contains the sublated negativity
and is matter only by virtue of this determination.
It refers to it as an other only
because form is not posited in it,
because it is form only implicitly.
It contains form locked up inside it,
and it is an absolute receptivity for form
only because it has the latter within it absolutely,
because to be form is its implicit vocation.
Hence matter must be informed,
and form must materialize itself;
it must give itself self-identity
or subsistence in matter.

2. Consequently, form determines matter,
and matter is determined by form.
Because form is itself absolute self-identity
and hence implicitly contains matter;
and equally because matter in its pure abstraction
or absolute negativity possesses form within it,
the activity of the form on the matter
and the reception by the latter of the form determination is only
the sublating of the semblance of their indifference and distinctness.
Thus the determination referring each to the other is
the self-mediation of each through its own non-being.
But the two mediations are one movement,
and the restoration of their original identity is
the inner recollection of their exteriorization.

First, form and matter presuppose each other.
As we have seen, this only means that the one essential unity is
negative self-reference, and that it therefore splits,
determined as an indifferent substrate in the essential identity,
and as determining form in essential distinction or negativity.
That unity of essence and form, the two opposed to each other as
form and matter, is the absolute self-determining ground.
Inasmuch as this unity differentiates itself,
the reference connecting the two diverse terms,
because of the unity that underlies them,
becomes a reference of reciprocal presupposition.

Second, the form already is, as self-subsisting,
self-sublating contradiction;
but it is also posited as in this way self-sublating,
for it is self-subsisting and at the same time
essentially referred to another,
and consequently it sublates itself.
Since it is itself two-sided, its sublating also has two sides.
For one, form sublates its self-subsistence
and transforms itself into something posited,
something that exists in an other,
and this other is in its case matter.
For the other, form sublates its determinateness vis-à-vis matter,
sublates its reference to it, consequently its positedness,
and it thereby gives itself subsistence.
Its reflection in thus sublating its positedness is
its own identity into which it passes over.
But since form at the same time externalizes this identity
and posits it over against itself as matter,
that reflection of the positedness into itself is
a union with a matter in which it obtains subsistence.
In this union, therefore, it is equally both:
is united with matter as with something other
(in accordance with the first side, viz. in that it makes
itself into a positedness),
and, in this other, is united with its own identity.

The activity of form by which matter is determined consists,
therefore, in a negative relating of the form to itself.
But, conversely, form thereby negatively relates itself to matter also;
the movement, however, by which matter becomes determined is
just as much the form's own movement.
Form is free of matter, but it sublates its self-subsistence;
but this, its self-subsistence, is matter itself,
for it is in this matter that it has its essential identity.
It makes itself into a positedness, but this is one and the same
as making matter into something determinate.
But, considered from the other side,
the form's own identity is at the same time externalized,
and matter is its other;
for this reason, because form sublates its own self-subsistence,
matter is also not determined.
But matter only subsists vis-à-vis form;
as the negative sublates itself, so does the positive also.
And as the form sublates itself, the determinateness of matter
that the latter has vis-à-vis form also falls away
the determinateness, namely, of being the indeterminate subsistence.

What appears here as the activity of form is, moreover,
just as much the movement that belongs to matter itself.
The determination that implicitly exists in matter,
what matter is supposed to be, is its absolute negativity.
Through it matter does not just refer to form simply as to an other,
but this external other is the form rather that
matter itself contains locked up within itself.
Matter is in itself the same contradiction that form contains,
and this contradiction, like its resolution, is only one.
But matter is thus in itself self-contradictory because,
as indeterminate self-identity,
it is at the same time absolute negativity;
it sublates itself within:
its identity disintegrates in its negativity
while the latter obtains in it its subsistence.
Since matter is therefore determined by form as by something external,
it thereby attains its determination,
and the externality of the relating, for both form and matter,
consists in that each, or rather in that the original unity of each,
in positing is at the same time presupposing:
the result is that self-reference is at the same time
a reference to the self as sublated or is reference to its other.

Third, through this movement of form and matter,
the original unity of the two is, on the one hand, restored;
on the other hand, it is henceforth a posited unity.
Matter is just as much a self-determining as
this determining is for it an activity of form external to it;
contrariwise, form determines only itself,
or has the matter that it determines within it,
just much as in its determining it relates itself to another;
and both, the activity of form and the movement of matter,
are one and the same thing, only that the former is an activity,
that is, it is the negativity as posited,
while the latter is movement or becoming,
the negativity as determination existing in itself.
The result, therefore, is the unity of the in-itself and positedness.
Matter is as such determined or necessarily has a form,
and form is simply material, subsistent form.

Inasmuch as form presupposes a matter as its other, it is finite.
It is not a ground but only the active factor.
Equally so, matter, inasmuch as it presupposes
form as its non-being, is finite matter;
it is not the ground of its unity with form
but is for the latter only the substrate.
But neither this finite matter nor the finite form have any truth;
each refers to the other, or only their unity is their truth.
The two determinations return to
this unity and there they sublate their self-subsistence;
the unity thereby proves to be their ground.
Consequently, matter is the ground
of its form determination not as matter
but only inasmuch as it is the absolute unity of essence and form;
similarly, form is the ground of the subsistence of its determinations
only to the extent that it is that same one unity.
But this one unity, as absolute negativity,
and more specifically as exclusive unity,
is, in its reflection, a presupposing;
or again, that unity is one act,
of preserving itself as positedness in positing,
and of repelling itself from itself;
of referring itself to itself as itself
and to itself as to another.
Or, the act by which matter is determined by form
is the self-mediation of essence as ground, in one unity:
through itself and through the negation of itself.

Informed matter or form that possesses subsistence is now,
not only this absolute unity of ground with itself,
but also unity as posited.
The movement just considered is the one
in which the absolute ground has exhibited
its moments at once as self-sublating
and consequently as posited.
Or the restored unity, in withdrawing into itself,
has repelled itself from itself and
has determined itself;
for its unity has been established through negation
and is, therefore, also negative unity.
It is, therefore, the unity of form and matter,
as the substrate of both, but a substrate which is determinate:
it is formed matter, but matter at the same time
indifferent to form and matter,
indifferent to them because sublated and unessential.
This is content.

c. Form and content

Form stands at first over against essence;
it is then the ground-connection in general,
and its determinations are the ground and the grounded.
It then stands over against matter,
and so it is determining reflection,
and its determinations are the determination of reflection itself
and the subsistence of the latter.
Finally, it stands over against content,
and then its determinations are again itself and matter.
What was previously the self-identical
at first the ground,
then subsistence in general,
and finally matter
now passes under the dominion of form
and is once more one of its determinations.

Content has, first, a form and a matter that
belong to it essentially; it is their unity.
But, because this unity is at the same time determinate
or posited unity, content stands over against form;
the latter constitutes the positedness
and is the unessential over against content.
The latter is therefore indifferent towards form;
form embraces both the form as such as well as the matter,
and content therefore has a form and a matter,
of which it constitutes the substrate
and which are to it mere positedness.

Content is, second, what is identical in form and matter,
so that these would be only indifferent external determinations.
They are positedness in general, but a positedness
that has returned in the content to its unity or its ground.
The identity of the content with itself is,
therefore, in one respect that identity which is indifferent to form,
but in another the identity of ground.
The ground has at first disappeared into content;
but content is at the same time the negative reflection of
the form determinations into themselves;
its unity, at first only the unity indifferent to form, is
therefore also the formal unity or the ground-connection as such.
Content, therefore, has this ground-connection as its essential form,
and, contrariwise, the ground has a content.

The content of the ground is therefore the ground
that has returned into its unity with itself;
the ground is at first the essence that in its
positedness is identical with itself;
as diverse from and indifferent to its positedness,
the ground is indeterminate matter;
but as content it is at the same time informed identity,
and this form becomes for this reason a ground-connection,
since the determinations of its oppositions are posited
in the content also as negated.
Content is further determined within,
not like matter as an indifferent in general,
but like informed matter,
so that the determinations of form have
a material, indifferent subsistence.
On the one hand, content is the essential self-identity
of the ground in its positedness;
on the other hand, it is posited identity
as against the ground-connection;
this positedness, which is in this identity as determination of form,
stands over against the free positedness, that is to say,
over against the form as the whole connection of ground and grounded.
This form is the total positedness returning into itself;
the other form, therefore, is only the positedness as immediate,
the determinateness as such.

The ground has thus made itself into a determinate ground in general,
and the determinateness is itself twofold:
of form first, and of content second.
The former is its determinateness of being external to the content as such,
the content that remains indifferent to this external reference.
The latter is the determinateness of the content that the ground has.

B. DETERMINATE GROUND

a. Formal ground

The ground has a determinate content.
For the form, as we have seen,
the determinateness of content is the substrate,
the simple immediate as against the mediation of form.
The ground is negatively self-referring identity which,
for this reason, makes itself into a positedness;
it negatively refers to itself because in its negativity
it is identical with itself;
this identity is the substrate or the content
which thus constitutes the indifferent
or positive unity of the ground-connection
and, in this connection, is the mediating factor.

In this content, the determinateness that
the ground and the grounded have over
against one another has at first disappeared.
The mediation, however, is also negative unity.
The negative implicit in that indifferent substrate is
this substrate's immediate determinateness through which
the ground has a determinate content.
But then, the negative is the negative reference of form to itself.
What has been posited sublates itself on its side
and returns to its ground;
the ground, however, the essential self-subsistence,
refers negatively to itself and makes itself into a positedness.
This negative mediation of ground and grounded is
the mediation that belongs to form as such, formal mediation.
Now both sides of form, because each passes over into the other,
thereby mutually posit themselves into one identity as sublated;
in this, they presuppose the identity.
The latter is the determinate content
to which the formal mediation thus refers itself
through itself as to the positive mediating factor.
That content is the identical element of both,
and because the two are distinct,
yet in their distinction each is
the reference to the other,
it is their subsistence,
the subsistence of each as the whole itself.

Accordingly, the result is that
in the determinate ground we have the following.
First, a determinate content is considered from two sides,
once in so far as it is ground,
then again in so far as it is grounded.
The content itself is indifferent to these forms;
it is in each simply and solely one determination.
Second, the ground is itself just as much a moment of form
as what is posited by it; this is its identity according to form.
It is a matter of indifference which of
the two determinations is made the first,
whether the transition is
from the one as posited to the other as ground
or from the one as ground to the other as posited.
The grounded, considered for itself, is the sublating of itself;
it thereby makes itself on the one side into a posited,
and is at the same time the positing of the ground.
The same movement is the ground as such;
it makes itself into something posited,
and thereby becomes the ground of something,
that is to say, is present therein
both as a posited and also first as ground.
That there be a ground, of that the posited is the ground,
and, conversely, the ground is thereby the posited.
The mediation begins just as much from the one as from the other;
each side is just as much ground as posited,
and each is the whole mediation or the whole form.
Further, this whole form is itself, as self-identical,
the substrate of the two determinations
that constitute the two sides of the ground and the grounded;
form and content are thus themselves one and the same identity.

Because of this identity of the ground and the grounded,
according both to content and form,
the ground is sufficient
(the sufficiency being limited to this relation);
there is nothing in the grounded which is not in the ground.
Whenever one asks for a ground,
one expects to see the same determination
which is the content doubled,
once in the form of that which is posited,
and again in the form of existence
reflected into itself, of essentiality.

Now inasmuch as in the determined ground,
the ground and the grounded are each the whole form, and their content,
though determinate, is nevertheless one and the same,
the two sides of the ground do not as yet have a real determination,
do not have a different content;
the determinateness is only one simple determinateness
that has yet to pass over into the two sides;
the determinate ground is present
only in its pure form, as formal ground.
Because the content is only this simple determinateness,
one that does not have in it the form of the ground-connection,
the determinateness is a self-identical content indifferent to form,
and the form is external to it;
the content is other than the form.

b. Real ground

The determinateness of ground is, as we have seen,
on the one hand determinateness of the substrate
or content determination;
on the other hand,
it is the otherness in the ground-connection itself,
namely the distinctness of its content and the form;
the connection of ground and grounded strays
in the content as an external form,
and the content is indifferent to these determinations.
But in fact the two are not external to each other;
for this is what the content is:
to be the identity of the ground
with itself in the grounded,
and of the grounded in the ground.
The side of the ground
has shown itself to be itself a posited,
and the side of the grounded to be
itself ground;
each side is this identity of the whole within it.
But since they equally belong to form
and constitute its determinate difference,
each is in its determinateness the identity of the whole with itself.
Consequently, each has a diverse content as against the other.
Or, considering the matter from the side of the content,
since the latter is the self-identity of the ground-connection,
it essentially possesses this difference of form within,
and is as ground something other than what it is as grounded.

Now the moment ground and grounded have a diverse content,
the ground-connection has ceased to be a formal one;
the turning back to the ground and
the procession forward from ground to posited
is no longer a tautology; the ground is realized.
Henceforth, whenever we ask for a ground,
we actually demand another content determination for it
than the determination of the content whose ground we are asking for.

This connection now determines itself further.
For inasmuch as its two sides are of different content,
they are indifferent to each other;
each is an immediate, self-identical determination.
Moreover, as referred to each other as ground and grounded,
the ground reflects itself in the other,
as in something posited by it, back to itself;
the content on the side of the ground,
therefore, is equally in the grounded;
the latter, as the posited, has its
self-identity and subsistence only in the ground.
But besides this content of the ground,
the grounded also now possesses a content of its own
and is accordingly the unity of a twofold content.
Now this unity, as the unity of sides that are different,
is indeed their negative unity;
but since the two determinations of content are indifferent to each other,
that unity is only their empty reference to each other,
in itself void of content, and not their mediation;
it is a one or a something externally holding them together.

In the real grounding connection
there is present, therefore, a twofold.
For one thing, the content determination which is ground
extends continuously into the positedness,
so that it constitutes the simple identity
of the ground and the grounded;
the grounded thus contains the ground
fully within itself;
their connection is one of
undifferentiated essential compactness.
Anything else in the grounded
added to this simple essence is,
therefore, only an unessential form,
external determinations of the content
that, as such, are free from the ground
and constitute an immediate manifold.
Of this unessential more, therefore,
the essential is not the ground,
nor is it the ground of any connection
between it and the unessential in the grounded.
The unessential is a positively identical element
that resides in the grounded but does not posit itself
there in any distinctive form;
as self-referring content, it is rather
an indifferent positive substrate.
For another thing, that which in the something is
linked with this substrate is an indifferent content,
but as the unessential side.
The main thing is the connection of the substrate
and the unessential manifold.
But this connection, since the determinations
that it connects are an indifferent content,
is also not a ground;
true, one determination is determined as essential content
and the other as only unessential or as posited;
but this form is to each, as a self-referring content, an external one.
The one of the something that constitutes their connection is
for this reason not a reference of form,
but only an external tie that does not hold
the unessential manifold content as posited;
it too is therefore likewise only a substrate.

Ground, in determining itself as real,
because of the diversity of the content
that constitutes its reality,
thus breaks down into external determinations.
The two connections of the essential reality content,
as the simple immediate identity of ground and grounded;
and then the something connecting distinct contents
are two different substrates.
The self-identical form of ground,
according to which one and the same thing
is at one time the essential
and at another the posited, has vanished.
The ground-connection has thus become external to itself.

Consequently, it is an external ground that now
holds together a diversified content
and determines what is ground and what is posited by it;
this determination is not to be found in the two-sided content itself.
The real ground is therefore the reference to another,
on the one hand, of a content to another content
and, on the other, of the ground-connection itself
(the form) to another, namely to an immediate,
to something not posited by it.

c. Complete ground

1. In real ground, ground as content
and ground as connection are only substrates.
The former is only posited as essential and as ground;
the connection is what the grounded immediately is
as the indeterminate substrate of a diversified content,
a linking of this content which is not the content's own reflection
but is rather external and consequently a reflection which is only posited.
The real ground-connection is ground, therefore, rather as sublated;
consequently, it rather makes up the side of
the grounded or of the positedness.
As positedness, however, the ground itself
has now returned to its ground;
it is now something grounded: it has another ground.
This ground will therefore be so determined that,
first, it is identical with the ground by which it is grounded;
both sides have in this determination one and the same content;
the two content determinations and their linkage in
a something are equally to be found in the new ground.
But, second, the new ground into which
the previously merely posited and external link
is now sublated is the immanent reflection of this link:
the absolute reference of the two content determinations to each other.

Because real ground has itself thus returned to its ground,
the identity of ground and grounded
or the formality of ground reasserts itself in it.
The newly arisen ground-connection is
therefore the one which is complete,
which contains the formal and real ground in itself
at the same time and mediates the content determinations
which in the real ground confronted each other immediately.

2. Thus the ground-connection has more precisely
determined itself as follows.

First, something has a ground;
it contains the content determination which is the ground
and, in addition, a second determination as posited by the ground.
But, because of the indifference of content,
the one determination is not ground in itself,
nor is the other in itself one that is grounded by the first;
this connection of ground and grounded is rather
sublated in the immediacy of their content, is posited,
and as such has its ground in another such connection.

Since this second connection is
distinguished only according to form,
it has the same content as the first;
it still has the same two determinations of content
but is now their immediate linking together.
This linking, however, is of a general nature,
and the content, therefore, is diversified into determinations
that are indifferent to each other.
The linking is not, therefore, their true absolute connection
that would make one determination the element of
self-identity in the positedness,
and the other determination
the positedness of this same self-identity;
on the contrary, the two are supported by a something
and this something is what connects them,
but in a connection which is not reflected,
is rather only immediate and, therefore,
only a relative ground as against
the linking in the other something.
The two somethings are therefore the two distinct
connections of content that have transpired.
They stand in the identical ground-connection of form;
they are one and the same whole content,
namely the two content determinations and their connection;
they are distinct only by the kind of this connection,
which in the one is an immediate
and in the other a posited connection;
through this, they are distinguished
one from another as ground and grounded only according to form.
Second, this ground-connection is not only formal, but also real.
Formal ground passes over into real ground, as has been shown;
the moments of the form reflect themselves into themselves;
they are a self-subsistent content,
and the ground-connection contains
also one content with the character of ground
and another with that of grounded.
The content constitutes at first the immediate
identity of both sides of the formal ground;
so the two sides have one and the same content.
But the content also has the form in it,
and so it is a twofold content
that behaves as ground and grounded.
One of the two content determinations of
the two somethings is therefore determined,
not merely as being common to them
according to external comparison,
but as their identical substrate
and the foundation of their connection.
As against the other determination of the content,
this determination is essential
and is the ground of the other which is posited,
that is, posited in the something,
the connection of which is the grounded.
In the first something, which is the ground-connection,
this second determination of the content is
also immediately and in itself linked with the first.
But the other something only contains
the one determination in itself as that
in which it is immediately identical with the first something,
but the other as the one which is posited in it.
The former content determination is its
ground by virtue of its being originally linked
in the first something with
the other content determination.

The ground-connection of the content determinations
in the second something is thus mediated
through the connection present in the first something.
The inference is this:
since determination B is implicitly linked
with determination A in a something,
in a second something to which only
the one determination A immediately belongs,
also B is linked with it.
In the second something, not only is
this second determination mediated;
also mediated is that its immediate ground is mediated,
namely by virtue of its original connection
with B in the first something.
This connection is thus the ground of the ground A,
and the whole ground-connection is present in
the second something as posited or grounded.

3. Real ground shows itself to be the self-external reflection of ground;
its complete mediation is the restoration of its identity with itself.
But because this identity has in the process equally acquired
the externality of real ground,
the formal ground-connection in this unity
of itself and real ground is just as much
self-positing as self-sublating ground;
the ground-connection mediates itself with itself through its negation.
The ground is at first, as the original connection,
the connection of immediate content determinations.
The ground-connection, being essential form,
has for sides such that are sublated or are as moments.
Consequently, as the form of immediate determinations,
it connects itself with itself as self-identical
while at the same time connecting with their negation;
accordingly, it is ground not in and for itself
but as connected with the sublated ground-connection.
Second, the sublated connection or the immediate,
which in the original and in the posited connection
is the identical substrate, is likewise real ground
not in and for itself; that it is ground is
rather posited by virtue of that original link.

Thus the ground-connection is in its totality
essentially presupposing reflection;
formal ground presupposes the immediate content determination,
and this content presupposes form as real ground.
Ground is therefore form as an immediate linkage
but in such a manner that it repels itself from itself
and rather presupposes immediacy,
referring itself therein as to another.
This immediate is the content determination, the simple ground;
but as such, that is, as ground, it is equally repelled from itself
and refers itself to itself equally as to an other.
Thus the total ground-connection has taken on
the determination of conditioning mediation.

C. CONDITION

a. The relatively unconditioned

1. Ground is the immediate,
and the grounded the mediated.
But ground is positing reflection;
as such, it makes itself into positedness
and is presupposing reflection;
as such it refers itself to itself
as to something sublated,
to an immediate through which
it is itself mediated.
This mediation, as an advance
from the immediate to the ground,
is not an external reflection
but, as we have seen, the ground's own doing
or, what is the same, the ground-connection,
as reflection into its self-identity,
is just as essentially self-externalizing reflection.
The immediate to which ground refers as
to its essential presupposition is condition;
real ground is accordingly essentially conditioned.
The determinateness that it contains is
the otherness of itself.
Condition is therefore,
first, an immediate, manifold existence.
Second, it is this existence referred to an other,
to something which is ground,
not of this existence but in some other respect,
for existence itself is immediate and without ground.
According to this reference, it is something posited;
as condition, the immediate existence is supposed to be
not for itself but for another.
But this, that it thus is for another, is at the same time
itself only a positedness;
that it is posited is sublated in its immediacy:
an existence is indifferent to being a condition.
Third, condition is something immediate in the sense
that it constitutes the presupposition of ground.
In this determination, it is the form-connection of ground
withdrawn into self-identity, hence the content of ground.
But content is as such only the indifferent unity of ground,
as in the form: without form, no content.
It nevertheless frees itself
from this indifferent unity
in that the ground-connection,
in the complete ground,
becomes a connection external to its identity,
whereby content acquires immediacy.
In so far, therefore, as condition is
that in which the ground-connection has
its identity with itself,
it constitutes the content of ground;
but since this content is indifferent to form,
it is only implicitly the content of form,
is something which has yet to become content
and hence constitutes the material for the ground.
Posited as condition,
and in accordance with the second moment,
existence is determined to lose its indifferent immediacy
and to become the moment of another.
By virtue of its immediacy, it is indifferent to this connection;
inasmuch as it enters into it, however,
it constitutes the in-itself of the ground
and is for it the unconditioned.
In order to be condition,
it has its presupposition in the ground
and is itself conditioned;
but this condition is external to it.

2. Something is not through its condition;
its condition is not its ground.
Condition is for the ground
the moment of unconditioned immediacy,
but is not itself the movement and the positing
that refers itself to itself negatively
and that makes itself into a positedness.
Over against condition there stands,
therefore, the ground-connection.
Something has, besides its condition, also a ground.
This ground is the empty movement of reflection,
for the latter has the immediacy
which is its presupposition outside it.
But it is the whole form
and the self-subsistent process of mediation,
for the condition is not its ground.
Since this mediating refers itself to itself as positing,
it equally is according to this side
something immediate and unconditioned;
it does indeed presuppose itself,
but as an externalized or sublated positing;
whatever it is in accordance with its determination,
that it is, on the contrary, in and for itself.
Inasmuch as the ground-connection is
thus a self-subsisting self-reference
and has within it the identity of reflection,
it has a content which is peculiarly its own
as against the content of the condition.
The one content is that of the ground
and is therefore essentially informed;
the other content, that of the condition,
is on the contrary only an immediate material
whose connecting reference to the ground,
while at the same time constituting
the in-itself of the latter,
is also equally external to it;
it is thus a mingling of a self-subsisting content
that has no reference to the content of the ground determination
and of the content that enters into the latter
and, as its material,
should become a moment of it.

3. The two sides of the whole,
condition and ground,
are thus, on the one hand,
indifferent and unconditioned
with respect to each other:
the one as the non-referred-to side,
to which the connecting reference
in which it is the condition is external;
the other as the connecting reference, or form,
for which the determinate existence of
the condition is only a material,
something passive whose form,
such as it possesses on its own account,
is unessential.
On the other hand, the two sides are also mediated.
Condition is the in-itself of the ground;
so much is it the essential moment of the ground-connection,
that it is the simple self-identity of the ground.
But this also is sublated;
this in-itself is only something posited;
immediate existence is indifferent to being a condition.
The fact, therefore, that condition is the in-itself
of the ground constitutes the side of it
by which it is a mediated condition.
Likewise, the ground-connection has
in its self-subsistence also a presupposition;
it has its in-itself outside itself.
Consequently, each of the two sides is this contradiction,
that they are indifferent immediacy and essential mediation,
both in one reference
or the contradiction of independent subsistence
and of being determined as only moments.

b. The absolutely unconditioned

At first, each of the two relatively unconditioned sides
reflectively shines in the other;
condition, as an immediate, is reflected
in the form connection of the ground,
and this form in the immediate existence as its positedness;
but each, apart from this reflective shine of its other in it,
stands out on its own and has a content of its own.

Condition is at first immediate existence;
its form has these two moments:
that of positedness, according to which it is, as condition,
material and moment of the ground;
and that of the in-itself, according to which
it constitutes the essentiality of ground
or its simple reflection into itself.
Both sides of the form are external to immediate existence,
for the latter is the sublated ground-connection.

But, first, existence is in it only this:
to sublate itself in its immediacy
and to founder, going to the ground.
Being is as such only the becoming of essence;
it is its essential nature to
make itself into a positedness
and into an identity which is
an immediacy through the negation of itself.
The form determinations of positedness
and of self-identical in-itself,
the form through which immediate existence is condition,
are not, therefore, external to that existence;
the latter is, rather, this very reflection.

Second, as condition, being is now posited as
that which it essentially is,
namely as a moment and consequently as the being of an other,
and at the same time as the in-itself of an other;
it is in itself but only through the negation of itself,
namely through the ground and through its self-sublating
and consequent presupposing reflection;
the in-itself of being is thus only something posited.
This in-itself of the condition has two sides:
one side is its essentiality as essentiality of the ground,
while the other is the immediacy of its existence.
Or rather, both sides are the same thing.
Existence is an immediate, but immediacy
is essentially something mediated,
namely through the self-sublating ground.
Existence, as this immediacy mediated by a self-sublating mediating,
is at the same time the in-itself of the ground and its unconditioned side;
but again, this in-itself is at the same time itself
equally only moment or positedness, since it is mediated.
Condition is, therefore, the whole form of the ground-connection;
it is the presupposed in-itself of the latter,
but, consequently, is itself a positedness
and its immediacy is this, to make itself into a positedness
and thereby to repel itself from itself,
in such as way that it both founders to the ground and is ground,
the ground that makes itself into a positedness
and thereby into a grounded, and both are one and the same.

Likewise in the conditioned ground, the in-itself is not
just as the reflective shining of an other in it.
This ground is the self-subsistent,
that is, self-referring reflection of the positing,
and consequently the self-identical;
or it is in it its in-itself and its content.
But it is at the same time presupposing reflection;
it negatively refers to itself
and posits its in-itself as an other opposite to it,
and condition, according to both its moment of in-itself
and of immediate existence, is
the ground-connection's own moment;
the immediate existence essentially is only through its ground
and is a moment of itself as a presupposing.
This ground, therefore, is equally the whole itself.

What we have here, therefore, is only one whole of form,
but equally so only one whole of content.
For the proper content of condition is essential content
only in so far as it is the self-identity of reflection in the form,
or the ground-connection is in it this immediate existence.
Further, this existence is condition only through
the presupposing reflection of the ground;
it is the ground's self-identity, or its content,
to which the ground posits itself as opposite.
Therefore, the existence is not a merely
formless material for the ground-connection;
on the contrary, because it has this form in it, it is informed matter,
and because in its identity with it it is at the same time
indifferent to it, it is content.
Finally, it is the same content as that possessed by the ground,
for it is precisely content as that which is self-identical
in the form connection.

The two sides of the whole,
condition and ground,
are therefore one essential unity,
as content as well as form.
They pass into one another,
or, since they are reflections,
they posit themselves as sublated,
refer themselves to this their negation,
and reciprocally presuppose each other.
But this is at the same time only one reflection of the two,
and their presupposing is, therefore, one presupposing only;
the reciprocity of this presupposing ultimately amounts to this,
that they both presuppose one identity
for their subsistence and their substrate.
This substrate, the one content and unity of form of both,
is the truly unconditioned; the fact in itself.
Condition is, as it was shown above, only the relatively unconditioned.
It is usual, therefore, to consider it as itself something conditioned
and to ask for a new condition,
whereby the customary progression ad infinitum
from condition to condition is set in motion.
But now, why is it that at one condition
a new condition is asked for, that is,
why is that condition assumed to be something conditioned?
Because it is some finite determinate existence or other.
But this is a further determination of condition
that does not enter into its concept.
Condition is as such conditioned solely because
it is the posited in-itselfness;
it is, therefore, sublated in the absolutely unconditioned.

Now this contains within itself the two sides,
condition and ground, as its moments;
it is the unity to which they have returned.
Together, the two constitute its form or its positedness.
The unconditioned fact is the condition of both,
but the condition which is absolute, that is to say,
one which is itself ground.
As ground, the fact is now the negative identity
that has repelled itself into those two moments:
first, in the shape of the sublated ground-connection,
the shape of an immediate manifold void of
unity and external to itself,
one that refers to the ground as an other to it
and at the same time constitutes its in-itself;
second, in the shape of an inner, simple form which is ground,
but which refers to the self-identical
immediate as to an other, determining it as condition,
that is, determining the in-itself of it as its own moment.
These two sides presuppose the totality,
presuppose that it is that which posits them.
Contrariwise, because they presuppose the totality,
the latter seems to be in turn also conditioned by them,
and the fact to spring forth from its condition and its ground.
But since these two sides have shown themselves to be an identity,
the relation of condition and ground has disappeared;
the two are reduced to a mere reflective shine;
the absolutely unconditioned is in its movement of positing
and presupposing only the movement in which this shine sublates itself.
It is the fact's own doing that it conditions itself
and places itself as ground over against its conditions;
but in connecting conditions and ground,
the fact is a reflection shining in itself;
its relation to them is a rejoining itself.

c. Procession of the fact into concrete existence

The absolutely unconditioned is the absolute ground
that is identical with its condition,
the immediate fact as the truly essential.
As ground, it refers negatively to itself
and makes itself into a positedness;
but this positedness is a reflection
that is complete in both its sides
and is in them the self-identical form of connection,
as has transpired from its concept.
This positedness is therefore first the sublated ground,
the fact as an immediacy void of reflection,
the side of the conditions.
This is the totality of the determinations of the fact,
the fact itself, but the fact as thrown into
the externality of being, the restored circle of being.
In condition, essence lets go of the unity of its immanent reflection;
but it lets it go as an immediacy that now carries
the character of being a conditioning presupposition
and of essentially constituting only one of its sides.
For this reason the conditions are the whole content of the fact,
because they are the unconditioned in the form of formless being.
But because of this form, they also have yet another shape besides
the conditions of the content as this is in the fact as such.
They appear as a manifold without unity,
mingled with extra-essential elements
and other circumstances that do not belong
to the circle of existence as constituting
the conditions of this determinate fact.
For the absolute, unrestricted fact,
the sphere of being itself is the condition.
The ground, returning into itself, posits
that sphere as the first immediacy
to which it refers as to its unconditioned.
This immediacy, as sublated reflection,
is reflection in the element of being,
which thus forms itself as such into a whole;
form proliferates as determinateness of being
and thus appears as a manifold distinct from
the determination of reflection
and as a content indifferent to it.
The unessential, which is in the sphere of being
but which the latter sheds in so far as it is condition,
is the determinateness of the immediacy into which
the unity of form has sunk.
This unity of form, as the connection of being,
is in the latter at first as becoming the passing over
of a determinateness of being into another.
But the becoming of being is also the coming to be
of essence and a return to the ground.
The existence that constitutes the conditions, therefore,
is in truth not determined as condition by an other
and is not used by it as material;
on the contrary, it itself makes itself, through itself,
into the moment of an other.
Further, the becoming of this existence
does not start off from itself
as if it were truly the first and immediate;
on the contrary, its immediacy is
something only presupposed, and the movement of
its becoming is the doing of reflection itself.
The truth of existence is thus that it is condition;
its immediacy is solely by virtue of
the reflection of the ground-connection
that posits itself as sublated.
Consequently, like immediacy, becoming is only
the reflective shine of the unconditioned
inasmuch as this presupposes itself
and has its form in this presupposing,
and hence the immediacy of being is essentially
only a moment of the form.

The other side of this reflective shining of
the unconditioned is the ground-connection as such,
determined as form as against the immediacy
of the conditions and the content.
But this side is the form of the absolute fact
that possesses the unity of its form with itself
or its content within it,
and, in determining this content as condition,
in this very positing sublates the diversity of the content
and reduces it to a moment;
just as, contrariwise, as a form void of essence,
in this self-identity it gives itself
the immediacy of subsistence.
The reflection of the ground
sublates the immediacy of the conditions,
connecting them and making them
moments within the unity of the fact;
but the conditions are that which
the unconditioned fact itself presupposes
and the latter, therefore, sublates its own positing;
consequently, its positing converts itself
just as immediately into a becoming.
The two, therefore, are one unity;
the internal movement of the conditions is a becoming,
the return into the ground and the positing of the ground;
but the ground as posited, and this means as sublated, is the immediate.
The ground refers negatively to itself,
makes itself into a positedness and grounds the conditions;
in this, however, in that the immediate existence is
thus determined as a positedness,
the ground sublates it and only then makes itself into a ground.
This reflection is therefore the self-mediation of
the unconditioned fact through its negation.
Or rather, the reflection of the unconditioned is
at first a presupposing,
but this sublating of itself is
immediately a positing which determines;
secondly, in this positing the reflection is
immediately the sublating of the presupposed
and a determining from within itself;
this determining is thus in turn the sublating of the positing:
it is a becoming within itself.
In this, the mediation as a turning back
to itself through negation has disappeared;
mediation is simple reflection
reflectively shining within itself
and groundless, absolute becoming.
The fact's movement of being posited,
on the one hand through its conditions,
and on the other hand through its ground,
now is the disappearing of
the reflective shine of mediation.
The process by which the fact is posited
is accordingly a coming forth,
the simple self-staging of
the fact in concrete existence,
the pure movement of the fact to itself.

When all the conditions of a fact are at hand,
the fact steps into concrete existence.
The fact is, before it exists concretely;
it is, first, as essence or as unconditioned;
second, it has immediate existence or is determined,
and this in the twofold manner just considered,
on the one hand in its conditions
and on the other in its ground.
In the former case, it has given itself the form
of the external, groundless being,
for as absolute reflection the fact is
negative self-reference
and makes itself into its presupposition.
This presupposed unconditioned is,
therefore, the groundless immediate
whose being is just to be there, without grounds.
If, therefore, all the conditions of the fact are at hand,
that is, if the totality of the fact is
posited as a groundless immediate,
then this scattered manifold
internally recollects itself.
The whole fact must be there,
within its conditions,
or all the conditions belong
to its concrete existence;
for the all of them constitutes
the reflection of the fact.
Or again, immediate existence,
since it is condition,
is determined by form;
its determinations are therefore determinations of reflection
and with the positing of one the rest also are essentially posited.
The recollecting of the conditions is at first
the foundering to the ground of immediate existence
and the coming to be of the ground.
But the ground is thereby a posited ground, that is,
to the extent that it is ground,
to that extent it is sublated as ground
and is immediate being.
If, therefore, all the conditions of the fact are at hand,
they sublate themselves as immediate existence and as presupposition,
and the ground is equally sublated.
The latter proves to be only a reflective shine
that immediately disappears;
this coming forth is thus the tautological movement
of the fact to itself:
its mediation through the conditions and through the ground
is the disappearing of both of these.
The coming forth into concrete existence is therefore so immediate,
that it is mediated only by the disappearing of the mediation.

The fact proceeds from the ground.
It is not grounded or posited by it
in such a manner that the ground
would still stay underneath, as a substrate;
on the contrary, the positing is
the outward movement of ground to itself
and the simple disappearing of it.
Through its union with the conditions,
it obtains the external immediacy
and the moment of being.
But it does not obtain them
as a something external,
nor by referring to them externally;
rather, as ground it makes
itself into a positedness;
its simple essentiality rejoins
itself in the positedness
and, in this sublating of itself,
it is the disappearing of
its difference from its positedness,
and is thus simple essential immediacy.
It does not, therefore, linger on
as something distinct from the grounded;
on the contrary, the truth of the grounding is
that in grounding the ground unites with itself,
and its reflection into another is
consequently its reflection into itself.
The fact is thus the unconditioned
and, as such, equally so the groundless;
it arises from the ground only in so far as
the latter has foundered and is no longer ground:
it rises up from the groundless, that is,
from its own essential negativity or pure form.

This immediacy, mediated by ground and condition
and self-identical through the sublating of mediation,
is concrete existence.
