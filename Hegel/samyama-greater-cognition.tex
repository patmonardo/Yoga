A. THE IDEA OF THE TRUE

At first the subjective idea is impulse.
For it is the contradiction of the concept
that it has itself for the subject matter
and is to itself the reality without, however,
the subject matter being an other that subsists
on its own over against it,
or without the differentiation
of itself from itself
having at the same time
the essential determination of diversity
and of indifferent existence.
The specific nature of this impulse is therefore
to sublate its own subjectivity,
to make that first abstract reality a concrete one,
filling it with the content of the world
presupposed by its subjectivity.
From the other side, the impulse is determined in this way:
the concept is indeed the absolute certainty of itself;
however, opposite its being-for-itself
there stands its presupposition of a world
that exists in itself,
but one whose indifferent otherness
has for the concept's certainty of itself
the status of something merely unessential;
the concept is therefore the impulse
to sublate this otherness
and, in the object, to intuit
its identity with itself.
This immanent reflection is
the sublated opposition
and the singularity that originally
makes its appearance as
the presupposed being-in-itself
but is now posited and made actual for the subject;
accordingly, this being-in-itself is
the self-identity of the form
as it has issued from the opposition,
an identity which is therefore determined as
indifferent towards the form
in its differentiation.
It is content.

Consequently this impulse is
the impulse of truth in so far as
the truth is in cognition,
and therefore of truth in its strict sense
as theoretical idea.
Although objective truth is
the idea itself as the reality
that corresponds to the concept,
and to this extent a subject matter
may or may not possess truth,
nevertheless the more precise meaning of truth is
that it is such for or in
the subjective concept, in knowledge.
Truth is the relation of the
judgment of the concept,
the concept that proved to be
the formal judgment of truth;
for the predicate in this judgment is
not only the objectivity of the concept,
but the comparison connecting
the concept of the fact
with the actuality of it.
This realization of the concept is theoretical
in so far as the concept still has, as form,
the determination of subjectivity, or in so far
as it has for the subject the determination of
being its own determination.
Because cognition is the idea as purpose
or as subjective idea,
the negation of the world,
presupposed as existing in itself,
is the first negation;
the conclusion in which the objective is posited
in the subjective has at first, therefore,
only the meaning that what exists in itself is
posited only as something subjective,
only in conceptual determination,
and consequently does not exist as
so posited in and for itself.
Thus the conclusion only attains
to a neutral unity, or a synthesis,
that is, to a unity of terms that
are originally separate, only externally conjoined.
Hence, since in this cognition the concept
posits the object as its own,
the idea gives itself at first
only a content of which the foundation is given,
in which only the form of externality has been sublated.
To this extent, this cognition still retains
its finitude in its realized purpose;
in the realized purpose, it has at the same time
not attained its purpose,
and in its truth it has not arrived at the truth.
For in so far as in the result
the content still has
the determination of a given,
the presupposed being-in-itself confronting
the concept is not sublated;
the unity of concept and reality, the truth,
is thereby equally not contained in it.
Remarkable is that this side of finitude is
the one that of late has been clung to
and accepted as the absolute relation of cognition
as if the finite as such were to be the absolute!
On this view, an unknown thinghood-in-itself is
attributed to the object, behind cognition,
and this thinghood, and the truth also along with it,
are regarded for cognition as an absolute beyond.
Thought determinations in general, the categories,
the determinations of reflection,
as well as the formal concept and its moments,
acquire on this view the status of
determinations that are finite,
not in and for themselves,
but in the sense of being
something subjective as against
this empty thinghood-in-itself;
the fallacy of taking this
untrue relation of cognition as
the true relation has become
the universal opinion of modern times.

It is immediately clear from this definition
of finite cognition that it is
a contradiction that sublates itself;
it is the contradiction of a truth that
is supposed at the same time not to be truth,
of a cognition of what is that at the same time
does not know the thing-in-itself.
In the collapse of this contradiction, its content,
subjective cognition and the thing-in-itself,
collapses, that is, proves itself to be an untruth.
But it is incumbent upon cognition itself to resolve
its finitude by its own forward movement
and along with it its contradiction.
What we have said is a consideration which
we bring to it and remains a reflection external to it.
But cognition is itself the concept
which is a purpose unto itself
and, therefore, through its
realization fulfills itself,
and precisely in this fulfillment
sublates its subjectivity
and the presupposed being-in-itself.
We must examine cognition, therefore,
in its positive activity within it.
Because this idea, as we have shown,
is the concept's impulse
to realize itself for itself,
its activity consists in determining the object,
and by virtue of this determining
to refer itself to itself in it as identical.
The object is simply the determinable as such,
and in the idea it has this essential side of
not being in and for itself opposed to the concept.
Because this cognition is still finite,
not speculative cognition,
the presupposed objectivity
does not as yet have for it
the shape of something
which is inherently the concept
simply and solely
and does not hold anything particular
for itself as against the cognition.
But by thus having the status of
a beyond that exists in itself,
the determination of being determinable
through the concept is essential to it;
for the idea is the concept
that exists for itself,
is that which is absolutely infinite in itself,
in which the object is implicitly sublated,
and the aim is still to sublate it explicitly.
The object, therefore, is indeed presupposed
by the idea of cognition as existing in itself,
but as so essentially related to the idea
that the latter, certain of itself
and of the nothingness of this opposition,
arrives in the object at the
realization of its concept.

In the syllogism whereby
the subjective idea now rejoins objectivity,
the first premise is the same form of
immediate seizure and connection of the concept
with respect to the object as
we see in the purposive connection.
The determining activity of the concept
upon the object is an immediate communication of
itself to the object, an unresisted invasion of it.
In all this the concept remains in pure self-identity;
but this immediate immanent reflection equally has
the determination of objective immediacy;
that which for the concept is its own determination,
is equally a being, for it is
the first negation of the presupposition.
The posited determination equally has the status,
therefore, of a presupposition which is merely found,
the apprehension of a given wherein the activity of
the concept consists rather in being negative towards itself,
in holding itself back away from what is found
and passive towards it,
in order that the latter be allowed to show itself,
not as determined by the subject,
but as it is in itself.

In this premise, therefore,
this cognition does not in any way appear as
an application of logical determinations,
but as a reception and apprehension
of such determinations as already found,
and its activity appears restricted
simply to the removing from
the subject matter of a subjective obstacle,
an external veil.
This cognition is analytic cognition.

a. Analytic cognition

The difference between analytic and synthetic cognition is
sometimes said to be that the one proceeds
from the known to the unknown
and the other from the unknown to the known.
On closer examination, however, it is difficult to
find any definite thought behind this difference,
even less a concept.
It may be said that in general
cognition begins with ignorance,
for one does not learn to know something
with which one is already acquainted.
Conversely, it also begins with the known,
for it is a tautological proposition that
that with which cognition begins,
what it therefore actually knows,
is for that reason a known;
what is as yet not known,
and is expected to be known only later,
is still an unknown.
In this respect it must be said that cognition,
once it has begun, always proceeds
from the known to the unknown.

The specific difference of analytic cognition is
already established by the fact that,
since it is the first premise of the whole syllogism,
mediation does not as yet belong to it;
analytic cognition is rather
the immediate communication of the concept,
a communication that does not as yet contain otherness
and in which activity divests itself of its negativity.
Yet this immediacy of the connection is
for that reason itself mediation,
for it is a negative reference of the concept to
the object that annuls itself
and thereby makes itself simple and identical.
This immanent reflection is only subjective,
because in its mediation the difference is
present still in the form of a presupposition
existing in itself,
as the object's difference within itself.
The determination that results
through this connection, therefore,
is the form of simple identity,
of abstract universality.
Accordingly, analytic cognition has in general
this identity for its principle,
and the transition into an other,
the linking of different terms is
excluded from it and from its activity.

If we look now more closely at analytic cognition,
we see that it starts from a presupposition,
hence from some singular, concrete subject matter,
whether for representation this subject matter is
already completed or in the form of a task,
that is, given to it only under
certain circumstances and conditions
rather than disengaged from these
on its own and presented in simple independence.
Now the analysis of this subject matter
cannot consist just in resolving it
into the particular representations
possibly contained within it;
such a resolution and the apprehension of
the particular representations is
an affair that would not belong to cognition,
but would rather be a matter of closer acquaintance,
a determination within the sphere of representing.
Analysis, since it is based on the concept,
has for its products determinations
that are essentially conceptual,
though such as are contained in the subject matter immediately.
We have seen from the nature of the idea of cognition
that the activity of the subjective concept
must be regarded from one side only as
the explication of what is already in the object,
for the object itself is nothing
but the totality of the concept.
It is just as one-sided to portray analysis
as though there were nothing in the subject matter
that is not imported into it,
as it is to suppose that
the resulting determinations are
only extracted from it.
The former way of stating the case corresponds,
as is well known, to subjective idealism,
which takes the activity of cognition in analysis
to be only a one-sided positing,
beyond which the thing-in-itself remains hidden.
The other way belongs to the so-called realism,
for which the subjective concept is an
empty identity that imports
the thought determinations from outside.
Since analytical cognition,
the transformation of the given material
into logical determinations,
has shown itself to be a positing
that immediately determines itself to
be equally a presupposing,
to be both in one,
the logical element can appear
on account of this presupposing to be in
the subject matter as something already completed,
just as because of the positing it can
appear as the product of a merely subjective activity.
But the two moments are not to be separated.
In the abstract form to which analysis raises it,
the logical element is of course
only to be found in cognition,
just as conversely it is not only something posited
but something that rather exists in itself.

Now in so far as analytical cognition is
the indicated transformation,
it does not go through further middle terms;
on the contrary, the determination is
to that extent immediate
and has precisely the meaning of
being the subject matter's own determination,
of belonging to it in itself,
and therefore of being apprehended directly
from it without subjective mediation.
But further, cognition is also
supposed to be a progress,
an explication of differences.
But because, according to
the determination that it has here,
it is void of concept and undialectical,
it only possesses a given difference,
and its progression happens solely
in the determinations of the material.
It seems to have an immanent progress
only in so far as the derived thought
determinations can in turn be analyzed,
in so far as they are still something concrete;
the highest and final term of this analyzing is
the abstract highest essence
or the abstract subjective identity
and, over against it, the difference.
This progress, however, is nothing
but just the repetition of
the one original activity of analysis, namely,
the re-determination as a concretion of
what has already been taken up in abstract conceptual form,
and following upon that the analysis of this concretion,
and then again the renewed determination of
the resulting abstraction as concrete, and so forth.
But the thought determinations also
seem to contain a transition in themselves.
If the subject matter is defined as a whole,
then of course one advances from it
to the other determination of part;
from cause to the other determination
of effect, and so forth.
But this is no progress,
for part and whole, cause and effect,
are relations indeed,
in the context of this formal cognition
they are such consummate relations
that the one determination is
already found essentially linked to the other.
The subject matter that has been determined
whether as cause or as part is
thus determined by the whole relation,
already by both sides of it.
Although this relation is in itself something synthetic,
this connection is for analytical cognition
just as much of a mere given as is
any other connection in its material,
and therefore outside its sphere of competence.
Whether this connection is otherwise determined as
a priori or a posteriori is here indifferent,
for it is apprehended as already given,
or, as it has also been called,
as a fact of consciousness
namely the fact that with the determination of whole there
is linked the determination of part, and so on.
Kant made the profound observation that
there are synthetic principles a priori,
and he recognized as their root
the unity of self-consciousness,
hence the self-identity of the concept.
However, he takes the specific connection,
the relational concepts,
and the synthetic principles,
from formal logic as given;
the deduction of these should have been the exposition of
the transition of that simple unity of self-consciousness
into these determinations and distinctions;
but Kant spared himself the effort of demonstrating
this truly synthetic progression,
that of the self-producing concept.

It is well known that “analytical science” and “analysis”
are the names of preference of arithmetic
and the sciences of discrete magnitude in general.
And in fact their typical method of cognition is
most immanently analytical
and we must now briefly consider why this is so.
Any other analytical cognition
begins from a concrete material
that has an accidental manifoldness within;
every distinction of content and every advance to
further content depend on this material.
The material of arithmetic and of algebra is, on the contrary,
an already totally abstract and indeterminate product
from which every peculiarity of relation has been eliminated,
and to which, therefore, every determination
and every joining is something external.
This product is the principle of discrete magnitude, the one.
This relationless atom can be increased to
a plurality and externally determined
and unified into a sum;
the increasing and the limiting are
an empty progression and an empty determining
that never gets past the same principle
of the abstract one.
How the numbers are further combined and separated
depends solely on the positing activity of the knowing subject.
Magnitude is in general the category
within which these determinations are conducted;
it is the determinateness that has become indifferent,
so that the subject matter has no
determinateness which is immanent to it
and is therefore given to cognition.
Since cognition has from the start provided itself with
an accidental assortment of numbers,
these now constitute the material
for further elaboration and manifold relations.
Such relations, their discovery and elaboration,
do not seem, it is true, to be anything immanent
in analytical cognition,
but seem rather something accidental and given;
moreover, these same relations
and the operations connected with them
are also routinely conducted one after the other,
as diverse, with no notice of any internal connectedness.
But it is easy to recognize
the presence of a guiding principle;
it is none other than the immanent
principle of analytical identity,
an identity that in diversity appears as equality;
progression is the reduction of
the unequal to ever greater equality.
To give an example from the first elementary operations,
addition is the combining of quite accidentally unequal numbers;
multiplication, on the contrary, is the combination of equal numbers,
upon which there then follows the relation of
equality of number of times and unit,
and the relation of powers then comes in.

Now because the determinateness of the subject matter
and of the relations is a posited one,
any further operation with them is also wholly analytic;
accordingly, analytical science has not so much theorems
as it has problems.
The analytical theorem contains the problem
as already resolved on its own terms;
the wholly external distinction that attaches
to the two sides which it equates is so unessential
that, as theorem, it would appear to be a trivial identity.
To be sure, Kant has declared the proposition
5 + 7 = 12 to be synthetic,
because the same is exhibited on the one side in the form
of a plurality, 5 + 7,
and on the other in the form of a unity, 12.
But, if the analytic proposition is
to mean more than just the totally abstract
identity and tautology of 12 = 12
and is to contain any progression within it at all,
then there must be present some sort of distinction,
though not one based on a quality,
on a reflective and still less conceptual determinateness.
5 + 7 and 12 are absolutely the very same content;
but the first side also expresses the demand
that 5 and 7 be combined into one expression,
that is to say, that just as 5 is the product
of a process of counting that was arbitrarily interrupted
but might just as well have been carried farther,
so now the counting is to be resumed as
before with the stipulation that the
ones to be added should be seven.
The 12 is therefore the result of 5 and 7
and of a pre-set operation
which is by nature also
a completely external and thoughtless act,
one that a machine can also therefore perform.
Here there is not the slightest transition to an other;
what there is, is the mere continuation,
that is, the repetition, of the same operation
that produced 5 and 7.

The proof of a theorem of this kind
and it would require a proof if
it were a synthetic proposition
would consist simply in the operation of
continuing counting, starting from 5,
up to 7 as the pre-determined limit,
and in the recognition of the
agreement of the product of this counting
with what is otherwise called 12,
a figure which is again nothing more
than that same counting up to a determined limit.
For this reason we state the proof
in the form of a problem rather than a theorem
as a matter of course.
We demand to perform an operation,
that is to say,
we state only one side of the equation
that would constitute the theorem
and whose other side is now to be found.
The problem contains the content and
assigns the specific operation to be performed with it.
The operation is not constrained by
any recalcitrant material endowed with specifying relations,
but is rather an external subjective act,
the determinations of which are received
with indifference by the material where they are posited.
The whole difference between
the conditions stipulated in the problem
and the result in the solution is only this,
that the union or separation as stipulated
in the problem is actual in the solution.

It is, therefore, a supremely superfluous piece of scaffolding
to apply here the form of geometrical method
that goes with synthetic propositions,
and to add to the problem,
over and above the solution,
a proof as well.
Such a proof can express no more than
the tautology that the solution is correct
because the prescribed operation has been performed.
If the problem is to add several numbers,
then the solution is to add them;
the proof shows that the solution is correct
because addition was prescribed
and addition was performed.
If the problem involves more complex expressions and operations,
as for instance the multiplication of decimal numbers,
and the solution only states the mechanical procedure,
a proof will then indeed be necessary;
but it can consist in nothing more than
the analysis of the expressions
and of the operation from which
the solution proceeds of itself.
By this separation of the solution
as a mechanical procedure,
and of the proof as a reminder of
the nature of the subject matter to be treated
and of the operation itself,
we lose precisely the advantage of the analytic problem,
namely that the construction can be derived directly from the
problem and presented, therefore,
as intelligible in and for itself;
in the other way,
the construction is expressly given a defect
which is typical of the synthetic method.
In higher analysis, especially in connection with
the relations of powers,
where qualitative relations of discrete magnitudes
dependent on conceptual determinacies come into play,
the problems and
the theorems do of course contain synthetic determinations;
in these cases,
other expressions and relations than are
given by the problem or theorem
must be taken as intermediary links.
But here also, the determinations
enlisted as an aid must be such as
to be based on recalling or developing
one side or other of the problem or theorem;
the look of synthesis comes
solely from the fact that the problem
or theorem has not as yet already
identified that side.
For instance, the problem of finding
the sum of the powers of the roots of an equation is
solved through the examination and then the joining of
the functions that are the coefficients of
the equation of the roots.
The determination of the functions of these coefficients
and their link here enlisted as an aid is not
already expressed in the problem,
but for the rest the development is totally analytical.
The same applies to the solution of the equation
x m 1 = 0 with the aid of the sine,
and also to its immanent algebraic solution,
famously discovered by Gauß,
which takes into consideration as an aid
the residuum of x m 1
1 divided by 1,
and the so-called primitive roots
one of the most important extensions of analysis in modern times.
These solutions are synthetic,
for the determinations enlisted in their aid,
the sine or the examination of the residua,
are not a determination of the problem itself.

We gave in the first part of this Logic
a detailed account of the nature of the analysis
which is dedicated to the so-called
infinite differentiations of variable magnitudes,
the analysis of differential and integral calculus.
It was shown there that underlying this analysis
there is a fundamental qualitative determination of magnitude
that can be comprehended only by the concept.
The transition to this determination from magnitude
as such is no longer analytic;
to this day, therefore, mathematics has been
incapable of justifying internally,
that is, mathematically, the operations based on it,
for the transition is not of a mathematical nature.
We said in the same place that Leibniz,
famed for having rendered the calculation of
infinitesimals into a calculus,
executed that transition in a way
which is utterly deficient,
just as totally void of concept as unmathematical.
But of course, once the transition is presupposed
and in the present state of the science
it is no more than a presupposition
the further course is only a series of
ordinary analytical operations.

We have said that analysis becomes synthetic
when it comes to determinations that are
no longer posited by the problems themselves.
But the general transition from
analytic to synthetic cognition
lies in the necessary transition from
the form of immediacy to mediation,
from abstract identity to difference.
Analysis in general restricts its activity
to determinations in so far as
these are self-referential;
yet by virtue of their determinateness
they also essentially refer to an other by nature.
We have already remarked that
analytic cognition remains such even
when it advances to relations that
are not an externally given material
but are rather thought determinations,
since for it these relations are also given.
But because the abstract identity
which this cognition knows to be solely its own is
essentially an identity in difference,
even as such it must be cognition's own identity,
and the connection as well must become for the concept
one which is posited by it
and is identical with it.

b. Synthetic cognition

Analytic cognition is the first premise of the whole syllogism
the immediate reference of the concept to the object.
Identity is, therefore, the determination
which analytic cognition recognizes as its own,
and analytic cognition is the apprehension of what is.
Synthetic cognition aims at the comprehension of what is,
that is, at grasping the manifoldness of
determinations in their unity.
It is, therefore, the second premise of the syllogism,
the one in which the diverse as such is connected.
Its aim, therefore, is necessity in general.
The diverse terms that are combined stand,
on the one hand, in a relation in which
they are both connected yet mutually
indifferent and self-subsistent;
but, on the other hand, they are linked together
in the concept which is their simple yet determinate unity.
Now inasmuch as in a first moment
synthetic cognition passes over
from abstract identity to relation,
or from being to reflection,
it is not the absolute reflection
of the concept that the latter
recognizes in its subject matter;
the reality that the concept
gives itself is the next stage,
namely the said identity in diversity as such,
an identity that equally is, therefore,
still inner, and only necessity;
it is not the subjective identity existing for itself,
hence not as yet the concept as such.
Synthetic cognition, therefore, does also have
for its content the determinations of the concept,
the object is posited in them;
but they stand only in relation to one another
or in immediate unity,
and for that very reason not in the unity
by which the concept exists as subject.

This is what constitutes the finitude of this cognition.
Because the identity which this real side of
the idea has in it is still an inner one,
the determinations of that identity are
still external to themselves;
and because the identity is not as subjectivity,
the concept's own specific presence in
the subject matter still lacks singularity;
although what in the object corresponds to
the concept is no longer the abstract
but the determinate form of the concept
and hence the concept's particularity;
the singularizing element in the object is
nevertheless still a given content.
Consequently, although this cognition
transforms the objective world into concepts,
what it gives to it in accordance
with conceptual determinations is only the form;
as for the object in its singularity,
in its determinate determinateness,
this it must find;
the cognition is not yet self-determining.
It likewise finds propositions and laws,
and proves their necessity;
but it proves the latter not as a necessity
inherent in a fact in and for itself,
that is to say, it does not
demonstrate it from the concept;
it proves it rather as the necessity
inherent to a cognition
that delves into given determinations,
into phenomenal differences,
and cognizes for itself the proposition as
a unity and relation,
or cognizes the ground of appearance
from the appearance itself.

We must now examine the detailed
moments of synthetic cognition.

1. Definition

To start with, the still given objectivity is
transformed into simple form, as the first form
and therefore as the form of the concept;
the moments of this apprehension are none other, therefore,
than the moments of the concept;
universality, particularity, and singularity.
The singular is the object itself as
an immediate representation;
it is that which is to be defined.
The universal of this singular object
took the form of genus in
the determination of the objective judgment,
or the judgment of necessity;
more precisely, it took the form of the proximate genus,
that is to say, of the universal with the determinateness
which is at the same time the principle for
the differentiation of the particular.
This is a difference that the subject matter
receives in its specific non-indifference,
the one that makes it a determinate species
and is the basis of its disjunction
from the remaining species.

Definition, in thus reducing
the subject matter to its concept,
gets rid of the externalities
that are requisite for its concrete existence;
it abstracts from what is added
to the concept in its realization,
whereby the concept issues first into idea
and secondly into external concrete existence.
Description is for representation;
it collects this extra content that belongs to reality.
But definition reduces this
wealth of manifold determinations of
the intuited existence to its simplest moments;
what is contained in the concepts are
the form of these simple elements
and how they are determined
with respect to one another.
The subject matter is thus apprehended, as we just said,
as a universal which is determined at the same time.
The subject matter is the third,
the singular in which genus and particularization
are posited in one,
an immediate which is posited outside the concept,
for it is not yet self-determining.

In these determinations,
in the difference of form of the definition,
the concept finds itself;
there it finds the reality that corresponds to it.
But since the reflection of the moments of
the concept into themselves
which is singularity is not as yet contained in this reality,
and since the object, in so far as it is in cognition,
is consequently not as yet determined as subjective,
it is cognition which, on the contrary, is
subjective and has an external beginning;
that is to say, because of its external
beginning in a singular it is subjective.
The content of the concept is
therefore something given and contingent.
The concrete concept itself is thus contingent in two respects:
once because its content is contingent;
and again because it is a matter of accident
which content determinations, from the many qualities
which the intended object has in external existence,
are chosen for the concept as constituting its moments.

This last respect requires closer consideration.
Since singularity is a determinate
way of existing in and for itself,
it escapes the conceptual determination
proper of synthetic cognition.
There is in fact no principle, therefore,
for determining which aspects of the subject matter are
to be regarded as belonging to its conceptual determination
and which only to its external reality.
In the case of definitions, this constitutes a difficulty
which for synthetic cognition cannot be eliminated.
A distinction must nonetheless be made here.
In the first place, so far as the products of
self-conscious purposiveness are concerned,
it is easy enough to discover their definition,
for the purpose which they should serve is
a determination that is generated by a subjective resolution
and constitutes the essential particularization,
the form of the concrete existent,
on which alone everything depends here.
The further nature of the material of
the existent thing or its other external properties,
in so far as they correspond to the purpose,
are contained in the thing's determination;
the rest are unessential for it.

Secondly, geometrical objects are abstract determinations of space;
the underlying abstraction, the so-called absolute space,
has lost all other concrete determinations
and now possesses no further shapes and configurations
than are posited in it;
essentially, therefore, such shapes and configurations are
only what they are intended to be;
their conceptual determination in general,
and more proximately their specific difference,
have unfettered reality in them;
in this respect, therefore, they are the same as
the products of external purposiveness,
and in this they also agree with
the objects of arithmetic in which
the underlying determination is also
only one that has been posited in them.
Of course, space has yet further determinations:
its tri-dimensionality, its continuity and divisibility which
are not first posited in it by external determination.
But these belong to the material under consideration
and are immediate presuppositions;
synthetic relations and laws are produced only
through the combination and the interweaving of
these subjective determinations with
this distinctive nature of their field
into which they have been imported.
In the case of number determinations,
since they are based on the simple principle of the one,
their combination and further determination is
only an entirely posited product;
on the other hand, determinations in space,
which for its part is a continuous externality,
run a further course of their own
and have a reality that exceeds their concept,
but it no longer belongs to the immediate definition.

But, thirdly, in the case of
the definitions of concrete objects,
of nature as well as of spirit,
the situation is quite different.
For representation, such subject matters are
in general things of many properties.
In their case all depends on apprehending
what is their proximate genus,
and then what is their specific difference.
We have to determine, therefore,
which of the many properties pertains
to the subject matter as genus, which as species,
and which among these properties is the essential one;
this further involves recognizing
how the properties hang together,
whether one is already posited with the other.
For this, however, no other criterion is
yet available than existence itself.
For the definition, in which the property is
to be posited as simple undeveloped determinateness,
the essentiality of the property is its universality.
But in existence this universality is empirical;
it is a universality in time
(whether the property persists while
the rest ostensibly come and go within
the permanence of the whole),
or a universality resulting from comparison
with other concrete wholes,
in which case it does not get beyond commonality.
Now if comparison gives as a common foundation
the total habitus, such as is empirically given,
then reflection must gather it together
into one simple thought determination
and grasp the simple character of the resulting totality.
But the only possible attestation that a thought determination,
or any single one of the immediate properties,
constitutes the simple and determinate
essence of the subject matter is
its derivation from the concrete constitution of the latter.
But this would require an analysis that transforms
the immediate elements of this constitution into thoughts
and reduces their concreteness to a simple thought determination;
and this is an analysis of a higher order than
the one just considered,
for it would not be abstractive;
on the contrary, in the universal it should still retain
the singular character of the concrete,
should unify it and show that it is dependent on
the simple thought determination.

The connections of the manifold determinations of
immediate existence to the simple concept
would however require theorems,
and these need proof.
But definition is the first, still undeveloped concept,
and in that it has to apprehend
the simple determinateness of the subject matter,
and this apprehension is to be something simple,
it can employ for the purpose only one of
the subject's immediate so-called properties,
a determination of sensuous existence or of representation;
the singling out, then, of this property through abstraction is
what constitutes the simplicity,
and for universality and essentiality
the concept must resort to empirical universality,
to persistence under altered circumstances
and to the reflection that seeks
the determination of the concept
in external existence and in pictorial representation,
seeks it, that is, where it is not to be found.
Defining, therefore, by its own doing also forfeits
the true concept determinations that would by essence
be the principles of the subject matter,
and contents itself with marks, that is,
determinations in which that they are essential
to the subject matter is a matter of indifference
and whose only purpose is rather to be
markers for external reflection.
Any such single, external determinateness is
too disproportionate with respect to the concrete totality
and to the nature of its concept to justify its being singled
out or to assume that a concrete whole
would find in it its true expression and determination.
For example, as Blumenbach observes,
the lobe of the ear is something lacking
to all other animals and is therefore perfectly entitled,
in accordance with ordinary ways of
speaking about common and distinguishing markers,
to be used as the distinctive characteristic in the
definition of the physical human being.
But how disproportionate such
a totally external determination
at once appears when measured
against the representation of
the total habitus of the physical human being,
and against the demand that
the concept determination shall be something essential!
It is entirely accidental whether the markers taken up
into the definition are pure makeshifts like this one
or approximate the nature of a principle instead.
From their externality one can also see that
the cognition based on concepts did not begin with them;
it was rather an obscure feeling,
an indeterminate but profound sense,
an intimation of the essential
that preceded the discovery of
genera in nature and spirit,
and only afterwards was a specific
externality sought for the understanding.
Since in existence the concept has entered into externality,
it has unfolded into its differences
and cannot be absolutely attached to
any single one of such properties.
The properties, as the externality of the thing,
are external to themselves;
for this reason, as we demonstrated
in the sphere of appearance
in connection with the thing of many properties,
do the properties essentially become
even self-subsistent matters;
spirit, regarded from the standpoint of appearance,
turns into an aggregate of many independent forces.
Regarded in this way, the single property or force,
even where it is posited as indifferent to
the other properties,
ceases to be a characterizing principle,
with the result that the determinateness,
as the determinateness of the concept,
vanishes completely.

In the concrete things, together with
the diversity of the properties among themselves,
there also enters the difference between
the concept and its realization.
The concept has an external presentation
in nature and spirit
wherein its determinateness manifests itself
as dependence on the external,
as transitoriness and inadequacy.
Therefore, although an actual thing
will indeed manifest in itself
what it ought to be, yet,
in accordance with the negative judgment of the concept,
it may equally also show that its actuality
only imperfectly corresponds with this concept,
that it is bad.
Now the definition is supposed to
indicate the determinateness of
the concept in an immediate property;
yet there is no property against which
an instance could not be adduced
where the whole habitus indeed allows
the recognition of the concrete thing to be defined,
yet the property taken for its character shows itself
to be immature and stunted.
In a bad plant, a bad animal type,
a contemptible human individual, a bad state,
there are aspects of their concrete existence
that are defective or entirely missing
but that might otherwise be picked out
for the definition as the distinctive mark
and essential determinateness
in the existence of any such concrete entity.
A bad plant, a bad animal, etc., remains a plant,
an animal just the same.
If, therefore, the bad specimens are
also to be covered by the definition,
then the empirical search for essential properties is
ultimately frustrated,
because of the instances of malformation
in which they are missing;
for instance, in the case of the physical human being,
the essentiality of the brain is missing
in the instance of acephalous individuals;
or, in the case of the state, the essentiality of
the protection of life and of property is
missing in the instance of despotic states
and tyrannical governments.

If the concept is maintained despite
the contradicting instance
and the latter is declared,
as measured by the concept,
to be a bad specimen,
then the attestation of the concept is
no longer based on appearance.
But that the concept stands on its own
goes against the meaning of definition;
for definition is supposed to be the immediate concept,
and can therefore derive its determinations of
the subject matter only from the immediacy of existence
and justify itself only in what it already finds there.
Whether its content is in and for itself truth or contingency,
this lies outside the sphere of definition;
but for this reason, because the singular
subject matter under consideration
may well be a bad specimen,
formal truth, or the agreement of the concept
subjectively posited in the definition
and the actual subject matter outside it,
cannot be established.

The content of a definition is taken
in general from immediate existence,
and because it is immediate,
it has no justification;
the question regarding its necessity is
precluded by its origination;
by the very fact that the definition voices
the concept as something merely immediate,
it renounces comprehending it conceptually.
What it exhibits, therefore, is nothing but
the form determination of the concept in a given content,
without the reflection of the concept within itself,
that is, without its being-for-itself.

But immediacy proceeds as such only from mediation,
and must therefore pass over into it.
Or the determinateness of the content
contained in the definition is,
for the very reason that it is determinateness,
not only immediate but something mediated by its other.
Consequently definition can apprehend its subject matter
only by virtue of the opposite determination
and must therefore pass over into division.

2. Division

The universal must particularize itself;
to this extent, the necessity of division lies in the universal.
But because definition itself already begins with the particular,
its necessity for passing over into division lies in
the particular that points, as particular, to an other.
Conversely, the particular separates itself off
from the universal precisely by holding on to its determinateness
for the sake of keeping it distinct from an other than it;
the universal is therefore presupposed for division.
The way to proceed is therefore this:
the singular content of definition is
raised through particularity to the extreme of universality;
but universality must from now on
be assumed as the objective foundation
and, with it as the starting point,
division presents itself as
the disjunction of the universal,
the latter being the first.

A transition is now introduced which,
since it takes place from the universal to the particular,
is determined by the form of the concept.
Definition is as such something singular;
a greater number of definitions pertains to
a greater number of subject matters.
The advance from the universal to the particular
characteristic of the concept constitutes
the basis and the possibility of a synthetic science,
of a system, and of systematic cognition.

The first requirement for this is that, as indicated,
the beginning be made with the subject matter
in the form of a universal.
In the realm of actuality,
whether of nature or spirit,
it is the concrete singularity that is given to
subjective, natural cognition as first.
But in a cognition which is a conceptual comprehension,
at least inasmuch as it has
the form of the concept for its basis,
it is the simple, abstracted from the concrete,
that on the contrary comes first,
for only in this form does the subject matter have
the form of a self-referring universality
and of an immediacy that accords with the concept.
It may perhaps be objected to
this way of proceeding in matters scientific
that, since intuition is easier than cognition,
what can be intuited, that is, concrete actuality,
should be made the starting point of science;
that this way of proceeding would be more natural
than one that starts from an abstract subject matter
and then proceeds from it to its
particularization and concrete singularization.
But inasmuch as cognition is the issue,
any comparison with intuition has
already been decided and dismissed;
the only question allowed here is
what should be the first inside cognition,
and how one should then go from there;
what is required is not a method appropriate to nature
but one appropriate to cognition.
If the issue is merely one of easiness,
then it goes without saying that
it is easier for cognition to grasp
the abstract simple thought determination
than to grasp a concrete subject matter
which is a complex web of such thought
determinations and relations;
and it is in this manner,
no longer as it is given in intuition,
that the matter should be grasped.
The universal is in and for itself
the first moment of the concept,
because it is the simple,
and the particular only comes after it,
because it is the mediated;
and conversely the simple is the more universal,
and the concrete,
since it is internally differentiated
and hence mediated,
is what already presupposes the transition from a first.
This remark applies not only to the ordering
of the whole into the specific forms of
definitions, divisions, and propositions,
but also to the ordering of cognition as a whole
and simply with respect to the difference of
abstract and concrete in general.
Thus in learning to read for example,
the more rational way to begin is also not
with reading whole words or even syllables,
but with the elements of words and syllables
and with the signs of abstract sound inflections;
in alphabetic script the analysis of concrete words
into their abstract sound inflections
and their signs is already accomplished,
and for this reason learning to read
is a primary occupation with abstractions.
In geometry, the beginning has to be made
not with a concrete space configuration
but with point and line
and then with plane figures,
and among the latter not with polygons but
with the triangle, and among curves with the circle.
In physics, the singular natural properties or matters
must be freed from the manifold entanglements
in which they are found in concrete actuality,
and presented with their simple, necessary conditions;
they too, like space configurations,
are accessible to intuition,
but this intuition has to be prepared for,
in such a way that they finally appear
and are held free of all the modifications
due to circumstances which are extraneous
to their specific nature.
Magnetism, electricity, various kinds of gases, etc.,
are such subject matters as come to be known
in their specificity only by being apprehended
as removed from the concrete conditions
under which they appear in actuality.
Of course, an experiment will exhibit them
to intuition in some concrete case;
but, in order for the experiment to be scientific,
it must admit only such conditions as
are necessary to it,
and it must be repeated in various forms
in order to demonstrate that
the concrete cluster of conditions
that cannot be separated from
the matters under consideration are inessential,
for these can appear in one concrete configuration
and then again in another,
so that only their abstract form is left for cognition.
To mention yet one more example,
it might appear natural and reasonable to regard colors,
first, in the concrete appearance of the animal subjective sense;
next, as a spectral phenomenon hovering outside the subject,
and finally fixed in objects in external actuality.
But for cognition, the universal
and consequently truly primary form is
the middle one of the three mentioned,
color as it hovers between subjectivity and objectivity
in the well-known form of the spectrum,
still unentangled with subjective and objective circumstances.
For the pure consideration of the nature of
this object such circumstances are
at first only a source of interference,
because they behave as efficient causes
and therefore make it uncertain
whether specific alterations of colors,
specific transitions and relations,
are based on their own specific nature as color
or are rather to be attributed to
the specific pathology of the circumstances themselves,
to the healthy and sick particular affections
and effects of the organs of the subject,
or to the chemical, vegetable,
and animal forces of the objects.
Numerous other examples could be adduced
from the cognition of organic nature
and of the world of spirit;
everywhere the abstract must constitute
the starting point and the element
in which and from which the
particularities and rich shapes
of the concrete spread out.

Now, although with division or with the particular
the distinction of universal and particular is duly introduced,
this universal is nevertheless itself already something determinate
and therefore itself only a member of a division.
Hence there is a higher universal for it,
and a higher yet for this other universal,
and the same for the next all the way to infinity.
There is no immanent limit to
the cognition under consideration here,
because it proceeds from a given
and it is the form of abstract universality
that defines its “first.”
Any object, therefore, that seems to possess
an elementary universality is
made the subject matter of a specific science:
it makes an absolute beginning
because ordinary acquaintance with it is presupposed,
and the assumption made is
that it stands on its own with no need of derivation.
Definition takes it as an immediate.

Division is the immediately next step after this starting point
For this advance, only an immanent principle would be required,
that is, a beginning from the universal and the concept;
but the cognition under consideration
here lacks any such principle,
for it follows only upon
the form determination of the concept
without that form's immanent reflection,
and therefore takes the determinateness of
the content from what is given.
There is no specific reason for
the particular that enters into the division,
whether with respect to what constitutes
the basis of the division,
or with respect to the specific relation
that the members of the disjunction
are supposed to have to one another.
Consequently, in this respect the business of cognition
can only consist partly in orderly arranging
the particularities found in the empirical material,
and partly also in discovering universal determinations of
this material by means of comparison.
Such determinations then count as grounds of division,
of which there can be a variety,
just as there can be an equal variety of divisions based on them.
The relation of the members of a division to one another,
the relation of the species,
has only this one universal determination,
namely that the members, the species,
are determined relative to one another in accordance
with the assumed ground of division;
if their differentiation were
to rest on some other consideration,
their order would be arranged
along different lines accordingly.

Because of the lack of a principle of self-determination,
the only possible laws for this business of division
consist in formal, empty rules that lead nowhere.
Thus we see laid down as a rule that
division should exhaust the concept;
but in fact each single member of the division
must exhaust the concept.
Actually, what is meant is that
the determinateness of the concept
should be exhausted;
but there is nothing in an empirical manifold of species,
internally void of determination,
that contributes to this exhaustion
whether few or many of them have been discovered;
it is indifferent, for example,
to the exhaustion of the concept
whether in addition to
the sixty-seven species of parrots
another dozen are discovered.
The demand for exhaustion can only
mean this tautological proposition,
that all the species should be
listed in their completeness.
Now as empirical cognition expands,
it can well happen that species are
discovered that do not fit the
assumed determination of the genus,
for the genus is usually adopted more
in accordance with some obscure
representation of the whole habitus
than in accordance with the more or less
singular mark that should expressly
serve to determine it.
In such a case, the genus would have to be altered
and a justification would have to be given
for regarding another group of species as
the species of the one new genus;
that is to say, the genus would receive
its determination from what we group together
on the basis of some standpoint or other
that we choose to assume as a principle of unity;
this standpoint thus becomes itself the ground of division.
Conversely, should we hold on to the determinateness
originally assumed to define the genus,
then the material that we wanted to bring as
species in unity with the earlier ones,
would have to be excluded.
This way of carrying on without the concept
at one time by assuming a certain determinateness
as the essential moment of the genus,
subordinating the particulars to it
or excluding them from it accordingly;
at another time by starting with the particulars
and letting oneself be guided in grouping them
by some other determinateness
this way gives the appearance of a game
in which it is left up to chance to
decide which part or which side of
the concrete to fix on in order then to
bring order in accordance with it.
It is physical nature itself that presents
such a contingency in the principles of division;
because of the external dependency of its actuality,
it stands in a manifold of connectedness
which for it is likewise given;
there is, therefore, an assortment of principles to
which it has to adapt itself,
following one principle in one series of its
forms but another in another series,
while also producing hybrids that go
in different directions at once.
Thus it happens that in one series of natural
things certain marks come to the fore
as especially significant and essential
that in another series become inconspicuous and purposeless,
the result being that it is impossible to abide
by any such principle of division.
The general determinateness of empirical species
can only consist in this:
that they are simply diverse from one another
without being opposed.
The disjunction of the concept was
presented earlier in a determinate form;
if particularity is taken without
the negative unity of the concept,
as a particularity which is immediate and given,
then difference stays at only the reflective form of
diversity considered earlier.
The externality in which the concept is
pre-eminently to be found in nature
brings with it the total indifference of difference;
for this reason it is common to take from number
the determination for division.

Such is the contingency here of the particular
with respect to the universal,
and therefore of division in general,
that it may be attributed to an instinct of reason
when we discover in this cognition
bases of division and divisions which,
to the extent that sensuous properties allow it,
show themselves to be more adequate to the concept.
For example, in the case of animals,
the instruments for eating,
the teeth and the claws,
are used in the systems as
a far-reaching criterion of division.
They may be taken at first only as
features in which it is easier,
for the subjective purpose of cognition,
to detect distinguishing marks.
But in fact the differentiation
embodied in those organs is not one
that pertains just to external reflection;
such organs are rather
the vital point of animal individuality,
where the latter posits itself as
self-referring singularity
by cutting itself loose from
the otherness of its external nature
and from continuity with the other.
In the case of the plant,
it is the reproductive parts
that constitute the highest point of vegetable life,
the point at which the plant points to
the transition into sexual difference
and thereby into individual singularity.
For this reason the system has rightly turned to
this point for a base of division
which, though not sufficient,
is nonetheless far-reaching,
and has thereby laid down
for foundation a determinateness
which is such not merely
for external reflection,
for the purpose of comparison,
but is in and for itself
the highest of which the plant is capable.

3. The theorem

1. The third stage in this advance of cognition
based on concept determinations is
the transition of particularity to singularity;
this stage constitutes the content of the theorem.
To be considered here, therefore,
is the self-referring determinateness,
the internal differentiation of the subject matter
and the connection of the differentiated determinacies to one another.
Definition contains only one determinateness,
division contains determinateness as against the other;
in singularization the subject matter has parted internally.
Whereas definition stops at the universal concept,
in theorems the subject matter is known in its reality,
in the conditions and the forms of its real existence.
Together with the definition, therefore,
the subject matter exhibits the idea,
which is the unity of the concept and reality.
But the cognition being considered here,
a cognition that is still a seeking,
does not attain this presentation,
for in it reality does not proceed from the concept,
and therefore the dependency of reality on the concept
and consequently the unity itself is not cognized.

Now according to the definition just given,
the theorem is the properly
synthetic element of a subject matter,
because the relations of its determinacies are necessary,
that is, are grounded in the inner identity of the concept.
In definition and division the synthetic element is
a connectedness held together externally;
what is found given is brought into the form of the concept,
but, as given, the entire content is only displayed;
in the theorem, on the contrary,
it ought to be demonstratively displayed.
Since this cognition does not deduce
the content of its definitions
and of the principles of division,
it seems that it might also spare itself
the proof of the relations expressed by the theorem
and be satisfied here too just with perception.
But what distinguishes cognition
from mere perception and representation is
the form of the concept in general
that it imparts to the content;
this is done in definition and division;
but since the content of the theorem proceeds
from the concept's moment of singularity,
it consists in determinations of reality
that no longer have as their relation
just the simple and immediate
determinations of the concept;
in singularity the concept has gone over
to otherness, to reality,
thereby becoming idea.
The synthesis contained in the
theorem no longer has, therefore,
the form of the concept to justify it;
it is a joining together of such as are diverse;
consequently, the unity not yet thereby
posited still remains to be demonstrated;
here proof thus becomes necessary
to this cognition itself.

Now the first difficulty that we encounter here is
of distinctly distinguishing between
which of the determinations of the subject matter
can be admitted into the definitions,
and which must rather be relegated to the theorems.
In this matter, there is no principle readily available.
There might seem to be one in assuming, perhaps,
that what pertains to a subject matter
immediately also belongs to the definition,
while for the rest, since it is something mediated,
the mediation must first be demonstrated.
But the content of the definition is as such determined
and therefore itself essentially mediated;
its immediacy is only a subjective one, that is to say,
the subject makes an arbitrary beginning
letting a subject matter count as a presupposition.
Now since this subject matter is as such in itself concrete,
and must also be divided,
the result is a number of determinations
that are mediated by nature
and are taken as immediate and unproved,
not on the basis of any principle,
but only subjectively.
Even in Euclid, who has always been justly recognized
to be the master of this synthetic kind of cognition,
we find under the name of axiom a presupposition
about parallel lines that some have held to be in need of proof,
and various attempts have been made to fill this lack.
And there are several other theorems in which
people have thought to have discovered presuppositions
that should not have been immediately assumed
but ought to have been proved.
So far as the axiom of parallel lines is concerned,
it may be noted that precisely there is
where Euclid gives evidence of his good sense;
he had duly appreciated the nature of his science.
The proof of that axiom would have to be
derived from the concept of parallel lines;
any such proof, however, has no place in his science,
no more than does the deduction of his definitions,
of his axioms and his subject matter in general,
of space itself and its first determinations, the dimensions.
For any such deduction would have to be
carried out on the basis of the concept,
and this lies outside the proper domain
of Euclidean science;
these must remain for it, therefore,
necessary presuppositions, relative firsts.

Axioms, to take this opportunity of mentioning them,
belong to the same class.
They are commonly but wrongly taken to be absolute firsts,
as if they were not in need of proof
in and for themselves.
If such were in fact the case,
they would then be mere tautologies,
for it is only in abstract identity
that there is no diversity
and that therefore also no mediation is needed.
But if the axioms are more than just tautologies,
then they are propositions drawn from another science,
since within the science for which they serve
as axioms they are meant as presuppositions.
Strictly speaking, therefore, they are theorems,
and are indeed mostly drawn from logic.
The axioms of geometry are lemmas of this kind,
logical propositions,
and they come close, moreover, to being tautologies
because they are concerned with quantity alone
and every qualitative difference has
therefore been purged from it.
Of the principal axiom, the purely quantitative syllogism,
we spoke earlier.
Axioms, therefore, when considered in and for themselves,
are just as much in need of proof
as are definitions and divisions,
and they are not made into theorems
only for the reason that,
since they are relative firsts,
they are assumed for a certain standpoint
as presuppositions.

As regards the content of theorems,
there is one further precision to be made.
Because this content consists in a connection of
determinacies of the concept's reality,
such connections may be more or less incomplete
and single relations of the subject matter, or,
on the contrary, they may be one such relation
that encompasses the whole content of reality
and expresses the content's determinate connection.
But the unity of all the content determinacies is
equivalent to the concept;
a proposition that contains them is
therefore itself a definition again not one, however,
that expresses the concept only as immediately assumed,
but one that expresses it rather
as developed into its determinate, real differences,
or one that expresses the concept's complete existence.
The two together, therefore, present the idea.

If we closely compare the theorems of a synthetic science,
of geometry in particular, we find this distinction,
namely that some of the science's theorems
contain only singular relations of the subject matter,
whereas others contain relations in which
its full determinateness is expressed.
It is a very superficial view
that attributes equal value to all propositions,
on the ground that each generally contains a truth,
and in the formal progression from step to step of
a proof each is equally essential.
The difference with respect to the content of theorems is
most intimately connected with this progression itself,
and some further remarks concerning the latter will
serve to clarify that difference further,
as well as the nature of synthetic cognition.
To start with, Euclidean geometry
(which, as the representative of the synthetic method
of which it delivers the most accomplished exemplar,
shall serve as an example)
has from ancient times been renowned
for the order in the progression of its theorems,
whereby for each theorem the propositions required
for its construction and proof are
always found already proved.
This circumstance concerns the formal sequence of inference;
yet, important as it is, it still has primarily to do
with the external ordering of purposiveness,
and bears on its own no connection
to the essential difference of the concept
and the idea in which there lies
a higher principle of
the necessity of the progression.
That is to say, the definitions
with which the beginning is made
apprehend the sensuous subject matter
as immediately given,
and they determine it according to
its proximate genus and specific difference;
these are equally the simple and immediate
determinacies of the concept,
the universality and particularity
whose relation is developed no further.
Now the initial theorems themselves have
nothing at their disposal except
such immediate determinations
as are found in the definitions;
similarly their reciprocal dependence can only be
a matter at first of each being in general
determined through the other.
Thus Euclid's first propositions regarding
the triangle have to do only with congruence,
that is, how many parts must be determined
in a triangle in order that
the remaining parts of the one and same triangle,
or the whole of it, be determined in full.
That two triangles are compared with one another
and their congruence posited in
the fact that they coincide is
a detour needed by a method that must rely on
sensuous coincidence instead of the thought of determinateness.
Otherwise considered for themselves,
these theorems themselves contain two parts,
one of which can be regarded as the concept,
and the other as the reality
that completes the concept by realizing it.
For the understanding, whatever suffices for a full determination,
that is to say, the two sides and the enclosed angle in this case,
is already the entire triangle;
nothing further is needed for its complete determinateness;
the remaining two angles and the third side are
the superfluity of reality over and above
the determinateness of the concept.
Thus all that those theorems in fact do is
to reduce the sensuous triangle,
which of course requires
three sides and three angles,
to its simplest conditions.
The definition had mentioned,
quite in general,
only the three lines that enclose the plane figure
and make it a triangle;
it is a theorem that first expresses
the determinateness of the angles
through the determinateness of the sides,
just as the remaining theorems express
the dependence of three other parts on three others.

But the complete determinateness of
the magnitude of a triangle in terms of its sides
is contained in the Pythagorean theorem;
it is in this theorem that we first have
the equation of the sides of the triangle,
for the preceding sides bring the triangle
to a reciprocal determinateness of
part to part only in general, not to an equation.
This proposition is therefore the perfect,
real definition of the triangle
of the right-angled triangle in the first place,
the simplest in its differences and hence the most regular.
Euclid brings the first book to a close with this proposition,
for it does in fact attain a perfect determinateness.
And after he has reduced to a uniform type
those triangles which are not right-angles
and are affected by greater inequality,
he concludes the second book with
the reduction of the rectangle to the square,
with an equation between the self-equal, or the square,
and the internally unequal, or the rectangle;
similarly, in the Pythagorean theorem,
the hypotenuse that corresponds to the right-angle, the self-equal,
constitutes one side of the equation,
while the other side is made up by the self-unequal,
the two perpendicular sides.
The equation between the square and the rectangle is
at the basis of the second definition of the circle,
and this is again the Pythagorean theorem,
except that the two perpendicular sides of
the right-angle are assumed to be alterable;
the first equation of the circle is in precisely
the relationship of sensuous determinateness to
equation as holds between the two different
definitions of conic sections in general.

This truly synthetic progression is a transition
from universal to singularity,
namely to that which is determined in and for itself,
or to the unity of the subject matter in itself
inasmuch as this has come apart,
differentiated into its essential real determinacies.
In other science, however,
the common and quite imperfect way of
advancing from universality to singularity is
indeed to start from a universal,
but then to singularize and concretize it
by applying it to a material brought in from elsewhere;
in this way, the singularity of the idea is
strictly speaking an empirical addition.

Now whatever the content of the theorem,
whether imperfect or perfect,
it must be proved.
It is a relation of real determinations
that do not have the relation of concept determinations;
when they do have this relation,
as it can be shown to be the case for the propositions
we have called second or real definitions,
such definitions are for that very reason
in one respect definitions;
but since their content consists at the same time
of relations of real determinations,
not just of the relation of
universal and simple determinateness,
in comparison with such a first definition they
are also in need and capable of proof.
As real determinacies, they have the form of
indifferent subsistence and indifferent diversity;
hence they are not immediately one
and therefore their mediation is to be demonstrated.
The immediate unity in the first definition is
the one in accordance with which
the particular is in the universal.

2. Now the mediation which we must now consider more closely
may be simple or may go through several mediations.
The mediating members are joined together
with those to be mediated;
but since it is not on the basis of the concept,
to which the transition into an opposite is altogether alien,
that the mediation and the theorem are retraced in this cognition,
in the absence of any concept of connectedness,
the mediating determinations must be imported
from somewhere as a provisory material
for the scaffolding of the proof.
This preparation is the construction.

Now among the connections of the content of the theorem,
of which there can be a great number,
only those must be adduced and made to work
that are of service to the proof.
The supply of material only acquires meaning in this context;
in itself it appears blind and meaningless.
In retrospect it will of course become apparent
in the proof that there was a purpose
to drawing, for example, such or such
additional lines to a geometrical figure
as the construction specifies;
in the course of the construction itself, however,
this must be done blindly;
by itself, therefore, this operation is without understanding,
since the purpose motivating it is yet to be declared.

It is a matter of indifference
whether the operation is undertaken
for the purpose of a theorem
in the strict sense or a problem;
as it first appears before the proof,
the operation is not anything derived
from the given specification of the theorem
or the problem a meaningless act, therefore,
for anyone as yet not acquainted with its purpose,
and then always only directed by an external purpose.

This hidden purpose becomes apparent in the proof.
This contains, as stated, the mediation of what
the theorem declares as bound together,
and it is only by virtue of this mediation
that this connectedness first appears as necessary.
Just as the construction lacks on its own
the subjectivity of the concept,
so is the proof a subjective act lacking in objectivity.
For since the content determinations of the theorem are
not posited at the same time as determinations of the concept,
but are posited instead as indifferent parts standing in
a multitude of external relations to one another,
it is only in the formal, external concept
that the necessity manifests itself.
The proof is not a genesis of the relation
that constitutes the content of the theorem;
the necessity is present only to insight,
and the whole proof is only for
the subjective interest of cognition.
It is for this reason
a thoroughly external reflection
that proceeds from the outside to the inside,
that is, arrives in conclusion at
the inner constitution of the relation
on the basis of external circumstances.
These circumstances, which the construction has presented,
are a consequence of the nature of the subject matter;
here they are converted instead into the ground
and the mediating relations.
The middle term, the third term in which
the terms linked in the theorem
present themselves in their unity
and that provides the nerve of the proof,
is therefore only something in which
the connectedness appears and is external.
Because the sequence which the proof goes through is
rather the reverse of the nature of the fact,
what is considered in the proof as ground is a subjective ground,
one that brings out the nature of the fact only for cognition.

The foregoing considerations make clear
the necessary limit of this cognition,
a limit that usually goes unrecognized.
The science of geometry is the
most illustrious example of the synthetic method
but it has been inappropriately applied
to other sciences as well, even to philosophy.
Geometry is a science of magnitude;
hence formal inference is the one most appropriate to it;
since it treats the quantitative determination alone,
abstracting from anything qualitative,
it can confine itself to formal identity,
to the unity void of concept which is equality
and belongs to external, abstractive reflection.
The determinations of space that are its
subject matter are already abstract objects,
suitably prepared for the purpose of
obtaining a perfectly finite, external determinateness.
This science, because of its abstract subject matter,
on the one hand has an aura of sublimity about it,
for in these empty silent spaces color is extinguished
and the other sensuous properties have equally vanished,
and further, every other interest that would appeal to
a living individuality is silenced.
On the other hand, this abstract subject matter is
still space, a non-sensuous sensuous.
To be sure, intuition is raised to
a higher level in this abstraction;
space is now a form of intuition,
but it is still intuition
sensuous intuition,
the externality of the senses themselves,
their pure absence of concept.
Enough has been heard lately
of the pre-eminence of geometry in this respect.
There are those who say
that geometry's foremost advantage is
that it is based on the intuition of the senses,
even believe that its scientific pre-eminence
depends on this circumstance
and that its proofs rest on intuition.
This shallow view must be countered with
the plain reminder that no science can be
brought about by intuition, but only by thought.
The intuitive character that geometry possesses
because of its still sensuous material
only gives to it that level of evidence that
the senses generally provide to thoughtless spirit.
It is therefore regrettable that the same sensuousness of material
which is a sign of the inferiority of its standpoint
has been reckoned instead to its advantage.
It is solely to the abstraction of its sensuous subject matter
that geometry owes its aptitude for a higher scientific reach
and the advantage that it has over the collections of information
that people are also want to call sciences
but have for content only the concrete
perceptible material of the senses,
and only because of the order that they seek to bring to it
do they give any sign of a remote inkling and hint of
the requirements of the concept.

It is only because the space of geometry is
the abstract emptiness of externality
that it is possible for figures
to be drawn in its indeterminateness
in such a way that their determinations
remain perfectly at rest outside one another
with no immanent transition to the opposite.
The science of these figures is therefore
plainly and simply the science of the finite
which is compared according to magnitude
and has for its unity the external one of equality.
But now, since with these figures the start
is made from a variety of sides and points at once,
and the various figures fall into place of themselves,
in comparing them their qualitative unlikeness
and incommensurability also come into view.
Geometry is thus driven,
beyond the finitude within which
it advanced step by step orderly and securely,
to infinity to the positing as equal
of such as are qualitatively diverse.
Here it loses the evidence that it derived from
being otherwise based on fixed finitude
without having to deal with the concept
and the transition to the opposite
which is its manifestation.
As a finite science, geometry reaches its limit at this point,
for the necessity and the mediation of the synthetic realm is
no longer grounded in merely positive identity,
but in negative identity.

If geometry, like algebra, quickly runs up against
its limit with its abstract subject matter,
suited as this is only to the understanding,
it is evident from the start that the synthetic method is
all the more insufficient for other sciences,
and most insufficient of all for philosophy.
Regarding definition and division,
we have already made the relevant points,
and we should be left here to speak
only of theorems and proofs.
But, besides the presupposition of definition and division
that itself requires proof and presupposes it,
also problematic is the very position of
definition and division with respect to the theorems.
This position is especially noteworthy in
the empirical sciences, as for example physics
This is how they go about it.
The reflective determinations of particular forces,
or of otherwise inner and essential forms,
which are the results of an analysis of experience
and can be justified only as such results,
must be placed at the top, in order to obtain
from them a general foundation
that can then be applied to the singular
and be instantiated there.
Since these general foundations have no hold of their own,
we must simply grant them in the meantime;
it is only in the derived consequences
that we notice that the latter are
in fact the ground of those presuppositions.
The so-called explanation,
and the proof of the concrete brought into theorems,
turn out to be partly a tautology,
partly an obfuscation of the true relation,
and partly also an obfuscation that serves
to hide the deception of cognition.
For cognition has collected experiences tendentiously,
only so that it could attain its simple definitions and principles;
and it has pre-empted the possibility of empirical refutation
by taking experiences and accepting them as valid,
not in their concrete totality but selectively,
as examples that can then be used on behalf of its
hypotheses and theories.
In this subordination of concrete experience to
presupposed determinations,
the foundation of the theory is obscured
and is only indicated according to
the side that suits the theory;
and, quite in general, the unprejudiced examination of
concrete perceptions for their own sake is thereby much impeded.
Only by turning the whole procedure upside down
does the whole thing acquire the right relation
in which the link of ground and consequence can come into view.
One of the principal obstacles in
the study of these sciences is
thus the way we enter into them,
which we can only do by blindly taking
the presuppositions for granted
and, without being able to form any further concept of them,
often not even an exact representation,
at best by conjuring up in phantasy a confused picture of them,
we right there impress in our memory the determinations
of the forces and matters that we have assumed,
their hypothetical shapes, their directions and rotations.
If we are asked to produce the necessity
and the concept of these assumptions
in order to justify assuming their validity,
we discover that we are incapable of making
a step beyond the starting point.
We had occasion above to speak of
the inappropriateness of applying the
synthetic method to strictly analytic science.
Wolff extended this application to
every kind of bits of knowledge
that he dragged into philosophy and mathematics
cognitions which were partly of a wholly analytical nature,
and partly also devoted to
practical matters of an incidental kind.
The incongruity between this material,
easy to grasp and by nature incapable of
rigorous and scientific treatment
and the pompous scientific roundabouts icing it,
has alone demonstrated the clumsiness of such an application,
finally discrediting it.
Yet, this misuse has not sufficed to
shake the belief that this method is
both suited and essential to attaining scientific rigor in philosophy.
Spinoza's example, the way he presented his philosophy,
has long served as model in this regard.
But the fact is that Kant and Jacobi
did do away with this whole style of
the previous metaphysics
and its method along with it.
As for the content of that metaphysics,
Kant has in his own fashion shown that it leads
by strict demonstration to antinomies,
the same whose nature we have in other respects
elucidated at the appropriate places.
But Kant did not reflect on the nature of
the demonstration associated with them,
on the fact that such a demonstration is
inextricably bound to a finite content.
In his Principles of Natural Science,
he gave himself an example of how to
deal with a science of reflection on
its own methodological terms,
in a way that he thought would
vindicate it for philosophy.
While Kant attacked previous metaphysics
for the most part from the side of its content,
Jacobi did it especially from the side
of its method of demonstration
and, with great clarity and profundity,
he put his finger on precisely the point at issue,
namely that such a method of demonstration is
strictly bound to the cycle of rigid necessity
of finite reality,
and that freedom, that is, the concept
and with it everything that truly exists,
lies beyond it and is unattainable by it.
According to Kant's result,
it is the peculiar content of metaphysics
that leads it into contradictions;
the inadequacy of cognition is due to its subjectivity.
Jacobi's result is that the inadequacy is
due instead to the method
and the whole nature of cognition itself
that only grasps a concatenation of
conditions and dependency
and therefore proves itself inadequate
to what exists in and for itself,
to what is absolutely true.
And in fact, since the principle of philosophy is
the infinite free concept
and all its content rests on that alone,
the method suited to a finitude empty of concept is
inadequate to it.
The synthesis and the mediation of this method,
the process of proving, goes no further than a
necessity which is opposed to freedom,
that is, an identity of the dependent
which is only implicit,
whether it is apprehended as internal or as external,
and in which that which in it constitutes reality,
the differentiation that has emerged in concrete existence,
remains simply self-subsistent diversity
and therefore something finite.
In this reality, therefore, this identity does
not itself attain concrete existence
but remains only internal,
or again, is only external,
because its determinate content is given to it.
Either way, whether internal or external,
the identity is something abstract
that does not possess within it the side of reality,
is not posited as determinate
identity in and for itself;
therefore the concept,
which alone is the issue here
and which is the infinite
in and for itself,
is precluded from this cognition.

In synthetic cognition, therefore,
the idea achieves its purpose only to the extent
that the concept becomes for the concept according to
its moments of identity and real determinations,
or of universal and particular differences
further also as an identity
which is connectedness and dependence in diversity.
But this, its subject matter, is not adequate to the concept;
for in it, in this subject matter or in its reality,
the concept does not come to be the unity
of itself with itself;
in necessity its identity is for it,
but in this identity the necessity is
not itself the determinateness
but is on the contrary a material
external to it, that is to say,
is not determined by the concept
and the concept, therefore,
does not recognize itself in it.
Thus in general the concept is not for itself,
is not at the same time determined in and for itself
according to its unity.
For this reason the idea does not as yet attain
the truth in this cognition:
it does not because of the disproportion
between subject matter and subjective concept.
But the sphere of necessity is
the highest point of being and reflection;
of itself, in and for itself, it passes
over into the freedom of the concept,
inner identity passes over into its manifestation
which is the concept as concept.
How this transition from the sphere of necessity
to the concept occurs in itself has been shown
when considering necessity,
and we saw it also at the beginning of this Book as
the genesis of the concept.
In the present context, necessity has the position of
being the reality or the subject matter of the concept,
just as the concept into which it passes is
now the concept's subject matter.
But the transition itself is the same.
Here, too, it is at first only in itself,
still lying in our reflection outside cognition,
that is, itself still the inner necessity of cognition.
Only the result is for cognition.
The idea, in so far as the concept is now
for itself determined in and for itself,
is the practical idea, action.

B. THE IDEA OF THE GOOD

Inasmuch as the concept,
which is its own subject matter,
is determined in and for itself,
the subject is determined as singular.
As subjective it again has
an implicit otherness for its presupposition;
it is the impulse to realize itself,
the purpose that on its own wants togive itself
objectivity in the objective world
and realize itself.
In the theoretical idea the subjective concept,
as a universal that in and for itself lacks determination,
stands opposed to the objective world
from which it derives determinate content and filling.
But in the practical idea it is as actual
that it stands over against the actual;
but the certainty of itself that the subject possesses
in being determined in and for itself is
a certainty of its actuality
and of the non-actuality of the world;
it is the singularity of this world,
and the determinateness of its singularity,
not just its otherness as abstract universality,
which is a nullity for the subject.
The subject has here vindicated objectivity for itself;
its inner determinateness is the objective,
for it is the universality which is just
as much absolutely determined;
the previously objective world is on the
contrary only something still posited,
an immediate which is determined in
a multitude of ways but which,
because it is only immediately determined,
in itself eludes the unity of the concept
and is of itself a nullity.

This determinateness which is in the concept,
is equal to the concept,
and entails a demand for singular external actuality, is the good.
It comes on the scene with the dignity of being absolute,
because it is intrinsically the totality of the concept,
the objective which is at the same time
in the form of free unity and subjectivity.
This idea is superior to the idea of cognition just considered,
for it has not only the value of the universal
but also of the absolutely actual.
It is impulse, in so far as this actual is still
subjective, self-positing, without at the same time the form of immediate
presupposition; its impulse to realize itself is not, strictly speaking, to give
itself objectivity, for this it possesses within itself, but to give itself only
this empty form of immediacy.
The activity of purpose, therefore, is
not directed at itself, is not a matter of letting in a given determination
and making it its own, but of positing rather its own determination and,
by means of sublating the determinations of the external world, giving
itself reality in the form of external actuality.
The idea of the will as a
self-determining explicitly possesses content within itself. Now this content
is indeed a determinate content, and to this extent finite and restricted;
self-determination is essentially particularization, since the reflection of
the will is in itself, as negative unity as such, also singularity in the sense
that it excludes an other while presupposing it. Yet the particularity of the
content is at first infinite by virtue of the form of the concept, of which
it is the proper determinateness, and which in that content possesses its
negative self-identity, and consequently not only a particularity but its
infinite singularity. The mentioned finitude of the content in the practical
idea only means, therefore, that the idea is at first not yet realized; the
concept is for the content that which exists in and for itself; it is here the
idea in the form of objectivity existing for itself; on the one hand, the
subjective is for this reason no longer just something posited, arbitrary or
accidental, but is an absolute; but, on the other hand, this form of concrete
existence, this being-for-itself, does not as yet have the form of the being-in-
itself. Thus what from the side of the form as such appears as opposition,
appears in the form of the concept reflected into simple identity, that is,
appears in the content as its simple determinateness; the good, although
valid in and for itself, is thereby a certain particular purpose, but not one
that first receives its truth by being realized; on the contrary, it is for itself
already the true.

The syllogism of immediate realization
does not itself require closer exposition here;
it is none other than the previously considered syllogism
of external purposiveness;
only the content constitutes the difference.
In external as in formal purposiveness
it was an indeterminate finite content in general;
here, though also finite,
it is as such at the same time absolutely valid.
But in regard to the conclusion, the realized purpose,
a further difference enters in.
In being realized the finite purpose still attains
only the status of a means;
since it is not a purpose determined in and for itself
already from the beginning,
as realized it also remains something that does not
exist in and for itself.
If the good is again also fixed as something finite,
and is essentially such, then,
notwithstanding its inner infinity,
it too cannot escape the fate of finitude
a fate that manifests itself in several forms.
The realized good is good by virtue of what
it already is in the subjective purpose, in its idea;
the realization gives it an external existence,
but since this existence has only the status of
an externality which is in and for itself null,
what is good in it has attained only
an accidental, fragile existence,
not a realization corresponding to the idea.
Further, since this good is restricted in content,
there are several kinds of it;
in concrete existence a good is subject to
destruction not only due to external contingency and to evil,
but also because of collision and conflict in the good itself.
From the side of the objective world presupposed for it
(in the presupposition of which consists
the subjectivity and the finitude of the good,
and which as a distinct world runs its own course),
the realization itself of the good
is exposed to obstacles, indeed,
might even be made impossible.
The good thus remains an ought;
it exists in and for itself,
but being, as the ultimate abstract immediacy,
remains over against it also determined as a non-being.
The idea of the fulfilled good is
indeed an absolute postulate,
but no more than a postulate, that is,
the absolute encumbered with
the determinateness of subjectivity.
There still are two worlds in opposition,
one a realm of subjectivity in
the pure spaces of transparent thought,
the other a realm of objectivity in
the element of an externally manifold actuality,
an impervious realm of darkness.
The complete development of this unresolved contradiction,
between that absolute purpose
and the restriction of this reality
that stands opposed to it,
has been examined in detail in
the Phenomenology of Spirit (pp. 323ff.).
Inasmuch as the idea has within it
the moment of complete determinateness,
the other concept to which the concept in
it relates possesses in its subjectivity
at the same time the moment of an object;
consequently the idea enters here into
the shape of self-consciousness,
and in this one respect
coincides with its exposition.

But what the practical idea still lacks is
the moment of real consciousness itself,
namely that the moment of actuality in the concept
would have attained for itself
the determination of external being.
This lack can also be regarded in this way,
namely that the practical idea still lacks the moment
of the theoretical idea.
That is to say, in the latter there stands
on the side of the subjective concept
the concept that is in process of being intuited
in itself by the concept
only the determination of universality;
cognition only knows itself as apprehension,
as the identity of the concept with itself
which, for itself, is indeterminate;
the filling, that is, the objectivity
determined in and for itself,
is for this identity a given;
what truly exists is for it
the actuality present there independently
of any subjective positing.
For the practical idea, on the contrary,
this actuality constantly confronting it as
an insuperable restriction is in and for itself a nullity
that ought to receive its true determination and intrinsic value
only through the purposes of the good.
It is the will, therefore, that alone stands
in the way of attaining its goal,
because it separates itself from cognition
and because for it external actuality does not
receive the form of a true existence.
The idea of the good can therefore find its completion
only in the idea of the true.

But it makes this transition through itself.
In the syllogism of action,
one premise is the immediate reference of
the good purpose to the actuality which it appropriates
and which, in the second premise, it directs as
external means against the external actuality.
The good is for the subjective concept the objective;
actuality confronts it in existence
as an insuperable restriction
only in so far as it still has
the determination of immediate existence,
not of something objective in the sense
that it is being in and for itself;
it is rather either the evil or the indifferent,
the merely determinable,
whose worth does not lie within it.
But this abstract being that confronts the good
in the second premise has already been sublated
by the practical idea itself;
the first premise of this idea's action is
the immediate objectivity of the concept,
according to which purpose is communicated to
actuality without any resistance
and is in the simple connection of identity with it.
To this extent, therefore, what remains is to bring together
the thoughts of the two premises of the practical idea.
All that is added to what is already accomplished
in the first premise by the objective concept is
that in the second it is posited by way of mediation, hence for it.
Just as in purposive connection in general,
where the realized purpose is again only a means
but the means is conversely also the realized purpose,
so too now in the syllogism of the good
the second premise is already
immediately present in the first in itself,
except that this immediacy is not sufficient
and the second premise is for the first already postulated
the realization of the good in
the face of another actuality confronting it is
the mediation which is essentially necessary
for the immediate connection and consummation of the good.
For the first premise is only the first negation
or the otherness of the concept,
an objectivity that would be a state of immersion of
the concept into externality;
the second premise is the sublation of this otherness,
whereby the immediate realization of the purpose
first becomes the actuality of the good as
concept existing for itself,
for in that actuality the concept is
posited as identical with itself,
not with an other,
and in this way alone as free concept.
If it is now claimed that the purpose of the good is
thereby still not realized,
what we have is a relapse of the concept to
the standpoint that it assumes prior to its activity,
when the actual is determined as worthless
and yet presupposed as real.
This is a relapse that gives rise to
the progression to bad infinity.
Its sole ground is that in the sublating of that abstract reality
the sublating itself is just as immediately forgotten,
or what is forgotten is that this reality is
rather already presupposed as an actuality
which is in and for itself worthless, nothing objective.
This repetition of the presupposition of
the unrealized purpose after
the actual realization of the purpose
also means that the subjective attitude of
the objective concept is reproduced and perpetuated,
with the result that the finitude of the good,
with respect to both content and form,
appears as the abiding truth,
and its actualization always as
only a singular, never universal, act.
As a matter of fact this state has
already sublated itself
in the realization of the good;
what still limits the objective concept is
its own view of itself,
and this view vanishes in the reflection
on what its realization is in itself.
By this view the concept only stands in its own way,
and all that it has to do about it is to turn,
not against an external actuality, but against itself.

That is to say, the activity in the second premise
produces only a one-sided being-for-itself,
and its product therefore appears as
something subjective and singular,
and the first presupposition is
consequently repeated in it.
But this activity is in truth just as much
the positing of the implicit identity
of the objective concept and the immediate actuality.
This actuality is by presupposition determined
to have only the reality of an appearance,
to be in and for itself a nullity,
entirely open to determination by the objective concept.
As the external actuality is altered by
the activity of the objective concept
and its determination is consequently sublated,
the merely apparent reality,
the external determinability and worthlessness,
are by that very fact removed from it
and it is thereby posited as having existence
in and for itself.
In this the presupposition itself is sublated,
namely the determination of the good as
a merely subjective purpose restricted in content,
the necessity of first realizing it
by subjective activity,
and this activity itself.
In the result the mediation itself sublates itself;
the result is an immediacy which is not
the restoration of the presupposition,
but is rather the presupposition as sublated.
The idea of the concept that is determined in
and for itself is thereby posited,
no longer just in the active subject but
equally as an immediate actuality;
and conversely, this actuality is posited
as it is in cognition,
as an objectivity that truly exists.
The singularity of the subject with which
the subject was burdened by its presupposition
has vanished together with the presupposition.
Thus the subject now exists
as free, universal self-identity
for which the objectivity of the concept is
a given, just as immediately present to the subject
as the subject immediately knows itself to be
the concept determined in and for itself.
Accordingly, in this result cognition is
restored and united with the practical idea;
the previously discovered reality is
at the same time determined as
the realized absolute purpose,
no longer an object of investigation,
a merely objective world without
the subjectivity of the concept,
but as an objective world whose inner ground
and actual subsistence is rather the concept.
This is the absolute idea.
