
B. APPEARANCE

The essence must appear.

Its shining within itself is
the sublating itself
and becoming an immediacy
which, as reflection-in-itself, is
as much a subsisting (matter)
as it is form, reflection-in-another,
subsisting in the process of sublating itself.

Its shining is the determination
through which the essence is not being but essence,
and the shining, once developed, is the appearance.
The essence is thus not behind or beyond the appearance;
instead, by virtue of the fact that
it is the essence that exists concretely,
concrete existence is appearance.

a. The world of appearance

What appears concretely exists in such a way
that its subsisting is immediately sublated;
it is only one moment of the form itself.
The form encompasses in itself the subsisting
or the matter as one of its determinations.
What appears thus has its ground in the form
as its essence, its reflection-in-itself
as opposed to its immediacy, but thereby has it
only in another determinacy of the form.
This, its ground, is just as much something appearing,
and thus the appearance continues on to an infinite mediation
of the subsisting through the form
and thus equally through not subsisting.
This infinite mediation is at once
a unity of relation-to-itself,
and concrete existence develops into a totality
and world of appearance, of reflected finitude.

b. Content and form

The manner of being-outside-one-another that is
characteristic of the world of appearances is
a totality and completely contained in its relation-to-itself.
The relation of the appearance to itself is
thus completely determined, has the form in itself
and because [it is] in this identity,
has that form as its essential subsistence.
Thus the form is content and,
in keeping with its developed determinacy,
it is the law of the appearance.
The negative side of the appearance,
what is alterable and not self-sufficient,
falls to the form as not reflected in itself;
it is the indifferent, external form.

For the contrast of form and content,
it is essential to keep in mind that
the content is not formless but
instead has the form within itself
just as much as it [the form] is
something external to it.

A doubling of the form presents itself;
at one time,
insofar as it is reflected in itself,
it is the content and,
at another time,
as not reflected in itself,
it is the external concrete existence,
indifferent to the content.

What presents itself here in itself is
the absolute relation of content and of form, namely,
their turning over and into one another,
so that the content is nothing but the form turning into content
and the form nothing other than the content turning into the form.
This 'turning over' is one of the most important determinations.
It is posited, however, only in the absolute relationship.

The immediate concrete existence, however, is
the determinacy of the subsisting itself as well as of the form;
it is thus just as much external to the determinacy of the content
as this externality, which it has through the element of its subsisting,
is essential to it.
The appearance, so posited, is the relationship
such that one and the same, [namely] the content, is
as the developed form, as the externality and opposition of
self-standing concrete existences and their identical relation,
the relation in which alone the differentiated elements are what they are.

c. The relationship

(a) The immediate relationship is that of the whole and the parts:
the content is the whole and consists of the parts (the form),
the opposite of it.
The parts are diverse from one another
and are what is self-standing.
But they are only parts in their identical relation to one another
or insofar as, taken together, they make up the whole.
But that 'together' is the opposite and negation of the part.

(b) What is one and the same in this relationship
(the relation to itself that is on hand in it)
is thus an immediately negative relation to itself and,
to be sure, as the mediation to the effect that one and the same
is indifferent to the difference, and that
it is the negative relation to itself that repels itself,
as reflection-in-itself, towards the difference, and posits itself,
concretely existing as reflection-into-another and, in reverse direction,
conducts this reflection-into-another back to
the relation to itself and to the indifference:
the force and its expression.

The relationship of the whole and the parts is
the immediate relationship;
hence, the thoughtless relationship
and the process of the identity-with-itself
turning over into diversity.
There is a passage from the whole to the parts and
from the parts to the whole,
and in the one [the whole or the part]
the opposition to the other is forgotten since
each is taken as a self-standing concrete existence,
the one time the whole, the other time the parts.
Or since the parts are supposed to subsist in the whole
and the whole to consist of the part one time the one,
the other time the other is the subsisting and
the other is each time the unessential.
The mechanical relationship,
in its superficial form,
consists generally in the fact
that the parts are taken as self-sufficient
opposite one another and opposite the whole.

The infinite progression that concerns
the divisibility of matter
can avail itself of this relationship too,
and then it is the thoughtless oscillation
of both sides of the relationship.
A thing is taken one time as a whole,
then there is a passage to the determination of it as a part,
this determination is then forgotten
and what was a part is limited and thus acquires
its determinacy by means of an other outside it
regarded as a whole;
the determination of it as a part resurfaces
and so on, ad infinitum.
Taken as the negative that it is, however,
this infinity is the negative relation of
the relationship to itself, the force,
the whole that is identical with itself as being-in-itself,
and as this being-in-itself sublating itself and expressing itself
and, conversely, the expression that disappears
and goes back into the force.

This infinity notwithstanding,
the force is also finite.
For the content, the one and the same
that the force and the expression are, is
initially this identity only in itself.
The two sides of the relationship
are not yet themselves,
each for itself its concrete identity,
not yet the totality.
In relation to one another, they are thus diverse
and the relationship is a finite one.
The force is thus in need of solicitation from without;
it acts blindly, and, thanks to this deficiency of the form,
the content is also limited and contingent.
It is not yet truly identical with the form,
is not yet the concept and purpose
that is the determinate in and for itself.
This difference is supremely essential,
but not easy to grasp;
it has to be determined more precisely and
only in terms of the concept of purpose.
If it is overlooked, this leads to
the confusion of construing God as force,
a confusion from which Herder's God suffers especially.

It is usually said that the nature of force itself is unknown
and only its expression is known.
On the one hand, the entire determination of the content of force is
just the same as that of the expression;
on account of this, the explanation of a phenomenon
on the basis of a force is an empty tautology.
What is supposed to
remain unknown is therefore in fact nothing but the empty form
of the reflection-in-itself, by means of which alone the force is
distinguished from the expression, a form that is equally
something well known.
This form adds nothing in the slightest to the content and to the law,
which are supposed to be known simply on the basis of the phenomenon alone.
Assurances are also given everywhere that, with this, nothing is
supposed to be claimed about the force; as a result, it is impossible to
see why the form of force has been introduced into the sciences.
But, on the other hand, the nature of force is, of course, something
unknown since the necessity of the connection of its content is still
lacking, not only in itself but also and equally insofar as it is for itself
limited and thus acquires its determinacy by means of an other outside it.

As the whole that is, in its very self,
the negative relation to itself, force is this:
the process of repelling itself from itself
and expressing itself.
But since this reflection-in-another,
the difference of the parts,
is just as much a reflection-in-itself,
the expression is the mediation by means of which
the force that returns into itself is force.
Its expression is itself the sublating of
the diversity on both sides,
which is on hand in this relationship,
and the positing of the identity
that in itself makes up the content.
Its truth is, for that reason, the relationship,
the two sides of which are distinguished only as inner and outer.

(c) The inner is the ground as the mere form
of the one side of the appearance and the relationship,
the empty form of the reflection-in-itself.
Standing opposite it is concrete existence as the form likewise
of the other side of the relationship,
with the empty determination of the
reflection-in-another as outer.
Their identity is the fulfilled identity, the content,
the unity of the reflection-in-itself and
the reflection-in-another,
posited in the movement of force.
Both are the same, one totality,
and this unity makes them into the content.

The outer is thus, in the first place, the same content as the inner is.
What is internal is also on hand externally and vice versa.
The appearance shows nothing that is not in the essence and
there is nothing in the essence that is not manifested.

In the second place, however,
inner and outer are also opposed to one another
as determinations of the form
and, to be sure, unqualifiedly so,
as the abstractions of identity with itself
and of sheer multiplicity or reality.
Yet, since they are essentially identical
as moments of the one form,
what is only posited initially
in the one abstraction is
also immediately only in the other.
Hence, what is only something internal is
also, by this means, only something external
and what is only something external is
as yet also only something internal.

It is the usual mistake of reflection
to take the essence as the merely inner.
When it is taken merely in this way,
then this consideration is
also a completely external one
and that essence is the empty external abstraction.

    The inner side of nature a poet says
    No created spirit can penetrate,
    Fortunate enough if he knows merely the outer shell

It should have been said, rather,
that precisely when he determines
the essence of nature as something inner,
he knows only the outer shell.
Since in being in general
or even in merely sensory perception,
the concept is only the inner at first,
it is something external for it [sensory perception],
a subjective being as well as thinking, devoid of truth.
In nature as in the spirit, insofar as
the concept, purpose, law are at first
only inner dispositions, pure possibilities,
they are only an external, inorganic nature at first,
science of a third, alien power, and so forth.
As a human being is externally, in his actions
(not, of course, in his merely corporeal externality),
so he is internally;
and if he is only internally virtuous, moral, and so forth,
only in intentions and sentiments
and his outer life is not identical with them,
then the one is as hollow and empty as the other.

The empty abstractions, by means of which
the one identical content is still supposed
to obtain in the relationship,
sublate themselves in the immediate transition,
the one in the other;
the content is itself nothing other than their identity,
they are the shine of the essence, posited as shine.
Through the force's expression, the inner is posited in concrete existence;
this positing is the mediating by means of empty abstractions;
it vanishes in itself into the immediacy in which
the inner and outer are in and for themselves identical and
their difference is determined as mere positedness.
This identity is the actuality.
