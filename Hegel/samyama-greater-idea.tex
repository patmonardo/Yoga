SECTION III

The idea

The idea is the adequate concept,
the objectively true,
or the true as such.
If anything has truth,
it has it by virtue of its idea,
or something has truth only
in so far as it is idea.
The expression “idea” has otherwise also often been used
in philosophy as well as in ordinary life for “concept,”
or even for just a “representation.”
To say that I still have no idea of
this lawsuit, this building, this region,
means nothing more than I still have no representation of it.
It is Kant who reclaimed the expression “idea”
for the “concept of reason.”
Now according to Kant the concept of reason
should be the concept of the unconditional,
but a concept which is transcendent
with respect to appearances,
that is, one for which no adequate empirical use can be made.
The concepts of reason are supposed to serve
for the comprehension of perceptions,
those of the understanding for the understanding of them.
In fact, however, if these last
concepts of the understanding are truly concepts,
then they are comprehensions, which means concepts;
they will make comprehending possible,
and an understanding of perceptions
through concepts of the understanding
will be a comprehending.
But if understanding is only
the determining of perceptions by categories
such as whole and parts, force, cause, and the like,
then it signifies only a determining by means of reflection,
just as by understanding one may mean only the determinate
representation of a fully determined sensuous content;
as when someone is being shown the way,
that at the end of the wood he must turn left,
and he replies “I understand,”
understanding means nothing more than a grasp
in pictorial representation and in memory.
“Concept of reason,” too, is a somewhat clumsy expression;
for the concept is in general something rational,
and in so far as reason is distinguished
from the understanding and the concept as such,
it is the totality of the concept and objectivity.
The idea is the rational in this sense;
it is the unconditioned,
because only that has conditions
which essentially refers to an objectivity
that it does not determine itself
but which still stands over against it
in the form of indifference and externality,
just as the external purpose had conditions.

If we now reserve the expression “idea”
for the objective or real concept
and we distinguish it from the concept itself
and still more from mere representation,
then we must also even more definitely reject
that estimate of it according to which
the idea is something with no actuality,
and true thoughts are accordingly said to be only ideas.
If thoughts are something merely subjective and contingent,
then they certainly have no further value;
but in this they do not stand lower
than the temporal and contingent actualities
which likewise have no further value than
that of accidentalities and appearances.
But if, on the contrary, the idea is
supposed not to have the value of truth
because in regard to appearances it is transcendent,
because no congruent object can be given for it
in the world of the senses,
then this is indeed an odd misunderstanding,
for objective validity is being denied to it
on the ground that it lacks precisely what makes
of appearances the untrue being of the objective world.
In regard to the practical ideas, Kant recognizes that
“nothing can be more harmful and unworthy of
a philosopher than the vulgar appeal to experience,
which supposedly contradicts the idea.
Any such alleged contradiction would not be there at all
if, for example, political institutions were set up
at the right time in accordance with ideas,
and if crude concepts, crude just
because they are drawn from experience,
had not usurped the place of ideas
thus thwarting all good intentions.”
Kant regards the idea as something necessary,
the goal which, as the archetype,
we must strive to set up as a maximum
and to which we must bring actuality
as it presently stands ever closer.

But since the result now is that the idea is
the unity of the concept and objectivity, the true,
we must not regard it as just a goal
which is to be approximated
but itself remains always a kind of beyond;
we must rather regard everything as being actual
only to the extent that it has the idea
in it and expresses it.
It is not just that the subject matter,
the objective and the subjective world,
ought to be in principle congruent with the idea;
the two are themselves rather the
congruence of concept and reality;
a reality that does not correspond
to the concept is mere appearance,
something subjective, accidental, arbitrary,
something which is not the truth.
When it is said that there is no subject matter
to be found in experience
which is perfectly congruent with the idea,
the latter is opposed to the actual
as a subjective standard;
but there is no saying what
anything actual might possibly be in truth,
if its concept is not in it
and its objectivity does not
measure up to this concept;
it would then be a nothing.
Indeed, the mechanical and the chemical object,
like a subject devoid of spirit
and a spirit conscious only of finitude
and not of its essence,
do not, according to their various natures,
have their concept concretely existing in them
in its own free form.
But they can be something at all true
only in so far as they are the union
of their concept and reality,
of their soul and their body.
Wholes like the state and the church
cease to exist in concreto when the unity
of their concept and their reality is dissolved;
the human being, the living thing, is dead
when soul and body are parted in it;
dead nature, the mechanical and the chemical
world that is, when “the dead” is taken
to mean the inorganic world,
for the expression would otherwise have
no positive meaning at all,
this dead nature, then, if it is separated
into its concept and its reality,
is nothing but the subjective abstraction of
a thought form and a formless matter.
Spirit that were not idea,
not the unity of the concept with itself,
not the concept that has the concept itself as its reality,
would be dead spirit, spiritless spirit, a material object.

Since the idea is the unity of the concept and reality,
being has attained the significance of truth;
it now is, therefore, only what the idea is.
Finite things are finite because
and to the extent that,
they do not possess the reality of
their concept completely within them
but are in need of other things for it
or, conversely,
because they are presupposed as objects
and consequently the concept is in them
as an external determination.
The highest to which they attain on
the side of this finitude is external purposiveness.
That actual things are not congruent with the idea
constitutes the side of their finitude, of their untruth,
and it is according to this side that they are objects,
each in accordance with its specific sphere,
and, in the relations of objectivity, determined
as mechanical, chemical, or by an external purpose.
That the idea has not perfectly fashioned their reality,
that it has not completely subjugated it to the concept,
the possibility of that rests on the fact that
the idea itself has a restricted content;
that, as essentially as it is
the unity of the concept and reality,
just as essentially it is also their difference;
for only the object is the immediate unity, that is,
the unity that only exists in itself.
But if a subject matter, say the state,
did not at all conform to its idea,
that is to say, if it were not rather
the idea of the state;
if its reality, which is the self-conscious individuals,
did not correspond at all to the concept,
its soul and body would have come apart;
the soul would have taken refuge in the secluded regions of thought,
the body been dispersed into singular individualities.
But because the concept of the state is essential
to the nature of these individualities,
it is present in them as so mighty an impulse
that they are driven to translate it into reality,
be it only in the form of external purposiveness,
or to put up with it as it is,
or else they must needs perish.
The worst state, one whose reality least corresponds to the concept,
in so far as it still has concrete existence, is yet idea;
the individuals still obey the power of a concept.

But the idea has not only
the general meaning of true being,
of the unity of concept and reality,
but also the more particular one
of the unity of subjective concept and objectivity.
For the concept is as such itself
already the identity of itself and reality;
for the indeterminate expression “reality”
means nothing but determinate being,
and this the concept possesses
in its particularity and singularity.
Objectivity, moreover, is likewise the total concept
that has withdrawn into identity with itself
out of its determinateness.
In the subjectivity of the concept,
the determinateness
or the difference of the latter is
a reflective shine which is immediately sublated,
withdrawn into being-for-itself
or into negative unity,
an inhering predicate.
But in this objectivity the determinateness is
posited as immediate totality, as external whole.
Now the idea has shown itself to be the concept
liberated again into its subjectivity
from the immediacy into which it has sunk in the object;
it is the concept that
distinguishes itself from its objectivity
but an objectivity which is no less determined by it
and possesses its substantiality only in that concept.
This identity has therefore rightly been
designated as a subject-object,
for it is just as well the formal
or subjective concept
as it is the object as such.
But this is a point that needs further precision.
The concept, inasmuch as it has truly attained its reality,
is this absolute judgment whose subject distinguishes itself
as self-referring negative unity from its objectivity
and is the latter's being-in-and-for-itself;
but it refers to it essentially through itself
and is, therefore, self-directed purpose and impulse.
For this very reason, however,
the subject does not possess objectivity immediately in it
(it would then be only the totality of the object as such,
a totality lost in the objectivity)
but is the realization of the purpose,
an objectivity posited by virtue of
the activity of the purpose,
one which, as positedness, has its
subsistence and its form only as
permeated by its subject.
As objectivity, it has the moment of the externality
of the concept in it and is in general,
therefore, the side of finitude,
of alteration and appearance;
but this side retreats into the negative unity of
the concept and there it perishes;
the negativity whereby its indifferent
externality of being manifests itself as unessential
and as a positedness is the concept itself.
Despite this objectivity, the idea is therefore
absolutely simple and immaterial,
for the externality has being only
as determined by the concept
and as taken up into its negativity;
in so far as it exists as indifferent externality,
it is not only abandoned to mechanism in general
but exists only as the transitory and untrue.
Thus although the idea
has its reality in a materiality,
the latter is not an abstract being standing
over against the concept
but, on the contrary, it exists only as becoming,
as simple determinateness of the concept
by virtue of the negativity of the indifferent being.

This yields the following closer determinations of the idea.

First, the idea is the simple truth,
the identity of concept and objectivity as a
universal in which the opposition,
the presence of the particular,
is dissolved in its self-identical negativity
and is equality with itself.

Second, it is the connection of the subjectivity
of the simple concept, existing for itself,
and of the concept's objectivity which is distinguished from it;
the former is essentially the impulse to sublate this separation,
and the latter is indifferent positedness,
subsistence which in and for itself is null.
As this connection, the idea is
the process of disrupting itself into individuality
and into the latter's inorganic nature,
and of then bringing this inorganic nature again
under the controlling power of the subject
and back to the first simple universality.
The identity of the idea with itself is one with the process;
the thought that liberates actuality from
the seeming of purposeless mutability
and transfigures it into idea
must not represent this truth of actuality
as dead repose, as a mere picture, numb, without impulse and movement,
as a genus or number, or as an abstract thought;
the idea, because of the freedom which the concept has attained in it,
also has the most stubborn opposition within it;
its repose consists in the assurance and the certainty
with which it eternally generates that opposition
and eternally overcomes it, and in it rejoins itself.

But the idea is at first again only immediate or only in its concept;
the objective reality is indeed conformable to the concept
but has not yet been liberated into the concept,
and it does not concretely exist explicitly as the concept.
Thus the concept is indeed the soul,
but the soul is in the guise of an immediate,
that is, it is not determined as soul itself,
has not comprehended itself as soul,
does not have its objective reality within itself;
the concept is as a soul that is not yet fully animated.

Thus the idea is, first of all, life.
It is the concept which, distinct from its objectivity,
simple in itself, permeates that objectivity
and, as self-directed purpose, has its means within it
and posits it as its means, yet is immanent in this means
and is therein the realized purpose identical with itself.
The idea, on account of its immediacy, has singularity
for the form of its concrete existence.
But the reflection within it of its absolute process is
the sublating of this immediate singularity;
thereby the concept, which as universality is
in this singularity the inner,
transforms externality into universality,
or posits its objectivity as a self-equality.

Thus is the idea, in second place,
the idea of the true and the good,
as cognition and will.
It is at first finite cognition and finite will,
where the true and the good are still distinguished
and the two are at first only as a goal.
The concept has first liberated itself into itself,
giving itself only a still abstract objectivity for its reality.
But the process of this finite cognition and this finite action
transforms the initially abstract universality into totality,
whereby it becomes complete objectivity.
Or considered from the other side,
finite, that is, subjective spirit,
makes for itself the
presupposition of an objective world,
such a presupposition as life only has;
but its activity is the sublating of this presupposition
and the turning of it into something posited.
Thus its reality is for it the objective world,
or conversely the objective world is
the ideality in which it knows itself.

Third, spirit recognizes the idea
as its absolute truth,
as the truth that is in and for itself:
the infinite idea in which
cognizing and doing are equalized,
and which is the absolute knowledge of itself.

CHAPTER 1

Life

The idea of life has to do with
a subject matter so concrete,
and if you will so real,
that in dealing with it one may seem
according to the common notion of logic
to have overstepped its boundaries.
Needless to say, if the logic were to contain
nothing but empty, dead forms of thought,
then there could be no talk in it at all
of such a content as the idea, or life, are.
But if the subject matter of logic is the absolute truth,
and truth as such lies essentially in cognition,
then cognition at least would have to come in for consideration.
It is common practice to have the so-called pure logic
be followed by an applied logic,
a logic that has to do with concrete cognition,
quite apart from all the psychology and anthropology
that is commonly deemed necessary to interpolate into logic.
But the anthropological and psychological side of
cognition is concerned with the form in which cognition appears
when the concept does not as yet have an objectivity equal to it,
that is, when it does not have itself as object.
The part of the logic that deals with this
concrete cognition does not belong to applied logic as such;
if it did, then every science would have to be dragged into logic,
for each is an applied logic in so far as
it consists in apprehending its subject matter
in forms of thought and of concepts.
The subjective concept has presuppositions
that are exhibited in psychological, anthropological, and other forms.
But the presuppositions of the pure concept
belong in logic only to the extent that
they have the form of pure thoughts, of abstract essentialities,
such as are the determinations of being and essence.
The same goes for cognition,
which is the concept's comprehension of itself:
no other shape of its presupposition
but the one which is itself idea
is to be dealt with in the logic;
this, however, is a presupposition
which is necessarily treated in logic.
This presupposition is now the immediate idea;
for while cognition is the concept,
in so far as the latter exists for itself
but as a subjectivity referring to an objectivity,
then the concept refers to the idea
as presupposed or as immediate.
But the immediate idea is life.

To this extent the necessity of considering
the idea of life in logic would be based on
the necessity, itself recognized in other ways,
of treating the concrete concept.
But this idea has arisen through the concept's own necessity;
the idea, that which is true in and for itself,
is essentially the subject matter of the logic;
since it is first to be considered in its immediacy,
so that this treatment be not
an empty affair devoid of determination,
it is to be apprehended and cognized
in this determinateness in which it is life.
A comment may be in order here to
differentiate the logical view of life
from any other scientific view of it,
though this is not the place to concern
ourselves with how life is treated in non-philosophical sciences
but only with how to differentiate logical life as idea
from natural life as treated in the philosophy of nature,
and from life in so far as it is bound to spirit.
As treated in the philosophy of nature,
as the life of nature and to that extent
exposed to the externality of existence,
life is conditioned by inorganic nature
and its moments as idea are a manifold of actual shapes.
Life in the idea is without such presuppositions,
which are in shapes of actuality;
its presupposition is the concept
as we have considered it,
on the one hand as subjective,
and on the other hand as objective.
In nature life appears as the highest stage
that nature's externality can attain
by withdrawing into itself
and sublating itself in subjectivity.
It is in logic the simple in-itselfness
which in the idea of life has attained
the externality truly corresponding to it;
the concept that came on the scene earlier as
a subjective concept is the soul of life itself;
it is the impulse that gives itself reality
through a process of objectification.
Nature, as it reaches this idea
starting from its externality, transcends itself;
its end is not its beginning but is for it
as a limit in which it sublates itself.
Similarly, in the idea of life
the moments of life's reality do not
receive the shape of external actuality
but remain enveloped in conceptual form.

In spirit, however, life appears
both as opposed to it and as posited as at one with it,
in a unity reborn as the pure product of spirit.
For life is here to be taken generally in
its proper sense as natural life,
for what is called the life of spirit as spirit,
is spirit's own peculiar nature
that stands opposed to mere life;
just as we speak of the nature of spirit,
even though spirit is nothing natural
but stands rather in opposition to nature.
Thus life as such is for spirit in one respect a means,
and then spirit holds it over against itself;
in another respect, spirit is an individual,
and then life is its body;
in yet another respect, this unity of spirit
and its living corporeality is
born of spirit into ideality.
None of these connections of life to spirit
concerns logical life,
and life is to be considered here
neither as the instrument of a spirit,
nor as a living body,
nor again as a moment of the ideal and of beauty.
In both cases, as natural life and as referring to spirit,
life obtains a determinateness from its externality,
in one case through its presuppositions,
such as are other formations of nature,
and in the other case through the
purposes and the activity of spirit.
The idea of life by itself is free
from both the conditioning objectivity
presupposed in the first case
and the reference to subjectivity
of the second case.

Life, considered now more closely in its idea,
is in and for itself absolute universality;
the objectivity which it possesses is
throughout permeated by the concept,
and this concept alone it has as substance.
Whatever is distinguished as part,
or by some otherwise external reflection,
has the whole concept within it;
the concept is the soul omnipresent in it,
a soul which is simple self-reference
and remains one in the manifoldness
that accrues to the objective being.
This manifoldness, as self-external objectivity,
has an indifferent subsistence which in space and time,
if these could already be mentioned here,
is a mutual externality of entirely
diverse and atomistic matters.
But externality is in life at the same time
as the simple determinateness of its concept;
thus the soul flows omnipresently in this manifold
but remains at the same time the simple oneness
of the concrete concept with itself.
That way of thinking that clings to the determinations of
reflective relations and of the formal concept,
when it comes to consider life,
the unity of its concept in the externality of objectivity,
the absolute multiplicity of atomistic matter,
finds that all its thoughts are absolutely of no avail;
the omnipresence of the simple in the manifold externality
is for reflection an absolute contradiction
and also, since it cannot at the same time avoid
witnessing this omnipresence in the perception of life
and must therefore grant the actuality of this idea,
an incomprehensible mystery
for reflection does not grasp the concept,
nor does it grasp it as the substance of life.
But this simple life is not only omnipresent;
it is the one and only subsistence
and immanent substance of its objectivity;
but as subjective substance it is impulse,
more precisely the specific impulse of particular difference,
and no less essentially the one and universal impulse
of the specific that leads its particularization
back to unity and holds it there.
Only as this negative unity of its
objectivity and particularization is life self-referring,
life that exists for itself, a soul.
As such, it is essentially a singular that refers to
objectivity as to an other, an inanimate nature.
The originative judgment of life consists therefore in this,
that it separates itself off as individual subject from the objective
and, since it constitutes itself as the negative unity of the concept,
makes the presupposition of an immediate objectivity.

First, life is therefore to be considered as a living individual
that is for itself the subjective totality
and is presupposed as indifferent to an objectivity
that stands indifferent over against it.

Second, it is the life-process of sublating its presupposition,
of positing as negative the objectivity indifferent to it,
and of actualizing itself as the power
and negative unity of this objectivity.
By so doing, it makes itself into the universal
which is the unity of itself and its other.

Third, consequently life is the genus-process,
the process of sublating its singularization
and relating itself to its objective existence
as to itself.
Accordingly, this process is
on the one hand the turning back to its concept
and the repetition of the first forcible separation,
the coming to be of a new individuality
and the death of the immediate first;
but, on the other hand, the withdrawing into itself
of the concept of life is the becoming of
the concept that relates itself to itself,
of the concept that exists for itself,
universal and free, the transition into cognition.

CHAPTER 2

The idea of cognition

Life is the immediate idea, or the idea as
its still internally unrealized concept.
In its judgment, the idea is cognition in general.

The concept is for itself as concept
inasmuch as it freely and concretely exists
as abstract universality or a genus.
As such, it is its pure self-identity
that internally differentiates itself
in such a way that the differentiated is
not an objectivity but is rather
equally liberated into subjectivity
or into the form of simple self-equality;
consequently, the object facing
the concept is the concept itself.
Its reality in general is the form of its existence;
all depends on the determination of this form;
on it rests the difference between
what the concept is in itself, or as subjective,
and what it is when immersed in objectivity,
and then in the idea of life.
In this last, the concept is indeed distinguished
from its external reality and posited for itself;
however, this being-for-itself which it now has,
it has only as an identity that refers to itself
as immersed in the objectivity subjugated to it,
or to itself as indwelling, substantial form.
The elevation of the concept above life consists in this,
that its reality is the concept-form liberated into universality.
Through this judgment the idea is doubled,
into the subjective concept whose reality is the concept itself,
and the objective concept which is as life.
Thought, spirit, self-consciousness, are determinations
of the idea inasmuch as the latter has itself
as the subject matter, and its existence, that is, the determinateness
of its being, is its own difference from itself.

The metaphysics of the spirit or,
as was more commonly said in the past, of the soul,
revolved around the determinations
of substance, simplicity, immateriality.
These were determinations for which
spirit was supposed to be the ground,
but as a subject drawn from empirical consciousness,
and the question then was which predicates
agreed with the perceived facts.
But this was a procedure that could go
no further than the procedure of physics,
which reduces the world of appearance
to general laws and determinations of reflection,
for it is spirit still as phenomenal
that is taken as the foundation.
In fact, in so far as scientific stringency goes,
it also had to fall short of physics.
For not only is spirit infinitely richer than nature;
since its essence is constituted by
the absolute unity in the concept of opposites,
and in its appearance, therefore,
and in its connection with externality,
it exhibits contradiction at its most extreme form,
it must be possible to adduce an experience in support of
each of the opposite determinations of reflection,
or, starting from experiences, to proceed by way
of formal inference to the opposite determinations.
Since the predicates immediately drawn from
the appearances still belong to empirical psychology,
so far as metaphysical consideration goes,
all that is in truth left are
the entirely inadequate determinations of reflection.
In his critique of rational psychology, Kant insists that,
since this metaphysics is supposed to be a rational science,
the least addition of anything drawn from perception
to the universal representation of self-consciousness
would alter it into an empirical science,
thus compromising its rational purity
and its independence from all experience.
Accordingly, all that is left on this view is
the simple representation “I,”
a representation entirely devoid of content,
of which one cannot even say that it is a concept,
but must say that it is a mere consciousness,
one that accompanies every concept.
Now, as Kant argues further, this “I,”
or, if you prefer, this “it” (the thing) that thinks,
takes us no further than the representation
of a transcendental subject of thoughts = x,
a subject which is known only through
the thoughts that are its predicates,
and of which, taken in isolation,
we cannot ever have the least concept.
This “I” has the associated inconvenience that,
as Kant expresses it, in order to judge anything about it,
we must every time already make use of it,
for it is not so much one representation
by which a particular object is distinguished,
as it is rather a form of representation in general,
in so far as representation can be said to be cognition.
Now the paralogism that rational psychology incurs,
as Kant expresses it, consists in this:
that modes of self-consciousness in thinking are
converted into concepts of the understanding,
as if they were the concepts of an object;
that that “I think” is taken to be
a thinking being, a thing-in-itself;
that in this way, because I am present
in consciousness always as a subject,
am indeed as a singular subject,
identical in all the manifoldness of representation,
and distinguishing myself from this manifoldness as external to it,
the illegitimate inference is thereby drawn that I am a substance,
and a qualitatively simple being on top of that,
and a one, and a being that concretely exists
independently of the things of space and time.

I have cited this position in some detail
because one can clearly recognize in it
both the nature of the former metaphysics of the soul
and also, more to the point, of the Critique that put an end to it.
The former was intent on determining the abstract essence of the soul;
it went about this starting from observation,
and then converting the latter's empirical generalizations,
and the determination of purely external reflection
attaching to the singularity of the actual,
into the form of the determinations of essence just cited.
What Kant generally has in mind here is
the state of the metaphysics of his time
which, as a rule, stayed at these one-sided determinations
with no hint of dialectic;
he neither paid attention to, nor examined,
the genuinely speculative ideas of older
philosophers on the concept of spirit.
In his critique of those determinations he then
simply abided by the Humean style of skepticism;
that is to say, he fixes on how the “I” appears in self-consciousness,
but from this “I,” since it is its essence
(the thing in itself) that we want to cognize,
he removes everything empirical;
nothing then remains but this appearance of the “I think”
that accompanies all representations
and of which we do not have the slightest concept.
It must of course be conceded that,
as long as we are not engaged in comprehending
but confine ourselves to a simple, fixed representation or to a name,
we do not have the slightest concept of the “I,”
or of anything whatever, not even of the concept itself.
Peculiar indeed is the thought
(if one can call it a thought at all)
that I must make use of the “I” in order to judge the “I.”
The “I” that makes use of self-consciousness
as a means in order to judge:
this is indeed an x of which,
and also of the relation involved in this “making use,”
we cannot possibly have the slightest concept.
But surely it is laughable to label
the nature of this self-consciousness,
namely that the “I” thinks itself,
that the “I” cannot be thought without the “I” thinking it,
an awkwardness and, as if it were a fallacy, a circle.
The awkwardness, the circle, is in fact the relation
by which the eternal nature of self-consciousness
and of the concept is revealed in
immediate, empirical self-consciousness,
is revealed because self-consciousness is
precisely the existent and therefore
empirically perceivable pure concept;
because it is the absolute self-reference that,
as parting judgment, makes itself into an intended object
and consists in simply making itself thereby into a circle.
This is an awkwardness that a stone does not have.
When it is a matter of thinking or judging,
the stone does not stand in its own way;
it is dispensed from the burden of making
use of itself for the task;
something else outside it must shoulder that effort.

The defect, which these surely barbarous notions place
in the fact that in thinking the “I”
the latter cannot be left out as a subject,
then also appears the other way around,
in that the “I” occurs only as the subject of consciousness,
or in that I can use myself only as a subject,
and no intuition is available by which the “I” would be given as an object;
but the concept of a thing capable of existence only as a subject
does not as yet carry any objective reality with it.
Now if external intuition as determined in
time and space is required for objectivity,
and it is this objectivity that is missed,
it is then clear that by objectivity is meant only sensuous reality.
But to have risen above such a reality is precisely
the condition of thinking and of truth.
Of course, if the “I” is not grasped conceptually
but is taken as a mere representation,
in the way we talk about it in everyday consciousness,
then it is an abstract self-determination,
and not the self-reference that has itself
for its subject matter.
Then it is only one of the extremes,
a one-sided subject without its objectivity;
or else just an object without subjectivity,
which it would be were it not for the awkwardness just touched upon,
namely that the thinking subject will not
be left out of the “I” as object.
But as a matter of fact this awkwardness is
already found in the other determination,
that of the “I” as subject;
the “I” does think something,
whether itself or something else.
This inseparability of the two forms
in which the “I” opposes itself to itself
belongs to the most intimate nature of its concept
and of the concept as such;
it is precisely what Kant wants to keep away
in order to retain what is only a representation
that does not internally differentiate itself
and consequently, of course, is void of concept.
Now this kind of conceptual void may well oppose itself
to the abstract determinations of reflection
or to the categories of the previous metaphysics,
for in one-sidedness it stands at the same level with them,
though these are in fact on a higher level of thought;
but it appears all the more lame and empty
when compared with the profounder ideas
of ancient philosophy concerning
the concept of the soul or of thinking,
as for instance the truly speculative ideas of Aristotle.
If the Kantian philosophy subjected the
categories of reflection to critical investigation,
all the more should it have investigated the abstraction
of the empty “I” that he retained,
the supposed idea of the thing-in-itself.
The experience of the awkwardness complained of is
itself the empirical fact in which
the untruth of that abstraction finds expression.

The Kantian critique of rational psychology
only refers to Mendelssohn's proof of the persistence of the soul,
and I now also cite its refutation of that proof
because of the oddness of what it adduces against it.
Mendelssohn's proof is based on the simplicity of the soul,
by virtue of which it is supposed to be incapable
of alteration in time, of transition into an other.
Qualitative simplicity is in general the
form of abstraction earlier considered;
as qualitative determinateness, it was investigated
in the sphere of being
and it was then proved that the qualitative,
which is as such abstractly self-referring determinateness,
is precisely for that reason dialectical,
mere transition into an other.
In the case of the concept, however, it was shown that,
when considered in connection with persistence,
indestructibility, imperishableness,
it is that which exists for itself,
which is eternal, just because it is
not abstract but concrete simplicity
because it is not a determinateness
that refers to itself abstractly
but is the unity of itself and its other,
and it cannot therefore pass over into this other
as if it thereby altered in it;
it cannot precisely because it is itself the other,
the determinateness, and hence in this
passing over it only comes to itself.
Now the Kantian critique opposes to this
qualitative determination of the unity of the concept
a quantitative one.
As it says, although the soul is not
a manifold of reciprocally external parts
and contains no extensive magnitude,
yet consciousness has a degree,
and the soul, like every concretely existing being,
is an intensive magnitude;
with this magnitude, however, there is posited
the possibility of a transition into nothing through gradual vanishing.
Now what is this refutation but the application to spirit
of a category of being, of intensive magnitude,
a determination that has no truth in itself
but on the contrary is sublated in the concept?

Metaphysics even one that restricted itself
to the fixed concepts of the understanding
without rising to speculation,
to the nature of the concept and of the idea,
did have for its aim the cognition of truth;
it did probe its subject matter to ascertain
whether they were something true or not,
whether substances or phenomena.
The triumph of the Kantian critique
over this metaphysics consists, on the contrary,
in side-lining any investigation
that would have truth for its aim and this aim itself;
it simply does not pose the one question
which is of interest,
namely whether a determinate subject,
in this case the abstract “I” of representation,
has truth in and for itself.
But to stay at appearances
and at the mere representations
of ordinary consciousness is to give up
on the concept and on philosophy.
Anything beyond that is branded
by the Kantian critique as high-flown,
something to which reason has no claim.
As a matter of fact, the concept does fly high,
rising above what has no concept,
and the immediate justification
for going beyond it is, for one thing, the concept itself,
and, for another, on the negative side,
the untruth of appearance and of representation,
and also of such abstractions as the thing-in-itself
and the said “I” which is not supposed to be an object to itself.

In the context of this logical exposition,
it is from the idea of life that
the idea of spirit has emerged,
or what is the same thing, that has demonstrated
itself to be the truth of the idea of life.
As this result, the idea possesses its truth in and for itself,
with which one may then also compare the empirical reality
or the appearance of spirit to see how far it accords with it.
We have seen regarding life that it is the idea,
but at the same time it has shown itself
not to be as yet the true presentation
or the true mode of its existence.
For in life, the reality of the idea is singularity;
universality or the genus is the inwardness.
The truth of life as absolute negative unity consists,
therefore, in this:
to sublate the abstract or, what is the same,
the immediate singularity,
and as identical to be self-identical,
as genus, to be self-equal.
Now this idea is spirit.
In this connection, we may further remark
that spirit is here considered in the form
that pertains to this idea as logical.
For the idea also has other shapes
which we may now mention in passing;
in these it falls to the concrete sciences
of spirit to consider it, namely
as soul, consciousness, and spirit as such.

The name “soul” was used formerly to mean
singular finite spirit in general,
and rational or empirical psychology was
supposed to be synonymous with doctrine of spirit.
The expression, “soul,” evokes an image of it
as if it were a thing like other things.
One enquires regarding its seat,
the spatial location from which its forces operate;
still more, how this thing can be imperishable,
subjected to the conditions of temporality yet exempt
from alteration in it.
The system of monads elevates matter by making all
of it in principle a soul;
on this way of representing it, the soul is an
atom like the atoms of matter;
the atom that rises from a cup of coffee as vapor
is capable in favorable circumstances of developing into a soul;
only the greater obscurity of its ideation distinguishes it
from the kind of thing that is manifestly soul.
The concept that is for itself is
necessarily also in immediate existence;
in this substantial identity with life,
immersed in its externality,
the concept is the subject matter of anthropology.
But even anthropology would find alien a metaphysics
in which this form of immediacy is made into a soul-thing,
into an atom like the atoms of matter.
To anthropology must be left only that obscure region where spirit,
under influences which were once called sidereal and terrestrial,
lives as a natural spirit in sympathy with nature
and has presentiments of the latter's alterations
in dreams and presentiments,
and indwells the brain, the heart, the liver, and so forth.
To the liver, according to Plato,
God gave the gift of prophesy
above which the self-conscious human is exalted,
so that even the irrational part of the soul
would be provided for by his bounty
and made to share in higher things.
To this irrational side belongs further the
behavior of figurative representation,
and of higher spiritual activity in so far
as the latter is subject to the play
of an entirely corporeal constitution,
of external influences and particular circumstances.

This lowest of the concrete shapes
in which spirit is sunk into materiality has
the one immediately superior to it in consciousness.
In this form the free concept,
as the “I” existing for itself,
is withdrawn from objectivity,
but it refers to the latter as its other,
a subject matter that confronts it.
Since spirit is here no longer as soul,
but, in the certainty that it has of itself,
the immediacy of being has for it the significance
rather of a negative, its identity with itself
in the objectivity confronting it is at
the same time still only a reflective shining,
for that objectivity still also has the
form of a being that exists in itself.
This stage is the subject matter
of the phenomenology of spirit,
a science that stands midway between the science
of the natural spirit and of the spirit as such.
It considers spirit as it exists for itself,
but at the same time as referring to its other,
an other which, as we have just said, is
thereby determined both as an object
existing in itself and as a negative.
The science thus considers spirit as appearing,
as exhibiting itself in its contrary.

But the higher truth of this form is spirit for itself.
For this spirit, the subject matter which for consciousness
exists in itself has the form of its own determination,
the form of representation in general;
this spirit, which acts on the subject matter's determinations
as on its own, on feelings, on representations and thought,
is thus infinite in itself and in its form.
The consideration of this stage belongs
to the doctrine of spirit proper,
which would embrace the subject matter
of ordinary empirical psychology
but which, in order to be the science of spirit,
must not go about its work empirically
but must be conceived scientifically.
At this stage spirit is finite spirit in so far as
the content of its determinateness is an immediately given content;
the science of this finite spirit has
to display the course along which
it liberates itself from this determinateness
and goes on to grasp its truth, the infinite spirit.

The idea of spirit which is the subject matter of logic
already stands, on the contrary, inside pure science;
it has no need, therefore, to observe spirit's
tracing that course, to see how it gets entangled
with nature, with immediate determinateness, with matter,
or in other words with pictorial representation;
this is what the other three sciences investigate.
The idea of spirit has this course already behind it,
or what is the same, it has it rather ahead of it
behind in so far as logic is taken as the final science;
ahead in so far as it is taken as the first science
from which the idea first passes over into nature.
In the logical idea of spirit, therefore,
the “I” is from the start in the way it has emerged
from the concept of nature as the truth of nature,
the free concept which in its judgment is itself the subject matter
confronting it, the concept as its idea.
Also in this shape, however, the idea is still not consummated.

Although the idea is indeed the free concept
that has itself as its subject matter,
it is nonetheless immediate,
and just because it is immediate,
it is still the idea in its subjectivity,
and hence in its finitude in general.
It is the purpose that ought to realize itself,
or the absolute idea itself still in its appearance.
What the idea seeks is the truth,
this identity of the concept itself and reality;
but at first it only seeks it;
for it is here as it is at first,
still something subjective.
Consequently, although the subject matter
that is for the concept is here also a given subject matter,
it does not enter into the subject as affecting it,
or as confronting it with a constitution of its own
as subject matter, or as a pictorial representation;
on the contrary, the subject transforms it
into a conceptual determination;
it is the concept which is the active principle in it
which therein refers itself to itself,
and, by thus giving itself its reality in the object, finds truth.

Initially, therefore, the idea is one extreme of a syllogism,
the concept that as purpose has itself
at first for its subjective reality;
the other extreme is the restriction
of the subjective, the objective world.
The two extremes are identical in that they are the idea.
Their unity is, first, that of the concept,
a unity which in the one extreme is only for itself
and in the other only in itself.
Second, it is reality, abstract in the one extreme
and in the other in its concrete externality.
This unity is now posited through cognition,
and, because the latter is the subjective idea
which as purpose proceeds from itself,
it is at first only a middle term.
The knowing subject, through the
determinateness of its concept
which is the abstract being-for-itself,
refers to an external world;
nevertheless, it does this in
the absolute certainty of itself,
in order to elevate its implicit reality,
this formal truth, to real truth.
It has the entire essentiality of
the objective world in its concept;
its process consists in positing for itself
the concrete reality of that world
as identical with the concept,
and conversely in positing the latter
as identical with objectivity.

Immediately, the idea of appearance is
the theoretical idea, cognition as such.
For to the concept that exists for itself,
the objective world immediately has
the form of immediacy or of being,
just as that concept is to itself
at first only the abstract concept of itself,
is still shut up within itself.
The concept is therefore only as form,
of which only its simple determinations
of universality and particularity are
the reality that it possesses within,
while the singularity or the determinate determinateness,
the content, is received by it from the outside.

CHAPTER 3

The absolute idea

The absolute idea has shown itself to be
the identity of the theoretical and the practical idea,
each of which, of itself still one-sided, possesses the idea
only as a sought-for beyond and unattained goal;
each is therefore a synthesis of striving,
each possessing as well as not possessing the idea within it,
passing over from one thought to the other
without bringing the two together
but remaining fixed in the contradiction of the two.
The absolute idea, as the rational concept
that in its reality only rejoins itself,
is by virtue of this immediacy of its objective identity,
on the one hand, a turning back to life;
on the other hand, it has equally
sublated this form of its immediacy
and harbors the most extreme opposition within.
The concept is not only soul,
but free subjective concept
that exists for itself
and therefore has personality,
the practical objective concept
that is determined in and for itself
and is as person impenetrable, atomic subjectivity
but which is not, just the same, exclusive singularity;
it is rather explicitly universality and cognition,
and in its other has its own objectivity for its subject matter.
All the rest is error, confusion, opinion,
striving, arbitrariness, and transitoriness;
the absolute idea alone is being, imperishable life,
self-knowing truth, and is all truth.

It is the sole subject matter and content of philosophy.
Since it contains all determinateness within it,
and its essence consists in returning
through its self-determination
and particularization back to itself,
it has various shapes,
and the business of philosophy
is to recognize it in these.
Nature and spirit are in general
different modes of exhibiting its existence,
art and religion its different modes of apprehending itself
and giving itself appropriate existence.
Philosophy has the same content and the same
purpose as art and religion,
but it is the highest mode of apprehending the absolute idea,
because its mode, that of the concept, is the highest.
Hence it seizes those shapes
of real and ideal finitude,
as well of infinity and holiness,
and comprehends them and itself.
The derivation and cognition of these particular modes are
now the further business of the particular philosophical sciences.
Also the logicality of the absolute idea
can be called a mode of it;
but mode signifies a particular kind,
a determinateness of form,
whereas the logicality of the idea is
the universal mode in which
all particular modes are sublated and enveloped.
The logical idea is the idea itself in its pure essence,
the idea which is enclosed in
simple identity within its concept
and in reflective shining has as yet
to step into a form-determinateness.
The Logic thus exhibits the
self-movement of the absolute idea
only as the original word,
a word which is an utterance,
but one that in being externally uttered
has immediately vanished again.
The idea is, therefore, only in this
self-determination of apprehending itself;
it is in pure thought, where difference is not yet otherness,
but is and remains perfectly transparent to itself.
The logical idea thus has itself,
as the infinite form, for its content,
form that constitutes the opposite of content
inasmuch as the latter is the form determination
that has withdrawn into itself
and has been so sublated in identity
that this concrete identity stands over against
the identity developed as form;
the content has the shape of an other
and of something given as against the form
that as such stands simply in reference,
and whose determinateness is posited
at the same time as reflective shine.
More exactly, the absolute idea itself has
only this for its content,
namely that the form determination is
its own completed totality, the pure content.
Now the determinateness of the idea
and the entire course traversed by this determinateness
has constituted the subject matter of the science of logic,
and out of this course the absolute idea has come forth for itself;
thus to be for itself, however, has shown itself to amount to this,
namely that determinateness does not
have the shape of a content,
but that it is simply as form,
and that accordingly the idea is
the absolutely universal idea.
What is left to be considered here, therefore,
is thus not a content as such,
but the universal character of its form
that is, method.

Method may appear at first to be just
the manner in which cognition proceeds,
and this is in fact its nature.
But as method this manner of proceeding is
not only a modality of being determined in and for itself;
it is a modality of cognition,
and as such is posited as determined
by the concept and as form,
since form is the soul of all objectivity
and all otherwise determined content has its truth in form alone.
If the content is again assumed as given to the method
and of a nature of its own, then method, so understood,
is just like the logical realm in general,
a merely external form.
But against this assumption appeal can be made,
not only to the fundamental concept of what constitutes logic,
but to the entire logical course
in which all the shapes of a given content
and of objects came up for consideration.
This course has shown the transitoriness
and the untruth of all such shapes;
also that no given object is capable of being the foundation
to which the absolute form would relate
as only an external and accidental determination;
that, on the contrary, it is the absolute form
that has proved itself to be the absolute foundation
and the ultimate truth.
For this course the method has resulted as
the absolutely self-knowing concept,
as the concept that has the absolute,
both as subjective and objective,
as its subject matter,
and consequently as the pure correspondence
of the concept and its reality,
a concrete existence that is the concept itself.

Accordingly, what is to be considered as method here is
only the movement of the concept itself.
We already know the nature of this movement,
but it now has, first, the added significance
that the concept is all,
and that its movement is the universal absolute activity,
the self-determining and self-realizing movement.
The method is therefore to be acknowledged as the universal,
internal and external mode, free of restrictions,
and as the absolutely infinite force to which
no object that may present itself as something external,
removed from reason and independent of it,
could offer resistance,
or be of a particular nature opposite to it,
and could not be penetrated by it.
It is therefore soul and substance,
and nothing is conceived and known in its truth
unless completely subjugated to the method;
it is the method proper to each and every fact
because its activity is the concept.
This is also the truer meaning of its universality;
according to the universality of reflection,
it is taken only as the method for all things;
but according to the universality of the idea,
it is both the manner of cognition,
of the concept subjectively aware of itself,
and the objective manner, or rather the
substantiality of things, that is,
of concepts as they first appear as others
to representation and reflection.
It is therefore not only the highest force of reason,
or rather its sole and absolute force,
but also reason's highest and sole impulse
to find and recognize itself through itself in all things.
Second, here we also have the distinction of
the method from the concept as such,
the particularization of the method.
As the concept was considered for itself,
it appeared in its immediacy;
the reflection, or the concept considering it,
fell on the side of our knowledge.
The method is this knowledge itself,
for which the concept is not only as subject matter
but is as its own subjective act,
the instrument and the means of cognitive activity,
distinct from this activity
and yet the activity's own essentiality.
In cognition as enquiry,
the method likewise occupies the position of an instrument,
as a means that stands on the side of the subject,
connecting it with the object.
The subject in this syllogism is one extreme,
the object is the other,
and in conclusion the subject unites
through its method with the object
without however uniting with itself there.
The extremes remain diverse,
because subject, method, and object are not
posited as the one identical concept;
the syllogism is therefore always the formal syllogism;
the premise in which the subject posits the form
on its side as its method is an immediate determination
and contains therefore the determinations of the form
as we have seen, of definition, division, and so forth
as matters of fact found ready-made in the subject.
In true cognition, on the contrary, method is
not only an aggregate of certain determinations,
but the determinateness in-and-for-itself of the concept,
and the concept is the middle term only
because it equally has the significance of the objective;
in the conclusion, therefore, the objective does not attain
only an external determinateness by virtue of the method,
but is posited rather in its identity with the subjective concept.

In conclusion, there remains only this to be said of this idea,
that in it, in the first place,
the science of logic has apprehended its own concept.
In the sphere of being, at the beginning of its content,
its concept appears as a knowledge external to
that content in subjective reflection.
But in the idea of absolute cognition,
the concept has become the idea's own content.
The idea is itself the pure concept
that has itself as its subject matter
and which, as it runs itself as subject matter
through the totality of its determinations,
builds itself up to the entirety of its reality,
to the system of science,
and concludes by apprehending this
conceptual comprehension of itself,
hence by sublating its position
as content and subject matter
and cognizing the concept of science.
In second place, this idea is still logical;
it is shut up in pure thought,
the science only of the divine concept.
Its systematic exposition is of course itself a realization,
but one confined within the same sphere.
Because the pure idea of cognition is
to this extent shut up within subjectivity,
it is the impulse to sublate it,
and pure truth becomes as final result
also the beginning of another sphere and science.
It only remains here to indicate this transition.

The idea, namely, in positing itself
as the absolute unity of the pure concept and its reality
and thus collecting itself in the immediacy of being,
is in this form as totality:  nature.
This determination, however, is nothing that has become,
is not a transition, as was the case above
when the subjective concept in its totality becomes objectivity,
or the subjective purpose becomes life.
The pure idea into which the determinateness
or reality of the concept is itself
raised into concept is rather an
absolute liberation for which
there is no longer an immediate determination
which is not equally posited and is not concept;
in this freedom, therefore, there is
no transition that takes place;
the simple being to which the idea determines itself
remains perfectly transparent to it:
it is the idea that in its determination remains with itself.
The transition is to be grasped, therefore,
in the sense that the idea freely discharges itself,
absolutely certain of itself and internally at rest.
On account of this freedom, the form of
its determinateness is just as absolutely free:
the externality of space and time absolutely existing
for itself without subjectivity.
Inasmuch as this externality is only
in the abstract determinateness of being
and is apprehended by consciousness,
it is as mere objectivity and external life;
within the idea, however, it remains in
and for itself the totality of the concept,
and science in the relation of
divine cognition to nature.
But what is posited by this first resolve
of the pure idea to determine itself as external idea
is only the mediation out of which the concept,
as free concrete existence that from externality
has come to itself, raises itself up,
completes this self-liberation in the science of spirit,
and in the science of logic finds the highest concept of itself,
the pure concept conceptually comprehending itself.
