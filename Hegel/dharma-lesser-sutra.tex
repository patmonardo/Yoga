

C. ACTUALITY

IV.21
citer apratisamkramayas tad-akara-apattau svabuddhi-samvedanam

IV.22
drastr-drsya-uparaktam cittam sarva-artham

IV.23
tad asamkhyeya-vasanabhi citram api para-artham samhatya-karitvat

IV.24
visesa-darsina atma-bhava-bhavana-vinivrtti

IV.25
tada viveka-nimnam kaivalya-prag-bharam cittam

IV.26
tad-chidresu pratyaya-antarani samskarebhya

IV.27
hanam esam klesavad uktam

IV.28
prasankhyane api-akusidasya sarvatha viveka-khyater dharma-megha samadhi

IV.29
tata klesa-karma-nivrtti

IV.30
tada sarvavarana-malapetasya jnaanasyanantyaj jnaeyam alpam

IV.31
tata-krta-arthanam parinama-krama-samaptir gunanam

IV.32
ksana-pratiyogi parinama-aparanta-nirgrahya krama

IV.33
purusa-artha-sunyanam gunanam pratiprasava kaivalyam
svarupa-pratistha va citi-sakti iti

BOOK TWO

The Doctrine of Essence

ESSENCE

The truth of being is essence.

Being is the immediate.
Since the goal of knowledge is the truth,
what being is in and for itself,
knowledge does not stop at the immediate and its determinations,
but penetrates beyond it on the presupposition that
behind this being there still is something other than being itself,
and that this background constitutes the truth of being.
This cognition is a mediated knowledge,
for it is not to be found with and in essence immediately,
but starts off from an other, from being,
and has a prior way to make,
the way that leads over and beyond being
or that rather penetrates into it.
Only inasmuch as knowledge recollects
itself into itself out of immediate being,
does it find essence through this mediation.
The German language has kept “essence” (Wesen)
in the past participle (gewesen) of the verb “to be” (sein),
for essence is past [but timelessly past] being.

When this movement is represented as a pathway of knowledge,
this beginning with being and the subsequent advance
which sublates being and arrives at essence as a mediated term
appears to be an activity of cognition external to being
and indifferent to its nature.

But this course is the movement of being itself.
That it is being's nature to recollect itself,
and that it becomes essence by virtue of this interiorizing,
this has been displayed in being itself.

If, therefore, the absolute was at first determined as being,
now it is determined as essence.
Cognition cannot in general stop at the manifold of existence;
but neither can it stop at being, pure being;
immediately one is forced to the reflection that
this pure being, this negation of everything finite,
presupposes a recollection and a movement
which has distilled immediate existence into pure being.
Being thus comes to be determined as essence,
as a being in which everything determined and finite is negated.
So it is simple unity, void of determination,
from which the determinate has been removed in an external manner;
to this unity the determinate was itself something external
and, after this removal,
it still remains opposite to it;
for it has not been sublated in itself but relatively,
only with reference to this unity.
We already noted above that if pure essence is defined
as the sum total of all realities,
these realities are equally subject to
the nature of determinateness and abstractive reflection
and their sum total is reduced to empty simplicity.
Thus defined, essence is only a product, an artifact.
External reflection, which is abstraction, only lifts
the determinacies of being out of what is left over as essence
and only deposits them, as it were, somewhere else,
letting them exist as before.
In this way, however, essence is neither in itself nor for itself;
it is by virtue of another, through external abstractive reflection;
and it is for another, namely for abstraction
and in general for the existent
which still remains opposite to it.
In its determination, therefore,
it is a dead and empty absence of determinateness.

As it has come to be here, however,
essence is what it is,
not through a negativity foreign to it,
but through one which is its own:
the infinite movement of being.
It is being-in-and-for-itself,
absolute in-itselfness;
since it is indifferent to every determinateness of being,
otherness and reference to other have been sublated.
But neither is it only this in-itselfness;
as merely being-in-itself, it would be only the abstraction of pure essence;
but it is being-for-itself just as essentially;
it is itself this negativity,
the self-sublation of otherness and of determinateness.

Essence, as the complete turning back of being into itself,
is thus at first the indeterminate essence;
the determinacies of being are sublated in it;
it holds them in itself but without their being posited in it.
Absolute essence in this simple unity with itself has no existence.
But it must pass over into existence,
for it is being-in-and-for-itself;
that is to say, it differentiates
the determinations which it holds in itself,
and, since it is the repelling of itself from itself
or indifference towards itself, negative self-reference,
it thereby posits itself over against itself
and is infinite being-for-itself
only in so far as in thus
differentiating itself from itself
it is in unity with itself.
This determining is thus of another nature than
the determining in the sphere of being,
and the determinations of essence have another character
than the determinations of being.
Essence is absolute unity of being-in-itself and being-for-itself;
consequently, its determining remains inside this unity;
it is neither a becoming nor a passing over,
just as the determinations themselves are
neither an other as other nor references to some other;
they are self-subsisting but, as such,
at the same time conjoined in the unity of essence.
Since essence is at first simple negativity,
in order to give itself existence and then being-for-itself,
it must now posit in its sphere the determinateness
which it contains in principle only in itself.

Essence is in the whole what quality was in the sphere of being:
absolute indifference with respect to limit.
Quantity is instead this indifference in immediate determination,
limit being in it an immediate external determinateness;
quantity passes over into quantum;
the external limit is necessary to it and exists in it.
In essence, by contrast, the determinateness does not exist;
it is posited only by the essence itself,
not free but only with reference to
the unity of the essence.
The negativity of essence is reflection,
and the determinations are reflected,
posited by the essence itself
in which they remain as sublated.

Essence stands between being and concept;
it makes up their middle,
its movement constituting the transition
of being into the concept.
Essence is being-in-and-for-itself,
but it is this in the determination of being-in-itself;
for its general determination is that it emerges from being
or that it is the first negation of being.
Its movement consists in positing negation
or determination in being, thereby giving
itself existence and becoming
as infinite being-for-itself what it is in itself.
It thus gives itself its existence
which is equal to its being-in-itself
and becomes concept.
For the concept is the absolute as it is absolutely,
or in and for itself, in its existence.
But the existence which essence gives to itself is
not yet existence as it is in and for itself
but as essence gives it to itself or as posited,
and hence still distinct from the existence of the concept.

First, essence shines within itself or is reflection;
second, it appears;
third, it reveals itself.

In the course of its movement,
it posits itself in the following determinations:

I. As simple essence existing in itself,
remaining in itself in its determinations;

II. As emerging into existence,
or according to its concrete existence and appearance;

III. As essence which is one with its appearance,
as actuality.

SECTION I

Essence as Reflection Within

IV.1
jati-antara-parinama prakrti-apurat

Essence issues from being;
hence it is not immediately in and for itself
but is a result of that movement.
Or, since essence is taken at first as something immediate,
it is a determinate existence to which another stands opposed;
it is only essential existence, as against the unessential.
But essence is being which has been sublated in and for itself;
what stands over against it is only shine.
The shine, however, is essence's own positing.

First, essence is reflection.
Reflection determines itself;
its determinations are a positedness
which is immanent reflection at the same time.
Second, these reflective determinations
or essentialities are to be considered.
Third, as the reflection of its immanent determining,
essence turns into foundation and passes over
into concrete existence and appearance.

CHAPTER 1

Shine

IV.2
nimittam aprayojakam prakrtinam varana-bhedas tu tata ksetrikavat

As it issues from being, essence seems to stand over against it;
this immediate being is, first, the unessential.

IV.3
nirmana-cittani-asmita-matra

But, second, it is more than just the unessential;
it is being void of essence; it is shine.

IV.4
pravrtti-bhede prayojakam cittam ekam anekesam

Third, this shine is not something external,
something other than essence, but is essence's own shining.
This shining of essence within it is reflection.

CHAPTER 2

Foundation

IV.5
tatra dhyana-jam anasayam

The essentialities or the determinations of reflection

Reflection is determined reflection;
accordingly, essence is determined essence, or it is essentiality.

Reflection is the shining of essence within itself.

Essence, as infinite immanent turning back is
not immediate simplicity, but negative simplicity;
it is a movement across moments that are distinct,
is absolute mediation with itself.
But in these moments it shines;
the moments are, therefore, themselves
determinations reflected into themselves.

First, essence is simple self-reference, pure identity.
This is its determination, one by which it is rather
the absence of determination.

Second, the specifying determination is difference,
difference which is either external or indefinite,
diversity in general, or opposed diversity or opposition.

Third, as contradiction this opposition is reflected into itself
and returns to its foundation.

CHAPTER 3

Ground

Essence determines itself as ground.

Just as nothing is at first in simple immediate unity with being,
so here too the simple identity of essence is at first
in simple unity with its absolute negativity.
Essence is only this negativity which is pure reflection.
It is this pure reflection as the turning back of being into itself;
hence it is determined, in itself or for us,
as the ground into which being resolves itself.
But this determinateness is not posited by the essence itself;
in other words, essence is not ground precisely because
it has not itself posited this determinateness that it possesses.
Its reflection, however, consists in positing itself as
what it is in itself, as a negative, and in determining itself.
The positive and the negative constitute the essential determination
in which essence is lost in its negation.
These self-subsisting determinations of reflection sublate themselves,
and the determination that has foundered to the ground is
the true determination of essence.

Consequently, ground is itself one of
the reflected determinations of essence,
but it is the last, or rather,
it is determination determined as sublated determination.
In foundering to the ground, the determination of reflection
receives its true meaning that it is the absolute
repelling of itself within itself;
or again, that the positedness that accrues to essence is
such only as sublated,
and conversely that only the self-sublating positedness is
the positedness of essence.
In determining itself as ground,
essence determines itself as the not-determined,
and only the sublating of its being determined is its determining.
Essence, in thus being determined as self-sublating,
does not proceed from an other but is,
in its negativity, identical with itself.

Since the advance to the ground is made starting
from determination as an immediate first
(is done by virtue of the nature of determination itself
that founders to the ground through itself),
the ground is at first determined by that immediate first.
But this determining is, on the one hand,
as the sublating of the determining,
the merely restored, purified or manifested identity of essence
which the determination of reflection is in itself;
on the other hand, this negating movement is, as determining,
the first positing of that reflective determinateness
that appeared as immediate determinateness,
but which is posited only by the self-excluding reflection of ground
and therein is posited as only something posited or sublated.
Thus essence, in determining itself as ground, proceeds only from itself.
As ground, therefore, it posits itself as essence,
and its determining consists in just this positing of itself as essence.
This positing is the reflection of essence
that sublates itself in its determining;
on that side is a positing, on this side is the positing of essence,
hence both in one act.

Reflection is pure mediation in general;
ground, the real mediation of essence with itself.
The former, the movement of nothing through nothing back to itself,
is the reflective shining of one in an other;
but, because in this reflection opposition does not
yet have any self-subsistence,
neither is the one, that which shines, something positive,
nor is the other in which it reflectively shines something negative.
Both are substrates, actually of the imagination;
they are still not self-referring.
Pure mediation is only pure reference,
without anything being referred to.
Determining reflection, for its part, does posit
such terms as are identical with themselves;
but these are at the same time only determined references.
Ground, on the contrary, is mediation that is real,
since it contains reflection as sublated reflection;
it is essence that turns back into itself
through its non-being and posits itself.
According to this moment of sublated reflection,
what is posited receives the determination of immediacy,
of an immediate which is self-identical
outside its reference or its reflective shining.
This immediacy is being as restored by essence,
the non-being of reflection through which essence mediates itself.
Essence returns into itself as it negates;
therefore, in its turning back into itself,
it gives itself the determinateness that precisely
for this reason is the self-identical negative,
is sublated positedness, and consequently,
as the self-identity of essence as ground,
equally an existent.

IV.6
karma-asukla-akrsnam yogina trividham itaresam

IV.7
tatas tad-vipaka-anugunanam eva-abhivyakti vasananam

The ground is, first, absolute ground
one in which the essence is first of all
the general substrate for the ground-connection.
It then further determines itself as form and matter
and gives itself a content.

IV.8
jati-desa-kala vyavahitanam api-anantaryam smrti-samskarayo eka-rupatvat

IV.9
tasam anaditvam ca-asisa nityatvat

Second, it is determinate ground,
the ground of a determinate content.
Because the ground-connection, in being realized,
becomes as such external,
it passes over into conditioning mediation.

IV.10
hetu-phala-asraya-alambana sangrhitatvad esam abhave tad-abhava

Third, ground presupposes a condition;
but the condition equally presupposes the ground;
the unconditioned is the unity of the two,
the fact itself that, by virtue of
the mediation of the conditioning reference,
passes over into concrete existence.

SECTION II

Appearance

The essence must appear.

IV.11
atita-anagatam svarupato 'styadhva-bhedad dharmanam

IV.12
te vyakta-suksma guna-atmana

IV.13
parinama-ekatvad vastu-tattvam

IV.14
vastu-samye citta-bhedat tayo vibhakta pantha

IV.15
na ca-eka-citta-tantram vastu tad apramanakam tada kim syat

IV.16
tad-uparaga-apeksitvat-cittasya vastu jnata-ajnatam

IV.17
sada jnata citta-vrttaya tat-prabho purusasya-aparinamitvat

IV.18
na tat sva-abhasam drsyatvat

IV.19
eka-samaye ca-ubhaya-anavadharanam

IV.20
citta-antara-drsye buddhi-buddher atiprasanga smrti-sankara ca

Its shining within itself is
the sublating itself and
becoming an immediacy which,
as reflection-in-itself, is
as much a subsisting (matter)
as it is form, reflection-in-another,
subsisting in the process of sublating itself.

Its shining is the determination through which
the essence is not being but essence,
and the shining, once developed, is the appearance.
The essence is thus not behind or beyond the appearance;
instead, by virtue of the fact that it is
the essence that exists concretely,
concrete existence is appearance.

a. The world of appearance

What appears concretely exists in such a way
that its subsisting is immediately sublated;
it is only one moment of the form itself.
The form encompasses in itself the subsisting
or the matter as one of its determinations.
What appears thus has its ground in the form
as its essence, its reflection-in-itself
as opposed to its immediacy, but thereby has it
only in another determinacy of the form.
This, its ground, is just as much something appearing,
and thus the appearance continues on to an infinite mediation
of the subsisting through the form
and thus equally through not subsisting.
This infinite mediation is at once
a unity of relation-to-itself,
and concrete existence develops into a totality
and world of appearance, of reflected finitude.

b. Content and form

The manner of being-outside-one-another that is
characteristic of the world of appearances is
a totality and completely contained in its relation-to-itself.
The relation of the appearance to itself is
thus completely determined, has the form in itself
and because [it is] in this identity,
has that form as its essential subsistence.
Thus the form is content and,
in keeping with its developed determinacy,
it is the law of the appearance.
The negative side of the appearance,
what is alterable and not self-sufficient,
falls to the form as not reflected in itself;
it is the indifferent, external form.

    For the contrast of form and content,
    it is essential to keep in mind that
    the content is not formless but
    instead has the form within itself
    just as much as it [the form] is
    something external to it.

    A doubling of the form presents itself;
    at one time,
    insofar as it is reflected in itself,
    it is the content and,
    at another time,
    as not reflected in itself,
    it is the external concrete existence,
    indifferent to the content.

    What presents itself here in itself is
    the absolute relation of content and of form, namely,
    their turning over and into one another,
    so that the content is nothing but the form turning into content
    and the form nothing other than the content turning into the form.
    This 'turning over' is one of the most important determinations.
    It is posited, however, only in the absolute relationship.

The immediate concrete existence, however, is
the determinacy of the subsisting itself as well as of the form;
it is thus just as much external to the determinacy of the content
as this externality, which it has through the element of its subsisting,
is essential to it.
The appearance, so posited, is the relationship
such that one and the same, [namely] the content, is
as the developed form, as the externality and opposition of
self-standing concrete existences and their identical relation,
the relation in which alone the differentiated elements are what they are.

c. The relationship

(a) The immediate relationship is that of the whole and the parts:
the content is the whole and consists of the parts (the form),
the opposite of it.
The parts are diverse from one another
and are what is self-standing.
But they are only parts in their identical relation to one another
or insofar as, taken together, they make up the whole.
But that 'together' is the opposite and negation of the part.

(b) What is one and the same in this relationship
(the relation to itself that is on hand in it)
is thus an immediately negative relation to itself and,
to be sure, as the mediation to the effect that one and the same
is indifferent to the difference, and that
it is the negative relation to itself that repels itself,
as reflection-in-itself, towards the difference, and posits itself,
concretely existing as reflection-into-another and, in reverse direction,
conducts this reflection-into-another back to
the relation to itself and to the indifference:
the force and its expression.

    The relationship of the whole and the parts is
    the immediate relationship;
    hence, the thoughtless relationship
    and the process of the identity-with-itself
    turning over into diversity.
    There is a passage from the whole to the parts and
    from the parts to the whole,
    and in the one [the whole or the part]
    the opposition to the other is forgotten since
    each is taken as a self-standing concrete existence,
    the one time the whole, the other time the parts.
    Or since the parts are supposed to subsist in the whole
    and the whole to consist of the part one time the one,
    the other time the other is the subsisting and
    the other is each time the unessential.
    The mechanical relationship,
    in its superficial form,
    consists generally in the fact
    that the parts are taken as self-sufficient
    opposite one another and opposite the whole.

    The infinite progression that concerns
    the divisibility of matter
    can avail itself of this relationship too,
    and then it is the thoughtless oscillation
    of both sides of the relationship.
    A thing is taken one time as a whole,
    then there is a passage to the determination of it as a part,
    this determination is then forgotten
    and what was a part is limited and thus acquires
    its determinacy by means of an other outside it
    regarded as a whole;
    the determination of it as a part resurfaces
    and so on, ad infinitum.
    Taken as the negative that it is, however,
    this infinity is the negative relation of
    the relationship to itself, the force,
    the whole that is identical with itself as being-in-itself,
    and as this being-in-itself sublating itself and expressing itself
    and, conversely, the expression that disappears
    and goes back into the force.

    This infinity notwithstanding,
    the force is also finite.
    For the content, the one and the same
    that the force and the expression are, is
    initially this identity only in itself.
    The two sides of the relationship
    are not yet themselves,
    each for itself its concrete identity,
    not yet the totality.
    In relation to one another, they are thus diverse
    and the relationship is a finite one.
    The force is thus in need of solicitation from without;
    it acts blindly, and, thanks to this deficiency of the form,
    the content is also limited and contingent.
    It is not yet truly identical with the form,
    is not yet the concept and purpose
    that is the determinate in and for itself.
    This difference is supremely essential,
    but not easy to grasp;
    it has to be determined more precisely and
    only in terms of the concept of purpose.
    If it is overlooked, this leads to
    the confusion of construing God as force,
    a confusion from which Herder's God suffers especially.

    It is usually said that the nature of force itself is unknown
    and only its expression is known.
    On the one hand, the entire determination of the content of force is
    just the same as that of the expression;
    on account of this, the explanation of a phenomenon
    on the basis of a force is an empty tautology.
    What is supposed to
    remain unknown is therefore in fact nothing but the empty form
    of the reflection-in-itself, by means of which alone the force is
    distinguished from the expression, a form that is equally
    something well known.
    This form adds nothing in the slightest to the content and to the law,
    which are supposed to be known simply on the basis of the phenomenon alone.
    Assurances are also given everywhere that, with this, nothing is
    supposed to be claimed about the force; as a result, it is impossible to
    see why the form of force has been introduced into the sciences.
    But, on the other hand, the nature of force is, of course, something
    unknown since the necessity of the connection of its content is still
    lacking, not only in itself but also and equally insofar as it is for itself
    limited and thus acquires its determinacy by means of an other outside it.

As the whole that is, in its very self,
the negative relation to itself, force is this:
the process of repelling itself from itself
and expressing itself.
But since this reflection-in-another,
the difference of the parts,
is just as much a reflection-in-itself,
the expression is the mediation by means of which
the force that returns into itself is force.
Its expression is itself the sublating of
the diversity on both sides,
which is on hand in this relationship,
and the positing of the identity
that in itself makes up the content.
Its truth is, for that reason, the relationship,
the two sides of which are distinguished only as inner and outer.

(c) The inner is the ground as the mere form
of the one side of the appearance and the relationship,
the empty form of the reflection-in-itself.
Standing opposite it is concrete existence as the form likewise
of the other side of the relationship,
with the empty determination of the
reflection-in-another as outer.
Their identity is the fulfilled identity, the content,
the unity of the reflection-in-itself and
the reflection-in-another,
posited in the movement of force.
Both are the same, one totality,
and this unity makes them into the content.

The outer is thus, in the first place, the same content as the inner is.
What is internal is also on hand externally and vice versa.
The appearance shows nothing that is not in the essence and
there is nothing in the essence that is not manifested.

In the second place, however,
inner and outer are also opposed to one another
as determinations of the form
and, to be sure, unqualifiedly so,
as the abstractions of identity with itself
and of sheer multiplicity or reality.
Yet, since they are essentially identical
as moments of the one form,
what is only posited initially
in the one abstraction is
also immediately only in the other.
Hence, what is only something internal is
also, by this means, only something external
and what is only something external is
as yet also only something internal.

    It is the usual mistake of reflection
    to take the essence as the merely inner.
    When it is taken merely in this way,
    then this consideration is
    also a completely external one
    and that essence is the empty external abstraction.

        The inner side of nature a poet says
        No created spirit can penetrate,
        Fortunate enough if he knows merely the outer shell

    It should have been said, rather,
    that precisely when he determines
    the essence of nature as something inner,
    he knows only the outer shell.
    Since in being in general
    or even in merely sensory perception,
    the concept is only the inner at first,
    it is something external for it [sensory perception],
    a subjective being as well as thinking, devoid of truth.
    In nature as in the spirit, insofar as
    the concept, purpose, law are at first
    only inner dispositions, pure possibilities,
    they are only an external, inorganic nature at first,
    science of a third, alien power, and so forth.
    As a human being is externally, in his actions
    (not, of course, in his merely corporeal externality),
    so he is internally;
    and if he is only internally virtuous, moral, and so forth,
    only in intentions and sentiments
    and his outer life is not identical with them,
    then the one is as hollow and empty as the other.

The empty abstractions, by means of which
the one identical content is still supposed
to obtain in the relationship,
sublate themselves in the immediate transition,
the one in the other;
the content is itself nothing other than their identity,
they are the shine of the essence, posited as shine.
Through the force's expression, the inner is posited in concrete existence;
this positing is the mediating by means of empty abstractions;
it vanishes in itself into the immediacy in which
the inner and outer are in and for themselves identical and
their difference is determined as mere positedness.
This identity is the actuality.

SECTION III

Actuality

Actuality is the unity of essence and concrete existence;
in it, shapeless essence and unstable appearance
(subsistence without determination
and manifoldness without permanence)
have their truth.
Although concrete existence is the immediacy
that has proceeded from ground,
it still does not have form explicitly posited in it;
inasmuch as it determines and informs itself, it is appearance;
and in developing this subsistence that otherwise only is
a reflection-into-other into an immanent reflection,
it becomes two worlds, two totalities of content,
one determined as reflected into itself
and the other as reflected into other.
But the essential relation exposes
the formality of their connection,
and the consummation of the latter is
the relation of the inner and the outer
in which the content of both is equally
only one identical substrate
and only one identity of form.
This identity has come about also in regard to form,
the form determination of their difference is sublated,
and that they are one absolute totality is posited.

This unity of the inner and outer is absolute actuality.
But this actuality is, first, the absolute as such
(in so far as it is posited as a unity
in which the form has sublated itself)
making itself into the empty or external
distinction of an outer and inner.
Reflection relates to this absolute
as external to it;
it only contemplates it
rather than being its own movement.
But it is essentially this movement
and is, therefore, as the absolute's
negative turning back into itself.

Second, it is actuality proper.
Actuality, possibility, and necessity constitute
the formal moments of the absolute,
or its reflection.

Third, the unity of the absolute
and its reflection is
the absolute relation,
or rather the absolute as
relation to itself, substance.

CHAPTER 1

The absolute

The simple solid identity of the absolute is indeterminate, or rather,
every determinateness of essence and concrete existence,
or of being in general as well as of reflection,
has dissolved itself into it.
Accordingly, the determining of what is
the absolute appears to be a negating,
and the absolute itself appears only as
the negation of all predicates, as the void.
But since it must equally be spoken of
as the position of all predicates,
it appears as the most formal of contradictions.
In so far as that negating and this positing
belong to external reflection,
what we have is a formal, unsystematic dialectic
that has an easy time picking up
a variety of determinations here and there,
and is just as at ease demonstrating, on the one hand,
their finitude and relativity, as declaring, on the other,
that the absolute, which it vaguely envisages as totality,
is the dwelling place of all determinations,
yet is incapable of raising either
the positions or the negations to a true unity.
The task is indeed to demonstrate what the absolute is.
But this demonstration cannot be either
a determining or an external reflection
by virtue of which determinations of the absolute would result,
but is rather the exposition of the absolute,
more precisely the absolute's own exposition,
and only a displaying of what it is.

CHAPTER 2

Actuality

The absolute is the unity of inner and outer
as a first implicitly existent unit.
The exposition appeared as an external reflection
which, for its part, has the immediate as something it has found,
but it equally is its movement and
the reference connecting it to the absolute
and, as such, it leads it back to the latter,
determining it as a mere “way and manner.”
But this “way and manner” is the determination of the absolute itself,
namely its first identity or its mere implicitly existent unity.
And through this reflection, not only is that
first in-itself posited as essenceless determination,
but, since the reflection is negative self-reference,
it is through it that the in-itself becomes
a mode in the first place.
It is this reflection that,
in sublating itself in its determinations
and as a movement which as such turns back upon itself,
is first truly absolute identity
and, at the same time, the determining of
the absolute or its modality.
The mode, therefore, is the externality of the absolute,
but equally so only its reflection into itself;
or again, it is the absolute's own manifestation,
so that this externalization is its immanent reflection
and therefore its being in-and-for-itself.

So, as the manifestation that it is nothing,
that it has no content, save to be the manifestation of itself,
the absolute is absolute form.
Actuality is to be taken as this reflected absoluteness.
Being is not yet actual;
it is the first immediacy;
its reflection is therefore becoming
and transition into an other;
or its immediacy is not being-in-and-for-itself.
Actuality also stands higher than concrete existence.
It is true that the latter is the immediacy
that has proceeded from ground and conditions,
or from essence and its reflection.
In itself or implicitly, it is therefore
what actuality is, real reflection;
but it is still not the posited unity of reflection and immediacy.
Hence concrete existence passes over into appearance
as it develops the reflection contained within it.
It is the ground that has foundered to the ground;
its determination, its vocation, is to restore this ground,
and therefore it becomes essential relation,
and its final reflection is that its
immediacy be posited as immanent reflection and conversely.
This unity, in which concrete existence
or immediacy and the in-itself,
the ground or the reflected, are simply moments,
is now actuality.
The actual is therefore manifestation.
It is not drawn into the sphere of alteration by its externality,
nor is it the reflective shining of itself in an other.
It just manifests itself, and this means that in its externality,
and only in it, it is itself, that is to say,
only as a self-differentiating and self-determining movement.

Now in actuality as this absolute form,
the moments only are as sublated or formal, not yet realized;
their differentiation thus belongs at first to external reflection
and is not determined as content.

Actuality, as itself immediate form-unity of inner and outer,
is thus in the determination of immediacy
as against the determination of immanent reflection;
or it is an actuality as against a possibility.
The connection of the two to each other is the third,
the actual determined both as being reflected into itself
and as this being immediately existing.
This third is necessity.

But first, since the actual and the possible are formal distinctions,
their connection is likewise only formal, and consists only in this,
that the one just like the other is a positedness, or in contingency.

Second, because in contingency the actual
as well as the possible are a positedness,
because they have retained their determination,
real actuality now arises,
and with it also real possibility
and relative necessity.

Third, the reflection of relative necessity
into itself yields absolute necessity,
which is absolute possibility and actuality.

CHAPTER 3

The absolute relation

Absolute necessity is not so much the necessary,
even less a necessary, but necessity:
being simply as reflection.
It is relation because it is a distinguishing
whose moments are themselves the whole totality of necessity,
and therefore subsist absolutely,
but do so in such a way that their subsisting is one subsistence,
and the difference only the reflective shine of
the movement of exposition,
and this reflective shine is the absolute itself.
Essence as such is reflection or a shining;
as absolute relation, however, essence is the
reflective shine posited as reflective shine,
one which, as such self-referring, is absolute actuality.
The absolute, first expounded by external reflection,
as absolute form or as necessity now expounds itself;
this self-exposition is its self-positing,
and is only this self-positing.
Just as the light of nature is not a something,
nor is it a thing, but its being is rather only its shining,
so manifestation is self-identical absolute actuality.

The sides of the absolute relation are not, therefore, attributes.
In the attribute the absolute reflectively shines only in one of its moments,
as in a presupposition that external reflection has simply assumed.
But the expositor of the absolute is the absolute necessity
which, as self-determining, is identical with itself.
Since this necessity is the reflective shining
posited as reflective shining, the sides of this relation,
because they are as shine, are totalities;
for as shine, the differences are themselves and their opposite,
that is, they are the whole;
and, conversely, they thus are only shine because they are totalities.
Thus this distinguishing, this reflecting shining of the absolute,
is only the identical positing of itself.

This relation in its immediate concept is
the relation of substance and accidents,
the immediate internal disappearing and becoming
of the absolute reflective shine.
If substance determines itself as a being-for-itself over
against an other or is absolute relation as something real,
then we have the relation of causality.
Finally, when this last relation passes over into
reciprocal causality by referring itself to itself,
we then have the absolute relation also posited
in accordance with the determination it contains;
this posited unity of itself in its determinations,
which are posited as the whole itself
and consequently equally as determinations,
is then the concept.
