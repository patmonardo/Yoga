SECTION II

Objectivity

In Book One of the Objective Logic,
abstract being was presented as
passing over into existence,
but at the same time as
retreating into essence.

In Book Two, essence shows itself as
determining itself as ground,
thereby stepping into concrete existence
and realizing itself as substance,
but at the same time
retreating into the concept.

Of the concept, we have now first shown
that it determines itself as objectivity.
It should be obvious that this latter transition is
essentially the same as the proof from the concept,
that is to say, from the concept of God to his existence,
that was formerly found in Metaphysics,
or the so-called ontological proof.
Equally well known is that
Descartes's sublimest thought,
that God is that whose concept
includes his being within itself,
after having degenerated into
the bad form of the formal syllogism,
namely into the form of the said proof,
finally succumbed to the Critique of Reason
and to the thought that existence
cannot be extracted from the concept.
Some elucidations concerning this proof
have already been made earlier.
In Volume I, pp. 47 ff.,
where being has vanished into
its closest opposite, non-being,
and becoming has shown itself
to be the truth of both,
attention was called to the
confusion that arises in the case
of a determinate existence
when we concentrate, 
not on its being,
but on its determinate content,
and then imagine if we compare 
this determinate content
(e.g. one hundred dollars)
with another determinate content
(e.g. the context of my perception, of my financial situation)
and discover that it makes indeed a difference
whether the one content is added to the other or not;
that we are dealing with the distinction of being and non-being,
or even the distinction of being and the concept.
Further, in the same Volume
on pp. 64ff. and on p. 289 of Volume II,
the definition of a sum-total of all reality
which occurs in the ontological proof was elucidated.
But the essential subject matter of that proof,
the connectedness of concept and existence,
is the concern of the treatment of the concept just concluded
and of the entire course that the latter traverses
in determining itself to objectivity.
The concept, as absolutely
self-identical negativity,
is self-determining;
it was noted that the concept,
in resolving itself into judgment in singularity,
already posits itself as something real, an existent;
this still abstract reality completes itself in objectivity.

Now it might appear that the transition
from the concept into objectivity
is quite another thing than the transition
from the concept of God to God's existence.
But, on the one hand, it must be borne in mind
that the determinate content, God,
makes no difference in a logical progression,
and that the ontological proof is
only one application of this logical progression
to that particular content.
On the other hand, it is essential to be
reminded of the remark made above that
the subject obtains determinateness
and content only in its predicate;
that prior to the predicate,
whatever that content might otherwise be for
feeling, intuition, and representation,
so far as conceptual cognition is concerned
it is only a name;
but in the predicate, with determinateness,
there begins at the same time
the process of realization in general.
The predicates, however, must be grasped as
themselves still confined within the concept,
hence as something subjective with which
no move to existence has yet been made;
even for this reason, in judgment the realization of
the concept is certainly not completed yet.
But there is the further reason that
the mere determination of a subject matter through predicates,
without this determination being at the same time
the realization and objectification of the concept,
remains something so subjective that it is not even
a true cognition and determination of the
concept of the subject matter;
“subjective” in the sense of abstract reflection
and non-conceptual representation.
God as living God, and better still as absolute spirit,
is only recognized in what he does.
Humankind were directed early to recognize God in his works;
only from these can the determinations proceed
that can be called his properties,
and in which his being is also contained.
It is thus the conceptual comprehension of God's activity,
that is to say, of God himself,
that recognizes the concept of God in his being
and his being in his concept.
Being by itself, or even existence,
are such a poor and restricted determination,
that the difficulty of finding them in the concept
may well be due to not having considered
what being or existence themselves are.
Being as entirely abstract, immediate self-reference,
is nothing but the abstract moment of the concept;
it is its moment of abstract universality
that also provides what is required of being,
namely that it be outside the concept,
for inasmuch as universality is a moment of the concept,
it is also its difference or the abstract judgment
wherein the concept opposes itself to itself.
The concept, even as formal, already immediately contains
being in a truer and richer form,
in that, as self-referring negativity,
it is singularity.

But of course the difficulty of finding
being in the concept in general,
and equally so in the concept of God,
becomes insuperable if we expect
being to be something that
we find in the context of external experience
or in the form of sense-perception,
like the one hundred dollars
in the context of my finances,
as something graspable only by hand,
not by spirit, essentially visible to
the external and not the internal eye;
in other words, if the name of
being, reality, truth, is given
to that which things possess
as sensuous, temporal, and perishable.
The consequence of a philosophizing
that in regard to being fails
to rise above the senses is
that, in regard to the concept,
it also fails to let go of
merely abstract thought;
such thought stands opposed to being.

The customary practice of regarding
the concept as something just as one-sided
as abstract thought will already stand in the way
of accepting what has just been suggested,
namely, that we regard the transition of
the concept of God to his being as
an application of the logical course
of objectification of the concept presented above.
Yet if it is granted, as it commonly is,
that the logical element, as the formal element,
constitutes the form for the cognition
of every determinate content,
then that application at least
would have to be conceded,
unless even at the opposition of
concept and objectivity in general
one stops short at the untrue concept
and an equally untrue reality as an ultimate.
But in the exposition of the pure concept
it was further indicated that the latter is
the absolute divine concept itself.
In truth, therefore, what takes place is
not a relation of application
but the immediate display in the logical course
of God's self-determination as being.
But on this point it is to be remarked
that inasmuch as the concept is to be
presented as the concept of God,
it ought be apprehended as it is
when already taken up in the idea.
The said pure concept passes through
the finite forms of the judgment
and the syllogism precisely
because it is not yet posited
in and for itself as one with objectivity,
but is conceived rather only in
the process of becoming that objectivity.
The latter, too, is not yet the divine concrete existence,
not yet the reality reflectively shining in the idea.
And yet objectivity is just that much richer and higher
than the being or existence of the ontological proof,
as the pure concept is richer and
higher than that metaphysical vacuum
of the sum-total of all reality.
But I reserve for another occasion
the task of elucidating in greater detail
the manifold misunderstanding brought
upon the ontological proof of God's existence,
and also on the rest of the other so-called proofs,
by logical formalism.
We shall also elucidate Kant's critique of such proofs
in order to establish their true meaning
and thus restore the thoughts on which they are based
to their worth and dignity.

We have previously called attention to
the several forms of immediacy
that have already come on the scene,
but in different determinations.
In the sphere of being, immediacy is
being itself and existence;
in the sphere of essence,
it is concrete existence
and then actuality and substantiality;
in the sphere of the concept,
besides being immediacy as abstract universality,
it is now objectivity.
These expressions, when the exactitude
of philosophical conceptual distinctions is
not at stake, may be used as synonymous;
but the determinations are derived
from the necessity of the concept.
Being is as such the first immediacy,
and existence is the same immediacy
with a first determinateness.
Concrete existence, along with the thing,
is the immediacy that proceeds from ground,
from the self-sublating mediation
of the simple reflection of essence.
But actuality and substantiality are
the immediacy that proceeds from
the sublated difference of the still unessential
concrete existence as appearance and its essentiality.
Finally, objectivity is the immediacy
as which the concept has determined itself
by the sublation of its abstraction and mediation.
It is the privilege of philosophy to choose
such expressions from the language of ordinary life,
which is made for the world of imaginary representations,
as seem to approximate the determinations of the concept.
There is no question of demonstrating
for a word chosen from ordinary life
that in ordinary life too the same concept is associated
with that for which philosophy uses it,
for ordinary life has no concepts,
only representations of the imagination,
and to recognize the concept in what is otherwise
mere representation is philosophy itself.
It must therefore suffice if representation,
for those of its expressions
that philosophy uses for its definitions,
has only some rough approximation
of their distinctive difference;
it may also be the case that in these expressions
one recognizes pictorial adumbrations
which, as approximations, are close indeed to the
corresponding concepts.
One will be hard pressed, perhaps,
to concede that something can be
without actually existing;
but at least nobody will mistake, for instance,
being as the copula of the judgment
for the expression “to exist actually,”
and nobody will say that
“this article exists dear, suitable, etc.,”
“gold exists a metal or metallic,”
instead of “this article is dear, suitable, etc.,”
“gold is a metal.”
And surely it is common to distinguish
being from appearing,
appearance from actuality,
as also being as contrasted to actuality,
and still more all these expressions from objectivity.
But even if such expressions were used synonymously,
philosophy would in any case have the freedom
to take advantage of such empty superfluity of language
for the purpose of its distinctions.

Mention was made in connection with the apodictic judgment
where judgment attains completion
and the subject thus loses its determinateness
as against the predicate,
of the double meaning of subjectivity originating from it,
namely the subjectivity of the concept
and equally so of the externality
and contingency confronting the concept.
A similar objectivity also appears for the double meaning,
of standing opposed to the self-subsistent concept
yet of also existing in and for itself.
In the former sense, the object stands opposed
to the “I = I” which in subjective idealism
is declared to be the absolute truth.
It is then the manifold world in its immediate existence
with which the “I” or the concept is engaged in endless struggle,
in order, by the negation of the inherently nullity of this other,
to give to its first certainty of being a self,
the actual truth of its equality with itself.
In a broader sense, it means a subject matter in general
for whatever interest or activity of the subject.

In the opposite sense, however,
the objective signifies that which exists in and for itself,
without restriction and opposition.
Rational principles, perfect works of art, etc.,
are said to be objective to the extent
that they are free and above every accidentality.
Although rational principles, whether theoretical or ethical,
only belong to the sphere of the subjective, to consciousness,
this aspect of the latter of existing in and for itself
is nonetheless called objective;
the cognition of truth is made to rest on
the cognition of the object as free of
any addition by subjective reflection,
and right conduct on the adherence to objective laws,
such as are not of subjective origin
and are immune to arbitrariness
and to treatment that would compromise their necessity.

At the present standpoint of our treatise,
objectivity has the meaning first of all of
the being in and for itself of the concept
that has sublated the mediation posited
in its self-determination,
raising it to immediate self-reference.
This immediacy is therefore itself
immediately and entirely pervaded by the concept,
just as its totality is immediately identical with its being.
But further, since the concept equally has to restore
the free being-for-itself of its subjectivity,
it enters with respect to objectivity
into a relation of purpose
in which the immediacy of the objectivity
becomes a negative for it,
something to be determined through its activity.
This immediacy thus acquires the other significance,
namely that in and for itself,
in so far as it stands opposed to the concept,
it is a nullity.

First, then, objectivity is in its immediacy.
Its moments, on account of the totality of all moments,
stand in self-subsistent indifference
as objects each outside the other,
and as so related they possess
the subjective unity of the concept
only as inner or as outer.
This is mechanism.

But, second, inasmuch as in mechanism that unity
reveals itself to be the immanent law of the objects,
their relation becomes one of non-indifference,
each specifically different according to law;
a connection in which the objects'
determinate self-subsistence is sublated.
This is chemism.

Third, this essential unity of the objects is
thereby posited as distinct from their self-subsistence.
It is the subjective concept,
but posited as referring in and for itself
to the objectivity, as purpose.
This is teleology.

Since purpose is the concept posited
as within it referring to objectivity,
and through itself sublating its defect
of being subjective,
the at first external purposiveness becomes,
through the realization of the purpose, internal.
It becomes idea.

CHAPTER 1 Mechanism

Since objectivity is the totality of the concept
that has returned into its unity,
an immediate is thereby posited
which is in and for itself that totality,
and is also posited as such,
but in it the negativity of the concept has as yet
not detached itself from the immediacy of the totality;
in other words, the objectivity is not yet posited as judgment.
In so far as it has the concept immanent in it,
the difference of the concept is present in it;
but on account of the objective totality,
the differentiated moments are
complete and self-subsistent objects
that, consequently, even in connection
relate to one another as each standing on its own,
each maintaining itself in every combination as external.
This is what constitutes the character of mechanism,
namely, that whatever the connection that
obtains between the things combined,
the connection remains one that is alien to them,
that does not affect their nature,
and even when a reflective semblance
of unity is associated with it,
the connection remains nothing more than
composition, mixture, aggregate, etc.
Spiritual mechanism, like its material counterpart,
also consists in the things connected in the spirit
remaining external to one another and to spirit.
A mechanical mode of representation,
a mechanical memory, a habit, a mechanical mode of acting,
mean that the pervasive presence that is proper to spirit
is lacking in what spirit grasps or does.
Although its theoretical or practical mechanism
cannot take place without its spontaneous activity,
without an impulse and consciousness,
the freedom of individuality is still lacking in it,
and since this freedom does not appear in it,
the mechanical act appears as a merely external one.

CHAPTER 2

Chemism

In objectivity as a whole
chemism constitutes the moment of judgment,
of the difference that has become objective,
and of process.

Since it already begins with
determinateness and positedness,
and the chemical object is
at the same time objective totality,
the course it follows next is
simple and perfectly determined
by its presupposition.

CHAPTER 3

Teleology

Where there is the perception of a purposiveness,
an intelligence is assumed as its author;
required for purpose is thus the concept's
own free concrete existence.
Teleology is above all contrasted with mechanism,
in which the determinateness posited in the object,
being external, is one that gives no sign of self-determination.
The opposition between causæ efficientes and causæ finales,
between merely efficient and final causes,
refers to this distinction, just as,
at a more concrete level, the enquiry whether the absolute essence
of the world is to be conceived as blind mechanism
or as an intelligence that determines itself
in accordance with purposes also comes down to it.
The antinomy of fatalism, along with determinism,
and freedom is equally concerned with
the opposition of mechanism and teleology;
for the free is the concept in its concrete existence.

Earlier metaphysics has dealt with these concepts
as it dealt with others.
It presupposed a certain picture of the world
and strived to show that one or the other concept
of causality was adequate to it,
and the opposite defective because
not explainable from the presupposed picture,
all the while not examining the concept of
mechanical cause and that of purpose to see
which possesses truth in and for itself.
If this is established independently, it may turn out
that the objective world exhibits mechanical and final causes;
its actual existence is not the norm of what is true,
but what is true is rather the criterion for deciding
which of these concrete existences is its true one.
Just as the subjective understanding exhibits also errors in it,
so the objective world exhibits also aspects and stages of truth
that by themselves are still one-sided, incomplete,
and only relations of appearances.
If mechanism and purposiveness stand opposed to each other,
then by that very fact they cannot
be taken as indifferent concepts,
as if each were by itself a correct concept
and had as much validity as the other,
the only question being where
the one or the other may apply.
This equal validity of the two rests
only on the fact that they are,
that is to say, that we have them both.
But since they do stand opposed,
the necessary first question is,
which of the two concepts is the true one;
and the higher and truly telling question is,
whether there is a third which is their truth,
or whether one of them is the truth of the other.
But purposive connection has proved
to be the truth of mechanism.
Regarding chemism, what came under it
can be taken together with mechanism,
for purpose is the concept in free concrete existence,
and the concept's state of unfreedom,
its being sunk into externality,
stands opposed to it in any form.
Both, mechanism as well as chemism, are
therefore included under natural necessity:
mechanism, because in it the concept
does not exist in the object concretely,
for as mechanical the latter lacks self-determination;
chemism, either because the concept has in it
a one-sided concrete existence in a state of tension,
or because, emerging as the unity that creates
in the neutral object a tension of extremes,
it is external to itself in so far as it sublates this divide.

The closer the teleological principle is associated
with the concept of an extra-mundane intelligence,
and the more it has therefore enjoyed the favor of piety,
all the more it has seemed to depart from
the true investigation of nature,
which aims at a cognition of the properties of nature
not as extraneous, but as immanent determinacies,
and accepts only such cognition
as a valid conceptual comprehension.
Since purpose is the concept itself in its concrete existence,
it may seem strange that a cognition of objects
based on their concept rather appears as
an unjustified trespass into a heterogeneous element,
whereas mechanism, for which the determinateness
of an object is posited in it externally and by an other,
is accepted as a more immanent view of things than teleology.
Of course mechanism, at least the ordinary unfree mechanism,
and chemism as well, must be regarded as an immanent principle
in so far as the externally determining object
is itself again just another such object,
externally determined and indifferent to its being determined,
or, in the case of chemism, in so far as the other object
must likewise be one that is chemically determined;
in general, in so far as an essential moment
of the totality always lies in something external.
These principles remain confined, therefore,
within the same natural form of finitude;
but although they do not wish to transcend the finite
and, as regards appearances, lead only to finite causes
that themselves demand further causes,
they nonetheless equally expand themselves,
partly into a formal totality in the concept
of force, cause, or of such determinations of reflection
that are supposed to signify originariness, and partly,
through the medium of abstract universality,
also into a sum total of forces,
a whole of reciprocal causes.
Mechanism thus reveals itself to be a striving for totality
by the very fact that it seeks to comprehend nature by itself
as a whole that has no need of an other for its concept;
a totality that is not found in purpose
and the extra-mundane intelligence associated with it.

Now purposiveness presents itself from the first
as something of a generally higher nature,
as an intelligence that externally determines
the manifoldness of objects through a unity
that exists in and for itself,
so that the indifferent determinacies of the objects
become essential by virtue of this connection.
In mechanism they become so through the mere form of necessity
that leaves their content indifferent,
for they are supposed to remain external
and only the understanding as such is
expected to find satisfaction by recognizing
its principle of union, the abstract identity.
In teleology, on the contrary, the content becomes important,
for teleology presupposes a concept,
something determined in and for itself
and consequently self-determining,
and has therefore extracted from the connection of
differences and their reciprocal determinateness, from the form,
a unity that is reflected into itself,
something that is determined in and for itself
and is consequently a content.
But if this content is otherwise finite and insignificant,
then it contradicts what it is supposed to be,
for according to its form purpose is
a totality infinite within itself;
especially when the activity operating in accordance
with it is assumed to be an absolute will and intelligence.
For this reason has teleology drawn the
reproach of triviality so much upon itself,
for the purposes that it has espoused are,
as the case may be, more important or more trivial [than the content],
and it was inevitable that the connection of purposiveness
in objects would so often appear just a frivolity,
since it appears external and therefore contingent.
Mechanism, on the contrary, leaves to the determinacies of the objects,
as regards their content, their status as accidents indifferent to the object,
and these determinacies are not supposed to have,
whether for the objects or the subjective understanding,
any value higher than that.
This principle, combined with external necessity,
yields therefore a consciousness of infinite freedom
that contrasts with teleology,
which sets up as something absolute bits of its content
that are trivial and even contemptible,
where the more universal thought can only
find itself infinitely constricted,
even to the point of feeling disgust.

The formal disadvantage from which
this teleology immediately suffers
is that it only goes as far as external purposiveness.
The content of concept, since the latter is
thereby posited as something formal,
is for teleology also externally given to it
in the manifoldness of the objective world
in those very determinacies that
are also the content of mechanism,
but are there as something external and accidental.
Because of this commonality of content,
only the form of purposiveness constitutes by itself
the essential element of the teleological.
In this respect, without as yet considering
the distinction between external and internal purposiveness,
the connection of purpose in general has
proven itself to be the truth of mechanism.
Teleology possesses in general the higher principle,
the concept in its concrete existence,
which is in and for itself the infinite and absolute;
a principle of freedom which, utterly certain of its self-determination,
is absolutely withdrawn from the external determining of mechanism.

One of Kant's greatest services to philosophy was
in drawing the distinction between relative or
external purposiveness and internal purposiveness;
in the latter he opened up the concept of life, the idea,
and with that he positively raised philosophy
above the determinations of reflection
and the relative world of metaphysics,
something that the Critique of Reason does
only imperfectly, ambiguously, and only negatively.
We have remarked that the opposition of teleology and mechanism is
first of all the general opposition of freedom and necessity.
Kant treated the opposition in this form, among the antinomies of reason,
namely, as the third conflict of the transcendental ideas.
I cite his exposition, to which reference was made earlier,
very briefly because its essential point is so simple
that it does not need extensive explanation,
and moreover, the peculiarities of Kant's antinomies
have been elucidated in greater detail elsewhere.

The thesis of the antinomy now in question runs thus:
Causality according to the laws of nature is
not the only one from which the appearance of
the world can exhaustively be derived.
For their explanation, it is necessary to
assume yet another causality through freedom.

The antithesis: There is no freedom,
but everything in the world happens
solely according to laws of nature.

As in the other antinomies, the proof starts off apagogically
by assuming the opposite of each thesis;
secondly, in order to show the contradiction of this assumption,
its opposite (which is then the proposition to be proved)
is assumed in turn and presupposed as valid.
This whole roundabout proof could therefore be spared,
for the proof consists in nothing but the assertoric
assertion of the two opposite propositions.

Thus to prove the thesis we should first assume
that there is no other causality than
that according to the laws of nature, that is,
according to the necessity of mechanism in general, chemism being included.
This proposition contradicts itself,
because the law of nature consists just in this,
that nothing happens without a cause sufficiently determined a priori,
a cause that would have to contain an absolute spontaneity within it,
that is, the assumption opposed to the thesis is contradictory,
for the reason that it contradicts the thesis.

In support of the proof of the antithesis
we should assume that there is a freedom
as a particular kind of causality
for absolutely initiating a situation,
and together with it also a series of
consequences following upon it.
But now, since such a beginning presupposes
a situation that has no causal link with the one preceding it,
it contradicts the law of causality that alone makes the unity
of experience and experience in general possible;
that is, the assumption of the freedom
that is opposed to the antithesis cannot be made,
for the reason that it contradicts the antithesis.

We find in essence the same antinomy in
the Critique of the Teleological Judgement as
the opposition between the proposition that
every generation of material things happens
according to merely mechanical laws,
and the proposition that some cases
of generation of material things
are not possible according to such laws.
Kant's resolution of this antinomy is the same
as the general resolution of the rest,
namely that reason cannot prove
either the one or the other proposition
because we cannot have a priori
any determining principle of the possibility of things
according to merely empirical laws of nature;
further, that therefore the two propositions must be
regarded not as objective propositions but as subjective maxims;
that I ought to reflect on the events of nature every time
according to the principle of the mechanism of nature alone,
but that this does not prevent, when occasion permits,
following up certain natural forms in accordance with another maxim,
namely in accordance with the principal of final causes,
as if now these two maxims, which moreover are supposed to be
necessary only for human reason,
did not stand in the same opposition as
the two propositions in antinomy.
Missing in all this, as we remarked above,
is the one thing that alone is of philosophical interest,
namely the investigation of which of the two principles
has truth in and for itself.
On this standpoint, it makes no difference
whether the principles should be regarded as objective,
which means here, as externally existing
determinations of nature,
or as mere maxims of a subjective cognition;
what is subjective here is rather the
contingent cognition that applies one or the other maxim
as occasion demands, indeed, according to whether
it deems them fitting for given objects,
but for the rest does not ask about the truth
of these determinations themselves,
whether they both are determinations
of the objects or of cognition.

However unsatisfactory is for this reason Kant's discussion
of the teleological principle with respect to its essential viewpoint,
still worthy of note is the place that Kant assigns to it.
Since he ascribes it to a reflective faculty of judgment,
he makes it into a mediating link between
the universal of reason and the singular of intuition;
further, he distinguishes this reflective judgment
from the determining judgment, the latter one
that merely subsumes the particular under the universal.
Such a universal that only subsumes is an abstraction
that becomes concrete only in an other, in the particular.
Purpose, on the contrary, is the concrete universal
containing within itself the moment
of particularity and of externality;
it is therefore active and the impulse
to repel itself from itself.
The concept, as purpose, is of course an objective judgment
in which one determination, the subject,
namely the concrete concept, is self-determined,
while the other is not only a predicate but external objectivity.
But for that reason the connection of purpose is not
a reflective judgment that considers external objects
only according to a unity,
as though an intelligence had given them to us
for the convenience of our faculty of cognition;
on the contrary, it is the truth that exists in
and for itself and judges objectively,
determining the external objectivity absolutely.
The connection of purpose is therefore more than judgment;
it is the syllogism of the self-subsistent free concept
that through objectivity unites itself with itself in conclusion.

Purpose has resulted as the third
to mechanism and chemism;
it is their truth.
Inasmuch as it still stands
inside the sphere of objectivity
or of the immediacy of the total concept,
it is still affected by externality as such
and has an objective world over
against it to which it refers.
From this side, mechanical causality,
to which chemism is also in general to be added,
still makes its appearance in this purposive connection
which is the external one,
but as subordinated to it
and as sublated in and for itself.
As regards the more precise relation,
the mechanical object is, as immediate totality,
indifferent to its being determined and consequently,
conversely, to its being a determinant.
This external determinateness has now
progressed to self-determination
and accordingly the concept that
in the object was only inner
or, which amounts to the same,
only outer, is now posited;
purpose is, in the first instance,
precisely this concept which is
external to the mechanical object.
And so for chemism also, purpose is the self-determining
which brings the external determinateness conditioning it
back to the unity of the concept.
We have here the nature of the subordination of
the two preceding forms of the objective process.
The other, which in those forms lies in the infinite progress,
is the concept posited at first as external to them,
and this is purpose;
not only is the concept their substance
but externality is for them also
an essential moment constituting their determinateness.
Thus mechanical or chemical technique,
because of its character of being externally determined,
naturally offers itself to the connection of purpose,
which we must now examine more closely.
