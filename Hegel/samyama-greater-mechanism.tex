A. THE MECHANICAL OBJECT

The object is, as we have seen, the syllogism
whose mediation has attained equilibrium
and has therefore come to be immediate identity.
It is therefore in and for itself a universal: universality,
not in the sense of a commonality of properties,
but a universality that pervades particularity
and in it is immediate singularity.

1. To begin with, therefore,
the object does not differentiate
itself into matter and form,
matter being its presumed self-subsistent universal aspect
and form the particular and singular instead;
according to its concept, any such abstract differentiation of
singularity and universality has no place in the object;
if regarded as matter, the object must then be taken
to be in itself informed matter.
One can just as well take it as a thing with properties,
as a whole consisting of parts, as substance with accidents,
or as determined by the other relations of reflection.
But these are all past relations that
in the concept have come to an end.
The object, therefore, has neither properties nor accidents,
for these are separable from the thing or the substance,
whereas in the object particularity is
absolutely reflected into the totality.
In the parts of a whole,
there is indeed present that self-subsistence that
pertains to the differences of the object,
but these differences are at once themselves essentially objects,
totalities which, unlike parts, are not such as against the whole.

At first, therefore, the object is indeterminate,
for it has no determinate opposition within,
because it is the mediation
that has collapsed into immediate identity.
Inasmuch as the concept is essentially determined,
the object has in it the determinateness of a manifold
which, although complete, is otherwise
indeterminate, that is, relationless,
one that constitutes a totality
also not further determined at first;
sides or parts that may be distinguished
within it belong to an external reflection.
This totally indeterminate difference thus amounts just to this,
that there are several objects,
each of which only contains its determinateness
reflected into its universality
and does not reflectively shine outwardly.
Because this indeterminate determinateness is
essential to the object,
the object is in itself a plurality,
and must therefore be regarded as a composite, an aggregate.
Yet it does not consist of atoms, for atoms are not objects
because they are not totalities.
Leibniz's monad would be more of an object.
It is a total representation of the world
which, shut up within its intensive subjectivity,
in essence at least is supposed to be a one.
Yet the monad, determined as an exclusive one,
is a principle only assumed by reflection.
It is an object, however, both because
the ground of its manifold representations of the developed,
that is, the posited determinations of
its merely implicit totality, lies outside it,
and because it is equally a matter of indifference
for the monad that it constitutes an object
together with other objects;
in fact, therefore, it is not exclusive,
not self-determined for itself.

2. Since the object is now
a totality of determinateness, yet,
because of its indeterminateness and immediacy,
it is not the totality's negative unity,
it is indifferent towards the determinations as singulars,
determined in and for themselves,
just as these are themselves indifferent to each other.
These, therefore, are not comprehensible from it nor from one another;
the object's totality is the form of the overall reflectedness
of its manifoldness into a singularity in general
which is not in itself determinate.
The determinacies, therefore, which are in it do indeed pertain to it;
but the form that constitutes their difference
and combines them into a unity is an external one,
indifferent to them;
whether it be a mixture, or again an order,
a certain arrangement of parts and sides,
these are combinations that are indifferent to what they connect.

Consequently, like an existence in general,
the object has the determinateness of its totality outside it,
in other objects, and these again outside them,
and so forth to infinity.
The immanent turning back of this progression in infinitum
must indeed be likewise assumed,
and it must be represented as a totality, as a world,
but one which is nothing but a universality brought to closure
through a singularity that remains indeterminate,
a universe.

Since the object is thus determinate
yet indifferent to its determinateness,
through itself it points for its determinateness
outside and beyond itself,
constantly to objects for which it is however
likewise a matter of indifference
that they do the determining.
Consequently, nowhere is
a principle of self-determination to be found.
Determinism, which is the standpoint that cognition adopts
when it assumes as truth the object as we first have it here,
assigns for each determination of the object that of another object;
but this other object is likewise indifferent
both to its determinateness and its determining.
For this reason determinism is itself so indeterminate
as to be bound to an infinite progression;
it can halt at will anywhere, and be satisfied there,
because the object to which it has progressed,
being a formal totality, is shut up within itself
and indifferent to its being determined by another.
For this reason to explain the determination of an object,
and to this end to extend the representation of it beyond it,
is only an empty word,
for there is no self-determination
in the other object to which
the explanation has been extended.

3. Now since the determinateness of an object lies in an other,
there is no determinate diversity separating the two;
the determinateness is merely doubled,
once in the one object and then again in the other;
it is something utterly identical
and the explanation or comprehension is,
therefore, a tautology.
This tautology is an external back and forth movement;
since the determinateness fails to obtain from objects
that are indifferent to it any proper differentiation
and is therefore only identical,
there is only one determinateness at hand,
and that it should be doubled only expresses
precisely the externality and vacuity of a difference.
But the objects are at the same time
self-subsistent in regard to one another;
in that identity, therefore, they remain utterly external.
Thus there arises the contradiction of
a perfect indifference of objects to one another
and of an identity of determinateness of such objects,
or of the objects' perfect externality
in the identity of their determinateness.
This contradiction is consequently the
negative unity of a plurality of objects
reciprocally repelling each other in the unity.
This is the mechanical process.

B. THE MECHANICAL PROCESS

If objects are regarded only as self-enclosed totalities,
they cannot act on one another.
Regarded in this way, they are the same as the monads
which, precisely for that reason, were thought of
as having no influence on each other.
But the concept of a monad is for just this reason
a deficient reflection.
For, in the first place, the monad is
a determinate representation of its only implicit totality;
as a certain degree of development and positedness of its
representation of the world, it is determinate;
but since it is a self-enclosed totality,
it is also indifferent to this determinateness
and is, therefore, not its own determinateness
but a determinateness posited through another object.
In second place, it is an immediate in general,
for it is supposed to be just a mirroring;
its self-reference is therefore abstract universality
and hence an existence open to others.
It does not suffice, in order to gain the freedom of substance,
to represent the latter as a totality that,
complete in itself, would have nothing to receive from the outside.
On the contrary, a self-reference that grasps nothing conceptually
but is only a mirroring is precisely a passivity towards the other.
Likewise the determinateness, whether we now take it as
the determinateness of a being that exists or that mirrors,
as a degree of the monad's own internally generated development,
is something external;
the degree that the development achieves has its limit in an other.
To project the reciprocal influence of substances
into a predetermined harmony means nothing more
than to make it a presupposition,
in effect to remove it from the scope of the concept.
The need to avoid the interaction of substances was founded
on the moment of absolute self-subsistence
and originariness which was made a fundamental assumption.
But since the positedness, the degree of development,
does not correspond to this assumed in-itselfness,
it has for this reason its ground in an other.

In connection with the relation of substantiality,
we showed that it passes over into the relation of causality.
But now the existent no longer has the
determination of a substance but that of an object;
the causal relation has come to an end in the concept;
the originariness of one substance vis-à-vis another
has shown itself to be a reflective shine,
the substance's action a transition into the opposite substance.
This relation has therefore no objectivity.
Hence in so far as one object is posited
in the form of subjective unity, as efficient cause,
this no longer counts as an originary determination
but as something mediated;
the active object has this determination
only by means of another object.
Mechanism, since it belongs to the sphere of the concept,
has that posited within it which proved to be
the truth of the relation of causality, namely,
that the cause which is supposed to be something
existing in and for itself is in fact
effect just as well, positedness.
In mechanism, therefore,
the originary causality of the object is
immediately a non-originariness;
the object is indifferent to this
determination attributed to it;
that it is a cause is therefore
something accidental to it.
To this extent, it can be said
that the causality of substances is
only the product of representation.
But precisely this causality
as product of representation
is what mechanism is;
for mechanism is this,
that causality,
as identical determinateness of
a diversity of substances
and hence as the foundering into
this identity of their self-subsistence,
is mere positedness;
the objects are indifferent to this unity
and maintain themselves in the face of it.
But this also, their indifferent self-subsistence,
is a mere positedness,
and for this reason they are capable of
mixing and aggregating,
and as an aggregate of becoming one object.
Through this indifference both
to their transition and to their self-subsistence,
the substances are objects.

a. The formal mechanical process

The mechanical process is the positing of
that which is contained in the concept of mechanism,
hence the positing in the first place of a contradiction.

1. It follows from the just indicated concept
that the interaction of objects is
the positing of their identical connection.
This positing consists simply in giving to
the determinateness which is generated
the form of universality and this is communication,
which occurs without transition into the opposite.
Spiritual communication, which however takes place in
an element of universality in the form of universality,
is an idealized connection for itself,
one in which a determinateness continues undisturbed from
one person to another, generalizing itself unaltered,
like a scent freely spreading in the unresisting atmosphere.
But also in the communication between material objects
does their determinateness widen, so to speak,
in an equally idealizing manner;
personality is an infinitely more
intensive hardness than objects possess.
The formal totality of an object in general,
a totality indifferent to determinateness
and hence not a self-determination,
renders the object indistinct from another object
and thus makes interaction at first an unimpeded continuing
of the determinateness of the one into the other.

Now in the region of the spirit there is
an infinitely manifold content capable of communication,
for by being taken up into intelligence the content receives
this form of universality in which it becomes communicable.
But that which is a universal not only by virtue of form,
but in and for itself, is the objective as such,
both in the region of the spirit and of the body,
and as against it the singularity of
external objects, as of persons also,
is an unessential factor unable to offer
any resistance to it.
Laws, morals, rational conceptions in general,
are in the region of the spirit communicable entities of this kind;
they pervade individuals unconsciously imposing themselves on them.
In the region of the body, such entities are
motion, heat, magnetism, electricity, and the like,
all of which, even when one wants to imagine them
as stuffs or materials,
must be termed as imponderable agents,
for they lack that aspect of materiality
that grounds its singularization.

2. Now although in the interaction of objects
their identical universality is posited first,
it is equally necessary to posit
the other moment of the concept,
that of particularity;
the objects thus also demonstrate their self-subsistence;
they hold themselves outside each other,
and in that universality they produce singularity.
This production is reaction in general.
To begin with, this reaction is not to be conceived of
as a mere sublation of action
and of the communicated determinateness;
what is communicated is as
universal positively present in the particular objects
and particularizes itself only in their diversity.
To this extent, therefore, what is communicated
remains what it is, only distributed among
the objects or determined by their particularity.
the activity of the causal substance in its action;
but the active object only becomes a universal;
its action is from the start not a loss of its determinateness
but a particularization by virtue of which the object,
which was at first that whole determinateness present in it as single,
now becomes a species of it,
and the determinateness is thereby posited
for the first time as a universal.
The two, the raising in communication of
the singular determinateness into universality;
and the particularization of it in distribution,
the reduction of what was only a one to a species
are one and the same.

Now reaction is equal to action.
First, this is manifested by the other object
taking over the entire universal;
and so it is now active against the first.
Thus its reaction is the same as the action,
a reciprocal repulsion of the impulse.
Second, what is communicated is the objective;
it therefore remains the substantial determination of
the object on the presupposition of their diversity;
the universal thus at the same time specifies itself in them,
and consequently each object does not simply give back
the whole action but possesses its specific share.
But, third, reaction is a wholly negative action
in so far as each object,
because of the elasticity of its self-subsistence,
repels within it the positedness of an other
and retains its self-reference.
The specific particularity of the determinateness
communicated in the objects,
what was before called species,
returns to singularity,
and the object asserts its externality
as against the communicated universality.
The action thereby passes over into rest.
It proves to be only a superficial,
transient alteration within the self-enclosed
indifferent totality of the object.

3. This return constitutes the product of the mechanical process.
Immediately, the object is presupposed as a singular
then as a particular as against another particular;
but finally as indifferent towards its particularity,
as universal.
The product is the totality of the concept
previously presupposed but now posited.
It is the conclusion in which the communicated universal
is united with singularity through the particularity of the object.
In rest, however, the mediation is posited
at the same time as sublated;
or again, what is posited is
that the product is indifferent to this determining of it
and that the received determinateness is external in it.

Accordingly the product is the same as the object
that first enters the process.
But at the same time that object is
first determined through this movement;
the mechanical object is, as such,
an object only as product,
for what it is, is only by virtue of
the mediation of an other in it.
It is as product that it thus is
what it was supposed to be in and for itself,
a composite, a mixture, a certain arrangement of parts,
in general such that its determinateness is
not self-determination but something posited.

Yet the result of the mechanical process is
not already there ahead of that process itself;
its end is not in its beginning, as in the case of purpose.
The product is in the object a determinateness
which is externally posited in it.
Hence this product is indeed
according to its concept the same as
what the object already is at the beginning.
But at the beginning the external determinateness is
not yet there as posited.
The result is therefore something
quite other than the first existence of the object,
and is something utterly accidental for it.

b. The real mechanical process

The mechanical process passes over into rest.
That is to say, the determinateness that the object obtains
through that process is only an external one.
Just as external to it is this rest,
for although the latter is a determinateness
opposed to the activity of the object,
the two are each indifferent to the object.
Rest can also be viewed, therefore,
as brought about by an external cause,
just as much as it was indifferent to
the object to be active.
Now further, since the determinateness is a posited one,
and the concept of the object has gone back to itself
through the process of mediation,
the object contains the determinateness
as one that is reflected into itself.
Hence in the mechanical process the objects
and the process itself now
have a more closely determined relation.
They are not merely diverse,
but are determinedly differentiated as against one another.
Consequently the result of the formal process,
on the one hand a determinationless rest, is,
on the other hand, through
the immanently reflected determinateness,
the distribution among several objects
mechanically relating to one another
of the opposition which is in the object as such.
The object that on the one hand lacks all determination,
showing no elasticity and no self-subsistence in its relations,
has, on the other hand, a self-subsistence
impenetrable to other objects.
Objects now have also as against one another
this more determined opposition of
the self-subsistent singularity
and the non-self-subsistent universality.
The precise difference between any two may be
had merely quantitatively as a difference in
a body of diverse magnitudes of mass,
or of intensity, or in various other ways.
But in general the difference cannot be fixed
at just this abstract level;
also as objects, both are
positively self-subsistent.

Now the first moment of this real process is,
as before, communication.
The weaker can be seized and invaded
by the stronger only in so far as
it accepts the stronger and constitutes one sphere with it.
Just as in the material realm the weaker is
secured against the disproportionately strong
(as a sheet hanging freely in the air is
not penetrated by a musket ball;
a weak organic receptivity is not
as vulnerable to strong stimuli as it is to weak),
so is the wholly feeble spirit safer facing the strong
than one who stands closer to the strong.
Imagine, if you will, someone dull-witted and ignoble;
lofty intelligence will make no impression on such a one,
nor will nobility.
The one single effective defense against reason is
not to get involved with it at all.
To the extent that an object that has no standing of its own is
unable to make contact with one which is self-subsistent,
and no communication can take place between them,
the latter is also unable to offer resistance, that is,
cannot specify the communicated universal for itself.
If they were not in the same sphere,
their mutual connection would be an infinite judgment
and no process would be possible between them.
Resistance is the precise moment of
the overpowering of the one object by the other,
for it is the initial moment in
the distribution of the communicated universal
and in the positing of the self-referring negativity,
of the singularity to be established.
Resistance is overpowered when its determinateness is
not commensurate to the communicated universal
which the object has accepted
and which is supposed to be singularized in the latter.
The object's relative lack of self-subsistence is
manifested in the fact that its singularity lacks the capacity
for what is communicated to it
and is therefore shattered by it,
for it is unable to constitute itself
as subject in this universal,
cannot make the latter its predicate.
Violence against an object is
for the latter something alien
only according to this second aspect.
Power becomes violence when power, an objective universality,
is identical with the nature of the object,
yet its determinateness or negativity is
not the object's own immanent negative reflection
according to which the object is a singular.
In so far as the negativity of the object is not
reflected back into itself in the power,
and the latter is not the object's own self-reference,
the negativity, as against the power,
is only abstract negativity
whose manifestation is extinction.

Power, as objective universality and as violence against
the object is what is called fate.
a concept that falls within mechanism
in so far as fate is called blind,
that is, its objective universality is
not recognized by the subject in
its own specific sphere.
To add a few more remarks on the subject,
the fate of a living thing is in general the genus,
for the genus manifests itself through the fleetingness of
the living individuals that do not possess it as genus
in their actual singularity.
Merely animate natures, as mere objects,
like other things at lower levels
on the scale of being,
do not have fate.
What befalls them is a contingency;
however, in their concept as objects
they are self-external;
hence the alien power of fate is
simply and solely their own immediate nature,
externality and contingency itself.
Only self-consciousness has fate in a strict sense,
because it is free,
and therefore in the singularity of its “I”
it absolutely exists in and for itself
and can oppose itself to its objective universality
and alienate itself from it.
By this separation, however, it excites against itself
the mechanical relation of a fate.
Hence, for the latter to have violent power over it,
it must have given itself some determinateness or other
over against the essential universality;
it must have committed a deed.
Self-consciousness has thereby made itself into a particular,
and this existence, like abstract universality,
is at the same time the side open to
the communication of its alienated essence;
it is from this side that it is
drawn into the process.
A people without deeds is without blame;
it is wrapped up in objective, ethical universality,
is dissolved into it,
is without the individuality
that moves the unmoved,
that gives itself a determinateness on the outside
and an abstract universality separated
from the objective universality;
yet in this individuality the subject is also
divested of its essence,
becomes an object and enters into the relation of
externality towards its nature,
into that of mechanism.

c. The product of the mechanical process

The product of formal mechanism is
the object in general,
an indifferent totality
in which determinateness is as posited.
The object has hereby entered
the process as a determinate thing,
and, in the extinction of this process,
the result is, on the one hand, rest,
the original formalism of the object,
the negativity of its determinateness-for-itself.
But, on the other hand, it is
the sublation of the determinateness,
the positive reflection of it into itself,
the determinateness that has withdrawn into itself,
or the posited totality of the concept,
the true singularity of the object.
The object, determined at first
in its indeterminate universality,
then as particular, is now
determined as an objective singular,
so that in it that reflective semblance of singularity,
which is only a self-subsistence opposing itself
to the substantial universality,
is sublated.

This resulting immanent reflection,
the objective oneness of the objects,
is now a oneness which is an individual self-subsistence:
the center.

Secondly, the reflection of negativity is
the universality which is not a fate
standing over against determinateness,
but a rational fate, immanently determined,
a universality that particularizes itself from within,
the difference that remains at rest and fixed
in the unstable particularity of the objects
and their process;
it is the law.

This result is the truth,
and consequently also the foundation,
of the mechanical process.

C. ABSOLUTE MECHANISM

a. The center

The empty manifoldness of the object is
now gathered first into objective singularity,
into the simple self-determining middle point.
Secondly, in so far as the object retains as
an immediate totality its indifference to determinateness,
the latter too is present in it as unessential
or as an outside-one-another of many objects.
As against this immediate totality,
the prior or the essential determinateness constitutes
the real middle term between the many interacting objects;
it unites them in and for themselves
and is their objective universality.
Universality exhibited itself first in
the relation of communication,
as present only through positing;
as objective universality, however,
it is the pervading immanent
essence of the objects.

In the material world it is the central body
which is the genus or rather the individualized universality
of the single objects and their mechanical process.
The unessential single bodies relate
to one another by impact and pressure;
this kind of relation does not hold between the central body
and the objects of which it is the essence;
for their externality no longer constitutes
their fundamental determination.
Hence their identity with the central body is rather rest,
namely the being at their center;
this unity is their concept existing in and for itself.
It nevertheless remains only an ought,
since the objects' externality, still posited at the same time,
does not conform to that unity.
The striving which the objects consequently have towards
the center is their absolute universality,
one which is not posited through communication;
it constitutes the true rest,
itself concrete and not posited from the outside,
into which the process of instability must find its way back.
It is for this reason an empty abstraction
to assume in mechanics that a body
set in motion would go on moving
in a straight line to infinity
if it did not lose movement
because of external resistance.
Friction, or whatever other form resistance takes,
is only a phenomenon of centrality;
it is the latter that in principle
brings the body back to itself,
since that against which the body rubs
and incurs friction has its power of resistance
only because it is united with the center.
In things spiritual the center,
and the union with it, assume higher forms;
but the unity of the concept
and the reality of that unity,
which is here in a first instance mechanical centrality,
must there too constitute the fundamental determination.

The central body has therefore ceased to be a mere object,
for in the latter the determinateness is something unessential,
whereas now the central body no longer
has only being-in-itself, in-itselfness,
but also has the being-for-itself
of the objective totality.
For this reason it can be regarded as an individual.
Its determinateness is essentially different
from a mere order or arrangement
and external combination of diverse parts;
as a determinateness that exists
in and for itself it is an immanent form,
a self-determining principle to which the objects inhere
and in virtue of which they are bound together in a true One.

But this central individual is at first only
a middle term that as yet has no true extremes;
as the negative unity of the total concept
it dirempts itself rather into such extremes.
Or again: the previously non-self-subsistent,
self-external objects become likewise determined as
individuals by the retreat of the concept;
the self-identity of the central body, still a striving,
is burdened by an externality
to which, in being taken up into
the body's objective singularity,
the latter is communicated.
The objects, through this centrality of their own,
are positioned outside the original center
and are themselves centers for the non-self-subsistent objects.
These second centers and the non-self-subsistent objects are
brought into unity by the absolute middle term.

But the relative individual centers themselves
also constitute the middle term of a second syllogism.
This middle term is on the one hand subsumed
under a higher extreme,
the objective universality and power of the absolute center;
on the other hand, it subsumes under it
the non-self-subsistent objects whose
superficiality and formal singularization it supports.
These non-self-subsistent objects are in turn
the middle term of a third syllogism,
the formal syllogism,
for since the central individuality obtains through
them the externality by virtue of which,
in referring to itself, is also strives
towards an absolute middle point,
those non-self-subsistent objects are
the link between absolute and relative central individuality.
The formal objects have for their essence
the identical gravity of their immediate central body
in which they inhere as in their subject
and the extreme of singularity;
through the externality which they constitute,
this immediate central body is subsumed under
the absolute central body;
they therefore are the formal middle term of particularity.
But the absolute individual is the objectively
universal middle term that brings into unity
and holds firm the inwardness of
the relative individual and its externality.
Similarly, the government, the individual citizens,
and the needs or the external life of these,
are also three terms,
of which each is the middle term of the other two.
The government is the absolute center
in which the extreme of the singulars is
united with their external existence;
the singulars are likewise the middle term that incites
that universal individual into external concrete existence
and transposes their ethical essence into the extreme of actuality.
The third syllogism is the formal syllogism,
the syllogism of reflective shine in which
the singular citizens are tied by their needs
and external existence to this universal
absolute individuality;
this is a syllogism that, as merely subjective,
passes over into the others and has its truth in them.

This totality, whose moments are themselves
the completed relations of the concept,
the syllogisms in which each of
the three different objects
runs through the determination of
the middle term and the extreme,
constitutes free mechanism.
In it the different objects have objective universality
for their fundamental determination,
the pervasive gravity that persists
self-identical in the particularization.
The connections of pressure, impact, attraction,
and the like, as also of aggregations or mixtures,
belong to the relation of externality
which is at the basis of the third of the three syllogisms.
Order, which is the merely external
determinateness of the objects,
has passed over into immanent
and objective determination.
This is the law.

b. The law

In law, the more specific difference of
the idealized reality of objectivity
versus the external reality comes into view.
The object, as the immediate totality of the concept,
does not yet possess an externality
differentiated from the concept,
and the latter is not posited for itself.
Now that through the mediation of the process
the object has withdrawn into itself,
there has arisen the opposition of
simple centrality as against an externality
now determined as externality,
that is, one posited as not existing in and for itself.
That moment of identity or idealization of individuality is,
on account of the reference to externality, an ought;
it is the unity of the concept,
determined in-and-for-itself and self-determining,
to which that external reality does not correspond,
and therefore does not go past the mere striving towards it.
But individuality is, in and for itself,
the concrete principle of negative unity,
and as such is itself totality;
it is a unity that dirempts itself
into the specific differences of the concept
while abiding within its self-equal universality;
it is thus the central point expanded
inside its pure ideality by difference.
This reality that corresponds to the concept is
the idealized reality,
distinct from the reality that is only a striving;
it is difference, earlier a plurality of objects
but now in its essential nature,
and taken up into pure universality.
This real ideality is the soul of
the hitherto developed objective totality,
the identity of the system
which is now determined in and for itself.

The objective being-in-and-for-itself
thus manifests itself more precisely
in its totality as the negative unity of the center,
a unity that divides into
subjective individuality and external objectivity,
maintains the former in the latter
and determines it in an idealized difference.
This self-determining unity that absolutely
reduces external objectivity to ideality
is a principle of self-movement;
the determinateness of this animating principle,
which is the difference of the concept itself, is the law.
Dead mechanism was the mechanical process of objects
above considered that immediately appeared as self-subsisting,
but precisely for that reason are in truth non-self-subsistent
and have their center outside them;
this process that passes over into rest
exhibits either contingency and indeterminate difference
or formal uniformity.
This uniformity is indeed a rule, but not law.
Only free mechanism has a law,
the determination proper to pure individuality
or to the concept existing for itself.
As difference, the law is in itself
the inexhaustible source of a self-igniting fire
and, since in the ideality of its difference
it refers only to itself,
it is free necessity.

c. Transition of mechanism

This soul is however still immersed in its body.
The now determined but inner concept of objective totality is
free necessity in the sense that the law
has not yet stepped in opposite its object;
it is concrete centrality as a universality
immediately diffused in its objectivity.
Such an ideality does not have, therefore,
the objects themselves for its determinate difference;
these are self-subsistent individuals of the totality,
or also, if we look back at the formal stage,
non-individual, external objects.
The law is indeed immanent in them
and it does constitute their nature and power;
but its difference is shut up in its ideality
and the objects are not themselves differentiated
in the idealized non-indifference of the law.
But the object possesses its essential self-subsistence
solely in the idealized centrality and its laws;
it has no power, therefore, to put up resistance
to the judgment of the concept
and to maintain itself in abstract, indeterminate
self-subsistence and remoteness.
Because of the idealized difference
which is immanent in it,
its existence is a determinateness posited by the concept.
Its lack of self-subsistence is thus no longer
just a striving towards a middle point,
with respect to which, precisely because
its connection with it is only that of a striving,
it still has the appearance of
a self-subsistent external object;
it is rather a striving towards the object
determinedly opposed to it;
and likewise the center has itself
for that reason fallen apart
and its negativity has passed over
into objectified opposition.
Centrality, therefore, is
now the reciprocally negative
and tense connection of these objectivities.
Thus free mechanism determines itself to chemism.
