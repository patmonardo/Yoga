
1. Accordingly, what constitutes the method are
the determinations of the concept itself and their connections,
and these we must now examine in the significance
that they have as determinations of the method.
In this, we must begin from the beginning.
We spoke of this beginning at the very beginning of the Logic,
and also in connection with subjective cognition,
and we showed that, when not performed arbitrarily
and in the absence of categorial sensitivity,
though it may seem to present many difficulties,
it is nevertheless of an extremely simple nature.
Because it is the beginning, its content is an immediate,
but one that has the meaning and the form of abstract universality.
Or be it a content of being, or of essence or of the concept,
inasmuch as it is something immediate, it is assumed,
found in advance, assertoric.
But first of all it is not an immediate of
sense-intuition or of representation,
but of thought, which because of its immediacy can
also be called a supersensuous, inner intuiting.
The immediate of sense-intuition is a manifold and a singular.
Cognition, on the contrary, is a thinking that conceptualizes;
its beginning, therefore, is also only in the element of thought,
a simple and a universal.
We spoke of this form earlier, in connection with definition.
At the beginning of finite cognition universality is likewise
recognized as an essential determination,
but only as thought and concept
determination in opposition to being.
In fact this first universality is an immediate universality,
and for that reason it has equally the significance of being,
for being is precisely this abstract self-reference.
Being has no need of further derivation,
as if it came to the abstract element of definition
only because taken from the intuition of the senses or elsewhere,
and in so far as it can be pointed at.
This pointing and deriving involve a mediation
that is more than a mere beginning,
and is a mediation of a kind that does not
belong to the comprehension of thought,
but is rather the elevation of representation,
of empirical and ratiocinative consciousness,
to the standpoint of thinking.
According to the currently accepted opposition
of thought, or concept, and being,
it passes as a very important truth that
no being belongs as yet to thought as thought,
and that being has a ground of its own independent of thought.
But the simple determination of being is in itself so poor that,
if for that reason alone, not much fuss ought to be made about it;
the universal is immediately itself this immediate
because, as abstract, it is also the
abstract self-reference which is being.
In fact, the demand that being should be exhibited has
a further, inner meaning in which more is at issue
than just this abstract determination;
implied in it is the demand for the realization of the concept,
a realization that is missing at the beginning itself
but is rather the goal and the business of the
entire subsequent development of cognition.
Further, inasmuch as the content of the beginning is
to be justified and authenticated as something true or correct
by being exhibited in inner or outer perception,
it is no longer the form of universality as such
that is meant, but its determinateness,
about which more in a moment.
The authentication of the determinate content
with which the beginning is made seems to lie behind it,
but is in fact to be regarded as an advance,
in so far as it is a matter of conceptual cognition.

The beginning, therefore, has for the method no other determinateness
than that of being the simple and universal;
this is precisely the determinateness that makes it deficient.
Universality is the pure, simple concept,
and the method, as the consciousness of this concept,
is aware that universality is only a moment
and that in it the concept is still not determined in and for itself.
But with this consciousness that would want to carry
the beginning further only for the sake of method,
the method is only a formal procedure
posited in external reflection.
Where the method, however,
is the objective and immanent form,
the immediate character of the beginning must be
a lack inherent in the beginning itself,
which must be endowed with the
impulse to carry itself further.
But in the absolute method the universal
has the value not of a mere abstraction
but of the objective universal, that is,
the universal that is in itself the concrete totality,
but a totality as yet not posited, not yet for itself.
Even the abstract universal is as such,
when considered conceptually, that is, in its truth,
not just anything simple, but is, as abstract,
already posited afflicted by a negation.
For this reason also there is nothing so simple and so abstract,
be it in actuality or in thought, as is commonly imagined.
Anything as simple as that is a mere presumption
that has its ground solely in the lack of
awareness of what is actually there.
We said earlier that the beginning is
made with the immediate;
the immediacy of the universal is the same as
what is here expressed as the in-itself
that is without being-for-itself.
One may well say, therefore, that every
beginning must be made with the absolute,
just as every advance is only the exposition of it,
in so far as implicit in existence is the concept.
But because the absolute exists
at first only implicitly, in itself,
it equally is not the absolute
nor the posited concept,
and also not the idea,
for the in-itself is only
an abstract, one-sided moment,
and this is what they are.
The advance is not, therefore, a kind of superfluity;
this is what it would be if that which is
at the beginning were already the absolute;
the advance consists rather in this,
that the universal determines itself
and is the universal for itself,
that is, equally a singular and a subject.
Only in its consummation is it the absolute.

It may also be mentioned that a beginning
which is in itself a concrete totality may as such
also be free and its immediacy have
the determination of an external existence;
the germ of anything living,
and subjective purpose in general,
have shown themselves to be such beginnings;
hence both are themselves impulses.
The non-spiritual and inanimate, on the contrary,
are the concrete concept only as real possibility;
cause is the highest stage in which the concrete concept has,
as the beginning in the sphere of necessity,
an immediate existence;
but it is not yet a subject
that maintains itself as such
in the course of its effective realization.
The sun, for instance,
and in general all things inanimate,
are determinate concrete existences
in which real possibility remains an inner totality;
the moments of the latter are not posited in them
in subjective form and therefore,
in so far as they are realized,
they attain concrete existence
through other corporeal individuals.

2. The concrete totality which makes the beginning
possesses as such, within it, the beginning of
the advance and development.
As concrete, it is differentiated in itself,
but because of its initial immediacy, this first
differentiation is to start with a diversity.
However, as self-referring universality, as subject,
the immediate is also the unity of this diversity.
This reflection is the first stage of the forward movement
the emergence of non-indifference, judgment, and determining in general.
Essential is that the method find, and recognize,
the determination of the universal within it.
Whatever in this abstractive generation of the universal
is left out of the concrete is then picked up, still externally,
by the finite cognition of the understanding.
This is how the latter operates.
The absolute method, on the contrary, does not behave
in this manner of external reflection
but takes the determinate from its subject matter,
for it is itself its immanent principle and its soul.
This is what Plato demanded of cognition,
that it should consider things in and for themselves;
on the one hand, that it should consider them in their universality;
on the other hand, that it should not stray away from them
while it grasps at circumstances, examples, and comparisons,
but, on the contrary, should keep only them in view before it
and bring to consciousness what is immanent in them.
To this extent the method of absolute cognition is analytic.
That the method finds the further determinations
of its initial universal simply and solely in this universal,
constitutes the concept's absolute objectivity,
of which the method is the certainty.
Equally so, however, is the method synthetic,
for its subject matter,
while immediately determined as the simple universal,
through the determinateness which it has
in its very immediacy and universality,
proves to be an other.
Yet this connection in diversity
that the subject matter is thus in itself,
is no longer a synthesis as understood in finite cognition;
the no less thoroughly analytic determination of the subject matter,
the fact that the connection is within the concept,
already distinguishes it fully from the latter synthesis.

This no less synthetic than analytic moment of the judgment
through which the initial universal determines itself
from within itself as the other of itself
is to be called the dialectical moment.
Dialectic is one of those ancient sciences
that have been the most misjudged in
the metaphysics of the moderns,
and in general also by popular philosophy,
both ancient and recent.
Diogenes Laërtius said of Plato that,
just as Thales was the founder of natural philosophy
and Socrates of moral philosophy,
so Plato was the founder of the third of
the sciences that belong to philosophy,
of dialectic, a contribution for which
he was highly esteemed by the ancients
but that often goes quite unnoticed
by those who have the most to say about him.
Dialectic has often been regarded as an art,
as if it rested on a subjective talent
and did not belong to the objectivity of the concept.
What shape it received in Kantian philosophy,
and with what result, has already been
indicated in representative examples of that philosophy's view.
It must be regarded as an infinitely important step
that dialectic is once more being
recognized as necessary to reason,
although the result that must be drawn from it
is the opposite than Kant drew.

When dialectic is not presented, as it generally is,
as something incidental, it usually assumes
the following more precise form.
It is shown of a subject matter or other
(for instance: world, movement, point, and so on)
that a certain determination accrues to it
(for instance, in the order of the just mentioned examples:
finitude in space or time, being at this place,
absolute negation of space),
also the opposite determinations can then
just as necessarily be shown to accrue to it
(for example: infinity in space and time,
not-being at this place,
reference to space and hence spatiality).
The older Eleatic school directed
its dialectic especially against motion;
Plato commonly did it against accepted
notions and concepts of his time,
in particular those of the Sophists,
but also against the pure categories
and the determinations of reflection;
the later and more sophisticated form of skepticism
extended it not only to the immediate so-called facts of consciousness
and the maxims of ordinary life,
but also to all scientific concepts.
Now the conclusion drawn from this kind of dialectic is
in general the contradiction and nullity of the asserted claims.
But this can happen in two ways either in the objective sense,
that the subject matter that thus contradicts itself
internally cancels itself and is a non-thing
(this was, for instance, the conclusion of the Eleatics,
who denied the truth of the world, of movement, of the point);
or in the subjective sense, that cognition is deficient.
Now understood in this last subjective sense,
the conclusion may be taken in two further ways.
It may mean that it is this dialectic itself
that generates the artifice of an illusion.
This is the common view of the so-called
healthy common sense that takes its stand
on the evidence of the senses
and on customary notions and claims,
at times quietly, like Diogenes the cynic did,
who demonstrated the vacuity of the dialectic of motion
by silently walking up and down;
but often by getting itself all worked up,
declaring that dialectic is mere foolery or,
when important ethical matters are at issue,
the criminal attempt at unsettling essentially solid norms
and providing excuses for the wicked,
a view we see directed in the Socratic dialectic
against that of the Sophists, with an ire that,
turned into the opposite direction,
even cost Socrates his life.
As for the vulgar refutation that opposes to thinking,
as Diogenes did, sensuous consciousness
and in this latter believes that it finds the truth,
this we must leave to itself;
but in so far as dialectic sublates ethical determinations,
we must have confidence in reason that it will
know how to reinstate them,
but reinstate them in their truth
and in the consciousness of their right,
though also of their limitations.
Yet another view is that the
result of subjective nullity has
nothing to do with dialectic itself,
but that it affects the cognition
against which it is directed
and, in the view of skepticism
and likewise of the Kantian philosophy,
cognition in general.

The fundamental prejudice here is that
dialectic has only a negative result,
a point about which more in a moment.
First, regarding the said form in which dialectic
usually makes its appearance,
it is to be observed that according to that form
the dialectic and its result affect
a subject matter which is previously assumed
or also the subjective cognition of it,
and declare either the latter or the subject matter
to be null and void,
while, on the contrary, no attention is given to
the determinations which are exhibited in the subject matter
as in a third thing and presupposed as valid for themselves.
To have called attention to this uncritical procedure has
been the infinite merit of the Kantian philosophy,
and in so doing to have given the impetus to the restoration
of logic and dialectic understood as the examination of
thought determinations in and for themselves.
The subject matter, as it is apart from thought and conceptualization,
is a picture representation or also a name;
it is in the determinations of thought and of the concept
that it is what it is.
In fact, therefore, everything rests on these determinations;
they are the true subject matter and content of reason,
and anything else that might be understood by subject matter and content
in distinction from them has meaning only through them and in them.
It must not therefore be taken as the fault of a subject matter
or of the cognition that these determinations,
because of what they are and the way they are externally joined,
prove to be dialectical.
On this assumption, the subject matter and the cognition
are imagined to be a subject on which
the determinations are brought to bear,
in the form of predicates, properties,
or self-subsistent universals,
as fixed and independently correct,
so that these determinations are
brought into dialectical relations
and incur contradiction only by
extraneous and contingent conjunction
in and by a third thing.
But this kind of external and fixed
subject of imagination and understanding,
and also these abstract determinations,
cannot be regarded as ultimates,
as secure and permanent substrates.
On the contrary, they are to be regarded as themselves immediate,
precisely the kind of presuppositions and starting points which,
as we have shown above, must succumb to dialectic in and for themselves,
because they are to be taken as in themselves the concept.
The same applies to all oppositions that are assumed as fixed,
as for example the finite and the infinite, the singular and the universal.
These are not in contradiction through some external conjoining;
on the contrary, as an examination of their nature shows,
they are a transition in and for themselves;
the synthesis and the subject in which they appear is
the product of their concept's own reflection.
If a consideration that avoids
the concept stops short at their external relation,
isolates them and leaves them as fixed presuppositions,
it is the concept that,  on the contrary,
will fix its sight on them,
move them as their soul and bring out their dialectic.

Now this is the very standpoint indicated above
from which a universal prius,
considered in and for itself,
proves to be the other of itself.
Taken quite generally, this determination
can be taken to mean that what is at first immediate
is therewith posited as mediated, as referred to an other,
or that the universal is posited as a particular.
The second universal that has thereby arisen is
thus the negative of that first
and, in view of subsequent developments, the first negative.
From this negative side, the immediate has perished in the other;
but the other is essentially not an empty negative,
the nothing which is normally taken to be the result of dialectic,
but is rather the other of the first,
the negative of the immediate;
it is therefore determined as the mediated,
contains as such the determination of the first in it.
The first is thus essentially preserved and contained also in the other.
To hold fast to the positive in its negative,
to the content of the presupposition in the result,
this is the most important factor in rational cognition;
what is more, it takes only the simplest of reflections
to be convinced of the absolute truth and necessity of this requirement,
and as for examples of proofs that testify to this,
the whole Logic consists of such proofs.

So what we now have, taken first or also immediately, is the mediated,
also a simple determination,
for the first has perished in it,
and only the second is therefore at hand.
Now since the first is contained in the second,
and this second is the truth of the first,
this unity of the two can be expressed
in the form of a proposition in which
the immediate is placed as the subject
but the mediated as its predicate;
for example, “the finite is infinite,”
“one is many,” “the singular is the universal.”
The inadequacy of the form of such
propositions and judgments is however obvious.
In connection with judgment it was shown that its form in general,
and most of all the immediate form of the positive judgment,
is incapable of holding within its grasp
the speculative content and the truth.
Its closest complement, the negative judgment,
would have to be brought in at least in equal measure.
In judgment the first, as subject, conveys the
reflective semblance of an independent subsistence,
whereas it is in fact sublated in the predicate as in its other;
this negation is indeed contained in
the content of the above propositions,
but their positive form contradicts the content;
consequently, what is contained in them is not posited
whereas this was precisely the intent
behind the use of a proposition.

The second determination, the negative or mediated determination,
is moreover at the same time the one that mediates.
At first it may be taken as a simple determination,
but its truth is that it is a reference or relation;
for it is the negative, but the negative of the positive,
and it includes this positive within itself.
It is the other, therefore, not of a one to which it is indifferent;
in that case it would not be an other, nor a reference or relation.
It is rather the other in itself, the other of an other;
hence it includes its own other within itself
and is consequently the contradiction, the posited dialectic, of itself.
Because the first or the immediate is the concept in itself or implicitly,
and therefore is the negative also only implicitly,
the dialectical moment in it consists in the positing
of the difference that is implicitly contained in it.
The second is on the contrary itself the determinate,
the difference or relation;
hence the dialectical moment consists in its case
in the positing of the unity contained within it.
For this reason, if the negative, the determinate, relation, judgment,
and all the determinations falling under this second moment,
do not appear by themselves already as contradiction,
as dialectical, this is solely a defect on the part
of thinking that fails to bring its thoughts together.
For the material, the opposed determinations in one connection,
are already posited, already present for thought.
But formal thinking makes identity its law,
lets the contradictory content that it has before it
fall into the sphere of representation, in space and time,
where the contradictory is held in external moments,
next to and following each other,
parading before consciousness without reciprocal contact.
The firm principle that formal thinking lays down for itself here
is that contradiction cannot be thought.
But in fact the thought of contradiction is
the essential moment of the concept.
Formal thought does in fact think it,
only it at once looks away from it
and stating its principle it only passes over
from it into abstract negation.

Now the negativity just considered constitutes
the turning point of the movement of the concept.
It is the simple point of the negative self-reference,
the innermost source of all activity,
of living and spiritual self-movement;
it is the dialectical soul
which everything true possesses
and through which alone it is true;
for on this subjectivity alone rests
the sublation of the opposition
between concept and reality,
and the unity which is truth.
The second negative at which we have arrived,
the negative of the negative,
is this sublating of contradiction,
and it too, just like contradiction,
is not an act of external reflection;
for it is on the contrary the innermost,
objective moment of the life of spirit
by virtue of which a subject is a person, is free.
The self-reference of the negative is to be regarded
as the second premise of the entire syllogism.
If the terms analytic and synthetic are used as opposites,
the first premise may be regarded as the analytic moment,
for in it the immediate relates to its other immediately
and therefore passes over, or rather has passed over,
into it though this connection, as already remarked,
is for this very reason also synthetic,
for it is its other that it passes over into.
The second premise considered here
may be defined as synthetic,
because it is the connection of
the differentiated, as differentiated,
to that from which it is differentiated.
Just as the first premise is the moment
of universality and communication,
so is the second determined by singularity;
a singularity which in referring to the other is
at first exclusive, for itself, and different.
The negative appears as the mediating factor,
because it holds itself and the immediate
of which it is the negation within itself.
In so far as these two determinations are taken
as referring to each other externally
in some relation or other,
the negative is only the formal mediating factor;
but, as absolute negativity,
the negative moment of absolute mediation is
the unity which is subjectivity and soul.

In this turning point of the method,
the course of cognition returns at the
same time back into itself.
This negativity is as self-sublating contradiction
the restoration of the first immediacy, of simple universality;
for the other of the other, the negative of the negative,
is immediately the positive, the identical, the universal.
In the whole course, if one at all cares to count,
this second immediate is third to the first immediate and the mediated.
But it is also third to the first or formal negative
and to the absolute negativity or second negative;
now in so far as that first negative is already the second
term, the term counted as third can also be counted as fourth,
and instead of a triplicity, the abstract form
may also be taken to be a quadruplicity;
in this way the negative or the difference
is counted as a duality.
The third or the fourth is in general
the unity of the first and the second moment,
of the immediate and the mediated.
That it is this unity,
or that the entire form of the method is a triplicity,
is indeed nothing but the merely superficial,
external side of cognition;
but to have also demonstrated this superficiality,
and to have done it in the context of a specific application
(for the abstract form of number has been around for a long time,
as is well known, but without conceptual comprehension
and therefore without any result)
is again to be regarded as an infinite merit of the Kantian philosophy.
The syllogism, or the threefold, has always been recognized
to be the universal form of reason;
but it has had in general the value of a wholly external form
that does not determine the nature of the content;
moreover, since in its formalism it gets caught up
in the understanding's determination of mere identity,
it lacks the essential dialectical moment of negativity;
and yet this moment enters into the triplicity of the determinations,
because the third term is the unity of the two first determinations
and these, since they are diverse, can be in unity only as sublated.
Formalism, it is true, has also seized hold of triplicity,
attending to its empty schema;
the shallow nonsense and the barrenness of
the so-called construction of modern philosophy,
that consists in nothing but fastening that formal schema
everywhere for the sake of external order,
with no concept or immanent determination,
has rendered that form tedious and has given it a bad name.
Yet the insipidity of this use cannot rob it of its inner worth,
and the fact that the shape of reason was discovered,
albeit without conceptual comprehension at first,
is always to be highly valued.

Now, on closer examination, the third is the immediate,
but the immediate through sublation of mediation,
the simple through the sublating of difference,
the positive through the sublating of the negative;
it is the concept that has realized itself through its otherness,
and through the sublating of this reality has rejoined itself
and has restored its absolute reality,
its simple self-reference.
This result is therefore the truth.
It is just as much immediacy as mediation
though these forms of judgments,
that the third is immediacy and mediation,
or that it is the unity of the two,
are not capable of grasping it,
for it is not a dormant third
but, exactly like this unity,
self-mediating movement and activity.
Just as that with which we began was the universal,
so the result is the singular, the concrete, the subject;
what the former is in itself, the latter is now equally for itself:
the universal is posited in the subject.
The two first moments of triplicity are abstract,
untrue moments that are dialectical for that very reason,
and through this their negativity make themselves into the subject.
For us at first, the concept itself is
both the universal that exists in itself
and the negative that exists for itself,
and also the third term that exists in and for itself,
the universal that runs through all the moments of the syllogism;
but this third is the conclusion in which
the concept mediates itself with itself through its negativity
and is thereby posited for itself as the universal
and the identity of its moments.

Now this result, as the whole that has withdrawn into itself
and is identical with itself, has given itself again
the form of immediacy.
Consequently, it is now itself all that
the starting point had determined itself to be.
As simple self-reference it is a universal,
and in this universal the negativity that
constituted its dialectic and mediation has
likewise withdrawn into simple determinateness,
which can again be a beginning.
It may seem at first that this cognition of the result
is an analysis of it
and would therefore have to dissect these determinations again,
and the course that it went through in order to come to be
the course that we have examined.
But if the subject matter were in fact treated analytically in this manner,
it would belong to that stage of the idea considered above,
a mode of cognition that searches for its subject matter
and only states of it what it is,
without the necessity of its concrete identity and of its concept.
But the method of truth that comprehends the subject matter,
though analytic as we have seen,
since it remains strictly within the concept,
is however equally synthetic,
for through the concept the subject matter is
determined as dialectical and as other.
On the new foundation that the result
has now constituted as the subject matter,
the method remains the same as in the preceding subject matter.
The difference concerns solely the
status of the foundation as such;
although it is certainly still a foundation,
its immediacy is only form,
since it was a result as well;
hence its determinateness as content is
no longer something merely taken up
but is deduced and proved.

It is here that the content of cognition first enters
as such into the circle of consideration,
because as deduced it now belongs to the method.
The method itself expands with this moment into a system.
With respect to content, the beginning has to be
for the method at first wholly indeterminate;
to this extent the method appears as the merely formal soul,
for which and by which the beginning was determined
simply and solely only according to form,
that is to say, as the immediate and universal.
In the course of the movement we have indicated,
the subject matter has received a determinateness for itself
and this determinateness is a content,
for the negativity that has withdrawn
into simplicity is the sublated form,
and stands as simple determinateness over against its development,
and in the first instance against its very opposition to universality.

Now since this determinateness is the
proximate truth of the indeterminate beginning,
it denounces the incompleteness of the latter,
and it also denounces the method itself
which, starting from that beginning, was only formal.
This can now be expressed as the henceforth
determinate demand that the beginning,
since as against the determinateness of the result
it is itself something determinate,
ought to be taken not as immediate,
but as mediated and deduced.
This may appear as the demand for
an infinite retrogression in proof and deduction;
just as from the newly obtained beginning a result
likewise emerges as the method runs its course,
so that the movement would roll on forwards to infinity as well.

It has been repeatedly shown that
the infinite progression as such belongs
to a reflection void of concept;
the absolute method, which has the concept
for its soul and content, cannot lead into it.
Even such beginnings as being, essence, universality,
may seem at first to be of the kind that possess
the full universality and complete absence of content
that is required for an entirely formal beginning,
such as the beginning is supposed to be,
and therefore not to require or allow,
as absolutely first beginnings, further regress.
Since they refer purely to themselves,
they are immediate and indeterminate,
and so they do not of course have in them the difference
which is straightaway posited in some other beginning
between the universality of its form and its content.
But the very indeterminacy which these logical beginnings have
as their sole content is what constitutes their determinateness;
this determinateness consists in their negativity,
as sublated mediation;
the particularity of this negativity gives
a particularity also to their indeterminacy,
and it is by virtue of it that
being, essence, and universality, are differentiated.
Now the determinateness that accrues to them
when taken for themselves is their immediate determinateness,
and this is just as immediate as that of any content
and in need, therefore, of derivation;
for the method it is a matter of indifference
whether the determinateness is taken
as determinateness of form or of content.
That it gives itself a determination by
the first of its results does not mean that,
in fact, it is thereby set on a new footing;
it remains neither more nor less formal than before.
For since the method is the absolute form,
the concept that knows itself and everything as concept,
there is no content that would stand out over against it
and determine it as a one-sided external form.
Hence, just as the lack of content of the said beginnings
does not make them absolute beginnings,
so too it is not the content that would as such lead
the method into the infinite progress forwards or backwards.
In one respect, the determinateness that the method generates for itself
in its result is the moment through which it is self-mediation
and converts the immediate into a mediated beginning.
But conversely, it is through that determinateness
that this mediation of the method runs its course;
it goes through a content, as through a seeming other of itself,
back to its beginning, in such a way that it does not merely
restore that beginning, albeit as determinate,
but that the result is equally the sublated determinateness,
and hence also the restoration of the first immediacy in which it began.
This it accomplishes as a system of totality.
We now have to consider it in this determination.

The determinateness which was the result is,
as we have shown, itself a new beginning
because of the form of simplicity
into which it has withdrawn;
since this beginning is distinguished
from the one preceding it by this very determinateness,
cognition rolls onwards from content to content.
First of all, this forward movement determines itself
in that it begins from simple determinacies,
and the following become ever richer and more concrete.
For the result contains its beginning and its course
has enriched it with a new determinateness.
The universal constitutes the foundation;
the advance is not to be taken, therefore,
as a flowing from other to other.
In the absolute method,
the concept maintains itself in its otherness,
the universal in its particularization,
in judgment and reality;
at each stage of further determination,
the universal elevates the whole mass of its preceding content,
not only not losing anything through its dialectical advance,
or leaving it behind, but, on the contrary,
carrying with itself all that it has gained,
inwardly enriched and compressed.

This expansion may be regarded as
the moment of content,
and in the whole as the first premise;
the universal is communicated to the wealth of content,
is immediately received in it.
But the relation has also a second,
negative or dialectical side.
The enrichment proceeds in the
necessity of the concept,
it is contained by it,
and every determination is
a reflection into itself.
Each new stage of exteriorization,
that is, of further determination,
is also a withdrawing into itself,
and the greater the extension,
just as dense is the intensity.
The richest is therefore
the most concrete and the most subjective,
and that which retreats to the simplest depth is
the mightiest and the most all-encompassing.
The highest and most intense point is
the pure personality that,
solely by virtue of the
absolute dialectic which is its nature,
equally embraces and holds everything within itself,
for it makes itself into the supremely free,
the simplicity which is
the first immediacy and universality.

It is in this manner that each step of the advance
in the process of further determination,
while getting away from the indeterminate beginning,
is also a getting back closer to it;
consequently, that what may at first appear to be different,
the retrogressive grounding of the beginning
and the progressive further determination of it,
run into one another and are the same.
The method, which thus coils in a circle,
cannot however anticipate in a temporal development
that the beginning is as such already something derived;
sufficient for an immediate beginning is that it be simple universality.
Inasmuch as this is what it is, it has its complete condition;
and there is no need to deprecate the fact that
it may be accepted only provisionally and hypothetically.
Whatever might be adduced against it about
the limitations of human cognition;
about the need to reflect critically
on the instrument of cognition
before getting to the fact itself;
all these are themselves presuppositions,
concrete determinations that as such carry
with them the demand for mediation and grounding.
Therefore, since they formally have no advantage
over beginning with the fact itself as they protest against,
and, because of their more concrete content,
are on the contrary all the more in need of derivation,
singling them out for special attention
is to be considered as empty presumption.
They have an untrue content,
for they make into something
incontestable and absolute
what is known to be finite and untrue,
namely a restricted cognition
determined as form and instrument
in opposition to its content;
this untrue cognition is itself also the form,
the retroactive search for grounds.
The method of truth also knows that the beginning is incomplete,
because it is a beginning;
but at the same time it knows that
this incompleteness is necessary,
because truth is but the coming-to-oneself
through the negativity of immediacy.
The impatience that would merely transcend the determinate
be it called beginning, object, the finite,
or in whatever other form it is otherwise taken
in order that one would find oneself immediately in the absolute,
has nothing before it as cognition
 but the empty negative, the abstract infinite.
Or what it has before it is a presumed absolute,
presumed because not posited, not comprehended;
comprehended it will be only through the mediation of cognition,
of which the universal and immediate are a moment,
and as for the truth itself,
it resides only in the extended course of mediation and at the end.
To meet the subjective need and the impatience
that come with not knowing,
one may well provide an overview of the whole in advance
by means of a division for reflection that,
in the manner of finite cognition,
gives the particular of the universal as already there, to be waited
for as the science progresses.
Yet this affords nothing more than a picture
for representation;
for the true transition from the universal
to the particular and to the whole
which is determined in and for itself
and in which that first universal is in truth
itself again a moment;
this transition is alien to the division of reflection
and is the exclusive mediation of science itself.

By virtue of the nature of the method just indicated,
the science presents itself as a circle that winds around itself,
where the mediation winds the end back to the beginning
which is the simple ground;
the circle is thus a circle of circles,
for each single member ensouled by the method
is reflected into itself so that, in returning to the beginning
it is at the same time the beginning of a new member.
Fragments of this chain are the single sciences,
each of which has a before and an after
or, more accurately said,
has in possession only the before
and in its conclusion points to its after.

So the logic also has returned in the absolute idea
to this simple unity which is its beginning;
the pure immediacy of being,
in which all determination appears at first
as extinguished or removed by abstraction,
is the idea that through mediation,
that is, the sublation of mediation,
has come to the likeness corresponding to it.
The method is the pure concept
that only relates to itself;
it is, therefore, the simple self-reference which is being.
But it now is also the fulfilled concept,
the concept that comprehends itself conceptually,
being as the concrete and just as absolutely intensive totality.
