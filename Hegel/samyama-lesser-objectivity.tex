B. THE OBJECT

The object is immediate being by virtue of
the indifference towards the difference
that has sublated itself in it.
It is in itself the totality
and, at the same time,
since this identity is the identity of the moments
but an identity that only is in itself,
it is just as indifferent to its immediate unity.
It breaks down into differentiated [moments],
each of which is itself the totality.
The object is thus the absolute contradiction of
the complete self-sufficiency of the manifold
and the equally complete lack of self-sufficiency
of the differentiated [moments].

The definition 'the absolute is the object'
is contained in the most determinate manner
in the Leibnizian monad which is supposed to be an object,
yet [an object] in itself representing [things]
and, indeed, is supposed to be the totality of
the representation of the world.
In its simple unity,
every difference is merely something ideal,
something not self-sufficient.
Nothing enters into the monad from the outside;
it is in itself the entire concept,
only differentiated by its own
greater or lesser development.
By the same token, this simple totality breaks down
into the absolute plurality of differences such
that they are self-sufficient monads.
In the monad of monads and the pre-established harmony
of their inner developments,
these substances are just as much reduced in turn
to the level of something ideal
and lacking in self~sufficiency.
The Leibnizian philosophy is thus
the perfectly developed contradiction.

a. Mechanism

The object, taken first in its immediacy, is
(1) the concept only in itself,
it has the concept at first
as something subjective outside it,
and every determinacy is posited
as an external determinacy.
As the unity of differences,
it is thus something composite, an aggregate,
and the effect on another remains
an external relation: formal mechanism.
In this relation and lack of self-sufficiency,
the objects remain equally self-sufficient,
resistant, external to one another.

Just as pressure and impulse are mechanical relationships,
so we also know in a mechanical way, by rote,
insofar as the words are devoid of any sense for us
and remain external to the senses, representing, thinking;
they [the words] are equally external to themselves,
a senseless sequence.
Acting, piety, and so forth are equally mechanical
insofar as what a person does is determined by
ceremonial laws, a counsel of conscience, and so forth,
while his own spirit and will are not in his actions,
such that these actions are external to him himself.

Only insofar as the object is self-sufficient
(see the preceding section)
does it have the lack of self-sufficiency
in terms of which it suffers violence.
Insofar as the object is the posited concept in itself,
neither of these determinations sublates itself
in its other determination;
instead the object joins itself together with itself
through the negation of itself,
through its lack of self-sufficiency,
and only then is it self-sufficient.
Thus, at the same time, in the difference from externality,
and in its self-sufficiency negating this externality,
it [the object] is the negative unity with itself, centrality,
subjectivity in which it is itself directed and related to the external.
The latter is equally centred in itself
and, in that, just as much related
to the other centre,
having its centrality just as much in the other.
[Hence, the object in the second place is]
(2) a differentiated mechanism
(fall, desire, social drive, and the like).

The development of this relationship
forms the syllogistic inference
that the immanent negativity
as the central individuality of an object (the abstract centre)
relates itself to objects lacking self-sufficiency
as the other extreme,
relating to them through a middle [term]
that unifies the objects' centrality
and lack of self-sufficiency,
the relative centre.
[Hence, the object is]
(3) absolute mechanism.

The syllogism that has been given here (I-P-U)
is a triad of syllogistic inferences.
The flawed individuality of the objects lacking self-sufficiency,
in which the formal mechanism is at home, is,
in keeping with its lack of self-sufficiency,
just as much the external universality.
These objects are thus the middle also
between the absolute and the relative centre
(the form of the syllogism: U-I-P).
For it is by means of this lack of self-sufficiency
that those two are separated and are extremes
just as they are related to one another.
So, too, the absolute centrality as the substantial universal
(the gravity that remains identical),
which as the pure negativity
also encapsulates in itself the individuality,
is the mediating factor between the relative centre
and the objects lacking self-sufficiency;
(thus amounting toJ the form of the inference P-U-I)
and, to be sure, just as essential
in terms of the immanent individuality
where it functions to separate,
as it is in terms of the universality
as the identical cohesion and
as the undisturbed being-in~itself.

Like the solar system, the state, for instance, is,
in the practical sphere, a system of three syllogisms.
(1) The individual (the person) joins itself through its particularity
(physical and spiritual needs, what becomes the civil society,
once they have been further developed for themselves)
with the universal (the society, justice, law, government).
(2) The will, the activity of individuals, is the mediating factor
which satisfies the needs in relation to society, the law, and so forth,
just as it fulfils and realizes the society, the law, and so forth.
(3) But the universal (state, government, law)
is the substantial middle [term]
in which the individuals and their satisfaction
have and acquire their fulfilled reality, mediation, and subsistence.
Since the mediation joins each of the determinations with the other extreme,
each joins itself precisely in this way together with itself;
it produces itself and this production is its self-preservation.
It is only through the nature of this joining together,
through this triad of syllogisms with the same terminis,
that a whole is truly understood in its organization.

The immediacy of concrete existence
that objects have in absolute mechanism
is in itself negated by the fact that
their self-sufficiency is mediated by
their relations to one another, hence,
through their lack of self-sufficiency.
Thus, the object must be posited as
differentiated in its concrete existence,
opposite its other.

b. Chemism

The differentiated object has an immanent determinacy
constituting its nature and in that determinacy
it has concrete existence.
But as the posited totality of the concept,
it is the contradiction of this its totality
and the determinacy of its concrete existence;
it is thus the [process of] striving to sublate this contradiction
and make its existence equal to the concept.

The chemical process thus has as its product
the neutral dimension of these strung-out extremes,
a neutral dimension which these extremes are in themselves;
by means of the differentiation of the objects (the particularization),
the concept, the concrete universal, joins itself
with the individuality, the product,
and so merely with itself equally contained
in this process are the other syllogisms;
the individuality, as activity, is likewise
the mediating factor just like the concrete universal,
the essence of the strung-out extremes,
which enters into existence in the product.

Chemism, as the reflexive relationship of objectivity,
still presupposes, together with
the differentiated nature of the objects,
the immediate self-sufficiency of those same objects.
The process is that of passing back and forth
from one form into the other,
forms determinate properties that the extremes
had opposite one another are sublated.
This is, indeed, in keeping with the concept;
but the animating principle of differentiating
does not exist concretely in it
since it has sunk back into immediacy.
For this reason, the neutral dimension
is a separable dimension.
Yet the judging principle that severs
the neutral dimension into differentiated extremes
and gives the undifferentiated objects in general
their difference and animation opposite an other
falls outside that first process,
and so does the process as
the separation that strings things out.

The externality of these two processes,
the reduction of what are differentiated to something neutral
and the differentiation of the undifferentiated or neutral,
which allows them to appear as self-sufficient opposite one another,
shows its finitude in passing over into products
in which they are sublated.
Conversely, the process presents the presupposed immediacy
of the differentiated objects as a vacuous immediacy.
By means of this negation of externality and immediacy,
into which the concept as object was immersed,
it is posited freely and for itself
opposite that externality and immediacy,
as purpose.

c. Teleology

Purpose is the concept that is for itself
and that has entered into a free concrete existence
via the negation of immediate objectivity.
It is determined as something subjective,
in that this negation initially is abstract
and thus objectivity also only stands
over against it [i.e. the purpose] at first.
In contrast to the totality of the concept, however,
this determinacy of the subjectivity is one-sided and, indeed,
for it [the purpose] itself, since all determinacy has posited
itself as sublated in it.
Thus, too, for it [the purpose] the presupposed object is
only an ideal, in itself vacuous reality.
As this contradiction of its identity
with itself opposite the negation
and the opposition posited in it,
it is itself the sublating,
the activity of so negating the opposition
that it posits it as identical with itself.
This is the process of realizing the purpose in which,
by rendering itself something other than its subjectivity
and objectifying itself,
it has sublated the difference of both,
has joined itself together only with itself
and has preserved itself.

The concept of purpose is, on the one hand, superfluous;
on the other hand, it is rightly labelled a concept of reason
and contrasted with the understanding's abstract-universal
that relates itself to the particular
(which it does not have in itself)
only by way of subsuming it.
Furthermore, the difference of the purpose
as the final cause
from the merely efficient cause
(what is ordinarily called the cause)
is of the utmost importance.
The cause pertains to the not yet uncovered, blind necessity;
for this reason it appears to pass over into its other
and lose its originality in it in the course of being posited.
Only in itself or for us is the cause in the effect first a cause,
and does it come back into itself.
The purpose, by contrast, is posited as in itself the determinacy,
or what there [in efficient causality] still appears as being-other
contains the effect [here], so that, in its efficacy,
it does not pass over [into something else]
but instead preserves itself.
That is to say, it brings about itself alone
and is, in the end, what it was in the beginning,
in the original state.
What is truly original is so only
by means of this self-preservation.
The purpose requires a speculative construal,
as the concept that itself,
in its own unity and in the ideality of its determinations,
contains the judgment or the negation,
the opposition of the subjective and the objective,
and is equally the sublating of them.

With regard to the purpose,
one should not immediately
or should not merely think of
the form in which it is in consciousness,
as a determination on hand in the representation.
Through the concept of inner purposiveness,
Kant re-awakened the idea in general
and that of life in particular.
Aristotle's determination of life
already contains the inner purposiveness
and thus stands infinitely far beyond
the concept of modern teleology
which has only the finite,
the external purposiveness in view.

Need and drive are the examples of purpose lying closest at hand.
They are the flit contradiction that takes place within the living
subject itself and they enter into the activity of negating this
negation that is still mere subjectivity.
The satisfaction produces the peace between the subject and object,
in that the objective dimension standing over there
in the still on hand contradiction (to the need)
is equally sublated with respect to this, its one-sidedness,
through the unification with the subjective dimension.
Those who speak so much of the solidity and invincibility of
the finite have an example of the opposite in every drive.
The drive is, so to speak, the certainty that
the subjective dimension is only one-sided
and has just as little truth as the objective dimension.
The drive is, furthermore, the implementation of this, its certainty.
It manages to sublate this opposition;
that the subjective dimension
would be and remain only something subjective,
just as the objective dimension
would equally be and remain only something objective;
and [to sublate] this finitude of them.

With regard to the activity of the purpose,
attention may also be drawn to the fact that,
in the syllogism that conjoins the purpose with itself
through the means of the realization,
the negation of the termini surfaces;
the just mentioned negation of immediate subjectivity
that surfaces in the purpose as such,
like that of the immediate objectivity
(of the means and the presupposed objects).
This is the same negation that is exercised
in the elevation of the spirit to God
in contrast to the finite things of the world
as much as in contrast to one's own subjectivity.
This is the moment which is overlooked
and left aside in the form of the syllogisms
at the level of the understanding,
the form that is given to this elevation
in the so called proofs of God's existence.

The teleological relation in its immediacy is
initially the external purposiveness,
and the concept is opposite the object
which is something presupposed.
The purpose is thus finite,
partly in terms of the content,
partly in terms of the fact that
it has an external condition in an extant object
as the material of its realization.
To this extent, its self-determination is merely formal.
The immediacy entails, more precisely, that the particularity
(as a determination of form, the subjectivity of the purpose)
appears as reflected in itself,
the content as distinct from the totality of the form,
the subjectivity in itself, the concept.
This diversity constitutes the finitude of
the purpose within itself.
The content is, by this means,
as limited, contingent, and given
as the object is something particular and extant.

The teleological relation is the syllogism
in which the subjective purpose joins itself
together with the objectivity external to it
through a middle term that is the unity of the two,
both as the purposive activity and as the objectivity
immediately posited under the purpose, the means.

1. The subjective purpose is the syllogism
in which the universal concept joins
together with individuality
by means of particularity;
such that this [individuality]
as the self~determination judges.
That is to say, this individuality
both particularizes that still indeterminate universal,
making it a determinate content,
and also posits the opposition of
subjectivity and objectivity.
It [this individuality] is, in itself,
at the same time the return into itself
since it determines the concept's subjectivity
(presupposed as something opposite the objectivity)
to be deficient in relation to the totality
that has joined together with itself
and since at the same time it thereby turns outward.

2. This activity turned outward is the individuality that,
in the subjective purpose, is identical to the particularity
in which, next to the content,
the external objectivity is also included.
As such, this activity relates at the outset
immediately to the object
and takes control of it as a means.
The concept is this immediate power
because it is the negativity identical with itself,
the being of the object is thoroughly
determined only as something ideal.
The entire middle term is now this
inner power of the concept as activity,
with which the object is immediately unified
as means and under which it stands.

In the finite purposiveness, the middle term is this status of
being broken into two moments external to one another,
the activity and the object.
The relation of the purpose as power to this object
and the latter's being conquered by it is immediate,
it is the first premise of the inference,
insofar as in the concept qua ideality that is for itself
the object is posited as in itself nothing.
This relation or first premise becomes itself the middle term
which is, at the same time, the syllogism in itself,
since by means of this relation the purpose joins together its activity,
in which it remains contained and dominant, with objectivity.

3. The purposive activity with its means is still directed outward,
since the purpose is also not identical with the object;
thus it must first be mediated with the object.
The means, as the object in this second premise,
is in immediate relation with the other extreme of the syllogism,
the objectivity as presupposed, the material.
This relation is the sphere of the mechanism and chemism
now serving the purpose that is their truth and free concept.
That the subjective purpose, as the power of these processes
in which the objective dimension rubs up against itself and sublates itself,
keeps itself outside them and is what preserves itself in them;
this is the cunning of reason.

Thus, the realized purpose is the posited unity of
the subjective and the objective dimensions.
This unity, however, is essentially determined in such a way that
the subjective and objective dimensions are neutralized
and sublated only with respect to their one-sidedness,
while the objective dimension is subjected
and made to conform to the purpose as the free concept
and, thereby, to the power over it.
The purpose preserves itself
against and in the objective dimension
because, in addition to being
the one-sided subjective dimension (the particular),
it is also the concrete universal,
the identity of both, that is in itself.
This universal, that as simple is reflected in itself,
is the content that remains the same
through all three termini [terms] of
the syllogism and their movement.

In the finite purposiveness, however,
the purpose carried out is also
something as internally broken
as was the middle term and the initial purpose.
What has come about is thus only a form
posited externally in the material found before it,
a form that, on account of the restricted content of
the purpose, is likewise a contingent determination.
The purpose attained is thus only an object
that is also in turn a means
or material for other purposes
and so on ad infinitum.

What happens, however, in the process of
realizing the purpose in itself is
that the one-sided subjectivity
and the semblance of objective self-sufficiency
on hand opposite it are sublated.
In seizing the means, the concept posits itself
as the object's essence as it is in itself,
in the mechanical and chemical process,
the self-sufficiency of the object
has already evaporated in itself
and in the course it takes under
the dominance of the purpose,
the semblance of that self-sufficiency,
the negative dimension opposite the concept,
sublates itself.
Yet this object is immediately
already posited as vacuous in itself,
as only ideal by virtue of the fact
that the executed purpose is determined
only as means and material.
With this, the opposition of content and form
has vanished as well.
Since the purpose, by sublating the formal determinations,
joins itself together with itself,
the form is posited as identical with itself,
thus as content,
so that the concept as the activity of
the form has only itself as content.
It is thus posited through this process generally
what the concept of the purpose was:
the unity, being in itself,
of the subjective and the objective dimensions
now posited as being for itself:
the idea.
