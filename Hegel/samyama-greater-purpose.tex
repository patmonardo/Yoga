A. THE SUBJECTIVE PURPOSE

In the centrality of the objective sphere,
which is an indifference to determinateness,
the subjective concept has first rediscovered
and posited the negative point of unity,
and in chemism it has first rediscovered
and posited the objectivity of the determinations
of the concept by which it is first posited
as concrete objective concept.
Its determinateness or its simple difference now
has the determinateness of externality within it,
and its simple unity is therefore
the unity that repels itself from itself
and in this repelling maintains itself.
Purpose, therefore, is the subjective concept as
an essential striving and impulse
to posit itself externally.
In this, it is exempt from transition.
It is neither a force expressing itself,
nor a substance or a cause manifesting
itself in its accidents or effects.
To the extent that force has not expressed
itself, it is only an abstract inner;
or again, it first has existence in
an externalization to which it has to be solicited.
The same applies to cause and to substance.
Since they have actuality only in
the accidents and in the effects,
their activity is a transition against which
they do not maintain themselves in freedom.
Purpose can of course also be
defined as a force or a cause,
but these expressions cover only
an incomplete side of its signification;
if they are to be said of purpose according to its truth,
this can be done only in a way that sublates their concept,
as a cause that solicits itself to expression,
or a cause that is a cause of itself
or whose effect is immediately the cause.
When purposiveness is attributed to
an intelligence, as was said above,
this is done with specific reference
to a certain content.
But, as such, purpose is to be taken as
the rational in its concrete existence.
It manifests rationality by being
the concrete concept that holds
the objective difference in its absolute unity.
Within, therefore, it is essentially syllogism.
It is the self-equal universal;
more precisely, inasmuch as it contains
self-repelling negativity,
it is universal though at first
still indeterminate activity.
But since this activity is negative self-reference,
it determines itself immediately and
gives itself the moment of particularity,
and this particularity,
as likewise the totality of the form reflected into itself,
is content as against the posited differences of the form.
The same negativity, through its self-reference,
is just as immediately the reflection of the form
into itself and singularity.
From the one side, this reflection is
the inner universality of the subject;
from the other side, however,
it is outwards reflection;
and to this extent purpose is
still something subjective,
its activity still directed to
an external objectivity.

For purpose is the concept that
has come to itself in objectivity;
the determinateness that it has given itself
there is that of objective indifference
and externality of determinateness;
its self-repelling negativity is
therefore one whose moments,
being only determinations of the concept itself,
also have the form of objective indifference to one another.
Already in the formal judgment are subject and predicate
determined as self-subsistent over against each other;
but their self-subsistence is still only abstract universality.
It has now attained the determination of objectivity,
but, as a moment of the concept,
this complete difference is
enclosed within the simple unity of the concept.
Now in so far as purpose is this total reflection of
objectivity into itself
and is such immediately,
in the first place,
the self-determination
or the particularity as
simple reflection into itself is
distinguished from the concrete form,
and is a determinate content.
Accordingly, purpose is finite,
even though according to form
it is equally infinite subjectivity.
Secondly, since its determinateness has
the form of objective indifference,
it has the shape of a presupposition,
and from this side its finitude consists
in its having before it an objective,
mechanical and chemical world to which
its activity is directed
as to something already there;
its self-determining activity is in its identity
thus immediately external to itself,
reflection into itself just as much as reflection outwards.
To this extent purpose still has a truly
extra-mundane concrete existence;
to the extent, namely, that this objectivity stands opposed to it,
just as the latter, as a mechanical and chemical whole
still not determined and not pervaded by purpose,
stands on its side opposed to it.

Consequently, the movement of purpose can now be expressed
as being directed at sublating its presupposition,
that is, the immediacy of the object,
and at positing it as determined by the concept.
This negative relating to the object is
equally a negative attitude towards itself,
a sublating of the subjectivity of purpose.
Positively, this is the realization of purpose,
namely the unification of the objective being with it,
so that this being, which as a moment of purpose is
immediately the determinateness identical with it,
shall be as external determinateness,
and conversely the objective, as presupposition,
shall be posited rather as determined by the concept.

Purpose is in it the impulse to its realization;
the determinateness of the moments
of the concept is externality;
the simplicity of these moments
within the unity of the concept is
however incommensurable with what this unity is,
and the concept therefore repels itself from itself.
This repulsion is in general the resolution of
the self-reference of the negative unity
by virtue of which the latter is exclusive singularity;
but by this excluding the unity resolves itself,
that is to say, it discloses itself,
for it is self-determination,
the positing of itself.
On the one hand, in determining itself,
subjectivity makes itself into particularity,
gives itself a content which,
enclosed within the unity of the concept,
is still an inner content;
but this positing, the simple reflection into itself,
is, as we have seen, at the same time
immediately a presupposing;
and at the same moment in which
the subject of purpose determines itself,
it is referred to an indifferent, external objectivity
which is to be made equal by it
with the determinateness of that inner content,
that is to say, posited as something determined by the concept,
first of all as means

B. THE MEANS

The first immediate positing in purpose is
equally the positing of something internally determined,
that is, determined as posited, and, at the same time,
the presupposing of an objective world,
one indifferent to the determination of purpose.
But the subjectivity of purpose is
the absolutely negative unity;
its second determining is, therefore,
the sublation of this presupposition as such;
this sublation is an immanent turning back
inasmuch as that moment of the first negation
which is the positing of the negative over against the subject,
the external object, is sublated by it.
But as against the presupposition
or the immediacy of the determining,
as against the objective world,
it is as yet only the first,
itself immediate and hence external negation.
This positing is therefore not yet
the realized purpose itself
but only the beginning of this realization.
The object so determined is now the means.

Through a means the purpose unites with objectivity
and in objectivity unites with itself.
This means is the middle term of the syllogism.
Purpose is in need of a means for its realization,
because it is finite, in need of a means,
that is to say, of a middle term
that has at the same time the shape of an
external existence indifferent towards
the purpose itself and its realization.
The absolute concept has mediation
within itself in such a manner that
the first positing of it is not a
presupposition in whose object
the fundamental determination
would be an indifferent externality;
on the contrary, the world as creation has
only the form of such an externality;
it is its negativity and the positedness
that rather constitute its fundamental determination.
Accordingly, the finitude of purpose consists in this,
that its determining is as such external to itself;
accordingly, its first determining, as we have seen,
falls apart into a supposing and a presupposing;
the negation of this determining is therefore
only according to one side already immanent reflection;
according to the other side, it is rather only first negation.
Or again, the immanent reflection is itself also self-external
and a reflection outwards.

The means is therefore the
formal middle term of a formal syllogism;
it is something external to the extreme
of the subjective purpose as also,
therefore, to the extreme of the objective purpose;
just as particularity in the formal syllogism is
an indifferent medius terminus that can be replaced by others.
Moreover, just as this particularity is
a middle term by virtue of being
determinateness with reference to one extreme
but universality with reference to the other extreme,
and therefore obtains its mediating determination
by being related to an other,
so too the means is a mediating middle term
only because it is,
first, an immediate object,
and, second, because it is a means
by virtue of a reference
connecting it with the extreme
of purpose external to it,
a reference which is for it a form
to which it is indifferent.

Concept and objectivity, therefore,
are in the means only externally linked;
hence the means is only a merely mechanical object.
The reference of the object to purpose is
a premise or the immediate reference
which, as we have seen,
is with respect to purpose an immanent reflection;
the means is an inhering predicate;
its objectivity is subsumed under
the determination of purpose which,
on account of its concreteness, is universality.
Through this purposive determination present in it,
the means is now also subsumptive
with respect to the other extreme,
the at the moment still indeterminate objectivity.
Conversely, as contrasted with the subjective purpose,
the means has as immediate objectivity a universality of
existence which the subjective singularity of purpose still misses.
Thus, since purpose is in the means as
only an external determinateness at first,
it is itself, as the negative unity, outside the means;
the means, for its part, is a mechanical object
that possesses purpose only as a determinateness,
not as the simple concretion of totality.
But as the unifying means,
the middle term must itself be
the totality of the purpose.
It has been shown that the
determination of purpose is in the middle term
at the same time immanent reflection;
as this reflection, it is a formal self-reference,
since the determinateness is posited as real indifference,
as the objectivity of the middle term.
But precisely for this reason this subjectivity,
which is in one respect pure subjectivity,
is at the same time also activity.
In the subjective purpose the
negative self-reference is
still identical with determinateness as such,
with the content and the externality.
However, in the initial objectification of purpose
which is a becoming-other of the simple concept,
those moments come apart, each outside the other,
or, conversely, the becoming-other
or the externality itself consists in this coming apart.

This whole middle term is thus
the totality of the syllogism
in which the abstract activity
and the external means
constitute the extremes,
while the determinateness of
the object through the purpose,
by virtue of which it is a means,
constitutes the middle term.
But further, universality is
the connection of purposiveness and the means.
This means is object,
in itself the totality of the concept;
it does not have with respect to purpose
any of the power of resistance
that it initially has against
another immediate object.
To the purpose, therefore,
which is the posited concept,
it is utterly penetrable,
and it is receptive to this communication
because it is in itself identical with it.
But it is now also posited that
it is penetrable by the concept,
for in centrality it is an object
striving towards negative unity;
in chemism, too, whether as neutral or non-indifferent,
it is no longer self-subsistent.
Its non-self-subsistence consists
precisely in its being the
totality of the concept only implicitly;
but the concept is being-for-itself.
Consequently, with respect to purpose
the object has the character of being
powerless and of serving it;
purpose is the subjectivity or soul of the object
that has in the latter its external side.

The object, immediately subjected to purpose in this way,
is not an extreme of the syllogism;
on the contrary, this connection between the two
constitutes a premise of it.
But the means has also one side from which
it still has self-subsistence as against the purpose.
The objectivity which in the means is bound
with the purpose is still external to it,
because it is only immediately so connected;
and therefore the presupposition still persists.
The activity of the purpose through the means is
for that reason still directed against this presupposition,
and the purpose is activity,
no longer mere impulse and striving,
because in the means the moment of objectivity is
posited in its determinateness as something external,
and the simple unity of the concept now has
this objectivity as such within it.

C. THE REALIZED PURPOSE

1. Purpose is in its connection to the means
already reflected into itself,
but its objective immanent turning back is not yet posited.
The activity of purpose through its means is
still directed against objectivity as
an initial presupposition;
this is precisely what that activity is,
to be indifferent to determinateness.
If it were again to consist in
determining the immediate objectivity,
the product would again be only a means,
and so forth into infinity;
only a purposeful means would result,
but not the objectivity of the purpose itself
In being active in its means, therefore,
purpose must not determine the immediate object as
something external to it,
and the object, accordingly, must merge with it
in the unity of the concept through itself;
or again, the otherwise external activity of purpose
through its means must determine itself as mediation
and thus sublate itself as external.

The connection of the activity of purpose
with the external object through the means is
first of all the second premise of the syllogism
an immediate connection of the middle term
with the other extreme.
It is immediate because the middle term
has within it an external object
and the other extreme is likewise an external object.
The means is effective and potent
against this latter object
because its own is linked with
the self-determining activity,
whereas for the other the immediate determinateness
that it possesses is an indifferent one.
Their process in this connection is
none other than the mechanical or chemical one;
the previous relations come up again
in this objective externality,
but under the dominance of purpose.
But these processes, as they themselves showed,
return into purpose on their own.
If, therefore, the connection of the means
to the external object which it has to work upon is
at first an immediate one,
that connection has earlier exhibited
itself already as a syllogism,
for purpose proved to be their
true middle term and unity.
Since the means is therefore the object
that stands on the side of purpose
and has the latter's activity within it,
the mechanism that occurs here is
at the same time the turning back of
objectivity into itself, into the concept
which, however, is already presupposed as purpose;
the negative attitude of the purposeful activity
towards the object is therefore not
an external attitude but, on the contrary,
the objectivity's own alteration
and internal transition into it.

That the purpose immediately
refers to an object
and makes it into a means,
as also that through this means
it determines another object,
may be regarded as violence inasmuch as
purpose appears of an entirely
different nature than the object,
and the two objects are in like matter
mutually independent totalities.
But that the purpose posits itself in a
mediate connection with the object,
and between itself and this object inserts
another object, may be regarded as the cunning of reason.
As remarked, the finitude of rationality has this side,
that purpose relates to the object
as a presupposition, that is, as external.
In an immediate connection with that object,
purpose would itself enter into
the sphere of mechanism and chemism
and would therefore be subject to accidentality
and to the loss of its determining vocation
to be the concept that exists in and for itself.
But in this way,
by sending an object as
a means ahead of it,
it lets it do the slavish work of
externality in its stead,
abandons it to the wear and tear
while preserving itself behind it
against mechanical violence.

Since it is finite, the purpose further
has a finite content;
accordingly, it is not rational absolutely,
or simply in and for itself.
But the means is the external
middle term of the syllogism
which is the realization of purpose;
in the means, therefore, the rationality
in the purpose manifests itself as such
by maintaining itself in this external other,
and precisely through this externality.
To this extent the means is higher than
the finite purposes of external purposiveness:
the plough is more honorable than are immediately
the enjoyments which it procures
and which are the purposes.
The tool lasts while the immediate enjoyments
pass away and are forgotten.
It is in their tools that human beings
possess power over external nature,
even though with respect to their purposes
they are subjected to it.

But the purpose does not just keep
outside the mechanical process;
on the contrary, it keeps itself in it
and is its determination.
For purpose
(as the concept that concretely exists
freely over against the object and its process,
and is self-determining activity)
since it is equally the truth of
mechanism existing in and for itself,
in the latter only rejoins itself.
The power of purpose over the object is
this identity existing for itself,
and its activity is the manifestation of this identity.
The purpose as content is the determinateness as
it exists in and for itself,
present in the object as indifferent and external;
but the activity of the purpose is
the truth of the process on the one side,
and, as negative unity, the sublation of
the reflective shine of externality.
The indifferent determinateness of the object is
one that can abstractly be replaced
by another just as externally;
but the truth of the simple
abstraction of the determinateness is
the totality of the negative,
the concrete concept that posits
the externality within itself.

The content of the purpose is its negativity as
simple determinateness reflected into itself,
distinguished from its totality as form.
On account of this simplicity,
the determinateness of which is in and for itself
the totality of the concept,
the content appears as that which remains identical
in the realization of the purpose.
The teleological process is
the translation of the concept
that concretely exists distinctly as
concept into objectivity;
as we see, this translation into
a presupposed other is
the rejoining of the concept
through itself with itself.
The content of the purpose is now
this identity concretely existing
in the form of the identical.
In every transition the concept maintains itself;
for instance, when the cause comes to effect,
it is the cause that in the effect only comes to itself.
But in the teleological transition,
what maintains itself is the concept that as
such already concretely exists as cause,
as the free concrete unity as against
objectivity and its external determinateness.
The externality into which
the purpose translates itself is,
as we have seen, itself already posited as
a moment of the concept,
as the form of its inner differentiation.
In the externality, therefore,
the purpose has its own moment;
and the content, as the content of the concrete unity,
is its simple form that does not
remain in its different moments
only implicitly equal with itself
(as subjective purpose, as means and mediating activity,
and as objective purpose)
but also exists concretely as
abidingly self-equal.

Of the teleological activity
one can say, therefore,
that in it
the end is the beginning,
the consequence the ground,
the effect the cause;
that it is a becoming of what has become;
that in it only that which already
concretely exists comes into existence, and so on;
that is to say, that quite in general
all the relation determinations that belong to
the sphere of reflection or of immediate being
have lost their distinction,
and what, like end, consequence, effect,
and so on, is spoken of as an other,
no longer has in purpose this determination of other,
but is rather posited as identical with the simple concept.

2. If we now examine the product of
teleological activity more closely,
we see that purpose comes to it only externally
if we take it as an absolute presupposition
over against a purpose which is subjective,
that is to say, in so far as
we stop short at a purposive activity
that relates to the object
through its means only mechanically,
positing in place of one indifferent
determinateness of the object an other
which is just as external to it.
A determinateness such as an object
possesses through purpose differs in
general from one which is merely mechanical
in that it is a moment of
a unity and consequently,
although external to the object,
is yet not in itself
something merely external.
The object that exhibits such a unity is
a whole with respect to which its parts,
its own externality, are indifferent;
it is a determinate, concrete unity that
unites different connections and
determinacies within itself.
This unity, which cannot be comprehended
from the specific nature of the object
and, as regards determinate content,
is of another content than the object's own,
is for itself not a mechanical determinateness,
yet still is in the object mechanically.
Just as in this product of purposive activity
the content of the purpose
and the content of the object are
external to each other,
so too do the determinations in
the other moments of  the syllogism
relate to each other externally,
in the connecting middle,
the purposive activity
and the object which is the means;
and in the subjective purpose,
which is the other extreme,
the infinite form as totality of the concept
and the content of the concept.
According to the connection by which
the subjective purpose is
syllogistically united with objectivity,
both premises are an immediate connection,
namely the connection of the object
determined as middle term with
the still external object,
and the connection of the subjective purpose
 with the object made into means.
The syllogism is therefore affected by
the deficiency of the formal syllogism in general,
namely that the connections
in which it consists are not
themselves conclusions or mediations
but already presuppose the conclusion
for the production of which they are
supposed to serve as means.

If we consider the one premise,
that of the immediate connection of
the subjective purpose and the object
that thereby becomes a means,
then the purpose cannot
connect with the object immediately,
for the latter is just as immediate as
the object of the other extreme
in which the purpose is to be
realized through mediation.
Since the two are thus posited as diverse,
a means for their connection must be
interjected between this objectivity
and the subjective purpose;
but such a means is equally an
object already determined by purpose,
and between this objectivity
and the teleological determination
a new means is to be interjected,
and so on to infinity.
The infinite progress of mediation is
thereby set in motion.
The same happens as regards the other premise,
the connection of the means
with the yet indeterminate object.
Since the two terms are utterly self-subsistent,
they can be united only in a third,
and so on to infinity.
Or conversely, since the premises
already presuppose the conclusion,
the latter can only be imperfect,
for it is based on those only immediate premises.
The conclusion or the product of
the purposive activity is
nothing but an object determined
by a purpose that is external to it;
thus it is the same as what the means is.
In such a product itself, therefore,
only a means has been derived,
not a realized purpose;
or again:
purpose has not truly attained any objectivity in it.
It is therefore entirely a matter of indifference
whether we consider an object
determined by external purpose
as realized purpose or only as means;
what we have is not an objective determination
but a relative one,
external to the object itself.
All objects in which an external purpose is
realized equally are, therefore, only a means of purpose.
Anything which is intended for the realization of a purpose
and is taken essentially as a means,
is such a means by virtue of
its vocation that it be used up.
But also the object that is supposed to
contain the realized purpose
and show itself to be its objectivity is perishable;
it likewise fulfills its purpose not by
a tranquil, self-preserving existence,
but only to the extent that it is worn out,
for only to this extent does
it conform to the unity of the concept,
namely in so far as its externality,
that is, its objectivity,
sublates itself in that unity.
A house, a clock, may appear as purposes
with respect to the instruments
employed in their production;
but the stones, the cross-beams,
or the wheels, the axles,
and the rest that make up
the actuality of the purpose,
fulfill this purpose only
through the pressure which they suffer,
through the chemical processes
to which they are exposed
with air, light, and water,
and from which they shield the human being;
through their friction, and so on.
They fulfill their vocation, therefore,
only through their being used up and worn out,
and only by virtue of their negation do
they correspond to what they are supposed to be.
They are not united with purpose positively,
because they possess self-determination only externally
and are only relative purposes,
or essentially only means.
These purposes thus in general have a restricted content;
their form is the infinite self-determination of the concept,
which through that content has restricted
itself to external singularity.
The restricted content renders
these purposes inadequate to
the infinity of the concept,
relegating them to untruth;
such a determinateness is
through the sphere of necessity,
through being, already at
the mercy of becoming
and alteration and passes away.

3. The result now is that external purposiveness,
which only has so far the form of teleology,
only goes so far as to be a means,
not to be an objective purpose,
because subjective purpose remains
an external, subjective determination.
Or in so far as purpose is
active and attains completion,
albeit only in a means,
it is still bound up
with objectivity immediately;
it is sunk into it.
Purpose is itself an object
and, as one may say,
it does not attain a means
because its realization is
needed before such a realization can be
brought about through a means.

But the result is in fact not only
an external purposive connection,
but the truth of such a connection,
inner purposive connection
and an objective purpose.
The self-subsistence of the object
over against the concept that
purpose presupposes is
posited in this presupposition
as an unessential reflective shine
and as already sublated in and for itself;
the activity of the purpose truly is,
therefore, only the exposure of
this reflective shine
and the sublation of it.
As the concept has demonstrated,
the first object becomes
by virtue of communication a means,
for it implicitly is the totality of the concept,
and its determinateness,
which is none other than the externality itself,
is posited as something only external and unessential,
is posited in purpose itself, therefore,
as the latter's own moment,
not as anything that stands
on its own over against it.
As a result, the determination of
the object as a means is
altogether immediate.
There is no need, therefore,
for the subjective purpose to exercise
any violence to make the object into a means,
no need of extra reinforcement;
the resolution, the resolve,
this determination of itself,
is the only posited externality of the object,
which is therein immediately subjected to purpose,
and has no other determination as
against it than that of the nothingness of
the being-in-and-for-itself.

The second sublating of objectivity through objectivity
differs from this first sublation in that the latter,
being the first, is the purpose in objective immediacy;
the second, therefore, is not only the sublating of
a first immediacy but of both,
of the objective as something merely posited
and of the immediate.
The negativity thus returns to itself in such a way
that it is equally the restoration of objectivity,
but of an objectivity which is identical with it,
and in this it is at the same time also the positing of it
as an external objectivity which is only determined by purpose.
Because of this positing,
the product remains as before also a means;
because of the identity with negativity,
the product is an objectivity
which is identical with the concept,
is the realized purpose
in which the side of being a means
is the reality itself of purpose.
In the completed purpose the means disappears
because it would be simply and solely
the objectivity immediately subsumed under that purpose,
an objectivity which in the realized purpose is
the turning back of the purpose into itself;
further, there also disappears with it mediation itself,
as the relating of an external;
it disappears into both
the concrete identity of objective purpose,
and into the same identity as
abstract identity and immediacy of existence.

Herein is also contained the mediation
that was required for the first premise,
the immediate connection of the purpose with the object.
The realized purpose is also a means;
conversely, the truth of the means is just this,
to be the real purpose itself,
and the first sublation of objectivity
is already also the second,
just as the second proved to contain the first also.
For the concept determines itself,
its determinateness is the external indifference
which is immediately determined in the resolution as sublated,
that is to say, as inner, subjective indifference
and at the same time as presupposed object.
Its further procession out of itself that appeared,
namely as the immediate communication and subsumption of
the presupposed object under it,
is at one and the same time
the sublating of that determinateness of externality
which was internal, shut up in the concept,
that is, posited as sublated,
and the sublating of the presupposition of an object;
consequently, this apparently first sublating of
the indifferent objectivity is already the second as well,
an immanent reflection that has gone through mediation,
and the realized purpose.

Since the concept is here,
in the sphere of objectivity
where its determinateness has
the form of indifferent externality,
in reciprocal action with itself,
the exposition of its movement
becomes doubly difficult and intricate,
for such a movement is itself immediately doubled
and a first is always also a second.
In the concept taken for itself,
that is, in its subjectivity,
the difference of itself from itself is
as an immediate identical totality on its own;
but since its determinateness here is indifferent externality,
its self-identity is in this externality
immediately also self-repulsion again,
so that what is determined as external
and indifferent to the identity is
rather this identity itself,
and the identity as identity,
as self-reflected,
is rather its other.
Only by firmly attending to this
shall we comprehend the objective
turning back of the concept into itself,
that is, its true objectification;
only then shall we see that
every one of the single moments
through which this mediation runs its course is
itself the whole syllogism of the mediation.
Thus the original inner externality of the concept,
by virtue of which the concept is
self-repelling unity, purpose
and the striving of purpose towards objectivity,
is the immediate positing
or the presupposition of an external object;
the self-determination is also
the determination of an external
object not determined by the concept;
and conversely this determination
is self-determination, that is,
the sublated externality posited as inner,
or the certainty of the unessentiality
of the external object.
Of the second connection,
that of the determination of the object as a means,
we have just shown how it is within itself
the self-mediation of purpose in the object.
Likewise the third mode of connection, mechanism,
which proceeds under the dominance of purpose
and sublates the object by virtue of the object,
is on the one hand the sublating of the means,
of the object already posited as sublated,
and consequently a second sublation
and immanent reflection,
and on the other hand a first
determining of the external object.
This last, as we remarked,
is in the realized purpose
again the production of only a means;
the subjectivity of the finite concept,
by contemptuously rejecting the means,
has attained nothing better in its goal.
But this reflection, namely
that purpose is attained in the means
and that the means and the mediation are
preserved in the fulfilled purpose,
is the final result of
the external connection of purpose,
a result in which this connection has sublated itself
and which it has exhibited as its truth.
The last considered third syllogism differs from the rest
in that it is in the first instance
the subjective purposive activity of the preceding syllogism,
but also the sublation of external objectivity
and consequently of externality in general;
it is this through itself, and is, therefore,
the totality in its positedness.

We have now seen subjectivity,
the being-for-itself of the concept,
pass over into the concept's
being-in-itself, into objectivity,
and then the negativity of
that being-for-itself
reassert itself in objectivity;
the concept has so determined itself in that negativity
that its particularity is an external objectivity,
or has determined itself as the simple concrete unity
whose externality is its self-determination.
The movement of purpose has now attained this much,
namely that the moment of externality is
not just posited in the concept,
the purpose is not just an ought and a striving,
but as a concrete totality is
identical with immediate objectivity.
This identity is on the one hand the simple concept,
and the equally immediate objectivity,
but, on the other hand,
it is just as essentially mediation,
and it is that simple immediacy only
through this mediation sublating itself as mediation.
Thus the concept is essentially this:
to be distinguished, as an identity existing for itself,
from its implicitly existent objectivity,
and thereby to obtain externality,
but in this external totality to be
the totality's self-determining identity.
