SECTION II

Appearance

Essence must appear.

Being is the absolute abstraction;
this negativity is not something external to it,
but being is rather being,
and nothing but being,
only as this absolute negativity.
Because of this negativity,
being is only as self-sublating being
and is essence.
But, conversely,
essence as simple self-equality
is likewise being.
The doctrine of being contains
the first proposition, “being is essence.”
The second proposition, “essence is being,”
constitutes the content of the first section
of the doctrine of essence.
But this being into which
essence makes itself
is essential being,
concrete existence,
a being which has come forth
out of negativity and inwardness.

Thus essence appears.
Reflection is the internal shining of essence.
The determinations of this reflection are included
in the unity purely and simply as posited, sublated;
or reflection is essence immediately
identical with itself in its positedness.
But since this essence is ground,
through its self-sublating reflection,
or the reflection that which returns into itself,
essence determines itself as something real;
further, since this real determination,
or the otherness,
of the ground-connection sublates itself
in the reflection of the ground
and becomes concrete existence,
the form determinations acquire therein
an element of independent subsistence.
Their reflective shine comes to completion in appearance.

The essentiality that has advanced to immediacy is,

first, concrete existence,
and a concrete existent or thing,
an undifferentiated unity of
essence and its immediacy.
The thing indeed contains reflection,
but its negativity is at first
dissolved in its immediacy;
but, because its ground is
essentially reflection,
its immediacy is sublated
and the thing makes itself
into a positedness.

Second, then, it is appearance.
Appearance is what the thing is in itself,
or the truth of it.
But this concrete existence,
only posited and reflected into otherness,
is equally the surpassing of itself into its infinity;
opposed to the world of appearance
there stands the world that exists
in itself reflected into itself.
But the being that appears and essential being
stand referred to each other absolutely.

Thus concrete existence is, third, essential relation;
what appears shows the essential,
and the essential is in its appearance.
Relation is the still incomplete union of
reflection into otherness and reflection into itself;
the complete interpenetrating of the two is actuality.

CHAPTER 1

Concrete existence

Just as the principle of sufficient reason says
that whatever is has a ground,
or is something posited,
something mediated,
so there would also have to be
a principle of concrete existence saying
that whatever is, exists concretely.
The truth of being is to be,
not an immediate something,
but essence that has
come forth into immediacy.

But when it was further said
that whatever exists concretely
has a ground and is conditioned,
it also would have had to be said
that it has no ground and is unconditioned.
For concrete existence is the immediacy
that has come forth from the sublating
of the mediation that results
from the connection of ground and condition,
and which, in coming forth,
sublates this very coming forth.

Inasmuch as mention may be made here of
the proofs of the concrete existence of God,
it is first to be noted that besides
immediate being that comes first,
and concrete existence
(or the being that proceeds from essence)
that comes second, there is still a third being,
one that proceeds from the concept,
and this is objectivity.
Proof is, in general, mediated cognition.
The various kinds of being require or contain
each its own kind of mediation,
and so will the nature of the proof also vary accordingly.
The ontological proof wants to start from the concept;
it lays down as its basis the sum total of all realities,
where under reality also concrete existence is subsumed.
Its mediation, therefore, is that of the syllogism,
and syllogism is not yet under consideration here.
We have already commented above (Part 1, Section 1)
on Kant's objection to the ontological proof,
and have remarked that by concrete existence
Kant understands the determinate immediate existence
with which something enters into the context of total experience,
that is, into the determination of being an other
and of being in reference to an other.
As an existent concrete in this way,
something is thus mediated by an other,
and concrete existence is in general the side of its mediation.
But in what Kant calls the concept, namely,
something taken as only simply self-referring,
or in representation as such, this mediation is missing;
in abstract self-identity, opposition is left out.
Now the ontological proof would have
to demonstrate that the absolute concept,
namely the concept of God,
attains to a determinate existence, to mediation,
or to demonstrate how simple essence
mediates itself with mediation.
This is done by the just mentioned
subsumption of concrete existence
under its universal, namely reality,
which is assumed as the middle term
between God in his concept, on the one hand,
and concrete existence, on the other.
This mediation, inasmuch as it has the form of a syllogism,
is not at issue here, as already said.
However, how that mediation of
essence and concrete existence truly comes about,
this is contained in the preceding exposition.
The nature of the proof itself will be considered
in the doctrine of cognition.
Here we have only to indicate what pertains
to the nature of mediation in general.

The proofs of the existence of God
adduce a ground for this existence.
It is not supposed to be
an objective ground of the existence of God,
for this existence is in and for itself.
It is, therefore, solely a ground for cognition.
It thereby presents itself as a ground
that vanishes in the subject matter
that at first seems to be grounded by it.
Now the ground which is derived
from the contingency of the world
entails the regress of the latter
into the absolute essence,
for the accidental is that which is
in itself groundless and self-sublating.
In this way, therefore,
the absolute essence does indeed proceed
from that which has no ground,
for the ground sublates itself
and with this there also vanishes
the reflective shine of the relation
that was given to God,
that it is grounded in an other.
This mediation is therefore true mediation.
But the reflection involved
in that proof does not know
the nature of the mediation that it performs.
On the one hand, it takes itself
to be something merely subjective,
and it consequently distances
its mediation from God himself;
on the other hand, for that same reason
it also fails to recognize its mediating movement,
that this movement is in the essence itself
and how it is there.
The true relation of reflection consists
in being both in one:
mediation as such but, of course, at the same time
a subjective, external mediation,
that is to say, a self-external mediation
which in turn internally sublates itself.
In that other presentation, however,
concrete existence is given the false relation
of appearing only as mediated or posited.

So, on the other side, concrete existence also
cannot be regarded merely as an immediate.
Taken in the determination of an immediacy,
the comprehension of God's concrete existence
has been declared to be beyond proof
and the knowledge of it
an immediate consciousness only, a faith.
Knowledge should arrive at
the conclusion that it knows nothing,
and this means that it gives up its mediating movement
and the determinations themselves
that have come up in the course of it.
This is what has also occurred in the foregoing;
but it must be added that reflection,
by ending up with the sublation of itself,
does not thereby have nothing for result,
so that the positive knowledge of the essence
would then be an immediate reference to it,
divorced from that result and self-originating,
an act that starts only from itself;
on the contrary, the end itself,
the foundering of the mediation,
is at the same time the ground
from which the immediate proceeds.
In “zu Grunde gehen,” the German language unites,
as we remarked above,
the meaning of foundering and of ground;
the essence of God is said to be the
abyss (Abgrund in German) for finite reason.
This it is, indeed, in so far as
reason surrenders its finitude therein,
and sinks its mediating movement;
but this abyss, the negative ground,
is at the same time the positive ground
of the emergence of the existent,
of the essence immediate in itself;
mediation is an essential moment.
Mediation through ground sublates itself
but does not leave the ground standing under it,
so that what proceeds from it would be a posited
that has its essence elsewhere;
on the contrary, this ground is,
as an abyss, the vanished mediation,
and, conversely,
only the vanished mediation is
at the same time the ground
and, only through this negation,
the self-equal and immediate.

Concrete existence, then, is not to be taken here
as a predicate, or as a determination of essence,
of which it could be said in a proposition,
“essence exists concretely,” or “it has concrete existence.”
On the contrary, essence has passed over into concrete existence;
concrete existence is the absolute self-emptying of essence,
an emptying that leaves nothing of the essence behind.
The proposition should therefore run:
“Essence is concrete existence;
it is not distinct from its concrete existence.”
Essence has passed over into concrete existence
inasmuch as essence as ground
no longer distinguishes itself from itself as grounded,
or inasmuch as the ground has sublated itself.
But this negation is no less essentially its position,
or the simply positive continuity with itself;
concrete existence is
the reflection of the ground into itself,
its self-identity as attained in its negation,
therefore the mediation that has posited itself
as identical with itself and through that is immediacy.

Now because concrete existence is
essentially self-identical mediation,
it has the determinations of mediation in it,
but in such a way that the determinations are
at the same time reflected into themselves
and have essential and immediate subsistence.
As an immediacy which is posited through sublation,
concrete existence is negative unity and being-within-itself;
it therefore immediately determines itself
as a concrete existent and as thing.

A. THE THING AND ITS PROPERTIES

Concrete existence as a concrete existent is posited in
the form of the negative unity which it essentially is.
But this negative unity is at first only immediate determination,
hence the oneness of the something in general.
But the concretely existent something is different
from the something that exists immediately.
The former is essentially an immediacy that has arisen
through the reflection of mediation into itself.
The concretely existent something is thus a thing.

The thing is distinct from its concrete existence
just as the something can be distinguished from its being.
The thing and the concrete existent
are immediately one and the same.
But because concrete existence is not
the first immediacy of being
but has the moment of mediation within it,
its further determination as thing
and the distinguishing of the two is
not a transition but truly an analysis.
Concrete existence as such contains
this very distinction in the moment of its mediation:
the distinction of thing-in-itself
and external concrete existence.

a. The thing in itself and concrete existence

1. The thing in itself is the concrete existent
as the essential immediate that has resulted
from the sublated mediation.
Mediation is therefore equally essential to it;
but this distinction in this first
or immediate concrete existence
falls apart into indifferent determinations.
The one side, namely the mediation of the thing,
is its non-reflected immediacy,
and hence its being in general;
and this being, since it is
at the same time determined as mediation,
is an existence which is other to itself,
manifold and external within itself.
But it is not just immediate existence;
it also refers to the sublated mediation
and the essential immediacy;
it is therefore immediate existence
as unessential, as positedness.
(When the thing is differentiated
from its concrete existence,
it is then the possible,
the thing of representation,
or the thing of thought,
which as such is at the same time
not supposed to exist.
However, the determination of possibility
and of the opposition of the thing
and its concrete existence comes later.)
But the thing-in-itself and its mediated being are
both contained in the concrete existence,
and both are themselves concrete existences;
the thing-in-itself exists concretely
and is the essential concrete existence,
but the mediated being is
the thing's unessential concrete existence.

The thing in itself, as the simple reflectedness of
the concrete existence within itself,
is not the ground of unessential existence;
it is the unmoved, indeterminate unity,
for it has precisely the determination of
being the sublated mediation,
and is therefore the substrate of that existence.
For this reason reflection, too,
as an immediate existence
which is mediated through some other,
falls outside the thing-in-itself.
The latter is not supposed to have
any determinate manifold in it;
for this reason it obtains it only
when exposed to external reflection,
though it remains indifferent to it.
(The thing-in-itself has color
only when exposed to the eye,
smell when exposed to the nose, and so on.)
Its diversity consists of aspects
which an other picks out,
specific points of reference
which this other assumes
with respect to the thing-in-itself
and which are not the thing's own determinations.

2. Now this other is reflection
which, determined as external, is,
first, external to itself and determinate manifoldness.
Second, it is external to the essential concrete existent
and refers to it as to its absolute presupposition.
These two moments of external reflection,
its own manifoldness and its reference to
the thing-in-itself as its other,
are however one and the same.
For this concrete existence is
external only in so far as it refers to
the essential identity as to an other.
The manifoldness, therefore, does not have
an independent  subsistence of its own
besides the thing-in-itself
but, over against it,
it is rather only as reflective shine;
in its necessary reference to it,
it is like a reflex refracting itself in it.
Diversity, therefore, is present as the reference
of an other to the thing-in-itself;
but this other is nothing that subsists on its own
but is only as reference to the thing-in-itself;
but at the same time it only is in being repelled from it;
thus it is the unsupported rebound of itself within itself.

Now since the thing-in-itself is
the essential identity of the concrete existence,
this essenceless reflection does not accrue to it
but collapses within itself externally to it.
It founders to the ground
and thus itself comes to be essential identity
or thing-in-itself.
This can also be looked at in this way:
the essenceless concrete existence has
in the thing-in-itself its reflection into itself;
it refers to it in the first place as to its other;
but as the other over against that which is in itself,
it is only the sublation of its self,
and its coming to be in the in-itself.
The thing-in-itself is thus identical
with external concrete existence.

This is exhibited in the thing-in-itself as follows.
The thing-in-itself is self-referring
essential concrete existence;
it is self-identity only in so far as
it holds negativity's reflection in itself;
that which appeared as concrete existence
external to it is, consequently, a moment in it.
It is for this reason also self-repelling thing-in-itself
which thus relates itself to itself as to an other.
Hence, there are now a plurality of things-in-themselves
standing in the reciprocal reference of external reflection.
This unessential concrete existence is
their reciprocal relation as others;
but it is, further, also essential to them
or, in other words, this unessential concrete existence,
in collapsing internally, is thing-in-itself,
but a thing-in-itself which is other than the first,
for that first is immediate essentiality
whereas the present proceeds from
the unessential concrete existence.
But this other thing-in-itself is only an other in general;
for, as self-identical thing, it has no
further determinateness vis-à-vis the first;
like the first, it is the reflection within itself
of the unessential concrete existence.
The determinateness of the various things-in-themselves
over against one another falls therefore into external reflection.

3. This external reflection is henceforth a relating of
the things-in-themselves to one another,
their reciprocal mediation as others.
The things-in-themselves are thus
the extreme terms of a syllogism,
the middle term of which is made up
by their external concrete existence,
the concrete existence by virtue of which
they are other to each other and distinct.
This, their difference, falls only
in their connecting reference;
they send determinations, as it were,
from their surface into the reference,
while remaining themselves indifferent to it.
This relation now constitutes
the totality of the concrete existence.
The thing-in-itself is drawn into
a reflection external to it
in which it has a manifold of determinations;
this is the repelling of itself from itself
into another thing-in-itself,
a repelling which is its rebounding back into itself,
for each thing-in-itself is an other
only as reflected back from the other;
it has its supposition not in itself but in the other,
is determined only through the determinateness of the other;
this other is equally determined only
through the determinateness of the first.
But the two things-in-themselves,
since each has its difference
not in it but in the other,
are not therefore distinct things;
the thing-in-itself, in relating as it should to
the other extreme as to another thing-in-itself,
relates to it as to something non-distinguished from it,
and the external reflection that should constitute
the mediating reference between the extremes is a
relation of the thing-in-itself only to itself,
or is essentially its reflection within itself;
the reflection is, therefore,
determinateness existing in itself,
or the determinateness of the thing-in-itself.
The latter, therefore, does not have this determinateness
in a reference, external to it,
to another thing-in-itself,
and of this other to it;
the determinateness is not just its surface
but is rather the essential mediation of
itself with itself as with an other.
The two things-in-themselves that should
constitute the extremes of the reference,
since they are supposed not
to have any contrasting determinateness,
collapse in fact into one;
it is only one thing-in-itself that
relates itself to itself in the external reflection,
and it is its own reference to itself as to another
that constitutes its determinateness.

This determinateness of the thing-in-itself is
the property of the thing.

b. Property

Quality is the immediate determinateness of something;
the negative itself by virtue of which being is something.
The property of the thing is, for its part,
the negativity of reflection,
by virtue of which concrete existence
in general is a concrete existent
and, as simple self-identity, is thing-in-itself.
But the negativity of reflection, the sublated mediation,
is itself essentially mediation and reference,
though not to an other in general like quality
which is not reflected determinateness;
it is rather reference to itself as to an other,
or mediation which immediately is no less self-identity.
The abstract thing-in-itself is itself this relation
which turns from another back to itself;
it is thereby determined in itself;
but its determinateness is constitution,
which is as such itself determination,
and in relating to the other
it does not pass over into otherness
and is excluded from alteration.

A thing has properties;

these are, first, its determinate references to something other;
the property is there only as a way of reciprocal relating;
it is, therefore, the external reflection of the thing
and the side of its positedness.

But, second, in this positedness the thing is in itself;
it maintains itself in its reference to the other
and thus is admittedly only a surface
where the concrete existence is exposed to
the becoming of being and to alteration;
the property is not lost in this.
A thing has the property to effect this or that in an other,
and in this connection to express itself in some characteristic way.
It demonstrates this property only under the condition
that another thing has a corresponding constitution,
but at the same time the property is characteristically
the thing's own and its self-identical substrate;
for this reason this reflected quality is called property.
The thing thereby passes over into an externality,
but the property maintains itself in this transition.
Through its properties the thing becomes cause,
and to be a cause is this, to preserve itself as effect.
However, the thing is here still the static thing of many properties;
it is not yet determined as actual cause;
it is so far only the reflection of
its determinations immediately existing in itself,
not yet itself the reflection that posits them.

Essentially, therefore, the thing-in-itself has
just shown itself to be thing-in-itself
not only in such a way that its properties are
the positedness of an external reflection;
on the contrary, those properties are its own determinations
by virtue of which it relates in some determinate manner;
it is not an indeterminate substrate located on
the other side of its external concrete existence
but is present in its properties rather as ground,
that is to say, it is self-identity in its positedness;
but, at the same time, it is conditioned ground,
that is to say, its positedness is
equally reflection external to itself;
it is reflected into itself and in itself
only to the extent that it is external.
Through concrete existence the thing-in-itself
enters into external references,
and the concrete existence consists
precisely in this externality;
it is the immediacy of being
and because of that the thing is
subjected to alteration;
but it is also the reflected immediacy of the ground,
hence the thing in itself in its alteration.
This mention of the ground-connection is
not however to be taken here as if
the thing in general were determined
as the ground of its properties;
thinghood itself is, as such, the ground-connection;
the property is not distinguished from its ground,
nor does it constitute just the positedness
but is rather the ground that has
passed over into its externality
and is consequently truly reflected into itself;
the property is itself, as such,
the ground, implicitly existent positedness;
it is the ground, in other words,
that constitutes the form of the property's identity,
and the property's determinateness is
the self-external reflection of the ground;
the whole is the ground which in its repelling and determining,
in its external immediacy, refers itself to itself.
The thing-in-itself thus concretely exists essentially,
and that it concretely exists essentially means,
conversely, that concrete existence, as external immediacy,
is at the same time in-itselfness.

c. The reciprocal action of things

The thing-in-itself exists in concreto by essence;
external immediacy and determinateness
belong to its being-in-itself,
or to its immanent reflection.
The thing in-itself is thus a thing that has properties,
and hence there are a number of things distinct from one another,
not because of some viewpoint alien to them but through themselves.
These many diverse things stand in essential reciprocal action
by virtue of their properties;
the property is this reciprocal connecting reference itself,
apart from which the thing is nothing;
the reciprocal determination,
the middle term of the things-in-themselves
that are taken as extreme terms
indifferent to the reference connecting them,
is itself the self-identical reflection
and the thing-in-itself
which those extremes were supposed to be.
Thinghood is thus reduced to the
form of indeterminate self-identity
having its essentiality only in its property.
Thus, if one speaks of a thing
or of things in general without a determinate property,
then their difference is merely indifferent, quantitative.
What is considered as a thing can just as well be made into
a plurality of things or be considered as a plurality of things;
their separation or their union is an external one.
A book is a thing, and each of its pages is also a thing,
and equally so every tiny piece of its pages,
and so on to infinity.
The determinateness, in virtue of which
a thing is this thing only,
lies solely in its properties.
It is through them that the thing
differentiates itself from other things,
for the property is the negative reflection
and the differentiating;
only in its property, therefore, does the thing possess
in it the difference of itself from others.
This is the difference reflected into itself,
by virtue of which the thing, in its positedness,
that is, in its reference to others,
is equally indifferent to the other
and to its reference to it.
Without its properties, therefore,
there is nothing that remains to the thing
except the unessential compass
and the external gathering of an abstract in-itselfness.
With this, thinghood has passed over into property.

The thing, as the extreme term that exists in itself,
was supposed to relate to the property,
and this property to constitute the middle term
between things that stand connected.
But this connection is where the things meet
as self-repelling reflection,
where they are distinguished and connected.
This, their distinction and their connecting reference,
is one reflection and one continuity of both.
Accordingly, the things themselves fall only
within this continuity which is the property;
they vanish as would-be self-subsisting extremes
that would have a concrete existence outside this property.

The property, which was supposed to connect
the self-subsisting extremes,
is therefore itself self-subsistent.
The things are, on the contrary, the unessential.
They are something essential only as
the self-differentiating
and self-referring reflection;
but this is the property.
The latter is in the thing,
therefore, not as something sublated,
not just a moment of it;
on the contrary, the truth of the thing is
that it is only an unessential compass
which is indeed a negative unity,
but only like the one of the something,
that is to say, a one which is immediate.
Whereas earlier the thing was determined as
an unessential compass because it was made such
by an external abstraction that omits the property,
this abstraction now happens through the transition of
the thing-in-itself into the property itself.
But there is now an inversion of values,
for the earlier abstraction still envisaged
the abstract thing without its property
as being the essential,
and the property as an external determination,
whereas it is the thing as such which is now reduced,
through itself, to the determination of
an indifferent external form of the property.
The latter is henceforth thus freed of
the indeterminate and impotent bond
which is the unity of the thing;
the property is what constitutes
the subsistence of the thing;
it is a self-subsisting matter.
Since this matter is simple continuity with itself,
it only possesses at first the form of diversity.
There is, therefore, a manifold of
these self-subsisting matters,
and the thing consists of them.

B. THE CONSTITUTION OF THE THING OUT OF MATTERS

The transition of property into a matter
or into a self-subsistent stuff is the familiar
transition performed on sensible matter by chemistry
when it seeks to represent the properties of color, smell, etc.,
as luminous matter, coloring matter, odorific matter,
sour, bitter matter and so on;
or when it simply assumes others,
like calorific matter, electrical, magnetic matter,
in the conviction that it has thereby gotten hold
of properties as they truly are.
Equally current is the saying that
things consist of various matters or stuffs.
One is careful about calling these matters or stuffs “things,”
even though one will readily admit that,
for example, a pigment is a thing;
but I do not know whether luminous matter,
for instance, or calorific matter,
or electrical matter, etc., are called things.
The distinction is made between things and their components
without any exact statement as to whether these components also,
and to what extent, are things or perhaps just half-things;
but they are at least concretes in general.

The necessity of making the transition
from properties to matters,
or of assuming that the properties are truly matters,
has resulted from the fact that they are
what is the essential in things
and consequently their true self-subsistence.
At the same time, however,
the reflection of the property into itself
constitutes only one side of the whole reflection,
namely the sublation of the distinction
and the continuity of the property
(which was supposed to be a concrete existence for an other)
with itself.
Thinghood, as immanent negative reflection
and as a distinguishing that repels itself from the other,
has consequently been reduced to an unessential moment;
at the same time, however, it has further determined itself.

First, this negative moment has preserved itself,
for property has become a matter continuous with itself
and self-subsisting only inasmuch as
the difference of things has sublated itself;
thus the continuity of the property in the otherness
itself contains the moment of the negative,
and, as this negative unity,
its self-subsistence is at the same time
the restored something of thinghood,
negative self-subsistence versus
the positive self-subsistence of the stuff.

Second, the thing has thereby progressed
from its indeterminacy to full determinateness.
As thing in itself, it is abstract identity,
simple negative concrete existence,
or this concrete existence
determined as the indeterminate;
it is then determined through its properties,
by virtue of which it is supposed to be
distinguished from other things;
but, since through the property the thing is
rather continuous with other things,
this imperfect distinction is sublated;
the thing has thereby returned into itself
and is now determined as determined;
it is determined in itself or is this thing.

But, third, this turning back into itself,
though a self-referring determination,
is at the same time an unessential determination;
the self-continuous subsistence makes up
the self-subsistent matter
in which the difference of things,
their determinateness existing in and for itself,
is sublated and is something external.
Therefore, although the thing as this thing
is complete determinateness,
this determinateness is such
in the element of inessentiality.

Considered from the side of the movement of the property,
this result follows in this way.
The property is not only external determination
but concrete existence immediately existing in itself.
This unity of externality and essentiality repels itself from itself,
for it contains reflection-into-itself and reflection-into-other,
and, on the one hand, it is determination as simple,
self-identical and self-referring self-subsistent
in which the negative unity,
the one of the thing, is sublated;
on the other hand, it is this determination over against an other,
but likewise as a one which is reflected into itself
and is determined in itself;
it is, therefore, the matters and this thing.
These are the two moments of self-identical externality,
or of property reflected into itself.
The property was that by which things
were supposed to be distinguished.
Since the thing has freed itself of its
negative side of inhering in an other,
it has thereby also become free
from its being determined by other things
and has returned into itself
from the reference connecting it to the other.
At the same time, however, it is only the thing-in-itself
now become the other of itself,
for the manifold properties on their part
have become self-subsistent
and their negative connection
in the one of the thing is
now only a sublated connection.
Consequently, the thing is self-identical negation
only as against the positive continuity of the material.

The “this” thus constitutes the
complete determinateness of the thing,
a determinateness which is at the same time
an external determinateness.
The thing consists of self-subsistent matters
indifferent to the connection they have in the thing.
This connection is therefore only
an unessential linking of them,
the difference of one thing from another
depending on whether there is in it
a more or less of particular matters
and in what amount.
These matters overrun this thing,
continue into others,
and that they belong to this thing
is no restriction for them.
Just as little are they, moreover,
a restriction for one another,
for their negative connection is
only the impotent “this.”
Hence, in being linked together in it,
they do not sublate themselves;
they are as self-subsistent,
impenetrable to each other;
in their determinateness they refer only to themselves
and are a mutually indifferent manifold of subsistence;
the only limit of which they are capable is a quantitative one.
The thing as this is just their merely quantitative connection,
a mere collection, their “also.”
The thing consists of some quantum or other of a matter,
also of the quantum of another, and also of yet another;
this combination, of not having any combination
alone constitutes the thing.

C. DISSOLUTION OF THE THING

This thing, in the manner it has determined itself
as the merely quantitative combination of free matters,
is the absolutely alterable.
Its alteration consists in one or more matters
being dropped from the collection,
or being added to this “also,”
or in the rearrangement of the matters'
respective quantitative ratio.
The coming-to-be and the passing-away of this thing is
the external dissolution of such an external bond,
or the binding of such for which it is indifferent
whether they are bound or not.
The stuffs circulate unchecked in or out of “this” thing,
and the thing itself is absolute porosity
without measure or form of its own.

So the thing, in the absolute determinateness
through which it is a “this,”
is the absolutely dissoluble thing.
This dissolution is an external process of being determined,
just like the being of the thing;
but its dissolution and the externality of its being
is the essential of this being;
the thing is only the “also”;
it consists only of this externality.
But it consists also of its matters,
and not just the abstract “this” as such
but the “this” thing whole is the dissolution of itself.
For the thing is determined as an external collection
of self-subsisting matters;
such matters are not things,
they lack negative self-subsistence;
it is the properties which are rather self-subsistent,
that is to say, are determined with a being
which, as such, is reflected into itself.
Hence the matters are indeed simple, referring only to themselves;
but it is their content which is a determinateness;
the immanent reflection is only the form of this content,
a content which is not, as such, reflected-into-itself
but refers to an other according to its determinateness.
The thing, therefore, is not only their “also,”
is not their reference to each other as indifferent
but is, on the contrary, equally so their negative reference;
and on account of their determinateness
the matters are themselves this negative reflection
which is the puncticity of the thing.
The one matter is not what the other is
according to the determinateness of its content
as contrasted to that of an other;
and the one is not to the extent that the other is,
in accordance with their self-subsistence.

The thing is, therefore, the connecting reference of
the matters of which it consists to each other,
in such a manner that the one matter,
and the other also, subsist in it,
and yet, at the same time,
the one matter does not subsist
in it in so far as the other does.
To the extent, therefore, that
the one matter is in the thing,
the other is thereby sublated;
but the thing is at the same time
the “also,” or the subsistence of the other matter.
In the subsistence of the one matter, therefore,
the other matter does not subsist,
and it also no less subsists in it;
and so with all these diverse matters
in respect to each other.
Since it is thus in the same respect
as the one matter subsists
that the other subsists also,
and this one subsistence of both is
the puncticity or the negative unity of the thing,
the two interpenetrate absolutely;
and since the thing is at the same time
only the “also” of the matters,
and these are reflected into their determinateness,
they are indifferent to one another,
and in interpenetrating they do not touch.
The matters are, therefore, essentially porous,
so that the one subsists in the pores
or in the non-subsistence of the others;
but these others are themselves porous;
in their pores or their non-subsistence
the first and also all the rest subsist;
their subsistence is at the same time
their sublatedness and the subsistence of others;
and this subsistence of the others is
just as much their sublatedness
and the subsisting of the first
and equally so of all others.
The thing is, therefore,
the self-contradictory mediation of
independent self-subsistence through its opposite,
that is to say, through its negation,
or of one self-subsisting matter
through the subsisting and non-subsisting of an other.

In “this” thing, concrete existence has attained its completion,
namely, that it is at once being that exists in itself,
or independent subsistence, and unessential concrete existence.
The truth of concrete existence is thus this:
that it has its in-itself in unessentiality,
or that it subsists in an other,
indeed in the absolute other,
or that it has its own nothingness for substrate.
It is, therefore, appearance.

CHAPTER 2

Appearance

Concrete existence is the immediacy of being
to which essence has again restored itself.
In itself this immediacy is the reflection of essence into itself.
As concrete existence, essence has stepped out of its ground
which has itself passed over into it.
Concrete existence is this reflected immediacy
in so far as, within, it is absolute negativity.
It is now also posited as such,
in that it has determined itself as appearance.

At first, therefore, appearance is
essence in its concrete existence;
essence is immediately present in it.
That it is not immediate,
but rather reflected concrete existence,
constitutes the moment of essence in it;
or concrete existence,
as essential concrete existence,
is appearance.

Something is only appearance,
in the sense that concrete existence is
as such only a posited being,
not something that is in-and-for-itself.
This is what constitutes its essentiality,
to have the negativity of reflection,
the nature of essence, within it.
There is no question here of an alien,
external reflection to which essence would belong
and which, by comparing this essence with concrete existence,
would declare the latter to be appearance.
On the contrary, as we have seen,
this essentiality of concrete existence,
that it is appearance, is
concrete existence's own truth.
The reflection by virtue of which
it is this is its own.

But if it is said that something is only appearance,
meaning that as contrasted with it
immediate concrete existence is the truth,
then the fact is that appearance is the higher truth,
for it is concrete existence as essential,
whereas concrete existence is appearance
that is still void of essence
because it only contains in it
the one moment of appearance,
namely that of concrete existence
as immediate, not yet negative, reflection.
When appearance is said to be essenceless,
one thinks of the moment of its negativity as if,
by contrast with it, the immediate were
the positive and the true;
in fact, however, this immediate does not
yet contain essential truth in it.
Concrete existence rather ceases to be essenceless
by passing over into appearance.

Essence reflectively shines at first
just within, in its simple identity;
as such, it is abstract reflection,
the pure movement of nothing
through nothing back to itself.
Essence appears, and so it now is real shine,
since the moments of the shine have concrete existence.
Appearance, as we have seen, is the thing as
the negative mediation of itself with itself;
the differences which it contains
are self-subsisting matters
which are the contradiction of
being an immediate subsistence,
yet of obtaining their subsistence
only in an alien self-subsistence,
hence in the negation of their own,
but then again, just because of that,
also in the negation of that alien self-subsistence
or in the negation of their own negation.
Reflective shine is this same mediation,
but its fleeting moments obtain in appearance
the shape of immediate self-subsistence.
On the other hand, the immediate self-subsistence
which pertains to concrete existence is reduced to a moment.
Appearance is therefore the unity of
reflective shine and concrete existence.

Appearance now determines itself further.
It is concrete existence as essential;
as essential, concrete existence
differs from the concrete existence
which is unessential,
and these two sides
refer to each other.

Appearance is, therefore,
first, simple self-identity
which also contains
diverse content determinations
and, both as identity
and as the connecting reference
of these determinations,
is that which remains self-equal
in the flux of appearance;
this is the law of appearance.

But, second, the law which is
simple in its diversity
passes over into opposition;
the essential moment of appearance becomes
opposed to appearance itself
and, confronting the world of appearance,
the world that exists in itself
comes onto the scene.

Third, this opposition returns into its ground;
that which is in itself is in the appearance
and, conversely, that which appears is determined
as taken up into its being-in-itself.
Appearance becomes relation.

A. THE LAW OF APPEARANCE

1. Appearance is the concrete existent
mediated through its negation,
which constitutes its subsistence.
This, its negation, is
indeed another self-subsistent;
but the latter is just as
essentially something sublated.
The concrete existent is consequently
the turning back of itself into itself
through its negation and through
the negation of this negation;
it has, therefore, essential self-subsistence,
just as it is equally immediately an absolute positedness
that has a ground and an other for its subsistence.
In the first place, therefore, appearance is
concrete existence along with its essentiality,
the positedness along with its ground;
but this ground is the negation,
and the other self-subsistent,
the ground of the first,
is equally only a positedness.
Or the concrete existent is,
as an appearance,
reflected into an other
and has this other for its ground,
and this ground is itself only this,
to be reflected into another.
The essential self-subsistence
that belongs to it because
it is a turning back into itself is,
for the sake of the negativity of the moments,
the return of nothing through nothing back to itself;
the self-subsistence of the concrete existent is
therefore only the reflective shine of essence.
The linkage of the reciprocally grounding
concrete existents consists, therefore,
in this reciprocal negation,
namely that the subsistence of the one is not
the subsistence of the other but is its positedness,
where this connection of positedness
alone constitutes their subsistence.
The ground is present as it is in truth,
namely as being a first which is only a presupposed.

This now constitutes the negative side of appearance.
In this negative mediation, however,
there is immediately contained the positive identity of
the concrete existent with itself.
For this concrete existent is not positedness
vis-à-vis an essential ground,
or is not the reflective shine in a self-subsistent,
but is rather positedness that refers itself to a positedness,
or a reflective shine only in a reflective shine.
In this, its negation, or in its other
which is itself something sublated,
it refers to itself and is thus
self-identical or positive essentiality.
This identity is not the immediacy
that pertains to concrete existence as such
and only is its unessential moment of subsisting in an other.
It is rather the essential content of appearance which has two sides:
first, to be in the form of positedness or external immediacy;
second, to be positedness as self-identical.
According to the first side, it is as a determinate being,
but one which in keeping with its immediacy is accidental, unessential,
and subject to transition, to coming-to-be and passing-away.
According to the other side, it is the simple content determination
exempted from that flux, the permanent element in it.

This content, besides being in general
the simple element of the transient,
is also a determined content, varied in itself.
It is the reflection of appearance,
of the negative determinate being, into itself,
and therefore contains determinateness essentially.
Appearance is however the multifarious diversity of
immediately existing beings that revels in unessential manifoldness;
its reflected content, on the other hand,
is its manifoldness reduced to simple difference.
Or, more precisely, the determinate essential content
is not just determined in general but,
as the essential element of appearance,
is complete determinateness;
the one and its other.
Each of these two has in appearance
its subsistence in the other,
but in such a way that it is at the same time
only in the other's non-subsistence.
This contradiction sublates itself;
and its reflection into itself is
the identity of their two-sided subsistence,
namely that the positedness of the one is
also the positedness of the other.
The two constitute one subsistence,
each at the same time as a different content
indifferent to the other.
In the essential side of appearance,
the negativity of the unessential content,
that it sublates itself, has thus gone back into identity;
it is an indifferent subsistence
which is not the sublatedness of the other
but rather its subsistence.

This unity is the law of appearance.

2. The law is thus the positive element
of the mediation of what appears.
Appearance is at first concrete existence
as negative self-mediation,
so that the concrete existent,
through its own non-subsistence,
through an other and again through
the non-subsistence of this other,
is mediated with itself.
In this there is contained,
first, the merely reflective shining
and the disappearing of both,
the unessential appearance;
second, also the persistence or the law;
for each of the two concretely exists
in the sublation of the other,
and their positedness is as
their negativity at the same time
the identical positive positedness of both.

This permanent subsistence which appearance
obtains in the law is thus,
as it has determined itself,

first, opposed to the immediacy
of the being which concrete existence has.
This immediacy is indeed one which is in itself reflected,
namely the ground that has gone back into itself;
but in appearance this simple immediacy is now distinguished
from the reflected immediacy that first began
to separate itself in the “thing.”
The concretely existing thing in its dissolution
has become this opposition;
the positive element of its dissolution is
the said self-identity of what appears,
a positedness in the positedness of its other.

Second, this reflected immediacy is itself determined
as positedness over against the immediate determinate
being of concrete existence.
This positedness is henceforth what is essential
and the true positive.
The German expression Gesetz [law] likewise contains this
note of positedness or Gesetztsein.
In this positedness there lies the essential connection of
the two sides of the difference that the law contains;
they are a diverse content,
each immediate with respect to the other,
and they are this as the reflection of
the disappearing content belonging to appearance.
As essential difference, the different sides are
simple, self-referring determinations of content.
But just as equally, neither is immediate,
just for itself, but is rather essential positedness,
or is only to the extent that the other is.

Third, appearance and law have one and the same content.
The law is the reflection of appearance into self-identity;
appearance, as an immediate which is null,
thus stands opposed to that which is immanently reflected,
and the two are distinguished according to form.
But the reflection of appearance
by virtue of which this difference is,
is also the essential identity of
appearance itself and its reflection,
and this is in general the nature of reflection;
it is what in the positedness is self-identical
and indifferent to that difference,
which is form or positedness
hence a content continuous
from appearance to law,
the content of the law
and of the appearance.

This content thus constitutes the substrate of appearance;
the law is this substrate itself,
appearance is the same content but contains still more,
namely the unessential content of its immediate being.
And so is also the form determination by which
appearance as such is distinguished from the law,
namely a content and equally a content distinguished
from the content of the law.
For concrete existence, as immediacy in general,
is likewise a self-identity of matter and form
which is indifferent to its form determinations
and is, therefore, a content;
the concrete existence is the thinghood
with its properties and matters.
But it is the content whose self-subsisting immediacy is
at the same time also only a non-subsistence.
But the self-identity of the content
in this its non-subsistence
is the other, essential content.
This identity, the substrate of appearance,
which constitutes law,
is appearances's own moment;
it is the positive side of the essentiality
by virtue of which concrete existence is appearance.

The law, therefore, is not beyond appearance
but is immediately present in it;
the kingdom of laws is the restful copy
of the concretely existing or appearing world.
But, more to the point, the two are one totality,
and the concretely existing world is itself
the kingdom of laws which, simple identity,
is at the same time self-identical in the positedness
or in the self-dissolving self-subsistence of concrete existence.
In the law, concrete existence returns to its ground;
appearance contains both of these, the simple ground
and the dissolving movement of the appearing universe,
of which the law is the essentiality.

3. The law is therefore the essential appearance;
it is the latter's reflection into itself in its positedness,
the identical content of itself and the unessential concrete existence.

In the first place, this identity of the law
with its concrete existence is now, to start with,
immediate, simple identity, and the law is indifferent
with respect to its concrete existence;
appearance still has another content as
contrasted with the content of the law.
That content is indeed the unessential one
and the return into the latter;
but for the law it is an original starting point not posited by it;
as content, therefore, it is externally bound up with the law.
Appearance is an aggregate of more detailed determinations
that belong to the “this” or the concrete,
and are not contained in the law
but are rather determined each by an other.

Secondly, that which appearance contains
distinct from the law determined itself as
something positive or as another content;
but it is essentially a negative;
it is the form and its movement is
a movement that belongs to appearance.
The kingdom of laws is
the restful content of appearance;
the latter is this same content
but displayed in restless flux
and as reflection-into-other.
It is the law as negative,
relentlessly self-mutating concrete existence,
the movement of the passing over into the opposite,
of self-sublation and return into unity.
This side of the restless form
or of the negativity
does not contain the law;
as against the law, therefore,
appearance is the totality,
for it contains the law but more yet,
namely the moment of the self-moving form.

Thirdly, this shortcoming is manifested
in the law in the mere diversity at first,
and the consequent internal indifference, of its content;
the identity of its sides with one another
is at first, therefore,
only immediate and hence inner,
not yet necessary in other words.
In a law two content determinations are
essentially bound together
(for instance, spatial and temporal magnitudes
in the law of falling bodies:
the traversed spaces vary as
the squares of the elapsed times);
they are bound together;
this connection is at first
only an immediate one.

At first, therefore, it is likewise only a posited connection,
just as the immediate has obtained in appearance
the meaning of positedness in general.
The essential unity of the two sides
of the law would be their negativity,
namely that each contains the other in it;
but in the law this essential unity has not yet come the fore.
(Thus it is not contained in the concept of
the space traversed by a falling body
that time corresponds to it as a square.
Because the falling is a sensible movement,
it is the ratio of space and time;
but first, that time refers to space and space to time
does not lie in the determination of time itself,
that is to say, in time as ordinarily represented;
it is said that time can very well be represented
without space and space without time;
the one thus comes to the other externally,
and their external reference to each other is movement.
Second, the more particular determination of
how the magnitudes further relate to
each other in movement is indifferent.
The relevant law here is drawn from experience
and is to this extent immediate;
there is still required a proof,
that is, a mediation,
in order to know that the law
not only occurs but is necessary;
the law as such does not contain
this proof and its objective necessity.)

The law is, therefore, only
the positive essentiality of appearance,
not its negative essentiality according to which
the content determinations are moments of the form,
as such pass over into their other
and are in their own selves
not themselves but their other.
In the law, therefore, although
the positedness of the one side of it is
the positedness of the other side,
the content of the two sides is
indifferent to this connection;
it does not contain this positedness in it.
Law, therefore, is indeed essential form,
but not as yet real form which is reflected
into its sides as content.

B. THE WORLD OF APPEARANCE AND THE WORLD-IN-ITSELF

1. The concrete existing world tranquilly
raises itself to a kingdom of laws;
the null content of its manifold determinate being
has its subsistence in an other;
its subsistence is therefore its dissolution.
In this other, however, that which appears also comes to itself;
thus appearance is in its changing also an enduring,
and its positedness is law.
Law is this simple identity of appearance with itself;
it is, therefore, its substrate and not its ground,
for it is not the negative unity of appearance
but, as its simple identity, is its immediate unity,
the abstract unity, alongside which, therefore,
its other content also occurs.
The content is this content; it holds together internally,
or has its negative reflection inside itself.
It is reflected into an other;
this other is itself a concrete existence of appearance;
the appearing things have their grounds and conditions
in other appearing things.

In fact, however, law is also
the other of appearance as appearance,
and its negative reflection as in its other.
The content of appearance,
which differs from the content of law,
is the concrete existent
which has negativity for its ground
or is reflected into its non-being.
But this other, which is also a concrete existent,
is such an existent as likewise reflected into its non-being;
it is thus the same and that which appears in it
is in fact reflected not into an other but into itself;
it is this very reflection of positedness into itself
which is law.
But as something that appears
it is essentially reflected into its non-being,
or its identity is itself essentially
just as much its negativity and its other.
The immanent reflection of appearance,
law, is therefore not only
the identical substrate of appearance
but the latter has in law its opposite,
and law is its negative unity.

Now through this, the determination of law
has been altered within the law itself.
At first, law is only a diversified content
and the formal reflection of positedness into itself,
so that the positedness of one of its sides is
the positedness of the other side.
But because it is also the negative reflection into itself,
its sides behave not only as different
but as negatively referring to each other.
Or, if the law is considered just for itself,
the sides of its content are indifferent to each other;
but they are no less sublated through their identity;
the positedness of the one is the positedness of the other;
consequently, the subsistence of each is
also the non-subsistence of itself.
This positedness of the one side in the other
is their negative unity,
and each positedness is not only the positedness
of that side but also of the other,
or each side is itself this negative unity.
The positive identity which they have in the law as such is
at first only their inner unity
which stands in need of proof and mediation,
since this negative unity is not yet posited in them.
But since the different sides of law are now determined
as being different in their negative unity,
or as being such that each contains the other within
while at the same time repelling this otherness from itself,
the identity of law is now also one which is posited and real.

Consequently, law has likewise obtained
the missing moment of the negative form of its sides,
the moment that previously still belonged to appearance;
concrete existence has thereby returned into itself fully
and has reflected itself into its absolute otherness
which has determinate being-in-and-for-itself.
That which was previously law, therefore,
is no longer only one side of the whole.
It is the essential totality of appearance,
so that it now obtains also the moment of
unessentiality that belonged to the latter
but as reflected unessentiality
that has determinate being in itself,
that is, as essential negativity.
As immediate content, law is determined in general,
distinguished from other laws,
of which there is an indeterminate multitude.
But because now it explicitly is essential negativity,
it no longer contains that merely
indifferent, accidental content determination;
its content is rather every determinateness in general,
essentially connected together in a totalizing connection.
Thus appearance reflected-into-itself is
now a world that discloses itself above
the world of appearance as one
which is in and for itself.

The kingdom of laws contains only
the simple, unchanging but diversified content
of the concretely existing world.
But because it is now the total reflection of this world,
it also contains the moment of its essenceless manifoldness.
This moment of alterability and alteration,
reflected into itself and essential,
is the absolute negativity
or the form in general as such:
its moments, however, have  the reality of
self-subsisting but reflected concrete existence
in the world  that has determinate being in-and-for-itself,
just as, conversely, this reflected
self-subsistence has form in it,
and its content is therefore not a mere manifold
but a content holding itself together essentially.

This world which is in and for itself is
also called the suprasensible world,
inasmuch as the concretely existing world
is characterized as sensible,
that is, as one intended for intuition,
which is the immediate attitude of consciousness.
The suprasensible world likewise has
immediate, concrete existence,
but reflected, essential concrete existence.
Essence has no immediate existence yet;
but it is, and in a more profound sense than being;
the thing is the beginning of the reflected concrete existence;
it is an immediacy which is not yet posited,
not yet essential or reflected;
but it is in truth not an immediate which is simply there.
Things are posited only as the
things of another, suprasensible, world
first as true concrete existences,
and, second, as the truth in contrast to that which just is.
What is recognized in them is that there is
a being distinguished from immediate being,
and this being is true concrete existence.
On the one side, the sense-representation
that ascribes concrete existence
only to the immediate being of
feeling and intuition is in this determination overcome;
but, on the other side, also overcome is
the unconscious reflection which,
although it possesses the representation of things,
forces, the inner, and so on, does not know
that such determinations are not sensible
or immediately existing beings,
but reflected concrete existences.

2. The world which is in and for itself is
the totality of concrete existence;
outside it there is nothing.
But, within it, it is absolute negativity or form,
and therefore its immanent reflection is
negative self-reference.
It contains opposition,
and splits internally
as the world of the senses
and as the world of otherness
or the world of appearance.
For this reason, since it is totality,
it is also only one side of the totality
and constitutes in this determination
a self-subsistence different from the world of appearance.
The world of appearance has its negative unity
in the essential world to which it founders
and into which it returns as to its ground.
Further, the essential world is also
the positing ground of the world of appearances;
for, since it contains the absolute form essentially,
it sublates its self-identity,
makes itself into positedness
and, as this posited immediacy,
it is the world of appearance.

Further, it is not only ground in general
of the world of appearance but its determinate ground.
Already as the kingdom of laws it is a manifold of content,
indeed the essential content of the world of appearance,
and, as ground with content, it is
the determinate ground of that other world.
But it is such only according to that content,
for the world of appearance still had other
and manifold content than the kingdom of laws,
because the negative moment was still the one peculiarly its own.
But because the kingdom of laws now has this moment likewise in it,
it is the totality of the content of the world of appearance
and the ground of all its manifoldness.
But it is at the same time the negative of
this manifoldness and thus a world opposed to it.
That is to say, in the identity of the two worlds,
because the one world is determined
according to form as the essential
and the other as the same world
but posited and unessential,
the connection of ground has indeed been restored.
But it has been restored as the ground-connection of appearance,
namely as the connection,
not of the two sides of an identical content,
nor of a mere diversified content, like law,
but as total connection,
or as negative identity and essential connection
of the opposed sides of the content.
The kingdom of laws is not only this,
that the positedness of a content
is the positedness of an other,
but rather that this identity, as we have seen,
is essentially also negative unity,
and in this negative unity
each of the two sides of law is in it,
therefore, its other content;
consequently, the other is not
an other in general, indeterminedly,
but is its other, equally containing
the content determination of that other;
and thus the two sides are opposed.
Now, because the kingdom of laws now has in it
this negative moment, namely opposition,
and thus, as totality, splits into a world
which exists in and for itself and a world of appearance,
the identity of these two is
the essential connection of opposition.
The connection of ground is, as such, the opposition
which, in its contradiction, has foundered to the ground;
and concrete existence is the ground that has come to itself.
But concrete existence becomes appearance;
ground is sublated in concrete existence;
it reinstates itself as the return of appearance into itself,
but does so as sublated ground, that is to say,
as the ground-connection of opposite determinations;
the identity of such determinations, however, is
essentially a becoming and a transition,
no longer the connection of ground as such.

The world that exists in and for itself is
thus itself a world distinguished within itself,
in the total compass of a manifold content.
That is to say, it is identical with the world of appearance
or the posited world and to this extent it is its ground.
But its identity connection is at the same time
determined as opposition,
because the form of the world of appearance is
reflection into its otherness
and this world of appearance, therefore, in the
world that exists in and for itself
has truly returned into itself,
in such a manner that
that other world is its opposite.
Their connection is, therefore, specifically this,
that the world that exists in and for itself is
the inversion of the world of appearance.

C. THE DISSOLUTION OF APPEARANCE

The world that exists in and for itself is
the determinate ground of the world of appearance
and is this only in so far as, within it,
it is the negative moment
and hence the totality of the content determinations
and their alterations that correspond to that world of appearance,
yet constitutes at the same time its completely opposed side.
The two worlds thus relate to each other in such a way
that what in the world of appearance is positive,
in the world existing in and for itself is negative,
and, conversely, what is negative in the former
is positive in the latter.
The north pole in the world of appearance
is the south pole in and for itself, and vice-versa;
positive electricity is in itself negative, and so forth.
What is evil in the world of appearance is
in and for itself goodness and a piece of good luck.

In fact it is precisely in this opposition
of the two worlds that their difference has disappeared,
and what was supposed to be
the world existing in and for itself is
itself the world of appearance
and this last, conversely,
the world essential within.
The world of appearance is in the first instance
determined as reflection into otherness,
so that its determinations and concrete existences have
their ground and subsistence in an other;
but because this other, as other,
is likewise reflected into an other,
the other to which they both refer is
one which sublates itself as other;
the two consequently refer to themselves;
the world of appearance is within it,
therefore, law equal to itself.
Conversely, the world existing in and for itself is
in the first instance self-identical content,
exempt from otherness and change;
but this content, as complete reflection of
the world of appearance into itself,
or because its diversity is difference
reflected into itself and absolute,
consequently contains negativity as a moment
and self-reference as reference to otherness;
it thereby becomes self-opposed, self-inverting, essenceless content.
Further, this content of the world existing in and for itself has
thereby also retained the form of immediate concrete existence.
For it is at first the ground of the world of appearance;
but since it has opposition in it, it is equally
sublated ground and immediate concrete existence.

Thus the world of appearance and the essential world are
each, each within it, the totality of
self-identical reflection and of reflection-into-other,
or of being-in-and-for-itself.
They are both the self-subsisting wholes of concrete existence;
the one is supposed to be only reflected concrete existence,
the other immediate concrete existence;
but each continues into the other and, within, is
therefore the identity of these two moments.
What we have, therefore, is this totality
that splits into two totalities,
the one reflected totality and the other immediate totality.
Both, in the first instance, are self-subsistent;
but they are this only as totalities,
and this they are inasmuch as each essentially contains
the moment of the other in it.
Hence the distinct self-subsistence of each,
one determined as immediate and one as reflected,
is now so posited as to be essentially the reference to the other
and to have its self-subsistence in this unity of the two.

We started off from the law of appearance;
this law is the identity of a content
and another content different from it,
so that the positedness of the one
is the positedness of the other.
Still present in law is this difference,
that the identity of its sides is
at first only an internal identity
which the two sides do not yet have in them.
Consequently the identity is, for its part, not realized;
the content of law is not identical
but indifferent, diversified.
This content, therefore, is on its side only in itself
so determined that the positedness of the one is
the positedness of the other;
this determination is not yet present in it.
But now law is realized;
its inner identity is existent at the same time
and, conversely, the content of law is raised to ideality;
for it is sublated within, is reflected into itself,
for each side has the other in it,
and therefore is truly identical
with it and with itself.

Thus is law essential relation.
The truth of the unessential world is at first
a world in and for itself and other to it;
but this world is a totality,
for it is itself and the first world;
both are thus immediate concrete existences
and consequently reflections in their otherness,
and therefore equally truly reflected into themselves.
“World” signifies in general
the formless totality of a manifoldness;
this world has foundered both as essential world
and as world of appearance;
it is still a totality or a universe
but as essential relation.
Two totalities of content have arisen in appearance;
at first they are determined as
indifferently self-subsisting vis-à-vis each other,
each having indeed form within it
but not with respect to the other;
this form has however demonstrated
itself to be their connecting reference,
and the essential relation is
the consummation of their unity of form.

CHAPTER 3

The essential relation

The truth of appearance is the essential relation.
Its content has immediate self-subsistence:
the existent immediacy and the reflected immediacy
or the self-identical reflection.
In this self-subsistence, however,
it is at the same time a relative content;
it is simply and solely as a reflection into its other,
or as unity of the reference with its other.
In this unity, the self-subsistent content is
something posited, sublated;
but precisely this unity is what constitutes
its essentiality and self-subsistence;
this reflection into an other is reflection into itself.
The relation has sides, since it is reflection into an other;
so its difference is internal to it,
and its sides are independent subsistence,
for in their mutually indifferent diversity
they are thrown back into themselves,
so that the subsistence of each equally has its meaning
only in its reference to the other
or in the negative unity of both.

The essential relation is therefore not yet
the true third to essence and to concrete existence
but already contains the determinate union of the two.
Essence is realized in it in such a way that
it has self-subsistent, concrete existents for its subsistence,
and these concrete existents have returned
from their indifference back into their essential unity
so that they have only this unity as their subsistence.
Also the reflective determinations
of positive and negative are
reflected into themselves only as
each is reflected into its opposite;
but they have no other determination
besides this their negative unity,
whereas the essential relation has sides
that are posited as self-subsistent totalities.
It is the same opposition as that of positive and negative,
but it is such as an inverted world.
The side of the essential relation is a totality
which, however, essentially has an opposite or a beyond;
it is only appearance;
its concrete existence,
rather than being its own,
is that of its other.
It is, therefore, something internally fractured;
but this, its sublated being, consists in
its being the unity of itself and its other,
therefore a whole, and precisely for this reason
it has self-subsistent concrete existence
and is essential reflection into itself.

This is the concept of relation.
At first, however, the identity it contains
is not yet perfect;
the totality which each relative is as relative,
is only an inner one;
the side of the relation is posited at first
in one of the determinations of negative unity;
what constitutes the form of the relation is
the specific self-subsistence of each of the two sides.
The identity of the form is therefore only a reference,
and the self-subsistence of the sides falls outside it,
that is to say, it falls in the sides;
we still do not have the reflected unity
of the identity of the relation
and of the self-subsistent concrete existents;
we still do not have substance.
It follows that the concept of relation has
indeed shown itself to be the unity
of reflected and immediate self-subsistence.
But it is this concept still immediately at first;
immediate are therefore its moments vis-à-vis each other,
and immediate is the unity of the reference
connecting them essentially;
a unity this, which only then is the true unity
that conforms to the concept,
when it has realized itself, that is to say,
through its movement has posited itself as this unity.

The essential relation is therefore immediately
the relation of the whole and the parts
the reference of reflected and immediate self-subsistence,
so that both are at the same time
mutually conditioning and presupposing.

In this relation, neither of the sides is
yet posited as moment of the other;
their identity is therefore itself one side,
or not their negative unity.
Hence, secondly, the relation passes over into one
in which one side is the moment of the other
and is present there as in its ground,
the true self-subsistent element of both.
This is the relation of force and its expression.

Third, the inequality still present
in this reference sublates itself,
and the final relation is that of inner and outer.
In this difference,
which has now become totally formal,
relation itself founders,
and substance or actuality come on the stage
as the absolute unity of
immediate and reflected concrete existence.
