
C. ACTUALITY

Actuality is that unity of essence and concrete existence,
of inner and outer, that has immediately come to be.
The expression of the actual is the actual itself,
so that in the expression it remains something equally essential
and is something essential only insofar as it is
in immediate, external concrete existence.

As forms of the immediate,
being and concrete existence surfaced earlier;
being is completely unreflected immediacy and
[the] passing over into an other.
The concrete existence is immediate unity of
being and reflection, thus appearance,
coming from the ground and returning to it.
The actual is the positedness of that unity,
relationship that has become identical with itself.
It is thus exempted from the passing over and
its externality is its energy;
in that externality, it is reflected in itself;
its existence is only the manifestation of itself,
not of an other.

The actuality, as this concrete [dimension],
contains those determinations and their difference;
it is, for that reason, also their development
so that they are determined in it at once
as a shine, as merely posited.

(a) As identity generally it is initially the possibility;
the reflection-in-itself that is posited as
the abstract and unessential essentiality
in contrast to the concrete unity of the actual.
Possibility is what is essential for actuality but
such that it is at the same time only possibility.

It is probably the determination of possibility that caused
Kant to regard it, along with actuality and necessity, as modalities,
because these determinations did not in the slightest add to the concept
as object but instead express only the connection to the
capacity of knowing [Critique of Pure Reason, B 266].
Possibility is indeed the empty abstraction of the reflection-in-itself,
what was previously called 'the inner', with the only difference that
it is now determined as the sublated, merely posited, external inner,
and thus, to be sure, is also posited as a mere modality,
as insufficient abstraction, something that, taken more concretely,
pertains only to subjective thinking.
Actuality and necessity are, by contrast, truly
anything but a mere sort and manner for an other; rather, they are
precisely the opposite, posited as the not merely posited but instead
as the concrete [dimension] that is complete in itself.
Because possibility, initially contrasted with
the concrete as something actual,
is the mere form of identity-with-itself,
the rule for it is merely that something not be self-contradictory and
thus everything is possible;
for this form of identity can be given to any content through abstraction.
But everything is just as much impossibLe, for in
every content, since it is something concrete, the determinacy can
be grasped as determinate opposition and thus as contradiction.
There is, thus, no more empty way of speaking than about this sort
of possibility and impossibility.
In philosophy, in particular, there should not be any talk of showing that
something is possible or that something else is also possible and
that something, as one also expresses it, is thinkable.
The historian is also directly advised not to use this
category that was already declared to be untrue for itself;
but the acumen of empty understanding is never more pleased with itself
than when it emptily devises possibilities and an abundant supply of them.

(b) In its difference from possibility as the reflection-in-itself, however,
the actual is itself only the externally concrete [dimension],
the immediate in an inessential way.
Or immediately, insofar as it initially is as the simple,
itself immediate unity of the inner and the outer,
it is what is external in an inessential way and
is thus at the same time what is only internal,
the abstraction of the reflection-in-itself;
it itself is thereby determined as something only possible.
In this value of a mere possibility, the actual is something contingent and,
vice versa, possibility is mere contingency itself.

Possibility and contingency are the moments of actuality, inner and outer,
posited as mere forms that constitute the externality of the actual.
In the actual qua determined in-itself,
in the content as the essential ground of their determination,
they have their reflection-in-itself.
The finitude of the contingent and possible thus consists,
more precisely, in the fact that the form determination is
differentiated from the content and, hence,
whether something is contingent and possible depends on the content.

That externality of actuality contains more precisely this:
that the contingency as immediate actuality is
essentially what is identical with itself only as being posited,
but a being posited that is just as much sublated,
an existing externality.
It is thus something presupposed,
the immediate existence of which
is at the same time a possibility
and has the determination of being sublated,
of being the possibility of another: the condition.

(c) This externality, developed in the manner depicted,
is a circle of determinations of possibility
and of the immediate actuality,
their mediation by one another,
the real possibility in general.
As such a circle, it is furthermore
the totality, thus the content,
the basic matter determined in and for itself,
and equally, in keeping with the
difference of determinations in this unity,
the concrete totality of the form for itself,
the immediate self-transposing of
the inner into the outer
and of the outer into the inner.
This self-moving of the form is activity,
activation of the basic matter as the real ground
that sublates itself and comes to be actual,
and activation of the contingent actuality, the conditions,
namely, their reflection-in-themselves and their self-sublating
to become another actuality,  the actuality of the basic matter.
If all conditions are at hand, the basic matter must become actual
and the basic matter is itself one of the conditions
since as something initially inner,
it is itself only something presupposed.
The developed actuality as the alternation
of the inner and the outer collapsing into one,
the alternation of its opposite movements
that are united into one movement, is necessity.

Necessity has been rightly defined, to be sure,
as the unity of possibility and actuality.
But this determination is superficial
and, for that reason, not understandable
if expressed only in this way.
The concept of necessity is very difficult
and, indeed, it is so because it is the concept itself
whose moments still are as actualities that,
nonetheless, have to be grasped at the same time merely as forms,
as in themselves broken and transitional.
For this reason, in both of the following sections,
the exposition of the moments that
constitute the necessity has to be given
in even greater detail.

Among the three moments, the condition, the basic matter, and the activity

a. the condition is

(a) something presupposed as only something supposed,
it is merely in the sense of being
relative to the basic matter,
but as presupposed it is in the sense of
a contingent, external condition,
concretely existing for itself
without regard for the basic matter.
But at the same time, in this contingency,
in regard to the basic matter which is the totality,
this presupposition is a complete circle of conditions.

(b) The conditions are passive,
they are used as material for the basic matter,
and thereby enter into the content of the basic matter.
They are just as much suited to this content
and already contain its entire determination within themselves.

b. The basic matter is equally

(a) something presupposed;
as supposed, it is initially merely something internal and possible
and, as pre-supposed, a self-sufficient content for itself

(b) Through the use of the conditions,
it obtains its external concrete existence,
realizing the determinations of its content,
determinations that correspond mutually to the conditions,
so that it both proves itself to be the basic matter
on the basis of these conditions and emerges from them.

c. The activity is

(a) also something self-sufficient and existing concretely for itself
(a human being, a character)
and, at the same time, it has its possibility
solely thanks to the conditions and the basic matter.

(b) It is the movement of translating
the conditions into the basic matter and
the basic matter into the conditions
as the side of concrete existence;
but the movement only of setting
the basic matter forth from the conditions
(in which it is on hand in itself and by way of sublating
the concrete existence of the conditions,
providing the basic matter with concrete existence.

Insofar as these three moments have the shape of a self-sufficient concrete
existence opposite one another,
this process is the external necessity.
This necessity has a limited content with respect to its basic matter.
For the basic matter is this whole in a simple determinacy.
But since it is in its form external to itself,
it is thereby also external to itself in itself and in its content,
and this externality with respect to the basic matter is
a limitation of its content.

Necessity is thus in itself the one essence,
identical with itself, but full of content,
the essence that shines in itself in such a way
that its differences have the form of self-sufficient actuals
and this identity, as the absolute form, is at the same time
the activity of sublating [immediacy] in mediated being and
the mediation in immediacy.
What is necessary is through an other that has broken up
into the mediating ground (the basic matter and the activity)
and an immediate actuality,
something contingent that is at the same time a condition.
Insofar as it is through an other, the necessary is
not in and for itself but instead something merely posited.
But this mediation is just as immediately the sublating of itself;
the ground and the contingent condition are transposed into immediacy,
by means of which that positedness is sublated to become actuality and
the basic matter has come together with itself.
In this return into itself, the necessary is
in an unqualified way, as unconditioned actuality.
The necessary is the way it is, mediated by a circle of circumstances,
it is so, because the circumstances are so;
and, at the same time, it is the way it is, unmediated,
it is so, because it is.

a. The relationship of substantiality

The necessary is in itself the absolute relationship,
the process (developed in the preceding sections) in which
the relationship equally sublates itself to become absolute identity.

In its immediate form, it is the relationship of substantiality and acciden-
tality. The absolute identity of this relationship with itself is the substance
as such which, as necessity, is the negativity of this form of interiority, thus
positing itself as actuality, but which is just as much the negativity of this
outer dimension, in keeping with which the actual as immediate is only
something accidental that, thanks to this, its mere possibility, passes over
into another actuality; a passing over which is the substantial identity as the
activity of the form·

The substance is accordingly the totality of the accidents in which it reveals
itself as their absolute negativity, as absolute power and at the same time
as the wealth of all content. This content, however, is nothing other than
this manifestation itself since the determinacy itself, reflected in itself [and
thus made into] the content, is only a moment of the form,
a moment that passes over into the power of the substance.
The substantiality is the
absolute activity of the form and the power of the necessity, and all content
is only a moment that belongs to this process alone,
the absolute turning over of form and content into one another.
Substance, qua absolute power, is the power that
relates itself to itself as only inner possibility,
determining itself thereby to accidentality,
whereby the externality thus posited is distinguished from it.
Just as it is substance in the first form of necessity,
so substance is, according to the moment just described,
genuine relationship: the relationship of causality.

b. The relationship of causality

Substance is cause insofar as it is reflected in itself
against its passing over into accidentality and
is thus the original basic matter,
but just as much supersedes the reflection-in-itself
or its mere possibility,
posits itself as the negative of itself and
in this way brings forth an effect,
an actuality which is only a posited actuality,
but through the process of effecting is
at the same time a necessary actuality.

As the original basic matter, the cause has
the determination of absolute self-sufficiency and
a subsisting that maintains itself opposite the effect.
But in the necessity, the identity of which
constitutes that originality itself,
it has merely passed over into the effect.
There is no content in the effect that is not in the cause,
insofar as it is possible again to talk of a determinate content.
That identity is the absolute content itself.
But it is also equally the determination of form,
the originality is sublated in the effect in which
it makes itself something posited.
With this, however, the cause has not vanished such that
the actual would be only the effect.
For this positedness is immediately superseded just as much;
it is indeed the reflection-in-itself of the cause, its originality;
the cause is first actual and cause in the effect.
The cause is thus in and for itself causa sui [cause of itself].
Jacobi, firmly caught up in the one-sided
representation of the mediation, took the causa sui (the effectus sui is
the same), this absolute truth of the cause, merely for a formalism.
He also put forward that God must be determined, not as ground,
but essentially as cause.
That this move did not achieve what he
intended would have emerged from thinking over the nature of
cause much more thoroughly.
Even in a finite cause and its representation,
this identity in regard to the content is at hand;
the rain, the cause, and the wetness, the effect, are one and the same
concretely existing water.
In regard to the form, the cause (the rain) thus
falls away in the effect (the wetness);
but so does the determination of the effect
that is nothing without the cause and
there remains only the indifferent wetness.

The cause in the common sense of the causal relationship is finite
insofar as its content is finite (as in the finite substance) and insofar
as cause and effect are represented as two different, self-sufficient
concrete existences - which they are only because one abstracts from
the relationship of causality in their case. Because in [the sphere
of] finitude one does not move beyond the difference between the
determinations of form in their relation, the cause is also alternately
determined as something posited or as effect. The latter then has
another cause in turn and in this way there arises here the
progression from effects to causes ad infinitum. The same holds for
the descending progression in that the effect, in keeping with its
identity with the cause, is itself determined as cause and at the same
time as another cause that has other effects in turn and so on ad
infinitum.

The effect is different from the cause;
the effect is, as such, a being-that-is-posited.
But positedness is equally reflection-in-itself and immediacy, and
the cause's effecting, its positing, is at the same time a presupposing,
insofar as the difference of the effect from the cause is maintained.
There is accordingly another substance at hand,
in regard to which the effect happens.
This [substance] is, as immediate,
not self-relating negativity and active, but passive instead.
But, as substance, it is equally active,
it sublates the presupposed immediacy and the effect posited in it;
it reacts, it sublates the activity of the first substance which, however, is
just as much this sublating of its immediacy or
the effect posited in it, and, with this,
sublates the activity of the other and reacts.
With this, causality has passed over into the relationship of determination,
and thus positing that vacuousness of the moments that is in itself
An effect is posited in the primordiality;
that is to say, the primordiality is sublated.
The action of a cause becomes a reaction, and so forth.

c. Reciprocity

The determinations that have been kept separate in reciprocity are

(a) in themselves the same;
one side like the other is cause, original, active, passive, and so forth.
So, too, presupposing another and having an effect on it,
the immediate primordiality and the positedness
by way of alternation are one and the same.
The cause assumed to be first is,
on account of its immediacy,
passive, a positedness, and an effect.
The difference between the causes, identified as two,
is thus empty and what is at hand is
in itself only one cause that, in its effect sublates itself
as substance just as much as it renders itself
self-sufficient in this effecting.

(b) But this unity is also for itself,
since this whole alternation is the cause's own positing,
and its being is nothing but this positing.
The vacuousness of the differences is not only
in itself or our reflection (see preceding section),
but this reciprocity is itself also the process of sublating
each of the posited determinations in turn,
inverting each into the opposite.

(c) This sheer alternation with itself is, accordingly,
the unveiled or posited necessity.
The bond of necessity as such is
the identity that is still inner and hidden
because it is the identity of
those [things] that count as actual,
but whose self-sufficiency is, nevertheless,
supposed to be precisely the necessity.
The course taken by the substance through
causality and reciprocity is thus merely
the process of positing that the self-sufficiency is
the infinite, negative relation to itself;
negative in the general sense that in it
the differentiating and mediating become an
original condition of actualities that are
self-sufficient vis-a-vis one another:
an infinite relation to itself, since their self-standing
status is precisely nothing other than their identity.

This truth of necessity is thus freedom, and
the truth of substance is the concept:
the self-sufficiency that is the repelling of itself from itself into
different self-sufficient [moments] and, as this repelling, is
identical with itself and, enduring by itself, is
this alternating movement only with itself.

The concept is accordingly the truth of being and essence,
since the shining of reflection within itself is
itself at the same time self-sufficient immediacy
and this being of diverse actuality is immediately
only a shining in itself.

In that the concept has proven itself to be
the truth of being and essence,
both of which have gone back into it
as into its ground, it has developed inversely,
from being as from its ground.

The former side of the progression can be considered
a deepening of being in itself,
the inner [dimension] of which
has been unveiled by this progression;
the latter side can be considered the emergence
of the more perfect from the less perfect.
Philosophy has been reproached for considering
such development from the latter side alone.
The more determinate content that the superficial thoughts
of the less perfect and the more perfect have here is
the difference between being qua immediate unity with itself,
and the concept qua free mediation with itself.
Since being has shown itself to be a moment of the concept,
the concept has demonstrated itself to be the truth of being;
as this, its reflection-in-itself,
and as the sublating of the mediation,
it presupposes the immediate,
a presupposing that is identical with the return-into-itself,
the identity that makes up the freedom and the concept.
If the moment is thus named the imperfect, then,
of course, the concept, the perfect, is this,
to develop itself from the imperfect,
for it is essentially this sublating of its presupposition.
However, at the same time, it is the concept alone that,
qua positing itself makes the presupposition,
as was the outcome in causality in general
and more specifically in reciprocity.

In relation to being and essence, the concept is determined
in such a way that it is the essence that has gone back to being
as simple immediacy, the essence whose shining thereby has actuality and
whose actuality is at the same time the process of freely shining in itself.
In this manner the concept has being as its simple relation
to itself or as the immediacy of its unity in itself,
being is so impoverished a determination that it is
the very least that can be pointed up in the concept.

The transition from necessity to freedom or
from the actual into the concept is the hardest transition,
because the self-sufficient actuality is supposed to be thought
as having its substantiality only in the process of passing over
and in the identity with the self-sufficient actuality other than it.
The concept is also the hardest then,
because it is itself precisely this identity.
The actual substance as such, however,
the cause that, in its being-for-itself,
does not wain to let anything penetrate into it,
is already subject to the necessity or fate
of passing over into positedness,
and this subjection is the hardest by far.
By contrast, thinking the necessity is
rather the dissolving of that hardness;
for it is the process of its coming-together
with itself in an other,
the liberation which is not the flight of abstraction
but instead the liberation of having itself
not as other but of having its own being and positing
in something else actual with which what is actual is
bound together by the power of necessity.
As concretely existing for itself,
this liberation is called 'I',
as developed in its totality 'free spirit',
as feeling 'love', as enjoyment 'blessedness'.
The great intuition of the Spinozistic substance is
only in itself the liberation from finite being-for-itself;
but the concept itself is for itself
the power of necessity and the actual freedom.
