SECTION III

Measure

Abstractly expressed, quality and quantity are in measure united.
Being as such is the immediate equality of determinateness with itself.
This immediacy of determinateness has sublated itself.
Quantity is being that has returned to itself in such a way
that it is a simple self-equality indifferent to determinateness.
But this indifference is only the externality of having
the determinateness not in itself but in an other.
As third, we now have the externality that refers itself to itself;
as self-reference, it is at the same time sublated externality and
carries the difference from itself in it;
a difference which, as externality, is the quantitative moment,
and, as taken back into itself, the qualitative.

Since among the categories of transcendental idealism
modality comes after quantity and quality,
with relation inserted in between,
this is an appropriate place to say something about it.
In transcendental idealism, this category has the meaning that
it is the connection of the subject matter to thought.
As understood by that idealism, thought is as such
essentially external to the thing-in-itself.
Hence, inasmuch as the other categories have
the transcendental determination of
belonging only to consciousness,
but as its objective moment,
so modality, which is the category of
the connection to the subject,
possesses the determination of
reflection in itself in a relative sense,
that is to say, the objectivity
which is granted to the other categories
is lacking in those of modality;
these, according to Kant's words,
do not add in the least to the concept
as a determination of the object
but only express its relation to the faculty of cognition
(Cr. of Pure R., 2nd edn, pp. 99, 266).
The categories which Kant groups under modality
(possibility, actuality, and necessity)
will come up later in their proper place.
Kant did not apply the form of triplicity
(an infinitely important form even though with him
it occurred only as a formal spark of light)
to the genera of his categories (to quantity, quality, etc.),
but only to their species to which
he also gave the name of genera.
He was therefore unable to hit upon
the third to quality and quantity.

With Spinoza, the mode is likewise
the third after substance and attribute;
Spinoza defines it as the affections of substance,
or as that which is in another through which it is also comprehended.
In this way of conceiving it, this third is externality as such;
as has already been mentioned, with Spinoza generally,
the rigidity of substance lacks the turning back into itself.

The remark just made extends to any of the systems of pantheism
which thought has in one way or other produced.
Being, the one, substance, the infinite, essence, is the first;
opposite this abstraction is the second which
can be mustered in an equally abstract form,
as is habitually done as the next step in any purely formalistic thinking,
namely all determinateness generally taken as the mere finite,
the mere accidental, the transitory, the extraneous and unessential, etc.
But the bond connecting this second with the first is too invasive
for the second not to be not equally grasped with the first;
thus with Spinoza the attribute is the whole of substance,
though as comprehended by the understanding,
which is itself a restriction of substance or mode;
and so the mode, the insubstantial as such which can
be grasped only through an other,
constitutes the opposite extreme of substance, the third.
Also Indian pantheism, taken abstractly, has attained
in its monstrous fantasies this refinement
which runs like a moderating thread across its excesses
as its one point of interest namely
that Brahma, the one of abstract thought,
progresses through the shape of Vishnu,
particularly in the form of Krishna, to the third, Shiva.
The determination of this third is that of mode, alteration,
coming-to-be and passing-away;
it is the field of externality in general.
This Indian trinity has tempted a comparison with the Christian,
and one must indeed acknowledge a common element in them.
But it is essential to be aware of the difference that separates them.
It is not just that this difference is infinite
but that the true infinity makes the difference.
The determination of the Indian third principle is that
it is the dispersal of the substantial unity into its opposite,
not its turning back to itself,
a spiritual void rather, not spirit.
In the true trinity, there is not only unity but unification;
the syllogism is brought to a unity which is full of content and actual,
a unity which in its totally concrete determination is spirit.
The principle of the mode and of alteration
does not, of course, exclude unity altogether.
In Spinozism, for instance, precisely the mode is as such untrue
while substance alone is what truly is;
everything is supposed to be reduced to substance,
and this is then a sinking of all content into
an only formal unity void of content.
As for Shiva, it too is again the great whole,
not distinct from Brahma, from Brahma itself, that is,
the distinction and the determinateness just disappear
without being preserved, without being sublated,
and the unity does not become concrete unity,
nor is the diremption reconciled.
The supreme goal of the human being,
relegated as he is to the sphere
of coming-to-be and passing-away,
of modality in general,
is to sink into unconsciousness,
into unity with Brahma, annihilation;
the Buddhist Nirvana, Nibbana, etc., is the same.

Now although the mode is as such abstract externality,
indifference to qualitative as well as quantitative determinations,
and nothing in the essence should depend on the external,
the unessential, it is nevertheless conceded that
in the many all depends on the how;
but this is to concede that the mode itself
essentially belongs to the substance of a thing,
a very indefinite connection but
one which at least implies that the externality of
the mode is not all that abstract an externality after all.
Here the mode has the definite meaning of being measure.
The Spinozistic mode, just like the Indian principle of alteration,
is the measureless.
The Greeks were aware that everything has a measure.
Parmenides himself introduces necessity after abstract being,
as the ancient limit which is imposed on all.
And this awareness, although still vague, is
the beginning of a much higher concept
than is contained in substance
and in the distinction of the mode from it.

Measure in its more developed, more reflected form is necessity.
Fate, Nemesis, ultimately comes down to a determination of measure.
Whatever renders itself beyond the pale, becomes too great, too high,
is brought down to the other extreme of being reduced to nothing,
so that the mean of measure, the medium, is restored.
That the Absolute, God, is the measure of all things, is not
a stronger statement of pantheism than the definition,
“the Absolute, God, is being,” but is infinitely truer.
Measure is indeed an external way of things, a more or less,
but one which is as at the same time reflected into itself,
not merely an indifferent and external determinateness
but one which exists in itself;
thus it is the concrete truth of being.
For this reason the nations have revered in it
the presence of something inviolable and sacred.

Already present in measure is the idea of essence,
namely of being self-identical in
the immediacy of being determined,
so that this immediacy is reduced
through the self-identity to something mediated,
just as the self-identity is equally mediated
only through this externality,
but the mediation is one with itself:
this is reflection, the moments of which indeed are,
but in this being are absolutely nothing but
moments of their negative unity.
In measure, the qualitative element is quantitative;
the determinateness or the difference is indifferent
and therefore a difference which is none;
it is sublated and this quantitativeness,
as an immanent turning back in which it is qualitative,
constitutes the being-in-and-for-itself which is essence.
But measure is essence only implicitly in itself or in its concept;
this concept of measure is not yet posited.
Measure is as such still the existent unity of the
qualitative and the quantitative element;
its moments are an existence, a quality and some quanta of this quality
which, in themselves, are indeed only indivisible,
but do not yet have the meaning of this reflected determination.
In the development of measure, these moments are differentiated
but at the same time referred to each other,
so that the identity which they are in themselves becomes
their connection explicitly, that is, is posited.
The meaning of this development is the realization of measure
in which the latter posits itself in relation to itself
and consequently as moment as well;
through this mediation, measure is determined as sublated;
its immediacy as well as that of its moments disappears;
these moments are reflected and thus measure,
having disclosed what it is according to its concept,
has passed over into essence.

Measure is at first the immediate unity of
the qualitative and the quantitative element, so that it is,

first, a quantum that has qualitative meaning and is as measure.
As so implicitly determined in itself, its further determination is that
the difference of its moments,
of its qualitatively and quantitatively determined being,
is disclosed in it.
These moments further determine themselves into
wholes of measure which as such are self-subsistent,
and, inasmuch as they refer to each other essentially,
measure becomes,

second, a ratio of specific quanta, each an independent measure.
But their self-subsistence also rests essentially
on a quantitative relation and a difference of magnitude,
and so the self-subsistence becomes a transition
of one measure into another.
The result is that measure collapses into the measureless.
But this beyond of measure is the negativity of measure
only in itself; thus,

third, the indifference of the determinations of
measure is thereby posited, and measure
(real measure because of the negativity contained within it)
is posited as an inverse ratio of measures
which, as self-subsistent qualities,
essentially rest on only their quantity
and their negative reference to each other,
and consequently prove to be only moments of
their truly self-subsistent unity.
This unity is the reflection-into-itself of each
and the positing of them; it is essence.

The development of measure,
which we have attempted in what follows,
is among the most difficult of subject matters.
Starting with immediate, external measure,
it would have to proceed, on the one hand,
to the further abstract determination of
the quantitative aspect of natural things
(of a mathematics of nature);
on the other side, it would have to indicate the link
between this determination of measure
and the qualitative aspect of those things,
at least in general, for the detailed demonstration of the
link between the qualitative and the quantitative aspects
as they originate in the concept of a concrete object
belongs to the particular science of the concrete
(examples of which, concerning the law of falling bodies
and the free movement of the heavens,
will be found in the Encyclopedia of the Philosophical Sciences).
We may remark quite in general in this connection
that the different forms in which measure is realized
also belong to different spheres of natural reality.
The complete, abstract indifference of developed measure,
that is, of its laws, can only be found in the sphere of mechanism
where concrete corporeity is only abstract matter itself;
the qualitative differences of this matter are
of an essentially quantitative nature;
space and time are nothing but pure externalities,
and the aggregates of matters, the masses, the intensity of weight,
are determinations which are just as external
and have their proper determinateness in the quantitative element.
On the other hand, in physical things but even more so in the organic,
this quantitative determinateness of abstract materiality is
already disturbed by the multiplicity
and consequently the conflict of qualities.
And the thus ensuing conflict is not just one of qualities as such,
but measure itself is subordinated here to higher relations
and its immanent development is reduced rather to
the simple form of immediate measure.
The limbs of the animal organism have a measure
which, as a simple quantum, stands
in a ratio to the other quanta of the other limbs;
the proportions of the human body are the fixed ratios of such quanta,
and the science of nature still has far to go in
discovering anything about the link that connects
these magnitudes with the organic functions
on which they are entirely dependent.
But the closest example of the reduction of an immanent
measure to a merely externally determined magnitude is motion.
In the heavenly bodies, motion is free motion,
one which is only determined by the concept from which alone,
consequently, its magnitudes equally depend (see above);
but in the organic body this free motion is reduced
to one which is arbitrary or mechanically regular,
that is, to one which is totally abstract and formal.

And in the realm of spirit there is even less of
a characteristic, free development of measure to be found.
For instance, it is obvious that a
republican constitution like the Athenian,
or an aristocratic constitution mixed with democracy,
is possible only in a state of a certain size;
it is also obvious that in civil society
the multitudes of individuals who belong to
the different occupations stand in a certain ratio to each other.
But none of this yields either laws of measures or typical forms of it.
In the spiritual realm as such there are indeed
distinctions of intensity of character,
strength of imagination, sensations, representations, and so on;
but in determining them one cannot go past this indefinite duo of
“strength” and “weakness.”
To see how lame and totally empty ultimately turn out to be
the so-called laws which have been established for
the relation of strength and weakness in sensations,
representations, and so on, one need only look at
the psychologies that busy themselves with just such matters.

CHAPTER 1

Specific quantity

Qualitative quantity is,

first, an immediate, specific quantum; and this quantum,

second, in relating itself to another,
becomes a quantitative specifying,
a sublating of the indifferent quantum.
This measure is to this extent a rule
and contains the two moments of measure as different;
namely, the quantitative determinateness
and the external quantum as existing in themselves.
In this difference, however, the two sides become qualities,
and the rule becomes a relation of the two;
measure presents itself thereby,

third, as a relation of qualities that have one single measure at first;
a measure, however, which further specifies itself in itself
into a difference of measures.

A. THE SPECIFIC QUANTUM

1. Measure is the simple self-reference of quantum,
its own determinateness in itself;
quantum is thus qualitative.

At first, as an immediate measure it is
an immediate quantum and hence some specific quantum;
equally immediate is the quality that belongs to it;
it is some specific quality or other.
Thus quantum, as this no longer indifferent limit
but as self-referring externality, is itself quality
and, although distinguished from it,
it does not extend past it,
just as quality does not extend past quantum.
Quantum is thus the determinateness
that has returned into simple self-equality,
which is at one with determinate existence
just as determinate existence is at one with it.

If a proposition is to be made of the determination just obtained,
it could be expressed thus:
“Whatever is, has a measure.”
Every existence has a magnitude,
and this magnitude belongs to the very nature of a something;
it constitutes its determinate nature and its in-itself.
The something is not indifferent to this magnitude,
as if, were the latter to alter, it would remain the same;
rather the alteration of the magnitude alters its quality.
As measure the quantum has ceased to be a limit which is none;
it is from now on the determination of a thing, so that,
were the latter to exceed or fall short of this quantum,
it would perish.

A measure, in the usual sense of a standard,
is a quantum which is arbitrarily assumed as
the unit determinate in itself as against an external amount.
Of course, such a unit can in fact also be determinate in itself,
like a foot or some such other original measure;
to the extent, however, that it is used as
the measuring standard for other things,
it is with respect to them only an external measure,
not their original measure.
Thus the diameter of the earth or the length of a pendulum
may be taken as a specific quantum on their own account.
But the choice of a fraction of the
earth's diameter
or of the pendulum's length,
and this last under which degree of latitude,
for use as a standard of measure is arbitrary.
And for other things such a standard is
something even more external.
These have further specified
the universal specific quantum in some particular way,
and they have thereby been made into particular things.
It is therefore foolish to speak of a natural standard of things.
Anyway, a universal standard is meant for use
only as an external comparison of things,
and in this superficial sense of universal standard
it is quite a matter of indifference
what is used for the purpose.
It is not meant to be a fundamental measure
in the sense that in it the natural measures
of particular things would be displayed and recognized,
according to a rule, as the specifications of one universal measure,
the measure of the things's universal body.
But without this meaning the sole interest and significance
of an absolute standard of measure is that of something common,
and any such standard is a universal
not in itself but only by convention.

Immediate measure is a simple determination of magnitude as,
for example, the size of organic beings, of their limbs, and so forth.
But any concrete existent has the size required for being what it is,
and for having existence in the first place.
As a quantum, the existent is an indifferent magnitude
open to external determination
and capable of fluctuating increases and decreases.
As a measure, however, it is at the same time
distinct from itself as quantum,
itself as such an indifferent determination,
and is a restriction on that indifferent fluctuation of a limit.
Since in the existence of anything
the quantitative determinateness is thus twofold,
in the sense that quality is tied to it
and yet the quantity can fluctuate without prejudice to quality
so the demise of anything that has a measure occurs
through the alteration of its quantum.
On the one hand, the demise appears unexpected,
inasmuch as there can be alteration
in the quantum without the measure
and the quality being altered;
but, on the other hand, it is made into something quite simple
to grasp by means namely of the concept of gradualness.
It is easy to turn to this category for visualizing
or “explaining” the disappearance of a quality or of a something,
for it gives the impression that one can witness this disappearance
as if before one's eyes:
since the quantum is posited as the external limit
which is by nature alterable,
the alteration (of quantum only) then follows by itself.
But in fact nothing is thereby explained,
for the alteration is at the same time essentially
the transition of one quality into another,
or the more abstract transition of one existence into a non-existence,
and therein lies another determination than just gradualness,
which is only a decrease or increase,
and the one-sided holding fast to magnitude.

2. The ancients had already taken notice of this coincidence,
that an alteration which appears to be only quantitative
suddenly changes into a qualitative one,
and they used popular examples to illustrate
the inconsistencies that arise when such
a coincidence is not understood.
Two such examples go under the familiar names
of “the bald” and “the heap.”
They are elenchi, that is, according to Aristotle's explanation,
two ways in which one is compelled to say
the opposite of what one has previously asserted.
The question was put:
does the plucking of one hair from someone's head
or from a horse's tail produce baldness,
or does a heap cease to be a heap
if one grain is removed?
The expected answer can safely be conceded,
for the removal amounts to a merely quantitative difference,
and an insignificant one at that.
And so one hair is removed, one grain,
and this is repeated with only one hair and one grain
being removed each time the answer is conceded.
At last the qualitative alteration is revealed:
the head or the tail is bald; the heap has vanished.
In conceding the answer, it was not only
the repetition that was each time forgotten,
but also that the individually insignificant quantities
(like the individually insignificant disbursements from a patrimony)
add up, and the sum constitutes the qualitative whole,
so that at the end this whole has vanished:
the head is bald, the purse is empty.

The embarrassment, the contradiction, produced by the result,
is not anything sophistic in the usual sense of the word,
as if the contradiction were a pretense.
The mistake is committed by the assumed interlocutor
(that is, our ordinary consciousness),
and that is of assuming a quantity to be only an indifferent limit,
that is, of taking it in the narrowly defined sense of a quantity.
But this assumption is confounded by the truth to which it is brought,
namely that quantity is a moment of measure
and is linked to quality;
refuted is the one-sided stubborn adherence to the
abstract determinateness of quantum.
Also those elenchi are, therefore,
not anything frivolous or pedantic but basically correct:
they attest to a mind which has an interest
in the phenomena that come with thinking.
Quantum, when it is taken as indifferent limit,
is the side from which an existence is
unsuspectedly attacked and laid low.
It is the cunning of the concept
that it would seize on an existence from this side
where its quality does not seem to come into play;
and it does it so well that the aggrandizement
of a State or of a patrimony, etc.,
which will bring about the misfortune of the State or the owner,
even appears at first to be their good fortune.

3. Measure is in its immediacy an ordinary quality
of a specific magnitude appropriate to it.
Now there is also the distinction between the side
according to which quantum is an indifferent limit
that can fluctuate without the quality altering
and the other side according to which it is
qualitative and specific.
Both sides are the magnitude determinations of
one and the same thing;
but because of the original immediacy of measure,
this distinction too is to be taken as immediate,
and accordingly the two sides each also have
a diverse concrete existence.
The concrete existence of measure,
which is the side of magnitude determinate in itself,
then behaves towards the concrete existence
of the alterable external side by sublating
the indifference of the latter;
this is a specifying of measure.

B. SPECIFYING MEASURE

First, this measure is a rule,
a measure external to the mere quantum.

Second, it is a specific quantity
determining the external quantum.

Third, the two sides, both as qualities
of specific quantitative determinacy,
relate to one another as one measure.

a. The rule

The rule, or the standard which we have just mentioned,
is in the first instance as a magnitude
which is determinate in itself
and is a unit with respect to a quantum
of a particular concrete existence:
this is a quantum with a concrete existence
which is other than the something of the rule,
is measured by the latter, that is,
is determined as the amount of the said unit.
This comparison is an external act,
and the unit itself is an arbitrary magnitude
which can in turn be posited as an amount
(the foot as an amount of inches).
But measure is not only an external rule;
as a specific measure its intrinsic nature is
that it relates to its other which is a quantum.

b. Specifying measure

Measure is a specific determining of the external magnitude, that is,
of the indifferent magnitude which is now posited
in the measuring something by some other concrete existence in general.
The something of the measure is indeed itself a quantum,
but with the difference that it is the qualitative side
determining the merely indifferent, external quantum.
It has intrinsically this side of being-for-other
to which the fluctuation in size belongs.
The immanent measuring is a quality of the something,
and this something is confronted by the same quality in another something;
in the latter, however, the quality is at first relative,
with a measureless quantum in general as against
the something determined as measuring.
Inasmuch as a something has an internal measure,
an alteration of the magnitude of its quality comes to it from outside,
and the something does not take on
the arithmetical aggregate of the alteration.
Its measure reacts against it,
behaves towards the aggregate as an intensive measure
and assimilates it in a way typically its own;
it alters the externally imposed alteration,
makes something else out of this quantum
and demonstrates through this specifying function
that in this externality it is for-itself.
This specifically assimilated aggregate is itself
a quantum which is also dependent on the other, that is,
on the other aggregate which is only external to it.
The specified aggregate is therefore also alterable,
but is not for that reason a quantum as such
but the external quantum as specified in a constant manner.
Measure thus has its determinate existence as a ratio,
and its specificity is in general the exponent of this ratio.

In intensive and extensive quantum,
as we saw when considering these determinations,
it is the same quantum which is present,
once in the form of intensity
and again in the form of extension.
In this difference the underlying quantum
does not suffer any alteration;
the difference is only an external form.
In the specifying measure, on the contrary,
the quantum is taken in one instance in its immediate magnitude,
but through the exponent of the ratio
it is taken in a second instance in another amount.

The exponent which constitutes the element of specificity can appear
at first to be a fixed quantum, as a quotient of the ratio between the
external and the qualitatively determined quantum.
But it would then be nothing more than an external quantum
whereas by the exponent we are to understand here nothing
but the qualitative moment itself that specifies the quantum as such.
The only strictly immanent qualitative determination of
the quantum is (as we saw earlier) that of the exponent,
and it must be such an exponential determination
which now constitutes the ratio and which,
as the internally existent determination,
comes to confront the quantum as externally constituted.
The principle of this quantum is the numerical one
which constitutes its internal determinateness:
the mode of connection of this numerical one is external,
and the alteration which is determined only
through the nature of the immediate quantum as such consists
essentially in the addition of one such numerical one
and then another and so forth.
As the external quantum alters in arithmetical progression in this way,
the specifying reaction of the qualitative nature produces another series
which refers to the first, increases and decreases with it,
not however in a ratio determined by a numerical exponent
but in a ratio which is numerically incommensurable,
in the manner of a power determination.

c. Relation of the two sides as qualities

1. The qualitative side of the quantum, in itself determined,
exists only as a reference to the external quantitative side;
as specifying the latter, it is a sublating of its externality
through which quantum as such is.
This qualitative side thus has a quantum
for the presupposition from which it starts.
But this quantum is also qualitatively distinguished from quality,
and this difference between the two must now be posited in the immediacy
of being in general which still characterizes measure.
The two sides thus stand to each other in a qualitative respect,
each a qualitative existence for itself,
and the one side that was at first only an internally indeterminate
formal quantum is the quantum of a something and of its quality,
and, just as their reciprocal reference is
now determined as measure in general,
so too is the specific magnitude of these qualities.
These qualities stand in relation to each other according
to a determination of measure.
This determination is their exponent,
but they are already implicitly connected
to each other in the being-for-itself of measure:
the quantum is in its double being
external quantum and specific quantum,
so that each of the distinct quantities has
this double determination in it
and is at the same time inextricably interwoven with the other;
it is in this way alone that the qualities are determined.
They are not, therefore, a determinate being in general
existing for each other but are rather posited as indivisible,
and the specific magnitude tied to them is a qualitative unity:
one determination of measure in which they are
implicitly bound together in accordance with their concept.
Measure is thus the immanent quantitative relating
of two qualities to each other.

2. In measure we have the essential determination of variable magnitude,
for measure is sublated quantum;
quantum, therefore, no longer as that
which it ought to be in order to be quantum,
but quantum as quantum and as something other besides.
This other is the qualitative element
and, as we have established, is nothing else
than the relation of powers of the quantum.
In immediate measure, this variability is not yet posited;
it is just one single quantum or other to which a quality is attached.
In the specifying of measure (the preceding determination),
which is an alteration of the merely external quantum
by the qualitative moment,
what is posited is the distinctness of two determinate magnitudes
and hence in general the plurality of measures
in one common, external quantum.
And it is in this differentiation of the quantum from itself
that the latter shows itself for the first time to be a real measure,
for it appears as a being which is one and the same
(e.g. the constant temperature of the medium)
and at the same time of diversified and indeed quantitative existence
(in the various temperatures of the bodies found in the medium).
This differentiation of quantum into the diverse qualities
(the different bodies) yields a further form of measure,
one in which the two sides relate to each other
as qualitatively determined quanta,
and this can be called the realized measure.

Magnitude, simply as such, is alterable, for its determinateness is
a limit which at the same time is none;
the alteration only affects, therefore,
a particular quantum in place of which another is posited.
But the genuine alteration is that of the quantum as such.
Here we have the determination, interesting when so understood,
of the variable magnitude of higher mathematics
in which we neither need to stop short at the formal deter-
mination of alteration or variability in general
nor introduce any other determination except
the simple determination of the concept by which
the other of the quantum is only the qualitative.
The genuine determination, therefore, of real variable magnitude is
that it is qualitative, that is, as we have sufficiently shown,
that it is determined by a ratio of powers.
Posited in this variable magnitude is
that quantum has no value as such but only
as determined in conformity with its other, that is, qualitatively.
The two sides in this relating have,
in keeping with their abstract side as qualitites in general,
some particular meaning or other, for instance, space and time.
Taken at first simply as determinacies of magnitude
in their ratio of measure,
one of them is the amount which increases and decreases
in external arithmetical progression;
the other is an amount specifically determined by the other amount,
which for it is the unit.
If each of these two sides were only a particular quality in general,
there would be no way of distinguishing which of them is
to be taken with respect to their determination of magnitude
as merely externally quantitative
and which as varying in quantitative specification.
If they are related, for instance, as root and square,
it is indifferent which is regarded as increasing or decreasing
in merely external arithmetical progression,
and which on the contrary has
its specific determination in this quantum.

But the two qualities are not indeterminate in their difference,
for they are the moments of measure
and the qualification of the latter ought to rest on them.
The closest determinateness of the qualitites themselves is,
of the one, the extensive, that it is an externality within;
and of the other, the intensive, that it exists in itself
or is the negative as against the other.
Accordingly, amount is the quantitative moment
that pertains to the former,
and unit the one that pertains to the latter;
in simple direct ratio, the former is to be taken
as the dividend and the latter as the divisor;
in specifying ratio, the former as the power
or the becoming-other and the latter as the root.
Inasmuch as we still count here, that is,
we reflect on the external quantum
(which is thus the totally accidental determinateness of
magnitude which we call empirical),
and accordingly also equally take the alteration
as advancing in external arithmetical progression,
then this falls on the side of the unit, the intensive quality;
the external extensive side, by contrast,
is to be represented as altering in the specified series.
But the direct ratio (like velocity as such, st ) is reduced here
to a formal determination which has no concrete existence
but belongs rather only to the abstraction of reflection;
and even though in the ratio of root and square (as in s = at 2 ),
the root is to be taken as an empirical quantum
varying in an arithmetical progression,
the other side as specified instead,
the higher realization of
the qualification of the quantitative moment,
one which would be more in keeping with the concept, is this:
that both sides are related in higher determinations of powers
(as in s^3 = at^2 ).

C. THE BEING-FOR-ITSELF IN MEASURE

1. In the form of specified measure just considered,
the quantitative moment of each side is qualitatively determined
(both in the ratio of powers);
they are thus moments of
one measure-determinateness of qualitative nature.
Here, however, the qualities are
still posited immediately, only as diverse;
they are not related in the manner
of their quantitative determinacies,
that is, that outside their relation
they would have neither meaning nor existence,
as is the case for the quantitative determinacies
as a ratio of powers.
The qualitative moment thus disguises itself as specifying,
not itself, but the determinateness of magnitude.
Only within the latter is it posited;
for itself it is instead immediate quality as such
which, besides the fact that it
posits the magnitude as non-indifferent,
and besides its connection with its other,
still has existence subsisting on its own.
Thus space and time, outside that specification
which their quantitative determinateness obtains
in the motion of falling bodies
or in the absolutely free motion,
both have the value of space in general and time in general,
space subsisting on its own outside and without the duration of time,
and time flowing on its own independently of space.

This immediacy of the qualitative moment as against
its specific measure-relation is, however,
equally bound up with a quantitative immediacy
and with the indifference to this same measure-relation
of the quantitative element in it;
the immediate quality also has only an immediate quantum.
For that reason, the specific measure also has
a side of an at first external alteration
which advances in merely arithmetical progression undisturbed by it
and in which falls the external and hence merely empirical
determination of magnitude.
Quality and quantum, even as they extend outside measure,
are at the same time connected with it;
the immediacy is a moment of their belonging to measure.
Thus the immediate qualities also belong to measure,
are likewise connected and stand in a ratio
which, outside the specified one of power determination,
is itself only a direct ratio and an immediate measure.
This conclusion and its consequences must now be indicated.

2. Although the immediately determined quantum is as
a moment of measure otherwise implicitly grounded in a conceptual nexus,
in connection with specific measure it is, as such, externally given.
But the immediacy which is thereby posited is
the negation of the determination of qualitative measure;
the same immediacy was shown just now
on the sides of the determination of measure
which appeared for that reason to be self-subsisting qualities.
Such a negation and the return to immediate
quantitative determinateness are present
in the qualitatively determined relation
because a relation of distinct terms
entails as such the connection of such terms
in one determinateness,
and here, in the quantitative sphere,
in distinction from the determination of the relation,
this determinateness is a quantum.
As the negation of the distinct, qualitatively determined sides,
this exponent is a being-for-itself,
an absolutely determined being,
but is so only in itself;
as existence, it is a simple immediate quantum,
the quotient or exponent of a ratio between the sides of measure
which is taken as direct;
as such, however, it appears in the quantitative side
of measure as an empirical unit.
In the motion of falling bodies,
the spaces traversed are proportional
to the square of the elapsed time, s = at^2.
This is a specifically determinate ratio,
one between space and time;
the other, the direct ratio, would pertain to space and time
as mutually indifferent qualities;
it is supposed to be the ratio of the space traversed
to the first moment of time.
The same coefficient, a, remains in all the succeeding time-points;
the unit of the amount, determined for its part by the specifying measure,
being an ordinary quantum.
This unit counts at the same time as
the exponent of that direct ratio
which pertains to the merely imagined, the bad,
that is, the reflectively formal velocity
which is not specifically determined by the concept.
Such a velocity does not have concrete existence here,
no more than does the one previously mentioned
which is supposed to pertain to a falling body
at the end of a moment of time.
That velocity is ascribed to
the first temporal moment in the fall,
but this so-called temporal moment is itself
only an assumed unit which has no existence
as such an atomic point.
The beginning of the motion
(its alleged smallness would make no difference)
is straight away a magnitude
and one which is specified by the law of falling bodies.
The said empirical quantum is attributed to the force of gravity,
and this force thus has itself no connection with the
specification at hand
(the determinateness of powers),
that is, with the determinateness characteristic of measure.
The immediate moment, that in the motion of falling bodies
the amount of some fifteen spatial units
which are assumed as feet enter into one unit of time
(call it a second, the so-called first unit),
is an immediate measure, just like the size of human limbs,
the distances and diameters of planets, etc.
The determination of such measures falls elsewhere
than within the qualitative determination of measure,
here of the law itself of falling bodies.
On what such numbers depend, these merely immediate
and therefore empirical appearances of a measure;
on this the concrete sciences have yet to give us any information.
Here we are only concerned with this conceptual determination,
namely that in any determination of measure
the empirical coefficient constitutes the being-for-itself,
but only the moment of the being-for-itself,
in so far as it is in principle and therefore as immediate.
The other moment is the development of this being-for-itself,
the specific measure-determinateness of the sides.
According to this second moment,
in the ratio expressing the motion of falling bodies
(this motion that is still half conditioned and half free)
gravity is to be regarded as a force of nature.
Its ratio is thus determined by the nature of time and space,
and the said specification, the ratio of powers, therefore falls in it;
the other moment mentioned above, the simple direct ratio,
expresses only a mechanical relation of time and space,
a reflectively formal velocity externally produced
and externally determined.

3. Measure has now taken on the determination
of being a specified quantitative relation,
one which, as qualitative, has within it the usual external quantum;
but this quantum is not a quantum in general
but essentially the determining moment
of the relation as such;
it is thus an exponent and,
because of the immediacy now of its determinateness,
an invariable exponent, consequently an exponent
of the already mentioned direct ratio
of the same qualities whose reciprocal quantitative relation is
at the same time specifically determined by the ratio.
In the example of measure that we have used,
that of the motion of falling bodies,
this direct ratio is, as it were,
anticipated and assumed as given,
but, as remarked, in this motion it still
does not have concrete existence.
However, that matter is now realized,
in that its sides are both measures,
one distinguished as immediate and external
and the other as internally specified,
while measure itself is the unity of the two
this constitutes a further determination.
As the unity of its two sides,
measure contains the relation in which the magnitudes,
by virtue of the nature of the qualities,
are posited as determined and non-indifferent;
its determinateness, therefore, is entirely immanent and self-subsisting,
and has at the same time collapsed into the being-for-itself
of an immediate quantum, the exponent of a direct ratio.
In this the self-determination of measure is negated,
for in this immediate quantum,
its other, measure has a final self-existent determinateness;
conversely, the immediate measure
which ought to be internally qualitative
assumes truly qualitative determinateness only in that other.
This negative unity is real being-for-itself,
the category of a something which is the unity of qualities
in a relation of measure, a complete self-subsistence.
The two resulting diverse relations also
immediately yield a twofold existence,
or, more precisely, that self-subsisting whole,
as a being that exists for itself, is as such in itself
a repulsion into distinct self-subsisting
somethings whose qualitative nature
and subsistence (materiality) lies
in their determinateness of measure.

CHAPTER 2

Real measure

Measure is now determined as a connection of measures
that make up the quality of distinct self-subsisting somethings,
or, in more common language, things.
The relations of measures just considered belong
to abstract qualities like space and time;
further examples of these now to be considered
are specific gravity and then chemical properties,
that is, determinations of concrete material existence.
Space and time are also moments of these measures,
but are now subordinated to other determinations
and no longer behave relative to one another
only according to their own conceptual determination.
In the case of sound, for instance,
the time within which a certain number of vibrations occur,
the spatial width and thickness of the sounding body,
are moments of its determination.
But the magnitudes of such idealized moments are externally determined;
they no longer assume the form of a ratio of powers
but relate in the usual direct way,
and harmony is reduced to the strictly
external simplicity of numbers in relations
which are most easy to grasp;
they therefore afford a satisfaction
which is the exclusive reserve of the senses,
for there is nothing there of representation,
imagery, thought, or the like,
that would satisfy spirit.
Since the sides which now constitute
the relation of measure are themselves measures,
but at the same time real somethings, their measures are,
in the first instance, immediate measures,
and the relations in them direct relations.
We now have to examine the further determination of
the relation of such relations.
Measure, now real measure, is as follows.

First, it is the independent measure of a type of body,
a measure which relates to other measures
and, in thus relating to them, specifies them as
well as their self-subsistent materiality.
This specification, as an external
connecting reference to many others in general,
produces other relations, and consequently other measures;
the specific self-subsistence, for its part,
does not remain fixed in one direct relation
but passes over into specific determinacies,
and this is a series of measures.

Second, the direct relations that thus result are in themselves
determinate and exclusive measures (elective affinities).
But because they are at the same time only
quantitatively different from one another,
what we have  is a progression of relations
which is in part merely external,
but is also interrupted by qualitative relations,
forming a nodal line of specifically self-subsisting things.

Third, what emerges in this progression for measure, however,
is the measureless:
the measureless in general and more specifically
the infinitude of measure in which the mutually exclusive
forms of self-subsistence are one with each other,
and anything self-subsistent comes to stand
in negative reference to itself.

CHAPTER 3

The becoming of essence

A. ABSOLUTE INDIFFERENCE

Being is abstract indifference,
and when this trait is to be thought by itself as being,
the abstract expression “indifferentness” has been used;
in which there is not supposed to be as yet
any kind of determinateness.
Pure quantity is this indifference in the sense of
being open to any determinations,
provided that these are external to it
and that quantity itself does not have
any link with them originating in it.
The indifference which can be called absolute,
however, is one which, through the negation of
every determinateness of being, of quality and quantity
and of their at first immediate unity, that is, of measure,
mediates itself with itself to form a simple unity.
Determinateness is in it still only a state, that is,
something qualitative and external
which has the indifference as a substrate.

But that which has thus been determined
as qualitative and external
is only a vanishing something;
as thus external with respect to being,
the qualitative sphere is the opposite of itself
and, as such, only the sublating of itself.
In this way, determinateness is still
only posited in the substrate
as an empty differentiation.
But it is precisely this empty differentiation
which is the indifference itself as result.
And this indifference is indeed concrete,
in the sense that it is self-mediated
through the negation of all the determinations of being.
As such a mediation, it contains negation and relation,
and what was called “state” is a differentiation
which is immanent to it and self-referring.
It is precisely this externality and its vanishing
which make the unity of being into an indifference:
consequently, they are inside this indifference,
which thereby ceases to be only a substrate
and, within, only abstract.

B. INDIFFERENCE AS INVERSE RATIO OF ITS FACTORS

We now have to see how this determination
of indifference is posited in the indifference itself
and the latter is posited, therefore, as existing for itself.

1. The reduction of at first
independently accepted measure-relations
establishes their one substrate;
this substrate is their continuing into one another
and is, therefore, the one indivisible independent measure
which is wholly present in its differentiation.
Present for this differentiation are
the two determinations contained
in the measure, quality and quantity,
and everything depends on how these two are posited in it.
But this is in turn determined by the fact that
the substrate is at first posited as result
and, though in itself mediation,
this mediation is not yet posited as such in it;
for this reason, it is in the first instance substrate
and, with respect to determinateness, indifference.

Consequently, the difference present in it is
at first essentially one which is only quantitative and external;
there simply are two different quanta of one and the same substrate
which would thus be their sum, itself posited as a quantum.
But the indifference is this fixed measure,
the implicitly existent absolute limit which,
as connected to those differences,
would not itself be in itself a quantum,
and would not in any way enter
into opposition with others,
whether as sum or also as exponent,
be those others sums or indifferences.
It is only the abstract determinateness
which falls into the indifference;
the two quanta, in order that
they may be posited in it as moments,
are alterable, indifferent, greater or smaller
relative to one another.
However, inasmuch as they are restricted
by the fixed limit of their sum,
they are at the same time related to
each other not externally, but negatively,
and this is now the qualitative determination
in which they stand to each other.
Accordingly, they stand in inverse ratio to each other.
This relation differs from the earlier formal inverted ratio
inasmuch as the limit is here a real substrate,
and each of the two sides is posited as
having to be in itself the whole.

According to the qualitative determinacy just stated,
the difference is present, further, in the form of two qualities,
each of which is sublated by the other
and yet, since the two are held together in the one unity
which they constitute, is inseparable from it.
The substrate itself, as the indifference,
is in itself likewise the unity of the two qualities;
consequently, each of the sides of the relation
equally contains both sides within itself
and is distinguished from the other
by a more of one quality
and a less of the other, or conversely.
The one quality, through its quantum,
only predominates on the one side,
as does the other quality on the other side.

Thus each side is in it an inverted relation
which, as formal, recurs in the two distinguished sides.
The sides themselves thus continue into each other
also according to their qualitative determinations;
each of the qualities relates itself in the other to itself
and is present in each of the two sides,
only in a different quantum.
Their quantitative difference is that indifference
in accordance with which they continue into each other,
and this continuation is the self-sameness of the qualities
in each of the two unities.
The sides, however, each containing
the whole of the determinations
and consequently the indifference itself,
are thus at the same time posited as
self-subsistent vis-a-vis each other.

2. As this indifference,
being is now the determinateness of measure
no longer in its immediacy
but in the developed manner just indicated;
it is indifference because it is in itself
the whole of the determinations of being
now resolved into this unity;
and it is existence as well,
as a totality of the posited realization,
in which the moments themselves are the totality of
the indifference existing in itself,
sustained by the latter as their unity.
But because the unity is held fast only as indifference
and consequently only implicitly in itself,
and the moments are not yet determined
as existing for themselves,
that is, are not yet determined as
sublating themselves into unity internally
and through each other,
what is here present is therefore simply
the indifference of the unity itself
towards itself as a developed determinateness.

This thus indivisible independent measure is
now to be more closely examined.
It is immanent in all its determinations
and in them it remains in unity with itself
and undisturbed by them.

But, (a) since the determinacies
sublated in it implicitly remain the totality,
they emerge in it groundlessly.
The implicit being of indifference
and its existence are thus unconnected;
the determinacies show up in the indifference in an immediate manner
and the indifference is in each of them entirely the same.
The difference between them is
thus posited at first as sublated, hence as quantitative;
for this reason, therefore, not as a self-repelling;
and the indifference not as self-determining,
but as having and becoming the determinate being
that it has only externally.

(b) The two moments are in inverse quantitative relation;
a fluctuating on the scale of magnitude
which is not however determined by the indifference,
which is precisely the indifference of the fluctuation,
but only externally.
For the determining appeal is made to an other
which lies outside the indifference.
The absolute, as indifference, has in this respect
the second defect of quantitative form,
namely that the determinateness of the difference is
not determined by the absolute itself,
just as it has the first defect in that
the differences emerge in it only in general, that is,
the positing of them is something immediate,
not a self-mediation.

(c) The quantitative determinateness of the moments which are now
sides of the relation constitutes the mode of their subsistence;
their existence is by virtue of this indifference
withdrawn from the transitoriness of quality.
But they do have a subsistence of their own in themselves,
one that differs from this quantitative existence,
for they are in themselves the indifference itself,
each the unity itself of the two qualities
into which the qualitative moment splits itself.
The difference of the two moments is restricted by
the fact that the one quality is posited
on the one side with a more
and in the other with a less,
and the other is posited
in inverse order accordingly.
Each side is thus in it the totality of the indifference.
Each of the two qualities taken singly for itself
likewise remains the same sum which the indifference is;
it continues from one side into the other
without being restricted by the quantitative limit
which is thereby posited in it.
At this, the determinations come into immediate opposition,
an opposition which develops into contradiction,
as we must now see.

3. Namely, each quality enters inside
each side in connection with the other,
and it does so in such a manner that, as has been determined,
this connection also is supposed to be only a quantitative difference.
If the two qualities are both self-subsistent
(something like sensible materials independent of each other)
then the whole determinateness of indifference falls apart;
their unity and totality would be empty names.
But they are at the same time determined as
comprised into one unity, as inseparable,
each having meaning and reality only in this one
qualitative connecting reference to the other.
But now, because their quantitativeness is simply
and solely of this qualitative nature,
 each reaches only as far as the other.
If they are assumed to differ as quanta,
then the one would reach beyond the other
and would have in this more an indifferent existence
which the other would not have.
As qualitatively connected, however,
each is only in so far as the other is.
The result is that they are in equilibrium,
so that to the extent that one increases or decreases,
the other likewise increases or decreases
and would do so in the same proportion.

On the basis, therefore, of their qualitative connection,
there is no question of a quantitative difference
or of a more of the one quality.
The more by which the one of the two connected moments
would exceed the other would be only an unstable determination,
or would only be the other itself again;
but, in this equality of the two, neither would then be there,
for their existence would have to rest on the inequality of their quantum.
Each of these supposed factors vanishes,
whether the one factor is assumed
to exceed the other or to be equal to it.
From the standpoint of quantitative representation,
the vanishing appears as a disturbance of the equilibrium,
one factor becoming greater than the other;
the sublation of the quality of the other
and its instability are thus posited;
the first factor becomes the predominant one
as the other diminishes with accelerated velocity
and is overcome by it;
this in turn constitutes itself as the one self-subsistent factor;
with this, however, there are no longer two specific moments
as factors but only the one whole.

This unity thus posited as the totality
of the process of determining,
itself determined in this process as indifference,
is a contradiction all around.
It must therefore be posited
as this self-sublating contradiction,
and be determined as subsistence existing for itself,
one which no longer has a merely indifferent unity
for result but a unity immanently negative and absolute.
This is essence.

C. TRANSITION INTO ESSENCE

Absolute indifference is the final determination of being
before the latter becomes essence;
but it does not attain essence.
It shows that it still belongs to the sphere of being
because it is still determined as indifferent,
and therefore difference is external to it, quantitative.
This externality is its existence, by which
it finds itself at the same time in the opposition of
being determined over against it as existing in itself,
not as being thought as the absolute that exists for itself.
Or again, it is external reflection
which insists that specific determinations,
whether in themselves or in the absolute,
are one and the same;
that their difference is only an indifferent one,
not a difference in itself.
What is still missing here is that
this reflection should sublate itself,
that it would cease to be the external
reflection of thought, of a subjective consciousness,
but that it would be rather the very determination of
the difference of that unity;
a unity which would then prove itself to be the absolute negativity,
the unity's indifference towards itself,
towards its own indifference no less than towards otherness.

But this self-sublation of the determination of
indifference has already manifested itself;
in the progressive positing of its being
it has shown itself on all sides to be contradiction.
Indifference is in itself the totality
in which all the determinations of being are sublated and contained;
thus it is the substrate,
but at first only in the one-sided determination of being-in-itself,
and consequently the differences,
the quantitative difference and the inverseratio of factors,
are present in it as external.
As thus the contradiction of itself and its determinateness,
of its implicitly existent determination
and of its posited determinateness,
it is the negative totality whose determinacies
have internally sublated themselves,
consequently, have also sublated the one-sidedness
of their substrate, their in-itselfness.
Indifference, now posited as what it in fact is,
is simple and infinitely negative self-reference,
the incompatibility of itself with itself,
the repelling of itself from itself.
Determining and being determined are
not a transition,
nor an external alteration,
nor again an emergence of determinations in it,
but its own referring to itself
which is the negativity of itself,
of its in-itselfness.

But as so repelled,
the determinations are not self-possessed;
do not emerge as self-subsistent or external
but are rather as moments:
first, as belonging to the unity
whose existence is still only implicit,
they are not let go by it but are rather borne by it
as their substrate and are filled by it alone;
and, second, as determinations immanent
to the unity as it exists for itself,
they are only through their repulsion from themselves.
Instead of some existent or other,
as they are in the whole sphere of being,
they now are simply and solely as posited,
with the sole determination and significance of
being referred to their unity
and hence each to the other and to negation,
marked by this their relativity.

Being in general and the being or immediacy
of the different determinacies have thereby vanished
just as much as the in-itselfness,
and the unity is being,
immediately presupposed totality,
so that it is only this simple self-reference,
mediated by the sublation of this presupposition,
and this pre-supposedness, the immediate being,
is itself only a moment of its repelling:
the original self-subsistence and self-identity are only
as the resulting infinite self-rejoining.
And so is being determined as essence:
being which, through the sublation of being,
is simple being with itself.
