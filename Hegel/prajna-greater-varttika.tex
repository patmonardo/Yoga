BOOK ONE

The Doctrine of Being

WITH WHAT MUST THE BEGINNING OF SCIENCE BE MADE?

It is only in recent times that
there has been a new awareness of
the difficulty of finding a beginning in philosophy,
and the reason for this difficulty,
and so also the possibility of resolving it,
have been discussed in a variety of ways.
The beginning of philosophy must be either
something mediated or something immediate,
and it is easy to show that it can be
neither the one nor the other;
so either way of beginning runs into contradiction.

The principle of a philosophy also
expresses a beginning, of course,
but not so much a subjective as an objective one,
the beginning of all things.
The principle is a somehow determinate content:
“water,” “the one,” “nous,” “idea,” or “substance,” “monad,” etc.
or, if it designates the nature of cognition
and is therefore meant simply as a criterion
rather than an objective determination,
as “thinking,” “intuition,” “sensation,” “I,” even “subjectivity,”
then here too the interest still lies in the content determination.
The beginning as such, on the other hand,
as something subjective in the sense that
it is an accidental way of introducing the exposition,
is left unconsidered, a matter of indifference,
and consequently also the need to ask
with what a beginning should be made remains
of no importance in face of the need for the principle
in which alone the interest of the fact seems to lie,
the interest as to what is the truth,
the absolute ground of everything.

But the modern perplexity about a beginning
proceeds from a further need which escapes
those who are either busy demonstrating
their principle dogmatically or skeptically looking for
a subjective criterion against dogmatic philosophizing,
and is outright denied by those who begin,
like a shot from a pistol,
from their inner revelation,
from faith, intellectual intuition, etc.
and who would be exempt from method and logic.
If earlier abstract thought is at first interested only
in the principle as content,
but is driven as philosophical culture advances
to the other side to pay attention to
the conduct of the cognitive process,
then the subjective activity has also been grasped
as an essential moment of objective truth,
and with this there comes the need to unite
the method with the content,
the form with the principle.
Thus the principle ought to be
also the beginning,
and that which has priority for thinking
ought to be also the first
in the process of thinking.

Here we only have to consider
how the logical beginning appears.
The two sides from which it can be
taken have already been named,
namely either by way of mediation as result,
or immediately as beginning proper.
This is not the place to discuss the question
apparently so important to present-day culture,
whether the knowledge of truth is
an immediate awareness that begins absolutely, a faith,
or rather a mediated knowledge.
In so far as the issue allows passing treatment,
this has already been done elsewhere
(in my Encyclopedia of the Philosophical Sciences, 3rd edn,
in the Prefatory Concept, §§21ff.).
Here we may quote from it only this,
that there is nothing in heaven or nature or spirit or anywhere else
that does not contain just as much immediacy as mediation,
so that both these determinations prove
to be unseparated and inseparable
and the opposition between them nothing real.
As for a scientific discussion,
a case in point is every logical proposition
in which we find the determinations of immediacy and mediacy
and where there is also entailed, therefore,
a discussion of their opposition and their truth.
This opposition, when connected
to thinking, to knowledge, to cognition,
assumes the more concrete shape
of immediate or mediated knowledge,
and it is then up to the science of logic
to consider the nature of cognition in general,
while the more concrete forms of
the same cognition fall within the scope of
the science of spirit and the phenomenology of spirit.
But to want to clarify the nature of cognition prior to science
is to demand that it should be discussed outside science,
and outside science this cannot be done,
at least not in the scientific manner
which alone is the issue here.

A beginning is logical in that it is to be made in
the element of a free, self-contained thought, in pure knowledge;
it is thereby mediated, for pure knowledge is
the ultimate and absolute truth of consciousness.
We said in the Introduction that the Phenomenology of Spirit is
the science of consciousness, its exposition;
that consciousness has the concept of science,
that is, pure knowledge, for its result.
To this extent, logic has for its presupposition
the science of spirit in its appearance,
a science which contains the necessity,
and therefore demonstrates the truth,
of the stand-point which is pure knowledge
and of its mediation.
In this science of spirit in its appearance
the beginning is made from empirical, sensuous consciousness,
and it is this consciousness which is
immediate knowledge in the strict sense;
there, in this science, is where its nature is discussed.
Any other consciousness, such as faith in divine truths,
inner experience, knowledge through inner revelation, etc.,
proves upon cursory reflection to be very ill-suited
as an instance of immediate knowledge.
In the said treatise, immediate consciousness is also
that which in the science comes first and immediately
and is therefore a presupposition;
but in logic the presupposition is
what has proved itself to be the result
of that preceding consideration,
namely the idea as pure knowledge.
Logic is the pure science, that is,
pure knowledge in the full compass of its development.
But in that result the idea has the determination of
a certainty that has become truth;
it is a certainty which, on the one hand, no longer stands
over and against a subject matter confronting it externally
but has interiorized it, is knowingly aware
that the subject matter is itself;
and, on the other hand, has relinquished
any knowledge of itself that would oppose it to objectivity
and would reduce the latter to a nothing;
it has externalized this subjectivity
and is at one with its externalization.

Now starting with this determination of pure knowledge,
all that we have to do to ensure that the beginning will remain
immanent to the science of this knowledge is to consider,
or rather, setting aside every reflection,
simply to take up, what is there before us.
Pure knowledge, thus withdrawn into this unity,
has sublated every reference to an other and to mediation;
it is without distinctions and as thus distinctionless
it ceases to be knowledge;
what we have before us is only simple immediacy.

Simple immediacy is itself an expression of reflection;
it refers to the distinction from what is mediated.
The true expression of this simple immediacy is therefore pure being.
Just as pure knowledge should mean nothing but knowledge as such,
so also pure being should mean nothing but being in general;
being, and nothing else, without further determination and filling.

Being is what makes the beginning here;
it is presented indeed as originating through mediation,
but a mediation which at the same time sublates itself,
and the presupposition is of a pure knowledge
which is the result of finite knowledge, of consciousness.
But if no presupposition is to be made,
if the beginning is itself to be taken immediately,
then the only determination of this beginning is
that it is to be the beginning of logic, of thought as such.
There is only present the resolve,
which can also be viewed as arbitrary,
of considering thinking as such.
The beginning must then be absolute
or, what means the same here, must be an abstract beginning;
and so there is nothing that it may presuppose,
must not be mediated by anything or have a ground,
ought to be rather itself the ground of the entire science.
It must therefore be simply an immediacy,
or rather only immediacy itself.
Just as it cannot have any determination with respect to an other,
so too it cannot have any within;
it cannot have any content, for any content would entail distinction
and the reference of distinct moments to each other,
and hence a mediation.
The beginning is therefore pure being.

After this simple exposition of what alone first belongs to
this simplest of all simples, the logical beginning,
we may add the following further reflections which should not serve,
however, as elucidation and confirmation of the exposition
[this is complete by itself]
but are rather occasioned by notions and reflections
which may come our way beforehand and yet,
like all other prejudices that antedate the science of logic,
must be disposed of within the science itself
and are therefore to be patiently deferred until then.

The insight that absolute truth must be a result,
and conversely, that a result presupposes a first truth
which, because it is first, objectively considered is
not necessary and from the subjective side is not known;
this insight has recently given rise to the thought
that philosophy can begin only with something
which is hypothetically and problematically true,
and that at first, therefore, philosophizing can be only a quest.
This is a view that Reinhold has repeatedly urged
in the later stages of his philosophizing,
and which must be given credit for being
motivated by a genuine interest in
the speculative nature of philosophical beginning.
A critical examination of this view will also be
an occasion for introducing a preliminary understanding of
what progression in logic generally means,
for the view has direct implications for the nature of this advance.
Indeed, as portrayed by it, progression in philosophy would be
rather a retrogression and a grounding,
only by virtue of which it then follows as result that
that, with which the beginning was made,
was not just an arbitrary assumption
but was in fact the truth,
and the first truth at that.

It must be admitted that it is an essential consideration
(one which will be found elaborated again within the logic itself)
that progression is a retreat to the ground,
to the origin and the truth on which
that with which the beginning was made,
and from which it is in fact produced, depends.
Thus consciousness, on its forward path from
the immediacy with which it began,
is led back to the absolute knowledge
which is its innermost truth.
This truth, the ground, is then also
that from which the original first proceeds,
the same first which at the beginning
came on the scene as something immediate.
It is most of all in this way that absolute spirit
(which is revealed as the concrete and supreme truth of all being)
comes to be known, as at the end of the development
it freely externalizes itself,
letting itself go into the shape of an immediate being,
resolving itself into the creation of a world
which contains all that fell within the development
preceding that result and which,
through this reversal of position with its beginning,
is converted into something dependent
on the result as principle.
Essential to science is not so much that
a pure immediacy should be the beginning,
but that the whole of science is in itself a circle
in which the first becomes also the last,
and the last also the first.

Conversely, it follows that it is
just as necessary to consider as result
that into which the movement returns as to its ground.
In this respect, the first is just as much the ground,
and the last a derivative;
since the movement makes its start from the first
and by correct inferences arrives at the last
as the ground, this last is result.
Further, the advance from that which constitutes the beginning is
to be considered only as one more determination of the same advance,
so that this beginning remains as the underlying ground of
all that follows without vanishing from it.
The advance does not consist in the derivation of an other,
or in the transition to a truly other:
inasmuch as there is a transition,it is equally sublated again.
Thus the beginning of philosophy is the ever present
and self-preserving foundation of all subsequent developments,
remaining everywhere immanent in its further determinations.

In this advance the beginning thus loses the one-sidedness
that it has when determined simply as something immediate and abstract;
it becomes mediated, and the line of scientific forward movement
consequently turns into a circle.
It also follows that what constitutes the beginning,
because it is something still undeveloped and empty of content,
is not yet truly known at that beginning,
and that only science, and science fully developed,
is the completed cognition of it,
replete with content and finally truly grounded.

But for this reason,
because it is as absolute ground
that the result finally emerges,
the progression of this cognition is not anything provisory,
still problematic and hypothetical,
but must be determined through the nature
of the matter at issue and of the content itself.
Nor is the said beginning an
arbitrary and only temporary assumption,
or something which seems to be an
arbitrary and tentative presupposition
but of which it is subsequently shown
that to make it the starting point
was indeed the right thing to do;
this is not as when we are instructed
to make certain constructions in order
to aid the proof of a geometrical theorem,
and only in retrospect, in the course of the proof,
does it become apparent that we did well to draw
precisely these lines and then, in the proof itself,
to begin by comparing them or the enclosed angles,
though the line-drawing or the comparing
themselves escape conceptual comprehension.

So we have just given, right within science itself,
the reason why in pure science
the beginning is made with pure being.
This pure being is the unity
into which pure knowledge returns,
or if this knowledge, as form,
is itself still to be kept
distinct from its unity,
then pure being is also its content.
It is in this respect that this pure being,
this absolute immediate, is just as absolutely mediated.
However, just because it is here as the beginning,
it is just as essential that it should be taken
in the one-sidedness of being purely immediate.
If it were not this pure indeterminacy,
if it were determined,
it would be taken as something mediated,
would already be carried further than itself:
a determinate something has the character of an other
with respect to a first.
It thus lies in the nature of a beginning itself
that it should be being and nothing else.
There is no need, therefore,
of other preparations to enter philosophy,
no need of further reflections or access points.

Nor can we derive a more specific determination
or a more positive content
for the beginning of philosophy
from the fact that it is such a beginning.
For here, at the beginning, where the fact
itself is not yet at hand,
philosophy is an empty word,
a received and yet unjustified notion.
Pure knowledge yields only this negative determination,
namely that the beginning ought to be abstract.
If pure being is taken as the content of pure knowledge,
then the latter must step back from its content,
allowing it free play and without determining it further.
Or again, inasmuch as pure being is to be considered
as the unity into which knowledge has collapsed
when at the highest point of union with its objectification,
knowledge has then disappeared into this unity,
leaving behind no distinction from it
and hence no determination for it.
Nor is there anything else present, any content whatever,
that could be used to make a more determinate beginning with it.

But, it may be said, the determination of being
assumed so far as the beginning can also be let go,
so that the only requirement would be
that a pure beginning should be made.
Nothing would then be at hand
except the beginning itself,
and we must see what this would be.
This position could be suggested also for the benefit
of those who are either not comfortable,
for whatever reason, with beginning with being
and even less with the transition into nothing
that follows from being,
or who simply do not know how else to make a beginning
in a science except by presupposing a representation
which is subsequently analyzed,
the result of the analysis then yielding
the first determinate concept in the science.
If we also want to test this strategy,
we must relinquish every particular object
that we may intend, since the beginning,
as the beginning of thought,
is meant to be entirely abstract,
entirely general, all form with no content;
we must have nothing, therefore,
except the representation of a mere beginning as such.
We have, therefore, only to see
what there is in this representation.

As yet there is nothing, and something is supposed to become.
The beginning is not pure nothing but a nothing, rather,
from which something is to proceed;
also being, therefore, is already contained in the beginning.
Therefore, the beginning contains both, being and nothing;
it is the unity of being and nothing,
or is non-being which is at the same time being,
and being which is at the same time non-being.

Further, being and nothing are present
in the beginning as distinguished;
for the beginning points to something other;
it is a non-being which refers to an other;
that which begins, as yet is not;
it only reaches out to being.
The being contained in the beginning is such, therefore,
that it distances itself from non-being
or sublates it as something which is opposed to it.
But further, that which begins already is,
but is also just as much not yet.
The opposites, being and non-being, are
therefore in immediate union in it;
or the beginning is their undifferentiated unity.

An analysis of the beginning would thus yield
the concept of the unity of being and non-being;
or, in a more reflected form, the concept of the unity
of differentiated and undifferentiated being;
or of the identity of identity and non-identity.
This concept could be regarded as the first,
purest, that is, most abstract, definition of the absolute;
as it would indeed be if the issue were just
the form of definitions and the name of the absolute.
In this sense, just as such an abstract concept
would be the first definition of the absolute,
so all further determinations and developments would be only
more determinate and richer definitions of it.
But let those who are not satisfied with being as the beginning,
since being passes over into nothing
and what emerges is the unity of the two,
let them consider what is more likely to satisfy them:
this beginning that begins with the representation
of the beginning and an analysis of it
(an analysis that is indeed correct yet
equally leads to the unity of being and non-being)
or a beginning which makes being the beginning.

But, regarding this strategy,
there is still a further observation to be made.
The said analysis presupposes that
the representation of the beginning is known;
its strategy follows the example of other sciences.
These presuppose their object and presume that
everyone has the same representation of it
and will find in it roughly the same determinations
which they have collected here or there,
through analysis, comparison, and sundry argumentation,
and they then offer as its representations.
But that which constitutes the absolute beginning must
likewise be something otherwise known;
now, if it is something concrete
and hence in itself variously determined,
then this connectedness which it is
in itself is presupposed as a known;
the connectedness is thereby adduced
as something immediate, which however it is not;
for it is connectedness only as
a connection of distinct elements
and therefore contains mediation within itself.
Further, the accidentality
and the arbitrariness of the analysis
and the specific mode of determination
affect the concrete internally.
Which determinations are elicited depends
on what each individual happens to discover
in his immediate accidental representation.
The connection contained within a concrete something,
within a synthetic unity, is necessary only in so far
as it is not found already given but is produced rather
by the spontaneous return of the moments back into this unity,
a movement which is the opposite of the analytical
procedure that occurs rather within the subject
and is external to the fact itself.

Here we then have the precise reason why
that with which the beginning is to be made
cannot be anything concrete,
anything containing a connection within its self.
It is because, as such, it would presuppose
within itself a process of mediation
and the transition from a first to an other,
of which process the concrete something,
now become a simple, would be the result.
But the beginning ought not itself to be
already a first and an other,
for anything which is
in itself a first and an other
implies that an advance has already been made.
Consequently, that which constitutes
the beginning, the beginning itself,
is to be taken as something unanalyzable,
taken in its simple, unfilled immediacy;
and therefore as being,
as complete emptiness.

If, impatient with this talk of an abstract beginning,
one should say that the beginning is to be made,
not with the beginning, but directly with the fact itself,
well then, this subject matter is nothing else
than that empty being.
For what this subject matter is,
that is precisely what ought to result
only in the course of the science,
what the latter cannot presuppose to know in advance.

On any other form otherwise assumed
in an effort to have a beginning
other than empty being,
that beginning would still
suffer from the same defects.
Let those who are still dissatisfied
with this beginning take upon themselves
the challenge of beginning in some other way
and yet avoiding such defects.

But we cannot leave entirely unmentioned
a more original beginning to philosophy
which has recently gained notoriety,
the beginning with the “I.”
It derived from both the reflection that
all that follows from the first truth
must be deduced from it,
and the need that this first truth
should be something with which one is already acquainted,
and even more than just acquainted,
something of which one is immediately certain.
This proposed beginning is not,
as such, an accidental representation,
or one which might be one thing
to one subject and something else to another.
For the “I,” this immediate consciousness of the self,
appears from the start to be both itself
an immediate something and something with which
we are acquainted in a much deeper sense than with any other representation;
true, anything else known belongs to this “I,”
but it belongs to it as a content which remains
distinct from it and is therefore accidental;
the “I,” by contrast, is the simple certainty of its self.
But the “I” is, as such, at the same time also a concrete,
or rather, the “I” is the most concrete of all things,
the consciousness of itself as an infinitely manifold world.
Before the “I” can be the beginning and foundation of philosophy,
this concreteness must be excised,
and this is the absolute act by virtue of which
the “I” purifies itself and makes its entrance
into consciousness as abstract “I.”
But this pure “I” is now not immediate,
is not the familiar, ordinary “I” of our consciousness
to which everyone immediately links science.
Truly, that act of excision would be
none other than the elevation to the standpoint of pure knowledge in
which the distinction between subject and object has disappeared.
But as thus immediately demanded,
this elevation is a subjective postulate;
before it proves itself as a valid demand,
the progression of the concrete “I”
from immediate consciousness to pure knowledge
must be demonstratively exhibited within the “I” itself,
through its own necessity.
Without this objective movement, pure knowledge,
also when defined as intellectual intuition,
appears as an arbitrary standpoint,
itself one of those empirical states of consciousness
for which everything depends on whether someone,
though not necessarily somebody else,
discovers it within himself
or is able to produce it there.
But inasmuch as this pure “I” must be essential, pure knowledge;
and pure knowledge is however one which is only posited in
individual consciousness through an absolute act of self-elevation,
is not present in it immediately;
we lose the very advantage which was to derive
from this beginning of philosophy,
namely that it is something with which
everyone is well acquainted,
something which everyone finds within himself
and to which he can attach further reflection;
that pure “I,” on the contrary,
in its abstract, essential nature,
is to ordinary consciousness an unknown,
something that the latter does not find within itself.
What comes with it is rather the disadvantage of
the illusion that we are speaking of something supposedly very familiar,
the “I” of empirical self-consciousness,
whereas at issue is in fact something far removed from the latter.
Determining pure knowledge as “I” acts as
a continuing reminder of the subjective “I”
whose limitations should rather be forgotten;
it leads to the belief that the propositions and relations
which result from the further development
of the “I” occur within ordinary consciousness
and can be found pregiven there,
indeed that the whole issue is about this consciousness.
This mistake, far from bringing clarity,
produces instead an even more glaring and bewildering confusion;
among the public at large, it has occasioned
the crudest of misunderstandings.

Further, as regards the subjective determinateness of the “I” in general,
pure knowledge does remove from it the restriction that it has
when understood as standing in unsurmountable opposition to an object.
But for this reason it would be at least superfluous
still to hold on to this subjective attitude
by determining pure knowledge as “I.”
For this determination not only carries with it
that troublesome duality of subject and object;
on closer examination, it also remains a subjective “I.”
The actual development of the science that proceeds from the “I”
shows that in the course of it the object has and retains
the self-perpetuating determination of an other
with respect to the “I”;
that therefore the “I” from which the start was made
does not have the pure knowledge that has truly overcome
the opposition of consciousness,
but is rather still entangled in appearance.
In this connection, there is the further
essential observation to be made that,
although the “I” might well be determined to be
in itself pure knowledge or intellectual intuition
and declared to be the beginning,
in science we are not concerned with
what is present in itself or as something inner,
but with the external existence rather of what in thought is inner
and with the determinateness which this inner assumes in that existence.
But whatever externalization there might be of
intellectual intuition at the beginning of science,
or if the subject matter of science is called
the eternal, the divine, the absolute,
of the eternal or absolute,
this cannot be anything else than a first, immediate, simple determination.
Whatever richer name be given to it than is expressed by mere being,
the only legitimate consideration is how
such an absolute enters into discursive knowledge
and the enunciation of this knowledge.
Intellectual intuition might well be the violent rejection
of mediation and of demonstrative, external reflection.
However, anything which it says over and above
simple immediacy would be something concrete,
and this concrete would contain a diversity of determinations in it.
But, as already remarked, the enunciation and exposition
of this concrete something is a process of mediation
which starts with one of the determinations and proceeds to another,
even though this other returns to the first;
and this is a movement which, moreover, is
not allowed to be arbitrary or assertoric.
Consequently, that from which the beginning is made in any
such exposition is not something itself concrete
but only the simple immediacy from which the movement proceeds.
Besides, what is lacking if we make something concrete the beginning
is the demonstration which the combination of the determinations
contained in it requires.

Therefore, if in the expression of the absolute,
or the eternal, or God
(and God would have the perfectly undisputed right
that the beginning be made with him),
if in the intuition or the thought of them,
there is more than there is in pure being,
then this more should first emerge in a knowledge
which is discursive and not figurative;
as rich as what is implicitly contained in knowledge may be,
the determination that first emerges in it is something simple,
for it is only in the immediate that no advance is
yet made from one thing to an other.
Consequently, whatever in the richer
representations of the absolute or God
might be said or implied over and above being,
all this is at the beginning
only an empty word and only being;
this simple determination which has no further meaning
besides, this empty something, is as such, therefore,
the beginning of philosophy.

This insight is itself so simple that this beginning is
as beginning in no need of any preparation or further introduction,
and the only possible purpose of this preliminary disquisition regarding it
was not to lead up to it but to dispense rather with all preliminaries.

I.40
paramanu-parama-mahattvanto 'sya vasikara

GENERAL DIVISION OF BEING

Being is determined, first, as against another in general;
secondly, it is internally self-determining;
thirdly, as this preliminary division is cast off,
it is the abstract indeterminateness and immediacy
in which it must be the beginning.

According to the first determination,
being partitions itself off from essence,
for further on in its development it proves to be
in its totality only one sphere of the concept,
and to this sphere as moment it opposes another sphere.
According to the second, it is the sphere
within which fall the determinations
and the entire movement of its reflection.

In this, being will posit itself in three determinations:

I. as determinateness; as such, quality;

II. as sublated determinateness; magnitude, quantity;

III. as qualitatively determined quantity; measure.

This division, as was generally remarked of
such divisions in the Introduction,
is here a preliminary statement;
its determinations must first arise
from the movement of being itself,
and receive their definitions and justification by virtue of it.
As regards the divergence of this division from
the usual listing of the categories,
namely quantity, quality, relation and modality,
(for Kant, incidentally, these are supposed to be
only classifications of his categories,
but are in fact themselves categories,
only more abstract ones;
about this, there is nothing to remark here,
since the entire listing will diverge from
the usual ordering and meaning of
the categories at every point.)

This only can perhaps be remarked,
that the determination of quantity
is ordinarily listed ahead of quality and as a rule
this is done for no given reason.
It has already been shown that
the beginning is made with being as such,
and hence with qualitative being.
It is clear from a comparison of quality with quantity
that the former is by nature first.
For quantity is quality which has already become negative;
magnitude is the determinateness which,
no longer one with being but already distinguished from it,
is the sublated quality that has become indifferent.
It includes the alterability of being
without altering the fact itself,
namely being, of which it is the determination;
qualitative determinateness is on the contrary one with its being,
it neither transcends it nor stays within it
but is its immediate restrictedness.
Hence quality, as the determinateness which is immediate,
is the first and it is with it that the beginning is to be made.

Measure is a relation, not relation in general
but specifically of quality and quantity to each other;
the categories dealt with by Kant under relation
will come up elsewhere in their proper place.
Measure, if one so wishes, can be considered also a modality;
but since with Kant modality is no longer
supposed to make up a determination of content,
but only concerns the reference of the content
to thought, to the subjective,
the result is a totally heterogeneous reference
that does not belong here.

The third determination of being falls within
the section Quality inasmuch as being, as abstract immediacy,
reduces itself to one single determinateness
as against its other determinacies inside its sphere.

SECTION I

Determinateness (Quality)

Being is the indeterminate immediate;
it is free of determinateness with respect to essence,
just as it is still free of any determinateness
that it can receive within itself.
This reflectionless being is being
as it immediately is only within.

Since it is immediate, it is being without quality;
but the character of indeterminateness attaches to it in itself
only in opposition to what is determinate or qualitative.
Determinate being thus comes to stand over and against being in general;
with that, however, the very indeterminateness of being
constitutes its quality.
It will therefore be shown that the first being is
in itself determinate, and therefore secondly,
that it passes over into existence, is existence;
that this latter, however, as finite being, sublates itself
and passes over into the infinite reference of being to itself;
it passes over, thirdly, into being-for-itself.

CHAPTER 1

Being

I.41
ksina-vrtter abhijatasyeva maner
grahitr-grahana-grahyesu tat-stha-tad-anjanata samapatti

A. BEING

Being, pure being, without further determination.
In its indeterminate immediacy it is equal only to itself
and also not unequal with respect to another;
it has no difference within it, nor any outwardly.
If any determination or content were posited in it as distinct,
or if it were posited by this determination or content
as distinct from an other,
it would thereby fail to hold fast to its purity.
It is pure indeterminateness and emptiness.
There is nothing to be intuited in it,
if one can speak here of intuiting;
or, it is only this pure empty intuiting itself.
Just as little is anything to be thought in it,
or, it is equally only this empty thinking.
Being, the indeterminate immediate is in fact nothing,
and neither more nor less than nothing.

B. NOTHING

Nothing, pure nothingness;
it is simple equality with itself,
complete emptiness,
complete absence of determination and content;
lack of all distinction within.
In so far as mention can be made here of
intuiting and thinking,
it makes a difference whether something or nothing is
being intuited or thought.
To intuit or to think nothing has therefore a meaning;
the two are distinguished and so nothing is (concretely exists)
in our intuiting or thinking;
or rather it is the empty intuiting and thinking itself,
like pure being.
Nothing is therefore the same determination
or rather absence of determination,
and thus altogether the same as what pure being is.

C. BECOMING

1. Unity of being and nothing

Pure being and pure nothing are therefore the same.
The truth is neither being nor nothing,
but rather that being has passed over into nothing
and nothing into being;
“has passed over,” not passes over.

But the truth is just as much that
they are not without distinction;
it is rather that they are not the same,
that they are absolutely distinct
yet equally unseparated and inseparable,
and that each immediately vanishes in its opposite.

Their truth is therefore this movement of
the immediate vanishing of the one into the other:
becoming, a movement in which the two are distinguished,
but by a distinction which has just as immediately dissolved itself.

2. The moments of becoming

Becoming is the unseparatedness of being and nothing,
not the unity that abstracts from being and nothing;
as the unity of being and nothing
it is rather this determinate unity,
or one in which being and nothing equally are.
However, inasmuch as being and nothing are
each unseparated from its other, each is not.
In this unity, therefore, they are,
but as vanishing, only as sublated.
They sink from their initially represented self-subsistence
into moments which are still distinguished
but at the same time sublated.

Grasped as thus distinguished,
each is in their distinguishedness
a unity with the other.
Becoming thus contains being and nothing as two such unities,
each of which is itself unity of being and nothing;
the one is being as immediate and as reference to nothing;
the other is nothing as immediate and as reference to being;
in these unities the determinations are of unequal value.

Becoming is in this way doubly determined.
In one determination, nothing is the immediate,
that is, the determination begins with nothing
and this refers to being;
that is to say, it passes over into it.
In the other determination, being is the immediate,
that is, the determination begins with being
and this passes over into nothing:
coming-to-be and ceasing-to-be.

Both are the same, becoming,
and even as directions that are so different
they interpenetrate and paralyze each other.
The one is ceasing-to-be;
being passes over into nothing,
but nothing is just as much the opposite of itself,
the passing-over into being, coming-to-be.
This coming-to-be is the other direction;
nothing goes over into being,
but being equally sublates itself
and is rather the passing-over into nothing;
it is ceasing-to-be.
They do not sublate themselves reciprocally
[the one sublating the other externally]
but each rather sublates itself in itself
and is within it the opposite of itself.

3. Sublation of becoming

The equilibrium in which coming-to-be and ceasing-to-be are poised
is in the first place becoming itself.
But this becoming equally collects itself in quiescent unity.
Being and nothing are in it only as vanishing;
becoming itself, however, is only by virtue of their being distinguished.
Their vanishing is therefore the vanishing of becoming,
or the vanishing of the vanishing itself.
Becoming is a ceaseless unrest that collapses into a quiescent result.

This can also be expressed thus:
becoming is the vanishing of being into nothing,
and of nothing into being,
and the vanishing of being and nothing in general;
but at the same time it rests on their being distinct.
It therefore contradicts itself in itself,
because what it unites within itself is self-opposed;
but such a union destroys itself.

This result is a vanishedness, but it is not nothing;
as such, it would be only a relapse into one of
the already sublated determinations
and not the result of nothing and of being.
It is the unity of being and nothing
that has become quiescent simplicity.
But this quiescent simplicity is being,
yet no longer for itself but as determination of the whole.

Becoming, as transition into
the unity of being and nothing,
a unity which is as existent
or has the shape of the one-sided
immediate unity of these moments,
is existence.

CHAPTER 2

Existence

I.42
tatra sabda-artha-jnana-vikalpa sankirna savitarka samapatti

Existence is determinate being;
its determinateness is existent determinateness, quality.
Through its quality, something is opposed to an other;
it is alterable and finite,
negatively determined not only towards an other,
but absolutely within it.
This negation in it,
in contrast at first
with the finite something,
is the infinite;
the abstract opposition
in which these determinations appear
resolves itself into oppositionless infinity,
into being-for-itself.

The treatment of existence is therefore in three divisions:

A. existence as such
B. something and other, finitude
C. qualitative infinity.

Transition

Ideality can be called the
quality of the infinite;
but it is essentially
the process of becoming,
and hence a transition,
like the transition
of becoming into existence.
We must now explicate this transition.
This immanent turning back,
as the sublating of finitude,
of finitude as such
and equally of the negative finitude
that only stands opposite to it,
is only negative finitude,
is self-reference, being.
Since there is negation
in this being,
the latter is existence;
but, further, since the
negation is essentially
negation of the negation,
self-referring negation,
it is the existence that
carries the name of
being-for-itself.

CHAPTER 3

Being-for-itself

I.43
smrti-parisuddha svarupa-sunya-iva-artha-matra-nirbhasa nirvitarka

In being-for-itself,
qualitative being is brought to completion;
it is infinite being;
the being of the beginning is void of determination;
existence is sublated but only immediately sublated being;
it thus contains, to begin with,
only the first negation, itself immediate;
being is of course retained as well,
and the two are united in existence in simple unity;
for this reason, however,
each is in itself still unlike the other,
and their unity is still not posited.
Existence is therefore the sphere of differentiation,
of dualism, the domain of finitude.
Determinateness is determinateness as such;
being which is relatively, not absolutely, determined.
In being-for-itself, the distinction
between being and determinateness,
or negation, is posited and equalized.
Quality, otherness, limit, as well as reality,
in-itselfness, ought, and so forth, are the
incomplete configurations of negation in being
which are still based on the differentiation of the two.
But since in finitude negation has passed over into infinity,
in the posited negation of negation,
negation is simple self-reference
and in it, therefore, the equalization with being:
absolutely determinate being.

First, being-for-itself is immediately
an existent-for-itself, the one.

Second, the one passes over
into a multiplicity of ones,
repulsion or the otherness of the one
which sublates itself into its ideality, attraction.

Third, we have the alternating
determination of repulsion and attraction
in which the two sink into a state of equilibrium;
and quality, driven to a head in being-for-itself,
passes over into quantity.

SECTION II

Magnitude (Quantity)

I.44
etaya-iva savicara nirvicara ca suksma-visaya vyakhyata

I.45
suksma-visayatvam calinga-paryavasanam

I.46
ta eva sabija samadhi

The difference between quantity and quality has been indicated.
Quality is the first, immediate determinateness.
Quantity is the determinateness that
has become indifferent to being;
a limit which is just as much no limit;
being-for-itself which is absolutely
identical with being-for-another:
the repulsion of the many ones
which is immediate non-repulsion,
their continuity.

Because that which exists for itself is now so posited
that it does not exclude its other
but rather affirmatively continues in it,
it is then otherness, inasmuch as
existence surfaces again on this continuity
and its determinateness is at the same time
no longer simple self-reference,
no longer the immediate determinateness
of the existent something,
but is posited as repelling itself from itself,
as referring to itself in the determinateness
rather of an other existence
(a being which exists for itself);
and since they are at the same time indifferent limits,
reflected into themselves and unconnected,
determinateness is as such outside itself,
an absolute externality and a something just as external;
such a limit, the indifference of the limit as limit
and the indifference of the something to the limit,
constitutes the quantitative determinateness of the something.

In the first place, we have to distinguish pure quantity
from quantity as determinate, from quantum.
First, pure quantity is real being-for-itself
turned back into itself, with as yet no determinateness in it:
a compact, infinite unity which continues itself into itself.

Second, this quantity proceeds to determinateness,
and this is posited in it as a determinateness
that at the same time is none, is only external.
Quantity becomes quantum.
Quantum is indifferent determinateness,
that is, one that transcends itself, negates itself;
as this otherness of otherness, it lapses into infinite progress.
Infinite quantum, however, is sublated indifferent determinateness:
it is the restoration of quality.

Third, quantum in qualitative form is quantitative ratio.
Quantum transcends itself only in general;
in the ratio, however, it transcends itself into its otherness,
in such a way that this otherness in which it has its determination
is at the same time posited, is another quantum.
With this we have quantum as turned back into itself
and referring to itself as into its otherness.

At the foundation of this relation there still
lies the externality of quantum;
it is indifferent quanta that
relate themselves to each other,
that is, they have the reference
that mutually connects them
in this being-outside-itself.
The ratio is, therefore, only
a formal unity of quality and quantity,
and dialectic is its transition into
their absolute unity, in measure.

CHAPTER 1

Quantity

A. PURE QUANTITY

Quantity is sublated being-for-itself.
The repelling one that behaved only
negatively towards the excluded one,
now that it has gone over in connection with it,
behaves towards the other as identical to itself
and has therefore lost its determination;
being-for-itself has passed over into attraction.
The absolute obduracy of the one has melted away
into this unity which, however, as containing the one, is
at the same time determined by the repulsion residing in it;
as unity of the self-externality, it is unity with itself.
Attraction is in this way the moment of continuity in quantity.

Continuity is therefore simple, self-same reference to itself
unbroken by any limit or exclusion;
not, however, immediate unity but the unity of ones
which have existence for themselves.
Still contained in it is the outside-one-another of plurality,
though at the same time as something without distinctions, unbroken.
Plurality is posited in continuity as it implicitly is in itself;
the many are each what the others are,
each is like the other,
and the plurality is, consequently,
simple and undifferentiated equality.
Continuity is this moment of self-equality
of the outsideness-of-one-another,
the self-continuation of the different ones
into the ones from which they are distinguished.

In continuity, therefore, magnitude immediately possesses
the moment of discreteness, repulsion as now a moment in quantity.
Steady continuity is self-equality,
but of many that do not become exclusive;
it is repulsion that first expands self-equality to continuity.
Hence discreteness is, for its part, a discreteness of confluents,
of ones that do not have the void to connect them,
not the negative, but their own steady advance
and, in the many, do not interrupt this self-equality.

Quantity is the unity of these moments,
of continuity and discreteness.
At first, however, it is this continuity
in the form of one of them, of continuity,
as a result of the dialectic of the being-for-itself
which has collapsed into the form of self-equal immediacy.
Quantity is as such this simple result
in so far as the being-for-itself has
not yet developed its moments
and has not posited them within it.
Quantity contains these moments at first
as being-for-itself posited in its truth.
It was the determination of being-in-itself
to be self-sublating self-reference,
a perpetual coming-out-of-itself.
But what is repelled is itself;
repulsion is thus a creative flowing away from itself.
On account of the sameness of what is repelled,
this discerning is unbroken continuity;
and on account of the coming-out-of-itself,
this continuity is at the same time,
without being broken off, a plurality,
a plurality which persists just as
immediately in its equality with itself.

B. CONTINUOUS AND DISCRETE MAGNITUDE

1. Quantity contains the two moments of continuity and discreteness.
It is to be posited in both, in each as its determination.
It is already from the start the immediate unity of the two,
that is, quantity is itself posited at first
only in one of the two determinations, that of continuity,
and as such is continuous magnitude.

Or continuity is indeed one of the moments of quantity
which is brought to completion only with the other, discreteness.
But quantity is concrete unity only in so far as
it is the unity of distinct moments.
These are to be taken, therefore, also as distinct,
without however resolving them again
into attraction and repulsion
but, rather, as they truly are,
each remaining in its unity with the other,
that is, remaining the whole.
Continuity is only the compact unity
holding together as unity of the discrete;
posited as such, it is no longer only moment
but the whole quantity: continuous magnitude.

2. Immediate quantity is continuous magnitude.
Quantity, however, is not as such an immediate;
immediacy is a determinateness,
the sublated being of which is precisely quantity.
Quantity is to be posited, therefore,
in the determinateness immanent to it,
and this is the one.
Quantity is discrete magnitude.

Discreteness is, like continuity, a moment of quantity,
but is itself also the whole quantity
just because it is a moment in it, the whole,
and therefore as distinct moment does not
diverge from its unity with the other moment.
Quantity is the outsideness-of-one-another as such,
and continuous magnitude is this outsideness-of-one-another
onwardly positing itself without negation
as an internally self-same connectedness.
On the other hand, discrete magnitude is this
outsideness-of-one-another as discontinuous, as broken off.
With this aggregate of ones, however,
the aggregate of atom and void,
repulsion in general, is not thereby reinstated.
Because discrete magnitude is quantity,
its discreteness is itself continuous.
Such a continuity in the discrete consists
in the ones being the same as one another,
or in that they have the same unity.
Discrete magnitude is therefore
the one-outside-the-other of
the many ones as of a same;
not the many ones in general,
but posited rather as the many of a unity.

C. THE LIMITING OF QUANTITY

Discrete magnitude has,
first, the one for its principle
and, second, is a plurality of ones;
third, it is essentially continuous,
it is the one as at the same time
sublated, as unity,
the self-continuing as such in
the discreteness of the ones.
Consequently, it is posited as one magnitude,
and the “one” is its determinateness;
a “one” which, in this posited and determinate existence,
excludes, is a limit to the unity.
Discrete magnitude as such is not supposed
to be immediately limited,
but, when distinguished from continuous magnitude,
it is an existence and a something,
the determinateness of which,
and in it also the first negation and limit,
is the “one.”

This limit, besides referring to the unity
and being the moment of negation in it,
is also, as one, self-referred;
thus it is enclosing, encompassing limit.
The limit here is not at first distinct
from the something of its existence,
but, as one, is essentially this negative point itself.
But the being which is here limited is essentially as continuity,
and in virtue of this continuity it transcends the limit,
transcends this one, and is indifferent to it.
Real, discrete quantity is thus one quantity, or quantum:
quantity as an existence and a something.

Since the one which is a limit encompasses
within it the many ones of discrete quantity,
it posits them equally as sublated in it;
it is a limit to continuity simply as such
and, consequently, the distinction between
continuous and discrete magnitude is here indifferent;
or, more precisely, it is a limit to
the continuity of the one
just as much as of the other;
in this, both pass over into being quanta.

CHAPTER 2

Quantum

Quantum, which in the first instance is quantity
with a determinateness or limit in general,
in its complete determinateness is number.
Second, quantum divides first into extensive quantum,
in which limit is the limitation of
a determinately existent plurality;
and then, inasmuch as the existence of
this plurality passes over into being-for-itself,
into intensive quantum or degree.
This last is for-itself but also,
as indifferent limit, equally outside itself.
It thus has its determinateness in an other.
Third, as this posited contradiction of
being determined simply in itself
yet having its determinateness outside itself
and pointing outside itself for it,
quantum, as thus posited outside itself within itself,
passes over into quantitative infinity.

CHAPTER 3

Ratio or the quantitative relation

The infinity of quantum has been determined up to
the point where it is the negative beyond of quantum,
a beyond which quantum, however, has within it.
This beyond is the qualitative moment in general.
The infinite quantum, as the unity of the two moments,
of the quantitative and the qualitative determinateness,
is in the first instance ratio.

In ratio, quantum no longer has
a merely indifferent determinateness
but is qualitatively determined as
simply referring to its beyond.
It continues in its beyond, and this beyond is
at first just an other quantum.
Essentially, however, the two do not refer
to each other as external quanta
but each rather possesses its determinateness
in this reference to the other.
In this, in their otherness, they have
thus returned into themselves;
what each is, that it is in its other;
the other constitutes the determinateness of each.
The quantum's self-transcendence does not now mean, therefore,
that quantum has simply changed either into some other
or into its abstract other, into its abstract beyond,
but that there, in the other, it has attained its determinateness;
in its other, which is an other quantum, it finds itself.
The quality of quantum, its conceptual determinateness,
is its externality as such,
and in ratio quantum is now posited as having
its determinateness in this externality, in another quantum;
as being in its beyond what it is.
It is quanta that stand to each other in the connection
that has now come on the scene.
This connection is itself also a magnitude;
quantum is not only in relation,
but is itself posited as relation;
it is a quantum as such that has
that qualitative determinateness in itself.
So, as relation (as ratio),
quantum gives expression to itself as self-enclosed totality
and to its indifference to limit by containing
the externality of its being-determined in itself:
in this externality it is only referred back to itself
and is thus infinite within.

Ratio in general is:

1. direct ratio.
In this, the qualitative moment does not
yet emerge explicitly as such;
in no other way except still as quantum is quantum posited
as having its determinateness in its externality.
In itself, however, the quantitative relation is
the contradiction of externality and self-reference,
the persistence of quanta and their negation.
Such a contradiction next sublates itself:

2. first inasmuch as in indirect or inverse ratio
the negation of each of the quanta is
as such co-posited in the alteration of the other,
and the variability of the direct ratio is itself posited;

3. but in the ratio of powers, the unity,
which in its difference refers back to itself,
proves to be a simple self-production of the quantum;
this qualitative moment itself,
finally posited in a simple determination
and as identical with the quantum, becomes measure.

About the nature of the following ratios,
much was anticipated in the preceding remarks
concerning the infinity of quantity, that is,
the qualitative moment in it;
it only remains to analyze, therefore,
the abstract concept of this ratio.

A. THE DIRECT RATIO

1. In the ratio which, as immediate, is direct,
the determinateness of each quantum lies in
the reciprocal determinateness of the other.
There is only one determinateness or limit of both
one which is itself a quantum,
namely the exponent of the ratio.

2. The exponent is some quantum or other;
however, in referring itself to itself
in the otherness which it has within it,
it is only a qualitatively determined quantum,
for its difference, its beyond and otherness, is in it.
This difference in the quantum is the difference of unit and amount;
the unit, which is the being-determined-for-itself;
the amount, which is the indifferent fluctuation of determinateness,
the external indifference of quantum.
Unit and amount were at first the moments of quantum;
now, in the ratio, in quantum as realized so far,
each of its moments appears as a quantum on its own
and as determinations of the existence of the quantum,
as delimitations against the otherwise
external, indifferent determinateness of magnitude.

The exponent is this difference as simple determinateness,
that is, it has the meaning of both
determinations immediately in it.
First, it is a quantum and thus an amount.
If the one side of the ratio which is taken as
unit is expressed in a numerical one,
and has only the value of one,
then the other, the amount, is
the quantum of the exponent itself.
Second, it is simple determinateness as
the qualitative moment of the sides of the ratio.
When the quantum of the one side is determined,
the other is also determined by the exponent
and it is a matter of total indifference
how the first is determined;
it no longer has any meaning
as a quantum determined for itself
but can just as well be any other
quantum without thereby altering
the determinateness of the ratio,
which rests solely on the exponent.
The one side which is taken as unit always
remains unit however great it becomes,
and the other, however great it too thereby becomes,
must remain the same amount of that unit.

3. Accordingly, the two truly constitute only one quantum;
the one side has only the value of unit
with respect to the other, not of an amount;
and the other only that of amount.
According to their conceptual determinateness,
therefore, they are themselves not complete quanta.
But this incompleteness is in them a negation,
and it is so not because of their general variability,
according to which one of them (any of the two)
can assume all possible magnitudes,
but because they are so determined that, as one is altered,
the other is increased or decreased in corresponding measure.
This means that, as indicated, only one of them,
the unit, is altered as quantum;
the other side, the amount, remains the same quantum of units,
and the first side too retains the value of a unit,
however much it is altered as quantum.
Each side is thus only one of the two moments of quantum,
and the self-subsistence which is their proper characteristic
is in principle negated;
in this qualitative combination they are to be
 posited as negative with respect to each other.

The exponent ought to be the complete quantum,
since the determinations of both sides come together in it;
but in fact, even as quotient the
exponent only has the value of amount, or of unit.
There is nothing available for determining
which of the two sides of the relation would have to
be taken as the unit or as the amount;
if one side, quantum B,
is measured against quantum A as unit,
then the quotient C is the amount of such units;
but if A is itself taken as amount,
the quotient C is the unit which is required by
the amount A for the quantum B.
As exponent, therefore, this quotient is
not posited for what it ought to be,
namely the determinant of the ratio,
or the ratio's qualitative unity.
It is posited as such only to
the extent that its value is that
of the unity of the two moments,
of unit and amount.
And since these two sides, as quanta,
are indeed present as they should be
in the explicated quantum, in the ratio,
but at the same time have the value,
which is specific to them as the sides of the ratio,
of being incomplete quanta and of counting
only as one of those qualitative moments,
they are to be posited with this negation qualifying them.
Thus there arises a more real ratio,
one more in accordance with its definition,
one in which the exponent has the meaning of the product of the sides.
In this determinateness, it is the inverse ratio.

B. THE INVERSE RATIO

1. The ratio as now before us is the sublated direct ratio.
It was an immediate relation and therefore not yet truly determinate;
henceforth, the newly introduced determinateness gives
the exponent the value of a product,
the unity of unit and amount.
In immediacy, as we have just seen,
it was possible for the exponent to be
indifferently taken as unit or amount.
Moreover, it also was only a quantum in general
and therefore an amount by choice.
One side was the unit,
and this was to be taken as a numerical one
with respect to which the other side would be a fixed amount
and at the same time the exponent.
The quality of the latter, therefore,
was only that this quantum is taken as fixed,
or rather that the constant only
has the meaning of quantum.

Now in the inverse ratio,
the exponent is as quantum likewise immediate,
a quantum or other which is assumed as fixed.
But to the one of the other quantum in the ratio,
this quantum is not a fixed amount; the ratio,
previously taken as fixed, is now posited instead as alterable;
if another quantum is taken as
the unit of the one side,
the other side now no longer remains
the same amount of units of the first side.
In direct ratio, this unit
is only the common element of both sides;
as such, it continues into the other side, the amount;
the amount itself, or the exponent,
is by itself indifferent to the unit.

But as the determinateness of the ratio now is,
the amount as such alters relative to the unit
with respect to which it makes up the other side of the ratio;
whenever another quantum is taken as the unit,
that amount alters.
Consequently, although the exponent is
still only an immediate quantum
and only arbitrarily assumed as fixed,
it does not remain fixed in the side of the ratio:
rather this side, and with it
the direct ratio of the sides, is alterable.
In the ratio now before us,
the exponent as the determining quantum is
thus posited as negative towards itself
as a quantum of the ratio,
and hence as qualitative, as limit;
the result is that the qualitative moment distinctly comes
to the fore for itself as against the quantitative moment.
In the direct ratio, the alteration of the two sides is
only the one alteration of the quantum
from which the unity which is the common element is taken;
by as much, therefore, as the one side is
increased or decreased, so also is the other;
the ratio itself is indifferent to this alteration
and the alteration external to the ratio.
In the indirect ratio, on the contrary,
although still arbitrary according to
the moment of quantitative indifference,
the alteration is contained within the ratio,
and its arbitrary quantitative extension is
limited by the negative determinateness
of the exponent as by a limit.

2. We must now consider this qualitative nature
of the indirect ratio more closely, as it is realized,
and sort out the entanglement of the affirmative
and the negative moments that are contained in it.
Quantum is posited as being quantum qualitatively,
that is, as self-determining,
as displaying its limit within it.
Accordingly, first, it is an immediate magnitude
as simple determinateness;
it is the whole as existent, an affirmative quantum.
But, second, this immediate determinateness is
at the same time limit;
for that purpose it is distinguished into two quanta
which are at first the other of each other;
but as the qualitative determinateness of these quanta,
and as a determinateness which is moreover complete,
quantum is the unity of the unit and the amount,
a product of which the two are the factors.
Thus on the one hand the exponent of their ratio
is in them identical with itself
and is their affirmative moment by which they are quanta;
on the other hand, as the negation posited in them,
it is in them the unity according to which each,
at first an immediate and limited quantum in general,
is at the same time so limited as to be
only implicitly in itself identical with its other.
Third, the exponent as the simple determinateness is
the negative unity of this differentiation of it into the two quanta,
and the limit of their reciprocal limiting.

In accordance with these determinations,
the two moments limit themselves inside the exponent
and each is the negative of the other,
for the exponent is their determinate unity;
the one moment becomes as many
times smaller as the other becomes greater;
each possesses a magnitude of its own
to the extent that this magnitude is in it that of their other,
that is, is the magnitude that the other lacks.
The magnitude of each in this way
continues into the other negatively;
how much it is in amount,
that much it sublates in the other as amount
and is what it is only through this negation
or limit which is posited in it by the other.
In this way, each contains the other as well,
and is proportioned to it,
for each is supposed to be only that quantum
which the other is not;
the magnitude of the other is
indispensable to the value of each,
and therefore inseparable from it.

This continuity of each in the other
constitutes the moment of unity
through which the two magnitudes stand in relation;
the moment of the one determinateness,
of the simple limit which is the exponent.
This unity, the whole, constitutes the in-itself
of each from which their given magnitude is distinct:
this is the magnitude according to which each is
only to the extent that it takes from the other
a part of their common in-itself which is the whole.
But each can take from the other only as much
as will make it equal to this in-itself;
it has its maximum in the exponent
which, in accordance with the stated second determination,
is the limit of the reciprocal delimitation.
And since each is a moment of the ratio only
to the extent that it limits the other
and is thereby limited by it,
it loses this, its determination,
by making itself equal to its in-itself;
in this loss, the other magnitude will not only
become a zero, but itself vanishes,
for what it ought to be is not just a quantum
but what it is as such a quantum,
namely only the moment of a ratio.
Thus each side is the contradiction
between the determination as its in-itself,
that is, as the unity of the whole which is the exponent,
and the determination as the moment of a ratio;
this contradiction is infinity again,
in a new form peculiar to it.

The exponent is the limit of the sides of its ratio,
within which limit the sides increase and decrease
proportionately to each other;
but they cannot become equal to this exponent
because of the latter's affirmative determinateness as quantum.
Thus, as the limit of their reciprocal limiting,
the exponent is (a) their beyond which they
infinitely approximate but can never attain.
This infinity in which the sides approximate their beyond is
the bad infinity of the infinite progression;
it is an infinity which is itself finite,
that finds its restriction in its opposite,
in the finitude of each side
and of the exponent itself,
and for this reason is only approximation.
But (b) the bad infinite is equally posited here
as what it is in truth, namely
the negative moment in general,
in accordance with which the exponent
is the simple limit as against the distinct quanta of the ratio:
it is the in-itself to which, as the absolutely alterable,
the finitude of the quanta is referred,
but which, as the quanta's negation,
remains absolutely different from them.
This infinite, which the quanta can only approximate,
is then equally found affirmatively present on their side:
the simple quantum of the exponent.
In it is attained the beyond with which
the sides of the ratio are burdened;
it is in itself the unity of the two or, consequently,
in itself the other side of each side;
for each side has only as much value
as the other does not have;
its whole determinateness thus rests in the other,
and this, their being-in-itself, is as affirmative infinity
simply the exponent.

3. With this, however, we have the transition of
the inverse ratio into a determination
other than the one it had at first.
This consisted in the fact that
the quantum is immediate
but at the same time so connected to
an other that the greater it is,
the smaller is the other,
that it is what it is by virtue of
negatively relating to the other;
also, a third magnitude is the
common restriction on this alteration
in magnitude that the two quanta undergo.
This reciprocal alteration,
as contrasted to the fixed qualitative limit,
is here their distinctive property;
they have the determination of
sum up are not only that this infinite beyond is
at the same time some present finite quantum or other,
but that its fixity
(which makes it with respect to the quantitative moment
the infinite beyond that it is,
and which is the qualitative moment of
being only as abstract self-reference)
has developed itself as a mediation
of itself with itself in its other,
the finite moments of the relation.
The general point is that the whole is as
such the limit of the reciprocal limiting of the two terms,
and that the negation of the negation
(and consequently infinity, the affirmative self-relation)
is therefore posited.
The more particular point is that, as product,
the exponent already implicitly is already
the unity of unit and amount,
whereas each of these two terms is
only one of two moments,
and for this reason the exponent encloses them in itself
and in them it implicitly refers itself to itself.
But in the inverse ratio the difference has developed
into the externality of quantitative being,
and the qualitative being is not only something fixed,
nor does it simply enclose the two moments
of the ratio immediately in it,
but in the externally existent otherness it rejoins itself.
It is this determination that stands out as a result
in the moments we have seen.
The exponent, namely, is found to be
the implicit being whose moments are realized in quanta
and in their generalized alterability.
The indifference of the magnitudes of these moments in the course of their
alteration displays itself as an infinite progression,
the basis of which is that in their indifference
their determinateness is to have their value
each in the value of the other.
Thus, (a) according to the affirmative side of their quantum,
the determinateness of the moments is that each is in itself
the whole of the exponent;
equally (b) they have the magnitude of the
exponent for their negative moment, for their reciprocal limiting;
their limit is that of the exponent.
The fact that such moments do not have any other immanent limit,
any fixed immediacy, is posited in the infinite progression
of their existence and in their limitation,
in the negation of every particular value.
This is, accordingly, the negation of the externality
of the exponent which is displayed in them, and the exponent
(itself equally a quantum as such and also expanded into quanta)
is thereby posited as preserving itself
in the negation of the indifferent subsistence of the moments,
as rejoining itself, and thus as the determining factor
in this movement of self-surpassing.

The ratio is hereby determined as the ratio of powers.

C. THE RATIO OF POWERS

1. Quantum, in positing itself as self-identical in its otherness
and in determining its own movement of self-surpassing,
has come to be a being-for-itself.
As such a qualitative totality, in positing itself as developed,
it has for its moments the conceptual determinations of number:
the unit and the amount.
This last, amount, is in the inverse ratio still an aggregate
which is not determined as amount by the unit itself but from elsewhere,
by a third determinate aggregate;
but now it is posited as determined only by the unit.
This is the case in the ratio of powers where the unit,
which in it is amount, is at the same time
the amount as against itself as unit.
The otherness, the amount of units, is the unit itself.
The power is an aggregate of units,
each of which is this aggregate itself.
The quantum, as indifferent determinateness, changes;
but inasmuch as the alteration is the raising to a power,
the otherness of the quantum is determined purely by itself.
The quantum is thus posited in the power as having returned into itself;
it is immediately itself and also its otherness.

The exponent of this ratio is no longer an immediate quantum,
as in the direct ratio and also in the inverse ratio.
In the ratio of powers, the exponent
is of an entirely qualitative nature;
it is this simple determinateness:
that the amount is the unit itself,
that the quantum is self-identical in its otherness.
And the side of its quantitative nature is to be found in this:
that the limit or negation is not an immediate existent,
but that existence is posited rather
as continuing in its otherness.
For the truth of quality is precisely to be quantity,
or immediate determinateness as sublated.

2. The ratio of powers appears at first as
an external alteration to which a given quantum is subjected;
but it has a closer connection with the concept of quantum,
namely, that in the existence into which the quantum
has developed in the ratio of powers,
quantum has attained that concept,
has realized its concept to the fullest.
The ratio of powers is the display of
what the quantum is implicitly in itself;
it expresses its determinateness of quantum
or the quality by which it is distinguished from another.
Quantum is indifferent determinateness posited as sublated,
that is to say, determinateness as limit,
one which is just as much no determinateness,
which continues in its otherness and in it,
therefore, remains identical with itself.
Thus is quantum posited in the ratio of powers:
its otherness, the surpassing of itself in another quantum,
as determined through the quantum itself.

If we compare the progressive realization
of quantum in the preceding ratios,
we find that quantum's quality of being
the difference of itself from itself is simply this:
that it is a ratio.
As the direct ratio, quantum is this posited difference
only in the first instance or immediately,
so that the self-reference which it has as exponent,
in contrast to its differences,
counts only as the fixity of an amount of the unit.
In the inverse ratio, as negatively determined,
quantum is a relating of itself to itself
(to itself as to its negation
in which, however, it has its value);
as an affirmative self-reference,
it is an exponent which, as quantum,
is only implicitly in itself the determinant of its moments.
But in the ratio of powers quantum is present
in the difference as a difference of itself from itself.
This externality of determinateness is
the quality of quantum and is thus posited,
in conformity to the concept of quantum,
as quantum's own determining,
as its reference to itself, its quality.

3. By being thus posited as it is in conformity to its concept,
quantum has passed over into another determination;
or, as we can also say, its determination is
now also as the determinateness,
the in-itself also as existence.
It is quantum in so far as the externality
or the indifference of its determining
(as we say, it is that which can be increased or decreased)
is simply accepted and immediately posited;
it has become the other of itself, namely quality,
in so far as that same externality is now posited
as mediated by quantum itself
and hence as a moment of quantum,
so that in that very externality
quantum refers itself to itself,
is being as quality.
At first quantity as such thus appears
in opposition to quality;
but quantity is itself a quality,
self-referring determinateness as such,
distinct from the determinateness which is its other,
from quality as such.
Except that quantity is not only a quality,
but the truth of quality itself is quantity,
and quality has demonstrated itself as passing over into it.
Quantity, in its truth, is instead the externality
which has returned into itself,
which is no longer indifferent.
Thus is quantity quality itself,
in such a way that outside this determination
quality as such would yet not be anything at all.

For the totality to be posited, a double transition is required,
not only the transition of one determinateness into the other,
but equally the transition of this other into the first,
its going back into it.
Through the first transition,
the identity of the two is present at first only in itself:
quality is contained in quantity,
but the latter still is only a one-sided determinateness.

Conversely, that quantity is equally contained in quality,
that it is equally also only as sublated,
this results in the second transition,
the going back into the first determinateness.
This remark regarding the necessity of the double transition
is everywhere of great importance for scientific method.

Quantum is henceforth no longer an indifferent
or external determination but is sublated as such,
and it is a quality and that by virtue of which
anything is what it is;
the truth of quantum is to be measure.

SECTION III

Measure

I.47
nirvicara-vaisaradye 'dhyatma-prasada

I.48
rtambhara tatra prajna

I.49
sruta-anumana-prajnabhyam anyavisaya visesa-arthatvat

I.50
taj-ja samskaro 'nya-samskara-pratibandhi

I.51
tasyapi nirodhe sarva-nirodhan nirbija samadhi

Abstractly expressed, quality and quantity are in measure united.
Being as such is the immediate equality of determinateness with itself.
This immediacy of determinateness has sublated itself.
Quantity is being that has returned to itself in such a way
that it is a simple self-equality indifferent to determinateness.
But this indifference is only the externality of having
the determinateness not in itself but in an other.
As third, we now have the externality that refers itself to itself;
as self-reference, it is at the same time sublated externality and
carries the difference from itself in it;
a difference which, as externality, is the quantitative moment,
and, as taken back into itself, the qualitative.

Since among the categories of transcendental idealism
modality comes after quantity and quality,
with relation inserted in between,
this is an appropriate place to say something about it.
In transcendental idealism, this category has the meaning that
it is the connection of the subject matter to thought.
As understood by that idealism, thought is as such
essentially external to the thing-in-itself.
Hence, inasmuch as the other categories have
the transcendental determination of
belonging only to consciousness,
but as its objective moment,
so modality, which is the category of
the connection to the subject,
possesses the determination of
reflection in itself in a relative sense,
that is to say, the objectivity
which is granted to the other categories
is lacking in those of modality;
these, according to Kant's words,
do not add in the least to the concept
as a determination of the object
but only express its relation to the faculty of cognition
(Cr. of Pure R., 2nd edn, pp. 99, 266).
The categories which Kant groups under modality
(possibility, actuality, and necessity)
will come up later in their proper place.
Kant did not apply the form of triplicity
(an infinitely important form even though with him
it occurred only as a formal spark of light)
to the genera of his categories (to quantity, quality, etc.),
but only to their species to which
he also gave the name of genera.
He was therefore unable to hit upon
the third to quality and quantity.

With Spinoza, the mode is likewise
the third after substance and attribute;
Spinoza defines it as the affections of substance,
or as that which is in another through which it is also comprehended.
In this way of conceiving it, this third is externality as such;
as has already been mentioned, with Spinoza generally,
the rigidity of substance lacks the turning back into itself.

The remark just made extends to any of the systems of pantheism
which thought has in one way or other produced.
Being, the one, substance, the infinite, essence, is the first;
opposite this abstraction is the second which
can be mustered in an equally abstract form,
as is habitually done as the next step in any purely formalistic thinking,
namely all determinateness generally taken as the mere finite,
the mere accidental, the transitory, the extraneous and unessential, etc.
But the bond connecting this second with the first is too invasive
for the second not to be not equally grasped with the first;
thus with Spinoza the attribute is the whole of substance,
though as comprehended by the understanding,
which is itself a restriction of substance or mode;
and so the mode, the insubstantial as such which can
be grasped only through an other,
constitutes the opposite extreme of substance, the third.
Also Indian pantheism, taken abstractly, has attained
in its monstrous fantasies this refinement
which runs like a moderating thread across its excesses
as its one point of interest namely
that Brahma, the one of abstract thought,
progresses through the shape of Vishnu,
particularly in the form of Krishna, to the third, Shiva.
The determination of this third is that of mode, alteration,
coming-to-be and passing-away;
it is the field of externality in general.
This Indian trinity has tempted a comparison with the Christian,
and one must indeed acknowledge a common element in them.
But it is essential to be aware of the difference that separates them.
It is not just that this difference is infinite
but that the true infinity makes the difference.
The determination of the Indian third principle is that
it is the dispersal of the substantial unity into its opposite,
not its turning back to itself,
a spiritual void rather, not spirit.
In the true trinity, there is not only unity but unification;
the syllogism is brought to a unity which is full of content and actual,
a unity which in its totally concrete determination is spirit.
The principle of the mode and of alteration
does not, of course, exclude unity altogether.
In Spinozism, for instance, precisely the mode is as such untrue
while substance alone is what truly is;
everything is supposed to be reduced to substance,
and this is then a sinking of all content into
an only formal unity void of content.
As for Shiva, it too is again the great whole,
not distinct from Brahma, from Brahma itself, that is,
the distinction and the determinateness just disappear
without being preserved, without being sublated,
and the unity does not become concrete unity,
nor is the diremption reconciled.
The supreme goal of the human being,
relegated as he is to the sphere
of coming-to-be and passing-away,
of modality in general,
is to sink into unconsciousness,
into unity with Brahma, annihilation;
the Buddhist Nirvana, Nibbana, etc., is the same.

Now although the mode is as such abstract externality,
indifference to qualitative as well as quantitative determinations,
and nothing in the essence should depend on the external,
the unessential, it is nevertheless conceded that
in the many all depends on the how;
but this is to concede that the mode itself
essentially belongs to the substance of a thing,
a very indefinite connection but
one which at least implies that the externality of
the mode is not all that abstract an externality after all.
Here the mode has the definite meaning of being measure.
The Spinozistic mode, just like the Indian principle of alteration,
is the measureless.
The Greeks were aware that everything has a measure.
Parmenides himself introduces necessity after abstract being,
as the ancient limit which is imposed on all.
And this awareness, although still vague, is
the beginning of a much higher concept
than is contained in substance
and in the distinction of the mode from it.

Measure in its more developed, more reflected form is necessity.
Fate, Nemesis, ultimately comes down to a determination of measure.
Whatever renders itself beyond the pale, becomes too great, too high,
is brought down to the other extreme of being reduced to nothing,
so that the mean of measure, the medium, is restored.
That the Absolute, God, is the measure of all things, is not
a stronger statement of pantheism than the definition,
“the Absolute, God, is being,” but is infinitely truer.
Measure is indeed an external way of things, a more or less,
but one which is as at the same time reflected into itself,
not merely an indifferent and external determinateness
but one which exists in itself;
thus it is the concrete truth of being.
For this reason the nations have revered in it
the presence of something inviolable and sacred.

Already present in measure is the idea of essence,
namely of being self-identical in
the immediacy of being determined,
so that this immediacy is reduced
through the self-identity to something mediated,
just as the self-identity is equally mediated
only through this externality,
but the mediation is one with itself:
this is reflection, the moments of which indeed are,
but in this being are absolutely nothing but
moments of their negative unity.
In measure, the qualitative element is quantitative;
the determinateness or the difference is indifferent
and therefore a difference which is none;
it is sublated and this quantitativeness,
as an immanent turning back in which it is qualitative,
constitutes the being-in-and-for-itself which is essence.
But measure is essence only implicitly in itself or in its concept;
this concept of measure is not yet posited.
Measure is as such still the existent unity of the
qualitative and the quantitative element;
its moments are an existence, a quality and some quanta of this quality
which, in themselves, are indeed only indivisible,
but do not yet have the meaning of this reflected determination.
In the development of measure, these moments are differentiated
but at the same time referred to each other,
so that the identity which they are in themselves becomes
their connection explicitly, that is, is posited.
The meaning of this development is the realization of measure
in which the latter posits itself in relation to itself
and consequently as moment as well;
through this mediation, measure is determined as sublated;
its immediacy as well as that of its moments disappears;
these moments are reflected and thus measure,
having disclosed what it is according to its concept,
has passed over into essence.

Measure is at first the immediate unity of
the qualitative and the quantitative element, so that it is,

first, a quantum that has qualitative meaning and is as measure.
As so implicitly determined in itself, its further determination is that
the difference of its moments,
of its qualitatively and quantitatively determined being,
is disclosed in it.
These moments further determine themselves into
wholes of measure which as such are self-subsistent,
and, inasmuch as they refer to each other essentially,
measure becomes,

second, a ratio of specific quanta, each an independent measure.
But their self-subsistence also rests essentially
on a quantitative relation and a difference of magnitude,
and so the self-subsistence becomes a transition
of one measure into another.
The result is that measure collapses into the measureless.
But this beyond of measure is the negativity of measure
only in itself; thus,

third, the indifference of the determinations of
measure is thereby posited, and measure
(real measure because of the negativity contained within it)
is posited as an inverse ratio of measures
which, as self-subsistent qualities,
essentially rest on only their quantity
and their negative reference to each other,
and consequently prove to be only moments of
their truly self-subsistent unity.
This unity is the reflection-into-itself of each
and the positing of them; it is essence.

The development of measure,
which we have attempted in what follows,
is among the most difficult of subject matters.
Starting with immediate, external measure,
it would have to proceed, on the one hand,
to the further abstract determination of
the quantitative aspect of natural things
(of a mathematics of nature);
on the other side, it would have to indicate the link
between this determination of measure
and the qualitative aspect of those things,
at least in general, for the detailed demonstration of the
link between the qualitative and the quantitative aspects
as they originate in the concept of a concrete object
belongs to the particular science of the concrete
(examples of which, concerning the law of falling bodies
and the free movement of the heavens,
will be found in the Encyclopedia of the Philosophical Sciences).
We may remark quite in general in this connection
that the different forms in which measure is realized
also belong to different spheres of natural reality.
The complete, abstract indifference of developed measure,
that is, of its laws, can only be found in the sphere of mechanism
where concrete corporeity is only abstract matter itself;
the qualitative differences of this matter are
of an essentially quantitative nature;
space and time are nothing but pure externalities,
and the aggregates of matters, the masses, the intensity of weight,
are determinations which are just as external
and have their proper determinateness in the quantitative element.
On the other hand, in physical things but even more so in the organic,
this quantitative determinateness of abstract materiality is
already disturbed by the multiplicity
and consequently the conflict of qualities.
And the thus ensuing conflict is not just one of qualities as such,
but measure itself is subordinated here to higher relations
and its immanent development is reduced rather to
the simple form of immediate measure.
The limbs of the animal organism have a measure
which, as a simple quantum, stands
in a ratio to the other quanta of the other limbs;
the proportions of the human body are the fixed ratios of such quanta,
and the science of nature still has far to go in
discovering anything about the link that connects
these magnitudes with the organic functions
on which they are entirely dependent.
But the closest example of the reduction of an immanent
measure to a merely externally determined magnitude is motion.
In the heavenly bodies, motion is free motion,
one which is only determined by the concept from which alone,
consequently, its magnitudes equally depend (see above);
but in the organic body this free motion is reduced
to one which is arbitrary or mechanically regular,
that is, to one which is totally abstract and formal.

And in the realm of spirit there is even less of
a characteristic, free development of measure to be found.
For instance, it is obvious that a
republican constitution like the Athenian,
or an aristocratic constitution mixed with democracy,
is possible only in a state of a certain size;
it is also obvious that in civil society
the multitudes of individuals who belong to
the different occupations stand in a certain ratio to each other.
But none of this yields either laws of measures or typical forms of it.
In the spiritual realm as such there are indeed
distinctions of intensity of character,
strength of imagination, sensations, representations, and so on;
but in determining them one cannot go past this indefinite duo of
“strength” and “weakness.”
To see how lame and totally empty ultimately turn out to be
the so-called laws which have been established for
the relation of strength and weakness in sensations,
representations, and so on, one need only look at
the psychologies that busy themselves with just such matters.

CHAPTER 1

Specific quantity

Qualitative quantity is,

first, an immediate, specific quantum; and this quantum,

second, in relating itself to another,
becomes a quantitative specifying,
a sublating of the indifferent quantum.
This measure is to this extent a rule
and contains the two moments of measure as different;
namely, the quantitative determinateness
and the external quantum as existing in themselves.
In this difference, however, the two sides become qualities,
and the rule becomes a relation of the two;
measure presents itself thereby,

third, as a relation of qualities that have one single measure at first;
a measure, however, which further specifies itself in itself
into a difference of measures.

A. THE SPECIFIC QUANTUM

1. Measure is the simple self-reference of quantum,
its own determinateness in itself;
quantum is thus qualitative.

At first, as an immediate measure it is
an immediate quantum and hence some specific quantum;
equally immediate is the quality that belongs to it;
it is some specific quality or other.
Thus quantum, as this no longer indifferent limit
but as self-referring externality, is itself quality
and, although distinguished from it,
it does not extend past it,
just as quality does not extend past quantum.
Quantum is thus the determinateness
that has returned into simple self-equality,
which is at one with determinate existence
just as determinate existence is at one with it.

If a proposition is to be made of the determination just obtained,
it could be expressed thus:
“Whatever is, has a measure.”
Every existence has a magnitude,
and this magnitude belongs to the very nature of a something;
it constitutes its determinate nature and its in-itself.
The something is not indifferent to this magnitude,
as if, were the latter to alter, it would remain the same;
rather the alteration of the magnitude alters its quality.
As measure the quantum has ceased to be a limit which is none;
it is from now on the determination of a thing, so that,
were the latter to exceed or fall short of this quantum,
it would perish.

A measure, in the usual sense of a standard,
is a quantum which is arbitrarily assumed as
the unit determinate in itself as against an external amount.
Of course, such a unit can in fact also be determinate in itself,
like a foot or some such other original measure;
to the extent, however, that it is used as
the measuring standard for other things,
it is with respect to them only an external measure,
not their original measure.
Thus the diameter of the earth or the length of a pendulum
may be taken as a specific quantum on their own account.
But the choice of a fraction of the
earth's diameter
or of the pendulum's length,
and this last under which degree of latitude,
for use as a standard of measure is arbitrary.
And for other things such a standard is
something even more external.
These have further specified
the universal specific quantum in some particular way,
and they have thereby been made into particular things.
It is therefore foolish to speak of a natural standard of things.
Anyway, a universal standard is meant for use
only as an external comparison of things,
and in this superficial sense of universal standard
it is quite a matter of indifference
what is used for the purpose.
It is not meant to be a fundamental measure
in the sense that in it the natural measures
of particular things would be displayed and recognized,
according to a rule, as the specifications of one universal measure,
the measure of the things's universal body.
But without this meaning the sole interest and significance
of an absolute standard of measure is that of something common,
and any such standard is a universal
not in itself but only by convention.

Immediate measure is a simple determination of magnitude as,
for example, the size of organic beings, of their limbs, and so forth.
But any concrete existent has the size required for being what it is,
and for having existence in the first place.
As a quantum, the existent is an indifferent magnitude
open to external determination
and capable of fluctuating increases and decreases.
As a measure, however, it is at the same time
distinct from itself as quantum,
itself as such an indifferent determination,
and is a restriction on that indifferent fluctuation of a limit.
Since in the existence of anything
the quantitative determinateness is thus twofold,
in the sense that quality is tied to it
and yet the quantity can fluctuate without prejudice to quality
so the demise of anything that has a measure occurs
through the alteration of its quantum.
On the one hand, the demise appears unexpected,
inasmuch as there can be alteration
in the quantum without the measure
and the quality being altered;
but, on the other hand, it is made into something quite simple
to grasp by means namely of the concept of gradualness.
It is easy to turn to this category for visualizing
or “explaining” the disappearance of a quality or of a something,
for it gives the impression that one can witness this disappearance
as if before one's eyes:
since the quantum is posited as the external limit
which is by nature alterable,
the alteration (of quantum only) then follows by itself.
But in fact nothing is thereby explained,
for the alteration is at the same time essentially
the transition of one quality into another,
or the more abstract transition of one existence into a non-existence,
and therein lies another determination than just gradualness,
which is only a decrease or increase,
and the one-sided holding fast to magnitude.

2. The ancients had already taken notice of this coincidence,
that an alteration which appears to be only quantitative
suddenly changes into a qualitative one,
and they used popular examples to illustrate
the inconsistencies that arise when such
a coincidence is not understood.
Two such examples go under the familiar names
of “the bald” and “the heap.”
They are elenchi, that is, according to Aristotle's explanation,
two ways in which one is compelled to say
the opposite of what one has previously asserted.
The question was put:
does the plucking of one hair from someone's head
or from a horse's tail produce baldness,
or does a heap cease to be a heap
if one grain is removed?
The expected answer can safely be conceded,
for the removal amounts to a merely quantitative difference,
and an insignificant one at that.
And so one hair is removed, one grain,
and this is repeated with only one hair and one grain
being removed each time the answer is conceded.
At last the qualitative alteration is revealed:
the head or the tail is bald; the heap has vanished.
In conceding the answer, it was not only
the repetition that was each time forgotten,
but also that the individually insignificant quantities
(like the individually insignificant disbursements from a patrimony)
add up, and the sum constitutes the qualitative whole,
so that at the end this whole has vanished:
the head is bald, the purse is empty.

The embarrassment, the contradiction, produced by the result,
is not anything sophistic in the usual sense of the word,
as if the contradiction were a pretense.
The mistake is committed by the assumed interlocutor
(that is, our ordinary consciousness),
and that is of assuming a quantity to be only an indifferent limit,
that is, of taking it in the narrowly defined sense of a quantity.
But this assumption is confounded by the truth to which it is brought,
namely that quantity is a moment of measure
and is linked to quality;
refuted is the one-sided stubborn adherence to the
abstract determinateness of quantum.
Also those elenchi are, therefore,
not anything frivolous or pedantic but basically correct:
they attest to a mind which has an interest
in the phenomena that come with thinking.
Quantum, when it is taken as indifferent limit,
is the side from which an existence is
unsuspectedly attacked and laid low.
It is the cunning of the concept
that it would seize on an existence from this side
where its quality does not seem to come into play;
and it does it so well that the aggrandizement
of a State or of a patrimony, etc.,
which will bring about the misfortune of the State or the owner,
even appears at first to be their good fortune.

3. Measure is in its immediacy an ordinary quality
of a specific magnitude appropriate to it.
Now there is also the distinction between the side
according to which quantum is an indifferent limit
that can fluctuate without the quality altering
and the other side according to which it is
qualitative and specific.
Both sides are the magnitude determinations of
one and the same thing;
but because of the original immediacy of measure,
this distinction too is to be taken as immediate,
and accordingly the two sides each also have
a diverse concrete existence.
The concrete existence of measure,
which is the side of magnitude determinate in itself,
then behaves towards the concrete existence
of the alterable external side by sublating
the indifference of the latter;
this is a specifying of measure.

B. SPECIFYING MEASURE

First, this measure is a rule,
a measure external to the mere quantum.

Second, it is a specific quantity
determining the external quantum.

Third, the two sides, both as qualities
of specific quantitative determinacy,
relate to one another as one measure.

a. The rule

The rule, or the standard which we have just mentioned,
is in the first instance as a magnitude
which is determinate in itself
and is a unit with respect to a quantum
of a particular concrete existence:
this is a quantum with a concrete existence
which is other than the something of the rule,
is measured by the latter, that is,
is determined as the amount of the said unit.
This comparison is an external act,
and the unit itself is an arbitrary magnitude
which can in turn be posited as an amount
(the foot as an amount of inches).
But measure is not only an external rule;
as a specific measure its intrinsic nature is
that it relates to its other which is a quantum.

b. Specifying measure

Measure is a specific determining of the external magnitude, that is,
of the indifferent magnitude which is now posited
in the measuring something by some other concrete existence in general.
The something of the measure is indeed itself a quantum,
but with the difference that it is the qualitative side
determining the merely indifferent, external quantum.
It has intrinsically this side of being-for-other
to which the fluctuation in size belongs.
The immanent measuring is a quality of the something,
and this something is confronted by the same quality in another something;
in the latter, however, the quality is at first relative,
with a measureless quantum in general as against
the something determined as measuring.
Inasmuch as a something has an internal measure,
an alteration of the magnitude of its quality comes to it from outside,
and the something does not take on
the arithmetical aggregate of the alteration.
Its measure reacts against it,
behaves towards the aggregate as an intensive measure
and assimilates it in a way typically its own;
it alters the externally imposed alteration,
makes something else out of this quantum
and demonstrates through this specifying function
that in this externality it is for-itself.
This specifically assimilated aggregate is itself
a quantum which is also dependent on the other, that is,
on the other aggregate which is only external to it.
The specified aggregate is therefore also alterable,
but is not for that reason a quantum as such
but the external quantum as specified in a constant manner.
Measure thus has its determinate existence as a ratio,
and its specificity is in general the exponent of this ratio.

In intensive and extensive quantum,
as we saw when considering these determinations,
it is the same quantum which is present,
once in the form of intensity
and again in the form of extension.
In this difference the underlying quantum
does not suffer any alteration;
the difference is only an external form.
In the specifying measure, on the contrary,
the quantum is taken in one instance in its immediate magnitude,
but through the exponent of the ratio
it is taken in a second instance in another amount.

The exponent which constitutes the element of specificity can appear
at first to be a fixed quantum, as a quotient of the ratio between the
external and the qualitatively determined quantum.
But it would then be nothing more than an external quantum
whereas by the exponent we are to understand here nothing
but the qualitative moment itself that specifies the quantum as such.
The only strictly immanent qualitative determination of
the quantum is (as we saw earlier) that of the exponent,
and it must be such an exponential determination
which now constitutes the ratio and which,
as the internally existent determination,
comes to confront the quantum as externally constituted.
The principle of this quantum is the numerical one
which constitutes its internal determinateness:
the mode of connection of this numerical one is external,
and the alteration which is determined only
through the nature of the immediate quantum as such consists
essentially in the addition of one such numerical one
and then another and so forth.
As the external quantum alters in arithmetical progression in this way,
the specifying reaction of the qualitative nature produces another series
which refers to the first, increases and decreases with it,
not however in a ratio determined by a numerical exponent
but in a ratio which is numerically incommensurable,
in the manner of a power determination.

c. Relation of the two sides as qualities

1. The qualitative side of the quantum, in itself determined,
exists only as a reference to the external quantitative side;
as specifying the latter, it is a sublating of its externality
through which quantum as such is.
This qualitative side thus has a quantum
for the presupposition from which it starts.
But this quantum is also qualitatively distinguished from quality,
and this difference between the two must now be posited in the immediacy
of being in general which still characterizes measure.
The two sides thus stand to each other in a qualitative respect,
each a qualitative existence for itself,
and the one side that was at first only an internally indeterminate
formal quantum is the quantum of a something and of its quality,
and, just as their reciprocal reference is
now determined as measure in general,
so too is the specific magnitude of these qualities.
These qualities stand in relation to each other according
to a determination of measure.
This determination is their exponent,
but they are already implicitly connected
to each other in the being-for-itself of measure:
the quantum is in its double being
external quantum and specific quantum,
so that each of the distinct quantities has
this double determination in it
and is at the same time inextricably interwoven with the other;
it is in this way alone that the qualities are determined.
They are not, therefore, a determinate being in general
existing for each other but are rather posited as indivisible,
and the specific magnitude tied to them is a qualitative unity:
one determination of measure in which they are
implicitly bound together in accordance with their concept.
Measure is thus the immanent quantitative relating
of two qualities to each other.

2. In measure we have the essential determination of variable magnitude,
for measure is sublated quantum;
quantum, therefore, no longer as that
which it ought to be in order to be quantum,
but quantum as quantum and as something other besides.
This other is the qualitative element
and, as we have established, is nothing else
than the relation of powers of the quantum.
In immediate measure, this variability is not yet posited;
it is just one single quantum or other to which a quality is attached.
In the specifying of measure (the preceding determination),
which is an alteration of the merely external quantum
by the qualitative moment,
what is posited is the distinctness of two determinate magnitudes
and hence in general the plurality of measures
in one common, external quantum.
And it is in this differentiation of the quantum from itself
that the latter shows itself for the first time to be a real measure,
for it appears as a being which is one and the same
(e.g. the constant temperature of the medium)
and at the same time of diversified and indeed quantitative existence
(in the various temperatures of the bodies found in the medium).
This differentiation of quantum into the diverse qualities
(the different bodies) yields a further form of measure,
one in which the two sides relate to each other
as qualitatively determined quanta,
and this can be called the realized measure.

Magnitude, simply as such, is alterable, for its determinateness is
a limit which at the same time is none;
the alteration only affects, therefore,
a particular quantum in place of which another is posited.
But the genuine alteration is that of the quantum as such.
Here we have the determination, interesting when so understood,
of the variable magnitude of higher mathematics
in which we neither need to stop short at the formal deter-
mination of alteration or variability in general
nor introduce any other determination except
the simple determination of the concept by which
the other of the quantum is only the qualitative.
The genuine determination, therefore, of real variable magnitude is
that it is qualitative, that is, as we have sufficiently shown,
that it is determined by a ratio of powers.
Posited in this variable magnitude is
that quantum has no value as such but only
as determined in conformity with its other, that is, qualitatively.
The two sides in this relating have,
in keeping with their abstract side as qualitites in general,
some particular meaning or other, for instance, space and time.
Taken at first simply as determinacies of magnitude
in their ratio of measure,
one of them is the amount which increases and decreases
in external arithmetical progression;
the other is an amount specifically determined by the other amount,
which for it is the unit.
If each of these two sides were only a particular quality in general,
there would be no way of distinguishing which of them is
to be taken with respect to their determination of magnitude
as merely externally quantitative
and which as varying in quantitative specification.
If they are related, for instance, as root and square,
it is indifferent which is regarded as increasing or decreasing
in merely external arithmetical progression,
and which on the contrary has
its specific determination in this quantum.

But the two qualities are not indeterminate in their difference,
for they are the moments of measure
and the qualification of the latter ought to rest on them.
The closest determinateness of the qualitites themselves is,
of the one, the extensive, that it is an externality within;
and of the other, the intensive, that it exists in itself
or is the negative as against the other.
Accordingly, amount is the quantitative moment
that pertains to the former,
and unit the one that pertains to the latter;
in simple direct ratio, the former is to be taken
as the dividend and the latter as the divisor;
in specifying ratio, the former as the power
or the becoming-other and the latter as the root.
Inasmuch as we still count here, that is,
we reflect on the external quantum
(which is thus the totally accidental determinateness of
magnitude which we call empirical),
and accordingly also equally take the alteration
as advancing in external arithmetical progression,
then this falls on the side of the unit, the intensive quality;
the external extensive side, by contrast,
is to be represented as altering in the specified series.
But the direct ratio (like velocity as such, st ) is reduced here
to a formal determination which has no concrete existence
but belongs rather only to the abstraction of reflection;
and even though in the ratio of root and square (as in s = at 2 ),
the root is to be taken as an empirical quantum
varying in an arithmetical progression,
the other side as specified instead,
the higher realization of
the qualification of the quantitative moment,
one which would be more in keeping with the concept, is this:
that both sides are related in higher determinations of powers
(as in s^3 = at^2 ).

C. THE BEING-FOR-ITSELF IN MEASURE

1. In the form of specified measure just considered,
the quantitative moment of each side is qualitatively determined
(both in the ratio of powers);
they are thus moments of
one measure-determinateness of qualitative nature.
Here, however, the qualities are
still posited immediately, only as diverse;
they are not related in the manner
of their quantitative determinacies,
that is, that outside their relation
they would have neither meaning nor existence,
as is the case for the quantitative determinacies
as a ratio of powers.
The qualitative moment thus disguises itself as specifying,
not itself, but the determinateness of magnitude.
Only within the latter is it posited;
for itself it is instead immediate quality as such
which, besides the fact that it
posits the magnitude as non-indifferent,
and besides its connection with its other,
still has existence subsisting on its own.
Thus space and time, outside that specification
which their quantitative determinateness obtains
in the motion of falling bodies
or in the absolutely free motion,
both have the value of space in general and time in general,
space subsisting on its own outside and without the duration of time,
and time flowing on its own independently of space.

This immediacy of the qualitative moment as against
its specific measure-relation is, however,
equally bound up with a quantitative immediacy
and with the indifference to this same measure-relation
of the quantitative element in it;
the immediate quality also has only an immediate quantum.
For that reason, the specific measure also has
a side of an at first external alteration
which advances in merely arithmetical progression undisturbed by it
and in which falls the external and hence merely empirical
determination of magnitude.
Quality and quantum, even as they extend outside measure,
are at the same time connected with it;
the immediacy is a moment of their belonging to measure.
Thus the immediate qualities also belong to measure,
are likewise connected and stand in a ratio
which, outside the specified one of power determination,
is itself only a direct ratio and an immediate measure.
This conclusion and its consequences must now be indicated.

2. Although the immediately determined quantum is as
a moment of measure otherwise implicitly grounded in a conceptual nexus,
in connection with specific measure it is, as such, externally given.
But the immediacy which is thereby posited is
the negation of the determination of qualitative measure;
the same immediacy was shown just now
on the sides of the determination of measure
which appeared for that reason to be self-subsisting qualities.
Such a negation and the return to immediate
quantitative determinateness are present
in the qualitatively determined relation
because a relation of distinct terms
entails as such the connection of such terms
in one determinateness,
and here, in the quantitative sphere,
in distinction from the determination of the relation,
this determinateness is a quantum.
As the negation of the distinct, qualitatively determined sides,
this exponent is a being-for-itself,
an absolutely determined being,
but is so only in itself;
as existence, it is a simple immediate quantum,
the quotient or exponent of a ratio between the sides of measure
which is taken as direct;
as such, however, it appears in the quantitative side
of measure as an empirical unit.
In the motion of falling bodies,
the spaces traversed are proportional
to the square of the elapsed time, s = at^2.
This is a specifically determinate ratio,
one between space and time;
the other, the direct ratio, would pertain to space and time
as mutually indifferent qualities;
it is supposed to be the ratio of the space traversed
to the first moment of time.
The same coefficient, a, remains in all the succeeding time-points;
the unit of the amount, determined for its part by the specifying measure,
being an ordinary quantum.
This unit counts at the same time as
the exponent of that direct ratio
which pertains to the merely imagined, the bad,
that is, the reflectively formal velocity
which is not specifically determined by the concept.
Such a velocity does not have concrete existence here,
no more than does the one previously mentioned
which is supposed to pertain to a falling body
at the end of a moment of time.
That velocity is ascribed to
the first temporal moment in the fall,
but this so-called temporal moment is itself
only an assumed unit which has no existence
as such an atomic point.
The beginning of the motion
(its alleged smallness would make no difference)
is straight away a magnitude
and one which is specified by the law of falling bodies.
The said empirical quantum is attributed to the force of gravity,
and this force thus has itself no connection with the
specification at hand
(the determinateness of powers),
that is, with the determinateness characteristic of measure.
The immediate moment, that in the motion of falling bodies
the amount of some fifteen spatial units
which are assumed as feet enter into one unit of time
(call it a second, the so-called first unit),
is an immediate measure, just like the size of human limbs,
the distances and diameters of planets, etc.
The determination of such measures falls elsewhere
than within the qualitative determination of measure,
here of the law itself of falling bodies.
On what such numbers depend, these merely immediate
and therefore empirical appearances of a measure;
on this the concrete sciences have yet to give us any information.
Here we are only concerned with this conceptual determination,
namely that in any determination of measure
the empirical coefficient constitutes the being-for-itself,
but only the moment of the being-for-itself,
in so far as it is in principle and therefore as immediate.
The other moment is the development of this being-for-itself,
the specific measure-determinateness of the sides.
According to this second moment,
in the ratio expressing the motion of falling bodies
(this motion that is still half conditioned and half free)
gravity is to be regarded as a force of nature.
Its ratio is thus determined by the nature of time and space,
and the said specification, the ratio of powers, therefore falls in it;
the other moment mentioned above, the simple direct ratio,
expresses only a mechanical relation of time and space,
a reflectively formal velocity externally produced
and externally determined.

3. Measure has now taken on the determination
of being a specified quantitative relation,
one which, as qualitative, has within it the usual external quantum;
but this quantum is not a quantum in general
but essentially the determining moment
of the relation as such;
it is thus an exponent and,
because of the immediacy now of its determinateness,
an invariable exponent, consequently an exponent
of the already mentioned direct ratio
of the same qualities whose reciprocal quantitative relation is
at the same time specifically determined by the ratio.
In the example of measure that we have used,
that of the motion of falling bodies,
this direct ratio is, as it were,
anticipated and assumed as given,
but, as remarked, in this motion it still
does not have concrete existence.
However, that matter is now realized,
in that its sides are both measures,
one distinguished as immediate and external
and the other as internally specified,
while measure itself is the unity of the two
this constitutes a further determination.
As the unity of its two sides,
measure contains the relation in which the magnitudes,
by virtue of the nature of the qualities,
are posited as determined and non-indifferent;
its determinateness, therefore, is entirely immanent and self-subsisting,
and has at the same time collapsed into the being-for-itself
of an immediate quantum, the exponent of a direct ratio.
In this the self-determination of measure is negated,
for in this immediate quantum,
its other, measure has a final self-existent determinateness;
conversely, the immediate measure
which ought to be internally qualitative
assumes truly qualitative determinateness only in that other.
This negative unity is real being-for-itself,
the category of a something which is the unity of qualities
in a relation of measure, a complete self-subsistence.
The two resulting diverse relations also
immediately yield a twofold existence,
or, more precisely, that self-subsisting whole,
as a being that exists for itself, is as such in itself
a repulsion into distinct self-subsisting
somethings whose qualitative nature
and subsistence (materiality) lies
in their determinateness of measure.

CHAPTER 2

Real measure

Measure is now determined as a connection of measures
that make up the quality of distinct self-subsisting somethings,
or, in more common language, things.
The relations of measures just considered belong
to abstract qualities like space and time;
further examples of these now to be considered
are specific gravity and then chemical properties,
that is, determinations of concrete material existence.
Space and time are also moments of these measures,
but are now subordinated to other determinations
and no longer behave relative to one another
only according to their own conceptual determination.
In the case of sound, for instance,
the time within which a certain number of vibrations occur,
the spatial width and thickness of the sounding body,
are moments of its determination.
But the magnitudes of such idealized moments are externally determined;
they no longer assume the form of a ratio of powers
but relate in the usual direct way,
and harmony is reduced to the strictly
external simplicity of numbers in relations
which are most easy to grasp;
they therefore afford a satisfaction
which is the exclusive reserve of the senses,
for there is nothing there of representation,
imagery, thought, or the like,
that would satisfy spirit.
Since the sides which now constitute
the relation of measure are themselves measures,
but at the same time real somethings, their measures are,
in the first instance, immediate measures,
and the relations in them direct relations.
We now have to examine the further determination of
the relation of such relations.
Measure, now real measure, is as follows.

First, it is the independent measure of a type of body,
a measure which relates to other measures
and, in thus relating to them, specifies them as
well as their self-subsistent materiality.
This specification, as an external
connecting reference to many others in general,
produces other relations, and consequently other measures;
the specific self-subsistence, for its part,
does not remain fixed in one direct relation
but passes over into specific determinacies,
and this is a series of measures.

Second, the direct relations that thus result are in themselves
determinate and exclusive measures (elective affinities).
But because they are at the same time only
quantitatively different from one another,
what we have  is a progression of relations
which is in part merely external,
but is also interrupted by qualitative relations,
forming a nodal line of specifically self-subsisting things.

Third, what emerges in this progression for measure, however,
is the measureless:
the measureless in general and more specifically
the infinitude of measure in which the mutually exclusive
forms of self-subsistence are one with each other,
and anything self-subsistent comes to stand
in negative reference to itself.

CHAPTER 3

The becoming of essence

A. ABSOLUTE INDIFFERENCE

Being is abstract indifference,
and when this trait is to be thought by itself as being,
the abstract expression “indifferentness” has been used;
in which there is not supposed to be as yet
any kind of determinateness.
Pure quantity is this indifference in the sense of
being open to any determinations,
provided that these are external to it
and that quantity itself does not have
any link with them originating in it.
The indifference which can be called absolute,
however, is one which, through the negation of
every determinateness of being, of quality and quantity
and of their at first immediate unity, that is, of measure,
mediates itself with itself to form a simple unity.
Determinateness is in it still only a state, that is,
something qualitative and external
which has the indifference as a substrate.

But that which has thus been determined
as qualitative and external
is only a vanishing something;
as thus external with respect to being,
the qualitative sphere is the opposite of itself
and, as such, only the sublating of itself.
In this way, determinateness is still
only posited in the substrate
as an empty differentiation.
But it is precisely this empty differentiation
which is the indifference itself as result.
And this indifference is indeed concrete,
in the sense that it is self-mediated
through the negation of all the determinations of being.
As such a mediation, it contains negation and relation,
and what was called “state” is a differentiation
which is immanent to it and self-referring.
It is precisely this externality and its vanishing
which make the unity of being into an indifference:
consequently, they are inside this indifference,
which thereby ceases to be only a substrate
and, within, only abstract.

B. INDIFFERENCE AS INVERSE RATIO OF ITS FACTORS

We now have to see how this determination
of indifference is posited in the indifference itself
and the latter is posited, therefore, as existing for itself.

1. The reduction of at first
independently accepted measure-relations
establishes their one substrate;
this substrate is their continuing into one another
and is, therefore, the one indivisible independent measure
which is wholly present in its differentiation.
Present for this differentiation are
the two determinations contained
in the measure, quality and quantity,
and everything depends on how these two are posited in it.
But this is in turn determined by the fact that
the substrate is at first posited as result
and, though in itself mediation,
this mediation is not yet posited as such in it;
for this reason, it is in the first instance substrate
and, with respect to determinateness, indifference.

Consequently, the difference present in it is
at first essentially one which is only quantitative and external;
there simply are two different quanta of one and the same substrate
which would thus be their sum, itself posited as a quantum.
But the indifference is this fixed measure,
the implicitly existent absolute limit which,
as connected to those differences,
would not itself be in itself a quantum,
and would not in any way enter
into opposition with others,
whether as sum or also as exponent,
be those others sums or indifferences.
It is only the abstract determinateness
which falls into the indifference;
the two quanta, in order that
they may be posited in it as moments,
are alterable, indifferent, greater or smaller
relative to one another.
However, inasmuch as they are restricted
by the fixed limit of their sum,
they are at the same time related to
each other not externally, but negatively,
and this is now the qualitative determination
in which they stand to each other.
Accordingly, they stand in inverse ratio to each other.
This relation differs from the earlier formal inverted ratio
inasmuch as the limit is here a real substrate,
and each of the two sides is posited as
having to be in itself the whole.

According to the qualitative determinacy just stated,
the difference is present, further, in the form of two qualities,
each of which is sublated by the other
and yet, since the two are held together in the one unity
which they constitute, is inseparable from it.
The substrate itself, as the indifference,
is in itself likewise the unity of the two qualities;
consequently, each of the sides of the relation
equally contains both sides within itself
and is distinguished from the other
by a more of one quality
and a less of the other, or conversely.
The one quality, through its quantum,
only predominates on the one side,
as does the other quality on the other side.

Thus each side is in it an inverted relation
which, as formal, recurs in the two distinguished sides.
The sides themselves thus continue into each other
also according to their qualitative determinations;
each of the qualities relates itself in the other to itself
and is present in each of the two sides,
only in a different quantum.
Their quantitative difference is that indifference
in accordance with which they continue into each other,
and this continuation is the self-sameness of the qualities
in each of the two unities.
The sides, however, each containing
the whole of the determinations
and consequently the indifference itself,
are thus at the same time posited as
self-subsistent vis-a-vis each other.

2. As this indifference,
being is now the determinateness of measure
no longer in its immediacy
but in the developed manner just indicated;
it is indifference because it is in itself
the whole of the determinations of being
now resolved into this unity;
and it is existence as well,
as a totality of the posited realization,
in which the moments themselves are the totality of
the indifference existing in itself,
sustained by the latter as their unity.
But because the unity is held fast only as indifference
and consequently only implicitly in itself,
and the moments are not yet determined
as existing for themselves,
that is, are not yet determined as
sublating themselves into unity internally
and through each other,
what is here present is therefore simply
the indifference of the unity itself
towards itself as a developed determinateness.

This thus indivisible independent measure is
now to be more closely examined.
It is immanent in all its determinations
and in them it remains in unity with itself
and undisturbed by them.

But, (a) since the determinacies
sublated in it implicitly remain the totality,
they emerge in it groundlessly.
The implicit being of indifference
and its existence are thus unconnected;
the determinacies show up in the indifference in an immediate manner
and the indifference is in each of them entirely the same.
The difference between them is
thus posited at first as sublated, hence as quantitative;
for this reason, therefore, not as a self-repelling;
and the indifference not as self-determining,
but as having and becoming the determinate being
that it has only externally.

(b) The two moments are in inverse quantitative relation;
a fluctuating on the scale of magnitude
which is not however determined by the indifference,
which is precisely the indifference of the fluctuation,
but only externally.
For the determining appeal is made to an other
which lies outside the indifference.
The absolute, as indifference, has in this respect
the second defect of quantitative form,
namely that the determinateness of the difference is
not determined by the absolute itself,
just as it has the first defect in that
the differences emerge in it only in general, that is,
the positing of them is something immediate,
not a self-mediation.

(c) The quantitative determinateness of the moments which are now
sides of the relation constitutes the mode of their subsistence;
their existence is by virtue of this indifference
withdrawn from the transitoriness of quality.
But they do have a subsistence of their own in themselves,
one that differs from this quantitative existence,
for they are in themselves the indifference itself,
each the unity itself of the two qualities
into which the qualitative moment splits itself.
The difference of the two moments is restricted by
the fact that the one quality is posited
on the one side with a more
and in the other with a less,
and the other is posited
in inverse order accordingly.
Each side is thus in it the totality of the indifference.
Each of the two qualities taken singly for itself
likewise remains the same sum which the indifference is;
it continues from one side into the other
without being restricted by the quantitative limit
which is thereby posited in it.
At this, the determinations come into immediate opposition,
an opposition which develops into contradiction,
as we must now see.

3. Namely, each quality enters inside
each side in connection with the other,
and it does so in such a manner that, as has been determined,
this connection also is supposed to be only a quantitative difference.
If the two qualities are both self-subsistent
(something like sensible materials independent of each other)
then the whole determinateness of indifference falls apart;
their unity and totality would be empty names.
But they are at the same time determined as
comprised into one unity, as inseparable,
each having meaning and reality only in this one
qualitative connecting reference to the other.
But now, because their quantitativeness is simply
and solely of this qualitative nature,
 each reaches only as far as the other.
If they are assumed to differ as quanta,
then the one would reach beyond the other
and would have in this more an indifferent existence
which the other would not have.
As qualitatively connected, however,
each is only in so far as the other is.
The result is that they are in equilibrium,
so that to the extent that one increases or decreases,
the other likewise increases or decreases
and would do so in the same proportion.

On the basis, therefore, of their qualitative connection,
there is no question of a quantitative difference
or of a more of the one quality.
The more by which the one of the two connected moments
would exceed the other would be only an unstable determination,
or would only be the other itself again;
but, in this equality of the two, neither would then be there,
for their existence would have to rest on the inequality of their quantum.
Each of these supposed factors vanishes,
whether the one factor is assumed
to exceed the other or to be equal to it.
From the standpoint of quantitative representation,
the vanishing appears as a disturbance of the equilibrium,
one factor becoming greater than the other;
the sublation of the quality of the other
and its instability are thus posited;
the first factor becomes the predominant one
as the other diminishes with accelerated velocity
and is overcome by it;
this in turn constitutes itself as the one self-subsistent factor;
with this, however, there are no longer two specific moments
as factors but only the one whole.

This unity thus posited as the totality
of the process of determining,
itself determined in this process as indifference,
is a contradiction all around.
It must therefore be posited
as this self-sublating contradiction,
and be determined as subsistence existing for itself,
one which no longer has a merely indifferent unity
for result but a unity immanently negative and absolute.
This is essence.

C. TRANSITION INTO ESSENCE

Absolute indifference is the final determination of being
before the latter becomes essence;
but it does not attain essence.
It shows that it still belongs to the sphere of being
because it is still determined as indifferent,
and therefore difference is external to it, quantitative.
This externality is its existence, by which
it finds itself at the same time in the opposition of
being determined over against it as existing in itself,
not as being thought as the absolute that exists for itself.
Or again, it is external reflection
which insists that specific determinations,
whether in themselves or in the absolute,
are one and the same;
that their difference is only an indifferent one,
not a difference in itself.
What is still missing here is that
this reflection should sublate itself,
that it would cease to be the external
reflection of thought, of a subjective consciousness,
but that it would be rather the very determination of
the difference of that unity;
a unity which would then prove itself to be the absolute negativity,
the unity's indifference towards itself,
towards its own indifference no less than towards otherness.

But this self-sublation of the determination of
indifference has already manifested itself;
in the progressive positing of its being
it has shown itself on all sides to be contradiction.
Indifference is in itself the totality
in which all the determinations of being are sublated and contained;
thus it is the substrate,
but at first only in the one-sided determination of being-in-itself,
and consequently the differences,
the quantitative difference and the inverseratio of factors,
are present in it as external.
As thus the contradiction of itself and its determinateness,
of its implicitly existent determination
and of its posited determinateness,
it is the negative totality whose determinacies
have internally sublated themselves,
consequently, have also sublated the one-sidedness
of their substrate, their in-itselfness.
Indifference, now posited as what it in fact is,
is simple and infinitely negative self-reference,
the incompatibility of itself with itself,
the repelling of itself from itself.
Determining and being determined are
not a transition,
nor an external alteration,
nor again an emergence of determinations in it,
but its own referring to itself
which is the negativity of itself,
of its in-itselfness.

But as so repelled,
the determinations are not self-possessed;
do not emerge as self-subsistent or external
but are rather as moments:
first, as belonging to the unity
whose existence is still only implicit,
they are not let go by it but are rather borne by it
as their substrate and are filled by it alone;
and, second, as determinations immanent
to the unity as it exists for itself,
they are only through their repulsion from themselves.
Instead of some existent or other,
as they are in the whole sphere of being,
they now are simply and solely as posited,
with the sole determination and significance of
being referred to their unity
and hence each to the other and to negation,
marked by this their relativity.

Being in general and the being or immediacy
of the different determinacies have thereby vanished
just as much as the in-itselfness,
and the unity is being,
immediately presupposed totality,
so that it is only this simple self-reference,
mediated by the sublation of this presupposition,
and this pre-supposedness, the immediate being,
is itself only a moment of its repelling:
the original self-subsistence and self-identity are only
as the resulting infinite self-rejoining.
And so is being determined as essence:
being which, through the sublation of being,
is simple being with itself.
