BOOK TWO

The Doctrine of Essence

ESSENCE

The truth of being is essence.

Being is the immediate.
Since the goal of knowledge is the truth,
what being is in and for itself,
knowledge does not stop at
the immediate and its determinations,
but penetrates beyond it
on the presupposition that
behind this being there still is
something other than being itself,
and that this background
constitutes the truth of being.
This cognition is a mediated knowledge,
for it is not to be found
with and in essence immediately,
but starts off from an other, from being,
and has a prior way to make,
the way that leads over and beyond being
or that rather penetrates into it.
Only inasmuch as knowledge recollects
itself into itself out of immediate being,
does it find essence through this mediation.
The German language has kept “essence” (Wesen)
in the past participle (gewesen) of the verb “to be” (sein),
for essence is past [but timelessly past] being.

When this movement is represented as a pathway of knowledge,
this beginning with being and the subsequent advance
which sublates being and arrives at essence as a mediated term
appears to be an activity of cognition external to being
and indifferent to its nature.

But this course is the movement of being itself.
That it is being's nature to recollect itself,
and that it becomes essence by virtue of this interiorizing,
this has been displayed in being itself.

If, therefore, the absolute was
at first determined as being,
now it is determined as essence.
Cognition cannot in general stop
at the manifold of existence;
but neither can it stop at being, pure being;
immediately one is forced to the reflection that
this pure being, this negation of everything finite,
presupposes a recollection and a movement
which has distilled immediate existence into pure being.
Being thus comes to be determined as essence,
as a being in which everything determined and finite is negated.
So it is simple unity, void of determination,
from which the determinate has been
removed in an external manner;
to this unity the determinate was
itself something external
and, after this removal,
it still remains opposite to it;
for it has not been sublated in itself but relatively,
only with reference to this unity.
We already noted above that if pure essence is defined
as the sum total of all realities,
these realities are equally subject to
the nature of determinateness and abstractive reflection
and their sum total is reduced to empty simplicity.
Thus defined, essence is only a product, an artifact.
External reflection, which is abstraction,
only lifts the determinacies of being
out of what is left over as essence
and only deposits them, as it were,
somewhere else, letting them exist as before.
In this way, however, essence is
neither in itself nor for itself;
it is by virtue of another,
through external abstractive reflection;
and it is for another, namely for abstraction
and in general for the existent
which still remains opposite to it.
In its determination, therefore,
it is a dead and empty absence of determinateness.

As it has come to be here, however,
essence is what it is,
not through a negativity foreign to it,
but through one which is its own:
the infinite movement of being.
It is being-in-and-for-itself,
absolute in-itselfness;
since it is indifferent to
every determinateness of being,
otherness and reference to other have been sublated.
But neither is it only this in-itselfness;
as merely being-in-itself, it would be only
the abstraction of pure essence;
but it is being-for-itself just as essentially;
it is itself this negativity,
the self-sublation of otherness
and of determinateness.

Essence, as the complete turning back of being into itself,
is thus at first the indeterminate essence;
the determinacies of being are sublated in it;
it holds them in itself but without their being posited in it.
Absolute essence in this simple unity with itself has no existence.
But it must pass over into existence,
for it is being-in-and-for-itself;
that is to say, it differentiates
the determinations which it holds in itself,
and, since it is the repelling of itself from itself
or indifference towards itself, negative self-reference,
it thereby posits itself over against itself
and is infinite being-for-itself
only in so far as in thus
differentiating itself from itself
it is in unity with itself.
This determining is thus of another nature than
the determining in the sphere of being,
and the determinations of essence have another character
than the determinations of being.
Essence is absolute unity of being-in-itself and being-for-itself;
consequently, its determining remains inside this unity;
it is neither a becoming nor a passing over,
just as the determinations themselves are
neither an other as other nor references to some other;
they are self-subsisting but, as such,
at the same time conjoined in the unity of essence.
Since essence is at first simple negativity,
in order to give itself existence and then being-for-itself,
it must now posit in its sphere the determinateness
which it contains in principle only in itself.

Essence is in the whole what quality was in the sphere of being:
absolute indifference with respect to limit.
Quantity is instead this indifference in immediate determination,
limit being in it an immediate external determinateness;
quantity passes over into quantum;
the external limit is necessary to it and exists in it.
In essence, by contrast, the determinateness does not exist;
it is posited only by the essence itself,
not free but only with reference to
the unity of the essence.
The negativity of essence is reflection,
and the determinations are reflected,
posited by the essence itself
in which they remain as sublated.

Essence stands between being and concept;
it makes up their middle,
its movement constituting the transition
of being into the concept.
Essence is being-in-and-for-itself,
but it is this in the determination of being-in-itself;
for its general determination is that it emerges from being
or that it is the first negation of being.
Its movement consists in positing negation
or determination in being,
thereby giving itself existence
and becoming as infinite being-for-itself
what it is in itself.
It thus gives itself its existence
which is equal to its being-in-itself
and becomes concept.
For the concept is the absolute
as it is absolutely,
or in and for itself,
in its existence.
But the existence which essence gives to itself is
not yet existence as it is in and for itself
but as essence gives it to itself or as posited,
and hence still distinct from the existence of the concept.

First, essence shines within itself
or is reflection;
second, it appears;
third, it reveals itself.

In the course of its movement,
it posits itself in the following determinations:

I. As simple essence existing in itself,
remaining in itself in its determinations;

II. As emerging into existence,
or according to its concrete existence and appearance;

III. As essence which is one with its appearance,
as actuality.

SECTION I

Essence as Reflection Within

Essence issues from being;
hence it is not immediately in and for itself
but is a result of that movement.
Or, since essence is taken at first as something immediate,
it is a determinate existence to which another stands opposed;
it is only essential existence, as against the unessential.
But essence is being which has been sublated in and for itself;
what stands over against it is only shine.
The shine, however, is essence's own positing.

First, essence is reflection.
Reflection determines itself;
its determinations are a positedness
which is immanent reflection at the same time.
Second, these reflective determinations
or essentialities are to be considered.
Third, as the reflection of its immanent determining,
essence turns into foundation and passes over
into concrete existence and appearance.

CHAPTER 1

Shine

As it issues from being, essence seems to stand over against it;
this immediate being is, first, the unessential.

But, second, it is more than just the unessential;
it is being void of essence; it is shine.

Third, this shine is not something external,
something other than essence, but is essence's own shining.
This shining of essence within it is reflection.

A. THE ESSENTIAL AND THE UNESSENTIAL

Essence is sublated being.

It is simple equality with itself
but is such as the negation of
the sphere of being in general.
And so it has immediacy over against it,
as something from which it has come to be
but which has preserved and maintained itself in this sublating.
Essence itself is in this determination
an existent immediate essence,
and with reference to it
being is only something negative,
nothing in and for itself:
essence, therefore, is a determined negation.
Being and essence relate to each other in this fashion
as against others in general which are mutually indifferent,
for each has a being, an immediacy,
and according to this being they stand in equal value.

But as contrasted with essence,
being is at the same time the unessential;
as against essence, it has the determination of something sublated.
And in so far as it thus relates to essence
as an other only in general,
essence itself is not essence proper
but is just another existence, the essential.

The distinction of essential and unessential has
made essence relapse into the sphere of existence,
for as essence is at first,
it is determined with respect to being
as an existent and therefore as an other.
The sphere of existence is thus laid out as foundation,
and that in this sphere being is being-in-and-for-itself,
is a further determination external to existence,
just as, contrariwise, essence is indeed being-in-and-for-itself,
but only over against an other, in a determinate respect.
Consequently, inasmuch as essential and unessential aspects are
distinguished in an existence from each other,
this distinguishing is an external positing,
a taking apart that leaves the existence itself untouched;
it is a separation which falls on the side of
a third and leaves undetermined
what belongs to the essential
and what belongs to the unessential.
It is dependent on some external standpoint or consideration
and the same content can therefore sometimes be considered
as essential, sometimes as unessential.

On closer consideration, essence becomes something
only essential as contrasted with an unessential
because essence is only taken,
is as sublated being or existence.
In this fashion, essence is only the first negation,
or the negation, which is determinateness,
through which being becomes only existence,
or existence only an other.
But essence is the absolute negativity of being;
it is being itself, but not being determined only as an other:
it is being rather that has sublated itself
both as immediate being
and as immediate negation,
as the negation which is affected by an otherness.
Being or existence, therefore, does not persist
except as what essence is,
and the immediate which still differs from essence is not just
an unessential existence but an immediate
which is null in and for itself;
it only is a non-essence, shine.

B. SHINE

1. Being is shine.

The being of shine consists solely
in the sublatedness of being,
in being's nothingness;
this nothingness it has in essence,
apart from its nothingness,
apart from essence, it does not exist.
It is the negative posited as negative.

Shine is all that remains of the sphere of being.
But it still seems to have an immediate side
which is independent of essence
and to be, in general, an other of essence.
Other entails in general the two moments
of existence and non-existence.
Since the unessential no longer has a being,
what is left to it of otherness is
only the pure moment of non-existence;
shine is this immediate non-existence,
a non-existence in the determinateness of being,
so that it has existence only with reference to another,
in its non-existence;
it is the non-self-subsistent
which exists only in its negation.
What is left over to it is thus only
the pure determinateness of immediacy;
it is as reflected immediacy, that is,
one which is only by virtue of
the mediation of its negation
and which, over against this mediation, is
nothing except the empty determination
of the immediacy of non-existence.

Shine, the “phenomenon” of skepticism,
and also the “appearance” of idealism,
is thus this immediacy which is not
a something nor a thing in general,
not an indifferent being that would exist apart
from its determinateness and connection with the subject.
Skepticism did not permit itself to say “It is,”
and the more recent idealism did not permit itself
to regard cognitions as a knowledge of the thing-in-itself.
The shine of the former was supposed absolutely
not to have the foundation of a being:
the thing-in-itself was not supposed
to enter into these cognitions.
But at the same time skepticism allowed
a manifold of determinations for its shine,
or rather the latter turned out to have
the full richness of the world for its content.
Likewise for the appearance of idealism:
it encompassed the full range of these manifold determinacies.
So, the shine of skepticism and the appearance of idealism
do immediately have a manifold of determination.
This content, therefore, might well have no being as foundation,
no thing or thing-in-itself;
for itself, it remains as it is;
it is simply transposed from being into shine,
so that the latter has within itself those manifold
determinacies that exist immediately,
each an other to the other.
The shine is thus itself something immediately determined.
It can have this or that content;
but whatever content it has, it has not posited it
but possesses it immediately.
Idealism, whether Leibnizian, Kantian, Fichtean, or in any
other form, has not gone further than skepticism in this:
it has not advanced beyond being as determinateness.
Skepticism lets the content of its shine to be given to it;
the shine exists for it immediately,
whatever content it might have.
The Leibnizian monad develops its representations from itself
but is not their generating and controlling force;
they rise up in it as a froth, indifferent,
immediately present to each other and to the monad as well.
Likewise Kant's appearance is a given content of perception
that presupposes affections, determinations of the subject
which are immediate to each other and to the subject.
As for the infinite obstacle of Fichte's Idealism,
it might well be that it has no thing-in-itself for foundation,
so that it becomes a determinateness purely within the “I.”
But this determinateness that the “I” makes its own,
sublating its externality,
is to the “I” at the same time an immediate determinateness,
a limitation of the “I” which the latter may transcend
but which contains a side of indifference,
and on account of this indifference,
although internal to the “I,”
it entails an immediate non-being of it.

2. Shine thus contains an immediate presupposition,
an independent side vis-à-vis essence.
But the task, inasmuch as this shine is distinct from essence,
is not to demonstrate that it sublates itself
and returns into essence,
for being has returned into essence in its totality;
shine is the null as such.
The task is to demonstrate that the determinations which
distinguish it from essence are the determinations of essence itself;
further, that this determinateness of essence,
which shine is, is sublated in essence itself.

What constitutes the shine is
the immediacy of non-being;
this non-being, however, is nothing else than
the negativity of essence within essence itself.
In essence, being is non-being.
Its inherent nothingness is the
negative nature of essence itself.
But the immediacy or indifference
which this non-being contains is
essences's own absolute in-itself.
The negativity of essence is its self-equality
or its simple immediacy and indifference.
Being has preserved itself in essence inasmuch
as this latter, in its infinite negativity,
has this equality with itself;
it is through this that essence is itself being.
The immediacy that the determinateness has
in shine against essence is
thus none other than essence's own immediacy,
though not the immediacy of an existent
but rather the absolutely mediated
or reflective immediacy which is shine;
being, not as being, but only as
the determinateness of being as against mediation;
being as moment.

These two moments [nothingness but as subsisting],
and being but as moment;
or again, negativity existing in itself and reflected immediacy,
these two moments that are the moments of shine,
are thus the moments of essence itself;
it is not that there is a shine of being in essence,
or a shine of essence in being:
the shine in the essence is not the shine of an other
but is rather shine as such, the shine of essence itself.

Shine is essence itself in the determinateness of being.
Essence has a shine because it is determined within itself
and is therefore distinguished from its absolute unity.
But this determinateness is as determinateness
just as absolutely sublated in it.
For essence is what stands on its own:
it exists as self-mediating through a negation which it itself is.
It is, therefore, the identical unit of absolute negativity and immediacy.
The negativity is negativity in itself;
it is its reference to itself and thus immediacy in itself.
But it is negative reference to itself,
a self-repelling negating;
thus the immediacy existing in itself is
the negative or the determinate over against the negativity.
But this determinateness is itself absolute negativity
and this determining, which as determining immediately sublates itself,
is a turning back into itself.

Shine is the negative which has a being,
but in another, in its negation;
it is a non-self-subsisting-being
which is sublated within and null.
And so it is the negative which returns into itself,
the non-subsistent as such, internally non-subsistent.
This reference of the negative or
the non-subsistent to itself is
the immediacy of this non-subsistent;
it is an other than it;
it is its determinateness over against it,
or the negation over against the negative.
But this negation which stands over against the negative is
negativity as referring solely to itself,
the absolute sublation of the determinateness itself.

The determinateness that shine is in essence is,
therefore, infinite determinateness;
it is only the negative which coincides with itself
and hence a determinateness that, as determinateness,
is self-subsistence and not determined.
Contrariwise, the self-subsistence, as self-referring immediacy,
equally is just determinateness and moment,
negativity solely referring to itself.
This negativity which is identical with immediacy,
and thus the immediacy which is identical with negativity, is essence.
Shine is, therefore, essence itself,
but essence in a determinateness, in such a way, however,
that the determinateness is only a moment,
and the essence is the shining of itself within itself.

In the sphere of being, non-being arises over against being,
each equally an immediate, and the truth of both is becoming.
In the sphere of essence, we have the contrast
first of essence and the non-essential,
then of essence and shine,
the non-essential and the shine
being both the leftover of being.
But these two, and no less the
distinction of essence from them,
consist solely in this:
that essence is taken at first as an immediate,
not as it is in itself,
namely as an immediacy which is immediacy
as pure mediacy or absolute negativity.
This first immediacy is thus only the determinateness of immediacy.
The sublating of this determinateness of essence consists, therefore,
in nothing further than showing that the unessential is only shine,
and that essence rather contains this shine within itself.
For essence is an infinite self-contained movement
which determines its immediacy as negativity
and its negativity as immediacy,
and is thus the shining of itself within itself.
In this, in its self-movement,
essence is reflection.

C. REFLECTION

Shine is the same as what reflection is;
but it is reflection as immediate.
For this shine which is internalized
and therefore alienated from its immediacy,
the German has a word from an alien language, “Reflexion.”

Essence is reflection, the movement of
becoming and transition that remains within itself,
wherein that which is distinguished is determined
simply and solely as the negative in itself, as shine.
In the becoming of being, it is being which lies
at the foundation of determinateness,
and determinateness is reference to an other.
Reflective movement is by contrast
the other as negation in itself,
a negation which has being only as self-referring.
Or, since this self-referring is
precisely this negating of negation,
what we have is negation as negation,
negation that has its being in its being-negated, as shine.
Here, therefore, the other is not
being with negation or limit,
but negation with negation.
But the first over against this other,
the immediate or being,
is only this self-equality itself of negation,
the negated negation, the absolute negativity.
This self-equality or immediacy, therefore, is
not a first from which the beginning is made
and which would pass over into its negation;
nor is there an existent substrate which would
go through the moves of reflection;
immediacy is rather just this movement itself.

In essence, therefore, the becoming,
the reflective movement of essence,
is the movement from nothing to nothing
and thereby back to itself.
Transition or becoming sublates itself in its transition;
the other which comes to be in this transition is
not the non-being of a being but the nothingness of a nothingness,
and this, to be the negation of a nothingness, constitutes being.
Being is only as the movement of nothingness to nothingness,
and so it is essence;
and this essence does not have this movement in itself,
but the movement is rather the absolute shine itself,
the pure negativity which has nothing outside it
which it would negate but which rather negates only its negative,
the negative which is only in this negating.
This pure absolute reflection, which is the movement
from nothing to nothing, further determines itself.

It is, first, positing reflection.

Second, it takes as its starting point
the presupposed immediate,
and then it is external reflection.

Third, it sublates however this presupposition,
and because in the sublating of the presupposition
it presupposes at the same time,
it is determining reflection.

1. Positing reflection

Shine is a nothingness or a lack of essence.
But a nothingness or that which is void of essence
does not have its being in an other in which it shines,
but its being is its own equality with itself;
this conversion of the negative with itself has been determined
as the absolute reflection of essence.

This self-referring negativity is
therefore the negating of itself.
It is thus just as much
sublated negativity as it is negativity.
Or again, it is itself the negative
and the simple equality with itself or immediacy.
It consists, therefore, in being itself
and not being itself,
and the two in one unity.

Reflection is at first the movement of
the nothing to the nothing,
and thus negation coinciding with itself.
This self-coinciding is in general
simple equality with itself, immediacy.
But this falling together is not
the transition of negation into equality
as into a being other than it;
reflection is transition rather
as the sublating of transition,
for it is the immediate falling together
of the negative with itself.
And so this coinciding is, first,
self-equality or immediacy;
but, second, this immediacy is
the self-equality of the negative,
and hence self-negating equality,
immediacy which is in itself the negative,
the negative of itself:
its being is to be what it is not.

The self-reference of the negative is
therefore its turning back into itself;
it is immediacy as the sublating of the negative,
but immediacy simply and solely as this reference
or as turning back from a one,
and hence as self-sublating immediacy.
This is positedness,
immediacy purely as determinateness
or as self-reflecting.
This immediacy, which is only as
the turning back of the negative into itself,
is the immediacy which constitutes the determinateness of shine,
and from which the previous reflective movement seemed to begin.
But, far from being able to begin with this immediacy,
the latter first is rather as the turning back
or as the reflection itself.
Reflection is therefore the movement which,
since it is the turning back,
only in this turning is that
which starts out or returns.

It is a positing, inasmuch as it is
immediacy as a turning back;
that is to say, there is not an other beforehand,
one either from which or to which it would turn back;
it is, therefore, only as a turning back
or as the negative of itself.
But further, this immediacy is sublated negation
and sublated return into itself.
Reflection, as the sublating of the negative, is
the sublating of its other, of the immediacy.
Because it is thus immediacy as a turning back,
the coinciding of the negative with itself,
it is equally the negation of the negative as negative.
And so it is presupposing.
Or immediacy is as a turning back
only the negative of itself,
just this, not to be immediacy;
but reflection is the sublating
of the negative of itself,
coincidence with itself;
it therefore sublates its positing,
and inasmuch as it is in its positing
the sublating of positing, it is presupposing.
In presupposing, reflection determines the turning back
into itself as the negative of itself,
as that of which essence is the sublating.
It is its relating to itself,
but to itself as to the negative of itself;
only so is it negativity which abides with itself,
self-referring negativity.
Immediacy comes on the scene simply and solely
as a turning back and is that negative
which is the semblance of a beginning,
the beginning which the return negates.
The turning back of essence is therefore its self-repulsion.
Or inner directed reflection is essentially
the presupposing of that from which
the reflection is the turning back.

It is only by virtue of the sublating of
its equality with itself
that essence is equality with itself.
Essence presupposes itself,
and the sublating of this presupposing is essence itself;
contrariwise, this sublating of its presupposition is
the presupposition itself.
Reflection thus finds an immediate before it
which it transcends and from which it is the turning back.
But this turning back is only the presupposing of
what was antecedently found.
This antecedent comes to be only by being left behind;
its immediacy is sublated immediacy.
The sublated immediacy is, contrariwise, the turning
back into itself,
essence that arrives at itself,
simple being equal to itself.
This arriving at itself is thus
the sublating of itself
and self-repelling, presupposing reflection,
and its repelling of itself from itself is
the arriving at itself.

It follows from these considerations that
the movement of reflection is to be taken as
an absolute internal counter-repelling.
For the presupposition of
the turning back into itself
[that from which essence arises,
essence being only as this coming back]
is only in the turning back itself.
Transcending the immediate from which reflection begins
occurs rather only through this transcending;
and the transcending of the immediate is
the arriving at the immediate.
The movement, as forward movement, turns immediately
around into itself and so is only self-movement:
a movement which comes from itself in so far as
positing reflection is presupposing reflection, yet,
as presupposing reflection, is simply positing reflection.

Thus is reflection itself and its non-being,
and only is itself by being the negative of itself,
for only in this way is the sublating of the negative
at the same time a coinciding with itself.

The immediacy which reflection,
as a process of sublating,
presupposes for itself is
simply and solely a positedness,
something in itself sublated
which is not diverse from
reflection's turning back into itself
but is itself only this turning back.
But it is at the same time determined as a negative,
as immediately in opposition to something,
and hence to an other.
And so is reflection determined.
According to this determinateness,
because reflection has a presupposition
and takes its start from the immediate as its other,
it is external reflection.

2. External reflection

Reflection, as absolute reflection,
is essence shining within,
essence that posits only shine,
only positedness, for its presupposition;
and as presupposing reflection,
it is immediately only positing reflection.
But external or real reflection
presupposes itself as sublated,
as the negative of itself.
In this determination, it is doubled.
At one time it is as what is presupposed,
or the reflection into itself which is the immediate.
At another time, it is as the reflection
negatively referring to itself;
it refers itself to itself as
to that its non-being.

External reflection thus presupposes a being,
at first not in the sense that
its immediacy is only positedness or moment,
but in the sense rather that
this immediacy refers to itself
and the determinateness is only as moment.
Reflection refers to its presupposition in such a way
that the latter is its negative,
but this negative is thereby sublated as negative.
Reflection, in positing, immediately sublates its positing,
and so it has an immediate presupposition.
It therefore finds this presupposition before it
as something from which it starts,
and from which it only makes its way back into itself,
negating it as its negative.
But that this presupposition is a negative
or a positedness is not its concern;
this determinateness belongs only to positing reflection,
whereas in the presupposing positedness
it is only as sublated.
What external reflection determines and posits in the immediate
are determinations which to that extent are external to it.
In the sphere of being, external reflection was the infinite;
the finite stands as the first,
as the real from which the beginning is made
as from a foundation that abides,
whereas the infinite is the reflection into itself
standing over against it.

This external reflection is the syllogism
in which the two extremes are
the immediate and the reflection into itself;
the middle term is the reference connecting the two,
the determinate immediate, so that one part of this connecting reference,
the immediate, falls to one extreme alone, and the other,
the determinateness or the negation, only to the other extreme.

But if one takes a closer look at what the external reflection does,
it turns out that it is, secondly, the positing of the immediate,
an immediate which thus becomes the negative or the determined;
but it is immediately also the sublating of this positing,
for it presupposes the immediate;
in negating, it is the negating of its negating.
But thereby it immediately is equally a positing,
the sublating of the immediate which is its negative;
and this negative, from which it seemed to begin
as from something alien,
only is in this its beginning.
In this way, the immediate is not only implicitly in itself
(that is, for us or in external reflection)
the same as what reflection is,
but is posited as being the same.
For the immediate is determined by reflection as
the negative of the latter or as the other of it,
but it is reflection itself which negates this determining.
The externality of reflection vis-à-vis
the immediate is consequently sublated;
its self-negating positing is its coinciding
with its negative, with the immediate,
and this coinciding is the immediacy of essence itself.
It thus transpires that external reflection is not external
but is just as much the immanent reflection of immediacy itself;
or that the result of positing reflection is
essence existing in and for itself.
External reflection is thus determining reflection.

3. Determining reflection

Determining reflection is in general
the unity of positing and external reflection.
This is now to be examined more closely.

1. External reflection begins from immediate being,
positing reflection from nothing.
In its determining, external reflection posits another in the
place of the sublated being, but this other is essence;
the positing does not posit its determination in the place of an other;
it has no presupposition.
But, precisely for this reason,
it is not complete as determining reflection;
the determination which it posits is consequently only a posited;
this is an immediate, not however as equal to itself
but as self-negating;
its connection with the turning back into itself is absolute;
it is only in the reflection-into-itself
but is not this reflection itself.

The posited is therefore an other,
but in such a manner that the self-equality
of reflection is retained;
for the posited is only as sublated,
as reference to the turning back into itself.
In the sphere of being, existence was the being
that had negation in it, and being was the immediate ground
and element of this negation which was,
therefore, itself immediate negation.
In the sphere of essence,
positedness is what corresponds to existence.
Positedness is equally an existence,
but its ground is being as essence
or as pure negativity;
it is a determinateness or a negation,
not as existent but immediately as sublated.
Existence is only positedness;
this is the principle of the essence of existence.
Positedness stands on the one side over against existence,
and over against essence on the other:
it is to be regarded as the means which conjoins
existence with essence and essence with existence.
If it is said, a determination is only a positedness,
the claim can thus have a twofold meaning,
according to whether the determination is such
in opposition to existence or in opposition to essence.
In either meaning, existence is taken for
something superior to positedness,
which is attributed to external reflection, to the subjective.
In fact, however, positedness is the superior, because, as posited,
existence is what it is in itself something negative,
something that refers simply and solely to the turning back into itself.
For this reason positedness is only a positedness
with respect to essence:
it is the negation of this turning back
as achieved return into itself.

2. Positedness is not yet a determination of reflection;
it is only determinateness as negation in general.
But the positing is now united with external reflection;
in this unity, the latter is absolute presupposing, that is,
the repelling of reflection from itself
or the positing of determinateness as its own.
As posited, therefore, positedness is negation;
but as presupposed, it is reflected into itself.
And in this way positedness is a determination of reflection.

The determination of reflection is distinct
from the determinateness of being, of quality;
the latter is immediate reference to other in general;
positedness also is reference to other,
but to immanently reflected being.
Negation as quality is existent negation;
being constitutes its ground and element.
The determination of reflection, on the contrary,
has for this ground immanent reflectedness.
Positedness gets fixed in determination precisely
because reflection is self-equality in its negatedness;
the latter is therefore itself reflection into itself.
Determination persists here, not by virtue of being
but because of its self-equality.
Since the being which sustains quality is
unequal to the negation, quality is
consequently unequal within itself,
and hence a transient moment which disappears in the other.
The determination of reflection is
on the contrary positedness as negation,
negation which has negatedness for its ground,
is therefore not unequal to itself within itself,
and hence essential rather than transient determinateness.
What gives subsistence to it is the self-equality of reflection
which has the negative only as negative,
as something sublated or posited.

Because of this reflection into themselves,
the determinations of reflection appear as
free essentialities, sublated in the void
without reciprocal attraction or repulsion.
In them the determinateness has become entranced
and infinitely fixed by virtue of the reference to itself.
It is the determinate which has subjugated its transitoriness
and its mere positedness to itself, that is to say,
has deflected its reflection-into-other into reflection-into- itself.
These determinations hereby constitute the determinate shine
as it is in essence, the essential shine.
Determining reflection is for this reason
reflection that has exited from itself;
the equality of essence with itself is
lost in the negation, and negation predominates.

Thus there are two distinct sides to the determination of reflection.
First, reflection is positedness, negation as such;
second, it is immanent reflection.
According to the side of positedness,
it is negation as negation,
and this already is its unity with itself.
But it is this unity at first only implicitly or in itself,
an immediate which sublates itself within, is the other of itself.
To this extent, reflection is a determining that abides in itself.
In it essence does not exit from itself;
the distinctions are solely posited,
taken back into essence.
But, from the other side, they are not posited
but are rather reflected into themselves;
negation as negation is equality with itself,
not in its other, not reflected into its non-being.

3. Now keeping in mind that the determination of reflection is
both immanently reflected reference and positedness as well,
its nature immediately becomes more transparent.
For, as positedness, the determination is negation as such,
a non-being as against another, namely,
as against the absolute immanent reflection or as against essence.
But as self-reference, it is reflected within itself.
This, the reflection of the determination,
and that positedness are distinct;
its positedness is rather the sublatedness of the determination
whereas its immanent reflectedness is its subsisting.
In so far as now the positedness is
at the same time immanent reflection,
the determinateness of the reflection is
the reference in it to its otherness.
It is not a determinateness that exists quiescent,
one which would be referred to an other
in such a way that the referred term
and its reference would be different,
each something existing in itself,
each a something that excludes its other
and its reference to this other from itself.
Rather, the determination of reflection is
within it the determinate side
and the reference of this determinate side as determinate,
that is, the reference to its negation.
Quality, through its reference, passes over into another;
its alteration begins in its reference.
The determination of reflection, on the contrary,
has taken its otherness back into itself.
It is positedness, negation which has however deflected
the reference to another into itself,
and negation which, equal to itself,
is the unity of itself and its other,
and only through this is an essentiality.
It is, therefore, positedness, negation,
but as reflection into itself it is at the same time
the sublatedness of this positedness,
infinite reference to itself.

CHAPTER 2

Foundation

The essentialities or the determinations of reflection

Reflection is determined reflection;
accordingly, essence is determined essence, or it is essentiality.

Reflection is the shining of essence within itself.

Essence, as infinite immanent turning back is
not immediate simplicity, but negative simplicity;
it is a movement across moments that are distinct,
is absolute mediation with itself.
But in these moments it shines;
the moments are, therefore, themselves
determinations reflected into themselves.

First, essence is simple self-reference, pure identity.
This is its determination, one by which it is rather
the absence of determination.

Second, the specifying determination is difference,
difference which is either external or indefinite,
diversity in general, or opposed diversity or opposition.

Third, as contradiction this opposition is reflected into itself
and returns to its foundation.

A. IDENTITY

1. Essence is simple immediacy as sublated immediacy.
Its negativity is its being;
it is equal to itself in its absolute negativity
by virtue of which otherness and reference to other
have as such simply disappeared into pure self-equality.
Essence is therefore simple self-identity.

This self-identity is the immediacy of reflection.
It is not that self-equality which being is, or also nothing,
but a self-equality which, in producing itself as unity,
does not produce itself over again, as from another,
but is a pure production, from itself and in itself,
essential identity.

It is not, therefore, abstract identity
or an identity which is the result
of a relative negation preceding it,
one that separates indeed
what it distinguishes from it
but, for the rest, leaves it existing outside it,
the same after as before.

Being, and every determinateness of being,
has rather sublated itself not relatively,
but in itself, and this simple negativity,
the negativity of being in itself,
is the identity itself.

In general, therefore,
it is still the same as essence.

B. DIFFERENCE

1. Absolute difference

Difference is the negativity that
reflection possesses in itself,
the nothing which is said in identity discourse,
the essential moment of identity itself
which, as the negativity of itself,
at the same time determines itself
and is differentiated from difference.

1. This difference is difference in and for itself,
absolute difference, the difference of essence.
It is difference in and for itself,
not difference through something external
but self-referring, hence simple, difference.
It is essential that we grasp absolute difference as simple.
In the absolute difference of A and not-A from each other,
it is the simple “not” which, as such,
constitutes the difference.
Difference itself is a simple concept.
“In this,” so it is said, “two things differ, in that etc.”
“In this,” that is, in one and the same respect,
relative to the same basis of determination.
It is the difference of reflection,
not the otherness of existence.
One existence and another existence are
posited as lying outside each other;
each of the two existences thus
determined over against each other
has an immediate being for itself.
The other of essence, by contrast,
is the other in and for itself,
not the other of some other
which is to be found outside it;
it is simple determinateness in itself.
Also in the sphere of existence
did otherness and determinateness
prove to be of this nature,
simple determinateness, identical opposition;
but this identity showed itself only as
the transition of a determinateness into the other.
Here, in the sphere of reflection,
difference comes in as reflected,
so posited as it is in itself.

2. Difference in itself is the difference
that refers itself to itself;
thus it is the negativity of itself,
the difference not from another
but of itself from itself;
it is not itself but its other.
What is different from difference, however, is identity.
Difference is, therefore, itself and identity.
The two together constitute difference;
difference is the whole and its moment.
One can also say that difference,
as simple difference, is no difference;
it is such only with reference to identity;
even better, that as difference it entails
itself and this reference equally.
Difference is the whole and its own moment,
just as identity equally is its whole and its moment.
This is to be regarded as
the essential nature of reflection
and as the determined primordial origin
of all activity and self-movement.
Both difference and identity make themselves
into moment or positedness
because, as reflection, they are negative self-reference.
Difference, thus as unity of itself and of identity,
is internally determined difference.
It is not the transition into another,
not reference to another outside it;
it has its other, identity, within,
and in like manner identity,
in being included in the determination of difference,
has not lost itself in it as its other
but retains itself therein is
the reflection-into-itself of difference, its moment.

3. Difference has both these moments,
identity and difference;
thus the two are both a positedness, determinateness.
But in this positedness each refers to itself.
The one, identity, is itself immediately
the moment of immanent reflection;
but no less is the other, difference,
difference in itself, reflected difference.
Difference, inasmuch as it has two such moments
which are themselves reflections into themselves,
is diversity.

2. Diversity

1. Identity internally breaks apart into diversity
because, as absolute difference in itself,
it posits itself as the negative of itself
and these, its two moments
(itself and the negative of itself),
are reflections into themselves,
are identical with themselves;
or precisely because it itself
immediately sublates its negating
and is in its determination reflected into itself.
The different subsists as diverse,
indifferent to any other,
because it is identical with itself,
because identity constitutes its base and element;
or, the diverse remains what it is
even in its opposite, identity.

Diversity constitutes the otherness
as such of reflection.
The other of existence has immediate being,
where negativity resides, for its foundation.
But in reflection it is self-identity,
the reflected immediacy, that constitutes
the subsistence of the negative and its indifference.

The moments of difference are identity and difference itself.
These moments are diverse when reflected into themselves,
referring themselves to themselves;
thus, in the determination of identity,
they are only self-referring;
identity is not referred to difference,
nor is difference referred to identity;
hence, inasmuch as each of these moments is
referred only to itself, the two
are not determined
with respect to each other.
Now because in this way the two are not differentiated within,
the difference is external to them.
The diverse moments, therefore,
conduct themselves with respect to each other,
not as identity and difference,
but only as moments different in general,
indifferent to each other and to their determinateness.

2. In diversity, as the indifference of difference,
reflection has in general become external;
difference is only a positedness or as sublated,
but is itself the whole reflection.
On closer consideration, both, identity and difference
are reflections, as we have just established;
each is the unity of it and its other,
each is the whole.
But the determinateness,
to be only identity or only difference,
is thus a sublated something.
They are not, therefore, qualities,
since their determinateness,
because of the immanent reflection,
is at the same time only as negation.
What we have is therefore this duplicity,
immanent reflection as such
and determinateness as negation or positedness.
Positedness is the reflection that is external to itself;
it is negation as negation
and consequently, indeed in itself
self-referring negation and immanent reflection,
but only in itself, implicitly;
its reference is to a something external.

Reflection in itself and external reflection are
thus the two determinations
in which the moments of difference,
identity and difference, are posited.
They are these moments themselves
as they have determined themselves at this point.
Immanent reflection is identity,
but determined to be indifferent to difference,
not to have difference at all but to conduct
itself towards difference as identical with itself;
it is diversity.
It is identity that has so reflected itself into itself
that it truly is the one reflection of
the two moments into themselves;
both are immanent reflections.
Identity is this one reflection of the two,
the identity which has difference within it
only as an indifferent difference
and is diversity in general.
External reflection, on the contrary,
is their determinate difference,
not as absolute immanent reflection,
but as a determination towards which
the implicitly present reflection is indifferent;
its two moments, identity and difference themselves,
are thus externally posited,
are not determinations that
exist in and for themselves.

Now this external identity is likeness,
and external difference is unlikeness.
Likeness is indeed identity,
but only as a positedness,
an identity which is not in and for itself.
Unlikeness is equally difference,
but an external difference which is not, in and for itself,
the difference of the unlike itself.
Whether something is like or unlike something else is
not the concern of either the like or the unlike;
each refers only to itself, each is in and for itself what it is;
identity or non-identity, in the sense of likeness or unlikeness,
depend on the point of view of a third external to them.

3. External reflection connects diversity by
referring it to likeness and unlikeness.
This reference, which is a comparing,
moves back and forth from likeness
to unlikeness and from unlikeness to likeness.
But this back and forth referring of
likeness and unlikeness is
external to these determinations themselves;
moreover, they are not referred to each other,
but each, for itself, is referred to a third.
In this alternation,
each immediately stands out on its own.
External reflection is as such external to itself;
determinate difference is negated absolute difference;
it is not simple difference, therefore,
not an immanent reflection,
but has this reflection outside it;
hence its moments come apart
and both refer,
each also outside the other,
to the immanent reflection
confronting them.

In reflection thus alienated from itself,
likeness and unlikeness present themselves,
therefore, as themselves unconnected,
and reflection keeps them apart,
for it refers them to one and the same
something by means of “in so far,”
“from this side or that,”
and “from this view or that.”
Thus diverse things that are one and the same,
when likeness and unlikeness are said of them,
are from one side like each other,
but from another side unlike,
and in so far as they are alike,
to that extent they are not unlike.
Likeness thus refers only to itself,
and unlikeness is equally only unlikeness.

Because of this separation from each other,
they sublate themselves.
Precisely that which should save them
from contradiction and dissolution,
namely that something is like another in one respect
but unlike in another precisely this keeping of
likeness and unlikeness apart, is their destruction.
For both are determinations of difference;
they are references to each other,
each intended to be what the other is not;
the like is not the unlike,
and the unlike is not the like;
both have this connecting reference essentially,
and have no meaning outside it;
as determinations of difference,
each is what it is as different from its other.
But because of their indifference to each other,
the likeness is referred to itself,
and similarly is unlikeness a point of view of
its own and a reflection unto itself;
each, therefore, is like itself;
difference has vanished, since they have no
determinateness to oppose them;
in other words, each is consequently only likeness.

Accordingly, this indifferent viewpoint
or the external difference sublates itself
and it is in itself the negativity of itself.
It is the negativity which in comparing
belongs to that which does the comparing.
This latter oscillates from likeness
to unlikeness and back again;
hence it lets the one disappear into the other
and is in fact the negative unity of both.
This negative unity transcends at first
what is compared as well as
the moments of the comparing as
a subjective operation that falls outside them.
But the result is that this unity is
in fact the nature of likeness and unlikeness themselves.
Even the independent viewpoint
that each of these is,
is rather the self-reference
that sublates their distinctness
and so, too, themselves.

From this side, as moments of external reflection
and as external to themselves,
likeness and unlikeness disappear together into their likeness.
But this, their negative unity,
is in addition also posited in them;
for their reflection implicitly exists
outside them, that is, they are the likeness and
unlikeness of a third,
of another than they themselves are.
Thus the like is not the like of itself,
and the unlike, as the unlike not of itself
but of an unlike to it,
is itself the like.
The like and the unlike is
each therefore the unlike of itself.
Each is thereby this reflection:
likeness, that it is itself and the unlikeness;
unlikeness, that it is itself and the likeness.
Likeness and unlikeness constituted
the side of positedness as against
what is being compared or the diverse
which, as contrasted with them,
had determined itself as implicitly existent reflection.
But this positedness has consequently equally
lost its determinateness as against this reflection.

Likeness and unlikeness,
the determinations of external reflection,
are precisely the merely
implicitly existent reflection
which the diverse as such was supposed to be,
its only indeterminate difference.
Implicitly existent reflection is
self-reference without negation,
abstract self-identity
and therefore positedness itself.
The merely diverse thus passes over
through the positedness
into negative reflection.
The diverse is difference
which is merely posited,
hence a difference which is no difference,
hence a negation that negates itself within.
Likeness and unlikeness themselves, the positedness,
thus return through indifference
or through implicitly existing reflection
back into negative unity with themselves,
into the reflection which is
the implicit difference of likeness and unlikeness.
Diversity, the indifferent sides of which
are just as much simply and solely
moments of a negative unity, is opposition.

3. Opposition

In opposition, the determinate reflection,
difference, is brought to completion.
Opposition is the unity of identity and diversity;
its moments are diverse in one identity,
and so they are opposites.

Identity and difference are the moments of
difference as held inside difference itself;
they are reflected moments of its unity.
Likeness and unlikeness are instead
the externalized reflection;
their self-identity is not only the indifference
of each towards the other differentiated from it,
but towards being-in-and-for-itself as such;
theirs is a self-identity that contrasts with
identity reflected into itself,
hence an immediacy which is not reflected into itself.
The positedness of the sides of
external reflection is therefore a being,
just as their non-positedness is a non-being.

On closer consideration, the moments of opposition are
positedness reflected into itself
or determination in general.
Positedness is likeness and unlikeness;
these two, reflected into themselves,
constitute the determinations of opposition.
Their immanent reflection consists in that
each is within it the unity of likeness and unlikeness.
Likeness is only in a reflection
which compares according to the unlikeness
and is therefore mediated by its
other indifferent moment; similarly,
unlikeness is only in the same
reflective reference in which likeness is.
Each of these moments, in its determinateness,
is therefore the whole.
It is the whole because it also contains its other moment;
but this, its other, is an indifferent existent;
thus each contains a reference to its non-being,
and it is reflection-into-itself, or the whole,
only as essentially referring to its non-being.

This self-likeness, reflected into itself
and containing the reference to
unlikeness within it, is the positive;
and the unlikeness that contains within itself
the reference to its non-being,
to likeness, is the negative.
Or again, both are positedness;
now in so far as the differentiated determinateness is
taken as a differentiated determinate reference of
positedness to itself, opposition is, on the one hand,
positedness reflected into its likeness with itself;
and, on the other hand, it is the same positedness
reflected into its inequality with itself:
the positive and the negative.
The positive is positedness as reflected into self-likeness;
but what is reflected is positedness, that is,
the negation as negation,
and so this immanent reflection has
the reference to the other for its determination.
The negative is positedness as reflected into unlikeness;
but positedness is the unlikeness itself,
and so this reflection is therefore
the identity of unlikeness with itself
and absolute self-reference.
Each, therefore, equally has the other in it:
positedness reflected into self-likeness has the unlikeness;
and positedness reflected into self-unlikeness, the likeness.

The positive and the negative are thus
the sides of opposition that have become self-subsisting.
They are self-subsisting because they are
the reflection of the whole into itself,
and they belong to opposition in so far
as the latter is determinateness
which, as the whole, is reflected into itself.
Because of their self-subsistence,
the opposition which they constitute is
implicitly determinate.
Each is itself and its other;
for this reason, each has its determinateness
not in an other but within.
Each refers itself to itself
only as referring itself to its other.
This has a twofold aspect.
Each is the reference to its non-being as
the sublating of this otherness in itself;
its non-being is thus only a moment in it.
But, on the other hand, here positedness
has become a being, an indifferent subsistence;
the other of itself which each contains is
therefore also the non-being of that in which
it should be contained only as a moment.
Each is, therefore, only to the
extent that its non-being is,
the two in an identical reference.

The determinations which constitute
the positive and the negative consist,
therefore, in that the positive and the negative are,
first, absolute moments of opposition;
their subsistence is indivisibly one reflection;
it is one mediation in which each is
by virtue of the non-being of its other,
hence by virtue of its other
or its own non-being.
Thus they are simply opposites;
or each is only the opposite of the other;
the one is not yet the positive
and the other not yet the negative,
but both are negative with
respect to each other.
Each, therefore, simply is,
first, to the extent that the other is;
it is what it is by virtue of the other,
by virtue of its own non-being;
it is only positedness.
Second, it is to the extent that
the other is not; it is what it is
by virtue of the non-being of the other;
it is reflection into itself.
The two, however, are both
the one mediation of opposition as such
in which they simply are only posited moments.

Moreover, this mere positedness is
reflected into itself in general
and, according to this moment of external reflection,
the positive and the negative are indifferent towards
this first identity where they are only moments;
or again, because that first reflection is
the positive's and the negative's own
reflection into itself,
each is indifferent towards its reflection
into its non-being, towards its own positedness.
The two sides are thus merely diverse,
and because their determinateness
that they are positive or negative
constitutes their positedness as against each other,
each is not specifically so determined internally
but is only determinateness in general;
to each side, therefore, there belongs indeed
one of the two determinacies,
the positive or the negative;
but the two can be interchanged,
and each side is such as
can be taken equally as positive or negative.

But, in third place, the positive and the negative are
not only a posited being,
nor are they something merely indifferent,
but their positedness,
or the reference to the other in the one unity
which they themselves are not,
is rather taken back into each.
Each is itself positive and negative within;
the positive and the negative are
the determination of reflection in and for itself;
only in this reflection of the opposite into itself is
the opposite either positive or negative.
The positive has within it the reference to
the other in which the determinateness of the positive consists.
And the same applies to the negative:
it is not negative as contrasted with another
but has the determinateness by which it is negative within.

Each is thus self-subsistent unity existing for itself.
The positive is indeed a positedness,
but in such a way that the positedness is
for it posited being as sublated.
It is the non-opposed, the sublated opposition,
but as the side of the opposition itself.
As positive, it is indeed a something
which is determined with reference to an otherness,
but in such a way that its nature
is not to be something posited;
it is the immanent reflection
that negates otherness.
But its other, the negative,
is itself no longer positedness or a moment
but itself a self-subsisting being
and so the negating reflection
of the positive is internally
determined to exclude this being,
which is its non-being, from itself.

Thus the negative, as absolute reflection,
is not the immediate negative
but is the negative as sublated positedness,
the negative in and for itself
which positively rests upon itself.
As immanent reflection,
it negates its reference to its other;
its other is the positive,
a self-subsisting being
hence its negative reference
to this positive is
the excluding of it from itself.
The negative is the independently existing opposite,
over against the positive
which is the determination of the sublated opposition,
the whole opposition resting upon itself,
opposed to the self-identical positedness.

The positive and the negative are such, therefore,
not just in themselves, but in and for themselves.
They are in themselves positive and negative
when they are abstracted from their excluding
reference to the other
and are taken only in accordance
with their determination.
Something is in itself positive or negative
when it is not supposed to be
determined as positive or negative
merely in contrast with the other.
But the positive and the negative,
taken not as a positedness
and hence not as opposed,
are each an immediate,
being and non-being.
They are, however, moments of opposition:
their in-itself constitutes only
the form of their immanent reflectedness.
Something is said to be positive in itself,
outside the reference to something negative,
and something negative in itself,
outside the reference to something negative:
in this determination, merely the abstract moment of
this reflectedness is held on to.
However, to say that the positive and the negative
exist in themselves essentially implies that
to be opposed is not a mere moment,
nor that it is just a matter of comparison,
but that it is the determination of the sides
themselves of the opposition.
The sides, as positive or negative in themselves,
are not, therefore, outside the reference to the other;
on the contrary, this reference, precisely as exclusive,
constitutes their determination or their in-itselfness;
in this, therefore, they are at the same time
in and for themselves.

C. CONTRADICTION

1. Difference in general contains both its sides as moments;
in diversity, these sides fall apart as indifferent to each other;
and in opposition as such, they are the moments of difference,
each determined by the other and hence only moments.
But in opposition these moments are equally determined within,
indifferent to each other and mutually exclusive,
self-subsisting determinations of reflection.

One is the positive and the other the negative,
but the former as a positive which is such within,
and the latter as a negative which is such within.
Each has indifferent self-subsistence for itself
by virtue of having the reference to
its other moment within it;
each moment is thus the whole self-contained opposition.
As this whole, each moment is self-mediated
through its other and contains this other.
But it is also self-mediated
through the non-being of its other
and is, therefore, a unity existing for itself
and excluding the other from itself.

Since the self-subsisting determination of reflection
excludes the other in the same respect as it contains it
and is self-subsisting for precisely this reason,
in its self-subsistence the determination excludes
its own self-subsistence from itself.
For this self-subsistence consists in
that it contains the determination
which is other than it in itself
and does not refer to anything external
for just this reason;
but no less immediately in that
it is itself and excludes from itself
the determination that negates it.
And so it is contradiction.

Difference as such is already implicitly contradiction;
for it is the unity of beings which are,
only in so far as they are not one
and it is the separation of beings which are,
only in so far as they are separated
in the same reference connecting them.
The positive and the negative, however,
are the posited contradiction,
for, as negative unities,
they are precisely their self-positing
and therein each the sublating of itself
and the positing of its opposite.
They constitute determining reflection as exclusive;
for the excluding is one act of distinguishing
and each of the distinguished beings,
as exclusive, is itself the whole act of excluding,
and so each excludes itself internally.

If we look at the two self-subsisting
determinations of reflection on their own,
the positive is positedness as reflected
into likeness with itself
positedness which is not reference to another,
hence subsistence inasmuch as
the positedness is sublated and excluded.
But with this the positive makes itself
into the reference of a non-being into a positedness.
In this way the positive is contradiction in that,
as the positing of self-identity by the
excluding of the negative,
it makes itself into a negative,
hence into the other
which it excludes from itself.
This last, as excluded, is posited
free of the one that excludes;
hence, as reflected into itself and itself as excluding.
The reflection that excludes is thus
the positing of the positive as excluding the other,
so that this positing immediately is
the positing of its other which excludes it.

This is the absolute contradiction of the positive;
but it is immediately the absolute contradiction of the negative;
the positing of both in one reflection.
Considered in itself as against the positive,
the negative is positedness as reflected into unlikeness to itself,
the negative as negative.
But the negative is itself the unlike,
the non-being of another;
consequently, reflection is in its unlikeness
its reference rather to itself.
Negation in general is the negative
as quality or immediate determinateness;
but taken as negative, it is referred to
the negative of itself, to its other.
If this second negative is taken only
as identical with the first,
then it is also only immediate,
just like the first;
they are not taken, therefore,
as each the other of the other,
hence not as negatives:
the negative is not at all an immediate.
But now, since each is moreover equally
the same as what the other is,
this reference connecting them as unequal
is just as much their identical connection.

This is therefore the same contradiction which the positive is,
namely positedness or negation as self-reference.
But the positive is only implicitly this contradiction,
is contradiction only in itself;
the negative, on the contrary, is the posited contradiction;
for in its reflection into itself,
as a negative which is in and for itself
or a negative which is identical with itself,
its determination is to be the not-identical,
the exclusion of identity.
The negative is this,
to be identical with itself over against identity,
and consequently, because of this excluding reflection,
to exclude itself from itself.

The negative is therefore the whole opposition
the opposition which, as opposition, rests upon itself;
distinction that absolutely does not refer itself to another;
distinction which, as opposition, excludes identity from itself,
but thereby also excludes itself,
for as reference to itself it determines itself
as the very identity which it excludes.

2. Contradiction resolves itself.

In the self-excluding reflection
we have just considered,
the positive and the negative,
each in its self-subsistence,
sublates itself;
each is simply the passing over,
or rather the self-translating of itself into its opposite.
This internal ceaseless vanishing of the opposites is
the first unity that arises by virtue of contradiction;
it is the null.

But contradiction does not contain merely the negative;
it also contains the positive;
or the self-excluding reflection is
at the same time positing reflection;
the result of contradiction is not only the null.
The positive and the negative constitute
the positedness of the self-subsistence;
their own self-negation sublates it.
It is this positedness which in truth
founders to the ground in contradiction.

The immanent reflection by virtue of which
the sides of opposition are turned into
self-subsistent self-references is,
first of all, their self-subsistence as distinct moments;
thus they are this self-subsistence only in themselves,
for they are still opposites,
and that they are in themselves self-subsistent
constitutes their positedness.
But their excluding reflection
sublates this positedness,
turns them into self-subsistent beings
existing in and for themselves,
such as are self-subsistent not only in themselves
but by virtue of their negative reference to their other;
in this way, their self-subsistence is also posited.
But, further, by thus being posited as self-subsistent,
they make themselves into a positedness.
They fate themselves to founder,
since they determine themselves as self-identical,
yet in their self-identity they are rather the negative,
a self-identity which is reference-to-other.

However, on closer examination,
this excluding reflection is not only
this formal determination.
It is self-subsistence existing in itself,
and the sublating of this positedness
is only through this sublating a unity that
exists for itself and is in fact self-subsistent.
Of course, through the sublating of otherness or positedness,
positedness or the negative of an other is indeed present again.
But in fact, this negation is not just a return to the first
immediate reference to the other,
is not positedness as sublated immediacy,
but positedness as sublated positedness.
The excluding reflection of self-subsistence,
since it is excluding,
makes itself a positedness but is just as much
the sublation of its positedness.
It is sublating reference to itself;
in that reference, it first sublates the negative
and it secondly posits itself as a negative,
and it is only this posited negative that it sublates;
in sublating the negative,
it both posits and sublates it at the same time.
In this way the exclusive determination is
itself that other of itself of which it is the negation;
the sublation of this positedness is not, therefore,
once more positedness as the negative of an other,
but is self-withdrawal, positive self-unity.
Self-subsistence is thus unity that turns back into itself
by virtue of its own negation,
for it turns into itself through the negation of its positedness.
It is the unity of essence to be identical with itself
through the negation not of an other, but of itself.

3. According to this positive side,
since self-subsistence in opposition,
as excluding reflection,
makes itself into a positedness
and equally sublates this positedness,
not only has opposition foundered
but in foundering it has gone back
to its foundation, to its ground.
The excluding reflection of
the self-subsisting opposition turns it
into a negative, something only posited;
it thereby reduces its formerly self-subsisting determinations,
the positive and the negative,
to determinations which are only determinations;
and the positedness, since it is now made into positedness,
has simply gone back to its unity with itself;
it is simple essence, but essence as ground.
Through the sublating of the determinations of essence,
which are in themselves self-contradictory,
essence is restored,
but restored in the determination of
an exclusive, reflective unity
a simple unity which determines itself as negation,
but in this positedness is immediately like itself
and withdrawn into itself.

In the first place, therefore, because of its contradiction,
the self-subsisting opposition goes back into a ground;
this opposition is what comes first,
the immediate from which the beginning is made,
while the sublated opposition
or the sublated positedness is itself a positedness.
Accordingly, essence is as ground a positedness,
something that has become.
But conversely, only this has been posited,
namely that the opposition or the positedness is
something sublated, only is as positedness.
As ground, therefore, essence is excluding reflection
because it makes itself into a positedness;
because the opposition from which the start
was just now made and was the immediate is
the merely posited determinate self-subsistence of essence;
because opposition only sublates itself within,
whereas essence is in its determinateness reflected into itself.
As ground, therefore, essence excludes
itself from itself, it posits itself;
its positedness which is what is excluded
is only as positedness,
as identity of the negative with itself.
This self-subsistent is the negative
posited as the negative,
something self-contradictory
which, consequently, remains in
the essence as in its ground.

The resolved contradiction is therefore ground,
essence as unity of the positive and the negative.
In opposition, determinateness has progressed to self-subsistence;
but ground is this self-subsistence as completed;
in it, the negative is self-subsistent essence, but as negative;
and, as self-identical in this negativity,
ground is thus equally the positive.
In ground, therefore, opposition and its contradiction
are just as much removed as preserved.
Ground is essence as positive self-identity
which, however, at the same time
refers itself to itself as negativity
and therefore determines itself,
making itself into an excluded positedness;
but this positedness is the whole self-subsisting essence,
and essence is ground, self-identical in its negation and positive.
The self-contradictory self-subsistent opposition
was itself, therefore, already ground;
all that was added to it was the determination of self-unity
which emerges as each of the self-subsisting opposites
sublates itself and makes itself into its other,
thereby founders and sinks to the ground
but therein also reunites itself with itself;
thus in this foundering, that is,
in its positedness or in the negation,
it rather is for the first time the essence
that is reflected into itself and self-identical.

CHAPTER 3

Ground

Essence determines itself as ground.

Just as nothing is at first
in simple immediate unity with being,
so here too the simple identity of essence is
at first in simple unity with its absolute negativity.
Essence is only this negativity which is pure reflection.
It is this pure reflection as
the turning back of being into itself;
hence it is determined, in itself or for us,
as the ground into which being resolves itself.
But this determinateness is not posited by the essence itself;
in other words, essence is not ground precisely because
it has not itself posited this determinateness that it possesses.
Its reflection, however, consists in positing itself as
what it is in itself, as a negative, and in determining itself.
The positive and the negative constitute the essential determination
in which essence is lost in its negation.
These self-subsisting determinations of reflection sublate themselves,
and the determination that has foundered to the ground is
the true determination of essence.

Consequently, ground is itself one of
the reflected determinations of essence,
but it is the last, or rather,
it is determination determined as sublated determination.
In foundering to the ground, the determination of reflection
receives its true meaning that it is the absolute
repelling of itself within itself;
or again, that the positedness that accrues to essence is
such only as sublated,
and conversely that only the self-sublating positedness is
the positedness of essence.
In determining itself as ground,
essence determines itself as the not-determined,
and only the sublating of its being determined is its determining.
Essence, in thus being determined as self-sublating,
does not proceed from an other but is,
in its negativity, identical with itself.

Since the advance to the ground is made starting
from determination as an immediate first
(is done by virtue of the nature of determination itself
that founders to the ground through itself),
the ground is at first determined by that immediate first.
But this determining is, on the one hand,
as the sublating of the determining,
the merely restored, purified or manifested identity of essence
which the determination of reflection is in itself;
on the other hand, this negating movement is, as determining,
the first positing of that reflective determinateness
that appeared as immediate determinateness,
but which is posited only by the self-excluding reflection of ground
and therein is posited as only something posited or sublated.
Thus essence, in determining itself as ground, proceeds only from itself.
As ground, therefore, it posits itself as essence,
and its determining consists in just this positing of itself as essence.
This positing is the reflection of essence
that sublates itself in its determining;
on that side is a positing, on this side is the positing of essence,
hence both in one act.

Reflection is pure mediation in general;
ground, the real mediation of essence with itself.
The former, the movement of nothing through nothing back to itself,
is the reflective shining of one in an other;
but, because in this reflection opposition does not
yet have any self-subsistence,
neither is the one, that which shines, something positive,
nor is the other in which it reflectively shines something negative.
Both are substrates, actually of the imagination;
they are still not self-referring.
Pure mediation is only pure reference,
without anything being referred to.
Determining reflection, for its part, does posit
such terms as are identical with themselves;
but these are at the same time only determined references.
Ground, on the contrary, is mediation that is real,
since it contains reflection as sublated reflection;
it is essence that turns back into itself
through its non-being and posits itself.
According to this moment of sublated reflection,
what is posited receives the determination of immediacy,
of an immediate which is self-identical
outside its reference or its reflective shining.
This immediacy is being as restored by essence,
the non-being of reflection through which essence mediates itself.
Essence returns into itself as it negates;
therefore, in its turning back into itself,
it gives itself the determinateness that precisely
for this reason is the self-identical negative,
is sublated positedness, and consequently,
as the self-identity of essence as ground,
equally an existent.

The ground is, first, absolute ground
one in which the essence is first of all
the general substrate for the ground-connection.
It then further determines itself as form and matter
and gives itself a content.

Second, it is determinate ground,
the ground of a determinate content.
Because the ground-connection, in being realized,
becomes as such external,
it passes over into conditioning mediation.

Third, ground presupposes a condition;
but the condition equally presupposes the ground;
the unconditioned is the unity of the two,
the fact itself that, by virtue of
the mediation of the conditioning reference,
passes over into concrete existence.
