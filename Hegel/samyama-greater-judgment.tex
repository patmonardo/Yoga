A. THE JUDGMENT OF EXISTENCE

In the subjective judgment we expect to see
one and the same object double,
once in its singular actuality,
and again in its essential identity
or in its concept:
the singular raised into its universality
or, what is the same thing,
the universal made singular into its actuality.
The judgment is thus truth,
for it is the agreement of the concept and reality.
But it is not at first constituted in this way,
for at first the judgment is immediate,
since as yet no reflection and no movement
of the determinations has been found in it.
This immediacy renders the first judgment
a judgment of immediate existence;
we can also call it a qualitative judgment,
but only in so far as quality does not apply
to the determinateness of being alone
but also extends to the universality
which, because of its simplicity,
likewise has the form of immediacy.

The judgment of existence is also
the judgment of inherence because,
though immediacy is its determination,
it is the subject that in the distinction
between subject and predicate is the immediate
and hence the first and the essential
term in the judgment,
and the predicate consequently takes on
the form of something that does not subsist on its own
but has its foundation in the subject.

a. The positive judgment

1. The subject and predicate, as we have just said,
are names at first that receive their actual determination
only as the judgment runs its course.
However, as sides of the judgment,
the judgment being the posited determinate concept,
they have the determination of moments of the concept,
but, on account of their immediacy,
this determination is as yet quite simple,
still not enriched by mediation
and also still caught up in
the abstract opposition of
abstract singularity
and abstract universality.
The predicate, to speak of it first,
is the abstract universal;
this abstract is conditioned by mediation,
by the sublation of singularity and particularity,
but so far such a mediation is here only a presupposition.
In the sphere of the concept
there can be no other immediacy
than the one that contains
mediation in and for itself
and has arisen only through its sublation;
this is the immediacy of the universal.
Thus qualitative being also is
in its concept a universal;
as being, however, the immediacy is
not yet posited as such;
it is only as universality
that immediacy is the concept determination
in which it is posited that
negativity essentially belongs to it.
This connection is given in the judgment
in which universality is the predicate of a subject.
Similarly the subject is an abstract singular,
or the immediate which is supposed to be such
and therefore the singular as a something in general.
The subject constitutes, therefore,
the abstract side of the judgment,
the side in it according to which
the concept has passed over into externality.
As these two concept determinations are determined,
so is also their connection,
the “is” or the copula;
it too can have no other meaning than
that of an immediate, abstract being.
It is because of this connection,
which still does not contain any mediation or negation,
that this judgment is called “positive.”

2. The first pure expression of
the positive judgment is, therefore,
the proposition: the singular is universal.
This expression must not be put
in the form of “A is B,”
for A and B are totally formless
and hence meaningless names,
whereas judgment in general,
and therefore already the judgment of existence,
has determinations of the concept for its extremes.
“A is B” can stand just as well
for any mere proposition as for a judgment.
But what is asserted in every judgment,
even one more richly determined in form,
is the proposition that has this determined content,
namely, “the singular is universal,”
for every judgment is in principle
also an abstract judgment.
(Regarding the negative judgment,
how far it likewise comes under this expression,
of this we shall speak presently.)
However, if no thought is given to
the fact that with every judgment,
the positive at least,
the assertion is made that the singular is universal,
this happens either because no attention is
given to the determinate form differentiating subject and object
(for it is taken for granted that the judgment is
nothing but the connecting of two concepts)
or also likely because the further content of the judgment,
“Gaius is learned,” or “the rose is red,”
comes drifting in before the mind,
and the latter, busy with the picture of Gaius etc.,
fails to reflect on the form;
even though, at least, such a content as “Gaius,”
which is the one that usually has to be dragged in as an example,
is much less interesting than the form,
and is indeed chosen because it is uninteresting,
not to divert attention from the form to itself.

The objective meaning of the proposition stating
that the singular is universal conveys,
as already incidentally noted,
both the perishableness of singular things
and their positive subsistence in the concept in general.
The concept itself is imperishable,
but that which emerges from it in its division
is subjected to alteration
and to falling back into its universal nature.
But the universal, conversely,
gives itself a determinate existence.
Just as essence goes out into
reflective shine in its determinations;
or ground into concrete existence in appearance;
and substance into manifestation in its accidents,
so does the universal resolve itself into the singular;
judgment is this resolution of the universal,
the development of the negativity
which, implicitly, it already is.
This last circumstance is expressed
by the converse proposition,
“the universal is singular,”
which is also equally spoken in the positive judgment.
The subject, the immediate singular at first,
is in the judgment itself referred to its other,
namely the universal;
it is thereby posited as the concrete
according to the category of being,
as a something of many qualities;
or as the concrete of reflection,
a thing of manifold properties,
an actual of manifold possibilities,
a substance of precisely such accidents.
Because these manifolds here belong
to the subject of the judgment, the something, the thing, etc.,
is in its qualities, properties, or accidents,
reflected into itself, or continues across them,
maintaining itself in them and them in itself.
Positedness or determinateness belongs to being
which is in and for itself.
The subject is therefore inherently the universal.
The predicate, on the contrary,
being this universality not as real or concrete,
but as abstract,
is in contrast to the subject the determinateness;
it contains only one moment of the subject's totality
to the exclusion of the others.
On account of this negativity,
which as an extreme of the judgment is at the
same time self-referring,
the predicate is an abstract singular.
For instance, in the proposition,
“the rose is fragrant,”
the predicate expresses only one of
the many properties of the rose;
it isolates it, whereas in the subject
the property is joined with the others;
likewise in the dissolution of the thing,
the manifold properties that inhere in it become
isolated in acquiring self-subsistence as materials.
From this side, then, the proposition of the judgment says:
the universal is singular.

By juxtaposing this reciprocal determination of
subject and predicate in the judgment,
we thus obtain this twofold result.
(1) Immediately, the subject is indeed
an existent or the singular,
while the predicate is the universal.
But because the judgment connects the two,
and the subject is determined as
universal by the predicate,
the subject is then the universal.
(2) The predicate is determined in the subject,
for it is not a determination in general
but the determination rather of the subject.
“The rose is fragrant.”
This fragrance is not some indeterminate fragrance or other,
but the fragrance of the rose.
The predicate is therefore a singular.
Now since the subject and predicate stand related in the judgment,
they should retain the opposition of concept determinations;
likewise, in the reciprocity of causality,
before the latter attains its truth,
the two sides are still
supposed to remain self-subsistent
and mutually opposed as against
the equality of their determination.
Therefore, when the subject is determined as universal,
the predicate should not also be taken
in its determination of universality,
for then we would have no judgment;
it must rather be taken only in its
determination of singularity.
And if the subject is determined as singular,
then the predicate is to be taken as universal.
If we reflect on the mere identity above,
then we have these two identical propositions,
“the singular is singular,”
“the universal is universal,”
in which the sides of the judgment
would have completely fallen apart;
only the self-reference of each is
expressed while the reference connecting
them to each other is dissolved;
and thus the judgment would be sublated.
Of the two propositions we drew,
the first, “the universal is singular,”
expresses the judgment according to its content,
as an isolated determination in the predicate
and as the totality of determinations in the subject.
The other, “the singular is universal,”
expresses it according to form as immediately given
through the judgment itself.
In the immediate positive judgment,
the extremes are still simple:
form and content are therefore still united.
Or, in other words, it does not
consist of two propositions;
the twofold connection that it yielded
immediately constitutes the one positive judgment.
For its extremes are
(a) the self-subsisting abstract determinations of judgment,
and (b) each side of the determination is determined
through the other by virtue of the copula connecting them.
Implicitly, however, the difference of form and content
is for this reason present in it, as we have seen;
and indeed, what the first proposition contains,
that the singular is universal, belongs to form,
for the proposition expresses the
immediate determinateness of the judgment.
The relation, on the contrary,
which the other proposition expresses,
that the universal is singular
or that the subject is determined as universal
whereas the predicate is determined as particular or singular,
concerns the content for its determinations are only
the result of an immanent reflection
by virtue of which the immediate determinacies of
the judgment are sublated
and the form is thereby converted into an identity
that has withdrawn into itself
and persists over against the distinction of form:
it converts itself into content.

3. If now the two propositions,
the one of form and the other of content,

(Subject) (Predicate)
The singular is universal
The universal is singular,

were to be united because
they are contained in the one positive judgment,
so that both, the subject as well as the predicate,
were determined as the unity of singularity and universality,
then both the subject and predicate would be the particular,
and this must be recognized as
implicitly their inner determination.
However, this combination would be arrived at only
through an external reflection;
moreover, the proposition that results from it,
“the particular is the particular,”
would no longer be a judgment
but an empty identical proposition
as were the two propositions already found
in the positive judgment,
“the singular is singular,”
and “the universal is universal.”
Singularity and universality cannot
yet be united into particularity,
because in the positive judgment
they are still posited as immediate.
Or again, the judgment must still be
distinguished according to its form and its content,
because the subject and predicate are themselves
still distinguished as immediacy and mediated,
or because the judgment, according to its connection,
is both the self-subsistence of the connected terms
and their reciprocal determination or mediation.

In first place, then, the meaning of the judgment
when considered according to its form is
that the singular is universal.
But in fact such an immediate singular is
definitely not universal;
its predicate is of wider extension,
does not correspond to it.
The subject is a being existing
immediately for itself,
and hence the opposite of that abstraction,
of that universality posited through mediation
that was supposed to be predicated of it.

In second place, if the judgment is
considered according to its content,
or as the proposition, “the universal is singular,”
then the subject is a universe of qualities,
an infinitely determined concrete universe,
and since its determinacies are as yet
qualitites, properties, or accidents,
its totality is the bad infinite plurality of them.
Such a subject, therefore, is not at all
the one single property that its predicate declares.
Consequently, both propositions must be united,
and the positive judgment must be posited as negative instead.

b. The negative judgment

1. We spoke earlier of the common notion
that whether the content of a judgment is true or false
depends solely on the content itself,
since logical truth concerns only the form
and its only requirement is that such
content shall not contradict itself.
Nothing else is reckoned as the form of judgment except
that the latter is a connection of two concepts.
But we have seen that these two concepts are not
just the relationless determination of a sum,
but that they relate to each other as singular and universal.
These are the determinations that
constitute the truly logical content
and also, abstracted in that way,
the content of the positive judgment;
whatever other content is in a judgment
(“the sun is round,” “Cicero was a greatRoman orator,”
 “it is daytime now,” etc.)
does not concern the judgment as such;
the judgment only says that the subject is predicate,
or, since these are only names,
that the singular is universal and vice versa.
It is because of this purely logical content
that the positive judgment is not true
but has its truth in the negative judgment.
In judgment, so it is required,
the content simply ought not to contradict itself;
but it does contradict itself in the positive judgment,
as we have just seen.
At any rate, it makes absolutely no difference
if that logical content is called form,
and by content is understood only the remaining empirical filling,
for even then the form would not contain a mere empty identity outside
which the content determination would then lie.
The positive judgment has in fact
no truth through its form as positive judgment;
whoever calls truth the correctness of an intuition or a perception,
the agreement of representation with the subject matter,
has for a minimum no expression left for that
which is the subject matter and the aim of philosophy.
We should at least say of these that they are the truth of reason,
and it will surely be granted that such judgments
as “Cicero was a great orator,”
that “it is daytime now,”
are definitely not truths of reason.
But they are not such truths,
not because they have an empirical content as it were contingently,
but because they are only positive judgments that can have,
and ought to have, no other content  than an immediate singular
and an abstract determinateness.

The positive judgment first attains its truth in the negative judgment:
the singular is not abstractly universal,
but rather, the predicate of the singular,
because it is such a predicate,
or because, if considered by itself
without reference to the subject,
it is an abstract universal,
is for that very reason itself something determinate;
from the start, therefore, the singular is a particular.
Furthermore, with respect to the other proposition
that the positive judgment contains,
the meaning of the negative judgment is
that the universal is not abstractly singular
but that this predicate, “singular,”
by the very fact that it is a predicate,
or because it refers to a universal subject,
is more than just mere singularity,
and the universal, accordingly,
is from the start equally a particular.
Since this universal, as subject,
is itself in the judgment determination of singularity,
the two propositions both reduce to one:
“the singular is a particular.”

We may remark that
(a) the particularity that here comes to the predicate has
already come up for consideration before;
here, however, it is not posited by external reflection
but has arisen rather as mediated by the negative connection
indicated in the judgment.
(b) This determination results here only for the predicate.
In the immediate judgment, the judgment of existence,
the subject is the underlying basis;
the determination seems at first, therefore, to occur in the predicate.
But in fact this first negation cannot as yet be a determination,
or cannot truly be the positing of the singular,
for such a positing is only a second moment,
the negative of the negative.

The singular is a particular:
this is the positive expression of the negative judgment.
This expression, therefore, is not the positive judgment itself,
for the latter, because of its immediacy,
has an abstraction for its extremes,
while the particular,
precisely through the positing of the judgment connection,
results as the first mediated determination.
But this determination is not to be taken
only as a moment of the extremes,
but also as the determination of the connection,
as it truly is from the start;
in other words, the judgment is also
to be considered as negative.

This transition is founded on
the relation of the extremes
and on their connection
in the judgment as such.

The positive judgment is the connection of
the singular and the universal
which are such immediately and each, therefore,
is not at the same time what the other is.
The connection is therefore just as
essentially separation, or negative;
for this reason the positive judgment was to be posited as negative.
There was no need, therefore,
for the logicians to make such a fuss about
the not of the negative judgment
being attached to the copula.
In the judgment, the determination of the extremes is
equally a determinate connection.
The judgment determination, or the extreme,
is not the purely qualitative one of immediate being that
only stands over against an other outside it.
Nor is it the determination of reflection,
which, in accordance with its general form,
behaves positively and negatively,
posited in either case as exclusive,
only implicitly identical with the other.
The judgment determination,
as the determination of the concept,
is a universal within, posited as extending continuously in its other.
Conversely, the judgment connection is
the same deterination as the
extremes have;
for it is precisely this universality and continuous extension
of each into the other;
in so far as these are distinguished,
the connection also has negativity in it.

The just stated transition from
the form of the connection
to the form of the determination
has the immediate consequence
that the not of the copula
must just as equally be attached to the predicate
and that the latter must be determined as the not-universal.
But, through a no less immediate consequence,
the not-universal is the particular.
If the focus is on the negative
according to the totally abstract
determination of immediate non-being,
then the predicate is the
totally indeterminate not-universal.
This is the determination which is normally treated in logic
in connection with the contradictory concepts,
and the further point is made,
a point considered important
that in the negative of a concept
one should only focus on the negative,
taking it as the mere indeterminate extent
of the other of the positive concept.
Thus the mere not-white would be
just as much red, yellow, blue, etc. as black.
White, however, is an unconceptualized determination of intuition;
the not of white is equally, then, unconceptualized not-being,
the abstraction that came in for consideration
at the very beginning of the Logic
where becoming was recognized to be its closest truth.
To use as an example,
in the consideration of judgment determinations,
an unconceptualized content of this sort,
drawn from intuition and the imagination,
and to take the determinations of being, and of reflection,
as such judgment determinations,
is the same uncritical practice as
when Kant applies the concepts of the understanding
to the infinite idea of reason, the so-called thing-in-itself;
the concept, to which the judgment proceeding from it also
belongs, is the true thing-in-itself or the rational;
those other determinations belong to being and essence;
they are not yet forms developed into the shape
where they are in their truth, in the concept.
If we stop at white, red, as representations of the senses,
then we call concept what is only a
determination of pictorial representation.
This is common practice.
But then, surely, the not-white, the not-red,
will be nothing positive,
just as the not-triangular will be
something totally indeterminate,
for a determination based as such on number and quantum is
essentially something indifferent, void of concept.
Yet, like non-being itself, such a sensuous content
ought to be conceptualized;
ought to shed that indifference and abstract immediacy
with which it is affected in the blind immobility
of pictorial representation.
Already in the sphere of immediate existence,
the non-being which is otherwise void of thought becomes limit,
and by virtue of this limit
the something refers to an other despite itself.
In the sphere of reflection, on the other hand,
it is the negative that refers essentially to a positive,
and is thereby determined;
a negative is no longer that indeterminate non-being,
for it is posited to be only to the extent
that the positive stands over against it,
and as third comes their ground;
the negative is thus held circumscribed in a sphere
within which the non-being of one is something determinate.
But it is all the more in the absolutely
fluid continuity of the concept
that the not is immediately a positive,
and the negation is not just determinateness
but is taken up into universality
and is posited as identical with it.
The non-universal is therefore directly the particular.

2. Since negation has to do with the connection of judgment,
and we are considering the negative judgment still as such,
the latter is in the first instance still a judgment;
we thus have the relation of subject and predicate,
or of singularity and universality,
and their connection, the form of the judgment.
The subject, as the immediate underlying basis,
remains untouched by the negation;
it retains, therefore,
its determination of having a predicate,
or its reference to the universality.
Consequently, what is negated in the predicate
is not the universality as such,
but the abstraction or the determinateness
of the predicate that appeared as
content in contrast to that universality.
The negative judgment is not, therefore, total negation;
the universal sphere which contains the predicate remains standing;
the connection of subject and the predicate is
therefore still essentially positive;
the yet remaining determination of
the predicate is no less connection.
When it is said that, for instance,
the rose is not red,
only the determinateness of the predicate is thereby denied
and thus separated from the universality
which equally attaches to it;
the universal sphere, color, is retained;
if the rose is not red,
it is nonetheless assumed that it has a color,
though another color.
From the side of this universal sphere,
the judgment is still positive.

“The singular is a particular.”
This positive form of the negative judgment
immediately expresses that the particular contains universality.
In addition, it also expresses that
the predicate is not just a universal
but also one which is still determinate.
The negative form contains the same,
for although the rose, for instance, is not red,
it is supposed, nevertheless, not only
still to retain the universal sphere of color as predicate,
but to have some other determinate color as well;
the singularity of determinateness
of the rose is therefore only sublated;
and not only is the universal sphere left standing
but determinateness too is retained,
although transformed into an indeterminate determinateness,
a universal determinateness, that is to say, into particularity.

3. The particularity that has resulted as
the positive determination of the negative judgment is
the term mediating singularity and universality;
so the negative judgment is now that which
provides in general the mediation for the third step,
that of the reflection of the judgment of existence into itself.
This judgment is according to its objective meaning
only the moment of the alteration of accidents,
or, in the sphere of existence,
of the singularized properties of the concrete.
Through this alteration, the full determinateness of the predicate,
or the concrete, emerges as posited.

“The singular is particular” is
what the positive expression of the negative judgment says.
But the singular is also not particular,
for particularity is of wider extension than singularity;
it is a predicate, therefore,
that does not correspond to the subject,
one in which the latter,
therefore, does not as yet have its truth.
“The singular is only a singular”:
this is a negativity that refers to nothing else,
be it positive or negative, except itself.
The rose is not a thing of some color or other,
but one that only has the one determinate color
which is the rose-color.
The singular is not an indeterminate determinate
but the determinate determinate.

This negation of the negative judgment appears,
when one starts from its positive form,
to be again a first negation.
But this is not what it is.
The negative judgment is again, in and for itself,
already the second negation or the negation of negation,
and this, what it is in and for itself, is to be posited.
To wit: the judgment negates the determinateness of the predicate
of the positive judgment, its abstract universality,
or, considered as content,
the singular quality that it possesses of the subject.
But the negation of thedeterminateness is
already the second negation,
hence the infinite turning back of
the singularity into itself.
With this, therefore, the restoration of
the concrete totality of the subject has taken place,
or rather, the subject is now for the first time
posited as singular,
for through the negation
and the sublation of that negation
it is mediated with itself.
The predicate, for its part, has thereby passed over
from the first universality to absolute determinateness
and made itself equal to the subject.
Thus the judgment says: “the singular is singular.”
From the other side, since the subject was
equally to be taken as a universal,
and since in the negative judgment the predicate,
which as against that subject is the singular,
expanded into particularity;
moreover, since now the negation
of this determinateness is equally
the purification of the universality
contained in the predicate,
this judgment also says:
“the universal is the universal.”

In these two judgments, which were
earlier obtained through external reflection,
the predicate is already expressed in its positivity.
But the negation of the negative judgment
must itself first appear in the form of a negative judgment.
It has just been shown that there still remained
in this judgment a positive connection of subject and predicate
as well the universal sphere of the latter.
From this side, the negative judgment
thus contains a universality
which is more purified of limitation
than was contained by the positive judgment
and is for this reason all the more
to be negated of the subject as a singular.
In this manner, the whole extent
of the predicate is negated,
and there is no longer any positive connection
between it and the subject.
This is the infinite judgment.

c. The infinite judgment

The negative judgment is as little of a true judgment as the positive.
But the infinite judgment which is supposed to be its truth is,
according to its negative expression, the negative infinite,
a judgment in which even the form of judgment is sublated.
But this is a nonsensical judgment.
It ought to be a judgment,
and hence contains a connection of subject and predicate;
but any such connection ought not at the same time to be there.
The name of the infinite judgment does indeed
occur in the common textbooks of logic,
but without any clarification as to its meaning.
Examples of negatively infinite judgments are easy to come by.
It is a matter of picking determinations,
one of which does not contain not just
the determinateness of the other
but its universal sphere as well,
and of combining them negatively as subject and predicate,
as when we say, for example,
that spirit is not red, yellow, etc.,
is not acid, not alkali, etc.,
or that the rose is not an elephant,
the understanding is not a table, and the like.
These judgments are correct or true,
as it is said, and yet, any such truth
notwithstanding, nonsensical and fatuous.
Or, more to the point, they are not judgments at all.
A more realistic example of
the infinite judgment is the evil action.
In civil litigation, when a thing is negated
as the property of another party,
it is still conceded that the same thing would indeed belong
to that party if the latter had a right to it.
It is only under the title of right
that the possession of it is challenged;
in the negative judgment, therefore,
the universal sphere, “right,” is
still acknowledged and maintained.
But crime is the infinite judgment that negates,
not only the particular right,
but the universal sphere, the right as right.
It has correctness, in the sense that it is an effective action,
but since it stands in a thoroughly negative fashion
with respect to the morality that constitutes its sphere,
it is nonsensical.

The positive element of the infinite judgment,
the negation of the negation,
is the reflection of singularity into itself
by virtue of which the singularity is
first posited as the determinate determinate.
“The singular is singular” is what
the infinite judgment said
according to that reflection.
In the judgment of existence,
the subject is as the immediate singular,
hence more of just a something in general.
Through the mediation of
the negative and infinite judgment,
it is posited as singular for the first time.
The singular is thus posited as
expanding into its predicate,
which is identical with it;
to the same extent, therefore,
universality is also no longer
anything immediate but a summing of distincts.
The positively infinite judgment equally says,
“the universal is universal,”
and in this the universal is posited
also as a turning back into itself.

Now through the reflection of
the judgment determinations into themselves,
the judgment has sublated itself;
in the negatively infinite judgment,
the difference is, so to speak, too great
for it still to remain a judgment;
subject and predicate have no
positive connection whatsoever to each other;
in the positively infinite judgment,
on the contrary, only identity is present,
and because of this total lack of difference
there is no longer a judgment.

More precisely, it is the judgment of existence
that has sublated itself
and, consequently, there is posited
what the copula of the judgment contains,
namely that in its identity
the qualitative extremes are sublated.
But since this unity is the concept,
it is immediately torn apart
and is a judgment,
but one whose terms are
no longer immediately determined
but are reflected into themselves.
The judgment of existence has passed over
into the judgment of reflection.

B. THE JUDGMENT OF REFLECTION

In the judgment that has now arisen,
the subject is a singular as such;
and similarly, the universal is
no longer an abstract universality,
or a singular property,
but is posited as a universal
that has collected itself together
into a unity through the connection of different terms,
or, regarded from the standpoint of the content
of diverse determinations in general,
as the coalescing of
manifold properties and concrete existences.
If examples of predicates
of judgments of reflection are to be given,
they must be of another kind
than for the judgments of existence.
It is only in the judgment of reflection
that we first have a determinate content strictly speaking,
that is, a content as such;
for the content is the form determination
reflected into identity as distinct from the form
in so far as  this is a distinct determinateness
as it still is as judgment.
In the judgment of existence,
the content is merely an immediate,
or abstract, indeterminate content.
These may therefore serve as examples
of judgments of reflection:
the human being is mortal,
things are perishable,
this thing is useful, harmful;
hardness, elasticity of bodies, happiness, etc.,
are predicates of this particular kind.
They express an essentiality
which is however a relational determination,
or a comprehensive universality.
This universality, which will further determine
itself in the movement of the judgment of reflection,
is still distinct from the universality
of the concept as such;
although it is no longer
the abstract universality of the qualitative judgment,
it still has a connection to the immediate
from which it proceeds
and has the latter at the basis of its negativity.
The concept determines immediate existence,
in the first instance,
to relational determinations that extend
across the diverse multiplicity of concrete existence,
so that the true universal is indeed
the inner essence of that multiplicity,
but is such in the sphere of appearance,
and this relative nature or even its mark
is not as yet the element of the multiplicity
that exists in and for itself.

It may seem fitting to define the
judgment of reflection as a judgment of quantity,
just as the judgment of existence was defined
also as qualitative judgment.
But just as the immediacy in the latter was not just there,
but was an immediacy which is also essentially mediated and abstract,
so, here also, that same immediacy which is now sublated
is not just sublated quality,
and therefore not merely quantity;
on the contrary, just as quality is
the most external immediacy,
so is quantity, in the same way,
the most external determination
belonging to mediation.

Also to be noted concerning the determination as it appears
in the movement of the judgment of reflection is
that, in the judgment of immediate existence,
the movement of the determination showed itself in the predicate,
for this kind of judgment was in  the determination
of immediacy and its subject,
therefore, appeared as the underlying basis.
For a similar reason, in the judgment of reflection
the onward movement of determination runs
its course in the subject,
for this judgment has the reflected in-itselfness
for its determination.
Hence the essential is here the universal or the predicate,
and it is the latter, therefore,
that constitutes the basis against which
the subject is to be measured and determined accordingly.
Yet the predicate also receives a further determination
through the further development of the form of the subject,
but it receives it indirectly,
whereas the progression of the subject manifests itself,
for the reason just given,
as a direct advance in determination.

As regards the objective signification of the judgment,
the singular enters into existence by virtue of its universality,
but it does so in an essential determination which is relational,
in an essentiality that maintains itself across the manifold of appearance;
the subject is supposed to be that which is determined in and for itself;
this is the determinateness which it has in its predicate.
The singular, for its part, is reflected into this predicate
which is its universal essence;
to this extent, the subject is a concrete existence
and a phenomenal something.
In this judgment, the predicate no longer inheres in the subject,
for it is rather the implicit being under which the singular
subject is subsumed as an accidental.
If the judgments of existence can also
be defined as judgments of inherence,
then the judgments of reflection are
by contrast judgments of subsumption.

a. The singular judgment

Now the immediate judgment of reflection is again,
“the singular is universal,”
but with the subject and predicate
in the signification just explained.
More accurately, therefore,
it can also be expressed thus,
“this is an essential universal.”

But a “this” is not an essential universal.
That positive judgment
(positive according to form must
as judgment be taken negatively.
But inasmuch as the judgment of reflection is
not merely something positive,
the negation does not directly affect the predicate)
a predicate which does not inhere in the subject
but is rather its implicit being.
On the contrary, it is the subject
that is alterable and needs determination.
The negative judgment is therefore
to be understood as saying:
“'not a this' is a universal of reflection”;
such an in-itself has a more universal concrete existence
than it would have in a “this.”
Accordingly, the singular judgment has
its proximate truth in the particular judgment.

b. The particular judgment

The non-singularity of the subject
that must be posited in the first judgment of reflection
instead of the subject's singularity is particularity.
But particularity is determined in the judgment of reflection
as essential singularity;
particularity cannot be, therefore,
a simple, abstract determination
in which the singular would be sublated
and the concrete existent dissolved,
but is rather only an extension of
this singular in external reflection.
Thus the subject is: “these ones,”
or “a particular number of singulars.”

The judgment, “some singulars are a universal of reflection,”
appears at first to be a positive judgment,
but it is just as well also negative;
for “some” contains universality
and may, accordingly, be regarded as comprehensive;
but since it is particularity,
it is equally disproportionate
with respect to universality.
The negative determination which the subject has obtained
through the transition of the singular judgment also is,
as we have shown above, the determination
of the connection, the copula.
Implicated in the judgment,
“some humans are happy,”
is the immediate consequence:
“some humans are not happy.”
When some things are useful,
then, precisely for that reason,
there also are some that are not useful.
The positive and the negative judgment
no longer fall outside one another,
but the particular immediately contains both at the same time,
precisely because it is a judgment of reflection.
But the particular judgment is therefore indeterminate.

If, in the example of such a judgment,
we consider further the subject,
“some humans,” “some animals,” etc.,
we find that it contains,
besides the particular form determination of “some,”
also the content determination of “humans,” etc.
By the subject of the singular judgment one could mean,
“this human,” a singularity that properly pertains to external pointing;
it would best be expressed, therefore, by something like “Gaius.”
But the subject of the particular judgment
can no longer be “some Gaiuses,”
for Gaius is supposed to be a singular as singular.
To the “some,” therefore, there is added a more universal content,
say “humans,” “animals,” etc.
This is not a mere empirical content,
but one which is determined by the form of the judgment;
it is universal, that is, because “some” contains universality,
and the latter must at the same time be separated
from the singulars which the reflected singularity has as a basis.
More precisely, this universality is also the universal nature
or species “human,” “animal”
the universality which is the result of
the judgment of reflection, but anticipated;
just as the positive judgment,
since it has the singular for subject,
also anticipates the determination
which is the result of the judgment of existence.

Thus the subject that contains the singulars,
their connection to particularity,
and the universal nature,
is already posited as the totality of
the determinations of the concept.
But, to be precise, this consideration is an external one.
What is at first already posited in the subject
by virtue of its form, in reciprocal connection,
is the extension of the “this” to particularity;
but this generalization is not commensurate to the “this”;
the latter is perfectly determinate,
but “some” is indeterminate.
The extension ought to be appropriate to the “this”
and therefore, in conformity with it,
it ought to be completely determined;
such an extension is totality,
or, in the first instance,
universality in general.

This universality has the “this” for its basis,
for the singular is here the singular reflected into itself;
its further determinations run their course,
therefore, outside it, and just as for this reason
particularity determined itself as a “some,”
so the universality which the subject has attained
is an “allness,” and thus the particular judgment
has passed over into the universal.

c. The universal judgment

The universality of the subject of the universal judgment is
the external universality of reflection, “allness”;
the “all” is the all of all the singulars
in which the singular remains unchanged.
This universality is therefore only
a commonality of self-subsisting singulars,
an association of such singulars
as comes about only by way of comparison.
This is the association that first comes to mind
at a subjective level of representation
when there is talk of universality.
The most obvious reason given for viewing a determination
as universal is because it fits many.
Also in analysis is this conception of universality
the one most prevalent, as when, for instance,
the development of a function in a polynomial is
taken to have greater universal value than
its development in a binomial,
because the polynomial displays
more single terms than the binomial.
The demand that the function should be resolved
in its full universality would require, strictly speaking,
a pantonomial, the exhausted infinity.
But here is where the limitation of
that demand becomes apparent,
and where the display of the infinite number of terms
must rest satisfied with the ought it commands,
and therefore also with a polynomial.
But in fact the binomial is already the pantonomial
in those cases where the method or the rule concerns
only the dependence of one member on another,
and the dependence of several terms on
those that precede them does not particularize itself
but remains one and the same underlying function.
It is the method or the rule
which is to be regarded as the true universal;
in the progress of the development
or in the development of a polynomial,
the rule is only repeated,
so that it gains nothing in universality
through the increased number of terms.
We have already spoken earlier of
the bad infinity and its deception;
the universality of the concept is the achieved beyond,
whereas that bad infinity remains afflicted with a beyond
which is unattainable but remains a mere progression to infinity.
If it is allness that universality brings to mind,
a universality that ought to be exhausted in singulars as singulars,
then there has been a relapse into that bad infinity;
or else it is mere plurality which is taken for allness.
But plurality, however great it might be,
remains inescapably only particularity:
it is not allness.
Yet there is in all this an obscure intimation of
the universality of the concept
as it exists in and for itself;
it is the concept that violently strives
to reach beyond the stubborn singularity
to which pictorial representation clings
and beyond the externality of its reflection,
passing off allness as totality
or rather as the category of the in-and-for-itself.

This is apparent in other ways as well in the allness
which is above all empirical universality.
Inasmuch as the singular is presupposed as something immediate
and is therefore pre-given and externally picked,
the reflection which collects it into an allness is
equally external to it.
But because the singular, as a “this,” is
absolutely indifferent to such a reflection,
the universality and the collected singularity
cannot combine to form a unity.
The empirical allness thus remains a task;
it is an ought which, as such,
cannot be represented  in the form of being.
Now an empirically universal proposition,
for nevertheless such are advanced,
rests on the tacit agreement that,
if no instance of the contrary can be adduced,
a plurality of cases ought to count for an allness;
or that a subjective allness,
namely the known cases,
may be taken for an objective allness.

Now a closer examination of
the universal judgment before us shows
that the subject, as we have just noted,
contains the achieved universality as presupposed;
it even contains it as posited in it.
“All humans” expresses, first, the species “human”;
second, this species in its singularization,
but in such a way that the singulars are at the same time
expanded to the universality of the species;
conversely, through this conjunction with singularity,
the universality is just as perfectly determined as singularity,
and the posited universality has thereby
become equal to what was presupposed.

But, strictly speaking, we should not anticipate the presupposed
but should rather consider the result for itself
in the form determination.
The singularity, inasmuch as it is expanded to allness,
is posited as negativity,
and this is identical self-reference.
It has not remained, therefore, that first singularity
(of Gaius, for instance)
but is a determination identical with universality,
or the absolute determinateness of the universal.
That first singularity of the singular judgment was not
the immediate singularity of the positive judgment,
but came about through the dialectical movement of
the judgment of existence in general;
it was already determined to be the negative identity of
the determinations of that judgment.
This is the true presupposition in the judgment of reflection;
as contrasted to the positing that runs its course in that judgment,
that first determinatenessof singularity was the latter's in-itself;
consequently, what singularity is in itself,
through the movement of the judgment of reflection is now posited,
posited, that is, as the identical self-reference of the determinate.
Therefore the reflection that expanded
the singularity to allness is not external to it;
on the contrary, it only makes explicit what was before implicit.
Hence the result is in truth the objective universality.
The subject has thus shed the form determination of the judgment of reflection
that made its way from the “this” to the “allness” through the “some.”
Instead of “all humans,” we now have to say “the human being.”

The universality that has thereby arisen is the genus,
or the universality which is concrete in its universality.
The genus does not inhere in the subject;
it is not one property of it or a property at all;
it contains all singular determinacies
dissolved into its substantial purity.
Because it is thus posited as this negative self-identity,
it is for that reason essentially subject,
but one that is no longer subsumed under its predicate.
Consequently the nature of the judgment of reflection is
now altogether altered.

This judgment was essentially a judgment of subsumption.
The predicate was determined, in contrast to its subject,
as the implicit universal;
according to its content, it could be taken as an
essentially relational determination
or also as a mark,
a determination which makes the subject
essentially only an appearance.
But when determined to objective universality,
the subject ceases to be subsumed
under such a relational determination
or the collecting grasp of reflection;
with respect to this objective universality,
a predicate of this sort is rather a particular.
The relation of subject and predicate
has thus reversed itself,
and to this extent the judgment has
at this point sublated itself.

This sublation of the judgment coincides
with what the determination of the copula becomes,
as we still have to consider;
the sublation of the determinations of judgment
and their transition into the copula are one and same.
For inasmuch as the subject has raised itself to universality,
it has become in this determination equal to the predicate
which, as the reflected universality,
also contains particularity within itself;
subject and predicate are therefore identical, that is,
in the copula they have come to coincide.
This identity is the genus
or the nature of a thing in and for itself.
Inasmuch as this identity, therefore, again divides,
it is the inner nature by virtue of which
a subject and predicate are connected to each other.
This is a connection of necessity
wherein the two terms of the judgment are only
unessential distinctions.
That what belongs to all the singulars of a genus
belongs to the genus by nature,
is an immediate consequence.
It expresses what we have just seen;
that the subject, e.g. “all humans,”
sheds its form determination
and “the human being” is what it should say instead.
This combination, implicit and explicit,
constitutes the basis of a new judgment,
the judgment of necessity.

C. THE JUDGMENT OF NECESSITY

The determination to which universality has advanced is,
as we have seen, the universality
that exists in and for itself
or the objective universality that
in the sphere of essence corresponds to substantiality.
It is distinguished from the latter
because it belongs to the concept
and for this reason is not only the inner
but also the posited necessity of its determinations,
or in other words, the distinction is immanent to it,
whereas substance has its distinction only in its accidents,
does not have it as a principle within it.

In the judgment now, this objective universality is posited,
first, posited with this determinateness
as essential to it, immanent to it;
second, posited with it as diverse from it,
a particularity for which the said universality
constitutes the substantial basis.
In this way the universality is determined
as genus and species.

a. The categorical judgment

The genus essentially divides
or repels itself into species;
it is genus only in so far as
it comprehends the species under it;
the species is a species only in so far as,
on the one side, it exists in singulars,
and, on the other side, it possesses
in the genus a higher universality.
Now the categorical judgment has
for predicate such a universality
as in it the subject possesses its immanent nature.
But the categorical judgment is itself
the first or the immediate judgment of necessity;
consequently, the determinateness of the subject,
by virtue of which the latter is a singular
as contrasted to the genus or the species,
belongs to the immediacy of external concrete existence.
But objective universality also has here
only its first immediate particularization;
on the one hand, therefore,
it is itself a determinate genus
with respect to which there are higher genera;
on the other hand, it is not the most proximate genus,
that is, its determinateness is not directly the
principle of the specific particularity of the subject.
But what is necessary in it is
the substantial identity of subject and predicate,
in view of which the distinguishing mark of each is
only an unessential positedness or even only a name;
in its predicate, the subject is
reflected into its being-in-and-for-itself.
Such a predicate ought not to be classed
with the predicates of the preceding judgments.
For example, to throw together into one class
these judgments:

The rose is red,
The rose is a plant,
or This ring is yellow,
It is gold,

and thus to take such an external property
as the color of a flower
as a predicate equal to its vegetable nature,
is to overlook a difference which
the dullest mind would not miss.
The categorical judgment, therefore, is
definitely to be distinguished from
the positive and the negative judgment;
in these, what is said of the subject is
a singular accidental content;
in the former, the content is
the totality of the form reflected into itself.
In this content, therefore, the copula has
the meaning of necessity,
whereas in that of the other two it has only
the meaning of abstract, immediate being.

The determinateness of the subject,
which makes it a particular
with respect to the predicate,
is at first still something contingent;
subject and predicate are not connected
with necessity by the form or the determinateness;
the necessity is therefore still an inner one.
The subject is subject, however, only as a particular,
and to the extent that it possesses objective universality,
it has to possess it essentially in accordance with
that at first immediate determinateness.
The objective universal, in determining itself,
that is, in positing itself in a judgment,
is in a connection of identity with this repelled
determinateness as such, essentially,
that is, this determinateness is not to
be posited as merely accidental.
Only through this necessity of its immediate being
does the categorical judgment conform
to its objective universality
and, in this way, has passed over
into the hypothetical judgment.

b. The hypothetical judgment

“If A is, then B is”;
or “The being of A is not its own being
but the being of an other, of B.”
What is posited in this judgment is
the necessary connectedness of immediate determinacies,
a connectedness which in the categorical judgment
is not yet posited.
There are here two immediate,
or externally contingent concrete existences,
of which in the categorical judgment
there is at first only one, the subject;
but since one is external to the other,
this other is immediately also external
with respect to the first.
On account of this immediacy,
the contents of both sides are still
indifferent to each other;
at first, therefore, this judgment is
a proposition of empty form.
Now, first, the immediacy is as such
indeed self-subsistent, a concrete being;
but, second, what is essential is its connection;
this being is therefore just as much mere possibility;
the hypothetical judgment does not say
either that A is, or that B is,
but only that if the one is, then the other is;
only the connectedness of the extremes is
posited as existing, not the extremes themselves.
Indeed, each extreme is posited in this necessity
as equally the being of an other.
The principle of identity asserts that
A is only A, not B;
and B is only B, not A.
In the hypothetical judgment, on the contrary,
the being of finite things is posited through the concept
in accordance with their formal truth,
namely that the finite is its own being,
but equally is not its own being
but is the being of an other.
In the sphere of being,
the finite alters and comes to be an other.
In the sphere of essence,
it is appearance;
its being is posited to consist in
the reflective shining of an other in it,
and the necessity is the inner connection
not yet posited as such.
But the concept is this:
that this identity is posited;
that the existent is not abstract self-identity
but concrete self-identity
and is, immediately within it,
the being of an other.

The hypothetical judgment can be more closely determined
in terms of the relations of reflection as
a relation of ground and consequence,
condition and conditioned, causality etc.
Just as substantiality is present in the
categorical judgment in the form of its concept,
so is the connectedness of causality
in the hypothetical judgment.
This and the other relations all recur in it,
but they are there essentially only as moments of
one and the same identity.
However, in it they are as yet not opposed as
singular or particular and universal
according to the determinations of the concept,
but are only as moments in general at first.
The hypothetical judgment, therefore, has a shape
which is more that of a proposition;
just as the particular judgment is of indeterminate content,
so is the hypothetical of indeterminate form,
for the determination of its content does not conform
to the relation of subject and predicate.
Yet the being, since it is the being of the other,
is for that very reason in itself
the unity of itself and the other,
and therefore universality;
by the same token it is in fact only a particular,
for it is a determinate being
and does not refer in its determinateness merely to itself.
But it is not the simple, abstract particularity that is posited;
on the contrary, through the immediacy
which the determinacies possess,
the moments of particularity are differentiated;
at the same time, through the unity of these moments
as constituted by their connection,
the particularity is also their totality.
In truth, therefore, what is posited in this judgment is
universality as the concrete identity of the concept
whose determinations do not have any subsistence of their own
but are only particularities posited in that identity.
So it is the disjunctive judgment.

c. The disjunctive judgment

In the categorical judgment, the concept is
objective universality and an external singularity.
In the hypothetical, the concept manifests its presence
in this externality, in its negative identity.
Through this identity, the objective universality
and the external singularity obtain the determinateness,
now posited in the disjunctive judgment,
which in the hypothetical they possess immediately.
Hence the disjunctive judgment is objective universality
at the same time posited in union with the form.
It thus contains,
first, the concrete universality
or the genus in simple form, as the subject;
second, the same universality
but as the totality of its differentiated determinations.
“A is either B or C.”
This is the necessity of the concept in which,
first, the self-identity of the two extremes is
of the same extent, content, and universality.
Second, they are differentiated according
to the form of conceptual determination,
but, because of that identity,
this determination is a mere form.
Third, the identical objective universality appears
for that reason reflected into itself
as against the non-essential form,
as a content which however has the determinateness of form in it,
once as the simple determination of genus;
then again, as this determinateness developed in its difference,
and in this way it is the particularity of the species
and their totality, the universality of the genus.
The particularity constitutes in its development the predicate,
because, in containing the whole universal
sphere of the subject, and in containing it,
however, also in the articulation of particularity,
it is to that extent the greater universal.

Upon closer consideration of this particularization,
it is the genus that constitutes first of all
the substantial universality of the species;
the subject is thus B as well as C;
this “as well as” indicates the positive identity of
the particular with the universal;
this objective universal maintains itself
fully in its particularity.
Secondly, the species mutually exclude one another;
“A is either B or C”;
for they are the specific difference of the universal sphere.
This “either or” is their negative connection.
In this negative connection
they are just as identical as in the positive;
the genus is their unity as
a unity of determinate particulars.
If the genus were an abstract universality,
as in the judgments of existence,
then the species would also have to
be taken as diverse and mutually indifferent;
this universality, however, is
not the external one that arises only
through comparison and abstraction
but is, on the contrary, the universality
which is immanent to the genus and concrete.
An empirical disjunctive judgment is without necessity;
A is either B or C or D, etc.,
because the species B, C, D, etc.,
are found beforehand;
strictly speaking, therefore, there is
no question here of an “either or,”
for the completeness of these species is only a subjective one;
of course, one species excludes the other,
but the “either or” excludes
every other species and excludes within itself an entire sphere.
This totality has its necessity in
the negative unity of the objective universal
which has dissolved singularity within itself
and possesses, immanent in it,
the simple principle of differentiation
by which the species are determined and connected.
The empirical species, on the contrary,
have their differences in some accidentality
or other which is a principle external to them
and is not, therefore, their principle,
and consequently also not the immanent
determinateness of the genus;
for this reason, they are also not reciprocally
connected according to their determinateness.
Yet it is by virtue of their determinateness
that the species constitute the universality of the predicate.
Here is where the so-called
contrary and contradictory concepts
should find their proper place,
for the disjunctive judgment is
where the essential difference of the concept is posited;
but here they also equally find their truth,
namely that contrariness and contradictoriness are themselves
differentiated both as contraries and as contradictory.
Species are contrary inasmuch as they are merely diverse,
that is to say, inasmuch as they possess an immediate existence
as subsisting in and for themselves by virtue of
the genus which is their nature.
They are contradictory, inasmuch as they exclude one another.
But each of these determinations is
by itself one-sided and void of truth.
In the “either or” of the disjunctive judgment,
their unity is posited as their truth,
which is that the independent subsistence of
the species as concrete universality is
itself also the principle of the negative unity
by which they mutually exclude one another.

Through the identity just demonstrated of
subject and predicate in accordance
with the negative unity,
the genus is determined in the disjunctive judgment
as the proximate genus.
This expression indicates at first
the mere quantitative difference of
the more or less determinations
which a universal contains as contrasted to
a particularity coming under it.
On this account, which is the
truly proximate genus remains contingent.
But then, if the genus is taken as a universal
arrived at by the mere abstraction of determinations,
it cannot strictly speaking form a disjunctive judgment;
for it is contingent whether, as it were,
there is still left in it the determinateness
that constitutes the principle of the “either or”;
the genus would not be displayed in the species
according to its determinateness,
and these would only be capable of contingent completeness.
In the categorical judgment, the genus stands at first
over against the subject only in this abstract form,
is not, therefore, necessarily its proximate genus
and, to this extent, is external to it.
But when the genus is a concrete,
essentially determined universality,
then, as simple determinateness,
it is the unity of the moments of the concept;
moments that, only sublated in that simplicity,
have their real difference in the species.
Hence the genus is the proximate genus of a species,
for the latter possesses its specific difference
in the essential determinateness of the genus,
and the species have as such the determination
differentiating them in the nature of the genus.

What we have just considered constitutes
the identity of subject and predicate
from the aspect of determinateness in general.
This is an aspect that was posited by the hypothetical judgment,
the necessity of which is an identity of immediate and diverse things
and is, therefore, essentially a negative unity.
It is this negative unity that in principle
separates subject and predicate
but is now posited as itself differentiated –
in the subject, as simple determinateness;
in the predicate, as totality.
That parting of subject and predicate is
the difference of the concept;
the totality of the species in the predicate can
then be none other than this difference.
The reciprocal determination of the disjunctive terms is
therefore hereby given.
It reduces to the difference of the concept,
for it is the concept alone that disjoins itself
and manifests in its determination its negative unity.
Of course, the species comes up for consideration here only under the
aspect of its simple conceptual determinateness, not according to the shape
in which, proceeding from the idea,
it steps into a further self-subsistent reality.
This reality is of course dropped in the simple principle of the genus;
but the essential differentiation must be a moment of the concept.
In the judgment here considered, it is really now
the concept's own progressive determination
that itself posits its disjunction,
just as was the case for the concept itself,
as we saw when it was determined in and for itself
and was differentiated into determinate concepts.
Now because the concept is the universal,
the positive as well as the negative totality of the particulars,
for that reason it is immediately itself also
one of its disjunctive members;
the other member, however, is
this universality resolved into its particularity,
or the determinateness of the concept as determinateness,
in which the very universality displays itself as totality.
If the disjunction of a genus into species has
not yet attained this form,
this is proof that the disjunction has not risen
to the determinateness of the concept
and has not proceeded from it.
Color is either violet, indigo, blue, green, yellow, orange, or red;
even empirically, the confusion and impurity of
such a disjunction are at once apparent;
it is a barbarism even from this standpoint.
If color is conceived as the concrete
unity of light and darkness,
then this genus has within it the determinateness
that constitutes the principle of
its particularization into species.
Of these, however, one must be the utterly simple color
that holds the opposition in balance,
contained and negated in the color's intensity;
the relation of the opposition of light and darkness
must then take its place over against it,
and, since this relation is a natural phenomenon,
the indifferent neutrality of the opposition must be further added to it.
Taking for genus such mixtures as violet, and orange,
or shades of difference like indigo blue and light blue,
betrays a totally inconsiderate procedure
that shows too little reflection even for empiricism.
But this is not the place to discuss the different
and more finely determined forms that disjunction
may indeed assume in the element of nature or spirit.

In the first instance, the disjunctive judgment has
the members of the disjunction in the predicate.
But the judgment is itself equally disjoined;
its subject and predicate are the members of the disjunction;
they are the moments of the concept
posited in their determinateness
but at the same time as identical;
identical, (a) in the objective universality
which is in the subject as the simple genus,
and in the predicate as the universal sphere
and totality of the moments of the concept;
and (b) in the negative unity,
the developed connectedness of necessity,
in accordance with which the simple determinateness
in the subject has fallen apart
into the difference of the species
and these, in this very difference,
have their essential connection and self-identity.

This unity, the copula of this judgment
in which the extremes have come together
through their identity,
is thus the concept itself,
indeed the concept as posited;
the mere judgment of necessity has
thereby risen to the judgment of the concept.

D. THE JUDGMENT OF THE CONCEPT

To know how to form judgments of existence,
such as “the rose is red,” “the snow is white,” etc.,
hardly counts as a sign of great power of judgment.
The judgments of reflection are more in the nature of propositions;
to be sure, in the judgment of necessity
the subject matter is present in its objective universality,
but it is only in the judgment now to be considered
that its connection with the concept is to be found.
The concept is at the basis of this judgment,
and it is there with reference to the subject matter,
as an ought to which reality may or may not conform.
This is the judgment, therefore, that first contains true adjudication;
the predicates, “good,” “bad,” “true,” “right,” etc.,
express that the fact is measured against the concept
as an ought which is simply presupposed,
and is, or is not, in agreement with it.

The judgment of the concept has been called
the judgment of modality,
and has been regarded as containing
the form of the connection of subject and predicate
as this obtains in an external understanding,
and as concerned with the value of the copula
only in connection with thought.
Accordingly, judgment is said to be problematic
when the affirmation or negation is taken
as optional or possible;
assertoric, when it is taken as true, that is, actual,
and apodictic when it is taken as necessary.
It is easy to see why it would be an easy step
in this judgment to go outside the judgment itself
and to regard its determination as something merely subjective.
For it is the concept here, the subjective,
that comes into play again in judgment
and relates to an immediate actuality.
But this subjectivity is not to be confused
with external reflection,
which is of course also something subjective
but in a different sense than the concept itself;
on the contrary, the concept that has again emerged
out of the disjunctive judgment is
the very opposite of a mere mode or manner.
The earlier judgments are subjective in this sense,
for they rest on an abstract one-sidedness
in which the concept is lost.
But the judgment of the concept is instead objective
and, as contrasted with the others,
it is the truth,
for it rests on the concept precisely
in its determinateness as concept,
not in some external reflection
or with reference to some subjective,
that is, accidental, thought.

In the disjunctive judgment,
the concept was posited as the identity of
universal nature and its particularization,
and with that the relation of the judgment was sublated.
This concretion of universality and particularization
is at first a simple result;
it must now further develop itself into totality,
for its moments have at first collapsed into it
and do not as yet stand over against each other
in determinate self-subsistence.
The shortcoming of that result may also be stated
more incisively by saying that
although in the disjunctive judgment
the objective universality has attained completion
in its particularization,
the negative unity of the latter
has only retreated into it
and has not as yet determined itself as the third moment,
that of singularity.
But to the extent that the result is
itself negative unity,
it is already this singularity;
it is then this one determinateness alone
that must now posit its negativity,
that must part itself into extremes
and in this way concludes its development
in the syllogistic conclusion.

The proximate diremption of this unity is
the judgment in which the unity is posited
first as subject,
as an immediate singular,
and then as predicate,
as the determinate connection of its moments.

a. The assertoric judgment

The judgment of the concept is at first immediate;
as such, it is the assertoric judgment.
The subject is a concrete singular in general,
and the predicate expresses this same singular
as the connection of its actuality,
its determinateness or constitution,
to its concept.
(“This house is bad,” “this action is good.”)
More closely considered, it contains, therefore,
(a) that the subject ought to be something;
its universal nature has posited itself as
the self-subsistent concept;
(b) that particularity is something constituted
or an external concrete existence,
not only because of its immediacy,
but because it expressly differs
from its self-subsisting universal nature;
its external concrete existence, for its part,
because of this self-subsistence of the concept,
is also indifferent with respect to the universal
and may or may not conform to it.
This constitution is the singularity
which in the disjunctive judgment escapes
the necessary determination of the universal,
a determination that exists only as
the particularization of the species
and as the negative principle of the genus.
Thus the concrete universality that has come out of
the disjunctive judgment divides in the assertoric judgment
into the form of extremes to which the concept itself,
as the posited unity connecting them, is still lacking.

For this reason the judgment is so far only assertoric;
its credential is only a subjective assurance.
That something is good or bad, right, suitable or not,
hangs on an external third.
But to say that the connectedness is
thus externally posited  is the same as
saying that it is still only in itself or internal.
When we say that something is good or bad, etc.,
we certainly do not mean to say that it is good
only in a subjective consciousness
but may perhaps be bad in itself,
or that “good and bad,” “right,” “suitable,” etc.
may not be predicates of the object itself.
The merely subjective character of
the assertion of this judgment consists,
therefore, in the fact that the implicitly
present connectedness of subject and predicate
has not been posited yet,
or, what amounts to the same thing,
that it is only external;
the copula still is an immediate abstract being.

Thus the assurance of the assertoric judgment can
with right be confronted by an opposing one.
When the assurance is given that
“this action is good,”
the opposite,
“this action is bad,”
has equal justification.
Or, considering the judgment in itself,
since its subject is an immediate singular,
in this abstraction it still does not have,
posited in it, the determinateness
that would contain its connection
with the universal concept;
it still is a contingent matter, therefore,
whether there is or there is not conformity to the concept.
Essentially, therefore, the judgment is problematic.

b. The problematic judgment

The problematic judgment is the assertoric judgment
in so far as the latter must be taken
positively as well as negatively.
According to this qualitative side,
the particular judgment is likewise a problematic one,
for it has positive just as much as negative value
(equally problematic is also the
being of the subject and predicate in the hypothetical judgment),
and also posited through this side is
that the singular judgment and the categorical
are still something merely subjective.
In the problematic judgment as such, however,
this positing is more immanent than it is in these others,
for in it the content of the predicate is
the connection of the subject to the concept;
here, therefore, the determination of the immediate
as something contingent is itself present.

Whether the predicate ought to be
or not to be coupled with a certain subject
appears at first only as problematic,
and to this extent the indeterminateness
falls on the side of the copula.
The predicate has no determination
to gain from this coupling,
since it is already the objective, concrete universality.
The problematic element falls, therefore,
on the immediacy of the subject,
which is thereby determined as a contingency.
But further, we must not for that reason
abstract from the singularity of the subject;
purified of such a singularity,
the subject would be only a universal,
whereas the predicate entails precisely this,
that the concept of the subject ought to be posited
with reference to its singularity.
We may not say, “the house or a house is good,”
but, “so indeed it is in the way it is made.”
The problematic element in the subject itself
constitutes its moment of contingency,
the subjectivity of the fact it expresses
as contrasted with its objective nature or its concept,
its mere mode and manner or its constitution.

Consequently the subject is itself differentiated
into its universality or objective nature, that is, its ought,
and the particularized constitution of immediate existence.
It thereby contains the ground for being or not being
what it ought to be.
In this way, it is equated with the predicate.
Accordingly, the negativity of
the problematic character of the judgment,
inasmuch as it implicates the immediacy of the subject,
only amounts to this original partition of the latter
into its moments of universal and particular
of which it is already the unity,
a partition which is the judgment itself.

One more comment that can be made is
that both sides of the subject,
its concept and the way it is constituted,
could each be called its subjectivity.
The concept is the universal essence of a fact,
withdrawn into itself, the fact's negative self-unity;
this unity constitutes the fact's subjectivity.
But a fact is also essentially contingent
and has an external constitution;
this last may also be called its mere subjectivity,
as contrasted with the objectivity of the concept.
The fact consists just in this, that its concept,
as self-negating unity, negates its universality
and projects itself into the externality of singularity.
As this duplicity, the subject of the judgment is here posited;
the truth of those two opposite meanings of subjectivity is
that they are in one.
The meaning of subjective has itself become problematic
by having lost the immediate determinateness
that it had in the immediate judgment
and its determinate opposition to the predicate.
These opposite meanings of subjectivity
that surface even in the ratiocination of ordinary reflection
should by themselves at least call attention to the fact that
subjectivity has no truth in one of them alone.
The duplicity of meaning is the manifestation
of the one-sidedness of each when taken by itself.

When this problematic character of the judgment
is thus posited as the character of the fact,
the fact with its constitution,
the judgment itself is no longer
problematic but apodictic.

c. The apodictic judgment

The subject of the apodictic judgment
(“the house, as so and so constituted, is good,”
“the action, as so and so constituted, is right”)
includes, first, the universal,
or what it ought to be;
second, its constitution;
the latter contains the ground why a predicate of
the judgment of the concept does or does not pertain to it,
that is, whether the subject corresponds to its concept or not.
This judgment is now truly objective;
or it is the truth of the judgment in general.
Subject and predicate correspond to each other,
and have the same concept,
and this content is itself posited concrete universality;
that is to say, it contains the two moments,
the objective universal or the genus
and the singularized universal.
Here we have, therefore, the universal that is itself
and continues through its opposite,
and is a universal only in unity with the latter.
Such a universal, like “good,” “fitting,” “right,” etc.,
has an ought for its ground,
and contains at the same time
the correspondence of existence;
it is not the ought or the genus by itself,
but this correspondence which is the universality
that constitutes the predicate of the apodictic judgment.

The subject likewise contains
these two moments in immediate unity as fact.
The truth of the latter, however,
is that it is internally fractured
into its ought and its being;
this is the absolute judgment on all actuality.
That this original partition,
which is the omnipotence of the concept,
is equally a turning back into the concept's unity
and the absolute connection of “ought” and “being” to each other,
is what makes the actual into a fact;
the fact's inner connection,
this concrete identity, constitutes its soul.

The transition from the immediate simplicity of the fact
to the correspondence which is the determinate connection
of its ought and its being, the copula,
now shows itself upon closer examination
to lie in the particular determinateness of the fact.
The genus is the universal existing in and for itself
which, to that extent, appears as unconnected;
the determinateness, however, is that
which in that universality is reflected into itself
but at the same time into an other.
The judgment, therefore, has its ground
in the constitution of the subject
and is thereby apodictic.
Consequently, we now have
the determinate and accomplished copula
which hitherto consisted in the abstract “is”
but has now further developed into ground in general.
It first attaches to the subject as immediate determinateness,
but it is equally the connection to the predicate,
a predicate that has no other content than
this correspondence itself,
or the connection of the subject to the universality.

Thus the form of judgment has passed away,
first, because subject and predicate are
in themselves the same content;
but, second, because through its determinateness
the subject points beyond itself
and connects itself to the predicate;
but again, third, this connecting has
equally passed over into the predicate,
only constitutes the content of it,
and so it is the connecting as posited
or the judgment itself.
The concrete identity of the concept
that was the result of the disjunctive judgment
and constitutes the inner foundation of
the judgment of the concept
(the identity that was posited
at first only in the predicate)
is thus recovered in the whole.

On closer examination, the positive factor in this result
which is responsible for the transition of the judgment
into another form is that, as we have just seen,
the subject and predicate are in the apodictic judgment
each the whole concept.
The unity of the concept, as the determinateness
constituting the copula that connects them,
is at the same time distinct from them.
At first, it stands only on
the other side of the subject
as the latter's immediate constitution.
But since its essence is to connect,
it is not only that immediate constitution
but the universal that runs through
the subject and predicate.
While subject and predicate have the same content,
it is the form of their connection
that is instead posited through
the determinateness of the copula,
the determinateness as a universal or the particularity.
Thus it contains in itself both
the form determinations of the extremes
and is the determinate connection of the subject and predicate:
the accomplished copula of the judgment,
the copula replete of content,
the unity of the concept that re-emerges from the judgment
wherein it was lost in the extremes.
By virtue of this repletion of the copula,
the judgment has become syllogism.
