A. THE LIVING INDIVIDUAL

1. The concept of life or universal life is the immediate idea,
the concept that has an objectivity corresponding to it;
but the objectivity corresponds to it only to the extent
that the concept is the negative unity of this externality,
that is to say, posits it as corresponding to it.
The infinite reference of the concept to itself is
as negativity a self-determining,
the diremption of itself within itself
as subjective singularity
and itself as indifferent universality.
The idea of life in its immediacy is as yet
only the creative universal soul.
Because of this immediacy,
the first internal negative reference of the idea
is the self-determination of itself as concept
an implicit positing which is explicit only
as a turning back into itself;
this is the creative presupposing.
By virtue of this self-determining,
universal life is particularized;
it has thus split itself into
the two extremes of the judgment
which immediately becomes syllogism.

The determinations of the opposition are
the universal determinations of the concept,
for the splitting into two is
the affair of the concept;
the filling of them, however, is the idea.
One determination is the unity of
the concept and of reality,
which is the idea as the immediate unity
that earlier assumed the form of objectivity.
Here, however, it is in another determination.
There, it was the unity of the concept and of reality
in so far as the concept has gone over into reality
and is lost in it;
it did not stand opposite to it, or,
since it is for the reality only an inner,
it is itself for it only an external reflection.
the immediate itself in immediate mode.
Here, on the contrary, it has proceeded
only from the concept,
so that its essence is positedness,
or that it exists as negative.
It is to be regarded as the side of
the universality of the concept,
hence as an abstract universality,
essentially only inhering in the subject
and in the form of immediate being
which, posited for itself, is
indifferent to the subject.
Hence the totality of the concept
that attaches to the objectivity is,
as it were, only lent to it;
the last self-subsistence that
objectivity has over against the subject
is this being whose truth is only
that moment of the concept
in which the latter, as a presupposing,
is in the first determinateness of
a positing that exists implicitly
and is not yet as positing,
as immanently reflected unity.
Having proceeded from the idea,
self-subsisting objectivity is
therefore immediate being only as
the predicate of the judgment of
the concept's self-determination,
a being that is indeed distinct from the subject
but is at the same time essentially posited as
a moment of the concept.

According to content, this objectivity is
the totality of the concept, a totality,
however, that has the subjectivity of the concept,
or its negative unity, standing over against it,
and this subjectivity or negativity is
what constitutes the true centrality,
that is to say, the concept's free unity with itself.
This subject is the idea in the form of singularity,
as simple but negative self-identity,
the living individual.

This individual is in the first place life as soul,
as the concept of itself, fully determined within itself,
the initiating self-moving principle.
In its simplicity the concept contains determinate externality,
as a simple moment enclosed within itself.
But, further, this soul is
in its immediacy immediately external,
and has an objective being within it,
a reality which is subjugated to purpose, the immediate means,
at first the objectivity which is predicated of the subject,
but then also the middle term of the syllogism,
for the corporeity of the soul is that
whereby the soul links itself to external objectivity.
The living being has this corporeity at first as a reality
immediately identical with the concept;
to this extent, the corporeity has
this reality in general by nature.

Now because this objectivity is
the predicate of the individual
and is taken up in the subjective unity,
the earlier determinations of the objects
do not attach to it,
not the relation of mechanism or of chemism,
and even less so the reflective relations
of whole and part, and the like.
As externality, it is indeed capable of such relations,
but to that extent it is no longer a living being;
when a living thing is taken to be a whole consisting of parts,
something exposed to the action of mechanical or chemical causes,
itself a mechanical or chemical product
(whether merely as such or as also
determined by some external purpose),
then the concept is taken as external to it,
the individual itself as something dead.
Since the concept is immanent in it,
the purposiveness of the living being
is to be grasped as inner;
it is present in it as a determinate concept,
distinguished from its externality
but, in thus distinguishing itself from it,
pervading it thoroughly and self-identical.
This objectivity of the living being is the organism;
it is the means and instrument of purpose, fully purposive,
for the concept constitutes its substance;
but precisely for this reason this means and instrument
is itself the accomplished purpose
in which the subjective purpose thus
immediately closes in upon itself.
As for its externality,
the organism is a manifold,
not of parts but of members.
(a) These members exist as such
only in the individuality;
they are separable inasmuch as they are external
and can be grasped in this externality,
but as thus separated they revert to
the mechanical and chemical relations
of common objectivity.
(b) Their externality is opposed to
the negative unity of the living individuality.
This individuality is therefore the impulse
 to posit as a concretely real difference
the otherwise abstract moment of
the determinateness of the concept;
and since this concretely real
difference exists as immediate,
it is the impulse of each singular,
specific moment to produce itself
and equally to raise its particularity to universality,
to sublate the other moments external to it
and promote itself at their cost,
but no less to sublate itself
and make itself a means for the other.

2. This process of the living individuality
is restricted to itself
and still falls entirely within the individuality.
The first premise of the syllogism of external purposiveness,
where the purpose immediately refers to objectivity
and makes it a means,
was earlier taken in the sense that
although in it the purpose remains self-equal
and has gone back into itself,
the objectivity has not yet sublated itself within,
and consequently the purpose is not
in it in and for itself
but becomes such only in the conclusion.
The process of the living being
with itself is this same premise,
but in so far as the premise is also the conclusion,
in so far as the immediate reference of
the subject to the objectivity,
by virtue of which the latter
becomes means and instrument,
is at the same time the negative unity
of the concept within itself,
the purpose realizes itself in this externality
by being the subjective power over it
and the process in which the externality
displays its self-dissolution
and its return into this
negative unity of the purpose.
The unrest and the mutability
of the external side of the living being is
the manifestation in it of the concept,
and the concept,
as in itself negativity,
has objectivity in so far as
this objectivity's indifferent subsistence
shows itself to be self-sublating.
Thus the concept produces itself
through its impulse in such a way
that the product, being its essence,
is itself the producing factor:
is product, in other words, only as an externality
that equally posits itself negatively,
or as the process of production.

3. Now the idea as just considered is
the concept of the living subject and of its process;
the determinations that stand in relation to one another are
the self-referring negative unity of the concept
and the objectivity which is the concept's means
but also where the concept has returned into itself.
But since these moments of the idea of life
do not go past the concept of life,
they are not the determinate conceptual moments of the living
individual in its reality.
This individual's objectivity or its corporeity
is a concrete totality;
those moments are the sides out of which
the living reality constitutes itself;
they are not, therefore, the moments of this reality
as already constituted by the idea.
But the living objectivity of the individual,
since as objectivity it is ensouled by the concept
and has the latter for its substance,
has also in it, for its essential difference,
such determinations as pertain to the concept,
universality, particularity, and singularity;
hence the shape in which the determinations in it are
externally differentiated is divided or incised (insectum)
in accordance with these.

Thus that shape is in this instance universality,
the purely internal pulsating of living reality, sensibility.
The concept of universality, as we saw it earlier,
is the simple immediacy which is such, however,
only as inherently absolute negativity.
This concept of absolute difference with
its negativity dissolved into simplicity and self-equal,
is brought to intuition in sensibility.
It is the in-itselfness not as abstract simplicity,
but as an infinite determinable receptivity
that does not become in its determinateness
anything manifold and external
but is absolutely reflected into itself.
Determinateness is present in this universality as simple principle;
the singular external determinateness, the so-called impression,
goes back from its external and manifold determination
into this simplicity of self-feeling.
Sensibility may therefore be regarded
as the external existence of the inward soul,
for it takes in all externality but reduces it
to the complete simplicity of self-equal universality.

The second determination of the concept is particularity.
This is the moment of posited difference,
the opening up of the negativity otherwise
locked up in simple self-feeling
or present in it as abstractly ideal,
not yet concretely real determinateness.
It is irritability.
Because of the abstraction of its negativity,
feeling is impulse;
it determines itself;
the self-determination of the living being is
its judgment or the self-limiting whereby
it refers to the outside as to a presupposed objectivity
with which it is in reciprocal activity.
The living being, as a particular,
now stands on one side as one species
next to other species;
the formal reflection of
this indifferent diversity into itself
is the formal genus and its systematization;
but the individual reflection is this,
that as outwardly directed the particularity,
the negativity of the living being's determinateness,
is the self-referring negativity of the concept.

According to this third determination,
the living being is a singular.
This immanent reflection further
 determines itself in such a way
that in irritability the living being is
the externality of itself as against itself,
as against the objectivity
that it possesses immediately
as its means and instrument
and which is externally determinable.
The immanent reflection sublates this immediacy:
on the one side as theoretical reflection,
that is, in so far as the negativity is
the simple moment of sensibility
as was considered in the latter,
and which constitutes feeling;
on the other side as real reflection,
in that the unity of the concept posits itself
in its externality as negative unity,
and this is reproduction.
The two first moments, sensibility and irritability,
are abstract determinations;
in reproduction life is something concrete and vital;
in it alone does it also have feeling and power of resistance.
Reproduction is the negativity as simple moment of sensibility,
and irritability is only a vital power of resistance,
so that the relation to the external is
reproduction and identity of the individual with itself.
Each singular moment is essentially the totality of all;
their difference constitutes the ideal determination of form
which is posited in reproduction as
the concrete totality of the whole.
On the one hand, therefore, this whole is opposed
to the previous determinate totalities as a third,
namely as a concretely real totality;
on the other hand, however, it is their implicit essentiality
and also that in which they are comprehended as moments
and where they have their subject and subsistence.

With reproduction as a moment of singularity,
the living being posits itself as actual individuality,
a self-referring being-for-itself;
but it is at the same time a real outward reference,
the reflection of particularity or irritability
as against an other, as against the objective world.
The life-process enclosed within the individual
passes over into a reference to
the presupposed objectivity as such,
by virtue of the fact that,
as the individual posits itself as subjective totality,
the moment of its determinateness,
its reference to externality,
also becomes a totality.

B. THE LIFE-PROCESS

In shaping itself inwardly, the living individual
comes into tension with its original presupposing
and, as a subject existing in and for itself,
sets itself in opposition to the presupposed objective world.
The subject is a purpose unto itself,
the concept that has its means and subjective reality
in the objectivity subjugated to it.
As such, it is constituted as the idea existing in and for itself
and as an essentially self-subsistent being,
as against which the presupposed external world
has the value only of something negative
and without self-subsistence.
In its self-feeling the living being
has the certainty of the intrinsic nullity
of the otherness confronting it.
Its impulse is the need to sublate this otherness
and to give itself the truth of this certainty.
At first the individual is, as subject,
only the concept of the idea of life;
its inner subjective process in which it feeds upon itself,
and the immediate objectivity which it posits as
a natural means in conformity with its concept,
are mediated by the process that refers to
the fully posited externality,
to the objective totality standing indifferently alongside it.

This process begins with need, that is,
the twofold moment of self-determination of the living being
by which the latter posits itself as negated
and thereby refers itself to an other than it,
to the indifferent objectivity,
but in this self-loss it is equally not lost,
preserves itself in it
and remains the identity of the self-equal concept.
The living being is thereby the impulse to posit
as its own this world which is other than it,
to posit itself as equal to it,
to sublate the world and objectify itself.
Its self-determination has therefore
the form of objective externality,
and since it is at the same time self-identical,
it is the absolute contradiction.
The immediate shape of the living being is
the idea in its simple concept,
the objectivity conforming to the concept;
as such the shape is good by nature.
But since its negative moment realizes itself
as an objective particularity,
that is, since the essential moments of its unity are
each realized as a totality for itself,
the concept splits into two,
becoming an absolute inequality with itself;
and since even in this rupture the concept remains absolute identity,
the living being is for itself this rupture,
has the feeling of this contradiction which is pain.
Pain is therefore the prerogative of living natures;
since they are the concretely existing concept,
they are an actuality of infinite power,
so that they are in themselves the negativity of themselves,
that this their negativity exists for them,
that in their otherness they preserve themselves.
It is said that contradiction cannot be thought;
but in the pain of the living being
it is even an actual, concrete existence.

This internal rupture of the living being,
when taken up into the simple universality of the concept,
in sensibility, is feeling.
From pain begin the need and the impulse
that constitute the transition by which the individual,
in being for itself the negation of itself,
also becomes for itself identity,
an identity which only is as the negation of that negation.
The identity which is in the impulse as such is
the individual's subjective certainty of itself,
in accordance with which it relates to the indifferent,
concrete existence of its external world as to an appearance,
to an actuality intrinsically void of concept and unessential.
This actuality is to obtain its concept
only through the subject,
which is the immanent purpose.
The indifference of the objective world to determinateness
and hence to purpose is what constitutes
its external aptitude to conform to the subject;
whatever other specifications there might be in it,
its mechanical determinability,
the lack of the freedom of the immanent concept,
constitute its impotence in preserving itself against the living being.
In so far as the object confronts the living being
at first as something external and indifferent,
it can affect it mechanically,
but without in this way affecting it as a living thing;
and in so far as it does relate to it as a living thing,
it does not affect it as a cause but it rather excites it.
Because the living being is an impulse,
externality impinges upon it and penetrates it
only to the extent that in principle it is already in it;
hence the effect on the subject consists only in that
the latter finds that the externality at its disposal accords with it.
And should this externality not accord with it as a totality,
then it must at least accord with a particular side of it
a possibility lodged in the very fact that,
in its relation to the outside,
the subject is a particular.

Now the subject, in so far as
in being determined in its need
it connects with the outside
and consequently is itself something external
or an instrument,
exercises violence over the object.
Its particular character, its finitude in general,
falls into the more determinate appearance of this relation.
The external factor in this is the process of
objectivity in general, mechanism and chemism.
But this same process is immediately interrupted
and the externality transformed into interiority.
The external purposiveness which is at first
elicited in the indifferent object
by the activity of the subject
is thereby sublated,
for as against the concept the object is not a substance:
the concept, therefore, cannot become for it just an external form
but must rather posit itself as its essence
and as the determinateness immanently pervading it
through and through in conformity with
the concept's original identity.

By seizing hold of the object,
the mechanical process passes over into an internal process
by which the individual appropriates the object
in such a manner that it takes away from it
its distinctive make-up, makes it into a means,
and confers upon it its own subjectivity as its substance.
This assimilation thus coincides with
the individual's process of reproduction considered above;
in this process the individual feeds on itself,
in the sense that it makes its own objectivity its object;
the mechanical and chemical conflict of its members
with external things is an objective moment of itself.
The mechanical and chemical factor in the process is
a beginning of the dissolution of the living thing.
Since life is the truth of these processes,
and as a living being it is therefore
the concrete existence of this truth
and the power over the processes,
it infringes upon the latter,
permeates them as their universality,
and their product is entirely determined by it.
This transformation of them into the living individual
constitutes the turning back of this individual into itself,
with the result that the production that
as such would be the transition into an other becomes reproduction,
a reproduction in which the living being posits itself
as self-identical for itself.

The immediate idea is also
the immediate identity of concept and reality
but one that does not exist for itself;
through the objective process,
the living being gives itself its feeling of self;
for in that process it posits itself
as it is in and for itself, namely,
as self-identical in an otherness
posited as indifferent to it,
as the negative unity of the negative.
The individual, in thus rejoining the objectivity
at first presupposed as indifferent to it,
has equally constituted itself as actual singularity
and has sublated its particularity,
raising it to universality.
Its particularity consisted in the disruption
whereby life posited the individual life
and the objectivity external to it as its species.
Through the external life-process,
it has consequently posited itself
as real universal life, as genus.

C. THE GENUS

The living individual, at first cut off
from the universal concept of life,
is a presupposition yet unproven through itself.
Through its process with
the simultaneously presupposed world,
it has posited itself for itself as
the negative unity of its otherness,
as the foundation of itself;
thus it is the actuality of the idea,
so that the individual now brings
itself forth out of actuality,
whereas before it proceeded
only from the concept,
and its coming to be,
which was a presupposing,
now becomes its production.

But the further determination
that it has attained by
the sublation of the opposition is
that it is genus,
identity of itself with its
hitherto indifferent otherness.
This idea of the individual,
since it is this essential identity,
is essentially the particularization of itself.
This particularization, its disruption,
in keeping with the totality from which it proceeds,
is the duplication of the individual
the presupposing of an objectivity
which is identical with it,
and a relating of the living being to itself
as to another living being.

This universal is the third stage,
the truth of life in so far as
life is still shut up within itself.
This stage is the process of
the individual as it refers to itself,
where the externality is
the individual's immanent moment
and is, besides, itself a living totality,
an objectivity which for the individual
is the individual itself
an externality in which the individual has
certainty of itself not as being sublated,
but as subsisting.

Now because the relation of genus is
the identity of individual self-feeling
in such a one who is at the same time
another self-subsistent individual,
it is a contradiction;
accordingly, the living being is once more impulse.
The genus is indeed now the completion of the idea of life,
but at first it is still within the sphere of immediacy;
this universality is therefore actual in a singular shape;
it is the concept whose reality has
the form of immediate objectivity.
And consequently the individual, although it is the genus,
it is the genus in itself rather than for itself;
what is for it is as yet only another living individual;
the concept distinguished from itself has for object,
with which it is identical, not itself as concept,
but a concept rather that as a living being has
at the same time external objectivity for it,
a form which is therefore immediately reciprocal.

The identity with the other individual,
the universality of the individual,
is thus still only inner or subjective;
it therefore has the longing to posit
this identity and to realize itself as universal.
But this impulse of the genus can realize itself
only through the sublation of the singular individualities
which are still particular to each other.
At first, in so far as it is these individualities
which, in themselves universal, themselves satisfy
the tension of their longing and dissolve themselves
into the universality of their genus,
their realized identity is
the negative unity of the genus
reflecting itself into itself out of its rupture.
To this extent, it is the individuality of life itself,
no longer generated out of its concept
but out of the actual idea.
At first, it is itself only the concept
that still has to objectify itself,
but a concept which is actual
the germ of a living individual.
To ordinary perception what the concept is,
and that the subjective concept has external actuality,
are visibly present in it.
For the germ of the living being is
the complete concretion of individuality:
it is where all the living being's diverse sides,
its properties and articulated differences,
are contained in their entire determinateness;
where the at first immaterial, subjective totality is
present undeveloped, simple and non-sensuous.
Thus the germ is the whole living being
in the inner form of the concept.

From this side the genus obtains actuality
through its reflection into itself,
for the moment of negative unity
and individuality is thereby posited in it
the propagation of the living species.
The idea, which as life is still
in the form of immediacy,
thus falls back into actuality,
and its reflection is now only
the repetition and the infinite process
in which it does not step outside
the finitude of its immediacy.
But this going back to its first concept also has
the higher side that the idea has not only run through
the mediation of its processes inside immediacy,
but, just because it has run through them,
has sublated this immediacy
and has thereby elevated itself
to a higher form of its existence.

That is to say, the process of the genus
in which the single individuals sublate in one another
their indifferent, immediate, concrete existence,
and in this negative unity die away,
has further the realized genus
that has posited itself as identical
with the concept for the other side of its product.
In the process of the genus,
the isolated singularities of individual life perish;
the negative identity in which
the genus turns back into itself is
on the one side the generation of
singularity just as it is also,
on the other side, the sublation of it
is thus the genus rejoining itself,
the universality of the idea as
it comes to be explicitly for itself.
In copulation, the immediacy of
living individuality perishes;
the death of this life is
the coming to be of spirit.
The idea, implicit as genus, becomes explicit
in that it has sublated its particularity
that constituted the living species,
and has thereby given itself a reality
which is itself simple universality;
thus it is the idea that relates
itself to itself as idea,
the universal that has universality
for its determinateness and existence.
This is the idea of cognition.
