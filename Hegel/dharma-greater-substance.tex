A. THE RELATION OF SUBSTANTIALITY

Absolute necessity is absolute relation
because it is not being as such
but being that is because it is,
being as the absolute mediation
of itself with itself.
This being is substance;
as the final unity of essence and being,
it is the being in all being.
It is neither the unreflected immediate,
nor something abstract standing behind
concrete existence and appearance,
but the immediate actuality itself,
and it is this actuality
as being absolutely reflected into itself,
as a subsisting that exists in and for itself.
Substance, as this unity of being and reflection,
is essentially the shining and the positedness of itself.
The shining is a self-referring shining, thus it is;
this being is substance as such.
Conversely, this being is only the self-identical positedness,
and as such it is shining totality, accidentality.

This shining is identity as identity of form,
the unity of possibility and actuality.
It is becoming at first,
contingency as the sphere of
coming-to-be and passing-away;
for in the determination of immediacy
the connection of possibility and actuality is
the immediate conversion of the two
into each other as existents,
of each into its other as only an other to it.
But because being is shine,
their relation is also one of identical terms
or of terms shining in one another, that is, reflection.
The movement of accidentality, therefore,
exhibits in each of its moments the mutual
reflective shine of the categories of being
and of the reflective determinations of essence.
The immediate something has a content;
its immediacy is at the same time
reflected indifference towards the form.
This content is determinate,
and because this determinateness is one of being,
the something passes over into an other.
But quality is also a determinateness of reflection;
as such, it is indifferent diversity.
But this diversity is animated into opposition,
and returns to the ground which is the nothing,
but also immanent reflection.
This reflection sublates itself;
but it is itself also reflected in-itselfness:
so it is possibility, and this in-itselfness,
in its transition which is equally immanent reflection,
is necessary actuality.

This movement of accidentality is the actuosity of substance
as the tranquil coming forth of itself.
It is not active against something,
but only against itself as a simple unresisting element.
The sublating of a presupposition is the disappearing shine;
only in the act of sublating the immediate does this
immediate itself come to be, or is that shining;
the beginning that begins from itself is first of all
the positing of this itself from which the beginning is made.

Substance, as this identity of the reflective shining,
is the totality of the whole and embraces accidentality in itself,
and accidentality is the whole substance itself.
Its differentiation into the simple identity
of being and the flux of accidents
within it is one form of its shining.
That simple being is the formless substance of the imagination
for which the shine has not determined itself as shine,
but which holds on, as on an absolute,
to this indeterminate identity that has no truth
but only is the determinateness of immediate actuality,
or equally so of in-itselfness or possibility,
form determinations that fall into accidentality.
The other determination, the flux of accidents,
is the absolute form-unity of accidentality,
substance as absolute power.
The ceasing-to-be of the accident is
its return as actuality into itself,
as into its in-itself or into its possibility;
but this, its in-itself, is itself only a positedness
and therefore also actuality,
and because these form determinations are
equally determinations of content,
this possible is an actual differently determined
also according to content.
Substance manifests itself through the actuality,
with the content of the latter into which
it translates the possible, as creative power,
and, through the possibility to which
it reduces the actual, as destructive power;
the creating is destructive and the destructing creative,
for the negative and the positive, possibility and negativity
are in substantial necessity absolutely united.

The accidents as such, and there are several of them,
because plurality is one of the determinations of being,
have no power over each other.
They are the immediately existent something,
or the something that immediately exists for itself;
concretely existing things of manifold properties;
or wholes consisting of parts, self-subsisting parts;
forces in need of reciprocal solicitation
and conditioning each other.
In so far as such an accidental
being seems to exercise a power over an other,
that power is that of substance that
encompasses them both within itself
and, as negativity, posits an inequality of value:
one it determines as ceasing-to-be
and another as having a different content
and as coming-to-be,
the one as passing over into its possibility
and the other into actuality
accordingly ever dividing itself
into this difference of form and content
and ever purifying itself of this one-sidedness,
but in this purification ever falling back
into determination and division.
One accident thus drives out another
only because its own subsisting is
this very totality of form and content
into which it, as well as its other, equally perishes.

Because of this immediate identity and presence
of substance in the accidents,
there is still no real difference present.
In this first determination, substance is not yet manifested
according to its whole concept.
When substance, as self-identical being-in-and-for-itself,
is differentiated from itself as a totality of accidents,
it is substance itself, as power, that mediates the difference.
This power is necessity, the positive persistence
of the accidents in their negativity
and their mere positedness in their subsistence;
this middle is thus the unity of
substantiality and accidentality themselves,
a middle whose extremes have no subsistence of their own.
Substantiality is, therefore, only the relation as immediately vanishing;
it refers to itself not as a negative
and, as the immediate unity of power with itself,
is in the form only of its identity,
not of its negative essence;
only one of its moments, that of negativity or of difference,
vanishes altogether;
the other moment of identity does not.

Another way of considering the matter is this.

The shine or the accidentality is indeed
in itself substance by virtue of the power,
but is not thus posited as this self-identical shine;
and therefore substance has only the accidentality,
not itself, for its shape or positedness;
it is not substance as substance.
The relation of substantiality is at first, therefore, only this,
that substance manifests itself as a formal power
whose differences are not substantial;
in fact, substance only is as the inner of the accidents,
and these only are in the substance.
Or this relation is only the shining of totality as becoming;
but it is equally reflection and, for this reason,
the accidentality which substance is in itself is also posited as such;
it is thus determined as self-referring negativity over against itself,
determined as self-referring simple identity with itself;
and it is substance that exists in and for itself,
substance endowed with power.
Thus the relation of substantiality passes over
into the relation of causality.

B. THE RELATION OF CAUSALITY

Substance is power, power reflected into itself,
not transitive power but power that posits determinations
and distinguishes them from itself.
As self-referring in its determining,
it is itself that which it posits as a negative
or makes into a positedness.
This positedness is, as such, sublated substantiality,
the merely posited, the effect;
the substance that exists for itself is, however, cause.

This relation of causality is in the first place
only this relation of cause and effect;
as such, it is the formal relation of causality.

a. Formal causality

1. Cause is originative as against the effect.
As power, substance is the reflective shining,
or it has accidentality.
But in this shining, as power,
it equally is an immanent reflection;
it thus expounds its transition,
and this reflective shine is determined as reflective shine,
or the accident is posited
as being just this, something posited.
But in its determining substance
does not proceed from accidentality,
as if the latter were an other before-
hand and were determined as determinateness only then,
but the two are one actuosity.
Substance as power determines itself;
but this determining is immediately itself
the sublation of the determining and a turning back.
It determines itself: substance, that which determines, is
thus the immediate and that which is itself already determined;
in determining itself it therefore posits
the already determined as determined;
and thus it has sublated the
positedness and has returned into itself.
Conversely, because this turning back is
the negative reference of substance to itself,
it is itself a determining
or the repelling of itself from itself;
it is through this turning back that
the determinate comes to be from which
substance seems to begin
and now to posit as something
which it has found already determined.
Absolute actuosity is thus cause,
the power of substance in its truth
as the manifestation by which
that which is in itself,
the accident or the positedness,
is immediately expounded in its becoming,
is posited as positedness, as effect.
This effect is, therefore, first the same as
what the accidentality of the relation
of substance is, namely substance as positedness;
but, second, an accident
is substantially such only by vanishing, only as transient;
but as effect it is positedness as self-identical;
in the effect the cause is manifested as the whole substance,
that is to say, as reflected into itself
in the positedness itself as such.

2. Over against this positedness reflected into itself, this determined as
determined, there stands substance as the non-posited original. Because
substance is as absolute power a turning back into itself, yet this turning
back is itself a determining, it is no longer the mere in-itself of its accident
but is also posited as this in-itself. Substance has actuality, therefore, only
as cause. But this actuality in which its in-itself, its determinateness in the
relation of substantiality, is now posited as determinateness, is effect; there-
fore substance has the actuality which it has as cause only in its effect. –
This is the necessity which is cause. – It is actual substance, because as
power substance determines itself; but it is at the same time cause, because
it expounds this determinateness or posits it as positedness and thus
posits its actuality as positedness or effect. This is the other of cause, the
positedness as against the original and as mediated through it. But cause,
as necessity, equally sublates this mediating and, in determining itself as
the originally self-referring term, as against the mediated, turns back to
itself; for positedness is determined as positedness, and consequently as
self-identical; therefore, cause is truly actual and self-identical only in its
effect. – The effect is therefore necessary, because it is the manifestation of
the cause or is this necessity which the cause is. – Only as this necessity is
cause self-moving, self-initiating without being solicited by another, self-
subsisting source of production out of itself; it must effect; its originariness is
this, that it is because its immanent reflection is a positing that determines
and conversely; the two are one unity.

Consequently, an effect contains nothing whatever that the cause does not
contain. Conversely, a cause contains nothing that is not in its effect. A cause
is cause only to the extent that it produces an effect; to be cause is nothing
but this determination of having an effect, and to be effect is nothing but this
determination of having a cause. Cause as such entails its effect, and the
effect entails the cause; in so far as a cause has not acted yet or has ceased
to act, it is not a cause; and the effect, in so far as its cause is no longer
present, is no longer an effect but an indifferent actuality.

3. Now in this identity of cause and effect the form distinguishing them
respectively, as that which exists in itself and that which is posited, is
sublated. The cause is extinguished in its effect and the effect too is thereby
extinguished, for it only is the determinateness of the cause. Hence this
causality which has been extinguished in the effect is an immediacy which
is indifferent to the relation of cause and effect and comes to it externally.

b. The determinate relation of causality

1. The self-identity of cause in its effect is
the sublation of its power and negativity,
hence a unity which is indifferent to differences of form,
that is to say, content.
This content, therefore, refers to form (here causality)
only implicitly.
The two are thus posited as diverse,
and with respect to content the form is itself
a causality which is only immediately efficient,
a contingent causality.

Further, the content is as thus determined
an internally diversified content;
and the cause is determined in accordance with its content,
and so is therefore also the effect.
The content, since reflectedness here is also immediate actuality,
is to this extent actual, but finite, substance.

This is now the relation of causality in its reality and finitude.
As formal, it is the infinite relation of absolute power,
the content of which is pure manifestation or necessity.
As finite causality, on the contrary, it has a given content
and, as an external difference, it runs its course here
and there over it, this identical content
which in its determination is one and the same substance.

Because of this identity of content, this causality is an analytic propo-
sition. It is the same fact that comes up once as cause and then again as
effect, in one case as something subsisting on its own and in the other
as positedness or determination. Since these determinations of form are
an external reflection, it is up to the essentially tautological consideration
of a subjective understanding to determine an appearance as effect and to
rise from it to its cause in order to comprehend and explain it. The same
content is being repeated; there is nothing else in the cause which is not in
the effect. – For instance, rain is the cause of wetness which is its effect; “the
rain makes wet,” this is an analytical proposition; the same water which is
rain is wetness; as rain, this water is only in the form of a subject by itself;
as wetness or moisture, it is on the contrary in adjectival form, something
posited no longer meant to have a subsistence on its own; and the one
determination, just like the other, is external to water. – Again, the cause of
this color is a coloring agent, a pigment which is one and the same actuality,
once in the form of an agent external to it, that is, is externally linked to
an agent different from it; but again in the determination, equally external
to it, of an effect. – The cause of an act is the inner intention of the subject
who is the agent, and this intention is the same in content and value as the
existence which it attains through the action. If the movement of a body
is considered as effect, the cause of this effect is then a propulsive force;
but it is the same quantum of movement which is present before and after
the propulsion, the same concrete existence which the propulsive body
contained and which it communicated to the one propelled; and what it
communicated, it lost in equal measure.

The cause, say the painter or the propulsive body, does have yet another
content than, in the case of the painter, the colors and the form combining
these into a painting; and, in the other case, the movement of specific
strength and direction. But this further content is a contingent side-being
which has nothing to do with the cause; whatever other qualities the painter
might possess besides being the painter of this painting, this does not enter
into the painting; only those of his properties which are displayed in the
effect are present in him as cause; as for the rest, he is not a cause. Likewise,
whether the propulsive body is of stone or wood, green, yellow, and so
on, all this does not enter into its propulsion and, to this extent, is not a
cause.

It is worth noting in regard to this tautology of the relation of causality
that the tautology does not seem to occur whenever it is not the proximate,
but the remote cause which is at issue. The alteration of form which the
basic fact undergoes as it passes through several middle terms hides the
identity which it preserves across them. In this proliferation of causes
introduced between it and the last effect, that fact is linked to other things
and circumstances, so that it is not that first term, which is declared the
cause, but all these several causes together, that contain the complete effect. –
For instance, if a man developed his talents in circumstances due to the
loss of his father who was hit by a bullet in battle, then this shot (or still
further back, the war or some cause of the war, and on to infinity) could
be adduced as the cause of the man’s skillfulness. But it is clear that the
shot, for one, is not the cause by itself but only in conjunction with the
other efficient determinations. Or more precisely, the shot is not the cause
at all, but only a single moment that pertained to the circumstances of the
possibility.

But it is the inadmissible application of the relation of causality to the
relations of physico-organic and spiritual life that must be noted above all.
Here that which is called the cause does indeed show itself to be of a different
content than the effect, but this is because anything that has an effect on a
living thing is independently determined, altered, and transmuted by the
latter, for the living thing will not let the cause come to its effect, that is, it
sublates it as cause. Thus it is inadmissible to say that nourishment is the
cause of blood, or that such and such a dish, or chill and humidity, are the
causes of fever or of what have you; it is equally inadmissible to give the Ionic
climate as the cause of Homer’s works, or Caesar’s ambition as the cause of
the fall of Rome’s republican constitution. In history in general there are
indeed spiritual masses and individuals at play and influencing each other;
but it is of the nature of spirit, in a much higher sense than it is of the
character of living things, that it will not admit another originative principle
within itself, or that it will not let a cause continue to work its causality
in it undisturbed but will rather interrupt and transmute it. – But these
relations belong to the idea, and will come up for discussion then. – This
much can still be noted here, namely that in so far as the relation of cause
and effect is admitted, albeit in an inappropriate sense, the effect cannot
be greater than the cause. It has become a common witticism in history
to let great effects arise from small causes and to cite as the first cause of an
event of far-reaching and profound consequence an anecdote. Any such so-
called first cause is to be regarded as no more than an occasion, an external
stimulus, of which the inner spirit of the event had no need, or could have
used a countless number of others, in order to make its first appearance,
to give itself a first breath and announce itself. The converse is rather the
case. It is by the spirit that any such triviality and contingency is determined
in the first place to be the occasion of spirit. Historical arabesques that draw
a full-blown figure out of a slender stalk are no doubt an ingenious, but
highly superficial, practice. It is true that in the rise of the great out of
the small we witness everywhere the conversion that spirit works on the
external; but precisely for this reason the external is not the cause within
spirit; rather, that conversion itself sublates the relation of causality.
2. But this determinateness of the relation of causality, that content
and form are different and indifferent to each other, extends further. The
determination of form is also content determination; cause and effect, the
two sides of the relation, are therefore also another content. Or the content,
because it is only as the content of a form, has the difference of this form
within it and is essentially different. But this form of the content is the
relation of causality, which is a content identical in cause and the effect,
and consequently the different content is externally connected, on the one
hand with the cause and on the other with the effect; hence the content itself
does not enter into the effective action and into the relation.
This external content is therefore relationless – an immediate concrete
existence, or because it is as content the implicit identity of cause and effect,
it is also immediate, existent identity. This content is, therefore, anything
at all which has manifold determinations of its existence, among them also
this, that it is in some respect or other cause or also effect. In it, the form
determinations of cause and effect have their substrate, that is to say, their
essential subsistence – and each has a particular subsistence (since their
identity is their subsistence); but it is a subsistence which is at the same
time immediate, not their subsistence as unity of form or as relation.
But this thing is not only substrate but also substance, for it is identical
subsistence only as subsistence of the relation. Moreover, the substance is
finite substance, for it is determined as immediate over against its causality.
But it has causality at the same time, for it is just as much an identity as this
relation. 21 – Now this substrate is, as cause, negative reference to itself. But
this “itself” to which it refers is, first, a positedness because it is determined
as immediately actual; this positedness, as content, is any determination
whatever. – Second, causality is external to the substrate and itself constitutes,
therefore, its positedness. Now since it is causal substance, its causality consists
in negatively referring itself to itself, hence to its positedness and external
causality. The effective action of this substance thus begins from something
external, frees itself from this external determination, and its turning back
into itself is the preservation of its immediate concrete existence and the
sublation of the one which is posited, and consequently of its causality as
such.

Take a stone that moves. It is a cause.
Its movement is a determination which it has.
But, besides it, it contains yet many other determinations
(color, shape, and so on) that do not enter into its causality.
Because its immediate concrete existence is separated from its form-connection,
namely the form of causality, the latter is something external;
the stone’s movement and the causality attaching to it is in it only positedness.
But the causality of the stone is also the stone’s own causality,
as follows from the fact that its substantial subsistence is the stone’s identical self-reference,
but that this is now determined as positedness and is therefore at the same
time negative self-reference. – Its causality, which is directed against itself as
a positedness or as an externality, consists therefore in sublating this and
through its removal in returning to itself – to this extent, therefore, in not
being self-identical in its positedness but only in restoring its originariness. –
Or again, rain is the cause of wetness, which is the same water as the rain.
This water has the determination of being rain and cause because this
determination has been posited in it by another; another force, or what
have you, has lifted it into the air and compressed it into a mass, the weight
of which makes it fall. Its being removed from the earth is a determination
alien to its original self-identity, to its gravity; its causality consists in
removing such determination and in restoring its original identity; but this
means also sublating its causality.

We now consider the second determinateness of causality which concerns
form; this relation is causality external to itself, as the originariness which is
within just as much positedness or effect. This union of opposite determi-
nation in an existent substrate constitutes the infinite regress from cause to
cause. – We start from an effect; the latter has as effect a cause; but this
cause has a cause in turn, and so on. Why does the cause have a cause in
turn? That is to say, why is the same side which was previously determined
as cause now determined as effect and therefore demands a new cause? –
Because the cause is something finite, a determinate in general; determined
as one moment of the form as against the effect; so it has its determinate-
ness or negation outside it; but for this very reason it is itself finite, has its
determinateness within it and is thereby positedness or effect. Its identity as
this positedness is also posited, but it is a third term, the immediate sub-
strate; causality is therefore external to itself, because its originariness is here
an immediacy. The difference of form is therefore a first determinateness,
not yet determinateness posited as determinateness; it is existent otherness.
Finite reflection, on the one hand, stops short at this immediate, removes
the unity of form from it and makes it be cause in one respect and effect in
another; on the other hand, it transfers the unity of form into the infinite,
and through the endless progression expresses its impotence in attaining
and holding fast to this unity.

Exactly the same is the case of the effect, or rather the endless progression
from effect to effect is one and the same as the regression from cause to cause.
Just as in the latter a cause becomes an effect which has another cause
in turn, so too, conversely, the effect becomes a cause which has another
effect in turn.
 – The determinate cause under consideration begins from
an externality and returns in its effect back to itself, but not as cause;
on the contrary, it loses its causality in that process. But, conversely, the
effect arrives at a substrate which is substance, an original self-referring
subsistence; in it, therefore, that positedness becomes a positedness, that is
to say, this substance, as the effect is posited in it, behaves as cause. But
that first effect, the positedness that accrues to the substance externally, is
other than the second which the substance produces; for this second effect is
determined as the immanent reflection of substance whereas the first is in it
as an externality. – But because causality is here causality external to itself,
it also equally fails to return in its effect back to itself but becomes therein
external to itself; its effect becomes again a positedness in a substrate – as
in another substance which however equally makes this positedness into a
positedness, in other words, manifests itself as cause, again repels its effect
from itself, and so on, into bad infinity.

3. We now have to see what has resulted from the movement of deter-
minate causality. – Formal causality expires in the effect and the element
of identity of these two moments emerges as a result, but it does so only
as an implicit unity of cause and effect to which the form connection is
external. – For this reason, the element of identity is immediate also with
respect to both of the two determinations of immediacy, first as in-itself,
as a content on which causality is deployed externally; second, as a concrete
existent substrate in which cause and effect inhere as different determina-
tions of form. In this substrate, the two determinations are implicitly one,
but, on account of this implicitness or of the externality of form, each is
external to itself and hence, in its unity with the other, is also determined
as other with respect to it. Consequently, the cause has indeed an effect and
is at the same time itself effect; and the effect not only has a cause but is itself
also cause. But the effect which the cause has, and the effect which it is, are
different – as are also the cause which the effect has and the cause which it
is.

The outcome of the movement of the determinate relation of causality
is then this, that the cause does not just expire in the effect, and thereby the
effect as well, as in formal causality, but that by expiring in the effect the
cause comes to be again; that the effect vanishes in the cause, but equally
comes to be again in it. Each of these determinations sublates itself in its
positing, and posits itself in its sublating; what we have is not an external
transition of causality from one substrate to another, but its becoming-
other is at the same time its own positing. Causality thus pre-supposes itself
or conditions itself. The previously only implicit identity, the substrate, is
therefore now determined as presupposition or posited as against the efficient
causality, and the reflection hitherto only external to the identity is now in
relation to it.

c. Action and reaction

Causality is a presupposing activity. The cause is conditioned; it is a negative
reference to itself as a presupposed, as an external other which in itself, but
only in itself, is causality itself. This other is, as we have seen, 22 the substantial
identity into which formal causality passes over, which now has determined
itself as against this causality as its negative. Or it is the same as the substance
of the causal relation, but a substance which is confronted by the power
of accidentality as itself substantial activity. – It is the passive substance. –
Passive is that which is immediate, or which exists-in-itself but is not also
for itself – pure being or essence in just this determinateness of abstract self-
identity. – Confronting the passive substance is the negatively self-referring
substance, the efficient substance. It is cause inasmuch as in determinate
causality it has restored itself out of the effect through the negation of itself –
a reflected being 23 which in its otherness or as an immediate behaves
essentially as a positing activity and through its negation mediates itself.
Here, therefore, causality no longer has a substrate in which it inheres; it is
not a determination of form as against this identity but is itself substance,
or in other words, causality alone is at the origin. – The substrate is the
passive substance which causality has presupposed for itself.

This cause now acts, for it is the negative power over itself; at the same
time it is its own presupposition; thus it acts upon itself as upon an other,
upon the passive substance. – Hence, it first sublates the otherness of this
substance and returns in it back to itself; second, it determines this same
substance, posits this sublation of its otherness or the substance’s turning
back into itself as a determinateness. This positedness, because it is at the
same time the substance’s turning back into itself, is at first its effect.
But conversely, because as presupposing it determines itself as its other, it
then posits the effect in this other, in the passive substance. – Or again,
because the passive substance is itself this double – namely a self-subsistent
other, and at the same time something presupposed and already implicitly
identical with the efficient cause – because of this, the action of the passive
substance is therefore itself double. It is at once both the sublation of
its determinateness, namely of its condition, or the sublation of the self-
subsistence of the passive substance; and also, in sublating its identity as
it sublates this substance, the pre-supposing of itself, that is, the positing
or supposing of itself as other. – Through this last moment, the passive
substance is preserved; that first sublation of it appears in this respect at
the same time also in this way, namely that only some determinations are
sublated in it, and its identity in the effect with the efficient cause occurs
in it externally.

To this extent it suffers violence. – Violence is the appearance of power,
or power as external. But power is something external only in so far as in its
action, that is, in the positing of itself, the causal substance is at the same
time a presupposing, that is, posits itself as sublated. Conversely, the act of
violence is therefore equally an act of power. The violent cause acts only on
an other which it presupposes; its effect on it is its negative self-reference,
or the manifestation of itself. The passive is the self-subsistent which is only
a posited, something internally fractured – an actuality which is condition,
though a condition that now is in its truth as an actuality that is only a
possible, or, conversely, an in-itself that is only the determinateness of the
in-itself, is only passive. To that which suffers violence, therefore, not only
is it possible to do violence, but violence must be done to it; that which
has dominion over an other, only has it because its power is that of the
other, a power which in that dominion manifests both itself and the other.
Through violence the passive substance is only posited as what it is in truth,
namely, that because it is the simple positive or the immediate substance,
for that very reason it is only something posited; the “pre-” 24 that it has as
condition is the reflective shine of immediacy that the efficient causality
strips off from it.

Passive substance, therefore, is only given its due by the action on it of
another power. What it loses is the immediacy it had, the substantiality alien
to it. What comes to it as an alien something, namely that it is determined
as a positedness, is its own determination. – But now in being determined
in its positedness, or in its own determination, the result is that it is not
sublated but rather that it only rejoins itself and in its being determined
is, therefore, an originariness. – On the one hand, therefore, the passive
substance is preserved or posited by the active, namely in so far as the latter
sublates itself; but, on the other hand, it is the act of the passive substance
itself to rejoin itself and thus to make itself into what is originary and a
cause. The being posited by an other and its own becoming are one and the
same.

Now, because the passive substance has been converted into a cause, it
follows, first, that the effect is sublated in it; therein consists its reaction in
general. As passive substance, it is in itself as positedness; also, positedness
has been posited in it by the other substance, namely in so far as it received
its effect within it. Its reaction contains, therefore, a twofold aspect. For
one, what it is in itself is posited. And two, what it is as posited displays itself
as its in-itself; it is positedness in itself, hence through the other substance it
receives an effect within; but, conversely, this positedness is its own in-itself,
it is thus its own effect, it itself displays itself as a cause.

Second, the reaction is directed at the first efficient cause. For the effect
which the hitherto passive substance sublates within itself is precisely the
effect of that other cause. But a cause has its substantial actuality only in
its effect; inasmuch as this effect is sublated, so is also the causal substan-
tiality of the other cause. This happens first in itself through itself, in that
the cause makes itself into an effect; its negative determination disappears
in this identity and the cause becomes passive; and, second, it happens
through the hitherto passive, but now reacting substance, which sublates its
effect. – Now in determinate causality the substance acted upon becomes
a cause, for it acts against the positing of an effect in it. But it did not
react against the cause of that effect but posited its effect rather in another
substance, and thus there arose the progression to infinity of effects –
for here the cause is only implicitly identical with itself in the effect, and
hence, on the one hand, it expires into an immediate identity as it comes
to rest, but, on the other hand, it revives in another substance. – In con-
ditioned causality, on the contrary, the cause refers back to itself in the
effect, for the latter is as a condition, as a presupposition, its other, and its
act is therefore just as much a becoming as a positing and sublating of the
other.

Further, causality behaves in all this as passive substance; but, as we
have seen, 25 the latter becomes causal through the effect it incurs. That first
cause, the one which acts first and receives its effect back into itself as a
reaction, thus comes up again as a cause, whereby the activity which in
finite causality runs into the bad infinite progression is bent around and
becomes an action that returns to itself, an infinite reciprocal action.

C. RECIPROCITY OF ACTION

In finite causality it is substances that actively relate to each other. Mech-
anism consists in this externality of causality, where the cause’s reflection
in its effect into itself is at the same time a repelling being, or where, in
the self-identity which the causal substance has in its effect, the substance
is equally immediately external to itself and the effect is transposed into
another substance. In reciprocity of action this mechanism is now sublated,
for it contains first the disappearing of that original persistence of immediate
substantiality; second, the coming to be of the cause, and hence originariness
mediating itself with itself through its negation.

At first, the reciprocity of action takes on the form of a reciprocal causality
of substances that are presupposed and that condition each other; each is with
respect to the other both active and passive substance. Since the two are thus
passive and active at once, their difference is thereby already sublated; it
is a totally transparent reflective shine; they are substances only in being
the identity of the active and the passive. The reciprocity of action is itself,
therefore, only a still empty way and manner, and all that is still needed
is merely the external bringing together of what is already there, both in
itself and as posited. First of all, it is no longer substrates that are referred to
each other but substances; in the movement of conditional causality, the
still left over presupposed immediacy has been sublated, and what conditions
the causing activity is only an influence, or its own passivity. But this
influence, moreover, does not come from another substance originating it
but from precisely a causality which is conditioned by influence, or one
which is mediated. This at first external factor that accrues to the cause
and constitutes the side of its passivity is therefore mediated through the
causality itself, is produced through its own activity and is, consequently,
a passivity posited by its own very activity. – Causality is conditioned and
conditioning. As conditioning, it is passive; but it is equally so as conditioned.
This conditioning or passivity is the negation of the cause through itself in
that it makes itself essentially into an effect and is cause precisely for that
reason. Reciprocity of action is, therefore, only causality itself; the cause does
not just have an effect but, in the effect, refers as cause back to itself.

Causality has thereby returned to its absolute concept and has at the
same time attained the concept itself. At first, it is real necessity, absolute
self-identity in which the difference between it and the determinations
referring to each other within it are substances, free actualities, over against
one another. Necessity is in this way inner identity; causality is the mani-
festation of it in which its reflective shine of substantial otherness has been
sublated, and necessity is elevated to freedom. – In the reciprocity of action,
originative causality displays itself as arising from its negation, from pas-
sivity, and as passing away into it, as a becoming, but in such a way that
this becoming is at the same time equally only shining; the transition into
otherness is reflection-into-itself; negation, which is the ground of the cause,
is its positive rejoining with itself.

In the reciprocity of action, therefore,
necessity and causality have disappeared;
they contain both the immediate identity
as combination and reference
and the absolute substantiality of the differences,
consequently their contingency,
the original unity of substantial difference
and therefore the absolute contradiction.
Necessity is being, because being is;
it is the unity of being with itself that has itself as ground,
but, conversely, because this being has a ground, it is not being;
it is simply and solely reflective shining, reference or mediation.
Causality is this posited transition of original being,
of cause, into reflective shine or mere positedness,
and, conversely, of positedness into originariness;
but the identity itself of being and reflective shine
still is the inner necessity.
This inwardness or this in-itself sublates
the movement of causality;
the result is that the substantiality of the sides
that stand in relation is lost, and necessity unveils itself.
Necessity does not come to be freedom by vanishing
but in that its still only inner identity is manifested,
and this manifestation is the identical movement immanent to
the different sides,
the immanent reflection of shine as shine.
Conversely, contingency thereby comes to be freedom at the same time,
for the sides of necessity, which have the shape of
independent, free actualities that do not reflectively shine into each other,
are now posited as an identity, so that now these totalities
of immanent reflection, in their differences, also shine as identical,
in other words, they are also posited as only one and the same reflection.

No longer, therefore, does absolute substance
as self-differentiating absolute form
repel itself as necessity from itself,
nor does it fall apart as contingency
into indifferent, external substances,
but, on the contrary, it differentiates itself:
on the one hand, into the totality
(the heretofore passive substance)
which is at the origin, as the reflection from internal determinateness,
as simple whole that contains its positedness within itself
and in this positedness is posited as self-identical;
this is the universal;
on the other hand, into the totality
(the hitherto causal substance)
which is the reflection, equally from internal determinateness,
into the negative determinateness
which, just as the self-identical determinateness, equally is the whole,
but posited as the self-identical negativity;
the singular.
But, because the universal is self-identical only in that
the determinateness that it holds within is sublated,
hence it is the negative as negative,
it immediately is the same negativity that singularity is.
And the singularity, because it equally is
the determinedly determined, the negative as negative,
immediately is the same identity that universality is.
This, their simple identity, is the particularity that,
from the singular, holds the moment of determinateness;
from the universal, that of immanent reflection,
the two in immediate unity.
These three totalities are therefore one and the same reflection
that, as negative self-reference, differentiates itself
into the other two totalities;
but as into a perfectly transparent difference,
namely into the determinate simplicity,
or into the simple determinateness,
which is their one same identity.
This is the concept,
the realm of subjectivity or of freedom.
