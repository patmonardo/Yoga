B. QUANTITY

a. Pure quantity

Quantity is pure being in which determinacy is posited as
no longer one with being itself, but as sublated or indifferent.

    1. The expression magnitude is unsuitable for quantity,
    insofar as it signifies first and foremost determinate quantity.

    2. Mathematics usually defines magnitude
    as what can be increased or decreased.
    As faulty as this definition is
    (inasmuch as it repeats what is to be defined),
    it still conveys this much, namely,
    that the determination of magnitude is such that
    it is posited as alterable and indifferent.
    Hence, apart from any alteration of it,
    e.g. an increase in extension or intensity,
    the basic matter, for instance, a house or red,
    does not cease to be a house or red.

    3. The absolute is pure quantity.
    This standpoint generally coincides
    with determining the absolute as matter
    in which the form is indeed on hand,
    but as an indifferent determination.
    Quantity also constitutes the basic
    determination of the absolute,
    when it is grasped in such a way
    that in it (as the absolutely indifferent)
    every distinction is only quantitative.
    Pure space, time, etc. may equally
    be taken as examples of quantity,
    insofar as one is supposed to construe
    the real as an indifferent filler of space or time.

Quantity, posited at first in
its immediate relation to itself
or in the determination of equality with itself
as posited by attraction, is continuous.
According to the other determination contained in it,
namely that of the One, it is a discrete magnitude.
The former quantity, however, is equally discrete,
since it is merely the continuity of the Many.
The latter is equally continuous,
for its continuity is the One
as the same in the many Ones,
the unity [of a mathematical unit].

    1. Continuous and discrete magnitudes
    thus must not be regarded as species,
    as though the determination of the one
    did not belong to the other.
    Rather, they differ only in virtue of
    the fact that the same whole is posited
    now under one and now under the other
    of itself determinations.

    2. The antinomy of space, time, or matter
    (with respect to their divisibility ad infinitum
    or their being composed of indivisibles)
    is nothing but the assertion that quantity is
    now continuous, now discrete.
    When space, time, etc. are posited only with
    the determination of continuous quantity,
    they are divisible ad infinitum;
    but with the determination of a discrete magnitude,
    they are in themselves divided
    and consist of indivisible ones.
    The one is as one-sided as the other.

b. Quantum

Quantity, posited essentially with the exclusive determinacy
that is contained in it, is quantum, limited quantity.

Quantum has its development and complete determinacy in number,
which contains the One as its element within itself
and, as its qualitative moment, the amount,
which is the moment of discreteness,
and the unity [of a mathematical unit]
which is the moment of continuity.

    In arithmetic, the kinds of calculation are
    usually listed as contingent ways of treating numbers.
    If there is to be any necessity
    and thus some rhyme and reason to them,
    it must lie in a principle,
    and that principle can lie only
    in the determinations that are
    contained in number itself.
    This principle shall be briefly expounded here.
    The determinations of the concept of number are
    the amount and the unity [of the mathematical unit],
    and number is the unity of both.
    But the unity [of the mathematical unit],
    when applied to empirical numbers,
    is merely the equality of them.
    Hence the principle of the kinds of calculation must be
    to put numbers into the relationship of
    the amount and the unity [of the mathematical unit],
    and to produce the equality of these determinations.

    Since the Ones or numbers are themselves indifferent towards
    each other, the unity into which they are placed appears generally
    to be an extraneous gathering together.
    For this reason, to calculate generally means to count,
    and the difference between the kinds of calculating
    resides exclusively in the qualitative make-up of
    the numbers that are being added together,
    and the determination of the unity [of the mathematical unit]
    and the amount is the principle of their qualitative make-up.

    To number or to generate number in general comes first,
    a matter of taking arbitrarily many Ones together.
    But calculation of a particular sort is a matter of
    counting together items that are already numbers,
    not the mere One.

    Numbers are immediately and at first
    quite undetermined numbers in general
    and, hence, unequal in general.
    Taking them together or counting them is adding.

    The next determination is that
    numbers are in general equal.
    Thus they constitute a unity [a mathematical unit]
    and there exists a certain amount of them.
    To count numbers such as these is to multiply,
    in which case it does not matter how the determinations
    of the amount and the unity [the mathematical unit] are distributed
    to the two numbers or factors,
    which is taken to be the amount
    and which the unity [the mathematical unit].

    The third determinacy, finally, is
    the equality of amount and unity [the mathematical unit].
    Counting together the numbers determined in this way is
    the raising of the power, and first of all squaring.
    The further raising of the power is the formal
    continuation of the multiplication of number with itself,
    a continuation that leads again to the indeterminate amount.
    Since perfect equality of the only difference on hand,
    that of the amount and their unity, is attained in
    this third determination, there cannot be more than
    these three kinds of calculation.
    To each of these ways of counting together there corresponds
    the dissolution of numbers in accordance with the same determinacies.
    Consequently, besides the three kinds listed,
    which in that regard could be called positive,
    there also exist three negative ones.

c. Degree

The limit is identical to the whole of quantum itself.
Insofar as it is, in itself, manifold,
it is the extensive magnitude,
but insofar as it is, in itself,
a simple determinateness,
it is intensive magnitude,
or degree.

    The difference between the continuous and discrete magnitudes
    and the extensive and intensive ones consists in the fact
    that the former apply to quantity in general,
    while the latter apply to the limit or its determinacy as such.
    Extensive and intensive magnitudes are likewise not two species,
    each of which would contain a determinacy that the other lacked.
    What is extensive magnitude is just as much
    intensive magnitude, and vice versa.

In the [concept of] degree, the concept of quantum is posited.
It is the magnitude as indifferently for itself and simple,
but in such a way that it has the determinacy through which
it is quantum entirely outside itself in other magnitudes.
With this contradiction, namely that the indifferent limit
that is for itself is the absolute externality,
the infinite quantitative progression is posited,
an immediacy that immediately changes over into its opposite,
into being mediated (transcending the quantum just posited)
and vice versa.

    Number is thought, but thought as a being
    that is utterly external to itself.
    It does not belong to intuition because it is thought,
    but it is the thought that has the externality of intuition
    for its determination.
    For this reason, not only can quantum be
    increased or decreased to infinity,
    it is through its concept this
    propelling of itself beyond itself.
    The infinite quantitative progression is
    likewise the thoughtless repetition
    of the same contradiction
    that quantum is in general
    and, when quantum is posited in its determinacy,
    of the same contradiction that degree is.
    Regarding the redundancy of expressing
    this contradiction in the form
    of an infinite progression,
    Zeno rightly says in Aristotle:
    'it is the same thing to say something once
    and to be saying it always.'

Quantum's being external to itself in the determinacy of
its being-for-itself constitutes its quality.
In being external, it is precisely itself and related to itself.
The externality, the quantitative,
and the being-for-itself, the qualitative,
are united therein.
Posited thus in itself, quantum is quantitative proportion
a determinacy that is just as much an immediate quantum
the exponent as it is mediation,
the relation  of a given quantum to another,
these being the two sides of the proportion
that at the same time are not to be taken
in their immediate value,
but whose value lies exclusively in this relation.

The sides of the proportion are still immediate quanta
and the qualitative and quantitative determinations are
still external to each other.
But as for what they truly are,
that the quantitative in its externality is
itself the relation to itself,
or that being-for-itself and the indifference
of the determinacy are united,
this is measure.
