BOOK TWO

The Doctrine of Essence

ESSENCE

The truth of being is essence.

Being is the immediate.
Since the goal of knowledge is the truth,
what being is in and for itself,
knowledge does not stop at
the immediate and its determinations,
but penetrates beyond it on
the presupposition that
behind this being there still is
something other than being itself,
and that this background
constitutes the truth of being.
This cognition is a mediated knowledge,
for it is not to be found
with and in essence immediately,
but starts off from an other, from being,
and has a prior way to make,
the way that leads over and beyond being
or that rather penetrates into it.
Only inasmuch as knowledge recollects
itself into itself out of immediate being,
does it find essence through this mediation.
The German language has kept “essence” (Wesen)
in the past participle (gewesen) of the verb “to be” (sein),
for essence is past [but timelessly past] being.

When this movement is represented as a pathway of knowledge,
this beginning with being and the subsequent advance
which sublates being and arrives at essence as a mediated term
appears to be an activity of cognition external to being
and indifferent to its nature.

But this course is the movement of being itself.
That it is being's nature to recollect itself,
and that it becomes essence by virtue of this interiorizing,
this has been displayed in being itself.

If, therefore, the absolute was
at first determined as being,
now it is determined as essence.
Cognition cannot in general stop
at the manifold of existence;
but neither can it stop at being, pure being;
immediately one is forced to the reflection that
this pure being, this negation of everything finite,
presupposes a recollection and a movement
which has distilled immediate existence into pure being.
Being thus comes to be determined as essence,
as a being in which everything determined and finite is negated.
So it is simple unity, void of determination,
from which the determinate has been
removed in an external manner;
to this unity the determinate was
itself something external
and, after this removal,
it still remains opposite to it;
for it has not been sublated in itself but relatively,
only with reference to this unity.
We already noted above that if pure essence is defined
as the sum total of all realities,
these realities are equally subject to
the nature of determinateness and abstractive reflection
and their sum total is reduced to empty simplicity.
Thus defined, essence is only a product, an artifact.
External reflection, which is abstraction, only lifts
the determinacies of being out of what is left over as essence
and only deposits them, as it were, somewhere else,
letting them exist as before.
In this way, however, essence is neither in itself nor for itself;
it is by virtue of another, through external abstractive reflection;
and it is for another, namely for abstraction
and in general for the existent
which still remains opposite to it.
In its determination, therefore,
it is a dead and empty absence of determinateness.

As it has come to be here, however,
essence is what it is,
not through a negativity foreign to it,
but through one which is its own:
the infinite movement of being.
It is being-in-and-for-itself,
absolute in-itselfness;
since it is indifferent to
every determinateness of being,
otherness and reference to other have been sublated.
But neither is it only this in-itselfness;
as merely being-in-itself, it would be only
the abstraction of pure essence;
but it is being-for-itself just as essentially;
it is itself this negativity,
the self-sublation of otherness
and of determinateness.

Essence, as the complete turning back of being into itself,
is thus at first the indeterminate essence;
the determinacies of being are sublated in it;
it holds them in itself but without their being posited in it.
Absolute essence in this simple unity with itself has no existence.
But it must pass over into existence,
for it is being-in-and-for-itself;
that is to say, it differentiates
the determinations which it holds in itself,
and, since it is the repelling of itself from itself
or indifference towards itself, negative self-reference,
it thereby posits itself over against itself
and is infinite being-for-itself
only in so far as in thus
differentiating itself from itself
it is in unity with itself.
This determining is thus of another nature than
the determining in the sphere of being,
and the determinations of essence have another character
than the determinations of being.
Essence is absolute unity of being-in-itself and being-for-itself;
consequently, its determining remains inside this unity;
it is neither a becoming nor a passing over,
just as the determinations themselves are
neither an other as other nor references to some other;
they are self-subsisting but, as such,
at the same time conjoined in the unity of essence.
Since essence is at first simple negativity,
in order to give itself existence and then being-for-itself,
it must now posit in its sphere the determinateness
which it contains in principle only in itself.

Essence is in the whole what quality was in the sphere of being:
absolute indifference with respect to limit.
Quantity is instead this indifference in immediate determination,
limit being in it an immediate external determinateness;
quantity passes over into quantum;
the external limit is necessary to it and exists in it.
In essence, by contrast, the determinateness does not exist;
it is posited only by the essence itself,
not free but only with reference to
the unity of the essence.
The negativity of essence is reflection,
and the determinations are reflected,
posited by the essence itself
in which they remain as sublated.

Essence stands between being and concept;
it makes up their middle,
its movement constituting the transition
of being into the concept.
Essence is being-in-and-for-itself,
but it is this in the determination of being-in-itself;
for its general determination is that it emerges from being
or that it is the first negation of being.
Its movement consists in positing negation
or determination in being,
thereby giving itself existence
and becoming as infinite being-for-itself
what it is in itself.
It thus gives itself its existence
which is equal to its being-in-itself
and becomes concept.
For the concept is the absolute
as it is absolutely,
or in and for itself,
in its existence.
But the existence which essence gives to itself is
not yet existence as it is in and for itself
but as essence gives it to itself or as posited,
and hence still distinct from the existence of the concept.

First, essence shines within itself
or is reflection;
second, it appears;
third, it reveals itself.

In the course of its movement,
it posits itself in the following determinations:

I. As simple essence existing in itself,
remaining in itself in its determinations;

II. As emerging into existence,
or according to its concrete existence and appearance;

III. As essence which is one with its appearance,
as actuality.

SECTION I

Essence as Reflection Within

IV.1
jati-antara-parinama prakrti-apurat

Essence issues from being;
hence it is not immediately in and for itself
but is a result of that movement.
Or, since essence is taken at first as something immediate,
it is a determinate existence to which another stands opposed;
it is only essential existence, as against the unessential.
But essence is being which has been sublated in and for itself;
what stands over against it is only shine.
The shine, however, is essence's own positing.

First, essence is reflection.
Reflection determines itself;
its determinations are a positedness
which is immanent reflection at the same time.
Second, these reflective determinations
or essentialities are to be considered.
Third, as the reflection of its immanent determining,
essence turns into foundation and passes over
into concrete existence and appearance.

CHAPTER 1

Shine

As it issues from being, essence seems to stand over against it;
this immediate being is, first, the unessential.

But, second, it is more than just the unessential;
it is being void of essence; it is shine.

Third, this shine is not something external,
something other than essence, but is essence's own shining.
This shining of essence within it is reflection.

IV.2
nimittam aprayojakam prakrtinam varana-bhedas tu tata ksetrikavat

A. THE ESSENTIAL AND THE UNESSENTIAL

Essence is sublated being.

It is simple equality with itself
but is such as the negation of
the sphere of being in general.
And so it has immediacy over against it,
as something from which it has come to be
but which has preserved and maintained itself in this sublating.
Essence itself is in this determination
an existent immediate essence,
and with reference to it
being is only something negative,
nothing in and for itself:
essence, therefore, is a determined negation.
Being and essence relate to each other in this fashion
as against others in general which are mutually indifferent,
for each has a being, an immediacy,
and according to this being they stand in equal value.

But as contrasted with essence,
being is at the same time the unessential;
as against essence, it has the determination of something sublated.
And in so far as it thus relates to essence
as an other only in general,
essence itself is not essence proper
but is just another existence, the essential.

The distinction of essential and unessential has
made essence relapse into the sphere of existence,
for as essence is at first,
it is determined with respect to being
as an existent and therefore as an other.
The sphere of existence is thus laid out as foundation,
and that in this sphere being is being-in-and-for-itself,
is a further determination external to existence,
just as, contrariwise, essence is indeed being-in-and-for-itself,
but only over against an other, in a determinate respect.
Consequently, inasmuch as essential and unessential aspects are
distinguished in an existence from each other,
this distinguishing is an external positing,
a taking apart that leaves the existence itself untouched;
it is a separation which falls on the side of
a third and leaves undetermined
what belongs to the essential
and what belongs to the unessential.
It is dependent on some external standpoint or consideration
and the same content can therefore sometimes be considered
as essential, sometimes as unessential.

On closer consideration, essence becomes something
only essential as contrasted with an unessential
because essence is only taken,
is as sublated being or existence.
In this fashion, essence is only the first negation,
or the negation, which is determinateness,
through which being becomes only existence,
or existence only an other.
But essence is the absolute negativity of being;
it is being itself, but not being determined only as an other:
it is being rather that has sublated itself
both as immediate being
and as immediate negation,
as the negation which is affected by an otherness.
Being or existence, therefore, does not persist
except as what essence is,
and the immediate which still differs from essence is
not just an unessential existence
but an immediate which is null in and for itself;
it only is a non-essence, shine.

IV.3
nirmana-cittani-asmita-matra

B. SHINE

1. Being is shine.

The being of shine consists solely
in the sublatedness of being,
in being's nothingness;
this nothingness it has in essence,
apart from its nothingness,
apart from essence, it does not exist.
It is the negative posited as negative.

Shine is all that remains of the sphere of being.
But it still seems to have an immediate side
which is independent of essence
and to be, in general, an other of essence.
Other entails in general the two moments
of existence and non-existence.
Since the unessential no longer has a being,
what is left to it of otherness is
only the pure moment of non-existence;
shine is this immediate non-existence,
a non-existence in the determinateness of being,
so that it has existence only with reference to another,
in its non-existence;
it is the non-self-subsistent
which exists only in its negation.
What is left over to it is thus only
the pure determinateness of immediacy;
it is as reflected immediacy, that is,
one which is only by virtue of
the mediation of its negation
and which, over against this mediation, is
nothing except the empty determination
of the immediacy of non-existence.

Shine, the “phenomenon” of skepticism,
and also the “appearance” of idealism,
is thus this immediacy which is not
a something nor a thing in general,
not an indifferent being that would exist apart
from its determinateness and connection with the subject.
Skepticism did not permit itself to say “It is,”
and the more recent idealism did not permit itself
to regard cognitions as a knowledge of the thing-in-itself.
The shine of the former was supposed absolutely
not to have the foundation of a being:
the thing-in-itself was not supposed
to enter into these cognitions.
But at the same time skepticism allowed
a manifold of determinations for its shine,
or rather the latter turned out to have
the full richness of the world for its content.
Likewise for the appearance of idealism:
it encompassed the full range of these manifold determinacies.
So, the shine of skepticism and the appearance of idealism
do immediately have a manifold of determination.
This content, therefore, might well have no being as foundation,
no thing or thing-in-itself;
for itself, it remains as it is;
it is simply transposed from being into shine,
so that the latter has within itself those manifold
determinacies that exist immediately,
each an other to the other.
The shine is thus itself something immediately determined.
It can have this or that content;
but whatever content it has, it has not posited it
but possesses it immediately.
Idealism, whether Leibnizian, Kantian, Fichtean, or in any
other form, has not gone further than skepticism in this:
it has not advanced beyond being as determinateness.
Skepticism lets the content of its shine to be given to it;
the shine exists for it immediately,
whatever content it might have.
The Leibnizian monad develops its representations from itself
but is not their generating and controlling force;
they rise up in it as a froth, indifferent,
immediately present to each other and to the monad as well.
Likewise Kant's appearance is a given content of perception
that presupposes affections, determinations of the subject
which are immediate to each other and to the subject.
As for the infinite obstacle of Fichte's Idealism,
it might well be that it has no thing-in-itself for foundation,
so that it becomes a determinateness purely within the “I.”
But this determinateness that the “I” makes its own,
sublating its externality,
is to the “I” at the same time an immediate determinateness,
a limitation of the “I” which the latter may transcend
but which contains a side of indifference,
and on account of this indifference,
although internal to the “I,”
it entails an immediate non-being of it.

2. Shine thus contains an immediate presupposition,
an independent side vis-à-vis essence.
But the task, inasmuch as this shine is distinct from essence,
is not to demonstrate that it sublates itself
and returns into essence,
for being has returned into essence in its totality;
shine is the null as such.
The task is to demonstrate that the determinations which
distinguish it from essence are the determinations of essence itself;
further, that this determinateness of essence,
which shine is, is sublated in essence itself.

What constitutes the shine is
the immediacy of non-being;
this non-being, however, is nothing else than
the negativity of essence within essence itself.
In essence, being is non-being.
Its inherent nothingness is the
negative nature of essence itself.
But the immediacy or indifference
which this non-being contains is
essences's own absolute in-itself.
The negativity of essence is its self-equality
or its simple immediacy and indifference.
Being has preserved itself in essence inasmuch
as this latter, in its infinite negativity,
has this equality with itself;
it is through this that essence is itself being.
The immediacy that the determinateness has
in shine against essence is
thus none other than essence's own immediacy,
though not the immediacy of an existent
but rather the absolutely mediated
or reflective immediacy which is shine;
being, not as being, but only as
the determinateness of being as against mediation;
being as moment.

These two moments [nothingness but as subsisting],
and being but as moment;
or again, negativity existing in itself and reflected immediacy,
these two moments that are the moments of shine,
are thus the moments of essence itself;
it is not that there is a shine of being in essence,
or a shine of essence in being:
the shine in the essence is not the shine of an other
but is rather shine as such, the shine of essence itself.

Shine is essence itself in the determinateness of being.
Essence has a shine because it is determined within itself
and is therefore distinguished from its absolute unity.
But this determinateness is as determinateness
just as absolutely sublated in it.
For essence is what stands on its own:
it exists as self-mediating through a negation which it itself is.
It is, therefore, the identical unit of absolute negativity and immediacy.
The negativity is negativity in itself;
it is its reference to itself and thus immediacy in itself.
But it is negative reference to itself,
a self-repelling negating;
thus the immediacy existing in itself is
the negative or the determinate over against the negativity.
But this determinateness is itself absolute negativity
and this determining, which as determining immediately sublates itself,
is a turning back into itself.

Shine is the negative which has a being,
but in another, in its negation;
it is a non-self-subsisting-being
which is sublated within and null.
And so it is the negative which returns into itself,
the non-subsistent as such, internally non-subsistent.
This reference of the negative or
the non-subsistent to itself is
the immediacy of this non-subsistent;
it is an other than it;
it is its determinateness over against it,
or the negation over against the negative.
But this negation which stands over against the negative is
negativity as referring solely to itself,
the absolute sublation of the determinateness itself.

The determinateness that shine is in essence is,
therefore, infinite determinateness;
it is only the negative which coincides with itself
and hence a determinateness that, as determinateness,
is self-subsistence and not determined.
Contrariwise, the self-subsistence, as self-referring immediacy,
equally is just determinateness and moment,
negativity solely referring to itself.
This negativity which is identical with immediacy,
and thus the immediacy which is identical with negativity, is essence.
Shine is, therefore, essence itself,
but essence in a determinateness, in such a way, however,
that the determinateness is only a moment,
and the essence is the shining of itself within itself.

In the sphere of being, non-being arises over against being,
each equally an immediate, and the truth of both is becoming.
In the sphere of essence, we have the contrast
first of essence and the non-essential,
then of essence and shine,
the non-essential and the shine
being both the leftover of being.
But these two, and no less the
distinction of essence from them,
consist solely in this:
that essence is taken at first as an immediate,
not as it is in itself,
namely as an immediacy which is immediacy
as pure mediacy or absolute negativity.
This first immediacy is thus only the determinateness of immediacy.
The sublating of this determinateness of essence consists, therefore,
in nothing further than showing that the unessential is only shine,
and that essence rather contains this shine within itself.
For essence is an infinite self-contained movement
which determines its immediacy as negativity
and its negativity as immediacy,
and is thus the shining of itself within itself.
In this, in its self-movement,
essence is reflection.

IV.4
pravrtti-bhede prayojakam cittam ekam anekesam

C. REFLECTION

Shine is the same as what reflection is;
but it is reflection as immediate.
For this shine which is internalized
and therefore alienated from its immediacy,
the German has a word from an alien language, “Reflexion.”

Essence is reflection, the movement of
becoming and transition that remains within itself,
wherein that which is distinguished is determined
simply and solely as the negative in itself, as shine.
In the becoming of being, it is being which lies
at the foundation of determinateness,
and determinateness is reference to an other.
Reflective movement is by contrast
the other as negation in itself,
a negation which has being only as self-referring.
Or, since this self-referring is
precisely this negating of negation,
what we have is negation as negation,
negation that has its being
in its being-negated, as shine.
Here, therefore, the other is
not being with negation or limit,
but negation with negation.
But the first over against this other,
the immediate or being,
is only this self-equality itself of negation,
the negated negation, the absolute negativity.
This self-equality or immediacy, therefore, is
not a first from which the beginning is made
and which would pass over into its negation;
nor is there an existent substrate which would
go through the moves of reflection;
immediacy is rather just this movement itself.

In essence, therefore, the becoming,
the reflective movement of essence,
is the movement from nothing to nothing
and thereby back to itself.
Transition or becoming sublates itself in its transition;
the other which comes to be in this transition is
not the non-being of a being but the nothingness of a nothingness,
and this, to be the negation of a nothingness, constitutes being.
Being is only as the movement of nothingness to nothingness,
and so it is essence;
and this essence does not have this movement in itself,
but the movement is rather the absolute shine itself,
the pure negativity which has nothing outside it
which it would negate but which rather negates only its negative,
the negative which is only in this negating.
This pure absolute reflection, which is the movement
from nothing to nothing, further determines itself.

It is, first, positing reflection.

Second, it takes as its starting point
the presupposed immediate,
and then it is external reflection.

Third, it sublates however this presupposition,
and because in the sublating of the presupposition
it presupposes at the same time,
it is determining reflection.

1. Positing reflection

Shine is a nothingness or a lack of essence.
But a nothingness or that which is void of essence
does not have its being in an other in which it shines,
but its being is its own equality with itself;
this conversion of the negative with itself has been determined
as the absolute reflection of essence.

This self-referring negativity is
therefore the negating of itself.
It is thus just as much
sublated negativity as it is negativity.
Or again, it is itself the negative
and the simple equality with itself or immediacy.
It consists, therefore, in being itself
and not being itself,
and the two in one unity.

Reflection is at first the movement of
the nothing to the nothing,
and thus negation coinciding with itself.
This self-coinciding is in general
simple equality with itself, immediacy.
But this falling together is not
the transition of negation into equality
as into a being other than it;
reflection is transition rather
as the sublating of transition,
for it is the immediate falling together
of the negative with itself.
And so this coinciding is, first,
self-equality or immediacy;
but, second, this immediacy is
the self-equality of the negative,
and hence self-negating equality,
immediacy which is in itself the negative,
the negative of itself:
its being is to be what it is not.

The self-reference of the negative is
therefore its turning back into itself;
it is immediacy as the sublating of the negative,
but immediacy simply and solely as this reference
or as turning back from a one,
and hence as self-sublating immediacy.
This is positedness,
immediacy purely as determinateness
or as self-reflecting.
This immediacy, which is only as
the turning back of the negative into itself,
is the immediacy which constitutes the determinateness of shine,
and from which the previous reflective movement seemed to begin.
But, far from being able to begin with this immediacy,
the latter first is rather as the turning back
or as the reflection itself.
Reflection is therefore the movement which,
since it is the turning back,
only in this turning is that
which starts out or returns.

It is a positing, inasmuch as it is
immediacy as a turning back;
that is to say, there is not an other beforehand,
one either from which or to which it would turn back;
it is, therefore, only as a turning back
or as the negative of itself.
But further, this immediacy is sublated negation
and sublated return into itself.
Reflection, as the sublating of the negative, is
the sublating of its other, of the immediacy.
Because it is thus immediacy as a turning back,
the coinciding of the negative with itself,
it is equally the negation of the negative as negative.
And so it is presupposing.
Or immediacy is as a turning back
only the negative of itself,
just this, not to be immediacy;
but reflection is the sublating
of the negative of itself,
coincidence with itself;
it therefore sublates its positing,
and inasmuch as it is in its positing
the sublating of positing, it is presupposing.
In presupposing, reflection determines the turning back
into itself as the negative of itself,
as that of which essence is the sublating.
It is its relating to itself,
but to itself as to the negative of itself;
only so is it negativity which abides with itself,
self-referring negativity.
Immediacy comes on the scene simply and solely
as a turning back and is that negative
which is the semblance of a beginning,
the beginning which the return negates.
The turning back of essence is therefore its self-repulsion.
Or inner directed reflection is essentially
the presupposing of that from which
the reflection is the turning back.

It is only by virtue of the sublating of its equality with itself
that essence is equality with itself.
Essence presupposes itself, and the sublating of
this presupposing is essence itself;
contrariwise, this sublating of its presupposition is
the presupposition itself.
Reflection thus finds an immediate before it
which it transcends and from which it is the turning back.
But this turning back is only the presupposing of
what was antecedently found.
This antecedent comes to be only by being left behind;
its immediacy is sublated immediacy.
The sublated immediacy is, contrariwise, the turning
back into itself,
essence that arrives at itself, simple being equal to itself.
This arriving at itself is thus the sublating of itself
and self-repelling, presupposing reflection,
and its repelling of itself from itself is
the arriving at itself.

It follows from these considerations that
the movement of reflection is to be taken as
an absolute internal counter-repelling.
For the presupposition of
the turning back into itself
[that from which essence arises,
essence being only as this coming back]
is only in the turning back itself.
Transcending the immediate from which reflection begins
occurs rather only through this transcending;
and the transcending of the immediate is
the arriving at the immediate.
The movement, as forward movement, turns immediately
around into itself and so is only self-movement:
a movement which comes from itself in so far as
positing reflection is presupposing reflection, yet,
as presupposing reflection, is simply positing reflection.

Thus is reflection itself and its non-being,
and only is itself by being the negative of itself,
for only in this way is the sublating of the negative
at the same time a coinciding with itself.

The immediacy which reflection,
as a process of sublating,
presupposes for itself is
simply and solely a positedness,
something in itself sublated
which is not diverse from
reflection's turning back into itself
but is itself only this turning back.
But it is at the same time determined as a negative,
as immediately in opposition to something,
and hence to an other.
And so is reflection determined.
According to this determinateness,
because reflection has a presupposition
and takes its start from the immediate as its other,
it is external reflection.

2. External reflection

Reflection, as absolute reflection,
is essence shining within,
essence that posits only shine,
only positedness, for its presupposition;
and as presupposing reflection,
it is immediately only positing reflection.
But external or real reflection
presupposes itself as sublated,
as the negative of itself.
In this determination, it is doubled.
At one time it is as what is presupposed,
or the reflection into itself which is the immediate.
At another time, it is as the reflection
negatively referring to itself;
it refers itself to itself as
to that its non-being.

External reflection thus presupposes a being,
at first not in the sense that
its immediacy is only positedness or moment,
but in the sense rather that
this immediacy refers to itself
and the determinateness is only as moment.
Reflection refers to its presupposition in such a way
that the latter is its negative,
but this negative is thereby sublated as negative.
Reflection, in positing, immediately sublates its positing,
and so it has an immediate presupposition.
It therefore finds this presupposition before it
as something from which it starts,
and from which it only makes its way back into itself,
negating it as its negative.
But that this presupposition is a negative
or a positedness is not its concern;
this determinateness belongs only to positing reflection,
whereas in the presupposing positedness
it is only as sublated.
What external reflection determines and posits in the immediate
are determinations which to that extent are external to it.
In the sphere of being, external reflection was the infinite;
the finite stands as the first,
as the real from which the beginning is made
as from a foundation that abides,
whereas the infinite is the reflection into itself
standing over against it.

This external reflection is the syllogism
in which the two extremes are
the immediate and the reflection into itself;
the middle term is the reference connecting the two,
the determinate immediate, so that one part of
this connecting reference, the immediate,
falls to one extreme alone, and the other,
the determinateness or the negation,
only to the other extreme.

But if one takes a closer look at what the external reflection does,
it turns out that it is, secondly, the positing of the immediate,
an immediate which thus becomes the negative or the determined;
but it is immediately also the sublating of this positing,
for it presupposes the immediate;
in negating, it is the negating of its negating.
But thereby it immediately is equally a positing,
the sublating of the immediate which is its negative;
and this negative, from which it seemed to begin
as from something alien,
only is in this its beginning.
In this way, the immediate is not only implicitly in itself
(that is, for us or in external reflection)
the same as what reflection is,
but is posited as being the same.
For the immediate is determined by reflection as
the negative of the latter or as the other of it,
but it is reflection itself which negates this determining.
The externality of reflection vis-à-vis
the immediate is consequently sublated;
its self-negating positing is its coinciding
with its negative, with the immediate,
and this coinciding is the immediacy of essence itself.
It thus transpires that external reflection is not external
but is just as much the immanent reflection of immediacy itself;
or that the result of positing reflection is
essence existing in and for itself.
External reflection is thus determining reflection.

3. Determining reflection

Determining reflection is in general
the unity of positing and external reflection.
This is now to be examined more closely.

1. External reflection begins from immediate being,
positing reflection from nothing.
In its determining, external reflection posits another in the
place of the sublated being, but this other is essence;
the positing does not posit its determination in the place of an other;
it has no presupposition.
But, precisely for this reason,
it is not complete as determining reflection;
the determination which it posits is consequently only a posited;
this is an immediate, not however as equal to itself
but as self-negating;
its connection with the turning back into itself is absolute;
it is only in the reflection-into-itself
but is not this reflection itself.

The posited is therefore an other,
but in such a manner that the self-equality
of reflection is retained;
for the posited is only as sublated,
as reference to the turning back into itself.
In the sphere of being, existence was the being
that had negation in it, and being was the immediate ground
and element of this negation which was,
therefore, itself immediate negation.
In the sphere of essence,
positedness is what corresponds to existence.
Positedness is equally an existence,
but its ground is being as essence
or as pure negativity;
it is a determinateness or a negation,
not as existent but immediately as sublated.
Existence is only positedness;
this is the principle of the essence of existence.
Positedness stands on the one side over against existence,
and over against essence on the other:
it is to be regarded as the means which conjoins
existence with essence and essence with existence.
If it is said, a determination is only a positedness,
the claim can thus have a twofold meaning,
according to whether the determination is such
in opposition to existence or in opposition to essence.
In either meaning, existence is taken for
something superior to positedness,
which is attributed to external reflection, to the subjective.
In fact, however, positedness is the superior, because, as posited,
existence is what it is in itself something negative,
something that refers simply and solely to the turning back into itself.
For this reason positedness is only a positedness
with respect to essence:
it is the negation of this turning back
as achieved return into itself.

2. Positedness is not yet a determination of reflection;
it is only determinateness as negation in general.
But the positing is now united with external reflection;
in this unity, the latter is absolute presupposing, that is,
the repelling of reflection from itself
or the positing of determinateness as its own.
As posited, therefore, positedness is negation;
but as presupposed, it is reflected into itself.
And in this way positedness is a determination of reflection.

The determination of reflection is distinct
from the determinateness of being, of quality;
the latter is immediate reference to other in general;
positedness also is reference to other,
but to immanently reflected being.
Negation as quality is existent negation;
being constitutes its ground and element.
The determination of reflection, on the contrary,
has for this ground immanent reflectedness.
Positedness gets fixed in determination precisely
because reflection is self-equality in its negatedness;
the latter is therefore itself reflection into itself.
Determination persists here, not by virtue of being
but because of its self-equality.
Since the being which sustains quality is
unequal to the negation, quality is
consequently unequal within itself,
and hence a transient moment which disappears in the other.
The determination of reflection is
on the contrary positedness as negation,
negation which has negatedness for its ground,
is therefore not unequal to itself within itself,
and hence essential rather than transient determinateness.
What gives subsistence to it is the self-equality of reflection
which has the negative only as negative,
as something sublated or posited.

Because of this reflection into themselves,
the determinations of reflection appear as
free essentialities, sublated in the void
without reciprocal attraction or repulsion.
In them the determinateness has become entranced
and infinitely fixed by virtue of the reference to itself.
It is the determinate which has subjugated its transitoriness
and its mere positedness to itself, that is to say,
has deflected its reflection-into-other into reflection-into- itself.
These determinations hereby constitute the determinate shine
as it is in essence, the essential shine.
Determining reflection is for this reason
reflection that has exited from itself;
the equality of essence with itself is
lost in the negation, and negation predominates.

Thus there are two distinct sides to the determination of reflection.
First, reflection is positedness, negation as such;
second, it is immanent reflection.
According to the side of positedness,
it is negation as negation,
and this already is its unity with itself.
But it is this unity at first only implicitly or in itself,
an immediate which sublates itself within, is the other of itself.
To this extent, reflection is a determining that abides in itself.
In it essence does not exit from itself;
the distinctions are solely posited,
taken back into essence.
But, from the other side, they are not posited
but are rather reflected into themselves;
negation as negation is equality with itself,
not in its other, not reflected into its non-being.

3. Now keeping in mind that the determination of reflection is
both immanently reflected reference and positedness as well,
its nature immediately becomes more transparent.
For, as positedness, the determination is negation as such,
a non-being as against another, namely,
as against the absolute immanent reflection or as against essence.
But as self-reference, it is reflected within itself.
This, the reflection of the determination,
and that positedness are distinct;
its positedness is rather the sublatedness of the determination
whereas its immanent reflectedness is its subsisting.
In so far as now the positedness is
at the same time immanent reflection,
the determinateness of the reflection is
the reference in it to its otherness.
It is not a determinateness that exists quiescent,
one which would be referred to an other
in such a way that the referred term
and its reference would be different,
each something existing in itself,
each a something that excludes its other
and its reference to this other from itself.
Rather, the determination of reflection is
within it the determinate side
and the reference of this determinate side as determinate,
that is, the reference to its negation.
Quality, through its reference, passes over into another;
its alteration begins in its reference.
The determination of reflection, on the contrary,
has taken its otherness back into itself.
It is positedness, negation which has however deflected
the reference to another into itself,
and negation which, equal to itself,
is the unity of itself and its other,
and only through this is an essentiality.
It is, therefore, positedness, negation,
but as reflection into itself it is at the same time
the sublatedness of this positedness,
infinite reference to itself.

CHAPTER 2

Foundation

IV.5
tatra dhyana-jam anasayam

The essentialities or the determinations of reflection

Reflection is determined reflection;
accordingly, essence is determined essence, or it is essentiality.

Reflection is the shining of essence within itself.

Essence, as infinite immanent turning back is
not immediate simplicity, but negative simplicity;
it is a movement across moments that are distinct,
is absolute mediation with itself.
But in these moments it shines;
the moments are, therefore, themselves
determinations reflected into themselves.

First, essence is simple self-reference, pure identity.
This is its determination, one by which it is rather
the absence of determination.

Second, the specifying determination is difference,
difference which is either external or indefinite,
diversity in general, or opposed diversity or opposition.

Third, as contradiction this opposition is reflected into itself
and returns to its foundation.

CHAPTER 3

Ground

Essence determines itself as ground.

Just as nothing is at first
in simple immediate unity with being,
so here too the simple identity of essence is
at first in simple unity with its absolute negativity.
Essence is only this negativity which is pure reflection.
It is this pure reflection as
the turning back of being into itself;
hence it is determined, in itself or for us,
as the ground into which being resolves itself.
But this determinateness is not posited by the essence itself;
in other words, essence is not ground precisely because
it has not itself posited this determinateness that it possesses.
Its reflection, however, consists in positing itself as
what it is in itself, as a negative, and in determining itself.
The positive and the negative constitute the essential determination
in which essence is lost in its negation.
These self-subsisting determinations of reflection sublate themselves,
and the determination that has foundered to the ground is
the true determination of essence.

Consequently, ground is itself one of
the reflected determinations of essence,
but it is the last, or rather,
it is determination determined as sublated determination.
In foundering to the ground, the determination of reflection
receives its true meaning that it is the absolute
repelling of itself within itself;
or again, that the positedness that accrues to essence is
such only as sublated,
and conversely that only the self-sublating positedness is
the positedness of essence.
In determining itself as ground,
essence determines itself as the not-determined,
and only the sublating of its being determined is its determining.
Essence, in thus being determined as self-sublating,
does not proceed from an other but is,
in its negativity, identical with itself.

Since the advance to the ground is made starting
from determination as an immediate first
(is done by virtue of the nature of determination itself
that founders to the ground through itself),
the ground is at first determined by that immediate first.
But this determining is, on the one hand,
as the sublating of the determining,
the merely restored, purified or manifested identity of essence
which the determination of reflection is in itself;
on the other hand, this negating movement is, as determining,
the first positing of that reflective determinateness
that appeared as immediate determinateness,
but which is posited only by the self-excluding reflection of ground
and therein is posited as only something posited or sublated.
Thus essence, in determining itself as ground, proceeds only from itself.
As ground, therefore, it posits itself as essence,
and its determining consists in just this positing of itself as essence.
This positing is the reflection of essence
that sublates itself in its determining;
on that side is a positing, on this side is the positing of essence,
hence both in one act.

Reflection is pure mediation in general;
ground, the real mediation of essence with itself.
The former, the movement of nothing through nothing back to itself,
is the reflective shining of one in an other;
but, because in this reflection opposition does not
yet have any self-subsistence,
neither is the one, that which shines, something positive,
nor is the other in which it reflectively shines something negative.
Both are substrates, actually of the imagination;
they are still not self-referring.
Pure mediation is only pure reference,
without anything being referred to.
Determining reflection, for its part, does posit
such terms as are identical with themselves;
but these are at the same time only determined references.
Ground, on the contrary, is mediation that is real,
since it contains reflection as sublated reflection;
it is essence that turns back into itself
through its non-being and posits itself.
According to this moment of sublated reflection,
what is posited receives the determination of immediacy,
of an immediate which is self-identical
outside its reference or its reflective shining.
This immediacy is being as restored by essence,
the non-being of reflection through which essence mediates itself.
Essence returns into itself as it negates;
therefore, in its turning back into itself,
it gives itself the determinateness that precisely
for this reason is the self-identical negative,
is sublated positedness, and consequently,
as the self-identity of essence as ground,
equally an existent.

The ground is, first, absolute ground
one in which the essence is first of all
the general substrate for the ground-connection.
It then further determines itself as form and matter
and gives itself a content.

Second, it is determinate ground,
the ground of a determinate content.
Because the ground-connection, in being realized,
becomes as such external,
it passes over into conditioning mediation.

Third, ground presupposes a condition;
but the condition equally presupposes the ground;
the unconditioned is the unity of the two,
the fact itself that, by virtue of
the mediation of the conditioning reference,
passes over into concrete existence.

IV.6
karma-asukla-akrsnam yogina trividham itaresam

IV.7
tatas tad-vipaka-anugunanam eva-abhivyakti vasananam

A. ABSOLUTE GROUND

a. Form and essence

The determination of reflection,
inasmuch as this determination returns into ground,
is a first immediate existence in general
from which the beginning is made.
But existence still has only the meaning of positedness
and essentially presupposes a ground,
in the sense that it does not really posit a ground;
that the positing is a sublating of itself;
that it is rather the immediate that is posited,
and the ground the non-posited.
As we have seen, this presupposing is the positing
that rebounds on that which posits;
as sublated determinate being, the ground is not an indeterminate
but is rather essence determined through itself,
but determined as indeterminate or as sublated positedness.
It is essence that in its negativity is identical with itself.
The determinateness of essence as ground is thus twofold:
it is the determinateness of the ground and of the grounded.
It is, first, essence as ground,
essence determined to be essence
as against positedness, as non-positedness.
Second, it is that which is grounded,
the immediate that, however, is not anything in and for itself:
is positedness as positedness.
Consequently, this positedness is equally identical with itself,
but in an identity which is that of the negative with itself.
The self-identical negative and the self-identical positive
are now one and the same identity.
For the ground is the self-identity
of the positive or even also of positedness;
the grounded is positedness as positedness,
but this its reflection-into-itself is the identity of the ground.
This simple identity, therefore, is not itself ground,
for the ground is essence posited as
the non-posited as against positedness.
As the unity of this determinate identity (the ground)
and of the negative identity (the grounded),
it is essence in general distinct from its mediation.

For one thing, this mediation,
compared with the preceding reflections
from which it derives,
is not pure reflection,
which is undistinguished from essence
and still does not have the negative in it,
consequently also does not as yet contain
the self-subsistence of the determinations.
These have their subsistence, rather,
in the ground understood as sublated reflection.
And it is also not the determining reflection
whose determinations have essential self-subsistence,
for that reflection has foundered, has sunk to the ground,
and in the unity of the latter
the determinations are only posited determinations.
This mediation of the ground is thus
the unity of pure reflection and determining reflection;
their determinations or that which is posited has self-subsistence,
and conversely the self-subsistence of
the determinations is a posited subsistence.
Since this subsistence of the determinations is
itself posited or has determinateness,
the determinations are consequently distinguished
from their simple identity,
and they constitute the form as against essence.

Essence has a form and determinations of this form.
Only as ground does it have a fixed immediacy or is substrate.
Essence as such is one with its reflection,
inseparable from its movement.
It is not essence, therefore, through which
this movement runs its reflective course;
nor is essence that from which the movement begins,
as from a starting point.
It is this circumstance that above all makes
the exposition of reflection especially difficult,
for strictly speaking one cannot say
that essence returns into itself,
that essence shines in itself,
for essence is neither before its movement nor in the movement:
this movement has no substrate on which it runs its course.
A term of reference arises in the ground only following upon the
moment of sublated reflection.
But essence as the referred-to term is determinate essence,
and by virtue of this positedness it has form as essence.
The determinations of form, on the contrary,
are now determinations in the essence;
the latter lies at their foundation
as an indeterminate which in its determination
is indifferent to them;
in it, they are reflected into themselves.
The determinations of reflection should have
their subsistence in them and be self-subsistent.
But their self-subsistence is their dissolution,
which they thus have in an other;
but this dissolution is itself this self-identity
or the ground of the subsistence that they give to themselves.
Everything determinate belongs in general to form;
it is a form determination inasmuch as it is something posited
and hence distinguished from that of which it is the form.
As quality, determinateness is one with its substrate, being;
being is the immediate determinate,
not yet distinct from its determinateness
or, in this determinateness,
still unreflected into itself,
just as the determinateness is, therefore,
an existent determinateness,
not yet one that is posited.
Moreover, the form determinations of essence are,
in their more specific determinateness,
the previously considered moments of reflections:
identity and difference, the latter as both
diversity and opposition.
But also the ground-connection belongs among
these form determinations of essence,
because through it, though itself the
sublated determination of reflection,
essence is at the same time as posited.
By contrast, the identity that has the ground immanent in it
does not pertain to form, because positedness,
as sublated and as such (as ground and grounded),
is one reflection, and this reflection constitutes
essence as simple substrate which is the subsistence of form.
But in ground this subsistence is posited,
or this essence is itself essentially as determinate and, consequently,
is in turn also the moment of the ground-connection and form.
This is the absolute reciprocal connecting reference of form and essence:
essence is the simple unity of ground and grounded
but, in this unity, is itself determined, or is a negative,
and it distinguishes itself as substrate from form,
but at the same time it thereby becomes itself
ground and moment of form.

Form is therefore the completed whole of reflection;
it also contains this determination of reflection, that it is sublated;
just like reflection, therefore, it is one unity of its determining,
and it is also referred to its sublatedness,
to another that is not itself form but in which the form is.
As essential self-referring negativity,
in contrast with that simple negative,
form is positing and determining;
simple essence, on the contrary, is indeterminate and inert
substrate in which the determinations of form have their subsistence
or their reflection into themselves.
External reflection normally halts at
this distinction of essence and form;
the distinction is necessary,
but the distinguishing itself of the two is their unity,
just as this unity of ground is essence repelling itself from itself
and making itself into positedness.
Form is absolute negativity itself
or the negative absolute self-identity
by virtue of which essence is indeed not being but essence.
This identity, taken abstractly, is essence as against form,
just as negativity, taken abstractly as positedness,
is the one determination of form.
But this determination has shown itself to be in truth
the whole self-referring negativity
which within, as this identity, thus is simple essence.
Consequently, form has essence in its own identity,
just as essence has absolute form in its negative nature.
One cannot therefore ask, how form comes to essence,
for form is only the internal reflective shining of essence,
its own reflection inhabiting it.
Form equally is, within it,
the reflection turning back into itself
or the identical essence;
in its determining, form makes the determination
into positedness as positedness.

Form, therefore, does not determine essence,
as if it were truly presupposed, separate from essence,
for it would then be the unessential,
constantly foundering determination of reflection;
here it rather is itself the ground of its sublating
or the identical reference of its determinations.
That the form determines the essence means, therefore,
that in its distinguishing form sublates this very distinguishing
and is the self-identity that essence is
as the subsistence of the determinations;
form is the contradiction of being sublated in its positedness
and yet having subsistence in this sublatedness;
it is accordingly ground as essence which
is self-identical in being determined or negated.
These distinctions, of form and of essence,
are therefore only moments of the simple reference of form itself.
But they must be examined and fixed more closely.
Determining form refers itself to itself as sublated positedness;
it thereby refers itself to its identity as to another.
It posits itself as sublated;
it therefore presupposes its identity;
according to this moment, essence is the indeterminate
to which form is an other.
It is not the essence which is absolute reflection within,
but essence determined as formless identity:
it is matter.

b. Form and matter

Essence becomes matter in that its reflection is
determined as relating itself
to essence as to the formless indeterminate.
Matter, therefore, is the simple identity,
void of distinction, that essence is,
with the determination that it is the other of form.
Hence it is the proper base or substrate of form,
since it constitutes the immanent reflection
of the determinations of form,
or the self-subsistent term,
to which such determinations refer
as to their positive subsistence.

If abstraction is made from every determination,
from every form of a something, matter is what is left over.
Matter is the absolutely abstract.
(One cannot see, feel, etc. matter;
what one sees or feels is a determinate matter,
that is, a unity of matter and form.)
This abstraction from which matter derives is not, however,
an external removal and sublation of form;
it is rather the form itself which, as we have just seen,
reduces itself by virtue of itself to this simple identity.

Further, form presupposes a matter to which it refers.
But for this reason the two do not find themselves
confronting each other externally and accidentally;
neither matter nor form derives from itself, is a se,
or, in other words, is eternal.
Matter is indifferent with respect to form,
but this indifference is the determinateness
of self-identity to which
form returns as to its substrate.
Form presupposes matter for the very reason
that it posits itself as a sublated,
hence refers to this, its identity,
as to something other.
Contrariwise, form is presupposed by matter;
for matter is not simple essence,
which immediately is itself absolute reflection,
but is essence determined as something positive,
that is to say, which only is as sublated negation.
But, on the other hand, since form posits itself as matter only in
sublating itself, hence in presupposing matter,
matter is also determined as groundless subsistence.
Equally so, matter is not determined as the ground of form;
but rather, inasmuch as matter posits itself as the abstract identity
of the sublated determination of form,
it is not that identity as ground,
and form is therefore groundless with respect to it.
Form and matter are consequently alike determined as
not to be posited each by the other,
each not to be the ground of the other.
Matter is rather the identity of
the ground and the grounded,
as the substrate that stands over
against this reference of form.
This determination of indifference that
the two have in common is
the determination of matter as such
and also constitutes their reciprocal reference.
The determination of form, that it is
the connection of the two as distinct,
equally is also the other moment of
the relating of the two to each other.
Matter, determined as indifferent,
is the passive as contrasted to form,
which is determined as the active.
This latter, as self-referring negative,
is inherently contradiction, self-dissolving,
self-repelling, and self-determining.
It refers to matter, and it is posited to refer to this matter,
which is its subsistence, as to another.
Matter is posited, on the contrary,
as referring only to itself
and as indifferent to the other;
but, implicitly, it does refer to the form,
for it contains the sublated negativity
and is matter only by virtue of this determination.
It refers to it as an other only
because form is not posited in it,
because it is form only implicitly.
It contains form locked up inside it,
and it is an absolute receptivity for form
only because it has the latter within it absolutely,
because to be form is its implicit vocation.
Hence matter must be informed,
and form must materialize itself;
it must give itself self-identity
or subsistence in matter.

2. Consequently, form determines matter,
and matter is determined by form.
Because form is itself absolute self-identity
and hence implicitly contains matter;
and equally because matter in its pure abstraction
or absolute negativity possesses form within it,
the activity of the form on the matter
and the reception by the latter of the form determination is only
the sublating of the semblance of their indifference and distinctness.
Thus the determination referring each to the other is
the self-mediation of each through its own non-being.
But the two mediations are one movement,
and the restoration of their original identity is
the inner recollection of their exteriorization.

First, form and matter presuppose each other.
As we have seen, this only means that the one essential unity is
negative self-reference, and that it therefore splits,
determined as an indifferent substrate in the essential identity,
and as determining form in essential distinction or negativity.
That unity of essence and form, the two opposed to each other as
form and matter, is the absolute self-determining ground.
Inasmuch as this unity differentiates itself,
the reference connecting the two diverse terms,
because of the unity that underlies them,
becomes a reference of reciprocal presupposition.

Second, the form already is, as self-subsisting,
self-sublating contradiction;
but it is also posited as in this way self-sublating,
for it is self-subsisting and at the same time
essentially referred to another,
and consequently it sublates itself.
Since it is itself two-sided, its sublating also has two sides.
For one, form sublates its self-subsistence
and transforms itself into something posited,
something that exists in an other,
and this other is in its case matter.
For the other, form sublates its determinateness vis-à-vis matter,
sublates its reference to it, consequently its positedness,
and it thereby gives itself subsistence.
Its reflection in thus sublating its positedness is
its own identity into which it passes over.
But since form at the same time externalizes this identity
and posits it over against itself as matter,
that reflection of the positedness into itself is
a union with a matter in which it obtains subsistence.
In this union, therefore, it is equally both:
is united with matter as with something other
(in accordance with the first side, viz. in that it makes
itself into a positedness),
and, in this other, is united with its own identity.

The activity of form by which matter is determined consists,
therefore, in a negative relating of the form to itself.
But, conversely, form thereby negatively relates itself to matter also;
the movement, however, by which matter becomes determined is
just as much the form's own movement.
Form is free of matter, but it sublates its self-subsistence;
but this, its self-subsistence, is matter itself,
for it is in this matter that it has its essential identity.
It makes itself into a positedness, but this is one and the same
as making matter into something determinate.
But, considered from the other side,
the form's own identity is at the same time externalized,
and matter is its other;
for this reason, because form sublates its own self-subsistence,
matter is also not determined.
But matter only subsists vis-à-vis form;
as the negative sublates itself, so does the positive also.
And as the form sublates itself, the determinateness of matter
that the latter has vis-à-vis form also falls away
the determinateness, namely, of being the indeterminate subsistence.

What appears here as the activity of form is, moreover,
just as much the movement that belongs to matter itself.
The determination that implicitly exists in matter,
what matter is supposed to be, is its absolute negativity.
Through it matter does not just refer to form simply as to an other,
but this external other is the form rather that
matter itself contains locked up within itself.
Matter is in itself the same contradiction that form contains,
and this contradiction, like its resolution, is only one.
But matter is thus in itself self-contradictory because,
as indeterminate self-identity,
it is at the same time absolute negativity;
it sublates itself within:
its identity disintegrates in its negativity
while the latter obtains in it its subsistence.
Since matter is therefore determined by form as by something external,
it thereby attains its determination,
and the externality of the relating, for both form and matter,
consists in that each, or rather in that the original unity of each,
in positing is at the same time presupposing:
the result is that self-reference is at the same time
a reference to the self as sublated or is reference to its other.

Third, through this movement of form and matter,
the original unity of the two is, on the one hand, restored;
on the other hand, it is henceforth a posited unity.
Matter is just as much a self-determining as
this determining is for it an activity of form external to it;
contrariwise, form determines only itself,
or has the matter that it determines within it,
just much as in its determining it relates itself to another;
and both, the activity of form and the movement of matter,
are one and the same thing, only that the former is an activity,
that is, it is the negativity as posited,
while the latter is movement or becoming,
the negativity as determination existing in itself.
The result, therefore, is the unity of the in-itself and positedness.
Matter is as such determined or necessarily has a form,
and form is simply material, subsistent form.

Inasmuch as form presupposes a matter as its other, it is finite.
It is not a ground but only the active factor.
Equally so, matter, inasmuch as it presupposes
form as its non-being, is finite matter;
it is not the ground of its unity with form
but is for the latter only the substrate.
But neither this finite matter nor the finite form have any truth;
each refers to the other, or only their unity is their truth.
The two determinations return to
this unity and there they sublate their self-subsistence;
the unity thereby proves to be their ground.
Consequently, matter is the ground
of its form determination not as matter
but only inasmuch as it is the absolute unity of essence and form;
similarly, form is the ground of the subsistence of its determinations
only to the extent that it is that same one unity.
But this one unity, as absolute negativity,
and more specifically as exclusive unity,
is, in its reflection, a presupposing;
or again, that unity is one act,
of preserving itself as positedness in positing,
and of repelling itself from itself;
of referring itself to itself as itself
and to itself as to another.
Or, the act by which matter is determined by form
is the self-mediation of essence as ground, in one unity:
through itself and through the negation of itself.

Informed matter or form that possesses subsistence is now,
not only this absolute unity of ground with itself,
but also unity as posited.
The movement just considered is the one
in which the absolute ground has exhibited
its moments at once as self-sublating
and consequently as posited.
Or the restored unity, in withdrawing into itself,
has repelled itself from itself and
has determined itself;
for its unity has been established through negation
and is, therefore, also negative unity.
It is, therefore, the unity of form and matter,
as the substrate of both, but a substrate which is determinate:
it is formed matter, but matter at the same time
indifferent to form and matter,
indifferent to them because sublated and unessential.
This is content.

c. Form and content

Form stands at first over against essence;
it is then the ground-connection in general,
and its determinations are the ground and the grounded.
It then stands over against matter,
and so it is determining reflection,
and its determinations are the determination of reflection itself
and the subsistence of the latter.
Finally, it stands over against content,
and then its determinations are again itself and matter.
What was previously the self-identical
at first the ground,
then subsistence in general,
and finally matter
now passes under the dominion of form
and is once more one of its determinations.

Content has, first, a form and a matter that
belong to it essentially; it is their unity.
But, because this unity is at the same time determinate
or posited unity, content stands over against form;
the latter constitutes the positedness
and is the unessential over against content.
The latter is therefore indifferent towards form;
form embraces both the form as such as well as the matter,
and content therefore has a form and a matter,
of which it constitutes the substrate
and which are to it mere positedness.

Content is, second, what is identical in form and matter,
so that these would be only indifferent external determinations.
They are positedness in general, but a positedness
that has returned in the content to its unity or its ground.
The identity of the content with itself is,
therefore, in one respect that identity which is indifferent to form,
but in another the identity of ground.
The ground has at first disappeared into content;
but content is at the same time the negative reflection of
the form determinations into themselves;
its unity, at first only the unity indifferent to form, is
therefore also the formal unity or the ground-connection as such.
Content, therefore, has this ground-connection as its essential form,
and, contrariwise, the ground has a content.

The content of the ground is therefore the ground
that has returned into its unity with itself;
the ground is at first the essence that in its
positedness is identical with itself;
as diverse from and indifferent to its positedness,
the ground is indeterminate matter;
but as content it is at the same time informed identity,
and this form becomes for this reason a ground-connection,
since the determinations of its oppositions are posited
in the content also as negated.
Content is further determined within,
not like matter as an indifferent in general,
but like informed matter,
so that the determinations of form have
a material, indifferent subsistence.
On the one hand, content is the essential self-identity
of the ground in its positedness;
on the other hand, it is posited identity
as against the ground-connection;
this positedness, which is in this identity as determination of form,
stands over against the free positedness, that is to say,
over against the form as the whole connection of ground and grounded.
This form is the total positedness returning into itself;
the other form, therefore, is only the positedness as immediate,
the determinateness as such.
The ground has thus made itself into a determinate ground in general,
and the determinateness is itself twofold:
of form first, and of content second.
The former is its determinateness of being external to the content as such,
the content that remains indifferent to this external reference.
The latter is the determinateness of the content that the ground has.

IV.8
jati-desa-kala vyavahitanam api-anantaryam smrti-samskarayo eka-rupatvat

IV.9
tasam anaditvam ca-asisa nityatvat

B. DETERMINATE GROUND

a. Formal ground

The ground has a determinate content.
For the form, as we have seen,
the determinateness of content is the substrate,
the simple immediate as against the mediation of form.
The ground is negatively self-referring identity which,
for this reason, makes itself into a positedness;
it negatively refers to itself because in its negativity
it is identical with itself;
this identity is the substrate or the content
which thus constitutes the indifferent
or positive unity of the ground-connection
and, in this connection, is the mediating factor.

In this content, the determinateness that
the ground and the grounded have over
against one another has at first disappeared.
The mediation, however, is also negative unity.
The negative implicit in that indifferent substrate is
this substrate's immediate determinateness through which
the ground has a determinate content.
But then, the negative is the negative reference of form to itself.
What has been posited sublates itself on its side
and returns to its ground;
the ground, however, the essential self-subsistence,
refers negatively to itself and makes itself into a positedness.
This negative mediation of ground and grounded is
the mediation that belongs to form as such, formal mediation.
Now both sides of form, because each passes over into the other,
thereby mutually posit themselves into one identity as sublated;
in this, they presuppose the identity.
The latter is the determinate content
to which the formal mediation thus refers itself
through itself as to the positive mediating factor.
That content is the identical element of both,
and because the two are distinct,
yet in their distinction each is
the reference to the other,
it is their subsistence,
the subsistence of each as the whole itself.

Accordingly, the result is that
in the determinate ground we have the following.
First, a determinate content is considered from two sides,
once in so far as it is ground,
then again in so far as it is grounded.
The content itself is indifferent to these forms;
it is in each simply and solely one determination.
Second, the ground is itself just as much a moment of form
as what is posited by it; this is its identity according to form.
It is a matter of indifference which of
the two determinations is made the first,
whether the transition is
from the one as posited to the other as ground
or from the one as ground to the other as posited.
The grounded, considered for itself, is the sublating of itself;
it thereby makes itself on the one side into a posited,
and is at the same time the positing of the ground.
The same movement is the ground as such;
it makes itself into something posited,
and thereby becomes the ground of something,
that is to say, is present therein
both as a posited and also first as ground.
That there be a ground, of that the posited is the ground,
and, conversely, the ground is thereby the posited.
The mediation begins just as much from the one as from the other;
each side is just as much ground as posited,
and each is the whole mediation or the whole form.
Further, this whole form is itself, as self-identical,
the substrate of the two determinations
that constitute the two sides of the ground and the grounded;
form and content are thus themselves one and the same identity.

Because of this identity of the ground and the grounded,
according both to content and form,
the ground is sufficient
(the sufficiency being limited to this relation);
there is nothing in the grounded which is not in the ground.
Whenever one asks for a ground,
one expects to see the same determination
which is the content doubled,
once in the form of that which is posited,
and again in the form of existence
reflected into itself, of essentiality.

Now inasmuch as in the determined ground,
the ground and the grounded are each the whole form, and their content,
though determinate, is nevertheless one and the same,
the two sides of the ground do not as yet have a real determination,
do not have a different content;
the determinateness is only one simple determinateness
that has yet to pass over into the two sides;
the determinate ground is present
only in its pure form, as formal ground.
Because the content is only this simple determinateness,
one that does not have in it the form of the ground-connection,
the determinateness is a self-identical content indifferent to form,
and the form is external to it;
the content is other than the form.

b. Real ground

The determinateness of ground is, as we have seen,
on the one hand determinateness of the substrate
or content determination;
on the other hand,
it is the otherness in the ground-connection itself,
namely the distinctness of its content and the form;
the connection of ground and grounded strays
in the content as an external form,
and the content is indifferent to these determinations.
But in fact the two are not external to each other;
for this is what the content is:
to be the identity of the ground
with itself in the grounded,
and of the grounded in the ground.
The side of the ground
has shown itself to be itself a posited,
and the side of the grounded to be
itself ground;
each side is this identity of the whole within it.
But since they equally belong to form
and constitute its determinate difference,
each is in its determinateness the identity of the whole with itself.
Consequently, each has a diverse content as against the other.
Or, considering the matter from the side of the content,
since the latter is the self-identity of the ground-connection,
it essentially possesses this difference of form within,
and is as ground something other than what it is as grounded.

Now the moment ground and grounded have a diverse content,
the ground-connection has ceased to be a formal one;
the turning back to the ground and
the procession forward from ground to posited
is no longer a tautology; the ground is realized.
Henceforth, whenever we ask for a ground,
we actually demand another content determination for it
than the determination of the content whose ground we are asking for.

This connection now determines itself further.
For inasmuch as its two sides are of different content,
they are indifferent to each other;
each is an immediate, self-identical determination.
Moreover, as referred to each other as ground and grounded,
the ground reflects itself in the other,
as in something posited by it, back to itself;
the content on the side of the ground,
therefore, is equally in the grounded;
the latter, as the posited, has its
self-identity and subsistence only in the ground.
But besides this content of the ground,
the grounded also now possesses a content of its own
and is accordingly the unity of a twofold content.
Now this unity, as the unity of sides that are different,
is indeed their negative unity;
but since the two determinations of content are indifferent to each other,
that unity is only their empty reference to each other,
in itself void of content, and not their mediation;
it is a one or a something externally holding them together.

In the real grounding connection
there is present, therefore, a twofold.
For one thing, the content determination which is ground
extends continuously into the positedness,
so that it constitutes the simple identity
of the ground and the grounded;
the grounded thus contains the ground
fully within itself;
their connection is one of
undifferentiated essential compactness.
Anything else in the grounded
added to this simple essence is,
therefore, only an unessential form,
external determinations of the content
that, as such, are free from the ground
and constitute an immediate manifold.
Of this unessential more, therefore,
the essential is not the ground,
nor is it the ground of any connection
between it and the unessential in the grounded.
The unessential is a positively identical element
that resides in the grounded but does not posit itself
there in any distinctive form;
as self-referring content, it is rather
an indifferent positive substrate.
For another thing, that which in the something is
linked with this substrate is an indifferent content,
but as the unessential side.
The main thing is the connection of the substrate
and the unessential manifold.
But this connection, since the determinations
that it connects are an indifferent content,
is also not a ground;
true, one determination is determined as essential content
and the other as only unessential or as posited;
but this form is to each, as a self-referring content, an external one.
The one of the something that constitutes their connection is
for this reason not a reference of form,
but only an external tie that does not hold
the unessential manifold content as posited;
it too is therefore likewise only a substrate.

Ground, in determining itself as real,
because of the diversity of the content
that constitutes its reality,
thus breaks down into external determinations.
The two connections of the essential reality content,
as the simple immediate identity of ground and grounded;
and then the something connecting distinct contents
are two different substrates.
The self-identical form of ground,
according to which one and the same thing
is at one time the essential
and at another the posited, has vanished.
The ground-connection has thus become external to itself.

Consequently, it is an external ground that now
holds together a diversified content
and determines what is ground and what is posited by it;
this determination is not to be found in the two-sided content itself.
The real ground is therefore the reference to another,
on the one hand, of a content to another content
and, on the other, of the ground-connection itself
(the form) to another, namely to an immediate,
to something not posited by it.

c. Complete ground

1. In real ground, ground as content
and ground as connection are only substrates.
The former is only posited as essential and as ground;
the connection is what the grounded immediately is
as the indeterminate substrate of a diversified content,
a linking of this content which is not the content's own reflection
but is rather external and consequently a reflection which is only posited.
The real ground-connection is ground, therefore, rather as sublated;
consequently, it rather makes up the side of
the grounded or of the positedness.
As positedness, however, the ground itself
has now returned to its ground;
it is now something grounded: it has another ground.
This ground will therefore be so determined that,
first, it is identical with the ground by which it is grounded;
both sides have in this determination one and the same content;
the two content determinations and their linkage in
a something are equally to be found in the new ground.
But, second, the new ground into which
the previously merely posited and external link
is now sublated is the immanent reflection of this link:
the absolute reference of the two content determinations to each other.

Because real ground has itself thus returned to its ground,
the identity of ground and grounded
or the formality of ground reasserts itself in it.
The newly arisen ground-connection is
therefore the one which is complete,
which contains the formal and real ground in itself
at the same time and mediates the content determinations
which in the real ground confronted each other immediately.

2. Thus the ground-connection has more precisely
determined itself as follows.

First, something has a ground;
it contains the content determination which is the ground
and, in addition, a second determination as posited by the ground.
But, because of the indifference of content,
the one determination is not ground in itself,
nor is the other in itself one that is grounded by the first;
this connection of ground and grounded is rather
sublated in the immediacy of their content, is posited,
and as such has its ground in another such connection.

Since this second connection is
distinguished only according to form,
it has the same content as the first;
it still has the same two determinations of content
but is now their immediate linking together.
This linking, however, is of a general nature,
and the content, therefore, is diversified into determinations
that are indifferent to each other.
The linking is not, therefore, their true absolute connection
that would make one determination the element of
self-identity in the positedness,
and the other determination
the positedness of this same self-identity;
on the contrary, the two are supported by a something
and this something is what connects them,
but in a connection which is not reflected,
is rather only immediate and, therefore,
only a relative ground as against
the linking in the other something.
The two somethings are therefore the two distinct
connections of content that have transpired.
They stand in the identical ground-connection of form;
they are one and the same whole content,
namely the two content determinations and their connection;
they are distinct only by the kind of this connection,
which in the one is an immediate
and in the other a posited connection;
through this, they are distinguished
one from another as ground and grounded only according to form.
Second, this ground-connection is not only formal, but also real.
Formal ground passes over into real ground, as has been shown;
the moments of the form reflect themselves into themselves;
they are a self-subsistent content,
and the ground-connection contains
also one content with the character of ground
and another with that of grounded.
The content constitutes at first the immediate
identity of both sides of the formal ground;
so the two sides have one and the same content.
But the content also has the form in it,
and so it is a twofold content
that behaves as ground and grounded.
One of the two content determinations of
the two somethings is therefore determined,
not merely as being common to them
according to external comparison,
but as their identical substrate
and the foundation of their connection.
As against the other determination of the content,
this determination is essential
and is the ground of the other which is posited,
that is, posited in the something,
the connection of which is the grounded.
In the first something, which is the ground-connection,
this second determination of the content is
also immediately and in itself linked with the first.
But the other something only contains
the one determination in itself as that
in which it is immediately identical with the first something,
but the other as the one which is posited in it.
The former content determination is its
ground by virtue of its being originally linked
in the first something with
the other content determination.

The ground-connection of the content determinations
in the second something is thus mediated
through the connection present in the first something.
The inference is this:
since determination B is implicitly linked
with determination A in a something,
in a second something to which only
the one determination A immediately belongs,
also B is linked with it.
In the second something, not only is
this second determination mediated;
also mediated is that its immediate ground is mediated,
namely by virtue of its original connection
with B in the first something.
This connection is thus the ground of the ground A,
and the whole ground-connection is present in
the second something as posited or grounded.

3. Real ground shows itself to be the self-external reflection of ground;
its complete mediation is the restoration of its identity with itself.
But because this identity has in the process equally acquired
the externality of real ground,
the formal ground-connection in this unity
of itself and real ground is just as much
self-positing as self-sublating ground;
the ground-connection mediates itself with itself through its negation.
The ground is at first, as the original connection,
the connection of immediate content determinations.
The ground-connection, being essential form,
has for sides such that are sublated or are as moments.
Consequently, as the form of immediate determinations,
it connects itself with itself as self-identical
while at the same time connecting with their negation;
accordingly, it is ground not in and for itself
but as connected with the sublated ground-connection.
Second, the sublated connection or the immediate,
which in the original and in the posited connection
is the identical substrate, is likewise real ground
not in and for itself; that it is ground is
rather posited by virtue of that original link.

Thus the ground-connection is in its totality
essentially presupposing reflection;
formal ground presupposes the immediate content determination,
and this content presupposes form as real ground.
Ground is therefore form as an immediate linkage
but in such a manner that it repels itself from itself
and rather presupposes immediacy,
referring itself therein as to another.
This immediate is the content determination, the simple ground;
but as such, that is, as ground, it is equally repelled from itself
and refers itself to itself equally as to an other.
Thus the total ground-connection has taken on
the determination of conditioning mediation.

IV.10
hetu-phala-asraya-alambana sangrhitatvad esam abhave tad-abhava

C. CONDITION

a. The relatively unconditioned

1. Ground is the immediate,
and the grounded the mediated.
But ground is positing reflection;
as such, it makes itself into positedness
and is presupposing reflection;
as such it refers itself to itself as to something sublated,
to an immediate through which it is itself mediated.
This mediation, as an advance
from the immediate to the ground,
is not an external reflection
but, as we have seen, the ground's own doing
or, what is the same, the ground-connection,
as reflection into its self-identity,
is just as essentially self-externalizing reflection.
The immediate to which ground refers as to
its essential presupposition is condition;
real ground is accordingly essentially conditioned.
The determinateness that it contains is the otherness of itself.
Condition is therefore, first, an immediate, manifold existence.
Second, it is this existence referred to an other,
to something which is ground,
not of this existence but in some other respect,
for existence itself is immediate and without ground.
According to this reference, it is something posited;
as condition, the immediate existence is supposed to be
not for itself but for another.
But this, that it thus is for another, is at the same time
itself only a positedness;
that it is posited is sublated in its immediacy:
an existence is indifferent to being a condition.
Third, condition is something immediate in the sense
that it constitutes the presupposition of ground.
In this determination, it is the form-connection of ground
withdrawn into self-identity, hence the content of ground.
But content is as such only the indifferent unity of ground,
as in the form: without form, no content.
It nevertheless frees itself
from this indifferent unity
in that the ground-connection,
in the complete ground,
becomes a connection external to its identity,
whereby content acquires immediacy.
In so far, therefore, as condition is
that in which the ground-connection has
its identity with itself,
it constitutes the content of ground;
but since this content is indifferent to form,
it is only implicitly the content of form,
is something which has yet to become content
and hence constitutes the material for the ground.
Posited as condition, and in accordance with the second moment,
existence is determined to lose its indifferent immediacy
and to become the moment of another.
By virtue of its immediacy, it is indifferent to this connection;
inasmuch as it enters into it, however,
it constitutes the in-itself of the ground
and is for it the unconditioned.
In order to be condition, it has its presupposition in the ground
and is itself conditioned;
but this condition is external to it.

2. Something is not through its condition;
its condition is not its ground.
Condition is for the ground
the moment of unconditioned immediacy,
but is not itself the movement and the positing
that refers itself to itself negatively
and that makes itself into a positedness.
Over against condition there stands,
therefore, the ground-connection.
Something has, besides its condition, also a ground.
This ground is the empty movement of reflection,
for the latter has the immediacy
which is its presupposition outside it.
But it is the whole form
and the self-subsistent process of mediation,
for the condition is not its ground.
Since this mediating refers itself to itself as positing,
it equally is according to this side
something immediate and unconditioned;
it does indeed presuppose itself,
but as an externalized or sublated positing;
whatever it is in accordance with its determination,
that it is, on the contrary, in and for itself.
Inasmuch as the ground-connection is thus
a self-subsisting self-reference and has within it the
identity of reflection,
it has a content which is peculiarly its own as against
the content of the condition.
The one content is that of the ground and is
therefore essentially informed;
the other content, that of the condition,
is on the contrary only an immediate material
whose connecting reference to the ground,
while at the same time constituting the in-itself of the latter,
is also equally external to it;
it is thus a mingling of a self-subsisting content
that has no reference to the content of the ground determination
and of the content that enters into the latter
and, as its material, should become a moment of it.

3. The two sides of the whole,
condition and ground,
are thus, on the one hand,
indifferent and unconditioned
with respect to each other:
the one as the non-referred-to side,
to which the connecting reference
in which it is the condition is external;
the other as the connecting reference, or form,
for which the determinate existence of
the condition is only a material,
something passive whose form,
such as it possesses on its own account,
is unessential.
On the other hand, the two sides are also mediated.
Condition is the in-itself of the ground;
so much is it the essential moment of the ground-connection,
that it is the simple self-identity of the ground.
But this also is sublated;
this in-itself is only something posited;
immediate existence is indifferent to being a condition.
The fact, therefore, that condition is the in-itself
of the ground constitutes the side of it
by which it is a mediated condition.
Likewise, the ground-connection has
in its self-subsistence also a presupposition;
it has its in-itself outside itself.
Consequently, each of the two sides is this contradiction,
that they are indifferent immediacy and essential mediation,
both in one reference
or the contradiction of independent subsistence
and of being determined as only moments.

b. The absolutely unconditioned

At first, each of the two relatively unconditioned sides
reflectively shines in the other;
condition, as an immediate, is reflected
in the form connection of the ground,
and this form in the immediate existence as its positedness;
but each, apart from this reflective shine of its other in it,
stands out on its own and has a content of its own.

Condition is at first immediate existence;
its form has these two moments:
that of positedness, according to which it is, as condition,
material and moment of the ground;
and that of the in-itself, according to which
it constitutes the essentiality of ground
or its simple reflection into itself.
Both sides of the form are external to immediate existence,
for the latter is the sublated ground-connection.

But, first, existence is in it only this:
to sublate itself in its immediacy
and to founder, going to the ground.
Being is as such only the becoming of essence;
it is its essential nature to
make itself into a positedness
and into an identity which is
an immediacy through the negation of itself.
The form determinations of positedness
and of self-identical in-itself,
the form through which immediate existence is condition,
are not, therefore, external to that existence;
the latter is, rather, this very reflection.

Second, as condition, being is now posited as
that which it essentially is,
namely as a moment and consequently as the being of an other,
and at the same time as the in-itself of an other;
it is in itself but only through the negation of itself,
namely through the ground and through its self-sublating
and consequent presupposing reflection;
the in-itself of being is thus only something posited.
This in-itself of the condition has two sides:
one side is its essentiality as essentiality of the ground,
while the other is the immediacy of its existence.
Or rather, both sides are the same thing.
Existence is an immediate, but immediacy
is essentially something mediated,
namely through the self-sublating ground.
Existence, as this immediacy mediated by a self-sublating mediating,
is at the same time the in-itself of the ground and its unconditioned side;
but again, this in-itself is at the same time itself
equally only moment or positedness, since it is mediated.
Condition is, therefore, the whole form of the ground-connection;
it is the presupposed in-itself of the latter,
but, consequently, is itself a positedness
and its immediacy is this, to make itself into a positedness
and thereby to repel itself from itself,
in such as way that it both founders to the ground and is ground,
the ground that makes itself into a positedness
and thereby into a grounded, and both are one and the same.

Likewise in the conditioned ground, the in-itself is not
just as the reflective shining of an other in it.
This ground is the self-subsistent,
that is, self-referring reflection of the positing,
and consequently the self-identical;
or it is in it its in-itself and its content.
But it is at the same time presupposing reflection;
it negatively refers to itself
and posits its in-itself as an other opposite to it,
and condition, according to both its moment of in-itself
and of immediate existence, is
the ground-connection's own moment;
the immediate existence essentially is only through its ground
and is a moment of itself as a presupposing.
This ground, therefore, is equally the whole itself.

What we have here, therefore, is only one whole of form,
but equally so only one whole of content.
For the proper content of condition is essential content
only in so far as it is the self-identity of reflection in the form,
or the ground-connection is in it this immediate existence.
Further, this existence is condition only through
the presupposing reflection of the ground;
it is the ground's self-identity, or its content,
to which the ground posits itself as opposite.
Therefore, the existence is not a merely
formless material for the ground-connection;
on the contrary, because it has this form in it, it is informed matter,
and because in its identity with it it is at the same time
indifferent to it, it is content.
Finally, it is the same content as that possessed by the ground,
for it is precisely content as that which is self-identical
in the form connection.

The two sides of the whole,
condition and ground,
are therefore one essential unity,
as content as well as form.
They pass into one another,
or, since they are reflections,
they posit themselves as sublated,
refer themselves to this their negation,
and reciprocally presuppose each other.
But this is at the same time only one reflection of the two,
and their presupposing is, therefore, one presupposing only;
the reciprocity of this presupposing ultimately amounts to this,
that they both presuppose one identity
for their subsistence and their substrate.
This substrate, the one content and unity of form of both,
is the truly unconditioned; the fact in itself.
Condition is, as it was shown above, only the relatively unconditioned.
It is usual, therefore, to consider it as itself something conditioned
and to ask for a new condition,
whereby the customary progression ad infinitum
from condition to condition is set in motion.
But now, why is it that at one condition
a new condition is asked for, that is,
why is that condition assumed to be something conditioned?
Because it is some finite determinate existence or other.
But this is a further determination of condition
that does not enter into its concept.
Condition is as such conditioned solely because
it is the posited in-itselfness;
it is, therefore, sublated in the absolutely unconditioned.

Now this contains within itself the two sides,
condition and ground, as its moments;
it is the unity to which they have returned.
Together, the two constitute its form or its positedness.
The unconditioned fact is the condition of both,
but the condition which is absolute, that is to say,
one which is itself ground.
As ground, the fact is now the negative identity
that has repelled itself into those two moments:
first, in the shape of the sublated ground-connection,
the shape of an immediate manifold void of
unity and external to itself,
one that refers to the ground as an other to it
and at the same time constitutes its in-itself;
second, in the shape of an inner, simple form which is ground,
but which refers to the self-identical
immediate as to an other, determining it as condition,
that is, determining the in-itself of it as its own moment.
These two sides presuppose the totality,
presuppose that it is that which posits them.
Contrariwise, because they presuppose the totality,
the latter seems to be in turn also conditioned by them,
and the fact to spring forth from its condition and its ground.
But since these two sides have shown themselves to be an identity,
the relation of condition and ground has disappeared;
the two are reduced to a mere reflective shine;
the absolutely unconditioned is in its movement of positing
and presupposing only the movement in which this shine sublates itself.
It is the fact's own doing that it conditions itself
and places itself as ground over against its conditions;
but in connecting conditions and ground,
the fact is a reflection shining in itself;
its relation to them is a rejoining itself.

c. Procession of the fact into concrete existence

The absolutely unconditioned is the absolute ground
that is identical with its condition,
the immediate fact as the truly essential.
As ground, it refers negatively to itself
and makes itself into a positedness;
but this positedness is a reflection
that is complete in both its sides
and is in them the self-identical form of connection,
as has transpired from its concept.
This positedness is therefore first the sublated ground,
the fact as an immediacy void of reflection,
the side of the conditions.
This is the totality of the determinations of the fact,
the fact itself, but the fact as thrown into
the externality of being, the restored circle of being.
In condition, essence lets go of the unity of its immanent reflection;
but it lets it go as an immediacy that now carries
the character of being a conditioning presupposition
and of essentially constituting only one of its sides.
For this reason the conditions are the whole content of the fact,
because they are the unconditioned in the form of formless being.
But because of this form, they also have yet another shape besides
the conditions of the content as this is in the fact as such.
They appear as a manifold without unity,
mingled with extra-essential elements
and other circumstances that do not belong
to the circle of existence as constituting
the conditions of this determinate fact.
For the absolute, unrestricted fact,
the sphere of being itself is the condition.
The ground, returning into itself, posits
that sphere as the first immediacy
to which it refers as to its unconditioned.
This immediacy, as sublated reflection,
is reflection in the element of being,
which thus forms itself as such into a whole;
form proliferates as determinateness of being
and thus appears as a manifold distinct from
the determination of reflection
and as a content indifferent to it.
The unessential, which is in the sphere of being
but which the latter sheds in so far as it is condition,
is the determinateness of the immediacy into which
the unity of form has sunk.
This unity of form, as the connection of being,
is in the latter at first as becoming the passing over
of a determinateness of being into another.
But the becoming of being is also the coming to be
of essence and a return to the ground.
The existence that constitutes the conditions, therefore,
is in truth not determined as condition by an other
and is not used by it as material;
on the contrary, it itself makes itself, through itself,
into the moment of an other.
Further, the becoming of this existence
does not start off from itself
as if it were truly the first and immediate;
on the contrary, its immediacy is
something only presupposed, and the movement of
its becoming is the doing of reflection itself.
The truth of existence is thus that it is condition;
its immediacy is solely by virtue of
the reflection of the ground-connection
that posits itself as sublated.
Consequently, like immediacy, becoming is only
the reflective shine of the unconditioned
inasmuch as this presupposes itself
and has its form in this presupposing,
and hence the immediacy of being is essentially
only a moment of the form.

The other side of this reflective shining of
the unconditioned is the ground-connection as such,
determined as form as against the immediacy
of the conditions and the content.
But this side is the form of the absolute fact
that possesses the unity of its form with itself
or its content within it,
and, in determining this content as condition,
in this very positing sublates the diversity of the content
and reduces it to a moment;
just as, contrariwise, as a form void of essence,
in this self-identity it gives itself the immediacy of subsistence.
The reflection of the ground sublates the immediacy of the conditions,
connecting them and making them moments within the unity of the fact;
but the conditions are that which
the unconditioned fact itself presupposes
and the latter, therefore, sublates its own positing;
consequently, its positing converts itself
just as immediately into a becoming.
The two, therefore, are one unity;
the internal movement of the conditions is a becoming,
the return into the ground and the positing of the ground;
but the ground as posited, and this means as sublated, is the immediate.
The ground refers negatively to itself,
makes itself into a positedness and grounds the conditions;
in this, however, in that the immediate existence is
thus determined as a positedness,
the ground sublates it and only then makes itself into a ground.
This reflection is therefore the self-mediation of
the unconditioned fact through its negation.
Or rather, the reflection of the unconditioned is at first a presupposing,
but this sublating of itself is immediately a positing which determines;
secondly, in this positing the reflection is immediately
the sublating of the presupposed
and a determining from within itself;
this determining is thus in turn the sublating of the positing:
it is a becoming within itself.
In this, the mediation as a turning back
to itself through negation has disappeared;
mediation is simple reflection
reflectively shining within itself
and groundless, absolute becoming.
The fact's movement of being posited,
on the one hand through its conditions,
and on the other hand through its ground,
now is the disappearing of the reflective shine of mediation.
The process by which the fact is posited is accordingly a coming forth,
the simple self-staging of the fact in concrete existence,
the pure movement of the fact to itself.

When all the conditions of a fact are at hand,
the fact steps into concrete existence.
The fact is, before it exists concretely;
it is, first, as essence or as unconditioned;
second, it has immediate existence or is determined,
and this in the twofold manner just considered,
on the one hand in its conditions
and on the other in its ground.
In the former case, it has given itself the form
of the external, groundless being,
for as absolute reflection the fact is
negative self-reference and makes itself into its presupposition.
This presupposed unconditioned is, therefore, the groundless immediate
whose being is just to be there, without grounds.
If, therefore, all the conditions of the fact are at hand,
that is, if the totality of the fact is posited as a groundless immediate,
then this scattered manifold internally recollects itself.
The whole fact must be there, within its conditions,
or all the conditions belong to its concrete existence;
for the all of them constitutes the reflection of the fact.
Or again, immediate existence, since it is condition, is determined by form;
its determinations are therefore determinations of reflection
and with the positing of one the rest also are essentially posited.
The recollecting of the conditions is at first
the foundering to the ground of immediate existence
and the coming to be of the ground.
But the ground is thereby a posited ground, that is,
to the extent that it is ground,
to that extent it is sublated as ground
and is immediate being.
If, therefore, all the conditions of the fact are at hand,
they sublate themselves as immediate existence and as presupposition,
and the ground is equally sublated.
The latter proves to be only a reflective shine
that immediately disappears;
this coming forth is thus the tautological movement
of the fact to itself:
its mediation through the conditions and through the ground
is the disappearing of both of these.
The coming forth into concrete existence is therefore so immediate,
that it is mediated only by the disappearing of the mediation.

The fact proceeds from the ground.
It is not grounded or posited by it
in such a manner that the ground
would still stay underneath, as a substrate;
on the contrary, the positing is
the outward movement of ground to itself
and the simple disappearing of it.
Through its union with the conditions,
it obtains the external immediacy and the moment of being.
But it does not obtain them as a something external,
nor by referring to them externally;
rather, as ground it makes itself into a positedness;
its simple essentiality rejoins itself in the positedness
and, in this sublating of itself,
it is the disappearing of its difference from its positedness,
and is thus simple essential immediacy.
It does not, therefore, linger on
as something distinct from the grounded;
on the contrary, the truth of the grounding is
that in grounding the ground unites with itself,
and its reflection into another is
consequently its reflection into itself.
The fact is thus the unconditioned
and, as such, equally so the groundless;
it arises from the ground only in so far as
the latter has foundered and is no longer ground:
it rises up from the groundless, that is,
from its own essential negativity or pure form.

This immediacy, mediated by ground and condition
and self-identical through the sublating of mediation,
is concrete existence.

SECTION II

Appearance

Essence must appear.

Being is the absolute abstraction;
this negativity is not something external to it,
but being is rather being,
and nothing but being,
only as this absolute negativity.
Because of this negativity,
being is only as self-sublating being
and is essence.
But, conversely, essence as simple self-equality
is likewise being.
The doctrine of being contains the first proposition,
“being is essence.”
The second proposition, “essence is being,”
constitutes the content of the first section
of the doctrine of essence.
But this being into which
essence makes itself
is essential being,
concrete existence,
a being which has come forth
out of negativity and inwardness.

Thus essence appears.
Reflection is the internal shining of essence.
The determinations of this reflection are included
in the unity purely and simply as posited, sublated;
or reflection is essence immediately identical
with itself in its positedness.
But since this essence is ground,
through its self-sublating reflection,
or the reflection that which returns into itself,
essence determines itself as something real;
further, since this real determination, or the otherness,
of the ground-connection sublates itself
in the reflection of the ground
and becomes concrete existence,
the form determinations acquire therein
an element of independent subsistence.
Their reflective shine comes to completion in appearance.

The essentiality that has advanced to immediacy is,

first, concrete existence,
and a concrete existent or thing,
an undifferentiated unity of essence and its immediacy.
The thing indeed contains reflection,
but its negativity is at first dissolved in its immediacy;
but, because its ground is essentially reflection,
its immediacy is sublated
and the thing makes itself into a positedness.

Second, then, it is appearance.
Appearance is what the thing is in itself,
or the truth of it.
But this concrete existence,
only posited and reflected into otherness,
is equally the surpassing of itself into its infinity;
opposed to the world of appearance there stands
the world that exists in itself reflected into itself.
But the being that appears and essential being
stand referred to each other absolutely.

Thus concrete existence is, third, essential relation;
what appears shows the essential,
and the essential is in its appearance.
Relation is the still incomplete union of
reflection into otherness and reflection into itself;
the complete interpenetrating of the two is actuality.

CHAPTER 1

Concrete existence

IV.11
atita-anagatam svarupato 'styadhva-bhedad dharmanam

IV.12
te vyakta-suksma guna-atmana

Just as the principle of sufficient reason says
that whatever is has a ground,
or is something posited, something mediated,
so there would also have to be
a principle of concrete existence saying
that whatever is, exists concretely.
The truth of being is to be,
not an immediate something,
but essence that has
come forth into immediacy.

But when it was further said
that whatever exists concretely
has a ground and is conditioned,
it also would have had to be said
that it has no ground and is unconditioned.
For concrete existence is the immediacy
that has come forth from the sublating
of the mediation that results
from the connection of ground and condition,
and which, in coming forth,
sublates this very coming forth.

Inasmuch as mention may be made here of
the proofs of the concrete existence of God,
it is first to be noted that besides
immediate being that comes first,
and concrete existence
(or the being that proceeds from essence)
that comes second, there is still a third being,
one that proceeds from the concept,
and this is objectivity.
Proof is, in general, mediated cognition.
The various kinds of being require or contain
each its own kind of mediation,
and so will the nature of the proof also vary accordingly.
The ontological proof wants to start from the concept;
it lays down as its basis the sum total of all realities,
where under reality also concrete existence is subsumed.
Its mediation, therefore, is that of the syllogism,
and syllogism is not yet under consideration here.
We have already commented above (Part 1, Section 1)
on Kant's objection to the ontological proof,
and have remarked that by concrete existence
Kant understands the determinate immediate existence
with which something enters into the context of total experience,
that is, into the determination of being an other
and of being in reference to an other.
As an existent concrete in this way,
something is thus mediated by an other,
and concrete existence is in general the side of its mediation.
But in what Kant calls the concept, namely,
something taken as only simply self-referring,
or in representation as such, this mediation is missing;
in abstract self-identity, opposition is left out.
Now the ontological proof would have
to demonstrate that the absolute concept,
namely the concept of God,
attains to a determinate existence, to mediation,
or to demonstrate how simple essence
mediates itself with mediation.
This is done by the just mentioned
subsumption of concrete existence
under its universal, namely reality,
which is assumed as the middle term
between God in his concept, on the one hand,
and concrete existence, on the other.
This mediation, inasmuch as it has the form of a syllogism,
is not at issue here, as already said.
However, how that mediation of
essence and concrete existence truly comes about,
this is contained in the preceding exposition.
The nature of the proof itself will be considered
in the doctrine of cognition.
Here we have only to indicate what pertains
to the nature of mediation in general.

The proofs of the existence of God
adduce a ground for this existence.
It is not supposed to be an objective
ground of the existence of God,
for this existence is in and for itself.
It is, therefore, solely a ground for cognition.
It thereby presents itself as a ground
that vanishes in the subject matter
that at first seems to be grounded by it.
Now the ground which is derived
from the contingency of the world
entails the regress of the latter
into the absolute essence,
for the accidental is that which is
in itself groundless and self-sublating.
In this way, therefore, the absolute essence
does indeed proceed from that which has no ground,
for the ground sublates itself
and with this there also vanishes
the reflective shine of the relation
that was given to God, that it is grounded in an other.
This mediation is therefore true mediation.
But the reflection involved in that proof does not know
the nature of the mediation that it performs.
On the one hand, it takes itself to be something merely subjective,
and it consequently distances its mediation from God himself;
on the other hand, for that same reason it
also fails to recognize its mediating movement,
that this movement is in the essence itself and how it is there.
The true relation of reflection consists
in being both in one:
mediation as such but, of course, at the same time
a subjective, external mediation,
that is to say, a self-external mediation
which in turn internally sublates itself.
In that other presentation, however,
concrete existence is given the false relation
of appearing only as mediated or posited.

So, on the other side, concrete existence also
cannot be regarded merely as an immediate.
Taken in the determination of an immediacy,
the comprehension of God's concrete existence
has been declared to be beyond proof
and the knowledge of it
an immediate consciousness only, a faith.
Knowledge should arrive at
the conclusion that it knows nothing,
and this means that it gives up its mediating movement
and the determinations themselves
that have come up in the course of it.
This is what has also occurred in the foregoing;
but it must be added that reflection,
by ending up with the sublation of itself,
does not thereby have nothing for result,
so that the positive knowledge of the essence
would then be an immediate reference to it,
divorced from that result and self-originating,
an act that starts only from itself;
on the contrary, the end itself,
the foundering of the mediation,
is at the same time the ground
from which the immediate proceeds.
In “zu Grunde gehen,” the German language unites,
as we remarked above,
the meaning of foundering and of ground;
the essence of God is said to be the
abyss (Abgrund in German) for finite reason.
This it is, indeed, in so far as
reason surrenders its finitude therein,
and sinks its mediating movement;
but this abyss, the negative ground, is at the same time
the positive ground of the emergence of the existent,
of the essence immediate in itself;
mediation is an essential moment.
Mediation through ground sublates itself but
does not leave the ground standing under it,
so that what proceeds from it would be a posited
that has its essence elsewhere;
on the contrary, this ground is, as an abyss,
the vanished mediation, and, conversely, only the
vanished mediation is at the same time the ground
and, only through this negation,
the self-equal and immediate.

Concrete existence, then, is not to be taken here
as a predicate, or as a determination of essence,
of which it could be said in a proposition,
“essence exists concretely,” or “it has concrete existence.”
On the contrary, essence has passed over into concrete existence;
concrete existence is the absolute self-emptying of essence,
an emptying that leaves nothing of the essence behind.
The proposition should therefore run:
“Essence is concrete existence;
it is not distinct from its concrete existence.”
Essence has passed over into concrete existence
inasmuch as essence as ground
no longer distinguishes itself from itself as grounded,
or inasmuch as the ground has sublated itself.
But this negation is no less essentially its position,
or the simply positive continuity with itself;
concrete existence is the reflection of the ground into itself,
its self-identity as attained in its negation,
therefore the mediation that has posited itself
as identical with itself and through that is immediacy.

Now because concrete existence is
essentially self-identical mediation,
it has the determinations of mediation in it,
but in such a way that the determinations are
at the same time reflected into themselves
and have essential and immediate subsistence.
As an immediacy which is posited through sublation,
concrete existence is negative unity and being-within-itself;
it therefore immediately determines itself
as a concrete existent and as thing.

IV.13
parinama-ekatvad vastu-tattvam

A. THE THING AND ITS PROPERTIES

Concrete existence as a concrete existent is posited in
the form of the negative unity which it essentially is.
But this negative unity is at first only immediate determination,
hence the oneness of the something in general.
But the concretely existent something is different
from the something that exists immediately.
The former is essentially an immediacy that has arisen
through the reflection of mediation into itself.
The concretely existent something is thus a thing.

The thing is distinct from its concrete existence
just as the something can be distinguished from its being.
The thing and the concrete existent
are immediately one and the same.
But because concrete existence is not
the first immediacy of being
but has the moment of mediation within it,
its further determination as thing
and the distinguishing of the two is
not a transition but truly an analysis.
Concrete existence as such contains
this very distinction in the moment of its mediation:
the distinction of thing-in-itself
and external concrete existence.

a. The thing in itself and concrete existence

1. The thing in itself is the concrete existent
as the essential immediate that has resulted
from the sublated mediation.
Mediation is therefore equally essential to it;
but this distinction in this first or immediate concrete
existence falls apart into indifferent determinations.
The one side, namely the mediation of the thing,
is its non-reflected immediacy,
and hence its being in general;
and this being, since it is at the same time determined as mediation,
is an existence which is other to itself,
manifold and external within itself.
But it is not just immediate existence;
it also refers to the sublated mediation
and the essential immediacy;
it is therefore immediate existence
as unessential, as positedness.
(When the thing is differentiated from its concrete existence,
it is then the possible, the thing of representation,
or the thing of thought,
which as such is at the same time not supposed to exist.
However, the determination of possibility
and of the opposition of the thing
and its concrete existence comes later.)
But the thing-in-itself and its mediated being are
both contained in the concrete existence,
and both are themselves concrete existences;
the thing-in-itself exists concretely
and is the essential concrete existence,
but the mediated being is
the thing's unessential concrete existence.

The thing in itself, as the simple reflectedness of
the concrete existence within itself,
is not the ground of unessential existence;
it is the unmoved, indeterminate unity,
for it has precisely the determination of
being the sublated mediation,
and is therefore the substrate of that existence.
For this reason reflection, too,
as an immediate existence
which is mediated through some other,
falls outside the thing-in-itself.
The latter is not supposed to have
any determinate manifold in it;
for this reason it obtains it only
when exposed to external reflection,
though it remains indifferent to it.
(The thing-in-itself has color only when exposed to the eye,
smell when exposed to the nose, and so on.)
Its diversity consists of aspects which an other picks out,
specific points of reference which this other assumes
with respect to the thing-in-itself
and which are not the thing's own determinations.

2. Now this other is reflection
which, determined as external, is,
first, external to itself and determinate manifoldness.
Second, it is external to the essential concrete existent
and refers to it as to its absolute presupposition.
These two moments of external reflection,
its own manifoldness and its reference to
the thing-in-itself as its other,
are however one and the same.
For this concrete existence is
external only in so far as it refers to
the essential identity as to an other.
The manifoldness, therefore, does not have
an independent  subsistence of its own
besides the thing-in-itself
but, over against it,
it is rather only as reflective shine;
in its necessary reference to it,
it is like a reflex refracting itself in it.
Diversity, therefore, is present as the reference
of an other to the thing-in-itself;
but this other is nothing that subsists on its own
but is only as reference to the thing-in-itself;
but at the same time it only is in being repelled from it;
thus it is the unsupported rebound of itself within itself.

Now since the thing-in-itself is
the essential identity of the concrete existence,
this essenceless reflection does not accrue to it
but collapses within itself externally to it.
It founders to the ground
and thus itself comes to be essential identity
or thing-in-itself.
This can also be looked at in this way:
the essenceless concrete existence has
in the thing-in-itself its reflection into itself;
it refers to it in the first place as to its other;
but as the other over against that which is in itself,
it is only the sublation of its self,
and its coming to be in the in-itself.
The thing-in-itself is thus identical
with external concrete existence.

This is exhibited in the thing-in-itself as follows.
The thing-in-itself is self-referring
essential concrete existence;
it is self-identity only in so far as
it holds negativity's reflection in itself;
that which appeared as concrete existence
external to it is, consequently, a moment in it.
It is for this reason also self-repelling thing-in-itself
which thus relates itself to itself as to an other.
Hence, there are now a plurality of things-in-themselves
standing in the reciprocal reference of external reflection.
This unessential concrete existence is
their reciprocal relation as others;
but it is, further, also essential to them
or, in other words, this unessential concrete existence,
in collapsing internally, is thing-in-itself,
but a thing-in-itself which is other than the first,
for that first is immediate essentiality
whereas the present proceeds from the unessential concrete existence.
But this other thing-in-itself is only an other in general;
for, as self-identical thing, it has no
further determinateness vis-à-vis the first;
like the first, it is the reflection within itself
of the unessential concrete existence.
The determinateness of the various things-in-themselves
over against one another falls therefore into external reflection.

3. This external reflection is henceforth a relating of
the things-in-themselves to one another,
their reciprocal mediation as others.
The things-in-themselves are thus
the extreme terms of a syllogism,
the middle term of which is made up
by their external concrete existence,
the concrete existence by virtue of which
they are other to each other and distinct.
This, their difference, falls only
in their connecting reference;
they send determinations, as it were,
from their surface into the reference,
while remaining themselves indifferent to it.
This relation now constitutes
the totality of the concrete existence.
The thing-in-itself is drawn into a reflection external to it
in which it has a manifold of determinations;
this is the repelling of itself from itself
into another thing-in-itself,
a repelling which is its rebounding back into itself,
for each thing-in-itself is an other
only as reflected back from the other;
it has its supposition not in itself but in the other,
is determined only through the determinateness of the other;
this other is equally determined only
through the determinateness of the first.
But the two things-in-themselves,
since each has its difference
not in it but in the other,
are not therefore distinct things;
the thing-in-itself, in relating as it should to
the other extreme as to another thing-in-itself,
relates to it as to something non-distinguished from it,
and the external reflection that should constitute
the mediating reference between the extremes is a
relation of the thing-in-itself only to itself,
or is essentially its reflection within itself;
the reflection is, therefore, determinateness existing in itself,
or the determinateness of the thing-in-itself.
The latter, therefore, does not have this determinateness
in a reference, external to it,
to another thing-in-itself,
and of this other to it;
the determinateness is not just its surface
but is rather the essential mediation of
itself with itself as with an other.
The two things-in-themselves that should
constitute the extremes of the reference,
since they are supposed not
to have any contrasting determinateness,
collapse in fact into one;
it is only one thing-in-itself that
relates itself to itself in the external reflection,
and it is its own reference to itself as to another
that constitutes its determinateness.

This determinateness of the thing-in-itself is
the property of the thing.

b. Property

Quality is the immediate determinateness of something;
the negative itself by virtue of which being is something.
The property of the thing is, for its part,
the negativity of reflection,
by virtue of which concrete existence in
general is a concrete existent
and, as simple self-identity, is thing-in-itself.
But the negativity of reflection, the sublated mediation,
is itself essentially mediation and reference,
though not to an other in general like quality
which is not reflected determinateness;
it is rather reference to itself as to an other,
or mediation which immediately is no less self-identity.
The abstract thing-in-itself is itself this relation
which turns from another back to itself;
it is thereby determined in itself;
but its determinateness is constitution,
which is as such itself determination,
and in relating to the other
it does not pass over into otherness
and  is excluded from alteration.

A thing has properties;

these are, first, its determinate references to something other;
the property is there only as a way of reciprocal relating;
it is, therefore, the external reflection of the thing
and the side of its positedness.

But, second, in this positedness the thing is in itself;
it maintains itself in its reference to the other
and thus is admittedly only a surface
where the concrete existence is exposed to
the becoming of being and to alteration;
the property is not lost in this.
A thing has the property to effect this or that in an other,
and in this connection to express itself in some characteristic way.
It demonstrates this property only under the condition
that another thing has a corresponding constitution,
but at the same time the property is characteristically
the thing's own and its self-identical substrate;
for this reason this reflected quality is called property.
The thing thereby passes over into an externality,
but the property maintains itself in this transition.
Through its properties the thing becomes cause,
and to be a cause is this, to preserve itself as effect.
However, the thing is here still the static thing of many properties;
it is not yet determined as actual cause;
it is so far only the reflection of
its determinations immediately existing in itself,
not yet itself the reflection that posits them.

Essentially, therefore, the thing-in-itself has
just shown itself to be thing-in-itself
not only in such a way that its properties are
the positedness of an external reflection;
on the contrary, those properties are its own determinations
by virtue of which it relates in some determinate manner;
it is not an indeterminate substrate located on
the other side of its external concrete existence
but is present in its properties rather as ground,
that is to say, it is self-identity in its positedness;
but, at the same time, it is conditioned ground,
that is to say, its positedness is
equally reflection external to itself;
it is reflected into itself and in itself only to
the extent that it is external.
Through concrete existence the thing-in-itself
enters into external references,
and the concrete existence consists
precisely in this externality;
it is the immediacy of being
and because of that the thing is
subjected to alteration;
but it is also the reflected immediacy of the ground,
hence the thing in itself in its alteration.
This mention of the ground-connection is
not however to be taken here as if
the thing in general were determined
as the ground of its properties;
thinghood itself is, as such, the ground-connection;
the property is not distinguished from its ground,
nor does it constitute just the positedness
but is rather the ground that has
passed over into its externality
and is consequently truly reflected into itself;
the property is itself, as such,
the ground, implicitly existent positedness;
it is the ground, in other words,
that constitutes the form of the property's identity,
and the property's determinateness is
the self-external reflection of the ground;
the whole is the ground which in its repelling and determining,
in its external immediacy, refers itself to itself.
The thing-in-itself thus concretely exists essentially,
and that it concretely exists essentially means,
conversely, that concrete existence, as external immediacy,
is at the same time in-itselfness.

c. The reciprocal action of things

The thing-in-itself exists in concreto by essence;
external immediacy and determinateness
belong to its being-in-itself,
or to its immanent reflection.
The thing in-itself is thus a thing that has properties,
and hence there are a number of things distinct from one another,
not because of some viewpoint alien to them but through themselves.
These many diverse things stand in
essential reciprocal action by virtue of their properties;
the property is this reciprocal connecting reference itself,
apart from which the thing is nothing;
the reciprocal determination, the middle term of
the things-in-themselves that are taken as extreme terms
indifferent to the reference connecting them,
is itself the self-identical reflection and the thing-in-itself
which those extremes were supposed to be.
Thinghood is thus reduced to the form of
indeterminate self-identity having its essentiality only in its property.
Thus, if one speaks of a thing or of things in general
without a determinate property, then their difference is
merely indifferent, quantitative.
What is considered as a thing can just as well be made into
a plurality of things or be considered as a plurality of things;
their separation or their union is an external one.
A book is a thing, and each of its pages is also a thing,
and equally so every tiny piece of its pages,
and so on to infinity.
The determinateness, in virtue of which
a thing is this thing only,
lies solely in its properties.
It is through them that the thing
differentiates itself from other things,
for the property is the negative reflection and the differentiating;
only in its property, therefore, does the thing possess
in it the difference of itself from others.
This is the difference reflected into itself,
by virtue of which the thing, in its positedness,
that is, in its reference to others, is equally indifferent
to the other and to its reference to it.
Without its properties, therefore, there is nothing that remains
to the thing except the unessential compass
and the external gathering of an abstract in-itselfness.
With this, thinghood has passed over into property.

The thing, as the extreme term that exists in itself,
was supposed to relate to the property,
and this property to constitute the middle term
between things that stand connected.
But this connection is where the things meet
as self-repelling reflection,
where they are distinguished and connected.
This, their distinction and their connecting reference,
is one reflection and one continuity of both.
Accordingly, the things themselves fall only
within this continuity which is the property;
they vanish as would-be self-subsisting extremes
that would have a concrete existence outside this property.

The property, which was supposed to connect
the self-subsisting extremes,
is therefore itself self-subsistent.
The things are, on the contrary, the unessential.
They are something essential only as
the self-differentiating and self-referring reflection;
but this is the property.
The latter is in the thing,
therefore, not as something sublated, not just a moment of it;
on the contrary, the truth of the thing is
that it is only an unessential compass
which is indeed a negative unity,
but only like the one of the something,
that is to say, a one which is immediate.
Whereas earlier the thing was determined as
an unessential compass because it was made such
by an external abstraction that omits the property,
this abstraction now happens through the transition of
the thing-in-itself into the property itself.
But there is now an inversion of values,
for the earlier abstraction still envisaged the abstract thing
without its property as being the essential,
and the property as an external determination,
whereas it is the thing as such which is now reduced,
through itself, to the determination of
an indifferent external form of the property.
The latter is henceforth thus freed of the indeterminate
and impotent bond which is the unity of the thing;
the property is what constitutes the subsistence of the thing;
it is a self-subsisting matter.
Since this matter is simple continuity with itself,
it only possesses at first the form of diversity.
There is, therefore, a manifold of
these self-subsisting matters,
and the thing consists of them.

IV.14
vastu-samye citta-bhedat tayo vibhakta pantha

B. THE CONSTITUTION OF THE THING OUT OF MATTERS

The transition of property into a matter
or into a self-subsistent stuff is the familiar
transition performed on sensible matter by chemistry
when it seeks to represent the properties of color, smell, etc.,
as luminous matter, coloring matter, odorific matter,
sour, bitter matter and so on;
or when it simply assumes others,
like calorific matter, electrical, magnetic matter,
in the conviction that it has thereby gotten hold
of properties as they truly are.
Equally current is the saying that
things consist of various matters or stuffs.
One is careful about calling these matters or stuffs “things,”
even though one will readily admit that,
for example, a pigment is a thing;
but I do not know whether luminous matter,
for instance, or calorific matter,
or electrical matter, etc., are called things.
The distinction is made between things and their components
without any exact statement as to whether these components also,
and to what extent, are things or perhaps just half-things;
but they are at least concretes in general.

The necessity of making the transition
from properties to matters,
or of assuming that the properties are truly matters,
has resulted from the fact that they are
what is the essential in things
and consequently their true self-subsistence.
At the same time, however,
the reflection of the property into itself
constitutes only one side of the whole reflection,
namely the sublation of the distinction
and the continuity of the property
(which was supposed to be a concrete existence for an other)
with itself.
Thinghood, as immanent negative reflection
and as a distinguishing that repels itself from the other,
has consequently been reduced to an unessential moment;
at the same time, however, it has further determined itself.

First, this negative moment has preserved itself,
for property has become a matter continuous with itself
and self-subsisting only inasmuch as the difference
of things has sublated itself;
thus the continuity of the property in the otherness
itself contains the moment of the negative,
and, as this negative unity, its self-subsistence is at the same time
the restored something of thinghood, negative self-subsistence
versus the positive self-subsistence of the stuff.

Second, the thing has thereby progressed
from its indeterminacy to full determinateness.
As thing in itself, it is abstract identity,
simple negative concrete existence,
or this concrete existence determined as the indeterminate;
it is then determined through its properties,
by virtue of which it is supposed to be
distinguished from other things;
but, since through the property the thing is
rather continuous with other things,
this imperfect distinction is sublated;
the thing has thereby returned into itself
and is now determined as determined;
it is determined in itself or is this thing.

But, third, this turning back into itself,
though a self-referring determination,
is at the same time an unessential determination;
the self-continuous subsistence makes up
the self-subsistent matter in which the difference of things,
their determinateness existing in and for itself,
is sublated and is something external.
Therefore, although the thing as this thing
is complete determinateness, this determinateness is such
in the element of inessentiality.

Considered from the side of the movement of the property,
this result follows in this way.
The property is not only external determination but
concrete existence immediately existing in itself.
This unity of externality and essentiality repels itself from itself,
for it contains reflection-into-itself and reflection-into-other,
and, on the one hand, it is determination as simple,
self-identical and self-referring self-subsistent in which the negative unity,
the one of the thing, is sublated;
on the other hand, it is this determination over against an other,
but likewise as a one which is reflected into itself
and is determined in itself;
it is, therefore, the matters and this thing.
These are the two moments of self-identical externality,
or of property reflected into itself.
The property was that by which things
were supposed to be distinguished.
Since the thing has freed itself of its
negative side of inhering in an other,
it has thereby also become free
from its being determined by other things
and has returned into itself
from the reference connecting it to the other.
At the same time, however, it is only the thing-in-itself
now become the other of itself,
for the manifold properties on their part
have become self-subsistent
and their negative connection
in the one of the thing is
now only a sublated connection.
Consequently, the thing is self-identical negation
only as against the positive continuity of the material.

The “this” thus constitutes the
complete determinateness of the thing,
a determinateness which is at the same time
an external determinateness.
The thing consists of self-subsistent matters
indifferent to the connection they have in the thing.
This connection is therefore only an unessential linking of them,
the difference of one thing from another depending on
whether there is in it a more or less of particular matters
and in what amount.
These matters overrun this thing, continue into others,
and that they belong to this thing is no restriction for them.
Just as little are they, moreover, a restriction for one another,
for their negative connection is only the impotent “this.”
Hence, in being linked together in it,
they do not sublate themselves;
they are as self-subsistent,
impenetrable to each other;
in their determinateness they refer only to themselves
and are a mutually indifferent manifold of subsistence;
the only limit of which they are capable is a quantitative one.
The thing as this is just their merely quantitative connection,
a mere collection, their “also.”
The thing consists of some quantum or other of a matter,
also of the quantum of another, and also of yet another;
this combination, of not having any combination alone
constitutes the thing.

IV.15
na ca-eka-citta-tantram vastu tad apramanakam tada kim syat

C. DISSOLUTION OF THE THING

This thing, in the manner it has determined itself
as the merely quantitative combination of free matters,
is the absolutely alterable.
Its alteration consists in one or more matters
being dropped from the collection,
or being added to this “also,”
or in the rearrangement of the matters'
respective quantitative ratio.
The coming-to-be and the passing-away of this thing is
the external dissolution of such an external bond,
or the binding of such for which it is indifferent
whether they are bound or not.
The stuffs circulate unchecked in or out of “this” thing,
and the thing itself is absolute porosity
without measure or form of its own.

So the thing, in the absolute determinateness
through which it is a “this,”
is the absolutely dissoluble thing.
This dissolution is an external process of being determined,
just like the being of the thing;
but its dissolution and the externality of its being
is the essential of this being;
the thing is only the “also”;
it consists only of this externality.
But it consists also of its matters,
and not just the abstract “this” as such
but the “this” thing whole is the dissolution of itself.
For the thing is determined as an external collection
of self-subsisting matters;
such matters are not things,
they lack negative self-subsistence;
it is the properties which are rather self-subsistent,
that is to say, are determined with a being
which, as such, is reflected into itself.
Hence the matters are indeed simple, referring only to themselves;
but it is their content which is a determinateness;
the immanent reflection is only the form of this content,
a content which is not, as such, reflected-into-itself
but refers to an other according to its determinateness.
The thing, therefore, is not only their “also,”
is not their reference to each other as indifferent
but is, on the contrary, equally so their negative reference;
and on account of their determinateness
the matters are themselves this negative reflection
which is the puncticity of the thing.
The one matter is not what the other is
according to the determinateness of its content
as contrasted to that of an other;
and the one is not to the extent that the other is,
in accordance with their self-subsistence.

The thing is, therefore, the connecting reference of
the matters of which it consists to each other,
in such a manner that the one matter,
and the other also, subsist in it,
and yet, at the same time,
the one matter does not subsist
in it in so far as the other does.
To the extent, therefore, that
the one matter is in the thing,
the other is thereby sublated;
but the thing is at the same time
the “also,” or the subsistence of the other matter.
In the subsistence of the one matter, therefore,
the other matter does not subsist,
and it also no less subsists in it;
and so with all these diverse matters
in respect to each other.
Since it is thus in the same respect
as the one matter subsists
that the other subsists also,
and this one subsistence of both is
the puncticity or the negative unity of the thing,
the two interpenetrate absolutely;
and since the thing is at the same time
only the “also” of the matters,
and these are reflected into their determinateness,
they are indifferent to one another,
and in interpenetrating they do not touch.
The matters are, therefore, essentially porous,
so that the one subsists in the pores
or in the non-subsistence of the others;
but these others are themselves porous;
in their pores or their non-subsistence
the first and also all the rest subsist;
their subsistence is at the same time
their sublatedness and the subsistence of others;
and this subsistence of the others is
just as much their sublatedness
and the subsisting of the first
and equally so of all others.
The thing is, therefore,
the self-contradictory mediation of
independent self-subsistence through its opposite,
that is to say, through its negation,
or of one self-subsisting matter
through the subsisting and non-subsisting of an other.

In “this” thing, concrete existence has attained its completion,
namely, that it is at once being that exists in itself,
or independent subsistence, and unessential concrete existence.
The truth of concrete existence is thus this:
that it has its in-itself in unessentiality,
or that it subsists in an other,
indeed in the absolute other,
or that it has its own nothingness for substrate.
It is, therefore, appearance.

CHAPTER 2

Appearance

Concrete existence is the immediacy of being
to which essence has again restored itself.
In itself this immediacy is the reflection of essence into itself.
As concrete existence, essence has stepped out of its ground
which has itself passed over into it.
Concrete existence is this reflected immediacy
in so far as, within, it is absolute negativity.
It is now also posited as such,
in that it has determined itself as appearance.

At first, therefore, appearance is
essence in its concrete existence;
essence is immediately present in it.
That it is not immediate,
but rather reflected concrete existence,
constitutes the moment of essence in it;
or concrete existence, as essential concrete existence,
is appearance.

Something is only appearance,
in the sense that concrete existence is
as such only a posited being,
not something that is in-and-for-itself.
This is what constitutes its essentiality,
to have the negativity of reflection,
the nature of essence, within it.
There is no question here of an alien,
external reflection to which essence would belong
and which, by comparing this essence with concrete existence,
would declare the latter to be appearance.
On the contrary, as we have seen,
this essentiality of concrete existence,
that it is appearance, is
concrete existence's own truth.
The reflection by virtue of which
it is this is its own.

But if it is said that something is only appearance,
meaning that as contrasted with it
immediate concrete existence is the truth,
then the fact is that appearance is the higher truth,
for it is concrete existence as essential,
whereas concrete existence is appearance
that is still void of essence
because it only contains in it
the one moment of appearance,
namely that of concrete existence
as immediate, not yet negative, reflection.
When appearance is said to be essenceless,
one thinks of the moment of its negativity as if,
by contrast with it, the immediate were
the positive and the true;
in fact, however, this immediate does not
yet contain essential truth in it.
Concrete existence rather ceases to be essenceless
by passing over into appearance.

Essence reflectively shines at first
just within, in its simple identity;
as such, it is abstract reflection,
the pure movement of nothing
through nothing back to itself.
Essence appears, and so it now is real shine,
since the moments of the shine have concrete existence.
Appearance, as we have seen, is the thing as
the negative mediation of itself with itself;
the differences which it contains are self-subsisting matters
which are the contradiction of being an immediate subsistence,
yet of obtaining their subsistence only in an alien self-subsistence,
hence in the negation of their own, but then again,
just because of that, also in the negation
of that alien self-subsistence
or in the negation of their own negation.
Reflective shine is this same mediation,
but its fleeting moments obtain in appearance
the shape of immediate self-subsistence.
On the other hand, the immediate self-subsistence which
pertains to concrete existence is reduced to a moment.
Appearance is therefore the unity
of reflective shine and concrete existence.

Appearance now determines itself further.
It is concrete existence as essential;
as essential, concrete existence differs
from the concrete existence which is unessential,
and these two sides refer to each other.

Appearance is, therefore, first, simple self-identity
which also contains diverse content determinations
and, both as identity and as the connecting reference
of these determinations,
is that which remains self-equal
in the flux of appearance;
this is the law of appearance.

But, second, the law which is
simple in its diversity
passes over into opposition;
the essential moment of appearance becomes
opposed to appearance itself
and, confronting the world of appearance,
the world that exists in itself
comes onto the scene.

Third, this opposition returns into its ground;
that which is in itself is in the appearance
and, conversely, that which appears is determined
as taken up into its being-in-itself.
Appearance becomes relation.

IV.16
tad-uparaga-apeksitvat-cittasya vastu jnata-ajnatam

A. THE LAW OF APPEARANCE

1. Appearance is the concrete existent
mediated through its negation,
which constitutes its subsistence.
This, its negation, is
indeed another self-subsistent;
but the latter is just as
essentially something sublated.
The concrete existent is
consequently the turning back of itself
into itself through its negation
and through the negation of this negation;
it has, therefore, essential self-subsistence,
just as it is equally immediately an absolute positedness
that has a ground and an other for its subsistence.
In the first place, therefore, appearance is
concrete existence along with its essentiality,
the positedness along with its ground;
but this ground is the negation,
and the other self-subsistent,
the ground of the first,
is equally only a positedness.
Or the concrete existent is, as an appearance,
reflected into an other
and has this other for its ground,
and this ground is itself only this,
to be reflected into another.
The essential self-subsistence
that belongs to it
because it is a turning back into itself is,
for the sake of the negativity of the moments,
the return of nothing through nothing back to itself;
the self-subsistence of the concrete existent is
therefore only the reflective shine of essence.
The linkage of the reciprocally grounding
concrete existents consists, therefore,
in this reciprocal negation,
namely that the subsistence of the one is not
the subsistence of the other but is its positedness,
where this connection of positedness
alone constitutes their subsistence.
The ground is present as it is in truth,
namely as being a first which is only a presupposed.

This now constitutes the negative side of appearance.
In this negative mediation, however,
there is immediately contained the positive identity of
the concrete existent with itself.
For this concrete existent is not positedness
vis-à-vis an essential ground,
or is not the reflective shine in a self-subsistent,
but is rather positedness that refers itself to a positedness,
or a reflective shine only in a reflective shine.
In this, its negation, or in its other which is itself something sublated,
it refers to itself and is thus self-identical or positive essentiality.
This identity is not the immediacy that pertains to concrete existence
as such and only is its unessential moment of subsisting in an other.
It is rather the essential content of appearance which has two sides:
first, to be in the form of positedness or external immediacy;
second, to be positedness as self-identical.
According to the first side, it is as a determinate being,
but one which in keeping with its immediacy is accidental, unessential,
and subject to transition, to coming-to-be and passing-away.
According to the other side, it is the simple content determination
exempted from that flux, the permanent element in it.

This content, besides being in general the simple element of the transient,
is also a determined content, varied in itself.
It is the reflection of appearance, of the negative determinate being,
into itself, and therefore contains determinateness essentially.
Appearance is however the multifarious diversity
of immediately existing beings that revels in unessential manifoldness;
its reflected content, on the other hand,
is its manifoldness reduced to simple difference.
Or, more precisely, the determinate essential content is not
just determined in general but, as the essential element of appearance,
is complete determinateness; the one and its other.
Each of these two has in appearance its subsistence in the other,
but in such a way that it is at the same time
only in the other's non-subsistence.
This contradiction sublates itself;
and its reflection into itself is
the identity of their two-sided subsistence,
namely that the positedness of the one is
also the positedness of the other.
The two constitute one subsistence,
each at the same time as a different content
indifferent to the other.
In the essential side of appearance,
the negativity of the unessential content,
that it sublates itself, has thus gone back into identity;
it is an indifferent subsistence which is not
the sublatedness of the other but rather its subsistence.

This unity is the law of appearance.

2. The law is thus the positive element
of the mediation of what appears.
Appearance is at first concrete existence
as negative self-mediation,
so that the concrete existent,
through its own non-subsistence,
through an other and again through
the non-subsistence of this other,
is mediated with itself.
In this there is contained,
first, the merely reflective shining
and the disappearing of both,
the unessential appearance;
second, also the persistence or the law;
for each of the two concretely exists
in the sublation of the other,
and their positedness is as
their negativity at the same time
the identical positive positedness of both.

This permanent subsistence which appearance
obtains in the law is thus,
as it has determined itself,

first, opposed to the immediacy
of the being which concrete existence has.
This immediacy is indeed one which is in itself reflected,
namely the ground that has gone back into itself;
but in appearance this simple immediacy is now distinguished
from the reflected immediacy that first began
to separate itself in the “thing.”
The concretely existing thing in its dissolution
has become this opposition;
the positive element of its dissolution is
the said self-identity of what appears,
a positedness in the positedness of its other.

Second, this reflected immediacy is itself determined
as positedness over against the immediate determinate
being of concrete existence.
This positedness is henceforth what is essential
and the true positive.
The German expression Gesetz [law] likewise contains this
note of positedness or Gesetztsein.
In this positedness there lies the essential connection of
the two sides of the difference that the law contains;
they are a diverse content,
each immediate with respect to the other,
and they are this as the reflection of
the disappearing content belonging to appearance.
As essential difference, the different sides are
simple, self-referring determinations of content.
But just as equally, neither is immediate,
just for itself, but is rather essential positedness,
or is only to the extent that the other is.

Third, appearance and law have one and the same content.
The law is the reflection of appearance into self-identity;
appearance, as an immediate which is null,
thus stands opposed to that which is immanently reflected,
and the two are distinguished according to form.
But the reflection of appearance by virtue of which this difference is,
is also the essential identity of appearance itself and its reflection,
and this is in general the nature of reflection;
it is what in the positedness is self-identical
and indifferent to that difference,
which is form or positedness hence a content continuous
from appearance to law, the content of the law and of the appearance.

This content thus constitutes the substrate of appearance;
the law is this substrate itself,
appearance is the same content but contains still more,
namely the unessential content of its immediate being.
And so is also the form determination by which
appearance as such is distinguished from the law,
namely a content and equally a content distinguished
from the content of the law.
For concrete existence, as immediacy in general,
is likewise a self-identity of matter and form
which is indifferent to its form determinations
and is, therefore, a content;
the concrete existence is the thinghood
with its properties and matters.
But it is the content whose self-subsisting immediacy is
at the same time also only a non-subsistence.
But the self-identity of the content
in this its non-subsistence
is the other, essential content.
This identity, the substrate of appearance,
which constitutes law,
is appearances's own moment;
it is the positive side of the essentiality
by virtue of which concrete existence is appearance.

The law, therefore, is not beyond appearance
but is immediately present in it;
the kingdom of laws is the restful copy
of the concretely existing or appearing world.
But, more to the point, the two are one totality,
and the concretely existing world is itself
the kingdom of laws which, simple identity,
is at the same time self-identical in the positedness
or in the self-dissolving self-subsistence of concrete existence.
In the law, concrete existence returns to its ground;
appearance contains both of these, the simple ground
and the dissolving movement of the appearing universe,
of which the law is the essentiality.

3. The law is therefore the essential appearance;
it is the latter's reflection into itself in its positedness,
the identical content of itself and the unessential concrete existence.

In the first place, this identity of the law
with its concrete existence is now, to start with,
immediate, simple identity, and the law is indifferent
with respect to its concrete existence;
appearance still has another content as
contrasted with the content of the law.
That content is indeed the unessential one
and the return into the latter;
but for the law it is an original starting point not posited by it;
as content, therefore, it is externally bound up with the law.
Appearance is an aggregate of more detailed determinations
that belong to the “this” or the concrete,
and are not contained in the law
but are rather determined each by an other.

Secondly, that which appearance contains distinct from the law
determined itself as something positive or as another content;
but it is essentially a negative;
it is the form and its movement is a movement that belongs to appearance.
The kingdom of laws is the restful content of appearance;
the latter is this same content but displayed in restless flux
and as reflection-into-other.
It is the law as negative, relentlessly self-mutating concrete existence,
the movement of the passing over into the opposite,
of self-sublation and return into unity.
This side of the restless form or of the negativity does
not contain the law;
as against the law, therefore, appearance is the totality,
for it contains the law but more yet,
namely the moment of the self-moving form.

Thirdly, this shortcoming is manifested in the law
in the mere diversity at first,
and the consequent internal indifference, of its content;
the identity of its sides with one another is at first, therefore,
only immediate and hence inner, not yet necessary in other words.
In a law two content determinations are essentially bound together
(for instance, spatial and temporal magnitudes in the law of falling bodies:
the traversed spaces vary as the squares of the elapsed times);
they are bound together; this connection is at first only an immediate one.

At first, therefore, it is likewise only a posited connection,
just as the immediate has obtained in appearance
the meaning of positedness in general.
The essential unity of the two sides of the law would be their negativity,
namely that each contains the other in it;
but in the law this essential unity has not yet come the fore.
(Thus it is not contained in the concept of the space traversed by
a falling body that time corresponds to it as a square.
Because the falling is a sensible movement, it is the ratio of space and time;
but first, that time refers to space and space to time
does not lie in the determination of time itself,
that is to say, in time as ordinarily represented;
it is said that time can very well be represented
without space and space without time;
the one thus comes to the other externally,
and their external reference to each other is movement.
Second, the more particular determination of how
the magnitudes further relate to each other in movement is indifferent.
The relevant law here is drawn from experience
and is to this extent immediate;
there is still required a proof, that is, a mediation,
in order to know that the law not only occurs but is necessary;
the law as such does not contain this proof and its objective necessity.)

The law is, therefore, only the positive essentiality of appearance,
not its negative essentiality according to which
the content determinations are moments of the form,
as such pass over into their other
and are in their own selves not themselves but their other.
In the law, therefore, although the positedness
of the one side of it is the positedness of the other side,
the content of the two sides is indifferent to this connection;
it does not contain this positedness in it.
Law, therefore, is indeed essential form,
but not as yet real form which is reflected into its sides as content.

IV.17
sada jnata citta-vrttaya tat-prabho purusasya-aparinamitvat

B. THE WORLD OF APPEARANCE AND THE WORLD-IN-ITSELF

1. The concrete existing world tranquilly
raises itself to a kingdom of laws;
the null content of its manifold determinate being
has its subsistence in an other;
its subsistence is therefore its dissolution.
In this other, however, that which appears also comes to itself;
thus appearance is in its changing also an enduring,
and its positedness is law.
Law is this simple identity of appearance with itself;
it is, therefore, its substrate and not its ground,
for it is not the negative unity of appearance
but, as its simple identity, is its immediate unity,
the abstract unity, alongside which, therefore,
its other content also occurs.
The content is this content; it holds together internally,
or has its negative reflection inside itself.
It is reflected into an other;
this other is itself a concrete existence of appearance;
the appearing things have their grounds and conditions
in other appearing things.

In fact, however, law is also
the other of appearance as appearance,
and its negative reflection as in its other.
The content of appearance,
which differs from the content of law,
is the concrete existent
which has negativity for its ground
or is reflected into its non-being.
But this other, which is also a concrete existent,
is such an existent as likewise reflected into its non-being;
it is thus the same and that which appears in it
is in fact reflected not into an other but into itself;
it is this very reflection of positedness into itself
which is law.
But as something that appears
it is essentially reflected into its non-being,
or its identity is itself essentially
just as much its negativity and its other.
The immanent reflection of appearance,
law, is therefore not only
the identical substrate of appearance
but the latter has in law its opposite,
and law is its negative unity.

Now through this, the determination of law
has been altered within the law itself.
At first, law is only a diversified content
and the formal reflection of positedness into itself,
so that the positedness of one of its sides is
the positedness of the other side.
But because it is also the negative reflection into itself,
its sides behave not only as different
but as negatively referring to each other.
Or, if the law is considered just for itself,
the sides of its content are indifferent to each other;
but they are no less sublated through their identity;
the positedness of the one is the positedness of the other;
consequently, the subsistence of each is
also the non-subsistence of itself.
This positedness of the one side in the other
is their negative unity,
and each positedness is not only the positedness
of that side but also of the other,
or each side is itself this negative unity.
The positive identity which they have in the law as such is
at first only their inner unity
which stands in need of proof and mediation,
since this negative unity is not yet posited in them.
But since the different sides of law are now determined
as being different in their negative unity,
or as being such that each contains the other within
while at the same time repelling this otherness from itself,
the identity of law is now also one which is posited and real.

Consequently, law has likewise obtained
the missing moment of the negative form of its sides,
the moment that previously still belonged to appearance;
concrete existence has thereby returned into itself fully
and has reflected itself into its absolute otherness
which has determinate being-in-and-for-itself.
That which was previously law, therefore,
is no longer only one side of the whole.
It is the essential totality of appearance,
so that it now obtains also the moment of
unessentiality that belonged to the latter
but as reflected unessentiality
that has determinate being in itself,
that is, as essential negativity.
As immediate content, law is determined in general,
distinguished from other laws,
of which there is an indeterminate multitude.
But because now it explicitly is essential negativity,
it no longer contains that merely
indifferent, accidental content determination;
its content is rather every determinateness in general,
essentially connected together in a totalizing connection.
Thus appearance reflected-into-itself is
now a world that discloses itself above
the world of appearance as one
which is in and for itself.

The kingdom of laws contains only
the simple, unchanging but diversified content
of the concretely existing world.
But because it is now the total reflection of this world,
it also contains the moment of its essenceless manifoldness.
This moment of alterability and alteration,
reflected into itself and essential,
is the absolute negativity
or the form in general as such:
its moments, however, have  the reality of
self-subsisting but reflected concrete existence
in the world  that has determinate being in-and-for-itself,
just as, conversely, this reflected
self-subsistence has form in it,
and its content is therefore not a mere manifold
but a content holding itself together essentially.

This world which is in and for itself is
also called the suprasensible world,
inasmuch as the concretely existing world
is characterized as sensible,
that is, as one intended for intuition,
which is the immediate attitude of consciousness.
The suprasensible world likewise has
immediate, concrete existence,
but reflected, essential concrete existence.
Essence has no immediate existence yet;
but it is, and in a more profound sense than being;
the thing is the beginning of the reflected concrete existence;
it is an immediacy which is not yet posited,
not yet essential or reflected;
but it is in truth not an immediate which is simply there.
Things are posited only as the
things of another, suprasensible, world
first as true concrete existences,
and, second, as the truth in contrast to that which just is.
What is recognized in them is that there is
a being distinguished from immediate being,
and this being is true concrete existence.
On the one side, the sense-representation
that ascribes concrete existence
only to the immediate being of
feeling and intuition is in this determination overcome;
but, on the other side, also overcome is
the unconscious reflection which,
although it possesses the representation of things,
forces, the inner, and so on, does not know
that such determinations are not sensible
or immediately existing beings,
but reflected concrete existences.

2. The world which is in and for itself is
the totality of concrete existence;
outside it there is nothing.
But, within it, it is absolute negativity or form,
and therefore its immanent reflection is negative self-reference.
It contains opposition, and splits internally
as the world of the senses and as the world of otherness
or the world of appearance.
For this reason, since it is totality,
it is also only one side of the totality
and constitutes in this determination
a self-subsistence different from the world of appearance.
The world of appearance has its negative unity
in the essential world to which it founders
and into which it returns as to its ground.
Further, the essential world is also
the positing ground of the world of appearances;
for, since it contains the absolute form essentially,
it sublates its self-identity, makes itself into positedness
and, as this posited immediacy, it is the world of appearance.

Further, it is not only ground in general
of the world of appearance but its determinate ground.
Already as the kingdom of laws it is a manifold of content,
indeed the essential content of the world of appearance,
and, as ground with content, it is
the determinate ground of that other world.
But it is such only according to that content,
for the world of appearance still had other
and manifold content than the kingdom of laws,
because the negative moment was still the one peculiarly its own.
But because the kingdom of laws now has this moment likewise in it,
it is the totality of the content of the world of appearance
and the ground of all its manifoldness.
But it is at the same time the negative of
this manifoldness and thus a world opposed to it.
That is to say, in the identity of the two worlds,
because the one world is determined
according to form as the essential
and the other as the same world
but posited and unessential,
the connection of ground has indeed been restored.
But it has been restored as the ground-connection of appearance,
namely as the connection,
not of the two sides of an identical content,
nor of a mere diversified content, like law,
but as total connection,
or as negative identity and essential connection
of the opposed sides of the content.
The kingdom of laws is not only this,
that the positedness of a content
is the positedness of an other,
but rather that this identity, as we have seen,
is essentially also negative unity,
and in this negative unity
each of the two sides of law is in it,
therefore, its other content;
consequently, the other is not
an other in general, indeterminedly,
but is its other, equally containing
the content determination of that other;
and thus the two sides are opposed.
Now, because the kingdom of laws now has in it
this negative moment, namely opposition,
and thus, as totality, splits into a world
which exists in and for itself and a world of appearance,
the identity of these two is
the essential connection of opposition.
The connection of ground is, as such, the opposition
which, in its contradiction, has foundered to the ground;
and concrete existence is the ground that has come to itself.
But concrete existence becomes appearance;
ground is sublated in concrete existence;
it reinstates itself as the return of appearance into itself,
but does so as sublated ground, that is to say,
as the ground-connection of opposite determinations;
the identity of such determinations, however, is
essentially a becoming and a transition,
no longer the connection of ground as such.

The world that exists in and for itself is
thus itself a world distinguished within itself,
in the total compass of a manifold content.
That is to say, it is identical with the world of appearance
or the posited world and to this extent it is its ground.
But its identity connection is at the same time
determined as opposition,
because the form of the world of appearance is
reflection into its otherness
and this world of appearance, therefore, in the
world that exists in and for itself
has truly returned into itself,
in such a manner that
that other world is its opposite.
Their connection is, therefore, specifically this,
that the world that exists in and for itself is
the inversion of the world of appearance.

IV.18
na tat sva-abhasam drsyatvat

IV.19
eka-samaye ca-ubhaya-anavadharanam

C. THE DISSOLUTION OF APPEARANCE

The world that exists in and for itself is
the determinate ground of the world of appearance
and is this only in so far as, within it, it is
the negative moment and hence the totality of
the content determinations and their alterations
that correspond to that world of appearance,
yet constitutes at the same time its completely opposed side.
The two worlds thus relate to each other in such a way
that what in the world of appearance is positive,
in the world existing in and for itself is negative,
and, conversely, what is negative in the former is
positive in the latter.
The north pole in the world of appearance is
the south pole in and for itself, and vice-versa;
positive electricity is in itself negative, and so forth.
What is evil in the world of appearance is
in and for itself goodness
and a piece of good luck.

In fact it is precisely in this opposition
of the two worlds that their difference has disappeared,
and what was supposed to be
the world existing in and for itself is
itself the world of appearance
and this last, conversely,
the world essential within.
The world of appearance is in the first instance
determined as reflection into otherness,
so that its determinations and concrete existences have
their ground and subsistence in an other;
but because this other, as other,
is likewise reflected into an other,
the other to which they both refer is
one which sublates itself as other;
the two consequently refer to themselves;
the world of appearance is within it,
therefore, law equal to itself.
Conversely, the world existing in and for itself is
in the first instance self-identical content,
exempt from otherness and change;
but this content, as complete reflection of
the world of appearance into itself,
or because its diversity is difference
reflected into itself and absolute,
consequently contains negativity as a moment
and self-reference as reference to otherness;
it thereby becomes self-opposed, self-inverting, essenceless content.
Further, this content of the world existing in and for itself has
thereby also retained the form of immediate concrete existence.
For it is at first the ground of the world of appearance;
but since it has opposition in it, it is equally
sublated ground and immediate concrete existence.

Thus the world of appearance and the essential world are
each, each within it, the totality of
self-identical reflection and of reflection-into-other,
or of being-in-and-for-itself.
They are both the self-subsisting wholes of concrete existence;
the one is supposed to be only reflected concrete existence,
the other immediate concrete existence;
but each continues into the other and, within, is
therefore the identity of these two moments.
What we have, therefore, is this totality
that splits into two totalities,
the one reflected totality and the other immediate totality.
Both, in the first instance, are self-subsistent;
but they are this only as totalities,
and this they are inasmuch as each essentially contains
the moment of the other in it.
Hence the distinct self-subsistence of each,
one determined as immediate and one as reflected,
is now so posited as to be essentially the reference to the other
and to have its self-subsistence in this unity of the two.

We started off from the law of appearance;
this law is the identity of a content and another content different from it,
so that the positedness of the one is the positedness of the other.
Still present in law is this difference,
that the identity of its sides is at first only an internal identity
which the two sides do not yet have in them.
Consequently the identity is, for its part, not realized;
the content of law is not identical but indifferent, diversified.
This content, therefore, is on its side only in itself so determined
that the positedness of the one is the positedness of the other;
this determination is not yet present in it.
But now law is realized;
its inner identity is existent at the same time
and, conversely, the content of law is raised to ideality;
for it is sublated within, is reflected into itself,
for each side has the other in it,
and therefore is truly identical with it and with itself.

Thus is law essential relation.
The truth of the unessential world is at first
a world in and for itself and other to it;
but this world is a totality,
for it is itself and the first world;
both are thus immediate concrete existences
and consequently reflections in their otherness,
and therefore equally truly reflected into themselves.
“World” signifies in general
the formless totality of a manifoldness;
this world has foundered both as essential world
and as world of appearance;
it is still a totality or a universe
but as essential relation.
Two totalities of content have arisen in appearance;
at first they are determined as
indifferently self-subsisting vis-à-vis each other,
each having indeed form within it
but not with respect to the other;
this form has however demonstrated
itself to be their connecting reference,
and the essential relation is
the consummation of their unity of form.

CHAPTER 3

The essential relation

IV.20
citta-antara-drsye buddhi-buddher atiprasanga smrti-sankara ca

The truth of appearance is the essential relation.
Its content has immediate self-subsistence:
the existent immediacy and the reflected immediacy
or the self-identical reflection.
In this self-subsistence, however,
it is at the same time a relative content;
it is simply and solely as a reflection into its other,
or as unity of the reference with its other.
In this unity, the self-subsistent content is
something posited, sublated;
but precisely this unity is what constitutes
its essentiality and self-subsistence;
this reflection into an other is reflection into itself.
The relation has sides, since it is reflection into an other;
so its difference is internal to it,
and its sides are independent subsistence,
for in their mutually indifferent diversity
they are thrown back into themselves,
so that the subsistence of each equally has its meaning
only in its reference to the other
or in the negative unity of both.

The essential relation is therefore not yet
the true third to essence and to concrete existence
but already contains the determinate union of the two.
Essence is realized in it in such a way that
it has self-subsistent, concrete existents for its subsistence,
and these concrete existents have returned
from their indifference back into their essential unity
so that they have only this unity as their subsistence.
Also the reflective determinations of positive and negative are
reflected into themselves only as each is reflected into its opposite;
but they have no other determination besides this their negative unity,
whereas the essential relation has sides
that are posited as self-subsistent totalities.
It is the same opposition as that of positive and negative,
but it is such as an inverted world.
The side of the essential relation is a totality
which, however, essentially has an opposite or a beyond;
it is only appearance;
its concrete existence,
rather than being its own,
is that of its other.
It is, therefore, something internally fractured;
but this, its sublated being, consists in
its being the unity of itself and its other,
therefore a whole, and precisely for this reason
it has self-subsistent concrete existence
and is essential reflection into itself.

This is the concept of relation.
At first, however, the identity it contains
is not yet perfect;
the totality which each relative is as relative,
is only an inner one;
the side of the relation is posited at first
in one of the determinations of negative unity;
what constitutes the form of the relation is
the specific self-subsistence of each of the two sides.
The identity of the form is therefore only a reference,
and the self-subsistence of the sides falls outside it,
that is to say, it falls in the sides;
we still do not have the reflected unity
of the identity of the relation
and of the self-subsistent concrete existents;
we still do not have substance.
It follows that the concept of relation has
indeed shown itself to be the unity
of reflected and immediate self-subsistence.
But it is this concept still immediately at first;
immediate are therefore its moments vis-à-vis each other,
and immediate is the unity of the reference
connecting them essentially;
a unity this, which only then is the true unity
that conforms to the concept,
when it has realized itself, that is to say,
through its movement has posited itself as this unity.

The essential relation is therefore immediately
the relation of the whole and the parts
the reference of reflected and immediate self-subsistence,
so that both are at the same time
mutually conditioning and presupposing.

In this relation, neither of the sides is
yet posited as moment of the other;
their identity is therefore itself one side,
or not their negative unity.
Hence, secondly, the relation passes over into one
in which one side is the moment of the other
and is present there as in its ground,
the true self-subsistent element of both.
This is the relation of force and its expression.

Third, the inequality still present
in this reference sublates itself,
and the final relation is that of inner and outer.
In this difference,
which has now become totally formal,
relation itself founders,
and substance or actuality come on the stage
as the absolute unity of
immediate and reflected concrete existence.

SECTION III

Actuality

Actuality is the unity of essence and concrete existence;
in it, shapeless essence and unstable appearance
(subsistence without determination
and manifoldness without permanence)
have their truth.
Although concrete existence is the immediacy
that has proceeded from ground,
it still does not have form explicitly posited in it;
inasmuch as it determines and informs itself, it is appearance;
and in developing this subsistence that otherwise only is
a reflection-into-other into an immanent reflection,
it becomes two worlds, two totalities of content,
one determined as reflected into itself
and the other as reflected into other.
But the essential relation exposes
the formality of their connection,
and the consummation of the latter is
the relation of the inner and the outer
in which the content of both is equally
only one identical substrate
and only one identity of form.
This identity has come about also in regard to form,
the form determination of their difference is sublated,
and that they are one absolute totality is posited.

This unity of the inner and outer is absolute actuality.
But this actuality is, first, the absolute as such
(in so far as it is posited as a unity
in which the form has sublated itself)
making itself into the empty or external
distinction of an outer and inner.
Reflection relates to this absolute
as external to it;
it only contemplates it
rather than being its own movement.
But it is essentially this movement
and is, therefore, as the absolute's
negative turning back into itself.

Second, it is actuality proper.
Actuality, possibility, and necessity constitute
the formal moments of the absolute,
or its reflection.

Third, the unity of the absolute
and its reflection is
the absolute relation,
or rather the absolute as
relation to itself, substance.

CHAPTER 1

The absolute

The simple solid identity of the absolute is indeterminate, or rather,
every determinateness of essence and concrete existence,
or of being in general as well as of reflection,
has dissolved itself into it.
Accordingly, the determining of what is
the absolute appears to be a negating,
and the absolute itself appears only as
the negation of all predicates, as the void.
But since it must equally be spoken of
as the position of all predicates,
it appears as the most formal of contradictions.
In so far as that negating and this positing
belong to external reflection,
what we have is a formal, unsystematic dialectic
that has an easy time picking up
a variety of determinations here and there,
and is just as at ease demonstrating, on the one hand,
their finitude and relativity, as declaring, on the other,
that the absolute, which it vaguely envisages as totality,
is the dwelling place of all determinations,
yet is incapable of raising either
the positions or the negations to a true unity.
The task is indeed to demonstrate what the absolute is.
But this demonstration cannot be either
a determining or an external reflection
by virtue of which determinations
of the absolute would result,
but is rather the exposition of the absolute,
more precisely the absolute's own exposition,
and only a displaying of what it is.

IV.21
citer apratisamkramayas tad-akara-apattau svabuddhi-samvedanam

A. THE EXPOSITION OF THE ABSOLUTE

The absolute is not just being, nor even essence.
The former is the first unreflected immediacy;
the latter, the reflected immediacy;
further, each is explicitly a totality,
but a determinate totality.
Being emerges in essence as concrete existence,
and the connection of being and essence develops
into the relation of inner and outer.
The inner is essence, but as a totality
whose essential determination is
to be referred to being and to be being immediately.
The outer is being, but with the essential determination of
being immediately connected with reflection
and, equally, in a relationless identity with essence.
The absolute itself is the absolute unity of the two;
it is that which constitutes in general
the ground of the essential relation
which, as only relation, has yet
to return into this its identity
and whose ground is not yet posited.

It follows that the determination of
the absolute is to be absolute form,
but at the same time not as an identity
whose moments only are simple determinacies,
but, on the contrary, as an identity
whose moments are each explicitly the totality
and hence, indifferent with respect to the form,
the complete content of the whole.
But, conversely, the absolute is absolute content
in such a way that this content,
which is as such indifferent plurality,
explicitly has the negative connection of form
by virtue of which its manifold is
only one substantial identity.

Thus the identity of the absolute is
for this reason absolute identity,
because each of its parts is itself the whole
or each determinateness is the totality, that is,
because determinateness has become as such
a thoroughly transparent reflective shine,
a difference that has disappeared in its positedness.
Essence, concrete existence, the world existing in itself,
whole, parts, force:
these reflected determinations appear to representation
as true being valid in and for itself;
but against them the absolute is the ground
into which they have foundered.
Because in the absolute the form is
now only simple self-identity,
the absolute does not determine itself,
for the determination is a difference of form
which is valid as such from the start.
But because the absolute at the same time contains
every difference and form determination in general,
or because it is itself absolute form and reflection,
the difference of content must also come into it.
But the absolute itself is the absolute identity;
to be this identity is its determination,
for the manifoldness of the world-in-itself
and of the phenomenal world has all been sublated in it.
In the absolute itself there is no becoming,
since the absolute is not being;
nor does the absolute determine itself reflectively,
for it is not the essence which determines itself only inwardly;
and it also does not externalize itself,
for it is the identity of inner and outer.
But in this way the movement of reflection
stands over against its absolute identity.
The movement is sublated in this identity
and is thus only its inner;
but consequently its outer.

At first, therefore, the movement consists only
in sublating its act in the absolute.
It is the beyond of the manifold differences
and determinations and of their movement,
a beyond that lies at the back of the absolute.
It is thus the negative exposition of the absolute
earlier alluded to.
In its true presentation, this exposition is
the preceding whole of the logical movement
of the spheres of being and essence,
the content of which has not been gathered in
from outside as something given and contingent;
nor has it been sunk into the abyss of the absolute
by a reflection external to it;
on the contrary, it has determined itself within it
by virtue of its inner necessity,
and, as being's own becoming
and as the reflection of essence,
has returned into the absolute
as into its ground.

But this exposition has itself also a positive side,
for in foundering to the ground the finite demonstrates
that its nature is to be referred to the absolute,
or to contain the absolute within.
However, this side is not as much
the positive exposition of the absolute
as it is rather the exposition of the determinations,
namely that these have the absolute for their abyss,
but also for their ground,
or that that which imparts subsistence to them,
to their reflective shine, is the absolute itself.
Being as shine is not nothing but reflection,
reference to the absolute;
or it is a shine inasmuch as
that which shines in it is the absolute.
This positive exposition thus halts the finite
just before its disappearing:
it considers it an expression and
a copy of the absolute.
But this transparency of the finite
that lets only the absolute transpire through it
ends up in complete disappearance,
for there is nothing in the finite
which would retain for it a difference
over against the absolute;
as a medium, it is absorbed by
that through which it shines.

This positive exposition of the absolute is
therefore itself only a reflective shine,
for the true positive, that which contains
the exposition and the expounded content,
is the absolute itself.
Whatever the further determinations that may occur,
the form in which the absolute reflectively shines is
a nullity which the exposition gathers up from outside
and in which it gains for itself
a starting point for its activity.
Any such determination has in the absolute,
not its beginning but its end.
This expository process, therefore,
though it is an absolute act
because of its reference to
the absolute into which it returns,
is not so at its starting point
which is a determination
external to the absolute.

But in actual fact
the exposition of the absolute
is the absolute's own doing,
an act that begins from itself
and arrives at itself.
The absolute, only as absolute identity,
is absolute in a determined guise,
that is, as identical absolute;
it is posited as such by reflection
over against opposition and manifoldness;
or it is only the negative of
reflection and determination in general.
It is not just the exposition of the absolute
which is therefore something incomplete,
but this absolute itself
which is only arrived at.
Or again, the absolute
which is only as absolute identity is
only the absolute of an external reflection.
It is, therefore, not the absolutely absolute
but the absolute in a determination,
or it is attribute.

But the absolute is not attribute just because
it is the subject matter of an external reflection
and is consequently something determined by it.
Or, reflection is not only external to it;
but, precisely because it is external to it,
it is immediately internal to it.
The absolute is absolute only because
it is not abstract identity
but is the identity of being and essence,
or the identity of the inner and the outer.
It is therefore itself the absolute form
that makes it reflectively shine within itself
and determines it as attribute.

IV.22
drastr-drsya-uparaktam cittam sarva-artham

B. THE ABSOLUTE ATTRIBUTE

The expression which we have used, “the absolute absolute,”
denotes the absolute which in its form
has returned back into itself
or whose form is equal to its content.
The attribute is just the relative absolute,
a combination which only signifies the absolute
in a form determination.
For at first, before its complete exposition,
the form is only internally
or, which is the same, only externally;
it is at first determinate form in general
or negation in general.
But because form is at the same time
as the form of the absolute,
the attribute is the whole content of the absolute;
it is the totality which earlier appeared as a world,
or as one of the sides of the essential relation,
each of which is itself the whole.
But both worlds, the phenomenal world
and the world that exists in and for itself,
were supposed to be opposed
to each other in their essence.
Each side of the essential relation was
indeed equal to the other:
the whole as much as the parts,
the expression of force the same content
as force itself,
and the outer everywhere the same as the inner.
But these sides were at the same time
supposed each to have still
an immediate subsistence of its own,
the one side as existent immediacy
and the other as reflected immediacy.
In the absolute, on the contrary,
these different immediacies have been
reduced to a reflective shine,
and the totality that the attribute is
is posited as its true and single subsistence,
while the determination in which it is
is posited as unessential subsistence.

The absolute is attribute because,
as simple absolute identity,
it is in the determination of identity;
now to the determination as such
other determinations can be attached,
for instance, also that there are several attributes.
But because absolute identity has only this meaning,
that not only all determinations have been sublated
but that reflection itself has also sublated itself,
all determinations are thus posited in it as sublated.
Or the totality is posited as absolute totality.
Or again, the attribute has the absolute for its
content and subsistence and, consequently, its form determination
by which it is attribute is also posited, posited immediately
as mere reflective shine; the negative is posited as negative.
The positive reflective shine that the exposition gives itself
through the attribute in that it does not take the finite
in its limitation as something that exists in and for itself
but dissolves its subsistence into the absolute
and expands it into attribute;
sublates precisely this, that the attribute is attribute;
it sinks it and its differentiating act into the simple absolute.

But since reflection thus reverts
from its differentiating act
only to the identity of the absolute,
it has not at the same time
left its externality behind
and has not arrived at the true absolute.
It has only reached the
indeterminate, abstract identity,
which is to say, the identity
in the determinateness of identity.
Or, since reflection determines the
absolute into attribute as inner form,
this determining is something
still distinct from externality;
the inner determination does not
penetrate the absolute;
the attribute's expression,
as something merely posited,
is to disappear into the absolute.

The form by virtue of which
the absolute would be attribute,
whether it is taken as outer or inner,
is therefore posited as something null in itself,
an external reflective shine,
or a mere way and manner.

IV.23
tad asamkhyeya-vasanabhi citram api para-artham samhatya-karitvat

C. THE MODE OF THE ABSOLUTE

The attribute is first the
absolute in simple self-identity.
Second, it is negation,
a negation which is as such
formal immanent reflection.
These two sides constitute at first
the two extremes of the attribute,
the middle term of which is the attribute itself,
since it is both the absolute and the determinateness.
The second of these extremes is the negative as negative,
the reflection external to the absolute.
Or inasmuch as the negative is
taken as the inner of the absolute
and its own determination is to posit itself as mode,
it is then the self-externality of the absolute,
the loss of itself in the
changeability and contingency of being,
its having passed over into its opposite
without turning back into itself,
the manifoldness of form and content
determinations that lacks totality.

But the mode, the externality of the absolute, is not just this.
It is rather externality posited as externality,
a mere way and manner,
hence the reflective shine as reflective shine,
or the reflection of form into itself;
hence, the self-identity which is the absolute.
In actual fact, therefore, the absolute is first posited
as absolute identity only in the mode;
it is what it is, namely self-identity,
only as self-referring negativity,
as reflective shining which is posited as reflective shining.

Hence, in so far as the exposition of the absolute
begins from its absolute identity
and passes over to the attribute
and from there to the mode,
it has therein exhaustively run through its moments.

But first, in this course it does not just behave
negatively towards these determinations;
its act is rather the reflective movement itself,
and it is only as such a movement that
the absolute truly is absolute identity.

Second, the exposition does not thereby deal with mere externality,
and the mode is not only the most external externality.
Rather, since the mode is reflective shine as shine,
it is an immanent turning back, the self-dissolving reflection,
and it is in being this reflection that
the absolute is absolute being.

Third, the reflective act of exposition seems to begin
from its own determinations and from something external,
to take up the modes or even the determinations of the attribute
as if they were found outside the absolute
and its contribution were only to reduce them
to undifferentiated identity.
But it has in fact found the determinateness
from which it begins in the absolute itself.
For as first undifferentiated identity,
the absolute is itself only the determinate absolute,
or attribute, because it is the unmoved, still unreflected absolute.
This determinateness, since it is determinateness,
belongs to the reflective movement,
and it is through this movement alone
that the absolute is determined as the first identity;
through it alone that it has absolute form
and does not just exist as self-equal
but posits itself as self-equal.

Accordingly the true meaning of mode is that
it is the absolute's own reflective movement;
it is a determining by virtue of which
the absolute would become, not an other,
but what it already is;
a transparent externality
which is a pointing to itself;
a movement out of itself,
but in such a way that being outwardly is
just as much inwardness,
and consequently equally a positing
which is not mere positedness
but absolute being.

When therefore one asks for a content of the exposition,
for what the absolute manifests,
the reply is that the distinction of form and content
in the absolute has been dissolved;
or that just this is the content of the absolute,
that it manifests itself.
The absolute is the absolute form
which in its diremption of itself is
utterly identical with itself,
is the negative as negative
or the negative that rejoins itself
and in this way alone is the absolute self-identity
which equally is indifferent towards its distinctions
or is absolute content.
The content is therefore only this exposition itself.

As this self-bearing movement of exposition,
as a way and manner which is its absolute identity with itself,
the absolute is expression, not of an inner, nor over against an other,
but simply as absolute manifestation of itself for itself.
Thus it is actuality.

CHAPTER 2

Actuality

The absolute is the unity of inner and outer
as a first implicitly existent unit.
The exposition appeared as an external reflection
which, for its part, has the immediate
as something it has found,
but it equally is its movement
and the reference connecting it to the absolute
and, as such, it leads it back to the latter,
determining it as a mere “way and manner.”
But this “way and manner” is the
determination of the absolute itself,
namely its first identity
or its mere implicitly existent unity.
And through this reflection, not only is
that first in-itself posited as essenceless determination,
but, since the reflection is negative self-reference,
it is through it that the in-itself becomes
a mode in the first place.
It is this reflection that,
in sublating itself in its determinations
and as a movement which as such turns back upon itself,
is first truly absolute identity
and, at the same time, the determining of
the absolute or its modality.
The mode, therefore, is the externality of the absolute,
but equally so only its reflection into itself;
or again, it is the absolute's own manifestation,
so that this externalization is its immanent reflection
and therefore its being in-and-for-itself.

So, as the manifestation that it is nothing,
that it has no content, save to be
the manifestation of itself,
the absolute is absolute form.
Actuality is to be taken as
this reflected absoluteness.
Being is not yet actual;
it is the first immediacy;
its reflection is therefore becoming
and transition into an other;
or its immediacy is not being-in-and-for-itself.
Actuality also stands higher than concrete existence.
It is true that the latter is the immediacy
that has proceeded from ground and conditions,
or from essence and its reflection.
In itself or implicitly, it is therefore
what actuality is, real reflection;
but it is still not the posited unity of reflection and immediacy.
Hence concrete existence passes over into appearance
as it develops the reflection contained within it.
It is the ground that has foundered to the ground;
its determination, its vocation, is to restore this ground,
and therefore it becomes essential relation,
and its final reflection is that its
immediacy be posited as immanent reflection and conversely.
This unity, in which concrete existence
or immediacy and the in-itself,
the ground or the reflected, are simply moments,
is now actuality.
The actual is therefore manifestation.
It is not drawn into
the sphere of alteration by its externality,
nor is it the reflective shining of itself in an other.
It just manifests itself,
and this means that in its externality,
and only in it, it is itself, that is to say,
only as a self-differentiating and self-determining movement.

Now in actuality as this absolute form,
the moments only are as sublated or formal, not yet realized;
their differentiation thus belongs at first to external reflection
and is not determined as content.

Actuality, as itself immediate form-unity of inner and outer,
is thus in the determination of immediacy
as against the determination of immanent reflection;
or it is an actuality as against a possibility.
The connection of the two to each other is the third,
the actual determined both as being reflected into itself
and as this being immediately existing.
This third is necessity.

But first, since the actual and the possible
are formal distinctions,
their connection is likewise only formal,
and consists only in this,
that the one just like the other
is a positedness, or in contingency.

Second, because in contingency
the actual as well as the possible
are a positedness,
because they have retained their determination,
real actuality now arises,
and with it also real possibility
and relative necessity.

Third, the reflection of relative necessity
into itself yields absolute necessity,
which is absolute possibility and actuality.

IV.24
visesa-darsina atma-bhava-bhavana-vinivrtti

IV.25
tada viveka-nimnam kaivalya-prag-bharam cittam

IV.26
tad-chidresu pratyaya-antarani samskarebhya

IV.27
hanam esam klesavad uktam

A. CONTINGENCY OR FORMAL ACTUALITY, POSSIBILITY, AND NECESSITY

1. Actuality is formal inasmuch as, as a first actuality,
it is only immediate, unreflected actuality,
and hence is only in this form determination
but not as the totality of form.
And so it is nothing more than a being,
or concrete existence in general.
But because by essence it is not mere concrete existence
but is the form-unity of the in-itselfness
or inwardness and externality,
it immediately contains in-itselfness or possibility.
What is actual is possible.

2. This possibility is actuality reflected into itself.
But this reflectedness, itself a first, is equally something formal
and consequently only the determination of self-identity
or of the in-itself in general.

But because the determination is here totality of form,
this in-itself is determined as sublated
or essentially only with reference to actuality;
as the negative of actuality, it is posited as negative.
Possibility entails, therefore, two moments.
It has first the positive moment
of being a being-reflected-into-itself.
But this being-reflected-into-itself,
since in the absolute form it is reduced to a moment,
no longer has the value of essence but has rather
the negative meaning that possibility is (in a second moment)
something deficient, that it points to an other, to actuality,
and is completed in this other.

According to the first, merely positive side,
possibility is therefore the mere
form determination of self-identity,
or the form of essentiality.
As such it is the relationless, indeterminate
receptacle of everything in general.
In this formal sense of possibility,
everything is possible
that does not contradict itself;
the realm of possibility is therefore
limitless manifoldness.
But every manifold is determined in itself
and as against an other:
it possesses negation within.
Indifferent diversity passes over
as such into opposition;
but opposition is contradiction.
Therefore, all things are
just as much contradictory
and hence impossible.

When we therefore say of something
that “it is possible,”
this purely formal assertion is
just as superficial and empty
as the principle of contradiction,
and any content that we put into it,
“A is possible,” says no more than “A is A.”
Left undeveloped, this content has
the form of simplicity;
only after being resolved
into its determinations,
does difference emerge within it.
To the extent that we stop at that
simple for the content remains
something self-identical
and hence a possible.
But we do not say anything by it,
just as we do not with the principle of identity.

Yet the possible amounts to more than just the principle of identity.
The possible is reflected immanent reflectedness;
or the identical simply as a moment of the totality,
hence also as determined not to be in itself;
it therefore has the second determination of being only a possible
and the ought-to-be of the totality of form.
Without this ought-to-be, possibility is essentiality as such;
but the absolute form entails this,
that essence itself is only a moment
and that it has no truth without being.
Possibility is this mere essentiality,
but so posited as to be only a moment,
to be disproportionate with respect to the absolute form.
It is the in-itself, determined as only a posited
or, equally, as not to be in itself.

Internally, therefore, possibility is contradiction,
or it is impossibility.

This finds expression at first in this way,
that possibility as form determination
posited as sublated possesses a content in general.
As possible, this content is an in-itself
which is at the same time something sublated
or an otherness.
But because this content is only a possible,
an other opposite to it is equally possible.
“A is A”; then, too, “not-A is not-A.”
These two statements each express
the possibility of its content determination.
But, as identical statements,
they are indifferent to each other;
that the other is also added,
is not posited in either.
Possibility is the connection comparing the two;
as a reflection of the totality,
it implies that the opposite also is possible.
It is therefore the ground for drawing the connection that,
because A equals A, not-A also equals not-A;
entailed in the possible A there is also the possible not-A,
and it is this reference itself connecting them
which determines both as possible.

But this connection, in which
the one possible also contains its other,
is as such a contradiction that sublates itself.
Now, since it is determined to be reflective
and, as we have just seen, reflectively self-sublating,
it is also therefore an immediate
and it consequently becomes actuality.

3. This actuality is not the first actuality
but reflected actuality,
posited as unity of itself and possibility.
What is actual is as such possible;
it is in immediate positive identity with possibility;
but the latter has determined itself as only possibility;
consequently the actual is also determined as only a possible.
And because possibility is immediately contained in actuality,
it is immediately in it as sublated, as only possibility.
Conversely, actuality which is in unity with possibility
is only sublated immediacy;
or again, because formal actuality is only immediate first actuality,
it is only a moment, only sublated actuality, or only possibility.

With this we also have a more precise expression of
the extent to which possibility is actuality.
Possibility is not yet all actuality;
there has been no talk yet of real and absolute actuality.
It is still only the possibility as it first presented itself,
namely the formal possibility that has determined itself
as being only possibility and hence the formless actuality
which is only being or concrete existence in general.
Everything possible has therefore in general
a being or a concrete existence.

This unity of possibility and actuality is contingency.
The contingent is an actual which is at the same time
determined as only possible, an actual whose other or opposite equally is.
This actuality is, therefore, mere being or concrete existence,
but posited in its truth as having the value
of a positedness or a possibility.
Conversely, possibility is immanent reflection
or the in-itself posited as positedness;
what is possible is an actual in this sense of actuality,
that it has only as much value as contingent actuality;
it is itself something contingent.

The contingent thus presents these two sides.
First, in so far as it has possibility immediately in it,
or, what is the same, in so far as
this possibility is sublated in it,
it is not positedness, nor is it mediated,
but is immediate actuality; it has no ground.
Because this immediate actuality pertains also to the possible,
the latter is determined no less than the actual as contingent
and is likewise groundless.

But, second, the contingent is the actual
as what is only possible, or as a positedness;
thus the possible also, as formal in-itself, is only positedness.
Consequently, the two are both not in and for themselves
but have their immanent reflection in an other,
or they do have a ground.

The contingent thus has no ground because it is contingent;
and for that same reason it has a ground, because it is contingent.

It is the posited, immediate conversion of inner and outer,
or of immanently-reflected-being and being, each into the other posited,
because possibility and actuality both have this determination in them
by being moments of the absolute form.
So actuality, in its immediate unity with possibility,
is only concrete existence and is determined as groundless,
something only posited or only possible;
or, as reflected and determined over against possibility,
it is separated from possibility,
from immanent reflectedness, and then, too, is
no less immediately only a possible.
Likewise possibility, as simple in-itself, is something immediate,
only an existent in general;
or, opposed to actuality, it equally is an in-itself
without actuality, only a possible,
but, for that very reason, again only a concrete,
not immanently reflected, existence in general.

This absolute restlessness of the becoming of
these two determinations is contingency.
But for this reason, because each determination
immediately turns into the opposite,
in this opposite each equally rejoins itself,
and this identity of the two,
of each in the other,
is necessity.

The necessary is an actual;
as such it is immediate, groundless;
but it equally has its actuality
through an other or in its ground
and is at the same time the positedness of this ground
and its reflection into itself;
the possibility of the necessary is a sublated one.
The contingent is therefore necessary
because the actual is determined as a possible;
its immediacy is consequently sublated
and is repelled into the ground or the in-itself,
and into the grounded, equally because its possibility,
this ground-grounded-connection,
 is simply sublated and posited as being.
What is necessary is, and this existent is itself the necessary.
At the same time it is in itself;
this immanent reflection is an other than that immediacy of being,
and the necessity of the existent is an other.
Thus the existent is not the necessary;
but this in-itself is itself only positedness;
it is sublated and itself immediate.
And so actuality, in that from which it is distinguished,
in possibility, is identical with itself.
As this identity, it is necessity.

IV.28
prasankhyane api-akusidasya sarvatha viveka-khyater dharma-megha samadhi

IV.29
tata klesa-karma-nivrtti

IV.30
tada sarva-avarana-mala-apetasya jnanasya-anantya jneyam alpam

IV.31
tata-krta-arthanam parinama-krama-samaptir gunanam

B. RELATIVE NECESSITY OR REAL ACTUALITY, POSSIBILITY, AND NECESSITY

1. The necessity which has resulted is formal
because its moments are formal,
that is, simple determinations which are a totality
only as an immediate unity,
or as an immediate conversion of the one into the other,
and thus lack the shape of self-subsistence.
The unity in this formal necessity is therefore simple at first,
and indifferent to its differences.
As the immediate unity of the form determinations,
this necessity is actuality,
but an actuality which, since its unity is now determined as indifferent
to the difference of the form determinations, has a content.
This content as an indifferent identity contains the form
also as indifferent that is, as a mere variety of determinations,
and is a manifold content in general.
This actuality is real actuality.

Real actuality is as such at first
the thing of many properties,
the concretely existing world;
but it is not the concrete existence
that dissolves into appearance
but, as actuality, it is at the same time
an in-itself and immanent reflection;
it preserves itself in the manifoldness of mere concrete existence;
its externality is an inner relating only to itself.
What is actual can act;
something announces its actuality by what it produces.
Its relating to an other is the manifestation of itself,
and this manifestation is
neither a transition
(the immediate something refers to the other in this way)
nor an appearing
(in this way the thing only is in relation to an other);
it is a self-subsistent which has its immanent reflection,
its determinate essentiality, in another self-subsistent.

Now real actuality likewise has possibility immediately present in it.
It contains the moment of the in-itself;
but, since it is in the first instance only immediate unity,
it is in one of the determinations of form
and hence distinguished, as immediate existent,
from the in-itself or possibility.

2. This possibility, as the in-itself of real actuality,
is itself real possibility, at first the in-itself full of content.
Formal possibility is immanent reflection only as abstract identity,
the absence of contradiction in a something.
But when we delve into the determinations,
the circumstances, the conditions of a fact
in order to discover its possibility,
we do not stop at this formal possibility
but consider its real possibility.

This real possibility is itself immediate concrete existence,
but no longer because possibility as such, as a formal moment,
is immediately its opposite, a non-reflected actuality,
but because this determination pertains to it
by the very fact of being real possibility.
The real possibility of a fact is therefore
the immediately existent manifoldness of
circumstances that refer to it.

This manifoldness of existence is therefore indeed
both possibility and actuality,
but their identity is at first only the content
which is indifferent to these form determinations;
they therefore constitute the form,
determined as against their identity.
Or the immediate real actuality, because it is immediate,
is determined as against its possibility;
as this determinate and hence reflected actuality,
it is real possibility.
This real possibility is now indeed the posited whole of the form,
but of the form in the determinateness of actuality as formal
or immediate and equally of possibility as the abstract in-itself.
This actuality, therefore, which constitutes the possibility of a fact,
is not its own possibility but the in-itself of an other actual;
itself, it is the actuality that ought to be sublated,
the possibility as only possibility.
Real possibility thus constitutes the totality of conditions,
a dispersed actuality which is not reflected into itself
but is determined to be the in-itself of an other
and intended in this determination to return to itself.

What is really possible is, therefore,
something formally identical according to its in-itself,
free of contradiction because of its simple content determination;
but, as self-identical, this something must also not contradict
itself according to its developed and differentiated circumstances
and all else connected with it.
But, secondly, because it is manifold in itself
and in manifold connection with others,
and variety inherently passes over into opposition,
it is contradictory.
Whenever a possibility is in question,
and the issue is to demonstrate its contradiction,
one need only fasten on to the multiplicity that it contains as content
or as its conditioned concrete existence,
and from this the contradiction will easily be discovered.
And this contradiction is not just a function of comparing;
on the contrary, the manifold of concrete existence is in itself this,
to sublate itself and to founder to the ground:
in this it explicitly has the determination of
being only a possibility.
Whenever all the conditions of a fact are completely present,
the fact is actually there;
the completeness of the conditions is
the totality as in the content,
and the fact is itself this content determined
as being equally actual as possible.
In the sphere of the conditioned ground,
the conditions have the form
(that is, the ground or the reflection that stands on its own)
outside them,
and it is this form that makes them moments
of the fact and elicits concrete existence in them.
Here, on the contrary, the immediate actuality is
not determined to be condition by virtue of
a presupposing reflection,
but the supposition is rather that the immediate actuality is
itself the possibility.

In self-sublating real possibility,
it is a twofold that is now sublated;
for this possibility is itself
the twofold of actuality and possibility.
(1) The actuality is formal, or is a concrete existence
which appeared to subsist immediately,
and through its sublating becomes reflected being,
the moment of an other,
and thus comes in possession of the in-itself.
(2) That concrete existence was also determined
as possibility or as the in-itself, but of an other.
As it sublates itself, this in-itself of the other is
also sublated and passes over into actuality.
This movement of self-sublating real possibility
thus produces the same moments that are already present,
but each as it comes to be out of the other;
in this negation, therefore, the possibility
is also not a transition but a self-rejoining.
In formal possibility, if something was possible,
then an other than it, not itself, was also possible.
Real possibility no longer has such an other over against it,
for it is real in so far as it is itself also actuality.
Therefore, as its immediate concrete existence,
the circle of conditions, sublates itself,
it makes itself into the in-itselfness which it already is,
namely the in-itself of an other.
And conversely, since its moment of in-itselfness
thereby sublates itself at the same time,
it becomes actuality, hence the moment
which it likewise already is.
What disappears is consequently this,
that actuality was determined as the possibility
or the in-itself of an other,
and, conversely, the possibility as an actuality
which is not that of which it is the possibility.

3. The negation of real possibility is thus its self-identity;
inasmuch as in its sublating it is thus within itself
the recoiling of this sublating, it is real necessity.

What is necessary cannot be otherwise;
but what is only possible can be,
for possibility is the in-itself
which is only positedness
and hence essentially otherness.
Formal possibility is this identity
as transition into the other as such;
but real possibility, since it has
the other moment of actuality within it,
is already itself necessity.
Hence what is really possible can no longer be otherwise;
under the given conditions and circumstances,
nothing else can follow.
Real possibility and necessity are, therefore,
only apparently distinguished;
theirs is an identity that does not first come to be
but is already presupposed at their base.
Real possibility is therefore a connection full of content,
for the content is that identity, existing in itself,
which is indifferent to form.

But this necessity is at the same time relative.
For it has a presupposition from which it begins;
it takes its start from the contingent.
For the real actual is as such the determinate actual,
and first has its determinateness as immediate being
in that it is a multiplicity of concretely existing circumstances;
but this immediate being as determinateness is also the negative of
itself, is an in-itself or possibility and so real possibility.
As this unity of the two moments, it is the totality of form,
but a totality which is still external to itself;
it is the unity of possibility and actuality in such a way that
(1) the manifold concrete existence is possibility immediately or positively:
it is a possible, something self-identical as such, because it is an actual;
(2) inasmuch as this possibility of concrete existence is posited,
it is determined as only possibility,
as the immediate conversion of actuality into its opposite, or as contingency.
Hence this possibility which immediate actuality has within
in so far as it is condition, is only the in-itself
or the possibility of an other.
Because this in-itself, as shown, sublates itself and this positedness
is itself posited, real possibility becomes indeed necessity;
but this necessity thus begins from that unity of the possible and the actual
which is not yet reflected into itself;
this presupposing and the movement which turns back
unto itself are still separate;
or necessity has not yet determined itself
out of itself into contingency.

The relativity of real possibility is manifested in the content
by the fact that the latter is at first only
the identity indifferent to form,
is therefore distinct from it
and a determinate content in general.
A necessary reality is for this reason any limited actuality
which, because of its limitation, is in some other respect
also only something contingent.

In actual fact, therefore, real necessity is in itself also contingency.
This first becomes apparent because real necessity,
although something necessary according to form,
is still something limited according to content,
and derives its contingency through the latter.
But this contingency is to be found also
in the form of real necessity because, as shown,
real possibility is the necessary only in itself,
but as posited it is the mutual otherness of
actuality and possibility.
Real necessity thus contains contingency;
it is the turning back into itself from the restless
being-the-other-of-each-other of actuality and possibility,
but not the turning back from itself to itself.
In itself, therefore, we have here the unity
of necessity and contingency;
this unity is to be called absolute actuality.

IV.32
ksana-pratiyogi parinama-aparanta-nirgrahya krama

C. ABSOLUTE NECESSITY

Real necessity is determinate necessity;
formal necessity does not yet have any content and determinateness in it.
The determinateness of necessity consists in
its having its negation, contingency, within it.
This is how it has shown itself to be.

But in its first simplicity this determinateness is actuality;
determinate necessity is therefore immediate actual necessity.
This actuality which is itself as such necessary,
since it contains necessity as its in-itself, is absolute actuality;
an actuality which can no longer be otherwise,
or its in-itself is not possibility but necessity itself.

But because this actuality is posited to be absolute,
that is to say, to be itself the unity of itself and possibility,
it is consequently only an empty determination, or it is contingency.
This emptiness of its determination makes it into a mere possibility,
one which can just as well be an other and is determined as possibility.
But this possibility is itself absolute possibility,
for it is precisely the possibility of being
equally determined as possibility and actuality.
For this reason, because it is this indifference towards itself,
it is posited as empty, contingent determination.

Thus real necessity not only contains contingency implicitly,
but the latter also becomes in it;
but this becoming, as externality,
is itself only the in-itself of the necessity,
because it is only an immediate determinateness.
But it is not only this but the necessity's own becoming
or the presupposition which it had is its own positing.
For as real necessity, it is the sublatedness
of actuality into possibility and of possibility into actuality;
because it is this simple conversion of one of these moments into the other,
it is also their positive unity, for in the other each rejoins itself.
And so it is actuality, yet an actuality
which is nothing but this rejoining of form with itself.
Its negative positing of these moments is thereby itself the presupposing
or the positing of itself as sublated, or the positing of immediacy.

But it is precisely in this positing that
this actuality is determined as the negative;
it rejoins itself from the actuality which was real possibility;
this new actuality thus comes to be only out of its in-itself,
out of the negation of itself.
Consequently, it is at the same time immediately determined
as possibility, as mediated by virtue of its negation.
But accordingly, this possibility is immediately nothing but
this mediating in which the in-itself,
namely the possibility itself and the mediating,
both in the same manner, are positedness.
Thus it is necessity which is equally the sublating of this positedness,
or the positing of immediacy and of the in-itself,
just as in this very sublating it is the determining of it as positedness.
It is necessity itself, therefore, that determines itself as contingency:
in its being it repels itself from itself,
in this very repelling has only returned to itself,
and in this turning back which is its being has repelled itself from itself.

Thus has form pervaded in its realization all its distinctions;
it has made itself transparent and, as absolute necessity, is only this
simple self-identity of being in its negation, or in essence.
The distinction itself of content and form
has thus equally vanished;
for that unity of possibility in actuality
and actuality in possibility is
the form which in its determinateness
or in positedness is indifferent towards itself:
it is the fact full of content on which
the form of necessity externally ran its course.
But necessity is thus this reflected identity of
the two determinations as indifferent to them,
and hence the form determination of the in-itself
as against the positedness,
and this possibility constitutes the limitation of
the content which real necessity had.
The resolution of this difference is
however the absolute necessity
whose content is this difference
which in this necessity penetrates itself.

Absolute necessity is therefore the truth
in which actuality and possibility in general
as well as formal and real necessity return.
As we have just seen, it is being which in its negation,
in essence, refers itself to itself and is being.
It is equally simple immediacy or pure being
and simple immanent reflection or pure essence;
it is this, that the two are one and the same.
The absolutely necessary only is because it is;
it otherwise has neither condition nor ground.
But it equally is pure essence,
its being the simple immanent reflection;
it is because it is.
As reflection, it has a ground and a condition
but has only itself for this ground and condition.
It is in-itself, but its in-itself is its immediacy,
its possibility is its actuality.
It is, therefore, because it is;
as the rejoining of being with itself,
it is essence;
but because this simple is equally immediate simplicity,
it is being.

Absolute necessity is thus the reflection or form of
the absolute, the unity of being and essence,
simple immediacy which is absolute negativity.
On the one hand, therefore, its differences are not like
the determinations of reflection but an existing manifoldness,
a differentiated actuality in the shape of others
independently subsisting over against each other.
On the other hand, since its connection is
that of absolute identity,
it is the absolute conversion of
its actuality into its possibility
and its possibility into its actuality.
Absolute necessity is therefore blind.
On the one hand, the two different terms
determined as actuality and possibility have
the shape of immanent reflection as being;
they are therefore free actualities,
neither of which reflectively shines in the other,
nor will either allow in it a trace of its reference to the other;
grounded in itself, each is inherently necessary.
Necessity as essence is concealed in this being;
the reciprocal contact of these actualities appears
therefore, as an empty externality;
the actuality of the one in the other is
the possibility which is only possibility, contingency.
For being is posited as absolutely necessary,
as the self-mediation which is the absolute negation of
mediation-through-other,
or being which is identical only with being;
consequently, an other that has actuality in being,
is therefore determined as something merely possible,
as empty positedness.

But this contingency is rather absolute necessity;
it is the essence of those free, inherently necessary actualities.
This essence is averse to light, because there is
no reflective shining in these actualities,
no reflex: because they
are  grounded purely in themselves,
are shaped for themselves,
manifest themselves only to themselves;
because they are only being.
But their essence will break forth in them
and will reveal what it is and what they are.
The simplicity of their being, their resting just on themselves,
is absolute negativity; it is the freedom of their reflectionless immediacy.
This negative breaks forth in them because being,
through this same negativity which is its essence, is self-contradiction;
it will break forth against this being in the form of being,
hence as the negation of those actualities,
a negation absolutely different from their being;
it will break forth as their nothing, as an
otherness which is just as free towards them as their being is free.
Yet this negative was not to be missed in them.
In their self-based shape they are indifferent to form,
are a content and consequently different actualities and a
determinate content.
This content is the mark that necessity impressed upon them
by letting them go free as absolutely actual;
for in its determination it is an absolute turning back into itself.
It is the mark to which necessity appeals as witness to its right,
and, overcome by it, the actualities now perish.
This manifestation of what determinateness is in its truth,
that it is negative self-reference, is a blind collapse into otherness;
in the sphere of immediate existence, the shining
or the reflection that breaks out in it is a becoming,
a transition of being into nothing.
But, conversely, being is equally essence,
and becoming is reflection or a shining.
Thus the externality is its inwardness;
their connection is one of absolute identity;
and the transition of the actual into the possible,
of being into nothing, is a self-rejoining;
contingency is absolute necessity;
it is itself the presupposing of that
first absolute actuality.

This identity of being with itself in its negation is now substance.
It is this unity as in its negation or as in contingency;
and so, as relation to itself, it is substance.
The blind transition of necessity is
rather the absolute's own exposition,
its movement in itself which, in its externalization,
reveals itself instead.

CHAPTER 3

The absolute relation

IV.33
purusa-artha-sunyanam gunanam pratiprasava kaivalyam
svarupa-pratistha va citi-sakti iti

Absolute necessity is not so much the necessary,
even less a necessary, but necessity:
being simply as reflection.
It is relation
because it is a distinguishing
whose moments are themselves
the whole totality of necessity,
and therefore subsist absolutely,
but do so in such a way that
their subsisting is one subsistence,
and the difference only the reflective shine of
the movement of exposition,
and this reflective shine is the absolute itself.
Essence as such is reflection or a shining;
as absolute relation, however, essence is the
reflective shine posited as reflective shine,
one which, as such self-referring, is absolute actuality.
The absolute, first expounded by external reflection,
as absolute form or as necessity now expounds itself;
this self-exposition is its self-positing,
and is only this self-positing.
Just as the light of nature is not a something,
nor is it a thing, but its being is rather only its shining,
so manifestation is self-identical absolute actuality.

The sides of the absolute relation are not, therefore, attributes.
In the attribute the absolute reflectively shines only in one of its moments,
as in a presupposition that external reflection has simply assumed.
But the expositor of the absolute is the absolute necessity
which, as self-determining, is identical with itself.
Since this necessity is the reflective shining
posited as reflective shining, the sides of this relation,
because they are as shine, are totalities;
for as shine, the differences are themselves and their opposite,
that is, they are the whole;
and, conversely, they thus are only shine because they are totalities.
Thus this distinguishing, this reflecting shining of the absolute,
is only the identical positing of itself.

This relation in its immediate concept is
the relation of substance and accidents,
the immediate internal disappearing and becoming
of the absolute reflective shine.
If substance determines itself as a being-for-itself over
against an other or is absolute relation as something real,
then we have the relation of causality.
Finally, when this last relation passes over into
reciprocal causality by referring itself to itself,
we then have the absolute relation also posited
in accordance with the determination it contains;
this posited unity of itself in its determinations,
which are posited as the whole itself
and consequently equally as determinations,
is then the concept.
