A. THE UNIVERSAL CONCEPT

The pure concept is the absolutely infinite,
unconditioned and free.
It is here, as the content of our treatise
begins to be the concept itself,
that we must look back once more at its genesis.
Essence came to be out of being,
and the concept out of essence,
therefore also from being.
But this becoming has the
meaning of a self-repulsion,
so that what becomes is rather
the unconditional and the originative.
In passing over into essence,
being became a reflective shine or a positedness,
and becoming or the passing over into an other
became a positing;
conversely, the positing or the reflection of essence
sublated itself and restored itself to
a non-posited, an original being.
The concept is the mutual penetration of these moments,
namely, the qualitative and the originative existent is
only as positing and as immanent turning back,
and this pure immanent reflection simply is
the becoming-other or determinateness
which is, consequently, no less
infinite, self-referring determinateness.

Thus the concept is absolute self-identity
by being first just this, the negation of negation
or the infinite unity of negativity with itself.
This pure self-reference of the concept,
which is such by positing itself through the negativity,
is the universality of the concept.
Universality seems incapable of explanation,
because it is the simplest of determinations;
explanation must rely on determinations and differentiations
and must apply predicates to its subject matter,
and this would alter rather than explain the simple.
But it is precisely of the nature of
the universal to be a simple that,
by virtue of absolute negativity,
contains difference and determinateness
in itself in the highest degree.
Being is simple as an immediate;
for this reason we can only intend it
without being able to say what it is;
therefore, it is immediately one
with its other, non-being.
The concept of being is just this,
that it is so simple as to vanish
into its opposite immediately;
it is becoming.
The universal is, on the contrary,
a simple that is at the same time
all the richer in itself,
for it is the concept.
First, therefore, it is simple self-reference;
it is only in itself.
But, second, this identity is
in itself absolute mediation
but not anything mediated.
Of the universal which is mediated,
that is to say, the abstract universal,
the one opposed to the particular and the singular,
of that we shall have to speak only in connection
with the determinate concept.
Yet, even the abstract universal entails this much,
that in order to obtain it there is required
the leaving aside of other
determinations of the concrete.
As determinations in general,
these determination are negations,
and leaving them aside is a further negating.
Even in the abstract universal, therefore,
the negation of negation is already present.
But this double negation comes to be represented
as if it were external to it,
both as if the properties of the concrete
that are left out were different
from the ones that are retained
as the content of the abstraction,
and as if this operation of leaving some aside
while retaining the rest went on outside them.
With respect to this movement,
the universal has not yet acquired
the determination of externality;
it is still in itself that absolute negation
which is, precisely, the negation of negation
or absolute negativity.

Accordingly, because of this original unity,
the first negative, or the determination, is not,
to begin with, a restriction for the universal;
rather, the latter maintains itself in it
and its self-identity is positive.
The categories of being were, as concepts,
essentially these identities of the determinations
with themselves in their restriction or their otherness;
but this identity was only implicitly the concept,
was not yet made manifest.
Consequently, the qualitative determination
perished as such in its other
and had as its truth
a determination diverse from it.
The universal, on the contrary,
even when it posits itself in a determination,
remains in it what it is.
It is the soul of the concrete which it inhabits,
unhindered and equal to itself
in its manifoldness and diversity.
It is not swept away in the becoming
but persists undisturbed through it,
endowed with the power of unalterable,
undying self-preservation.

It also does not simply shine reflectively in its other,
as does the determination of reflection.
This determination, as something relative,
does not refer only to itself but is a relating.
It lets itself be known in its other,
but at first it only shines reflectively in it,
and this reflective shining of each in the other,
or their reciprocal determination,
has the form of an external activity
alongside their self-subsistence.
The universal is posited, on the contrary,
as the essence of its determination,
as this determination's own positive nature.
For the determination that constitutes the negative of the
universal is in the concept simply and solely a positedness;
essentially, in other words,
it is at the same time the negative of the negative,
and only is as this self-identity of the negative
which is the universal.
To this extent, the universal is also
the substance of its determinations,
but in such a way that
what for the substance as such was an accident,
is the concept's own self mediation,
its own immanent reflection.
But this mediation,
which first raises the accidental to necessity,
is the manifested reference;
the concept is not the abyss of formless substance,
or the necessity which is the inner identity of things
or circumstances different from each other
and reciprocally constricting;
rather, as absolute negativity,
it is the informing and creative principle,
and since the determination is not as limitation
but is just as much simply sublated as determination,
is positedness, so is the reflective shine
the appearance as appearance of the identical.

The universal is therefore free power;
it is itself while reaching out
to its other and embracing it,
but without doing violence to it;
on the contrary, it is at rest
in its other as in its own.
Just as it has been called free power,
it could also be called free love
and boundless blessedness,
for it relates to that
which is distinct from it as to itself;
in it, it has returned to itself.
Mention has just been made of determinateness,
even though the concept has not yet progressed to it,
being at first only as the universal
and only self-identical.
But one cannot speak of the universal
apart from determinateness
which, to be more precise,
is particularity and singularity.
For in its absolute negativity
the universal contains determinateness
in and for itself, so that,
when speaking of determinateness
in connection with the universal,
the determinateness is not being
imported into the latter from outside.
As negativity in general, that is,
according to the first immediate negation,
the universal has determinateness in it
above all as particularity;
as a second universal, as the negation of negation,
it is absolute determinateness, that is,
singularity and concreteness.
The universal is thus the totality of the concept;
it is what is concrete, is not empty
but, on the contrary, has content
by virtue of its concept;
a content in which the universal
does not just preserve itself
but is rather the universal's own,
immanent to it.
It is of course possible to abstract from this content,
but what we have then is not
the universal element of the concept
but the abstract universal,
which is an isolated and imperfect
moment of the concept, void of truth.

More precisely, the universal shows itself
to be this totality as follows.
In so far as the universal possesses determinateness,
this determinateness is not only the first negation
but also the reflection of this negation into itself.
According to that first negation, taken by itself,
the universal is a particular,
and in this guise we shall consider it in a moment.
In the other determinateness, however,
the universal is still essentially universal,
and this side we have here still to consider.
For this determinateness, as it is in the concept,
is the total reflection a doubly reflective shine,
both outwards, as reflection into the other,
and inwards, as reflection into itself.
The outward shining establishes a distinction
with respect to an other;
the universal accordingly takes on a particularity
which is resolved in a higher universality.
Inasmuch as it now is also only a relative universal,
it does not lose its character of universality;
it preserves itself in its determinateness,
not just because it remains indifferent to it
for then it would be only posited together with it
but because of what has just been called the inward shining.
The determinateness, as determinate concept,
is bent back into itself;
it is the concept's own immanent character,
a character made essential by
being taken up into the universality
and by being pervaded by it,
just as it pervades it in turn
equal in extension and identical with it.
This is the character that belongs to the genus
as the determinateness which is not
separated from the universal.
To this extent, it is not an outwardly directed limitation,
but is positive, for by virtue of the universality
it stands in free self-reference.
Thus even the determinate concept remains
in itself infinitely free concept.

But in regard to the other side
in which the genus is limited
because of its determinate character,
we have just said that, as a lower genus,
it has its resolution in a higher universal.
This universal can also be grasped as a genus
but as a more abstract one;
it always pertains, however, only to
the side of the determinate concept
which is outwardly directed.
The truly higher universal is the one in which
this outwardly directed side is redirected inwardly;
this is the second negation in which
the determinateness is present simply and solely
as something posited, or as reflective shine.
Life, the “I,” spirit, absolute concept,
are not universals only as higher genera,
but are rather concretes whose determinacies are
also not mere species or lower genera
but determinacies which, in their reality,
are self-contained and self-complete.
Of course, life, the “I,” finite spirit,
are also only determinate concepts.
To this extent, however, their resolution is
in a universal which, as the truly absolute concept,
is to be grasped as the idea of infinite spirit,
the spirit whose posited being is
the infinite, transparent reality
in which it contemplates its creation
and, in this creation, itself.

The true, infinite universal,
the one which, immediately in itself,
is just as much particularity as singularity,
is now to be more closely examined as particularity.
It determines itself freely;
the process by which it becomes finite
is not a transition,
the kind that occurs only
in the sphere of being;
it is creative power as
self-referring absolute negativity.
As such, it differentiates itself internally,
and this is a determining,
because the differentiating is
one with the universality.
Accordingly, it is a positing of differences
that are themselves universals, self-referring.
They become thereby fixed, isolated differences.
The isolated subsistence of the finite
that was earlier determined as its being-for-itself,
also as thinghood, as substance,
is in its truth universality,
the form with which the infinite concept
clothes its differences,
a form which is equally itself one of its differences.
Herein consists the creativity of the concept,
a creativity which is to be comprehended
only in the concept's innermost core.

B. THE PARTICULAR CONCEPT

Determinateness as such belongs to being and the qualitative;
as the determinateness of the concept, it is particularity.
It is not a limit, as if it were related to an other beyond it,
but is rather, as just shown, the universal's own immanent moment;
in particularity, therefore, the universal is not
in an other but simply and solely with itself.

The particular contains the universality
that constitutes its substance;
the genus is unaltered in its species;
these do not differ from the universal
but only from each other.
The particular has one and the same universality as
the other particulars to which it is related.
The diversity of these particulars,
because of their identity with the universal, is
as such at the same time universal;
it is totality.
The particular, therefore,
does not only contain the universal
but exhibits it also through its determinateness;
accordingly the universal constitutes a sphere
that the particular must exhaust.
This totality, inasmuch as the determinateness
of the particular is taken as mere diversity,
appears as completeness.
In this respect, the species are complete
simply in so far as there are no more of them.
There is no inner standard or principle available for them,
for their diversity is just the dispersed difference
for which the universality,
which is for itself absolute unity,
is a merely external reflex
and an unconstrained, contingent completeness.
But diversity passes over into opposition,
into an immanent connection of diverse moments.
Particularity, however,
because it is universality,
is this immanent connection,
not by virtue of a transition,
but in and for itself.
It is totality intrinsically,
and simple determinateness,
essential principle.
It has no other determinateness than
that posited by the universal itself
and resulting from it in the following manner.

The particular is the universal itself,
but it is its difference or reference to an other,
its outwardly reflecting shine;
but there is no other at hand from which
the particular would be differentiated
than the universal itself.
The universal determines itself,
and so is itself the particular;
the determinateness is its difference;
it is only differentiated from itself.
Its species are therefore
only (a) the universal itself
and (b) the particular.
The universal is as concept itself and its opposite,
and this opposite is in turn the universal itself
as its posited determinateness;
the universal overreaches it
and, in it, it is with itself.
Thus it is the totality and the principle of its diversity,
which is determined wholly and solely through itself.

There is, therefore, no other
true logical division than this,
that the concept sets itself on one side
as the immediate, indeterminate universality;
it is this very indeterminateness
that makes its determinateness,
or that it is a particular.
The two are both a particular
and are therefore coordinated.
Both, as particular, are also
determinate as against the universal,
and in this sense they are subordinated to it.
But even this universal,
as against which the particular is determined,
is for that reason itself also
just one of the opposing sides.
When we speak of two opposing sides,
we must repeat that the two constitute the particular,
not just together, as if they were alike
in being particular only for external reflection,
but because their determinateness over against each other is
at the same time essentially only one determinateness;
it is the negativity which in the universal is simple.

Difference, as it presents itself here,
is in its concept and therefore in its truth.
All previous difference has this unity in the concept.
As it is present immediately in being,
difference is the limit of an other;
as present in reflection, it is relative,
posited as referring essentially to its other;
here is where the unity of the concept
thus begins to be posited;
at first, however, the unity is only
a reflective shine in an other.
The true significance of the transitoriness
and the dissolution of these determinations is just this,
that they attain to their concept, to their truth;
being, existence, something,
or whole and part, and so on,
substance and accidents, cause and effect,
are thought determinations on their own;
as determinate concepts, however,
they are grasped in so far as each
is cognized in unity with its others
or in opposition to them.
Whole and parts, for example,
or cause and effect, and so on,
are not yet diverse terms
that are determined as particular
relatively to each other,
for although they implicitly constitute one concept,
their unity has not yet attained the form of universality;
thus the difference as well which is in these relations,
does not yet have the form of being one determinateness.
Cause and effect, for example, are not two diverse concepts
but only one determinate concept,
and causality is, like every concept,
a simple concept.

With respect to completeness,
we have just seen that the determinate moment
of  particularity is complete in the difference
of the universal and the particular,
and that only these two make up the particular species.
To be sure, there are more than two species
to be found in any genus in nature,
and these many species cannot stand
in the same relation to each other
as we have shown.
This is the impotence of nature,
that it cannot abide by and exhibit the rigor of the concept
and loses itself in a blind manifoldness void of concept.
We can wonder at nature,
at the manifoldness of its genera and species,
in the infinite diversity of its shapes,
for wonder is without concept
and its object is the irrational.
It is allowed to nature,
since nature is the self-externality of the concept,
to indulge in this diversity,
just as spirit, even though it possesses
the concept in the shape of concept,
lets itself go into pictorial representation
and runs wild in the infinite manifoldness of the latter.
The manifold genera and species of nature must not be
esteemed to be anything more than arbitrary notions of spirit
engaged in pictorial representations.
Both indeed show traces and intimations of the concept,
but they do not exhibit it in trustworthy copy,
for they are the sides of its free self-externality;
the concept is the absolute power precisely
because it can let its difference go free
in the shape of self-subsistent diversity,
external necessity, accidentality, arbitrariness, opinion;
all of which, however, must not be taken as anything
more than the abstract side of nothingness.

As we have just seen, the determinateness of
the particular is simple as principle,
but it is also simple as a moment of the totality,
determinateness as against the other determinateness.
The concept, in determining or differentiating itself,
behaves negatively towards its unity
and gives itself the form of one
of its ideal moments of being;
as a determinate concept,
it has a determinate existence in general.
But this being no longer has
the significance of mere immediacy,
but has the significance rather of an immediacy
which is equal to itself by virtue of absolute mediation,
an immediacy that equally contains in itself
the other moment of essence or of reflection.
This universality, with which the determinate clothes itself,
is abstract universality.
The particular has this universality in it as its essence;
but in so far as the determinateness
of the difference is posited
and thereby has being,
the universality is form in it,
and the determinateness as such is its content.
Universality becomes form inasmuch as
the difference is something essential,
just as in the pure universal it is, on the contrary,
only absolute negativity
and not a difference posited as such.

Now the determinateness is indeed an abstraction,
as against the other determinateness;
but the other determinateness is only universality itself,
and this too is therefore abstract universality;
and the determinateness of the concept, or particularity,
is again nothing more than determinate universality.
In this universality, the concept is outside itself,
and because it is it, the concept,
which is there outside itself,
the abstract-universal contains
all the moments of the concept.
It is
(a) universality,
(b) determinateness,
(c) the simple unity of the two;
but this unity is immediate,
and the particularity is not
therefore as totality.
Implicitly it is this totality also, and mediation;
it is essentially a reference to the other excluding it,
or the sublation of negation,
namely of the other determinateness
an other that lingers on only as an intention,
for it vanishes immediately revealing itself to be
the same as its other is supposed to be.
Therefore, what makes this universality an abstraction
is that the mediation is only a condition,
or is not posited in it.
Because it is not posited,
the unity of the abstraction has the form of immediacy,
and the content has the form of indifference to its universality,
for the content is nothing but this totality
which is the universality of absolute negativity.
Hence the abstract universal is indeed the concept,
but the unconceptualized concept,
the concept not posited as such.

When we speak of the determinate concept,
what we ordinarily mean is
precisely just this abstract universal.
Even by concept as such,
what is generally understood is
only this unconceptualized concept,
and the understanding is designated as its faculty.
Demonstration belongs to this understanding
inasmuch as it proceeds by way of concepts,
that is to say, only in determinations.
This progression by way of concepts does not therefore
reach past finitude and necessity;
the highest it reaches is the negative infinite,
the abstraction of the highest essence
which is itself the determinateness
of the indeterminateness.
Absolute substance, too, although not this empty abstraction
but on the contrary a totality according to content,
is still abstract, for since it is without absolute form,
its innermost truth is not constituted by the concept;
although it is the identity of universality and particularity,
or of thought and externality,
this identity is not the determinateness of the concept;
there is rather an understanding outside it,
an understanding which is contingent precisely
because it is outside it in which and for which
substance exists in diverse attributes and modes.
Moreover, abstraction is not as empty as
it is usually said to be;
it is the determinate concept;
it has some determinateness or other for its content;
the highest essence also, the pure abstraction,
has the determinateness of indeterminateness,
as just mentioned;
but indeterminateness is a determinateness
because it is supposed to stand opposite the determinate.
But the moment one says what it is,
its intended meaning sublates itself by itself;
for it is spoken of on a par with determinateness,
and from this abstraction the concept and its truth are brought out.
To be sure, any determinate concept is empty
in so far as it does not contain the totality,
but only a one-sided determinateness.
Even when it has otherwise concrete content
such as, for instance, humankind, the state, animal, etc.,
it remains an empty concept inasmuch as its determinateness is
not the principle of its differentiation;
the principle contains the beginning and the essence
of its development and realization;
any other determinateness of the concept is however otiose.
To reproach the concept as such for being empty is
to ignore its absolute determinateness
which is the difference of the concept
and the only true content in the element of the concept.

Here we have the circumstance that explains
why the understanding is nowadays held in such a low repute
and is so much discredited when measured against reason;
it is the fixity which it imparts to determinacies
and consequently to anything finite.
This fixity consists in the form of the abstract universality
just considered that makes them unalterable.
For qualitative determinateness,
and also the determination of reflection,
are essentially limited,
and because of their limitation they entail a reference
to their other;
hence the necessity of their transition and passing away.
But the universality which they possess in the understanding
gives them the form of immanent reflection
and, because this form removes from them
the reference to the other,
they have become unalterable.
Now although this eternity belongs
to the pure concept by nature,
the determinations of the concept are
eternal essentialities only according to form;
but their content is not proportionate to this form
and, therefore, they are not truth, or imperishable.
Their content is not proportionate to the form
because it is not the determinateness itself as universal,
that is, not as totality of the difference of the concept,
or is not itself the whole form;
the form of the limited understanding is
for this reason itself imperfect universality,
that is to say, abstract universality.
But further, we must pay due respect to the
infinite force of the understanding
in splitting the concrete into abstract determinacies
and plumbing the depth of the difference,
this force which alone is at the same time
the mighty power causing the transition of the determinacies.
The concrete of intuition is a totality,
but a sensuous totality,
a real material that subsists in space and time,
part outside part, each indifferent to the other;
surely this lack of unity in a manifold that
makes it the content of intuition ought not to be credited
as privileging it over the universal of the understanding.
The mutability that the manifold exhibits
in intuition already points to the universal;
but all of the manifold that comes to intuition is
just more of the same, an equally alterable other,
not the universal that one would expect to appear and take its place.
But least of all should we reckon to the credit of such sciences
as for example Geometry and Arithmetic
that their material carries an intuitive element with it,
or imagine that their propositions are grounded by it.
On the contrary, the presence of that element renders
the material of these sciences of an inferior nature;
the intuition of figures or numbers is
of no help to the science of figures and numbers;
only the thought of them produces this science.
But if by intuition we understand
not merely a sensuous material
but the objective totality,
then the intuition is an intellectual one, that is,
its subject matter is not existence in its externalization
but that element in existence which is
unalterable reality and truth;
the reality only in so far as it is essentially
in the concept and is determined by it;
the idea, of whose more precise nature more will be said later.
What intuition as such is supposed to have
over the concept is external reality,
the reality that lacks the concept
and receives value only through the concept.

Consequently, since the understanding exhibits
the infinite force that determines the universal,
or conversely, since it is the understanding
that through the form of universality
imparts stable subsistence to the otherwise
inherent instability of determinateness,
then it is not the fault of the understanding
if there is no further advance.
It is a subjective impotence of reason
that allows these determinacies to remain so dispersed,
and is unable to bring them back to their unity
through the dialectical force opposed
to that abstract universality,
that is to say, through the determinacies'
own nature which is their concept.
To be sure, the understanding does give them
through the form of abstract universality a rigidity of being,
so to speak, which they do not otherwise possess
in the qualitative sphere and in the sphere of reflection;
but by thus simplifying them, the understanding
at the same time quickens them with spirit,
and it so sharpens them that only at that point,
only there, do they also obtain the capacity
to dissolve themselves and to pass over into their opposite.
The ripest maturity, the highest stage,
that anything can attain is
the one at which its fall begins.
The fixity of the determinacies
which the understanding appears to run up against,
the form of the imperishable,
is that of self-referring universality.
But this universality belongs to the concept as its own,
and for this reason what is found expressed in it,
infinitely close at hand,
is the dissolution of the finite.
This universality directly contradicts
the determinateness of the finite
and makes explicit its disproportion
with respect to it.
Or rather, that disproportion is already at hand;
the abstract determinate is posited as
one with universality
and, for this reason, not for itself
(for it would then be only a determinate)
but, on the contrary,
only as the unity of itself and the universal,
that is, as concept.

Therefore the common practice
of separating understanding and reason
is to be rejected on all counts.
On the contrary, to consider the concept
as void of reason should itself be
considered as an incapacity of reason
to recognize itself in the concept.
The determinate and abstract concept is the condition,
or rather an essential moment, of reason;
it is form quickened by spirit in which the finite,
through the universality in which it refers to itself,
is internally kindled, is posited as dialectical
and thereby is the beginning of the appearance of reason.

Since in the foregoing the determinate concept
has been presented in its truth,
it is only left to indicate what, as so presented,
it has already been posited as.
Difference, which is an essential moment of the concept
but in the pure universal is not yet posited as such,
receives its due in the determinate concept.
Determinateness in the form of universality is
united with the latter to form a simple;
determinate universality is self-referring determinateness,
determinate determinateness,
or absolute negativity posited for itself.
But self-referring determinateness is singularity.
Just as universality immediately is
particularity in and for itself,
no less immediately is particularity also
singularity in and for itself;
this singularity is at first to be regarded as
the third moment of the concept,
inasmuch as it is held fast
in opposition to the other two,
but also as the absolute turning back
of the concept into itself,
and at the same time
as the posited loss of itself.

C. THE SINGULAR

Singularity, as we have seen, is
already posited through particularity;
this is determinate universality
and hence self-referring determinateness,
the determinate determinate.

1. At first, therefore, singularity appears
as the reflection of the concept
out of its determinateness into itself.
It is the concept's self-mediation by virtue of which,
since its otherness has once more been made into an other,
it restores itself as self-equal,
but in the determination of absolute negativity.
The negative in the universal,
by virtue of which this universal is a particular,
was earlier determined as a doubly reflective shine.
In so far as the reflective shining is inward,
the particular remains a universal;
through the outward shining,
it is a determinate particular;
the turning back of this side
into the universal is twofold,
either by virtue of an abstraction
that lets the particular fall away
and climbs to a higher and the highest genus,
or by virtue of the singularity
to which the universality in
the determinateness itself descends.
Here is where the false start is made
that makes abstraction stray away
from the way of the concept,
abandoning the truth.
Its higher and highest universal
to which it rises is  only a surface
that becomes progressively more void of content;
the singularity which it scorns is
the depth in which the concept grasps itself
and where it is posited as concept.

Universality and particularity appeared, on the one hand,
as moments of the becoming of singularity.
But it has already been shown that the two are
in themselves the total concept;
consequently, that in singularity they do
not pass over into an other
but that, on the contrary,
what is posited in it is
what they are in and for themselves.
The universal is for itself because it
is absolute mediation in itself,
self-reference only as absolute negativity.
It is an abstract universal
inasmuch as this sublating is an external act
and so a dropping off of the determinateness.
This negativity, therefore, attaches indeed
to the abstract universal,
but it remains outside it,
as a mere condition of it;
it is the abstraction itself
that holds its universal opposite it,
and so the universal does not
have singularity in itself
and remains void of concept.
Life, spirit, God, as well as the pure concept,
are for this reason beyond the grasp of abstraction,
for abstraction keeps singularity away from its products,
and singularity is the principle of
individuality and personality.
And so it comes to nothing but lifeless universalities,
void of spirit, color, and content.

But the unity of the concept is so indissoluble
that these products of abstraction also,
though they are supposed to drop singularity,
are rather themselves singulars.
For in elevating the concrete to universality,
abstraction grasps the universal
as only a determinate universality,
and this is precisely the singularity
that presented itself as self-referring determinateness.
Thus abstraction is a partitioning of the concrete
and an isolating of its determinations;
only singular properties or moments are picked out by it,
for its product must contain what it itself is.
But the difference between this
singularity of its products
and the singularity of the concept is
that in the former the singular and the universal differ
from each other as content and form respectively,
precisely because the content is not the absolute form,
is not the concept itself,
or this form is not the totality of form.
However, this closer consideration shows
that the product of abstraction is
itself the unity of the singular content
and of abstract universality,
therefore that it is something concrete,
the opposite of what it is supposed to be.

The particular, for the same reason that
makes it only a determinate universal,
is also a singular, and conversely,
because the singular is a determinate universal,
it is equally a particular.
If we stay at this abstract determinateness,
then the concept has the
three particular determinations of
universal, particular, and singular,
whereas earlier we gave only
the universal and the particular as
species of the particular.
Because singularity is the turning of
the concept as a negative back to itself,
this turning back from abstraction,
which in the turning is truly sublated,
can itself be placed as an indifferent
moment alongside the others
and be counted with them.

If singularity is listed as one of the
particular determinations of the concept,
then particularity is the totality
which embraces them all
and, precisely as this totality,
it is the concretion of the determinations
or singularity itself.
But it is a concrete also according
to the previously mentioned side,
as determinate universality;
and then it is the immediate unity
in which none of these moments is posited
as distinct or as the determinant,
and in this form it will constitute
the middle term of the formal syllogism.

It follows that each of the determinations
established in the preceding exposition of the concept
has immediately dissolved itself
and has lost itself in its other.
Each distinction is confounded
in the course of the very reflection
that should isolate it and hold it fixed.
Only a way of thinking that is merely representational,
for which abstraction has isolated them,
is capable of holding the universal, the particular,
and the singular rigidly apart.
Then they can be counted;
and for a further distinction
this representation relies on one
which is entirely external to being,
on their quantity,
and nowhere is such a distinction as inappropriate as here.
In singularity, the earlier true relation,
the inseparability of the determinations of the concept,
is posited;
for as the negation of negation,
singularity contains the opposition
of those determinations
and this opposition itself at its ground
or the unity where the determinations have
come together, each in the other.
Because in this reflection
universality is in and for itself,
singularity is essentially the negativity of
the determinations of the concept,
but not merely as if it stood as
a third something distinct from them,
but because what is now posited is
that positedness is being-in-and-for-itself;
that is, what is posited is that
each of the distinct determinations is the totality.
The turning back of the determinate concept into itself
means that its determination is to be
in its determinateness the whole concept.

2. Singularity is not, however, only the turning back
of the concept into itself, but the immediate loss of it.
Through singularity, where it is internal to itself,
the concept becomes external to itself and steps into actuality.
Abstraction, which is the soul of singularity
and so the self-reference of the negative,
is, as we have seen, nothing external to
the universal and the particular
but is immanent in them,
and these are concreted through it,
they become a content, a singular.
But, as this negativity, singularity is
the determinate determinateness, differentiation as such,
and through this reflection of the difference into itself,
the difference becomes fixed;
the determining of the particular occurs
only by virtue of singularity,
for singularity is that abstraction which,
precisely as singularity,
is now posited abstraction.

The singular, therefore, is as self-referring negativity
the immediate identity of the negative with itself;
it exists for itself.
Or it is the abstraction determining
the concept as an immediate,
according to its ideal moment of being.
Thus the singular is a one which is qualitative, or a this.
In accordance with this qualitative character, it is,
first, the repulsion of itself from itself
by virtue of which many other ones are presupposed;
second, it is now a negative reference
with respect to these presupposed others,
and to this extent the singular is exclusive.
Universality, when referred to
these singulars as indifferent ones
and it must be referred to them,
for they are a moment of the concept of singularity,
is only their commonality.
If by the universal one understands
that which is common to several singulars,
the indifferent subsistence of these singulars is
then taken as the starting point,
thus mixing in the immediacy of being
into the determination of the concept.
The lowest conception one can have of
the universal as connected with the singular
is this external relation that it has
to the latter as a mere commonality.

The singular, which in the reflective sphere
of concrete existence is as a this,
does not have the excluding reference
to the other that is characteristic
of qualitative being-for-itself.
A this is a one reflected into itself, without repulsion;
or the repulsion is in this reflection one with abstraction,
a reflective mediation present in the this
that makes it a posited immediacy
pointed at by someone external to it.
The this is; it is immediate, it is a this,
however, only in so far as it is pointed at.
This “pointing at” is the reflective movement
that takes hold of itself and posits the immediacy,
but as something external to itself.
Now the singular surely is also a this,
as an immediate which is the result of mediation,
but does not have this mediation outside it;
it is itself repelling separation, posited abstraction,
yet is, precisely in its separation, a positive connection.

This act of abstraction by the singular is,
as the immanent reflection of difference,
the first positing of the differences as self-subsisting,
reflected into themselves.
They exist immediately;
but, further, this separating is reflection in general,
the reflective shining of one in the other;
the differences thus stand in essential relation.
They are, moreover, not singulars that
just exist next to each other;
a plurality of this kind belongs to being;
the singularity that posits itself as determinate
does not posit itself in an external difference
but in a difference of the concept;
singularity thus excludes the universal from itself,
but since this universal is a moment of it,
it refers to it just as essentially.

The concept, as this connection of
its self-subsistent determinations,
has lost itself,
for the concept itself is no longer
the posited unity of these determinations,
and these no longer are moments,
the reflective shining of the concept,
but subsist rather in and for themselves.
As singularity, the concept returns
in determinateness into itself,
and therewith the determinate has
itself become totality.
The concept's turning back into itself is
thus the absolute, originative partition of itself,
that is, as singularity it is posited as judgment.
