THE DOCTRINE OF ESSENCE

A. REFLECTION

IV.1
jatyantara-parinama prakrtya-apurat

IV.2
nimittam aprayojakam prakrtinam varana-bhedas tu tata ksetrikavat

IV.3
nirmana-cittani-asmita-matra

IV.4
pravrtti-bhede prayojakam cittam ekam anekesam

IV.5
tatra dhyana-jam anasayam

IV.6
karma-asukla-akrsnam yoginas trividham itaresam

IV.7
tatas tad-vipaka-anugunanam evabhivyaktir vasananam

IV.8
jati-desa-kala vyavahitanam apyanantaryam smrti-samskarayo eka-rupatvat

IV.9
tasam anaditvam caasiso nityatvat

IV.10
hetu-phala-asraya-alambana samgehitatvad esam abhave tad-abhava


IV.1
jatyantara-parinama prakrtya-apurat

The essence is the concept insofar as it is simply posited;
in the essence, the determinations are only relative,
they are not yet fully reflected in themselves.
For this reason, the concept is not yet for itself.

As being that mediates itself with itself in virtue of its negativity,
essence is relation to itself only insofar as
it is relation to an other that is, however, not immediately a being,
but something posited and mediated.

Being has not disappeared;
instead, in the first place,
the essence, as a simple relation to itself, is being;
in the second place, moreover,
in keeping with being's one-sided determination as something immediate,
being has been demoted to something merely negative, to a shine.
The essence is accordingly being as shining in itself.

    The absolute is essence.

    This definition is the same as the definition that it is being,
    insofar as being is also the simple relation to itself;
    but at the same time it is higher since the essence is
    being that has gone into itself, that is to say,
    its simple relation to itself is this relation,
    posited as the negation of the negative,
    as mediation of itself in itself with itself.
    However, when the absolute is determined as essence,
    negativity is frequently taken only in the sense of
    an abstraction from all determinate predicates.
    This negative act, the abstracting, then falls outside of the essence and
    the essence itself is thus only a result without these, its premises,
    the caput mortuum of abstraction.
    But since this negativity is not external to being,
    but instead is its own dialectic,
    then its truth, the essence,
    is the being that has gone into itself or is in itself,
    that reflection, its process of shining in itself,
    constitutes its difference from immediate being and
    is the distinctive determination of the essence itself.

IV.2
nimittam aprayojakam prakrtinam varana-bhedas tu tata ksetrikavat

The relation-to-itself within the essence is the form of identity,
of the reflection-in-itself, this has taken the place of immediacy here;
both are the same abstractions of the relation-to-itself.

    Sensoriness's thoughtlessness,
    of taking everything limited and finite to be a being,
    passes over into the understanding's stubbornness,
    of grasping it as something identical with itself,
    something not contradicting itself in itself.

Originating from being, this identity seems at first to be beset only with
determinations of being and related to it as something external.
If being is taken as thus detached from the essence,
it is called the inessential.

But the essence is being-in-itself, it is essential only insofar as
it possesses within itself the negative of itself,
the relation-to-another, the mediation.

It thus has in itself the inessential as its own shine.
But since the differentiating is contained in the shining or mediating
and since what is differentiated acquires the form of identity due to
its difference from the identity from which it emerges and in which it is not
or in which it lies only as a shine because of this,
what is differentiated is in the manner of
the immediacy that relates to itself, or of being.
By this route, the sphere of the essence becomes a
still imperfect combination of immediacy and mediation.
Everything is so posited in the sphere of essence
that it refers to itself and at the same time
has passed beyond it as a being of reflection,
a being in which an other shines and which in turn shines in an other.
It is thus also the sphere of the posited contradiction
that is only in itself in the sphere of being.

    Because the one concept is the substantial element in everything,
    the same determinations surface in the development of the essence
    as in the development of being, but in reflected form.
    Hence, instead of being and nothing,
    the forms of the positive and the negative now enter in,
    the former initially corresponding to
    the opposition-less being as identity,
    the latter (shining in itself) developed as the difference;
    then, further, in the same way,
    becoming as ground itself of existence that,
    as reflected onto the ground, is concrete existence, and so forth.

    This (the most difficult) part of logic contains pre-eminently
    the categories of metaphysics and the sciences in general,
    [containing them] as products of the understanding insofar as it reflects,
    assuming the differences to be self-standing and
    at the same time also positing their relativity,
    but merely combining both aspects as next to
    and after one another through an 'also',
    without bringing these thoughts together
    and unifying them into a concept.

A. THE ESSENCE AS GROUND OF CONCRETE EXISTENCE

a. The pure determinations of reflection

a. Identity

The essence shines within itself or is pure reflection and, as such,
it is only a relation to itself, not as immediate but instead as
reflected identity with itself.

    Formal identity or identity of the understanding is this identity
    insofar as one fastens on it and abstracts from the difference.
    Or the abstraction is rather the positing of this formal identity,
    the transformation of something in itself concrete into this form of
    simplicity - be it that a part of the manifold on hand in what is
    concrete is omitted (through so-called analysing) and only one of the
    manifold parts is taken up or that, with the omission of its diversity,
    the manifold determinations are pulled together into one.

    If identity is combined with the absolute as the subject of a sentence,
    the sentence reads as follows:
    'The absolute is what is identical with itself.'
    As true as this sentence is, it is ambiguous whether it is intended
    in its true significance.
    The expression of it at least is incomplete for this reason.
    For it is left undecided whether the abstract identity of the understanding,
    in contrast to the other determinations of the essence, is meant or
    whether the identity is meant as in itself concrete;
    in the latter sense it is, as will become evident,
    first the ground and then at a higher level of truth the concept.
    Even the word 'absolute' has itself frequently no further
    meaning than that of 'abstract';
    thus, absolute space, absolute time means nothing further than
    abstract space and abstract time.

    The determinations of essence, taken as essential determinations,
    become predicates of a presupposed subject that is everything because
    those determinations are essential.
    The sentences that arise thereby have been pronounced
    the universal laws of thinking.
    The principle of identity accordingly reads:
    'Everything is identical with itself, A = A';
    and negatively:
    'A cannot be A and not A at the same time'
    This principle, instead of being a true law of thinking,
    is nothing but the law of the abstract understanding.
    The form of the sentence already contradicts it itself
    since a sentence also promises a difference between subject and predicate,
    but this sentence does not accomplish what its form requires.
    But it will be sublated in particular by the subsequent so-called
    laws of thinking that make into laws the opposite of this law.
    If one maintains that this sentence cannot be proven but that
    each consciousness proceeds in accord with it and
    experientially concurs with it as soon as it hears it,
    then it is necessary to note,
    in opposition to this alleged experience of the school,
    the general experience that no consciousness thinks,
    has representations, and so forth, or speaks according to this law,
    that no concrete existence of any sort exists according to this law.
    Speaking according to this alleged law of truth
    ('a planet is a planet', 'magnetism is magnetism', 'the spirit is a spirit')
    is considered, quite correctly, to be silly;
    this is presumably a universal experience.
    The school in which alone such laws are valid has,
    along with its logic which seriously propounds them,
    long since been discredited in the eyes of
    healthy common sense and in the eyes of reason.

b. Difference

The essence is pure identity and shine within itself only insofar
as it is the negativity that relates itself to itself,
thus the repelling of itself from itself.
Hence, it essentially contains the determination of difference.

    Being other is here no longer the qualitative [sense of being other],
    the determinacy, the limit but instead, in the essence as
    relating itself to itself, negation is at the same time relation,
    difference, positedness, being-mediated.

Difference is (1) immediate difference,
the diversity in which each of what is differentiated is
for itself what it is and indifferent to its relation to the other
which is thus a relation external to it.
Because of the indifference of the diverse [things] to their difference,
that difference falls outside them into a third (thing), which does the comparing.
As the identity of the related [things],
this external difference is (their) likeness;
as their non-identity, it is their unlikeness.

    The understanding allows these determinations themselves to be
    so separate from one another that, although the comparison has one
    and the same substrate for likeness and unlikeness, these are
    supposed to be diverse sides and respects in the same [substrate].
    But likeness is for itself simply the foregoing, the identity,
    and unlikeness is for itself the difference.
    Diversity has likewise been transformed into a sentence,
    the principle that everything is diverse or
    that there are no two things that are completely like one another.
    Here everything is provided with a predicate
    that is the opposite of the identity attributed to it
    in the first principle;
    thus, a law contradicting the first [law of thinking] is given.
    Yet, insofar as diversity pertains only to the external comparison,
    something is supposed to be only identical with itself for itself
    and thus this second principle is supposed not to contradict the first.
    But then, too, diversity does not pertain to something or everything;
    it does not constitute any essential determination of this subject;
    thus, the second principle cannot be stated in this way at all.
    If, however, something is itself diverse, according to the principle,
    then it is so through its own determinacy;
    but with this then it is no longer diversity as such that is meant
    but the determinate difference instead.
    This is also the sense of the Leibnizian principle.

Likeness is an identity only of such as are
not the same, not identical to one another, and
unlikeness is a relation of what is not alike.
Hence, neither falls indifferently outside the other
into diverse sides or aspects;
instead, each is a shining into the other.
Diversity is thus difference of reflection or
difference in itself, determinate difference.

(2) Difference in itself is essential difference,
[the difference between] the positive and the negative,
such that the former is the identical relation to itself
in such a way that it is not the negative and
the latter is the differentiated for itself
in such a way that it is not the positive.
Because each is for itself insofar as it is not the other,
each shines in the other and is only insofar as the other is.
The difference of the essence is thus the opposition
according to which what is differentiated does not have
an other in general but instead has its other opposite it.
That is to say, each has its own determination
only in its relation to the other,
is only reflected in itself insofar as
it is reflected in the other
and the same holds for the other.
Each is thus the other's own other.

    Difference in itself yields the principle:

    'Everything is something essentially differentiated' or,
    as it has also been expressed,
    'Only one of two opposite predicates pertain to
    a particular something and there is no third.'

    This principle of the opposition contradicts
    the principle of identity in the most explicit way,
    since something, according to the one principle,
    is supposed to be merely the relation to itself,
    but according to the other, is something opposite,
    the relation to another.

    It is the peculiar thoughtlessness of abstraction to place
    two such contradictory principles as laws next to one another
    without even so much as comparing them.
    The principle of the excluded third is the
    principle of the determinate understanding that wants to refrain
    from contradiction and, in doing so, contradicts itself.
    A is supposed to be +A or -A;
    but the third, the A, is thereby articulated,
    something which is neither + nor - and that is posited just as much
    as +A and as -A are.
    If + W 6 means 6 miles in a westerly direction
    and - W 6 means 6 miles in an easterly direction,
    and + and - cancel one another, then the 6 miles of the way or
    space remain what they were with and without the opposition.
    Even the mere plus and minus of the number or the abstract direction
    have, if one will, zero as their third.
    But it should not be denied that the empty opposition of the understanding,
    signalled by + and -, also has its place in the case of such abstractions
    as number, direction, and so forth.
    In the doctrine of contradictory concepts one concept means, for
    example, 'blue' (since even something like the sensory presentation
    of a colour is named a concept in such a doctrine),
    the other 'not-blue' so that this other would not be something affirmative,
    such as yellow, but instead would be fixed upon merely [as]
    something negative in an abstract sense.
    That the negative in itself is just as much positive,
    see the following section; this also lies already
    in the determination that something opposed to another
    is its other.
    The emptiness of the opposition of so-called
    contradictory concepts was completely displayed in the, as it were,
    grandiose expression of a universal law that one of every such
    opposite predicate and not the other pertains to each thing, such
    that [for example,] the spirit is either white or not-white,
    yellow or not yellow, and so on ad infinitum.

    Because it is forgotten that identity and opposition are themselves opposed,
    the principle of opposition is also taken for that of identity
    in the form of the principle of contradiction, and
    a concept to which none or both of two mutually
    contradictory characteristics apply is declared logically false
    such as, for example, a circle with four corners.
    Now, although a circle with multiple corners and a rectilinear arc
    equally contradict this principle, geometers have no reservations about
    considering and treating the circle as a polygon with rectilinear sides.
    But something like a circle (its mere determinacy) is still no concept;
    in the concept of the circle, centre and periphery are equally essential and
    yet periphery and centre are opposed and contradictory to one another.

    The notion of polarity that is so prominent in physics contains
    within itself the more correct determination of opposition;
    but if physics, in regard to its thoughts, holds itself to the ordinary logic,
    then it would easily be aghast, were it to unfold [the concept of]
    polarity for itself and arrive at the thoughts that lie within it.

The positive is that diverse [aspect]
that is supposed to be for itself and at the same time
not indifferent to its relation to its other.
The negative is supposed to be equally self-standing,
the negative relation to itself, for itself,
but at the same time, as simply negative,
is supposed to have this its relation to itself,
its positive [aspect] only in the other.
Both are, accordingly, the posited contradiction;
both are in themselves the same.
Both are so also for themselves since each is
the sublating of the other and of itself.
With this they collapse, falling to the ground.
Or the essential difference, as difference in and for itself,
immediately is only the difference of itself from itself
and hence contains the identical.
Hence, identity belongs just as inherently as difference itself
to difference in and for itself and as a whole.
As self-referring, difference is likewise already
declared to be identical with itself and
the opposed is in general what contains the one and its other,
itself, and its opposite, in itself.
Essence's being-in-itself, so determined, is the ground.

c. Ground

The ground is the unity of identity and difference;
the truth of what the difference and the identity have turned out to be:
the reflection-in-itself that is just as much
reflection-in-another and vice versa.
It is the essence posited as totality.

    The principle of the ground reads:
    'Everything has its sufficient ground [or reason]';
    that is to say, the true essence of anything is not
    the determination of it as identical with itself or
    as diverse or as merely positive or merely negative.
    It is instead the fact that it has its being in an other that,
    as its identity-with-itself, is its essence.
    The latter is equally not an abstract reflection
    in itself but in an other instead.
    The ground is the essence being in itself and
    this is essentially ground and it is ground only
    insofar as it is ground of something, of an other.

The essence is at first shining and mediation within itself.
Now, as the totality of the mediation, its unity with itself is posited
as the self~sublating of the difference and thereby of the mediation.
This is therefore the re-establishment of immediacy or being,
but of being insofar as it is mediated by the sublating of mediation:
concrete existence.

    The ground has as yet no content that is determinate in and for itself;
    neither is it a purpose, thus it is not active, nor is it productive;
    instead a concrete existence merely emerges from the ground.
    For that reason, the determinate ground is something formal;
    it is any sort of a determinacy,
    insofar as it is related to itself, posited as affirmation,
    in relation to the immediate concrete existence connected with it.
    Precisely by the fact that it is ground, it is also a good ground,
    since 'good' quite abstractly also means nothing more than something
    affirmative and each determinacy is good that can be articulated in
    any way as something affirmative that is granted.
    Thus, a ground can be found and given for everything,
    and a good ground (e.g. a good ground of motivation for acting)
    can effect something or not, can have a consequence or not.
    A ground of motivation that effects something comes about,
    for example, through its assumption into a will that
    first makes it into something active and a cause.

IV.11
atita-anagatam svarupato ‘styadhva-bhedad dharmanam

b. Concrete existence

Concrete existence is the immediate unity of
reflection-in-itself and reflection-in-another.
It is thus the indeterminate set of concretely
existing entities as reflected-in-themselves
that are at the same time just as much
a shining-in-another, are relative, and
form a world of reciprocal dependency and
an infinite connection of grounds and grounded entities.
The grounds are themselves concrete existences and
the concretely existing entities are from multiple sides
just as much grounds as they are the grounded.

The reflection-in-another of what exists concretely is,
however, not separate from the reflection-in-itself;
the ground is their unity, from which the concrete existence has gone forth.
What exists thus concretely contains in itself relativity and
its multiple connection with other entities existing concretely.
Thus, too, it is reflected in itself as ground.
As such, what exists concretely is a thing.

The thing-in-itself that has come to be so famous in
Kantian philosophy shows itself here in its origin, namely,
as the abstract reflection-in-itself that is held on to
in its opposition to the reflection-in-another and
the differentiated determinations in general
as their empty foundation.

IV.14
vastu-samye citta-bhedat tayor vibhakta pantha

IV.15
na caika-citta-tantram vastu tad apramanakam tada kim syat

IV.16
tad-uparaga-apeksitvat-cittasya vastu jnata-ajnatam

c. The thing

IV.12
te vyaktasuksma guna-atmana

IV.13
parinama-ekatvad vastu-tattvam

The thing is the totality as the development, posited in one,
of the determinations of the ground and concrete existence.

According to one of its moments, the reflection-in-another,
it has the differences in it, and, in keeping with those differences,
it is a determinate and concrete thing.

(a) These determinations are diverse from one another;
they have their reflection-in-itself in the thing, not in themselves.
They are properties of the thing and their relation to it is one of having.

    Having enters as relation in place of being.
    Something, to be sure, also has qualities in it,
    but this transposition of having onto
    beings is imprecise because the determinacy as quality
    is immediately one with the something [that has the quality], and
    something ceases to be if it loses its quality.
    But, the thing is the reflection-in-itself as
    the identity that is also different from
    the difference, its determinations.
    Having is used in many languages to designate the past,
    rightly so, since the past is the sublated being and
    the spirit its reflection-in-itself,
    the spirit in which it alone still obtains,
    but which also distinguishes this being,
    sublated in it, from itself.

(b) But in the ground, the reflection-in-another is also in itself immediately
the reflection-in-itself. Thus, the properties are just as much identical with
themselves, self-standing, and freed from their being-bound to the thing.
However, because they are the thing's determinacies, different from one
another as reflected-in-themselves, they are not themselves things which
are concrete, but instead concrete existences, reflected in themselves as
abstract determinacies, sorts of matter.

    The sorts of matter, e.g. magnetic, electric sorts of matter, are also
    not called things. They are the genuine qualities, one with their being,
    the determinacy that has attained immediacy,
    but a being that is a reflected [being], concrete existence.

Matter is thus the abstract or indeterminate reflection-in-another or
the reflection-in-itself as determinate at the same time;
it is thus existing thingness, the subsisting of the thing.
In this way, the thing has, in the sorts of matter,
its reflection-in-itself;
it does not subsist in itself, but consists of sorts of matter and
is only their superficial combination, an external linkage of them.

(c) As the immediate unity of concrete existence with itself,
matter is also indifferent to the determinacy;
the many diverse sorts of matter thus go together into
the one matter, the concrete existence in
the determination-of-reflection of identity,
in contrast to which these differentiated determinacies and
their external relation, which they have to one another in the thing,
are the form:
the determination-of-reflection of the difference,
but as existing concretely and as the totality.

    This one matter, devoid of determination, is also the same as
    the thing-in-itself, only the latter is in itself completely abstract,
    the former is in itself also for another, initially a being for the form.

The thing thus breaks down into matter and form,
each of which is the totality of thinghood and
self-standing for itself.
But the matter, which is supposed to be the positive,
indeterminate concrete existence contains as concrete existence
just as much the reflection-in-another as being-in-itself.
As the unity of these determinations,
it is itself the totality of the form.
However, as the totality of the determinations,
the form already contains the reflection-in-itself or,
as self-referring form, it has what is supposed to
make up the determination of matter.
Both are in themselves the same.
This unity of them, qua posited, is in general
the relation of matter and form that are
just as much distinguished [from one another].

The thing as this totality is the contradiction of being
(in keeping with its negative unity)
the form in which the matter is determined and relegated to properties, and
at the same time of consisting of sorts of matter that,
in the reflection-in-itself of the thing, are
at once both self-standing and negated.
The thing, being thus the essential concrete existence
as one that sublates itself in itself, is appearance.

    The negation as well as the independence of the sorts of matter
    posited in the thing surface in physics as porosity.
    Each of the many sorts of matter
    (colour-matter, odorous matter, and other sorts of matter;
    according to some also sonorous matter, then caloric matter,
    electrical matter, and so forth)
    is also negated and in this, their negation,
    their pores, are the many other self-standing sorts of
    matter that are likewise porous and allow the others to concretely
    exist thus reciprocally in themselves.
    The pores are nothing empirical but instead contrivances of
    the understanding that represents the
    aspect of the negation of the self-standing sorts of matter in this
    way and covers the further development of the contradictions with
    that nebulous confusion in which everything is selfstanding and
    everything is likewise negated in one another.
    If in the same way in the spirit
    the faculties or activities are hypostasized,
    then their living unity likewise becomes
    the confusion of the acting of one on the other.

    (We are talking here, not of the pores in the organic,
    those of wood, skin, and so on, but instead of pores
    in the so-called sorts of matter, as in the colour-matter,
    caloric-matter, and so forth, or in metals, crystals, and the like.)
    Just as there is no verification of the pores in observation,
    so also matter itself is a product of the reflective understanding as
    is a form separated from the matter, the thing and its consisting of
    sorts of matter or that it itself subsists and has only properties.
    All are products of the reflective understanding that,
    while observing and alleging to present what it observes,
    generates instead a metaphysics that is from all sides a contradiction,
    albeit a contradiction that remains hidden from it.

B. APPEARANCE


IV.17
sada jnata citta-vrttaya tat-prabho purusasya-aparinamitvat

IV.18
na tat sva-abhasam drsyatvat

IV.19
eka-samaye ca-ubhaya-anavadharanam

IV.20
citta-antara-drsye buddhi-buddher atiprasanga smrti-sankara ca

The essence must appear.

Its shining within itself is
the sublating itself and
becoming an immediacy which,
as reflection-in-itself, is
as much a subsisting (matter)
as it is form, reflection-in-another,
subsisting in the process of sublating itself.

Its shining is the determination through which
the essence is not being but essence,
and the shining, once developed, is the appearance.
The essence is thus not behind or beyond the appearance;
instead, by virtue of the fact that it is
the essence that exists concretely,
concrete existence is appearance.

a. The world of appearance

What appears concretely exists in such a way
that its subsisting is immediately sublated;
it is only one moment of the form itself.
The form encompasses in itself the subsisting
or the matter as one of its determinations.
What appears thus has its ground in the form
as its essence, its reflection-in-itself
as opposed to its immediacy, but thereby has it
only in another determinacy of the form.
This, its ground, is just as much something appearing,
and thus the appearance continues on to an infinite mediation
of the subsisting through the form
and thus equally through not subsisting.
This infinite mediation is at once
a unity of relation-to-itself,
and concrete existence develops into a totality
and world of appearance, of reflected finitude.

b. Content and form

The manner of being-outside-one-another that is
characteristic of the world of appearances is
a totality and completely contained in its relation-to-itself.
The relation of the appearance to itself is
thus completely determined, has the form in itself
and because [it is] in this identity,
has that form as its essential subsistence.
Thus the form is content and,
in keeping with its developed determinacy,
it is the law of the appearance.
The negative side of the appearance,
what is alterable and not self-sufficient,
falls to the form as not reflected in itself;
it is the indifferent, external form.

    For the contrast of form and content,
    it is essential to keep in mind that
    the content is not formless but
    instead has the form within itself
    just as much as it [the form] is
    something external to it.

    A doubling of the form presents itself;
    at one time,
    insofar as it is reflected in itself,
    it is the content and,
    at another time,
    as not reflected in itself,
    it is the external concrete existence,
    indifferent to the content.

    What presents itself here in itself is
    the absolute relation of content and of form, namely,
    their turning over and into one another,
    so that the content is nothing but the form turning into content
    and the form nothing other than the content turning into the form.
    This 'turning over' is one of the most important determinations.
    It is posited, however, only in the absolute relationship.

The immediate concrete existence, however, is
the determinacy of the subsisting itself as well as of the form;
it is thus just as much external to the determinacy of the content
as this externality, which it has through the element of its subsisting,
is essential to it.
The appearance, so posited, is the relationship
such that one and the same, [namely] the content, is
as the developed form, as the externality and opposition of
self-standing concrete existences and their identical relation,
the relation in which alone the differentiated elements are what they are.

c. The relationship

(a) The immediate relationship is that of the whole and the parts:
the content is the whole and consists of the parts (the form),
the opposite of it.
The parts are diverse from one another
and are what is self-standing.
But they are only parts in
their identical relation to one another
or insofar as, taken together, they make up the whole.
But that 'together' is the opposite and negation of the part.

(b) What is one and the same in this relationship
(the relation to itself that is on hand in it)
is thus an immediately negative relation to itself and,
to be sure, as the mediation to the effect that one and the same
is indifferent to the difference, and that
it is the negative relation to itself that repels itself,
as reflection-in-itself, towards the difference, and posits itself,
concretely existing as reflection-into-another and, in reverse direction,
conducts this reflection-into-another back to
the relation to itself and to the indifference:
the force and its expression.

    The relationship of the whole and the parts is
    the immediate relationship;
    hence, the thoughtless relationship
    and the process of the identity-with-itself
    turning over into diversity.
    There is a passage from the whole to the parts and
    from the parts to the whole,
    and in the one [the whole or the part]
    the opposition to the other is forgotten since
    each is taken as a self-standing concrete existence,
    the one time the whole, the other time the parts.
    Or since the parts are supposed to subsist in the whole
    and the whole to consist of the part one time the one,
    the other time the other is the subsisting and
    the other is each time the unessential.
    The mechanical relationship,
    in its superficial form,
    consists generally in the fact
    that the parts are taken as self-sufficient
    opposite one another and opposite the whole.

    The infinite progression that concerns
    the divisibility of matter
    can avail itself of this relationship too,
    and then it is the thoughtless oscillation
    of both sides of the relationship.
    A thing is taken one time as a whole,
    then there is a passage to the determination of it as a part,
    this determination is then forgotten and
    what was a part is limited and thus acquires its determinacy
    by means of an other outside it.
    regarded as a whole;
    the determination of it as a part resurfaces and so on,
    ad infinitum.
    Taken as the negative that it is, however,
    this infinity is the negative relation of
    the relationship to itself, the force,
    the whole that is identical with itself as being-in-itself,
    and as this being-in-itself sublating itself and expressing itself
    and, conversely, the expression that disappears
    and goes back into the force.

    This infinity notwithstanding,
    the force is also finite.
    For the content, the one and the same
    that the force and the expression are, is
    initially this identity only in itself.
    The two sides of the relationship
    are not yet themselves,
    each for itself its concrete identity,
    not yet the totality.
    In relation to one another, they are thus diverse
    and the relationship is a finite one.
    The force is thus in need of solicitation from without;
    it acts blindly, and, thanks to this deficiency of the form,
    the content is also limited and contingent.
    It is not yet truly identical with the form,
    is not yet the concept and purpose
    that is the determinate in and for itself.
    This difference is supremely essential,
    but not easy to grasp;
    it has to be determined more precisely and
    only in terms of the concept of purpose.
    If it is overlooked, this leads to
    the confusion of construing God as force,
    a confusion from which Herder's God suffers especially.

    It is usually said that the nature of force itself is unknown
    and only its expression is known.
    On the one hand, the entire determination of the content of force is
    just the same as that of the expression;
    on account of this, the explanation of a phenomenon
    on the basis of a force is an empty tautology.
    What is supposed to
    remain unknown is therefore in fact nothing but the empty form
    of the reflection-in-itself, by means of which alone the force is
    distinguished from the expression, a form that is equally
    something well known.
    This form adds nothing in the slightest to the content and to the law,
    which are supposed to be known simply on the basis of the phenomenon alone.
    Assurances are also given everywhere that, with this, nothing is
    supposed to be claimed about the force; as a result, it is impossible to
    see why the form of force has been introduced into the sciences.
    But, on the other hand, the nature of force is, of course, something
    unknown since the necessity of the connection of its content is still
    lacking, not only in itself but also and equally insofar as it is for itself
    limited and thus acquires its determinacy by means of an other outside it.

As the whole that is, in its very self,
the negative relation to itself, force is this:
the process of repelling itself from itself
and expressing itself.
But since this reflection-in-another,
the difference of the parts,
is just as much a reflection-in-itself,
the expression is the mediation by means of which
the force that returns into itself is force.
Its expression is itself the sublating of
the diversity on both sides,
which is on hand in this relationship,
and the positing of the identity
that in itself makes up the content.
Its truth is, for that reason, the relationship,
the two sides of which are distinguished only as inner and outer.

(c) The inner is the ground as the mere form
of the one side of the appearance and the relationship,
the empty form of the reflection-in-itself.
Standing opposite it is concrete existence as the form likewise
of the other side of the relationship,
with the empty determination of the
reflection-in-another as outer.
Their identity is the fulfilled identity, the content,
the unity of the reflection-in-itself and
the reflection-in-another,
posited in the movement of force.
Both are the same, one totality,
and this unity makes them into the content.

The outer is thus, in the first place, the same content as the inner is.
What is internal is also on hand externally and vice versa.
The appearance shows nothing that is not in the essence and
there is nothing in the essence that is not manifested.

In the second place, however,
inner and outer are also opposed to one another
as determinations of the form
and, to be sure, unqualifiedly so,
as the abstractions of identity with itself
and of sheer multiplicity or reality.
Yet, since they are essentially identical
as moments of the one form,
what is only posited initially
in the one abstraction is
also immediately only in the other.
Hence, what is only something internal is
also, by this means, only something external
and what is only something external is
as yet also only something internal.

    It is the usual mistake of reflection
    to take the essence as the merely inner.
    When it is taken merely in this way,
    then this consideration is
    also a completely external one
    and that essence is the empty external abstraction.

        The inner side of nature a poet says
        No created spirit can penetrate,
        Fortunate enough if he knows merely the outer shell

    It should have been said, rather,
    that precisely when he determines
    the essence of nature as something inner,
    he knows only the outer shell.
    Since in being in general
    or even in merely sensory perception,
    the concept is only the inner at first,
    it is something external for it [sensory perception],
    a subjective being as well as thinking, devoid of truth.
    In nature as in the spirit, insofar as
    the concept, purpose, law are at first
    only inner dispositions, pure possibilities,
    they are only an external, inorganic nature at first,
    science of a third, alien power, and so forth.
    As a human being is externally, in his actions
    (not, of course, in his merely corporeal externality),
    so he is internally;
    and if he is only internally virtuous, moral, and so forth,
    only in intentions and sentiments
    and his outer life is not identical with them,
    then the one is as hollow and empty as the other.

The empty abstractions, by means of which
the one identical content is still supposed
to obtain in the relationship,
sublate themselves in the immediate transition,
the one in the other;
the content is itself nothing other than their identity,
they are the shine of the essence, posited as shine.
Through the force's expression, the inner is posited in concrete existence;
this positing is the mediating by means of empty abstractions;
it vanishes in itself into the immediacy in which
the inner and outer are in and for themselves identical and
their difference is determined as mere positedness.
This identity is the actuality.

C. ACTUALITY

IV.21
citer apratisamkramayas tad-akarapattau svabuddhi-samvedanam

IV.22
drastr-drsya-uparaktam cittam sarva-artham

IV.23
tad asamkhyeya-vasanabhi citram api para-artham samhatya-karitvat

IV.24
visesa-darsina atma-bhava-bhavana-vinivrtti

IV.25
tada viveka-nimnam kaivalya-prag-bharam cittam

IV.26
tad-chidresu pratyaya-antarani samskarebhya

IV.27
hanam esam klesavad uktam

IV.28
prasamkhyane ‘pyakusidasya sarvatha viveka-khyater dharma-megha samadhi

IV.29
tata klesa-karma-nivrtti

IV.30
tada sarvavarana-malapetasya jnaanasyanantyaj jnaeyam alpam

IV.31
tata-krta-arthanam parinama-krama-samaptir gunanam

IV.32
ksana-pratiyogi parinama-aparanta-nirgrahya krama

IV.33
purusa-artha-sunyanam gunanam pratiprasava kaivalyam
svarupa-pratistha va citi-sakti iti

Actuality is that unity of essence and concrete existence,
of inner and outer, that has immediately come to be.
The expression of the actual is the actual itself,
so that in the expression it remains something equally essential
and is something essential only insofar as it is
in immediate, external concrete existence.

    As forms of the immediate,
    being and concrete existence surfaced earlier;
    being is completely unreflected immediacy and
    [the] passing over into an other.
    The concrete existence is immediate unity of
    being and reflection, thus appearance,
    coming from the ground and returning to it.
    The actual is the positedness of that unity,
    relationship that has become identical with itself.
    It is thus exempted from the passing over and
    its externality is its energy;
    in that externality, it is reflected in itself;
    its existence is only the manifestation of itself,
    not of an other.

The actuality, as this concrete [dimension], contains
those determinations and their difference;
it is, for that reason, also their development
so that they are determined in it at once
as a shine, as merely posited.

(a) As identity generally it is initially the possibility;
the reflection-in-itself that is posited as
the abstract and unessential essentiality
in contrast to the concrete unity of the actual.
Possibility is what is essential for actuality but
such that it is at the same time only possibility.

    It is probably the determination of possibility that caused
    Kant to regard it, along with actuality and necessity, as modalities,
    because these determinations did not in the slightest add to the concept
    as object but instead express only the connection to the
    capacity of knowing [Critique of Pure Reason, B 266].
    Possibility is indeed the empty abstraction of the reflection-in-itself,
    what was previously called 'the inner', with the only difference that
    it is now determined as the sublated, merely posited, external inner,
    and thus, to be sure, is also posited as a mere modality,
    as insufficient abstraction, something that, taken more concretely,
    pertains only to subjective thinking.
    Actuality and necessity are, by contrast, truly
    anything but a mere sort and manner for an other; rather, they are
    precisely the opposite, posited as the not merely posited but instead
    as the concrete [dimension] that is complete in itself.
    Because possibility, initially contrasted with
    the concrete as something actual,
    is the mere form of identity-with-itself,
    the rule for it is merely that something not be self-contradictory and
    thus everything is possible;
    for this form of identity can be given to any content through abstraction.
    But everything is just as much impossibLe, for in
    every content, since it is something concrete, the determinacy can
    be grasped as determinate opposition and thus as contradiction.
    There is, thus, no more empty way of speaking than about this sort
    of possibility and impossibility.
    In philosophy, in particular, there should not be any talk of showing that
    something is possible or that something else is also possible and
    that something, as one also expresses it, is thinkable.
    The historian is also directly advised not to use this
    category that was already declared to be untrue for itself;
    but the acumen of empty understanding is never more pleased with itself
    than when it emptily devises possibilities and an abundant supply of them.

(b) In its difference from possibility as the reflection-in-itself, however,
the actual is itself only the externally concrete [dimension],
the immediate in an inessentiaL way.
Or immediately, insofar as it initially is as the simple,
itself immediate unity of the inner and the outer,
it is what is external in an inessentiaL way and
is thus at the same time what is only internal,
the abstraction of the reflection-in-itself;
it itself is thereby determined as something only possible.
In this value of a mere possibility, the actual is something contingent and,
vice versa, possibility is mere contingency itself.

Possibility and contingency are the moments of actuality, inner and outer,
posited as mere forms that constitute the externality of the actual.
In the actual qua determined in-itself,
in the content as the essential ground of their determination,
they have their reflection-in-itself.
The finitude of the contingent and possible thus consists,
more precisely, in the fact that the form determination is
differentiated from the content and, hence,
whether something is contingent and possible depends on the content.

That externality of actuality contains more precisely this:
that the contingency as immediate actuality is
essentially what is identical with itself only as being posited,
but a being posited that is just as much sublated,
 an existing externality.
It is thus something presupposed,
the immediate existence of which is
at the same time a possibility and has the determination of being sublated -
of being the possibility of another - the condition.

(c) This externality, developed in the manner depicted, is a circle of deter-
minations of possibility and of the immediate - actuality, their mediation
by one another, the real possibility in general. As such a circle, it is further-
more the totality, thus the content, the basic matter [Sache] determined in
and for itself, and equally, in keeping with the difference of determinations
in this unity, the concrete totality of the form for itself, the immediate self-
transposing of the inner into the outer and of the outer into the inner. This
self-moving of the form is activity, activation of the basic matter as the real
ground that sublates itself and coroes to be actual, and activation of the
contingent actuality, the conditions, namely, their reflection-in-themselves
and their self-sublating to become another actuality, the
actuality of the basic matter.
If all conditions are at hand, the basic matter must become actual
and the basic matter is itself one of the conditions
since as something initially inner,
it is itself only something presupposed.
The developed actuality as the alternation
of the inner and the outer collapsing into one,
the alternation of its opposite movements
that are united into one movement, is necessity.

    Necessity has been rightly defined, to be sure,
    as the unity of possibility and actuality.
    But this determination is superficial
    and, for that reason, not understandable
    if expressed only in this way.
    The concept of necessity is very difficult
    and, indeed, it is so because it is the concept itself
    whose moments still are as actualities that,
    nonetheless, have to be grasped at the same time merely as forms, as
    in themselves broken and transitional.
    For this reason, in both of the following sections,
    the exposition of the moments that
    constitute the necessity has to be given
    in even greater detail.

Among the three moments, the condition, the basic matter, and the activity

a. the condition is

(a) something presupposed as only something supposed,
it is merely in the sense of being
relative to the basic matter,
but as presupposed it is in the sense of
a contingent, external condition,
concretely existing for itself
without regard for the basic matter.
But at the same time, in this contingency,
in regard to the basic matter which is the totality,
this presupposition is a complete circle of conditions.

(b) The conditions are passive,
they are used as material for the basic matter,
and thereby enter into the content of the basic matter.
They are just as much suited to this content
and already contain its entire determination within themselves.

b. The basic matter is equally

(a) something presupposed;
as supposed, it is initially merely something internal and possible
and, as pre-supposed, a self-sufficient content for itself

(b) Through the use of the conditions,
it obtains its external concrete existence,
realizing the determinations of its content,
determinations that correspond mutually to the conditions,
so that it both proves itself to be the basic matter
on the basis of these conditions and emerges from them.

c. The activity is

(a) also something self-sufficient and existing concretely for itself
(a human being, a character)
and, at the same time, it has its possibility
solely thanks to the conditions and the basic matter.

(b) It is the movement of translating
the conditions into the basic matter and
the basic matter into the conditions
as the side of concrete existence;
but the movement only of setting
the basic matter forth from the conditions
(in which it is on hand in itself and by way of sublating
the concrete existence of the conditions,
providing the basic matter with concrete existence.

Insofar as these three moments have the shape of a self-sufficient concrete
existence opposite one another,
this process is the external necessity.
This necessity has a limited content with respect to its basic matter.
For the basic matter is this whole in a simple determinacy.
But since it is in its form external to itself,
it is thereby also external to itself in itself and in its content,
and this externality with respect to the basic matter is
a limitation of its content.

Necessity is thus in itself the one essence,
identical with itself, but full of content,
the essence that shines in itself in such a way
that its differences have the form of self-sufficient actuals
and this identity, as the absolute form, is at the same time
the activity of sublating [immediacy] in mediated being and
the mediation in immediacy.
What is necessary is through an other that has broken up
into the mediating ground (the basic matter and the activity)
and an immediate actuality,
something contingent that is at the same time a condition.
Insofar as it is through an other, the necessary is
not in and for itself but instead something merely posited.
But this mediation is just as immediately the sublating of itself;
the ground and the contingent condition are transposed into immediacy,
by means of which that positedness is sublated to become actuality and
the basic matter has come together with itself.
In this return into itself, the necessary is
in an unqualified way, as unconditioned actuality.
The necessary is the way it is, mediated by a circle of circumstances,
it is so, because the circumstances are so;
and, at the same time, it is the way it is, unmediated,
it is so, because it is.

a. The relationship of substantiality

The necessary is in itself the absolute relationship,
the process (developed in the preceding sections) in which
the relationship equally sublates itself to become absolute identity.

In its immediate form, it is the relationship of substantiality and acciden-
tality. The absolute identity of this relationship with itself is the substance
as such which, as necessity, is the negativity of this form of interiority, thus
positing itself as actuality, but which is just as much the negativity of this
outer dimension, in keeping with which the actual as immediate is only
something accidental that, thanks to this, its mere possibility, passes over
into another actuality; a passing over which is the substantial identity as the
activity of the form·

The substance is accordingly the totality of the accidents in which it reveals
itself as their absolute negativity, as absolute power and at the same time
as the wealth of all content. This content, however, is nothing other than
this manifestation itself since the determinacy itself, reflected in itself [and
thus made into] the content, is only a moment of the form,
a moment that passes over into the power of the substance.
The substantiality is the
absolute activity of the form and the power of the necessity, and all content
is only a moment that belongs to this process alone,
the absolute turning over of form and content into one another.
Substance, qua absolute power, is the power that
relates itself to itself as only inner possibility,
determining itself thereby to accidentality,
whereby the externality thus posited is distinguished from it.
Just as it is substance in the first form of necessity,
so substance is, according to the moment just described,
genuine relationship: the relationship of causality.

b. The relationship of causality

Substance is cause insofar as it is reflected in itself
against its passing over into accidentality and
is thus the original basic matter,
but just as much supersedes the reflection-in-itself
or its mere possibility,
posits itself as the negative of itself and
in this way brings forth an effect,
an actuality which is only a posited actuality,
but through the process of effecting is
at the same time a necessary actuality.

    As the original basic matter, the cause has
    the determination of absolute self-sufficiency and
    a subsisting that maintains itself opposite the effect.
    But in the necessity, the identity of which
    constitutes that originality itself,
    it has merely passed over into the effect.
    There is no content in the effect that is not in the cause,
    insofar as it is possible again to talk of a determinate content.
    That identity is the absolute content itself.
    But it is also equally the determination of form,
    the originality is sublated in the effect in which
    it makes itself something posited.
    With this, however, the cause has not vanished such that
    the actual would be only the effect.
    For this positedness is immediately superseded just as much;
    it is indeed the reflection-in-itself of the cause, its originality;
    the cause is first actual and cause in the effect.
    The cause is thus in and for itself causa sui [cause of itself].
    Jacobi, firmly caught up in the one-sided
    representation of the mediation, took the causa sui (the effectus sui is
    the same), this absolute truth of the cause, merely for a formalism.
    He also put forward that God must be determined, not as ground,
    but essentially as cause.
    That this move did not achieve what he
    intended would have emerged from thinking over the nature of
    cause much more thoroughly.
    Even in a finite cause and its representation,
    this identity in regard to the content is at hand;
    the rain, the cause, and the wetness, the effect, are one and the same
    concretely existing water.
    In regard to the form, the cause (the rain) thus
    falls away in the effect (the wetness);
    but so does the determination of the effect
    that is nothing without the cause and
    there remains only the indifferent wetness.

    The cause in the common sense of the causal relationship is finite
    insofar as its content is finite (as in the finite substance) and insofar
    as cause and effect are represented as two different, self-sufficient
    concrete existences - which they are only because one abstracts from
    the relationship of causality in their case. Because in [the sphere
    of] finitude one does not move beyond the difference between the
    determinations of form in their relation, the cause is also alternately
    determined as something posited or as effect. The latter then has
    another cause in turn and in this way there arises here the
    progression from effects to causes ad infinitum. The same holds for
    the descending progression in that the effect, in keeping with its
    identity with the cause, is itself determined as cause and at the same
    time as another cause that has other effects in turn and so on ad
    infinitum.

The effect is different from the cause;
the effect is, as such, a being-that-is-posited.
But positedness is equally reflection-in-itself and immediacy, and
the cause's effecting, its positing, is at the same time a presupposing,
insofar as the difference of the effect from the cause is maintained.
There is accordingly another substance at hand,
in regard to which the effect happens.
This [substance] is, as immediate,
not self-relating negativity and active, but passive instead.
But, as substance, it is equally active,
it sublates the presupposed immediacy and the effect posited in it;
it reacts, it sublates the activity of the first substance which, however, is
just as much this sublating of its immediacy or
the effect posited in it, and, with this,
sublates the activity of the other and reacts.
With this, causality has passed over into the relationship of determination,
and thus positing that vacuousness of the moments that is in itself
An effect is posited in the primordiality;
that is to say, the primordiality is sublated.
The action of a cause becomes a reaction, and so forth.

c. Reciprocity

The determinations that have been kept separate in reciprocity are

(a) in themselves the same;
one side like the other is cause, original, active, passive, and so forth.
So, too, presupposing another and having an effect on it,
the immediate primordiality and the positedness
by way of alternation are one and the same.
The cause assumed to be first is,
on account of its immediacy,
 passive, a positedness, and an effect.
The difference between the causes, identified as two,
is thus empty and what is at hand is
in itself only one cause that, in its effect sublates itself
as substance just as much as it renders itself
self-sufficient in this effecting.

(b) But this unity is also for itself,
since this whole alternation is the cause's own positing,
and its being is nothing but this positing.
The vacuousness of the differences is not only
in itself or our reflection (see preceding section),
but this reciprocity is itself also the process of sublating
each of the posited determinations in turn,
inverting each into the opposite.

(c) This sheer alternation with itself is, accordingly,
the unveiled or posited necessity.
The bond of necessity as such is
the identity that is still inner and hidden
because it is the identity of
those [things] that count as actual,
but whose self-sufficiency is, nevertheless,
supposed to be precisely the necessity.
The course taken by the substance through
causality and reciprocity is thus merely
the process of positing that the self-sufficiency is
the infinite, negative relation to itself;
negative in the general sense that in it
the differentiating and mediating become an
original condition of actualities that are
self-sufficient vis-a-vis one another:
an infinite relation to itself, since their self-standing
status is precisely nothing other than their identity.

This truth of necessity is thus freedom, and
the truth of substance is the concept:
the self-sufficiency that is the repelling of itself from itself into
different self-sufficient [moments] and, as this repelling, is
identical with itself and, enduring by itself, is
this alternating movement only with itself.

The concept is accordingly the truth of being and essence,
since the shining of reflection within itself is
itself at the same time self-sufficient immediacy
and this being of diverse actuality is immediately
only a shining in itself.

    In that the concept has proven itself to be
    the truth of being and essence,
    both of which have gone back into it
    as into its ground, it has developed inversely,
    from being as from its ground.

    The former side of the progression can be considered
    a deepening of being in itself,
    the inner [dimension] of which has been unveiled by this progression;
    the latter side can be considered the emergence of the more perfect
    from the less perfect.
    Philosophy has been reproached for considering
    such development from the latter side alone.
    The more determinate
    content that the superficial thoughts of the less perfect and the more
    perfect have here is the difference between being qua immediate
    unity with itself, and the concept qua free mediation with itself.
    Since being has shown itself to be a moment of the concept,
    the concept has demonstrated itself to be the truth of being;
    as this, its reflection-in-itself, and as the sublating of the mediation,
    it presupposes the immediate - a presupposing that is identical with
    the return-into-itself, the identity that makes up the freedom and
    the concept.
    If the moment is thus named the imperfect, then,
    of course, the concept, the perfect, is this, to develop itself from
    the imperfect, for it is essentially this sublating of its presupposition.
    However, at the same time, it is the concept alone that,
    qua positing itself makes the presupposition,
    as was the outcome in causality in general and
    more specifically in reciprocity.

    In relation to being and essence, the concept is determined
    in such a way that it is the essence that has gone back to being
    as simple immediacy, the essence whose shining thereby has actuality and
    whose actuality is at the same time the process of freely shining in itself.
    In this manner the concept has being as its simple relation
    to itself or as the immediacy of its unity in itself,
    being is so impoverished a determination that it is
    the very least that can be pointed up in the concept.

    The transition from necessity to freedom or
    from the actual into the concept is the hardest transition,
    because the self-sufficient actuality is supposed to be thought
    as having its substantiality only in the process of passing over and
    in the identity with the self-sufficient actuality other than it.
    The concept is also the hardest then, because
    it is itself precisely this identity.
    The actual substance as such, however,
    the cause that, in its being-for-itself,
    does not wain to let anything penetrate into it,
    is already subject to the necessity or fate
    of passing over into positedness,
    and this subjection is the hardest by far.
    By contrast, thinking the necessity is
    rather the dissolving of that hardness;
    for it is the process of
    its coming-together with itself in an other,
    the liberation which is not the flight of abstraction
    but instead the liberation of having itself
    not as other but of having its own being and positing
    in something else actual with which what is actual is
    bound together by the power of necessity.
    As concretely existing for itself,
    this liberation is called '1',
    as developed in its totality 'free spirit',
    as feeling 'love', as enjoyment 'blessedness'.
    The great intuition of the Spinozistic substance is
    only in itself the liberation from finite being-for-itself;
    but the concept itself is for itself
    the power of necessity and the actual freedom.
