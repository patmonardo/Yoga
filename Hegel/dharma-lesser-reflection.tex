
THE DOCTRINE OF ESSENCE

The essence is the concept insofar as
it is simply posited;
in the essence, the determinations are only relative,
they are not yet fully reflected in themselves.
For this reason, the concept is not yet for itself.

As being that mediates itself with itself
in virtue of its negativity,
essence is relation to itself only insofar as
it is relation to an other that is, however, not immediately a being,
but something posited and mediated.

Being has not disappeared;

instead, in the first place,
the essence, as a simple relation to itself,
is being;

in the second place, moreover,
in keeping with being's one-sided determination
as something immediate,
being has been demoted to
something merely negative,
to a shine.

The essence is accordingly
being as shining in itself.

The absolute is essence.

This definition is the same as the definition that it is being,
insofar as being is also the simple relation to itself;
but at the same time it is higher since the essence is
being that has gone into itself, that is to say,
its simple relation to itself is this relation,
posited as the negation of the negative,
as mediation of itself in itself with itself.
However, when the absolute is determined as essence,
negativity is frequently taken only in the sense of
an abstraction from all determinate predicates.
This negative act, the abstracting,
then falls outside of the essence
and the essence itself is thus only
a result without these, its premises,
the caput mortuum of abstraction.
But since this negativity is
not external to being,
but instead is its own dialectic,
then its truth, the essence,
is the being that has gone into itself or is in itself,
that reflection, its process of shining in itself,
constitutes its difference from immediate being
and is the distinctive determination of the essence itself.

The relation-to-itself within the essence is
the form of identity, of the reflection-in-itself,
this has taken the place of immediacy here;
both are the same abstractions of the relation-to-itself.

Sensoriness's thoughtlessness,
of taking everything limited and finite to be a being,
passes over into the understanding's stubbornness,
of grasping it as something identical with itself,
something not contradicting itself in itself.

Originating from being, this identity seems at first
to be beset only with determinations of being
and related to it as something external.
If being is taken as thus detached from the essence,
it is called the inessential.

But the essence is being-in-itself, it is essential only insofar as
it possesses within itself the negative of itself,
the relation-to-another, the mediation.
It thus has in itself the inessential as its own shine.
But since the differentiating is contained in the shining or mediating
and since what is differentiated acquires the form of identity due to
its difference from the identity from which it emerges and in which it is not
or in which it lies only as a shine because of this,
what is differentiated is in the manner of
the immediacy that relates to itself, or of being.
By this route, the sphere of the essence becomes a
still imperfect combination of immediacy and mediation.
Everything is so posited in the sphere of essence
that it refers to itself and at the same time
has passed beyond it as a being of reflection,
a being in which an other shines and which in turn shines in an other.
It is thus also the sphere of the posited contradiction
that is only in itself in the sphere of being.

Because the one concept is the substantial element in everything,
the same determinations surface in the development of the essence
as in the development of being, but in reflected form.
Hence, instead of being and nothing,
the forms of the positive and the negative now enter in,
the former initially corresponding to
the opposition-less being as identity,
the latter (shining in itself) developed as the difference;
then, further, in the same way,
becoming as ground itself of existence that,
as reflected onto the ground, is concrete existence,
and so forth.

This (the most difficult) part of logic contains pre-eminently
the categories of metaphysics and the sciences in general,
[containing them] as products of the understanding insofar as it reflects,
assuming the differences to be self-standing and
at the same time also positing their relativity,
but merely combining both aspects as next to
and after one another through an 'also',
without bringing these thoughts together
and unifying them into a concept.

A. THE ESSENCE AS GROUND OF CONCRETE EXISTENCE

a. The pure determinations of reflection

(a) Identity

The essence shines within itself or is pure reflection and, as such,
it is only a relation to itself, not as immediate but instead as
reflected identity with itself.

Formal identity or identity of the understanding is this identity
insofar as one fastens on it and abstracts from the difference.
Or the abstraction is rather the positing of this formal identity,
the transformation of something in itself concrete into this form of
simplicity - be it that a part of the manifold on hand in what is
concrete is omitted (through so-called analysing) and only one of the
manifold parts is taken up or that, with the omission of its diversity,
the manifold determinations are pulled together into one.

If identity is combined with the absolute as the subject of a sentence,
the sentence reads as follows:
'The absolute is what is identical with itself.'
As true as this sentence is, it is ambiguous whether it is intended
in its true significance.
The expression of it at least is incomplete for this reason.
For it is left undecided whether the abstract identity of the understanding,
in contrast to the other determinations of the essence, is meant or
whether the identity is meant as in itself concrete;
in the latter sense it is, as will become evident,
first the ground and then at a higher level of truth the concept.
Even the word 'absolute' has itself frequently no further
meaning than that of 'abstract';
thus, absolute space, absolute time means nothing further than
abstract space and abstract time.

The determinations of essence, taken as essential determinations,
become predicates of a presupposed subject that is everything because
those determinations are essential.
The sentences that arise thereby have been pronounced
the universal laws of thinking.
The principle of identity accordingly reads:
'Everything is identical with itself, A = A';
and negatively:
'A cannot be A and not A at the same time'
This principle, instead of being a true law of thinking,
is nothing but the law of the abstract understanding.
The form of the sentence already contradicts it itself
since a sentence also promises a difference between subject and predicate,
but this sentence does not accomplish what its form requires.
But it will be sublated in particular by the subsequent so-called
laws of thinking that make into laws the opposite of this law.
If one maintains that this sentence cannot be proven but that
each consciousness proceeds in accord with it and
experientially concurs with it as soon as it hears it,
then it is necessary to note,
in opposition to this alleged experience of the school,
the general experience that no consciousness thinks,
has representations, and so forth, or speaks according to this law,
that no concrete existence of any sort exists according to this law.
Speaking according to this alleged law of truth
('a planet is a planet', 'magnetism is magnetism', 'the spirit is a spirit')
is considered, quite correctly, to be silly;
this is presumably a universal experience.
The school in which alone such laws are valid has,
along with its logic which seriously propounds them,
long since been discredited in the eyes of
healthy common sense and in the eyes of reason.

(b) Difference

The essence is pure identity and shine within itself only insofar
as it is the negativity that relates itself to itself,
thus the repelling of itself from itself.
Hence, it essentially contains the determination of difference.

Being other is here no longer the qualitative [sense of being other],
the determinacy, the limit but instead, in the essence as
relating itself to itself, negation is at the same time relation,
difference, positedness, being-mediated.

Difference is (1) immediate difference,
the diversity in which each of what is differentiated is
for itself what it is and indifferent to its relation to the other
which is thus a relation external to it.
Because of the indifference of the diverse [things] to their difference,
that difference falls outside them into a third (thing), which does the comparing.
As the identity of the related [things],
this external difference is (their) likeness;
as their non-identity, it is their unlikeness.

The understanding allows these determinations themselves to be
so separate from one another that, although the comparison has one
and the same substrate for likeness and unlikeness, these are
supposed to be diverse sides and respects in the same [substrate].
But likeness is for itself simply the foregoing, the identity,
and unlikeness is for itself the difference.
Diversity has likewise been transformed into a sentence,
the principle that everything is diverse or
that there are no two things that are completely like one another.
Here everything is provided with a predicate
that is the opposite of the identity attributed to it
in the first principle;
thus, a law contradicting the first [law of thinking] is given.
Yet, insofar as diversity pertains only to the external comparison,
something is supposed to be only identical with itself for itself
and thus this second principle is supposed not to contradict the first.
But then, too, diversity does not pertain to something or everything;
it does not constitute any essential determination of this subject;
thus, the second principle cannot be stated in this way at all.
If, however, something is itself diverse, according to the principle,
then it is so through its own determinacy;
but with this then it is no longer diversity as such that is meant
but the determinate difference instead.
This is also the sense of the Leibnizian principle.

Likeness is an identity only of such as are
not the same, not identical to one another, and
unlikeness is a relation of what is not alike.
Hence, neither falls indifferently outside the other
into diverse sides or aspects;
instead, each is a shining into the other.
Diversity is thus difference of reflection or
difference in itself, determinate difference.

(2) Difference in itself is essential difference,
[the difference between] the positive and the negative,
such that the former is the identical relation to itself
in such a way that it is not the negative and
the latter is the differentiated for itself
in such a way that it is not the positive.
Because each is for itself insofar as it is not the other,
each shines in the other and is only insofar as the other is.
The difference of the essence is thus the opposition
according to which what is differentiated does not have
an other in general but instead has its other opposite it.
That is to say, each has its own determination
only in its relation to the other,
is only reflected in itself insofar as
it is reflected in the other
and the same holds for the other.
Each is thus the other's own other.

Difference in itself yields the principle:

'Everything is something essentially differentiated'
or, as it has also been expressed,
'Only one of two opposite predicates pertain to
a particular something and there is no third.'

This principle of the opposition contradicts
the principle of identity in the most explicit way,
since something, according to the one principle,
is supposed to be merely the relation to itself,
but according to the other, is something opposite,
the relation to another.

It is the peculiar thoughtlessness of abstraction to place
two such contradictory principles as laws next to one another
without even so much as comparing them.
The principle of the excluded third is the
principle of the determinate understanding
that wants to refrain from contradiction
and, in doing so, contradicts itself.
A is supposed to be +A or -A;
but the third, the A, is thereby articulated,
something which is neither + nor -
and that is posited just as much
as +A and as -A are.
If + W 6 means 6 miles in a westerly direction
and - W 6 means 6 miles in an easterly direction,
and + and - cancel one another, then the 6 miles of the way or
space remain what they were with and without the opposition.
Even the mere plus and minus of the number or the abstract direction
have, if one will, zero as their third.
But it should not be denied that the empty opposition of the understanding,
signalled by + and -, also has its place in the case of such abstractions
as number, direction, and so forth.
In the doctrine of contradictory concepts one concept means,
for example, 'blue' (since even something like the sensory presentation
of a colour is named a concept in such a doctrine),
the other 'not-blue' so that this other would not be something affirmative,
such as yellow, but instead would be fixed upon merely [as]
something negative in an abstract sense.
That the negative in itself is just as much positive,
see the following section; this also lies already
in the determination that something opposed to
another is its other.
The emptiness of the opposition of so-called
contradictory concepts was completely displayed in the, as it were,
grandiose expression of a universal law that one of every such
opposite predicate and not the other pertains to each thing, such
that [for example,] the spirit is either white or not-white,
yellow or not yellow, and so on ad infinitum.

Because it is forgotten that identity and opposition are themselves opposed,
the principle of opposition is also taken for that of identity
in the form of the principle of contradiction,
and a concept to which none or both of
two mutually contradictory characteristics apply
is declared logically false such as,
for example, a circle with four corners.
Now, although a circle with multiple corners
and a rectilinear arc equally contradict this principle,
geometers have no reservations about considering
and treating the circle as a polygon with rectilinear sides.
But something like a circle (its mere determinacy) is still no concept;
in the concept of the circle, centre and periphery are equally essential
and yet periphery and centre are opposed and contradictory to one another.

The notion of polarity that is so prominent in physics contains
within itself the more correct determination of opposition;
but if physics, in regard to its thoughts, holds itself to the ordinary logic,
then it would easily be aghast, were it to unfold [the concept of]
polarity for itself and arrive at the thoughts that lie within it.

The positive is that diverse [aspect]
that is supposed to be for itself and at the same time
not indifferent to its relation to its other.
The negative is supposed to be equally self-standing,
the negative relation to itself, for itself,
but at the same time, as simply negative,
is supposed to have this its relation to itself,
its positive [aspect] only in the other.
Both are, accordingly, the posited contradiction;
both are in themselves the same.
Both are so also for themselves since each is
the sublating of the other and of itself.
With this they collapse, falling to the ground.
Or the essential difference, as difference in and for itself,
immediately is only the difference of itself from itself
and hence contains the identical.
Hence, identity belongs just as inherently as difference itself
to difference in and for itself and as a whole.
As self-referring, difference is likewise already
declared to be identical with itself and
the opposed is in general what contains the one and its other,
itself, and its opposite, in itself.
Essence's being-in-itself, so determined, is the ground.

(c) Ground

The ground is the unity of identity and difference;
the truth of what the difference and the identity have turned out to be:
the reflection-in-itself that is just as much
reflection-in-another and vice versa.
It is the essence posited as totality.

The principle of the ground reads:
'Everything has its sufficient ground [or reason]';
that is to say, the true essence of anything is not
the determination of it as identical with itself or
as diverse or as merely positive or merely negative.
It is instead the fact that it has its being in an other that,
as its identity-with-itself, is its essence.
The latter is equally not an abstract reflection
in itself but in an other instead.
The ground is the essence being in itself and
this is essentially ground and it is ground only
insofar as it is ground of something, of an other.

The essence is at first shining and mediation within itself.
Now, as the totality of the mediation, its unity with itself is posited
as the self~sublating of the difference and thereby of the mediation.
This is therefore the re-establishment of immediacy or being,
but of being insofar as it is mediated by the sublating of mediation:
concrete existence.

The ground has as yet no content that is determinate in and for itself;
neither is it a purpose, thus it is not active, nor is it productive;
instead a concrete existence merely emerges from the ground.
For that reason, the determinate ground is something formal;
it is any sort of a determinacy,
insofar as it is related to itself, posited as affirmation,
in relation to the immediate concrete existence connected with it.
Precisely by the fact that it is ground, it is also a good ground,
since 'good' quite abstractly also means nothing more than something
affirmative and each determinacy is good that can be articulated in
any way as something affirmative that is granted.
Thus, a ground can be found and given for everything,
and a good ground (e.g. a good ground of motivation for acting)
can effect something or not, can have a consequence or not.
A ground of motivation that effects something comes about,
for example, through its assumption into a will that
first makes it into something active and a cause.

b. Concrete existence

Concrete existence is the immediate unity of
reflection-in-itself and reflection-in~another.
It is thus the indeterminate set of concretely
existing entities as reflected-in-themselves
that are at the same time just as much
a shining-in-another, are relative, and
form a world of reciprocal dependency and
an infinite connection of grounds and grounded entities.
The grounds are themselves concrete existences and
the concretely existing entities are from multiple sides
just as much grounds as they are the grounded.

The reflection-in-another of what exists concretely is,
however, not separate from the reflection-in-itself;
the ground is their unity, from which the concrete existence has gone forth.
What exists thus concretely contains in itself relativity and
its multiple connection with other entities existing concretely.
Thus, too, it is reflected in itself as ground.
As such, what exists concretely is a thing.

The thing-in-itself that has come to be so famous in
Kantian philosophy shows itself here in its origin, namely,
as the abstract reflection-in-itself that is held on to
in its opposition to the reflection-in-another and
the differentiated determinations in general
as their empty foundation.

c. The thing

The thing is the totality as the development, posited in one,
of the determinations of the ground and concrete existence.

According to one of its moments, the reflection-in-another,
it has the differences in it, and, in keeping with those differences,
it is a determinate and concrete thing.

(a) These determinations are diverse from one another;
they have their reflection-in-itself in the thing, not in themselves.
They are properties of the thing and their relation to it is one of having.

Having enters as relation in place of being.
Something, to be sure, also has qualities in it,
but this transposition of having onto
beings is imprecise because the determinacy as quality
is immediately one with the something [that has the quality], and
something ceases to be if it loses its quality.
But, the thing is the reflection-in-itself as
the identity that is also different from
the difference, its determinations.
Having is used in many languages to designate the past,
rightly so, since the past is the sublated being and
the spirit its reflection-in-itself,
the spirit in which it alone still obtains,
but which also distinguishes this being,
sublated in it, from itself.

(b) But in the ground, the reflection-in-another is also in itself immediately
the reflection-in-itself. Thus, the properties are just as much identical with
themselves, self-standing, and freed from their being-bound to the thing.
However, because they are the thing's determinacies, different from one
another as reflected-in-themselves, they are not themselves things which
are concrete, but instead concrete existences, reflected in themselves as
abstract determinacies, sorts of matter.

The sorts of matter, e.g. magnetic, electric sorts of matter, are also
not called things. They are the genuine qualities, one with their being,
the determinacy that has attained immediacy,
but a being that is a reflected [being], concrete existence.

Matter is thus the abstract or indeterminate reflection-in-another or
the reflection-in-itself as determinate at the same time;
it is thus existing thingness, the subsisting of the thing.
In this way, the thing has, in the sorts of matter,
its reflection-in-itself;
it does not subsist in itself, but consists of sorts of matter and
is only their superficial combination, an external linkage of them.

(c) As the immediate unity of concrete existence with itself,
matter is also indifferent to the determinacy;
the many diverse sorts of matter thus go together into
the one matter, the concrete existence in
the determination-of-reflection of identity,
in contrast to which these differentiated determinacies and
their external relation, which they have to one another in the thing,
are the form:
the determination-of-reflection of the difference,
but as existing concretely and as the totality.

This one matter, devoid of determination, is also the same as
the thing-in-itself, only the latter is in itself completely abstract,
the former is in itself also for another, initially a being for the form.

The thing thus breaks down into matter and form,
each of which is the totality of thinghood and
self-standing for itself.
But the matter, which is supposed to be the positive,
indeterminate concrete existence contains as concrete existence
just as much the reflection-in-another as being-in-itself.
As the unity of these determinations,
it is itself the totality of the form.
However, as the totality of the determinations,
the form already contains the reflection-in-itself or,
as self-referring form, it has what is supposed to
make up the determination of matter.
Both are in themselves the same.
This unity of them, qua posited, is in general
the relation of matter and form that are
just as much distinguished [from one another].

The thing as this totality is the contradiction of being
(in keeping with its negative unity)
the form in which the matter is determined and relegated to properties, and
at the same time of consisting of sorts of matter that,
in the reflection-in-itself of the thing, are
at once both self-standing and negated.
The thing, being thus the essential concrete existence
as one that sublates itself in itself, is appearance.

The negation as well as the independence of the sorts of matter
posited in the thing surface in physics as porosity.
Each of the many sorts of matter
(colour-matter, odorous matter, and other sorts of matter;
according to some also sonorous matter, then caloric matter,
electrical matter, and so forth)
is also negated and in this, their negation,
their pores, are the many other self-standing sorts of
matter that are likewise porous and allow the others to concretely
exist thus reciprocally in themselves.
The pores are nothing empirical but instead contrivances of
the understanding that represents the
aspect of the negation of the self-standing sorts of matter in this
way and covers the further development of the contradictions with
that nebulous confusion in which everything is selfstanding and
everything is likewise negated in one another.
If in the same way in the spirit
the faculties or activities are hypostasized,
then their living unity likewise becomes
the confusion of the acting of one on the other.

(We are talking here, not of the pores in the organic,
those of wood, skin, and so on, but instead of pores
in the so-called sorts of matter, as in the colour-matter,
caloric-matter, and so forth, or in metals, crystals, and the like.)
Just as there is no verification of the pores in observation,
so also matter itself is a product of the reflective understanding as
is a form separated from the matter, the thing and its consisting of
sorts of matter or that it itself subsists and has only properties.
All are products of the reflective understanding that,
while observing and alleging to present what it observes,
generates instead a metaphysics that is from all sides a contradiction,
albeit a contradiction that remains hidden from it.
