
BOOK ONE

The Doctrine of Being

GENERAL DIVISION OF BEING

Being is determined, first, as against another in general;
secondly, it is internally self-determining;
thirdly, as this preliminary division is cast off,
it is the abstract indeterminateness and immediacy
in which it must be the beginning.

According to the first determination,
being partitions itself off from essence,
for further on in its development it proves to be
in its totality only one sphere of the concept,
and to this sphere as moment it opposes another sphere.
According to the second, it is the sphere
within which fall the determinations
and the entire movement of its reflection.

In this, being will posit itself in three determinations:

I. as determinateness; as such, quality;

II. as sublated determinateness; magnitude, quantity;

III. as qualitatively determined quantity; measure.

This division, as was generally remarked of
such divisions in the Introduction,
is here a preliminary statement;
its determinations must first arise
from the movement of being itself,
and receive their definitions and justification by virtue of it.
As regards the divergence of this division from
the usual listing of the categories,
namely quantity, quality, relation and modality,
(for Kant, incidentally, these are supposed to be
only classifications of his categories,
but are in fact themselves categories,
only more abstract ones;
about this, there is nothing to remark here,
since the entire listing will diverge from
the usual ordering and meaning of
the categories at every point.)

This only can perhaps be remarked,
that the determination of quantity
is ordinarily listed ahead of quality and as a rule
this is done for no given reason.
It has already been shown that
the beginning is made with being as such,
and hence with qualitative being.
It is clear from a comparison of quality with quantity
that the former is by nature first.
For quantity is quality which has already become negative;
magnitude is the determinateness which,
no longer one with being but already distinguished from it,
is the sublated quality that has become indifferent.
It includes the alterability of being
without altering the fact itself,
namely being, of which it is the determination;
qualitative determinateness is on the contrary one with its being,
it neither transcends it nor stays within it
but is its immediate restrictedness.
Hence quality, as the determinateness which is immediate,
is the first and it is with it that the beginning is to be made.

Measure is a relation, not relation in general
but specifically of quality and quantity to each other;
the categories dealt with by Kant under relation
will come up elsewhere in their proper place.
Measure, if one so wishes, can be considered also a modality;
but since with Kant modality is no longer
supposed to make up a determination of content,
but only concerns the reference of the content
to thought, to the subjective,
the result is a totally heterogeneous reference
that does not belong here.

The third determination of being falls within
the section Quality inasmuch as being, as abstract immediacy,
reduces itself to one single determinateness
as against its other determinacies inside its sphere.

SECTION I

Determinateness (Quality)

Being is the indeterminate immediate;
it is free of determinateness with respect to essence,
just as it is still free of any determinateness
that it can receive within itself.
This reflectionless being is being
as it immediately is only within.

Since it is immediate, it is being without quality;
but the character of indeterminateness attaches to it in itself
only in opposition to what is determinate or qualitative.
Determinate being thus comes to stand over and against being in general;
with that, however, the very indeterminateness of being
constitutes its quality.
It will therefore be shown that the first being is
in itself determinate, and therefore secondly,
that it passes over into existence, is existence;
that this latter, however, as finite being, sublates itself
and passes over into the infinite reference of being to itself;
it passes over, thirdly, into being-for-itself.

CHAPTER 1

Being

A. BEING

Being, pure being, without further determination.
In its indeterminate immediacy it is equal only to itself
and also not unequal with respect to another;
it has no difference within it, nor any outwardly.
If any determination or content were posited in it as distinct,
or if it were posited by this determination or content
as distinct from an other,
it would thereby fail to hold fast to its purity.
It is pure indeterminateness and emptiness.
There is nothing to be intuited in it,
if one can speak here of intuiting;
or, it is only this pure empty intuiting itself.
Just as little is anything to be thought in it,
or, it is equally only this empty thinking.
Being, the indeterminate immediate is in fact nothing,
and neither more nor less than nothing.

B. NOTHING

Nothing, pure nothingness;
it is simple equality with itself,
complete emptiness,
complete absence of determination and content;
lack of all distinction within.
In so far as mention can be made here of
intuiting and thinking,
it makes a difference whether something or nothing is
being intuited or thought.
To intuit or to think nothing has therefore a meaning;
the two are distinguished and so nothing is (concretely exists)
in our intuiting or thinking;
or rather it is the empty intuiting and thinking itself,
like pure being.
Nothing is therefore the same determination
or rather absence of determination,
and thus altogether the same as what pure being is.

C. BECOMING

1. Unity of being and nothing

Pure being and pure nothing are therefore the same.
The truth is neither being nor nothing,
but rather that being has passed over into nothing
and nothing into being;
“has passed over,” not passes over.

But the truth is just as much that
they are not without distinction;
it is rather that they are not the same,
that they are absolutely distinct
yet equally unseparated and inseparable,
and that each immediately vanishes in its opposite.

Their truth is therefore this movement of
the immediate vanishing of the one into the other:
becoming, a movement in which the two are distinguished,
but by a distinction which has just as immediately dissolved itself.

2. The moments of becoming

Becoming is the unseparatedness of being and nothing,
not the unity that abstracts from being and nothing;
as the unity of being and nothing
it is rather this determinate unity,
or one in which being and nothing equally are.
However, inasmuch as being and nothing are
each unseparated from its other, each is not.
In this unity, therefore, they are,
but as vanishing, only as sublated.
They sink from their initially represented self-subsistence
into moments which are still distinguished
but at the same time sublated.

Grasped as thus distinguished,
each is in their distinguishedness
a unity with the other.
Becoming thus contains being and nothing as two such unities,
each of which is itself unity of being and nothing;
the one is being as immediate and as reference to nothing;
the other is nothing as immediate and as reference to being;
in these unities the determinations are of unequal value.

Becoming is in this way doubly determined.
In one determination, nothing is the immediate,
that is, the determination begins with nothing
and this refers to being;
that is to say, it passes over into it.
In the other determination, being is the immediate,
that is, the determination begins with being
and this passes over into nothing:
coming-to-be and ceasing-to-be.

Both are the same, becoming,
and even as directions that are so different
they interpenetrate and paralyze each other.
The one is ceasing-to-be;
being passes over into nothing,
but nothing is just as much the opposite of itself,
the passing-over into being, coming-to-be.
This coming-to-be is the other direction;
nothing goes over into being,
but being equally sublates itself
and is rather the passing-over into nothing;
it is ceasing-to-be.
They do not sublate themselves reciprocally
[the one sublating the other externally]
but each rather sublates itself in itself
and is within it the opposite of itself.

3. Sublation of becoming

The equilibrium in which coming-to-be and ceasing-to-be are poised
is in the first place becoming itself.
But this becoming equally collects itself in quiescent unity.
Being and nothing are in it only as vanishing;
becoming itself, however, is only by virtue of their being distinguished.
Their vanishing is therefore the vanishing of becoming,
or the vanishing of the vanishing itself.
Becoming is a ceaseless unrest that collapses into a quiescent result.

This can also be expressed thus:
becoming is the vanishing of being into nothing,
and of nothing into being,
and the vanishing of being and nothing in general;
but at the same time it rests on their being distinct.
It therefore contradicts itself in itself,
because what it unites within itself is self-opposed;
but such a union destroys itself.

This result is a vanishedness, but it is not nothing;
as such, it would be only a relapse into one of
the already sublated determinations
and not the result of nothing and of being.
It is the unity of being and nothing
that has become quiescent simplicity.
But this quiescent simplicity is being,
yet no longer for itself but as determination of the whole.

Becoming, as transition into
the unity of being and nothing,
a unity which is as existent
or has the shape of the one-sided
immediate unity of these moments,
is existence.

CHAPTER 2

Existence

Existence is determinate being;
its determinateness is existent determinateness, quality.
Through its quality, something is opposed to an other;
it is alterable and finite,
negatively determined not only towards an other,
but absolutely within it.
This negation in it,
in contrast at first
with the finite something,
is the infinite;
the abstract opposition
in which these determinations appear
resolves itself into oppositionless infinity,
into being-for-itself.

The treatment of existence is therefore in three divisions:

A. existence as such
B. something and other, finitude
C. qualitative infinity.

Transition

Ideality can be called the quality of the infinite;
but it is essentially the process of becoming,
and hence a transition, like the transition
of becoming into existence.
We must now explicate this transition.
This immanent turning back, as the sublating of finitude,
of finitude as such and equally of the negative finitude
that only stands opposite to it, is only negative finitude,
is self-reference, being.
Since there is negation in this being, the latter is existence;
but, further, since the negation is essentially
negation of the negation, self-referring negation,
it is the existence that carries the name of being-for-itself.

CHAPTER 3

Being-for-itself

In being-for-itself,
qualitative being is brought to completion;
it is infinite being;
the being of the beginning is void of determination;
existence is sublated but only immediately sublated being;
it thus contains, to begin with,
only the first negation, itself immediate;
being is of course retained as well,
and the two are united in existence in simple unity;
for this reason, however, each is in itself still unlike the other,
and their unity is still not posited.
Existence is therefore the sphere of differentiation,
of dualism, the domain of finitude.
Determinateness is determinateness as such;
being which is relatively, not absolutely, determined.
In being-for-itself, the distinction
between being and determinateness,
or negation, is posited and equalized.
Quality, otherness, limit, as well as reality, in-itselfness,
ought, and so forth, are the incomplete configurations of negation in being
which are still based on the differentiation of the two.
But since in finitude negation has passed over into infinity,
in the posited negation of negation,
negation is simple self-reference
and in it, therefore, the equalization with being:
absolutely determinate being.

First, being-for-itself is immediately
an existent-for-itself, the one.

Second, the one passes over into a multiplicity of ones,
repulsion or the otherness of the one
which sublates itself into its ideality, attraction.

Third, we have the alternating
determination of repulsion and attraction
in which the two sink into a state of equilibrium;
and quality, driven to a head in being-for-itself,
passes over into quantity.

SECTION II

Magnitude (Quantity)

The difference between quantity and quality has been indicated.
Quality is the first, immediate determinateness.
Quantity is the determinateness that
has become indifferent to being;
a limit which is just as much no limit;
being-for-itself which is absolutely
identical with being-for-another:
the repulsion of the many ones
which is immediate non-repulsion,
their continuity.

Because that which exists for itself is now so posited
that it does not exclude its other
but rather affirmatively continues in it,
it is then otherness, inasmuch as
existence surfaces again on this continuity
and its determinateness is at the same time
no longer simple self-reference,
no longer the immediate determinateness
of the existent something,
but is posited as repelling itself from itself,
as referring to itself in the determinateness
rather of an other existence
(a being which exists for itself);
and since they are at the same time indifferent limits,
reflected into themselves and unconnected,
determinateness is as such outside itself,
an absolute externality and a something just as external;
such a limit, the indifference of the limit as limit
and the indifference of the something to the limit,
constitutes the quantitative determinateness of the something.

In the first place, we have to distinguish pure quantity
from quantity as determinate, from quantum.
First, pure quantity is real being-for-itself
turned back into itself, with as yet no determinateness in it:
a compact, infinite unity which continues itself into itself.

Second, this quantity proceeds to determinateness,
and this is posited in it as a determinateness
that at the same time is none, is only external.
Quantity becomes quantum.
Quantum is indifferent determinateness,
that is, one that transcends itself, negates itself;
as this otherness of otherness, it lapses into infinite progress.
Infinite quantum, however, is sublated indifferent determinateness:
it is the restoration of quality.

Third, quantum in qualitative form is quantitative ratio.
Quantum transcends itself only in general;
in the ratio, however, it transcends itself into its otherness,
in such a way that this otherness in which it has its determination
is at the same time posited, is another quantum.
With this we have quantum as turned back into itself
and referring to itself as into its otherness.

At the foundation of this relation there still
lies the externality of quantum;
it is indifferent quanta that
relate themselves to each other,
that is, they have the reference
that mutually connects them
in this being-outside-itself.
The ratio is, therefore, only
a formal unity of quality and quantity,
and dialectic is its transition into
their absolute unity, in measure.

CHAPTER 1

Quantity

CHAPTER 2

Quantum

Quantum, which in the first instance is quantity
with a determinateness or limit in general,
in its complete determinateness is number.
Second, quantum divides first into extensive quantum,
in which limit is the limitation of
a determinately existent plurality;
and then, inasmuch as the existence of
this plurality passes over into being-for-itself,
into intensive quantum or degree.
This last is for-itself but also,
as indifferent limit, equally outside itself.
It thus has its determinateness in an other.
Third, as this posited contradiction of
being determined simply in itself
yet having its determinateness outside itself
and pointing outside itself for it,
quantum, as thus posited outside itself within itself,
passes over into quantitative infinity.

CHAPTER 3

Ratio or the quantitative relation

The infinity of quantum has been determined up to
the point where it is the negative beyond of quantum,
a beyond which quantum, however, has within it.
This beyond is the qualitative moment in general.
The infinite quantum, as the unity of the two moments,
of the quantitative and the qualitative determinateness,
is in the first instance ratio.

In ratio, quantum no longer has
a merely indifferent determinateness
but is qualitatively determined as
simply referring to its beyond.
It continues in its beyond, and this beyond is
at first just an other quantum.
Essentially, however, the two do not refer
to each other as external quanta
but each rather possesses its determinateness
in this reference to the other.
In this, in their otherness, they have
thus returned into themselves;
what each is, that it is in its other;
the other constitutes the determinateness of each.
The quantum's self-transcendence does not now mean, therefore,
that quantum has simply changed either into some other
or into its abstract other, into its abstract beyond,
but that there, in the other, it has attained its determinateness;
in its other, which is an other quantum, it finds itself.
The quality of quantum, its conceptual determinateness,
is its externality as such,
and in ratio quantum is now posited as having
its determinateness in this externality, in another quantum;
as being in its beyond what it is.
It is quanta that stand to each other in the connection
that has now come on the scene.
This connection is itself also a magnitude;
quantum is not only in relation,
but is itself posited as relation;
it is a quantum as such that has
that qualitative determinateness in itself.
So, as relation (as ratio),
quantum gives expression to itself as self-enclosed totality
and to its indifference to limit by containing
the externality of its being-determined in itself:
in this externality it is only referred back to itself
and is thus infinite within.

Ratio in general is:

1. direct ratio.
In this, the qualitative moment does not
yet emerge explicitly as such;
in no other way except still as quantum is quantum posited
as having its determinateness in its externality.
In itself, however, the quantitative relation is
the contradiction of externality and self-reference,
the persistence of quanta and their negation.
Such a contradiction next sublates itself:

2. first inasmuch as in indirect or inverse ratio
the negation of each of the quanta is
as such co-posited in the alteration of the other,
and the variability of the direct ratio is itself posited;

3. but in the ratio of powers, the unity,
which in its difference refers back to itself,
proves to be a simple self-production of the quantum;
this qualitative moment itself,
finally posited in a simple determination
and as identical with the quantum, becomes measure.

About the nature of the following ratios,
much was anticipated in the preceding remarks
concerning the infinity of quantity, that is,
the qualitative moment in it;
it only remains to analyze, therefore,
the abstract concept of this ratio.

SECTION III

Measure

Abstractly expressed, quality and quantity are in measure united.
Being as such is the immediate equality of determinateness with itself.
This immediacy of determinateness has sublated itself.
Quantity is being that has returned to itself in such a way
that it is a simple self-equality indifferent to determinateness.
But this indifference is only the externality of having
the determinateness not in itself but in an other.
As third, we now have the externality that refers itself to itself;
as self-reference, it is at the same time sublated externality and
carries the difference from itself in it;
a difference which, as externality, is the quantitative moment,
and, as taken back into itself, the qualitative.

Measure is at first the immediate unity of
the qualitative and the quantitative element, so that it is,

first, a quantum that has qualitative meaning and is as measure.
As so implicitly determined in itself, its further determination is that
the difference of its moments,
of its qualitatively and quantitatively determined being,
is disclosed in it.
These moments further determine themselves into
wholes of measure which as such are self-subsistent,
and, inasmuch as they refer to each other essentially,
measure becomes,

second, a ratio of specific quanta, each an independent measure.
But their self-subsistence also rests essentially
on a quantitative relation and a difference of magnitude,
and so the self-subsistence becomes a transition
of one measure into another.
The result is that measure collapses into the measureless.
But this beyond of measure is the negativity of measure
only in itself; thus,

third, the indifference of the determinations of
measure is thereby posited, and measure
(real measure because of the negativity contained within it)
is posited as an inverse ratio of measures
which, as self-subsistent qualities,
essentially rest on only their quantity
and their negative reference to each other,
and consequently prove to be only moments of
their truly self-subsistent unity.
This unity is the reflection-into-itself of each
and the positing of them; it is essence.

The development of measure,
which we have attempted in what follows,
is among the most difficult of subject matters.
Starting with immediate, external measure,
it would have to proceed, on the one hand,
to the further abstract determination of
the quantitative aspect of natural things
(of a mathematics of nature);
on the other side, it would have to indicate the link
between this determination of measure
and the qualitative aspect of those things,
at least in general, for the detailed demonstration of the
link between the qualitative and the quantitative aspects
as they originate in the concept of a concrete object
belongs to the particular science of the concrete
(examples of which, concerning the law of falling bodies
and the free movement of the heavens,
will be found in the Encyclopedia of the Philosophical Sciences).
We may remark quite in general in this connection
that the different forms in which measure is realized
also belong to different spheres of natural reality.
The complete, abstract indifference of developed measure,
that is, of its laws, can only be found in the sphere of mechanism
where concrete corporeity is only abstract matter itself;
the qualitative differences of this matter are
of an essentially quantitative nature;
space and time are nothing but pure externalities,
and the aggregates of matters, the masses, the intensity of weight,
are determinations which are just as external
and have their proper determinateness in the quantitative element.
On the other hand, in physical things but even more so in the organic,
this quantitative determinateness of abstract materiality is
already disturbed by the multiplicity
and consequently the conflict of qualities.
And the thus ensuing conflict is not just one of qualities as such,
but measure itself is subordinated here to higher relations
and its immanent development is reduced rather to
the simple form of immediate measure.
The limbs of the animal organism have a measure
which, as a simple quantum, stands
in a ratio to the other quanta of the other limbs;
the proportions of the human body are the fixed ratios of such quanta,
and the science of nature still has far to go in
discovering anything about the link that connects
these magnitudes with the organic functions
on which they are entirely dependent.
But the closest example of the reduction of an immanent
measure to a merely externally determined magnitude is motion.
In the heavenly bodies, motion is free motion,
one which is only determined by the concept from which alone,
consequently, its magnitudes equally depend (see above);
but in the organic body this free motion is reduced
to one which is arbitrary or mechanically regular,
that is, to one which is totally abstract and formal.

CHAPTER 1

Specific quantity

Qualitative quantity is,

first, an immediate, specific quantum; and this quantum,

second, in relating itself to another,
becomes a quantitative specifying,
a sublating of the indifferent quantum.
This measure is to this extent a rule
and contains the two moments of measure as different;
namely, the quantitative determinateness
and the external quantum as existing in themselves.
In this difference, however, the two sides become qualities,
and the rule becomes a relation of the two;
measure presents itself thereby,

third, as a relation of qualities that have one single measure at first;
a measure, however, which further specifies itself in itself
into a difference of measures.

CHAPTER 2

Real measure

Measure is now determined as a connection of measures
that make up the quality of distinct self-subsisting somethings,
or, in more common language, things.
The relations of measures just considered belong
to abstract qualities like space and time;
further examples of these now to be considered
are specific gravity and then chemical properties,
that is, determinations of concrete material existence.
Space and time are also moments of these measures,
but are now subordinated to other determinations
and no longer behave relative to one another
only according to their own conceptual determination.
In the case of sound, for instance,
the time within which a certain number of vibrations occur,
the spatial width and thickness of the sounding body,
are moments of its determination.
But the magnitudes of such idealized moments are externally determined;
they no longer assume the form of a ratio of powers
but relate in the usual direct way,
and harmony is reduced to the strictly
external simplicity of numbers in relations
which are most easy to grasp;
they therefore afford a satisfaction
which is the exclusive reserve of the senses,
for there is nothing there of representation,
imagery, thought, or the like,
that would satisfy spirit.
Since the sides which now constitute
the relation of measure are themselves measures,
but at the same time real somethings, their measures are,
in the first instance, immediate measures,
and the relations in them direct relations.
We now have to examine the further determination of
the relation of such relations.
Measure, now real measure, is as follows.

First, it is the independent measure of a type of body,
a measure which relates to other measures
and, in thus relating to them, specifies them as
well as their self-subsistent materiality.
This specification, as an external
connecting reference to many others in general,
produces other relations, and consequently other measures;
the specific self-subsistence, for its part,
does not remain fixed in one direct relation
but passes over into specific determinacies,
and this is a series of measures.

Second, the direct relations that thus result are in themselves
determinate and exclusive measures (elective affinities).
But because they are at the same time only
quantitatively different from one another,
what we have  is a progression of relations
which is in part merely external,
but is also interrupted by qualitative relations,
forming a nodal line of specifically self-subsisting things.

Third, what emerges in this progression for measure, however,
is the measureless:
the measureless in general and more specifically
the infinitude of measure in which the mutually exclusive
forms of self-subsistence are one with each other,
and anything self-subsistent comes to stand
in negative reference to itself.

Chapter 3

The becoming of essence
