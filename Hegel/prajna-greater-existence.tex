
A. EXISTENCE AS SUCH

In existence
(a) as such, its determinateness is first
(b) to be distinguished as quality.
The latter, however, is to be taken in
both the two determinations of
existence as reality and negation.
In these determinacies, however,
existence is equally reflected into itself,
and, as so reflected, it is posited as
(c) something, an existent.

a. Existence in general

Existence proceeds from becoming.
It is the simple oneness of being and nothing.
On account of this simplicity,
it has the form of an immediate.
Its mediation, the becoming, lies behind it;
it has sublated itself,
and existence therefore appears as a first
from which the forward move is made.
It is at first in the one-sided determination of being;
the other determination which it contains, nothing,
will likewise come up in it,
in contrast to the first.

It is not mere being but existence,
or Dasein [in German];
according to its [German] etymology,
it is being (Sein) in a certain place (da).
But the representation of space does not belong here.
As it follows upon becoming,
existence is in general
being with a non-being,
so that this non-being is taken up
into simple unity with being.
Non-being thus taken up into being
with the result that the concrete whole is
in the form of being, of immediacy,
constitutes determinateness as such.

The whole is likewise in the form
or determinateness of being,
since in becoming being has likewise
shown itself to be only a moment:
something sublated, negatively determined.
It is such, however, for us, in our reflection;
not yet as posited in it.
What is posited, however, is
the determinateness as such of existence,
as is also expressed by the da (or “there”) of the Dasein.
The two are always to be clearly distinguished.
Only that which is posited in a concept
belongs in the course of the elaboration
of the latter to its content.
Any determinateness not yet posited
in the concept itself
belongs instead to our reflection,
whether this reflection is directed to
the nature of the concept itself
or is a matter of external comparison.
To remark on a determinateness
of this last kind can only be
for the clarification or anticipation of the whole
that will transpire in the course of the development itself.
That the whole, the unity of being and nothing,
is in the one-sided determinateness of being
is an external reflection;
but in negation, in something and other,
and so forth, it will become posited.
It was necessary here to call attention to
the distinction just given;
but to comment on all
that reflection can allow itself,
to give an account of it,
would lead to a long-winded anticipation
of what must transpire in the fact itself.
Although such reflections may
serve to facilitate a general overview
and thus facilitate understanding,
they also bring the disadvantage of
being seen as unjustified assertions,
unjustified grounds and foundations,
of what is to follow.
They should be taken for no more than
what they are supposed to be
and should be distinguished from
what constitutes a moment in
the advance of the fact itself.

Existence corresponds to being in the preceding sphere.
But being is the indeterminate;
there are no determinations that therefore transpire in it.
But existence is determinate being, something concrete;
consequently, several determinations,
several distinct relations of its moments,
immediately emerge in it.

b. Quality

On account of the immediacy
with which being and nothing are
one in existence, neither oversteps the other;
to the extent that existence is existent,
to that extent it is non-being;
it is determined.
Being is not the universal,
determinateness not the particular.
Determinateness has yet to detach itself from being;
nor will it ever detach itself from it,
since the now underlying truth is
the unity of non-being with being;
all further determinations will transpire on this basis.
But the connection which determinateness now has
with being is one of the immediate unity of the two,
so that as yet no differentiation between the two is posited.

Determinateness thus isolated by itself,
as existent determinateness, is quality:
something totally simple, immediate.
Determinateness in general is the more universal
which, further determined, can be
something quantitative as well.
On account of this simplicity,
there is nothing further to say
about quality as such.

Existence, however, in which
nothing and being are equally contained,
is itself the measure of the one-sidedness of
quality as an only immediate or existent determinateness.
Quality is equally to be posited in the determination of nothing,
and the result is that the immediate or existent determinateness is
posited as distinct, reflected, and the nothing,
as thus the determinate element of determinateness,
will equally be something reflected, a negation.
Quality, in the distinct value of existent, is reality;
when affected by a negating, it is negation in general,
still a quality but one that counts as a lack
and is further determined as limit, restriction.

Both are an existence, but in reality,
as quality with the accent on being an existent,
that it is determinateness
and hence also negation is concealed;
reality only has, therefore,
the value of something positive
from which negating, restriction, lack, are excluded.
Negation, for its part, taken as mere lack,
would be what nothing is;
but it is an existence, a quality,
only determined with a non-being.

c. Something

In existence its determinateness has been distinguished as quality;
in this quality as something existing, the distinction exists:
the distinction of reality and negation.
Now though these distinctions are present in existence,
they are just as much null and sublated.
Reality itself contains negation;
it is existence, not indeterminate or abstract being.
Negation is for its part equally existence,
not the supposed abstract nothing
but posited here as it is in itself,
as existent, as belonging to existence.
Thus quality is in general unseparated from existence,
and the latter is only determinate, qualitative being.

This sublating of the distinction is more than
the mere retraction and external re-omission of it,
or a simple return to the simple beginning,
to existence as such.
The distinction cannot be left out, for it is.
Therefore, what de facto is at hand is this:
existence in general, distinction in it,
and the sublation of this distinction;
the existence, not void of distinctions as at the beginning,
but as again self-equal through the sublation of the distinction;
the simplicity of existence mediated through this sublation.
This state of sublation of the distinction is
existence's own determinateness;
existence is thus being-in-itself;
it is existent, something.

Something is the first negation of negation,
as simple existent self-reference.
Existence, life, thought, and so forth,
essentially take on the determination of an existent being,
a living thing, a thinking mind (“I”), and so forth.
This determination is of the highest importance
if we do not wish to halt at existence, life, thought,
and so forth, as generalities, also not at Godhood (instead of God).
In common representation, something rightly carries
the connotation of a real thing.
Yet it still is a very superficial determination,
just as reality and negation, existence and its determinateness,
though no longer the empty being and nothing,
still are quite abstract determinations.
For this reason they also are the most common expressions,
and a reflection that is still philosophically
unschooled uses them the most;
it casts its distinctions in them,
fancying that in them it has something
really well and firmly determined.
As something, the negative of the negative is
only the beginning of the subject;
its in-itselfness is still quite indeterminate.
It determines itself further on,
at first as existent-for-itself and so on,
until it finally obtains in the concept
the intensity of the subject.
At the base of all these determinations
there lies the negative unity with itself.
In all this, however, care must be taken
to distinguish the first negation, negation as negation in general,
from the second negation, the negation of negation
which is concrete, absolute negativity,
just as the first is on the contrary only abstract negativity.

Something is an existent as the negation of negation,
for such a negation is the restoration of
the simple reference to itself;
but the something is thereby equally
the mediation of itself with itself.
Present in the simplicity of something,
and then with greater determinateness
in being-for-itself,
in the subject, and so forth,
this mediation of itself with itself is
also already present in becoming,
but only as totally abstract mediation;
mediation with itself is posited in the something
in so far as the latter is determined as a simple identity.
Attention can be drawn to the presence of mediation in general,
as against the principle of the alleged bare immediacy of a knowledge
from which mediation should be excluded.
But there is no further need to draw
particular attention to the moment of mediation,
since it is to be found everywhere and on all sides,
in every concept.

This mediation with itself which something is in itself,
when taken only as the negation of negation,
has no concrete determinations for its sides;
thus it collapses into the simple unity which is being.
Something is, and is therefore also an existent.
Further, it is in itself also becoming,
but a becoming that no longer has only
being and nothing for its moments.
One of these moments, being, is now
existence and further an existent.
The other moment is equally an existent,
but determined as the negative of something; an other.
As becoming, something is a transition,
the moments of which are themselves something,
and for that reason it is an alteration,
a becoming that has already become concrete.
At first, however, something alters only in its concept;
it is not yet posited in this way, as mediated and mediating,
but at first only as maintaining itself
simply in its reference to itself;
and its negative is posited as equally qualitative,
as only an other in general.

B. FINITUDE

(a) Something and other:
at first they are indifferent to one another;
an other is also an immediate existent, a something;
the negation thus falls outside both.
Something is in itself in contrast to its being-for-other.
But the determinateness belongs also to its in-itself, and

(b) the determination of this in-itself
in turn passes over into constitution,
and this latter, as identical with determination,
constitutes the immanent and at the same time
negated being-for-another,
the limit of something which

(c) is the immanent determination of the something itself,
and the something thus is the finite.

In the first division where existence in general was considered,
this existence had, as at first taken up, the determination of an existent.
The moments of its development, quality and something,
are therefore of equally affirmative determination.
The present division, on the contrary, develops
the negative determination which is present in existence
and was there from the start only as negation in general.
It was then the first negation but has now been determined to
the point of the being-in-itself of the something,
the point of the negation of negation.

(a) Something and other:

1. Something and other are,
first, both existents or something.
Second, each is equally an other.
It is indifferent which is named first,
and just for this reason it is named something
(in Latin, when they occur in a proposition,
both are aliud, or “the one, the other,” alius alium;
in the case of an alternating relation,
the analogous expression is alter alterum).
If of two beings we call the one A and the other B,
the B is the one which is first determined as other.
But the A is just as much the other of the B.
Both are other in the same way.
“This” serves to fix the distinction
and the something which is to be taken in the affirmative sense.
But “this” also expresses the fact that the distinction,
and the privileging of one something,
is a subjective designation that falls outside the something itself.
The whole determinateness falls on the side of this external pointing;
also the expression “this” contains no distinctions;
each and every something is just as good a “this” as any other.
By “this” we mean to express something completely determinate,
overlooking the fact that language, as a work of the understanding,
only expresses the universal, albeit naming it as a single object.
But an individual name is something meaningless in the sense
that it does not express a universal.
It appears as something merely posited
and arbitrary for the same reason
that proper names can also be arbitrarily picked,
arbitrarily given as well as arbitrarily altered.

Otherness thus appears as a determination
alien to the existence thus pointed at,
or the other existence as outside this one existence,
partly because the one existence is determined as other
only by being compared by a Third,
and partly because it is so determined
only on account of the other which is outside it,
but is not an other for itself.
At the same time, as has been remarked,
even for ordinary thinking every existence
equally determines itself as an other existence,
so that there is no existence
that remains determined simply as an existence,
none which is not outside an existence
and therefore is not itself an other.

Both are determined as something as well as other:
thus they are the same and there is as yet
no distinction present in them.
But this sameness of determinations, too,
falls only within external reflection,
in the comparison of the two;
but the other, as posited at first,
though an other with reference to something,
is other also for itself apart from the something.

Third, the other is therefore to be taken in isolation,
with reference to itself, has to be taken abstractly as the other,
the 'to heteron' of Plato who opposes it to the one
as a moment of totality,
and in this way ascribes to the other a nature of its own.
Thus the other, taken solely as such,
is not the other of something,
but is the other within, that is, the other of itself.
Such an other, which is the other by its own determination,
is physical nature; nature is the other of spirit;
this, its determination, is at first
a mere relativity expressing not a quality of nature itself
but only a reference external to it.
But since spirit is the true something,
and hence nature is what it is within only in contrast to spirit,
taken for itself the quality of nature is just this,
to be the other within, that which-exists-outside-itself
(in the determinations of space, time, matter).

The other which is such for itself is the other within it,
hence the other of itself and so the other of the other;
therefore, the absolutely unequal in itself,
that which negates itself, alters itself.
But it equally remains identical with itself,
for that into which it alters is the other,
and this other has no additional determination;
but that which alters itself is not determined in
any other way than in this, to be an other;
in going over to this other, it only unites with itself.
It is thus posited as reflected into itself
with sublation of the otherness,
a self-identical something from which the otherness,
which is at the same time a moment of it, is therefore distinct,
itself not appertaining to it as something.

2. The something preserves itself in its non-being;
it is essentially one with it, and essentially not one with it.
It therefore stands in reference to an otherness
without being just this otherness.
The otherness is at once contained in it
and yet separated from it;
it is being-for-other.
Existence as such is an immediate, bare of references;
or, it is in the determination of being.
However, as including non-being within itself,
existence is determinate being,
being negated within itself,
and then in the first instance an other;
but, since in being negated it preserves itself
at the same time, it is only being-for-other.
It preserves itself in its non-being and is being;
not, however, being in general but being with reference
to itself in contrast to its reference to the other,
as self-equality in contrast to its inequality.
Such a being is being-in-itself.

Being-for-other and being-in-itself constitute
the two moments of something.
There are here two pairs of determinations:
(1) something and other;
(2) being-for-other and being-in-itself.
The former contain the non-connectedness of their determinateness;
something and other fall apart.
But their truth is their connection;
being-for-other and being-in-itself are
therefore the same determinations posited as
moments of one and the same unity,
as determinations which are connections
and which, in their unity,
remain in the unity of existence.
Each thus itself contains within it, at the same time,
also the moment diverse from it.
Being and nothing in their unity, which is existence,
are no longer being and nothing
(these they are only outside their unity);
so in their restless unity, in becoming,
they are coming-to-be and ceasing-to-be.
In the something, being is being-in-itself.
Now, as self-reference, self-equality,
being is no longer immediately,
but is self-reference only as the non-being of otherness
(as existence reflected into itself).
The same goes for non-being:
as the moment of something in this
unity of being and non-being:
it is not non-existence in general
but is the other, and more determinedly,
according as being is at the same time distinguished from it,
it is reference to its non-existence, being-for-other.

Hence being-in-itself is, first,
negative reference to non-existence;
it has otherness outside it and is opposed to it;
in so far as something is in itself,
it is withdrawn from being-other and being-for-other.
But, second, it has non-being also right in it;
for it is itself the non-being of being-for-other.
But being-for-other is, first, the negation of
the simple reference of being to itself
which, in the first place, is supposed
to be existence and something;
in so far as something is in an other or for an other,
it lacks a being of its own.
But, second, it is not non-existence as pure nothing;
it is non-existence that points to being-in-itself
as its being reflected into itself,
just as conversely the being-in-itself points to being-for-other.

3. Both moments are determinations of one and the same,
namely of something.
Something is in-itself in so far as it has returned
from the being-for-other back to itself.
But something has also a determination or circumstance,
whether in itself (here the accent is on the in) or in it;
in so far as this circumstance is in it externally,
it is a being-for-other.
This leads to a further determination.
Being-in-itself and being-for-other
are different at first.
But that something also has in it what it is in itself and
conversely is in itself also what it is as being-for-other
this is the identity of being-in-itself and being-for-other,
in accordance with the determination
that the something is itself
one and the same something of both moments,
and these are in it, therefore, undivided.
This identity already occurs formally in the sphere of existence,
but more explicitly in the treatment of essence
and later of the relations of interiority and externality,
and in the most determinate form in the treatment of the idea,
as the unity of concept and actuality.
Opinion has it that with the in-itself
something lofty is being said, as with the inner;
but what something is only in itself, is also only in it;
in-itself is a merely abstract,
and hence itself external determination.
The expressions:
there is nothing in it,
or there is something in it,
imply, though somewhat obscurely,
that what is in a thing also pertains
to its in-itselfness, to its inner, true worth.

It may be observed that here we have
the meaning of the thing-in-itself.
It is a very simple abstraction,
though it was for a while a very important determination,
something sophisticated, as it were,
just as the proposition that we know nothing of
what things are in themselves was a much valued piece of wisdom.
Things are called “in-themselves” in so far as abstraction
is made from all being-for-other, which really means,
in so far as they are thought without all determination, as nothing.
In this sense, of course, it is impossible to know
what the thing-in-itself is.
For the question “what?” calls for determinations to be produced;
but since the things of which the determinations are called for
are at the same time presumed to be things-in-themselves,
which means precisely without determination,
the impossibility of an answer is thoughtlessly implanted in the question,
or else a senseless answer is given.
The thing-in-itself is the same as that absolute
of which nothing is known except that in it all is one.
What there is in these things-in-themselves is therefore very well known;
they are as such nothing but empty abstractions void of truth.
What, however, the thing-in-itself in truth is,
what there basically is in it,
of this the Logic is the exposition.
But in this Logic something better is understood by the in-itself
than an abstraction, namely, what something is in its concept;
but this concept is in itself concrete:
as concept, in principle conceptually graspable;
and, as determined and as the connected whole
of its determinations, inherently cognizable.

Being-in-itself has at first the being-for-other
as a moment standing over against it.
But positedness also comes to be positioned over against it,
and, although in this expression being-for-other is also included,
the expression still contains the determination of the bending back,
which has already occurred,
of that which is not in itself into that wherein it is
positive, and this is its being-in-itself.
Being-in-itself is normally to be taken
as an abstract way of expressing the concept;
positing, strictly speaking, first occurs
in the sphere of essence, of objective reflection;
the ground posits that which is grounded through it;
more strongly, the cause produces an effect,
an existence whose subsistence is immediately negated
and which carries the meaning that it has its substance,
its being, in an other.
In the sphere of being, existence only emerges out of becoming.
Or again, with the something an other is posited;
with the finite, an infinite;
but the finite does not bring forth the infinite,
does not posit it.
In the sphere of being, the self-determining of
the concept is at first only in itself or implicit,
and for that reason it is called a transition or passing over.
And the reflecting determinations of being,
such as something and other,
or finite and infinite,
although they essentially point to one another,
or are as being-for-other,
also stand on their own qualitatively; the other exists;
the finite, like the infinite, is equally to be regarded
as an immediate existent that stands firm on its own;
the meaning of each appears complete even without its other.
The positive and the negative, on the contrary, cause and effect,
however much they are taken in isolation,
have at the same time no meaning each without the other;
their reflective shining in each other,
the shine in each of its other,
is present right in them.
In the different cycles of determination
and especially in the progress of the exposition,
or, more precisely, in the progress of the concept
in the exposition of itself,
it is of capital concern always to clearly distinguish
what still is in itself or implicitly
and what is posited,
how determinations are in the concept
and how they are as posited
or as existing-for-other.
This is a distinction that belongs only to
the dialectical development and one unknown
to metaphysical philosophizing
(to which the critical also belongs);
the definitions of metaphysics,
like its presuppositions, distinctions, and conclusions,
are meant to assert and produce only the existent
and that, too, as existent-in-itself.
In the unity of the something with itself,
being-for-other is identical with its in-itself;
the being-for-other is thus in the something.
The determinateness thus reflected into itself is
therefore again a simple existent
and hence again a quality: determination.

b. Determination, constitution, and limit

The in-itself, in which the something is
reflected into itself from its being-for-other,
no longer is an abstract in-itself
but, as the negation of its being-for-other,
is mediated through this latter,
which is thus its moment.
It is not only the immediate identity of
the something with itself,
but the identity by virtue of which
the something also has present in it
what it is in itself;
the being-for-other is present in it
because the in-itself is the sublation of it,
is in itself from it;
but, because it is still abstract,
and therefore essentially affected with negation,
it is equally affected with being-for-other.
We have here not only quality and reality,
existent determinateness,
but determinateness existent-in-itself;
and the development consists in positing
such determinateness as thus immanently reflected.

1. The quality which in the simple
something is an in-itself
essentially in unity with
the something's other moment,
its being-in-it,
can be named its determination,
provided that this word is distinguished,
in a more precise signification,
from determinateness in general.
Determination is affirmative determinateness;
it is the in-itself by which
a something abides in its existence
while involved with an other
that would determine it,
by which it preserves itself
in its self-equality,
holding on to it in its being-for-other.
Something fulfills its determination
to the extent that the further determinateness,
which variously accrues to it
in the measure of its being-in-itself
as it relates to an other,
becomes its filling.
Determination implies that
what something is in itself
is also present in it.

The determination of the human being,
its vocation, is rational thought:
thinking in general is his simple determinateness;
by it the human being is distinguished from the brute;
he is thinking in himself, in so far as this thinking is
distinguished also from his being-for-other,
from his own natural and sensuous being
that brings him in immediate association with the other.
But thinking is also in him;
the human being is himself thinking,
he exists as thinking,
thought is his concrete existence and actuality;
and, further, since thinking is in his existence
and his existence is in his thinking,
thinking is concrete,
must be taken as having content and filling;
it is rational thought and as such
the determination of the human being.
But even this determination is again
only in itself, as an ought, that is to say,
it is, together with the filling embodied in its in-itself,
in the form of an in-itself in general
as against the existence which is not embodied in it
but still lies outside confronting it,
immediate sensibility and nature.

2. The filling of the being-in-itself with determinateness
is also distinct from the determinateness
which is only being-for-other
and remains outside the determination.
For in the sphere of the qualitative,
the distinguished terms are left, in their sublated being,
also with an immediate, qualitative being contrasting them.
That which the something has in it thus separates itself
and is from this side the external existence of
the something and also its existence,
but not as belonging to its being-in-itself.
Determinateness is thus constitution.

Constituted in this or that way,
the something is caught up in external influences
and in external relationships.
This external connection on which the constitution depends,
and the being determined through an other,
appear as something accidental.
But it is the quality of the something
to be given over to this externality
and to have a constitution.

In so far as something alters,
the alteration falls on the side of its constitution;
the latter is that in the something which becomes an other.
The something itself preserves itself in the alteration;
the latter affects only this unstable surface
of the something's otherness, not its determination.

Determination and constitution are thus distinct from each other;
something, according to its determination,
is indifferent to its constitution.
But that which the something has in it is
the middle term of this syllogism connecting the two,
determination and constitution.
Or, rather, the being-in-the-something showed itself
to fall apart into these two extremes.
The simple middle term is determinateness as such;
its identity belongs to determination
just as well as to constitution.
But determination passes over into constitution on its own,
and constitution into determination.
This is implied in what has been said.
The connection, upon closer consideration, is this:
in so far as that which something is in itself is also in it,
the something is affected with being-for-other;
determination is therefore open, as such,
to the relation with other.
Determinateness is at the same time moment,
but it contains at the same time the qualitative distinction
of being different from being-in-itself,
of being the negative of the something,
another existence.
This determinateness which thus holds the other in itself,
united with the being-in-itself,
introduces otherness in the latter or in determination,
and determination is thereby reduced to constitution.
Conversely, the being-for-other, isolated as constitution
and posited on its own, is in it the same
as what the other as such is, the other in it,
that is, the other of itself;
but it consequently is self-referring existence,
thus being-in-itself with a determinateness, therefore determination.
Consequently, inasmuch as the two are also to be held apart,
constitution, which appears to be grounded in something external,
in an other in general, also depends on determination,
and the determining from outside is at the same time
determined by the something's own immanent determination.
And further, constitution belongs to that
which something is in itself:
something alters along with its constitution.

This altering of something is no longer
the first alteration of something merely
in accordance with its being-for-other.
The first was an alteration only implicitly present,
one that belonged to the inner concept;
now the alteration is also posited in the something.
The something itself is further determined,
and negation is posited as immanent to it,
as its developed being-in-itself.

The transition of determination and constitution
into each other is at first the sublation of their distinction,
and existence or something in general is thereby posited;
moreover, since this something in general results
from a distinction that also includes
qualitative otherness within it,
there are two somethings.
But these are, with respect to each other,
not just others in general,
so that this negation would still be abstract
and would occur only in the comparison of the two;
rather the negation now is immanent to the somethings.
As existing, they are indifferent to each other,
but this, their affirmation, is no longer immediate:
each refers itself to itself through the intermediary of
the sublation of the otherness which in determination is
reflected into the in-itselfness.

Something behaves in this way in relation
to the other through itself;
since otherness is posited in it as its own moment,
its in-itselfness holds negation in itself,
and it now has its affirmative existence
through its intermediary alone.
But the other is also qualitatively distinguished
from this affirmative existence
and is thus posited outside the something.
The negation of its other is only
the quality of the something,
for it is in this sublation of its other
that it is something.
The other, for its part,
truly confronts an existence
only with this sublation;
it confronts the first something only externally,
or, since the two are in fact inherently joined together,
that is, according to their concept,
their connectedness consists in this,
that existence has passed over into otherness,
something into other;
that something is just as much an other as the other is.
Now in so far as the in-itselfness is
the non-being of the otherness
that is contained in it
but is at the same time
also distinct as existent,
something is itself negation,
the ceasing to be of an other in it;
it is posited as behaving negatively in relation to
the other and in so doing preserving itself.
This other, the in-itselfness of
the something as negation of the negation,
is the something's being-in-itself,
and this sublation is as simple negation
at the same time in it, namely, as its negation
of the other something external to it.
It is one determinateness of the two somethings
that, on the one hand, as negation of the negation,
is identical with the in-itselfness of the somethings,
and also, on the other hand, since these negations are
to each other as other somethings,
joins them together of their own accord
and, since each negation negates the other,
equally separates them.
This determinateness is limit.

3. Being-for-other is indeterminate, affirmative
association of something with its other;
in limit the non-being-for-other is emphasized,
the qualitative negation of the other,
which is thereby kept out of the something
that is reflected into itself.
We must see the development of this concept,
a development that will rather look like
confusion and contradiction.
Contradiction immediately raises its head
because limit, as an internally reflected negation of something,
ideally holds in it the moments of something and other,
and these, as distinct moments, are at the same time
posited in the sphere of existence
as really, qualitatively, distinct.

(a) Something is therefore immediate, self-referring existence
and at first it has a limit with respect to an other;
limit is the non-being of the other,
not of the something itself;
in limit, something marks the boundary of its other.
But other is itself a something in general.
The limit that something has with respect to
an other is, therefore, also the limit of
the other as a something;
it is the limit of this something in virtue of which
the something holds the first something
as its other away from itself,
or is a non-being of that something.
The limit is thus not only the non-being of the other,
but of the one something just as of the other,
and consequently of the something in general.

But the limit is equally, essentially,
the non-being of the other;
thus, through its limit,
something at the same time is.
In limiting, something is of course thereby
reduced to being limited itself;
but, as the ceasing of the other in it,
its limit is at the same time itself
only the being of the something;
this something is what it is by virtue of it,
has its quality in it.
This relation is the external appearance of the fact
that limit is simple negation or the first negation,
whereas the other is, at the same time,
the negation of the negation,
the in-itselfness of the something.
Something, as an immediate existence,
is therefore the limit with respect
to another something;
but it has this limit in it
and is something through the mediation of that limit,
which is just as much its non-being.
The limit is the mediation in virtue of
which something and other each both is and is not.

(b) Now in so far as something in its limit
both is and is not,
and these moments are an immediate, qualitative distinction,
the non-existence and the existence of the something
fall outside each other.
Something has its existence outside its limit
(or, as representation would also have it, inside it);
in the same way the other, too,
since it is something, has it outside it.
The limit is the middle point between the two
at which they leave off.
They have existence beyond each other,
beyond their limit;
the limit, as the non-being of each,
is the other of both.

It is in accordance with this difference
of the something from its limit
that the line appears as line outside its limit, the point;
the plane as plane outside the line;
the solid as solid only outside its limiting plane.
This is the aspect of limit that
first occurs to figurative representation
(the self-external-being of the concept)
and is also most commonly assumed
in the context of spatial objects.

(c) But further, something as it is outside the limit,
as the unlimited something, is only existence in general.
As such, it is not distinguished from its other;
it is only existence
and, therefore, it and its other have the same determination;
each is only something in general or each is other;
and so both are the same.
But this, their at first immediate existence,
is now posited in them as limit:
in it both are what they are, distinct from each other.
But it is also equally their common distinguishedness,
the unity and the distinguishedness of both,
just like existence.
This double identity of the two,
existence and limit, contains this:
that something has existence only in limit,
and that, since limit and immediate existence are each
at the same time the negative of each other,
the something, which is now only in its limit,
equally separates itself from itself,
points beyond itself to its non-being
and declares it to be its being,
and so it passes over into it.
To apply this to the preceding example,
the one determination is this:
that something is what it is only in its limit.
Therefore, the point is the limit of line,
not because the latter just ceases at the point
and has existence outside it;
the line is the limit of plane,
not because the plane just ceases at it;
and the same goes for the plane as the limit of solid.
Rather, at the point the line also begins;
the point is its absolute beginning,
and if the line is represented
as unlimited on both its two sides,
or, as is said, as extended to infinity,
the point still constitutes its element,
just as the line constitutes the element of the plane,
and the plane that of the solid.
These limits are the principle
of that which they delimit;
just as one, for instance,
is as hundredth the limit,
but also the element,
of the whole hundred.

The other determination is the unrest of
the something in its limit in which it is immanent,
the contradiction that propels it beyond itself.
Thus the point is this dialectic of itself becoming line;
the line, the dialectic of becoming plane;
the plane, of becoming total space.
A second definition is given of line, plane, and whole space
which has the line come to be through the movement of the point;
the plane through the movement of the line, and so forth.
This movement of the point, the line, and so forth, is
however viewed as something accidental,
or as movement only in figurative representation.
In fact, however, this view is taken back by supposing that
the determinations from which the line, and so forth,
originate are their elements and principles,
and these are, at the same time,
nothing else but their limits;
the coming to be is not considered as accidental
or only as represented.
That the point, the line, the plane, are per se
self-contradictory beginnings which on their own
repel themselves from themselves,
and consequently that the point passes over
from itself into the line through its concept,
moves in itself and makes the line come to be,
and so on all this lies in the concept of the limit
which is immanent in the something.
The application itself, however,
belongs to the treatment of space;
as an indication of it here, we can say that
the point is the totally abstract limit,
but in a determinate existence;
this existence is still taken in total abstraction,
it is the so-called absolute, that is, abstract space,
the absolutely continuous being-outside-one-another.
Inasmuch as the limit is not abstract negation,
but is rather in this existence,
inasmuch as it is spatial determinateness,
the point is spatial, is the contradiction
of abstract negation and continuity
and is, for that reason, the transition
as it occurs and has already occurred
into the line, and so forth.
And so there is no point,
just as there is no line or plane.

The something, posited with its immanent limit
as the contradiction of itself by virtue of which
it is directed and driven out and beyond itself,
the finite.

c. Finitude

Existence is determinate.
Something has a quality,
and in this quality it is
not only determined but delimited;
its quality is its limit and, affected by it,
something remains affirmative, quiescent existence.
But, so developed that the opposition of its existence
and of the negation as the limit immanent to this existence is
the very in-itselfness of the something,
and this is thus only becoming in it,
this negation constitutes the finitude of the something.
When we say of things that they are finite,
we understand by this that they not only have a determinateness,
that their quality is not only reality
and existent determination,
that they are not merely limited
and as such still have existence outside their limit,
but rather that non-being constitutes their nature, their being.
Finite things are, but in their reference to themselves
they refer to themselves negatively
in this very self-reference
they propel themselves beyond themselves,
beyond their being.
They are, but the truth of this being is
(as in Latin) their finis, their end.
The finite does not just alter,
as the something in general does,
but perishes, and its perishing is
not just a mere possibility,
as if it might be without perishing.
Rather, the being as such of finite things is
to have the germ of this transgression
in their in-itselfness:
the hour of their birth is the hour of their death.

(a) The immediacy of finitude

The thought of the finitude of things
brings this mournful note with it
because finitude is qualitative negation driven to the extreme,
and in the simplicity of such a determination
there is no longer left to things
an affirmative being distinct from their determination,
as things destined to ruin.
Because of this qualitative simplicity of negation
that has returned to the abstract opposition of
nothing and perishing to being,
finitude is the most obstinate of
the categories of the understanding;
negation in general, constitution, limit,
are compatible with their other, with existence;
even the abstract nothing, by itself,
is given up as an abstraction;
but finitude is negation fixed in itself
and, as such, stands in stark contrast to its affirmative.
The finite thus does indeed let itself be submitted to flux;
this is precisely what it is,
that it should come to an end,
and this end is its only determination.
Its refusal is rather to let itself be brought
affirmatively to its affirmative, the infinite,
to be associated with it;
it is therefore inseparably posited with its nothing,
and thereby cut off from any reconciliation
with its other, the affirmative.
The determination of finite things does not go past their end.
The understanding persists in this sorrow of finitude,
for it makes non-being the determination of things
and, at the same time, this non-being imperishable and absolute.
Their transitoriness would only pass away in their other,
in the affirmative;
their finitude would then be severed from them;
but this finitude is their unalterable quality, that is,
their quality which does not pass over into their other, that is,
not into the affirmative;
and so finitude is eternal.

This is a very important consideration.
But that the finite is absolute is
certainly not a standpoint that any philosophy or outlook,
or the understanding, would want to endorse.
The opposite is rather expressly present
in the assertion of finitude:
the finite is the restricted, the perishable,
the finite is only the finite, not the imperishable;
all this is immediately part and parcel
of its determination and expression.
But all depends on whether
in one's view of finitude its being is insisted on,
and the transitoriness thus persists,
or whether the transitoriness and the perishing perish.
The fact is that this perishing of the perishing does not happen
on precisely the view that would make
the perishing the final end of the finite.
The official claim is that the finite is
incompatible with the infinite
and cannot be united with it;
that the finite is absolutely opposed to the infinite.
Being, absolute being, is ascribed to the infinite.
The finite remains held fast over against it as its negative;
incapable of union with the infinite,
it remains absolute on its own side;
from the affirmative, from the infinite,
it would receive affirmation and thus it would perish;
but a union with the infinite is precisely
what is declared impossible.
If the finite were not to persist over against the infinite
but were to perish, its perishing, as just said,
would then be the last of it;
not its affirmative, which would be only
a perishing of the perishing.
However, if it is not to perish into the affirmative
but its end is rather to be grasped as a nothing,
then we are back at that first, abstract nothing
that itself has long since passed away.

With this nothing, however, which is supposed to be only nothing
but to which a reflective existence is nevertheless granted
in thought, in representation or in speech,
the same contradiction occurs as we have
just indicated in connection with the finite,
except that in the nothing it just occurs
but in the finite it is instead expressed.
In the one case, the contradiction appears as subjective;
in the other, the finite is said to stand
in perpetual opposition to the infinite,
in itself to be null, and to be as null in itself.
This is now to be brought to consciousness.
The development of the finite will show that,
expressly as this contradiction, it collapses internally,
but that, in this collapse, it actually resolves the contradiction;
it will show that the finite is not just perishable, and that it perishes,
but that the perishing, the nothing, is rather not the last of it;
that the perishing rather perishes.

(b) Restriction and the ought

This contradiction is indeed abstractly present
by the very fact that the something is finite,
or that the finite is.
But something or being is no longer posited abstractly
but reflected into itself,
and developed as being-in-itself
that has determination and constitution in it,
or, more determinedly still, in such a way
that it has a limit within it;
and this limit, as constituting
what is immanent to the something
and the quality of its being-in-itself,
is finitude.
It is to be seen what moments are contained
in this concept of the finite something.

Determination and constitution arose as
sides for external reflection,
but determination already contained otherness
as belonging to the in-itself of something.
On the one side, the externality of otherness is
within the something's own inwardness;
on the other side, it remains as otherness distinguished from it;
it is still externality as such, but in the something.
But further, since otherness is determined as limit,
itself as negation of the negation,
the otherness immanent to the something is posited as the
connection of the two sides,
and the unity of the something with itself
(to which both determination and constitution belong)
is its reference turned back upon itself,
the reference to it of its implicitly existing determination
that in it negates its immanent limit.
The self-identical in-itself thus refers
itself to itself as to its own non-being,
but as negation of the negation,
as negating that which at the same time retains existence in it,
for it is the quality of its in-itselfness.
Something's own limit, thus posited by it as a
negative which is at the same time essential,
is not only limit as such, but restriction.
But restriction is not alone in being posited as negative;
the negation cuts two ways, for that which it posits as negated is limit,
and limit is in general what is common to something and other,
and is also the determinateness of the in-itself of determination as such.
This in-itself, consequently, as negative reference to its limit
(which is also distinguished from it),
as negative reference to itself as restriction,
is the ought.

In order for the limit that is in every something to be a restriction,
the something must at the same time transcend it in itself,
must refer to it from within as to a non-existent.
The existence of something lies quietly indifferent,
as it were, alongside its limit.
But the something transcends its limit only
in so far as it is the sublatedness of the limit,
the negative in-itselfness over against it.
And inasmuch as the limit is as restriction
in the determination itself,
the something thereby transcends itself.

The ought therefore contains the double determination:
once, as a determination which has
an in-itselfness over against negation;
and again, as a non-being which, as restriction,
is distinguished from the determination
but is at the same time itself
a determination existing in itself.

The finite has thus determined itself
as connecting determination and limit;
in this connection, the determination is the ought
and the limit is the restriction.
Thus the two are both moments of the finite,
and therefore both themselves finite,
the ought as well as the restriction.
But only restriction is posited as the finite;
the ought is restricted only in itself,
and therefore only for us.
It is restricted by virtue of its reference
to the limit already immanent within it,
though this restriction in it is shrouded in in-itselfness,
for according to its determinate being,
that is, according to its determinateness
in contrast to restriction,
it is posited as being-in-itself.

What ought to be is, and at the same time is not.
If it were, it would not be what merely ought to be.
The ought has therefore a restriction essentially.
This restriction is not anything alien;
that which only ought to be is determination
now posited as it is in fact,
namely as at the same time
only a determinateness.

The being-in-itself of the something is
thus reduced in its determination
to the ought because the very thing
that constitutes the something's in-itselfness is,
in one and the same respect, as non-being;
or again, because in the in-itselfness,
in the negation of the negation,
the said being-in-itself is as
one negation (what negates) a unity with the other,
and this other, as qualitatively other, is the limit
by virtue of which that unity is as reference to it.
The restriction of the finite is not anything external,
but the finite's own determination is rather also its restriction;
and this restriction is both itself and the ought;
it is that which is common to both,
or rather that in which the two are identical.

But further, as “ought” the finite transcends its restriction;
the same determinateness which is its negation is also sublated,
and is thus its in-itself;
its limit is also not its limit.

As ought something is thus elevated above its restriction,
but conversely it has its restriction only as ought.
The two are indivisible.
Something has a restriction in so far as
it has negation in its determination,
and the determination is also
the being sublated of the restriction.

(c) Transition of the finite into the infinite

The ought contains restriction explicitly, for itself,
and restriction contains the ought.
Their mutual connection is the finite itself,
which contains them both in its in-itself.
These moments of its determination are qualitatively opposed;
restriction is determined as the negative of the ought,
and the ought equally as the negative of restriction.
The finite is thus in itself the contradiction of itself;
it sublates itself, it goes away and ceases to be.
But this, its result, the negative as such,
is (a) its very determination;
for it is the negative of the negative.
So, in going away and ceasing to be,
the finite has not ceased;
it has only become momentarily
an other finite which equally is,
however, a going-away as a going-over
into another finite, and so forth to infinity.
But, (b) if we consider this result more closely,
in its going-away and ceasing-to-be,
in this negation of itself,
the finite has attained its being-in-itself;
in it, it has rejoined itself.
Each of its moments contains precisely this result;
the ought transcends the restriction,
that is, it transcends itself;
but its beyond, or its other, is only restriction itself.
Restriction, for its part, immediately points
beyond itself to its other,
and this is the ought;
but this ought is the same
diremption of in-itselfness and determinateness
as is restriction;
it is the same thing;
in going beyond itself,
restriction thus equally rejoins itself.
This identity with itself, the negation of negation,
is affirmative being, is thus the other of the finite
which is supposed to have the first negation
for its determinateness;
this other is the infinite.

C. INFINITY

The infinite in its simple concept
can be regarded, first of all,
as a fresh definition of the absolute;
as self-reference devoid of determination,
it is posited as being and becoming.
The forms of existence have no place in
the series of determinations that can be
regarded as definitions of the absolute,
since the forms of that sphere are
immediately posited for themselves
only as determinacies, as finite in general.
But the infinite is accepted
unqualifiedly as absolute,
since it is explicitly determined
as the negation of the finite;
the restrictedness
(to which being and becoming would somehow be susceptible
even if they do not have it or exhibit it)
is thereby both explicitly referred to and denied in it.
But, in fact, by just this negation the infinite is
not already free from restrictedness and finitude.
It is essential to distinguish
the true concept of infinity
from bad infinity,
the infinite of reason from
the infinite of the understanding.
The latter is in fact a finitized infinite,
and, as we shall now discover,
in wanting to maintain the infinite
pure and distant from the finite,
the infinite is by that very fact
only made finite.
The infinite
(a) in simple determination,
is the affirmative as negation of the finite;
(b) but is thereby in alternating
determination with the infinite,
and is abstract, one-sided infinite;
(c) is the self-sublation of this infinite
and of the finite in one process.
This is the true infinite.

a. The infinite in general

The infinite is the negation of negation, the affirmative,
being that has reinstated itself out of restrictedness.
The infinite is, in a more intense sense
than the first immediate being;
it is the true being;
the elevation above restriction.
At the mention of the infinite, soul and spirit light up,
for in the infinite the spirit is at home,
and not only abstractly;
rather, it rises to itself, to the light of its thinking,
its universality, its freedom.
What is first given with the concept of the infinite is this,
that in its being-in-itself existence is determined
as finite and transcends restriction.
It is the very nature of the finite
that it transcend itself,
that it negate its negation
and become infinite.
Consequently, the infinite does not stand
above the finite as something ready-made by itself,
as if the finite stood fixed outside or below it.
Nor is it we only, as a subjective reason,
who transcend the finite into the infinite,
as if, in saying that the infinite is a concept of reason
and that through reason we elevate ourselves above things temporal,
we did this without prejudice to the finite,
without this elevation
(which remains external to the finite)
affecting it.
In so far as the finite itself is being elevated to infinity,
it is not at all an alien force that does this for it;
it is rather its nature to refer itself to itself as restriction
(both restriction as such and as ought)
and to transcend this restriction,
or rather, in this self-reference,
to have negated the restriction
and gone above and beyond it.
It is not in the sublation of the finite in general
that infinity in general comes to be,
but the finite is rather just this,
that through its nature it comes to be itself the infinite.
Infinity is its affirmative determination,
its vocation, what it truly is in itself.
The finite has thus vanished into the infinite
and what is, is only the infinite.

b. Alternating determination of finite and infinite

The infinite is;
in this immediacy it is at the same time
the negation of an other, of the finite.
And so, as existent and at the same time as the
non-being of an other,
it has fallen back into the category of the something,
of something determinate in general.
More precisely: the infinite is
the existence reflected into itself
which results from the mediating sublation
of determinateness in general
and is consequently posited as existence
distinct from its determinateness;
therefore, it has fallen back into
the category of something with a limit.
In accordance with this determinateness,
the finite stands over against
the infinite as real existence;
they thus remain outside each other,
standing in qualitative mutual reference;
the immediate being of the infinite
resurrects the being of its negation,
of the finite again, which seemed at first
to have vanished into the infinite.

But the infinite and the finite are not
in these referential categories only;
the two sides are further determined in addition
to being as mere others to each other.
Namely, the finite is restriction posited as restriction;
it is existence posited with the determination
that it passes over into what is its in-itself
and becomes infinite.
Infinity is the nothing of the finite,
the in-itself that the latter ought to be,
but it is this at the same time
as reflected within itself,
as realized ought, as only
affirmative self-referring being.
In infinity we have the satisfaction
that all determinateness, alteration, all restriction
and the ought itself together with it,
have vanished, are sublated,
and the nothing of the finite is posited.
As this negation of the finite is
the being-in-itself determined which,
as negation of negation, is in itself affirmative.
Yet this affirmation is qualitatively
immediate self-reference, being;
and, because of this, the affirmative is led back
to the category of being that has
the finite confronting it as an other;
its negative nature is posited as existent negation,
and hence as first and immediate negation.
The infinite is in this way burdened
with the opposition to the finite,
and this finite, as an other, remains a real existence
even though in its being-in-itself, in the infinite,
it is at the same time posited as sublated;
this infinite is that which is not finite:
a being in the determinateness of negation.
Contrasted with the finite,
with the series of existent determinacies,
of realities,
the infinite is indeterminate emptiness,
the beyond of the finite,
whose being-in-itself is not in its existence
(which is something determinate).

As thus posited over against the finite,
the two connected by the qualitative mutual reference of others,
the infinite is to be called the bad infinite,
the infinite of the understanding,
for which it counts as the highest,
the absolute truth.
The understanding believes that
it has attained satisfaction
in the reconciliation of truth
while it is in fact entangled in
unreconciled, unresolved, absolute contradictions.
And it is these contradictions,
into which it falls on every side whenever
it embarks on the application and explication
of these categories that belong to it,
that must make it conscious of the fact.

This contradiction is present in the very fact
that the infinite remains over against the finite,
with the result that there are two determinacies.
There are two worlds, one infinite and one finite,
and in their connection the infinite is only
the limit of the finite and thus only a determinate,
itself finite infinite.

This contradiction develops its content into more explicit forms.
The finite is the real existence which persists as such
even when it has gone over into its non-being, the infinite.
As we have seen, this infinite has for its determinateness,
over against the finite,
only the first, immediate negation,
just as the finite, as negated,
has over against this negation
only the meaning of an other
and is, therefore, still a something.
When, therefore, the understanding,
elevating itself above this finite world,
rises to what is the highest for it,
to the infinite,
the finite world remains for it
as something on this side here,
and, thus posited only above the finite,
the infinite is separated from the finite
and, for the same reason,
the finite from the infinite:
each is placed in a different location,
the finite as existence here,
and the infinite, although the being-in-itself of the finite,
there as a beyond, at a nebulous, inaccessible distance
outside which there stands, enduring, the finite.

As thus separated, they are just as much
essentially connected with each other,
through the very negation that divides them.
This negation connecting them
(these somethings reflected into themselves)
is the common limit of each over against the other;
and that, too, in such a way that each
does not merely have this limit in it over against the other,
but the negation is rather the in-itselfness of each;
each thus has for itself, in its separation
from the other, the limit in it.
But the limit is the first negation;
both are thus limited, finite, in themselves.
Yet, as each affirmatively refers itself to itself,
each is also the negation of its limit;
each thus immediately repels
the negation from itself as its non-being,
and, qualitatively severed from it,
posits it as an other being outside it:
the finite posits its non-being as this infinite,
and the infinite likewise the finite.
It is readily conceded that the
finite passes over into the infinite necessarily
(that is, through its determination)
and is thereby elevated to what is its in-itself,
for while the finite is indeed determined as subsistent existence,
it is at the same time also a null in itself
and therefore destined to self-dissolution;
whereas the infinite, although burdened with negation and limit,
is equally also determined as the existent in-itself,
so that this abstraction of self-referring affirmation is
what constitutes its determination,
and hence finite existence is not present in it.
But it has been shown that the infinite itself
attains affirmative being only by the mediation of negation,
as negation of negation,
and that when its affirmation thus attained is
taken as just simple, qualitative being,
the negation contained in it is demoted
to simple immediate negation
and, therefore, to determinateness and limit;
and these, then, are excluded from the infinite
as contradicting its in-itself;
they are posited as not belonging to it
but rather as opposed to its in-itself,
as the finite.
Since each is in it and through its determination
the positing of its other, the two are inseparable.
But this unity rests hidden in their qualitative otherness;
it is their inner unity, one that lies only at their base.

The manner of the appearance of this unity has thereby been defined.
The unity is posited in existence as a turning over
or transition of the finite into the infinite, and vice-versa;
so that the infinite only emerges in the finite,
and the finite in the infinite,
the other in the other;
that is to say, each arises in the other
independently and immediately,
and their connection is only an external one.

The process of their transition
has the following, detailed shape.
We have the finite passing over into the infinite.
This passing over appears as an external doing.
In this emptiness beyond the finite, what arises?
What is there of positive in it?
On account of the inseparability of the infinite and the finite
(or because this infinite, which stands apart, is itself restricted),
the limit arises.
The infinite has vanished and the other,
the finite, has stepped in.
But this stepping in of the finite appears
as an event external to the infinite,
and the new limit as something that does not arise
out of the infinite itself but is likewise found given.
And with this we are back at the previous determination,
which has been sublated in vain.
This new limit, however, is itself only
something to be sublated or transcended.
And so there arises again the emptiness, the nothing,
in which we find again the said determination,
and so forth to infinity.

We have before us the alternating determination
of the finite and the infinite;
the finite is finite only with reference
to the ought or the infinite,
and the infinite is only infinite with reference to the finite.
The two are inseparable and at the same time
absolutely other with respect to each other;
each has in it the other of itself;
each is thus the unity of itself and its other,
and, in its determinateness not to be
what itself and what its other is:
it is existence.

This alternating determination of self-negating
and of negating the negating is what passes
as the progress to infinity,
which is accepted in so many shapes and applications
as an unsurpassable ultimate at which thought,
having reached this “and so on to infinity,”
has usually achieved its end.
This progress breaks out wherever relative determinations are
pressed to the point of opposition, so that,
though in inseparable unity,
each is nevertheless attributed an independent existence
over against the other.
This progress is therefore the contradiction
which is not resolved but is rather always
pronounced simply as present.

What we have before us is an abstract transcending
which remains incomplete because the transcending
itself has not been transcended.
Before us we have the infinite;
of course, this infinite is transcended,
for another limit is posited,
but just because of that only a return
is instead made back to the finite.
This bad infinite is in itself
the same as the perpetual ought;
it is indeed the negation of the finite,
but in truth it is unable to free itself from it;
the finite constantly resurfaces in it as its other,
since this infinite only is with reference to the finite,
which is its other.
The progress to infinity is
therefore only repetitious monotony,
the one and the same tedious
alternation of this finite and infinite.

The infinity of the infinite progress
remains burdened by the finite as such,
is thereby restricted, and is itself finite.
In fact, however, it is thereby posited as
the unity of the finite and the infinite.
Only, this unity is not reflected upon.
Yet it alone rouses the finite in the infinite,
and the infinite in the finite;
it is, so to speak, the impulse
driving the infinite progress.
This progress is the outside of this unity
at which representation remains fixated;
fixated at that perennial repetition
of one and the same alternation;
at the empty unrest of a progression
across the limit towards the infinite
which, in this infinity, finds a new limit
but is just as unable to halt at it
as it is at the infinite.
This infinite has the rigid determination
of a beyond that cannot be attained,
for the very reason that it ought not be attained,
since the determinateness of the beyond,
of an existent negation, has not been let go.
In this determination, the infinite has
the finite as a this-side over against it;
a finite that is likewise unable
to raise itself up to the infinite
just because it has this determination of an other,
that is, of an existence that perennially
regenerates itself in that beyond
precisely by being different from it.

c. Affirmative infinity

In this reciprocal determination of the finite and the infinite
alternating back and forth as just indicated,
the truth of these two is already implicitly present in itself,
and all that is needed is to take up what is there.
This back and forth movement constitutes the external realization
of the concept in which the content of the latter is posited,
but externally, as a falling out of the two;
all that is needed is the comparing of these two different moments
in which the unity is given which the concept itself gives.
“Unity of the finite and the infinite”
(as has often been already noted
but must especially be kept in mind at this juncture)
is the uneven expression for the unity as it is in truth;
but also the removal of this uneven determination must be
found in the externalization of the concept that lies ahead of us.

Taken in their first, only immediate determination,
the infinite is the transcending of the finite;
according to its determination, it is the negation of the finite;
the finite, for its part, is only that which must be transcended,
the negation in it of itself, and this is the infinite.
In each, therefore, there is the determinateness of the other,
whereas, according to the viewpoint of the infinite progression,
the two should be mutually excluded
and would have to follow one another only alternately;
neither can be posited and grasped without the other,
the infinite without the finite, the finite without the infinite.
In saying what the infinite is, namely the negation of the finite,
the finite itself is said also;
it cannot be avoided in the determination of the infinite.
One need only know what is being said in order to
find the determination of the finite in the infinite.
Regarding the finite, it is readily conceded that it is the null;
this very nothingness is however the infinite
from which it is inseparable.
Understood in this way, they may seem to be taken
according to the way each refers to its other.
Taken without this connecting reference,
and thus joined only through an “and,”
they subsist independently,
each only an existent over against the other.
We have to examine how they would be constituted in this way.
The infinite, thus positioned, is one of the two;
but, as only one of them, it is itself finite,
it is not the whole but only one side;
it has its limit in that which stands over against it;
and so it is the finite infinite.
We have before us only two finites.
The finitude of the infinite,
and therefore its unity with the finite,
lies in the very fact that it is separated from the finite
and placed, consequently, on one side.
The finite, for its part, removed from the infinite
and positioned for itself, is this self-reference
in which the relativity, its dependence and transitoriness, are removed;
it is the same self-subsistence and self-affirmation
which the infinite is presumed to be.

The two pathways of consideration,
even though they seem at first to have
each a different determinateness for their point of departure;
the former inasmuch as it assumes it to be
only the reference of infinite and finite
to each other, of each to the other;
and the latter their complete separation from each other
yield one and the same result.
The infinite and the finite, taken together
as referring to each other in a connection
which is presumed external but is in fact essential to them
(for without it, neither is what it is),
each contains its other in its own determination,
just as, when each is taken for itself,
when looked at on its terms,
each has the other present in it
as its own moment.

This yields, then, the scandalous unity
of the finite and the infinite,
the unity which is itself the infinite
that embraces both itself and the finite,
the infinite, therefore, understood in a sense
other than when the finite is separated from it
and placed on the other side from it.
Since they must now also be distinguished,
each is within it, as just shown,
itself the unity of both;
there are thus two such unities.
The common element, the unity of both determinacies,
as such a unity, posits them at first as negated,
for each is to be what it is in being distinguished;
in their unity, therefore, they lose their qualitative nature;
an important reflection for countering the incorrigible habit
of representing the infinite and the finite, in their unity,
as still holding on to the quality that
they would have when taken apart from each other;
of seeing in that unity, therefore, nothing except contradiction,
and not also the resolution of the contradiction by
the negation of the qualitative determinateness of each.
And so is the unity of the infinite and the finite,
at first simple and universal, falsified.

But further, since the two are now to be taken
also as distinguished, the unity of the infinite
which is itself both of these moments
is determined differently in each.
The infinite, determined as such, has in it
the finitude which is distinct from it;
in this unity, the infinite is the in-itself
while the finite is only determinateness,
the limit in the infinite.
But such a limit is the absolute
other of the infinite, its opposite.
The infinite's determination, which is the in-itself as such,
is corrupted by being saddled with a quality of this sort;
the infinite is thus a finitized infinite.
Likewise, since the finite is as such only the non-in-itself
but equally has its opposite in it by virtue
of the said unity,
it is elevated above its worth
and, so to speak, infinitely elevated;
it is posited as the infinitized finite.

Likewise, just as the simple unity of infinite and finite was
falsified before by the understanding, so too is the double unity.
Here also this happens because the infinite is taken
in one of the two unities not as negated but,
rather, as the in-itself in which, therefore,
determinateness and restriction should not be posited,
for they would debase and corrupt it.
Conversely, the finite is equally held fixed
as not negated, although null in itself;
so that, in combination with the infinite,
it is elevated to what it is not
and is thereby infinitized notwithstanding its determination
that has not vanished but is rather perpetuated.

The falsification that the understanding perpetrates
with respect to the finite and the infinite,
of holding their reciprocal reference fixed
as qualitative differentiation,
of maintaining that their determination is separate,
indeed, absolutely separate, comes from forgetting
what for the understanding itself is the concept of these moments.
According to this concept, the unity of the finite and the infinite
is not an external bringing together of them,
nor an incongruous combination that goes against their nature,
one in which inherently separate and opposed terms
that exist independently and are consequently incompatible,
would be knotted together.
Rather, each is itself this unity,
and this only as a sublating of itself
in which neither would have an advantage
over the other in in-itselfness and affirmative existence.
As has earlier been shown, finitude is only as a transcending of itself;
it is therefore within it that the infinite,
the other of itself, is contained.
Similarly, the infinite is only as the transcending of the finite;
it therefore contains its other essentially,
and it is thus within it that it is the other of itself.
The finite is not sublated by the infinite
as by a power present outside it;
its infinity consists rather in sublating itself.

This sublating is not, consequently,
alteration or otherness in general,
not the sublating of something.
That into which the finite is sublated is
the infinite as the negating of finitude.
But the latter has long since been only existence,
determined as a non-being.
It is only the negation, therefore,
that in the negation sublates itself.
Thus infinity is determined on its side
as the negative of the finite
and thereby of determinateness in general,
as an empty beyond;
its sublating of itself into the finite is
a return from an empty flight,
the negation of the beyond
which is inherently a negative.

Present in both, therefore, is the same negation of negation.
But this negation of negation is in itself self-reference,
affirmation but as turning back to itself, that is
through the mediation that the negation of negation is.
These are the determinations that it is essential to bring to view;
the second point, however, is that in the infinite progression
they are also posited, and how they are posited therein,
namely, not in their ultimate truth.

First, both are negated in that progression,
the infinite as well as the finite;
both are equally transcended.
Second, they are also posited as distinct,
one after the other, each positive for itself.
We sort out these two determinations
while comparing them, just as in the comparison (in an external comparing)
we have separated the two ways of considering them:
the finite and the infinite as referring to one another,
and each taken for itself.
The infinite progression, however, says more than this.
Also posited in it, though at first still only
as transition and alternation,
is the connectedness of the terms being distinguished.
We now only need to see, in one simple reflection,
what is in fact present in it.

In the first place,
the negation of the finite and the infinite
which is posited in the infinite progression
can be taken as simple,
and hence as mutual externality,
only a following of one upon the other.
Starting from the finite,
the limit is thus transcended, the finite negated.
We now have its beyond, the infinite,
but in this the limit rises up again;
so we have the transcending of the infinite.
This twofold sublation is nonetheless
partly only an external event
and an alternating of moments in general,
and partly still not posited as one unity;
each of these moves beyond is an independent starting point,
a fresh act, so that the two fall apart.
But, in addition, their connection is
also present in the infinite progression.
The finite comes first;
then there is the transcending of it,
and this negative, or this beyond of the finite,
is the infinite;
third, this negation is transcended in turn,
a new limit comes up, a finite again.
This is the complete, self-closing movement
that has arrived at that which made the beginning;
what emerges is the same as that
from which the departure was made,
that is, the finite is restored;
the latter has therefore rejoined itself,
in its beyond has only found itself again.

The same is the case regarding the infinite.
In the infinite, in the beyond of the limit,
only a new limit arises which has the same fate, namely,
that as finite it must be negated.
Thus what is again at hand is the same infinite
that just now disappeared in the new limit;
by being sublated, by traversing the new limit,
the infinite has not therefore advanced one jot further:
it has distanced itself neither from the finite
(for the finite is just this,
to pass over into the infinite),
nor from itself, for it has arrived at itself.

Thus the finite and the infinite are both
this movement of each returning
to itself through its negation;
they are only as implicit mediation,
and the affirmative of each
contains the negative of each,
and is the negation of the negation.
They are thus a result
and, as such, not in the determination
that they had at the beginning:
neither is the finite an existence on its side
nor the infinite an existence
or a being-in-itself beyond that existence,
that is, beyond existence in
the determination of finitude.
The understanding strongly resists
the unity of the finite and the infinite
only because it presupposes restriction
and finitude to remain,
like being-in-itself, constants.
It thereby overlooks the negation of both
which is in fact present in the infinite progression,
just as it equally overlooks that the two occur in this
progression only as moments of a whole,
that each emerges only through the mediation of its opposite
but, essentially, equally by means of the sublation of its opposite.

If this immanent turning back has for the moment been reckoned
to be just as much the turning back of the finite to itself
and of the infinite to itself,
noticeable in this very result is an error connected
with the one-sidedness just criticized:
the finite and then the infinite is
each taken as the starting point,
and only in this way two results ensue.
But it is a matter of total indifference
which is taken as the starting point
and, with this, the distinction caused
by the duality of results dissolves of itself.
This is likewise posited in the line
of the infinite progression,
open-ended on both sides,
wherein each of the moments
recurs in equal alternation,
and it is totally extraneous
at which position the progression is
arrested and taken as beginning.
The moments are distinguished in the progression
but each is equally only moment of the other.
Since both, the finite and the infinite,
are themselves moments of the progress,
they are jointly the finite,
and, since they are equally jointly
negated in it and in the result,
this result as the negation
of their joint finitude is
called with truth the infinite.
Their distinction is thus the
double meaning which they both have.
The finite has the double meaning,
first, of being the finite over against the infinite
which stands over against it,
and, second, of being at the same time
the finite and the infinite over against the infinite.
Also the infinite has the double meaning
of being one of the two moments
(it is then the bad infinite)
and of being the infinite in which the two moments,
itself and its other, are only moments.
Therefore, as in fact we now have it,
the nature of the infinite is
that it is the process in which it lowers itself
to be only one of its determinations over against the finite
and therefore itself only one of the finites,
and elevates this distinction of itself
and itself to be self-affirmation
and, through this mediation, the true infinite.

This determination of the true infinite
cannot be captured in the already criticized formula
of a unity of the finite and the infinite;
unity is abstract, motionless self-sameness,
and the moments are likewise unmoved beings.
But, like both its moments, the infinite is
rather essentially only as becoming,
though a becoming now further determined in its moments.
Becoming has for its determinations,
first, abstract being and nothing;
as alteration, it has existence, something and other;
now as infinite, it has finite and infinite,
these two themselves as in becoming.

This infinite, as being-turned-back-unto-itself,
as reference of itself to itself, is being;
but not indeterminate, abstract being,
for it is posited as negating the negation;
consequently, it is also existence or “thereness,”
for it contains negation in general
and consequently determinateness.
It is, and is there, present, before us.
Only the bad infinite is the beyond,
since it is only the negation of
the finite posited as real
and, as such, it is abstract first negation;
thus determined only as negative,
it does not have the affirmation of existence in it;
held fast only as something negative,
it ought not to be there, it ought to be unattainable.
However, to be thus unattainable is
not its grandeur but rather its defect,
which is at bottom the result of holding fast
to the finite as such, as existent.
It is the untrue which is the unattainable,
and what must be recognized is that
such an infinite is the untrue.
The image of the progression in infinity is the straight line;
the infinite is only at the two limits of this line,
and always only is where the latter
(which is existence)
is not but transcends itself,
in its non-existence, that is, in the indeterminate.
As true infinite, bent back upon itself,
its image becomes the circle,
the line that has reached itself,
closed and wholly present,
without beginning and end.

True infinity, thus taken in general as existence posited
as affirmative in contrast to abstract negation,
is reality in a higher sense than it was earlier
as simply determined; it has now obtained a concrete content.
It is not the finite which is the real, but rather the infinite.
Thus reality is further determined
as essence, concept, idea, and so forth.
In connection with the more concrete,
it is however superfluous to repeat such earlier
and more abstract categories as reality,
and to use them for determinations more
concrete than they are by themselves.
Such a repetition, as when it is said that essence,
or that the concept, is real, has its origin in the fact
that to uneducated thought the most abstract categories
such as being, existence, reality, finitude, are the most familiar.

The more immediate occasion, however,
for recalling here the categories of reality
is that the negation, against which reality is the affirmative,
is here the negation of negation,
and consequently itself posited over against
that reality which finite existence is.
Negation is thus determined as ideality;
the idealized is the finite as it is in the true infinite
(as a determination, a content, a distinct but
not a subsistent existent, a moment rather.
Ideality has this more concrete signification
which is not fully expressed through
the negation of finite existence.)
As regards reality and ideality,
the opposition of finite and infinite is, however,
so grasped that the finite assumes the value of “the real,”
whereas the infinite that of “the idealized”;
in the same way, further on, also the concept is
regarded as an idealization, that is, as a mere idealization,
in contrast to existence in general, which is regarded as “the real.”
When contrasted in this way, it is of course of no use to have
reserved for the said concrete determination of negation
the distinctive expression of “idealization”;
in that opposition of finite and infinite,
we are back to the one-sidedness of
the abstract negative characteristic of the bad infinite
and still fixed in the affirmative existence of the finite.
