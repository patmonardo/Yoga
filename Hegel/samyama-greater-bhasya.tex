
DIVISION

The concept, as considered so far,
has demonstrated itself to be
the unity of being and essence.
Essence is the first negation of being,
which has thereby become reflective shine;
the concept is the second negation,
or the negation of this negation,
and is therefore being
which has been restored once more,
but as in itself the infinite mediation
and negation of being.
In the concept, therefore,
being and essence no longer have
determination as being and essence,
nor are they only in such a unity
in which each would reflectively shine in the other.
Consequently, the concept does not differentiate
itself into these determinations.
The concept is the truth of the substantial relation
in which being and essence attain their perfect
self-subsistence and determination each through the other.
The truth of substantiality proved
to be the substantial identity,
an identity that equally is,
and only is, positedness.
Positedness is determinate existence and differentiation;
in the concept, therefore, being-in-and-for-itself
has attained a true existence adequate to it,
for that positedness is itself being-in-and-for-itself.
This positedness constitutes the difference
of the concept in the concept itself;
and because the concept is
immediately being-in-and-for-itself,
its differences are themselves the whole concept,
universal in their determinateness
and identical in their negation.

This is now the concept itself of the concept,
but at first only the concept of the concept
or also itself only concept.
Since the concept is being-in-and-for-itself
by being a positedness, or is absolute substance,
and substance manifests the necessity of
distinct substances as an identity,
this identity must itself posit what it is.
The moments of the movement of the substantial relation
through which the concept came to be
and the reality thereby exhibited are
only in the transition to the concept;
that reality is not yet the
concept's own determination,
one that has emerged out of it;
it fell in the sphere of necessity
whereas the reality of the concept
can only be its free determination,
a determinate existence in which
the concept is identical with itself
and whose moments are themselves concepts
posited through the concept itself.

YS III.1

    desa-bandha cittasya dharana

At first, therefore, the concept is
only implicitly the truth;
because it is only something inner,
it is equally only something outer.
It is at first simply an immediate
and in this shape its moments have
the form of immediate, fixed determinations.
It appears as the determinate concept,
as the sphere of mere understanding.
Because this form of immediacy is an existence
still inadequate to the nature of the concept,
for the concept is free and only refers to itself,
it is an external form in which the concept
does not exist in-and-for-itself,
but can only count as something posited or subjective.
The shape of the immediate concept
constitutes the standpoint that makes
of the concept a subjective thinking,
a reflection external to the subject matter.
This stage constitutes, therefore, subjectivity,
or the formal concept.
Its externality is manifested in
the fixed being of its determinations
that makes them come up each by itself,
isolated and qualitative,
and each only externally referred to the other.
But the identity of the concept,
which is precisely their inner or subjective essence,
sets them in dialectical movement,
and through this movement their singleness is sublated
and with it also the separation of
the concept from the subject matter,
and what emerges as their truth is
the totality which is the objective concept.

YS III.2

    tatra pratyaya-eka-tanata dhyanam

Second, in its objectivity the concept is
the fact itself as it exists in-and-for-itself.
The formal concept makes itself into the fact
by virtue of the necessary determination of its form,
and it thereby sheds the relation
of subjectivity and externality
that it had to that matter.
Or, conversely, objectivity is the real concept
that has emerged from its inwardness
and has passed over into existence.
In this identity with the fact,
the concept thus has an existence
which is its own and free.
But this existence is still a freedom
which is immediate and not yet negative.
Being at one with the subject matter,
the concept is submerged into it;
its differences are objective
determinations of existence
in which it is itself again the inner.
As the soul of objective existence,
the concept must give itself the form of subjectivity
that it immediately had as formal concept;
and so, in the form of the free concept
which in objectivity it still lacked,
it steps forth over against that objectivity
and, over against it, it makes therein the identity with it,
which as objective concept it has in and for itself,
into an identity that is also posited.

YS III.3

    tad evartha-matra-nirbhasa svarupa-sunyam iva samadhi

In this consummation in which
the concept has the form of freedom
even in its objectivity,
the adequate concept is the idea.
Reason, which is the sphere of the idea,
is the self-unveiled truth
in which the concept attains
the realization absolutely adequate to it,
and is free inasmuch as in this real world,
in its objectivity, it recognizes its subjectivity,
and in this subjectivity recognizes that objective world.

YS III.4

    trayam ekatra samyama

YS III.5

    taj-jayat prajna-aloka

YS III.6

    tasya bhumisu viniyoga

YS III.7

    trayam antar-angam purvebhya

YS III.8

    tad api bahir-angam nirbijasya

SECTION I

Subjectivity

The concept is, to start with, formal,
the concept in its beginning
or as the immediate concept.
In this immediate unity,
its difference or its positedness
is, first, itself initially simple
and only a reflective shine,
so that the moments of the difference
are immediately the totality of the concept
and only the concept as such.

But, second, because it is absolute negativity,
the concept divides and posits itself
as the negative or the other of itself;
yet, because it is still immediate concept,
this positing or this differentiation is
characterized by the reciprocal
indifference of its moments,
each of which comes to be on its own;
in this division the unity of the concept is
still only an external connection.
Thus, as the connection of its moments
posited as self-subsisting and indifferent,
the concept is judgment.

Third, although the judgment contains
the unity of the concept that has been lost
in its self-subsisting moments,
this unity is not posited.
It will become posited by virtue of
the dialectical movement of the judgment
which, through this movement,
becomes syllogistic inference,
and this is the fully posited concept,
for in the inference the moments of
the concept as self-subsisting extremes
and their mediating unity are both equally posited.

But since this unity itself, as unifying middle,
and the moments, as self-subsisting extremes,
stand at first immediately opposite one another,
this contradictory relation that occurs
in the formal inference sublates itself,
and the completeness of the concept passes over
into the unity of totality;
the subjectivity of the concept
into its objectivity.

CHAPTER 1

The concept

YS III.9

    vyutthana-nirodha-samskarayor abhibhava-pradur-bhavau
    nirodha-kshana-cittanvayo nirodha-parinama

    The Concept as such

    Manas, the understanding, first the faculty
    for the cognition of the general (of rules)

YS III.10

    tasya prasanta-vahita samskarat

The faculty of concepts is normally
associated with the understanding,
and the latter is accordingly distinguished
from the faculty of judgment
and from the faculty of syllogistic inferences
which is formal reason.
But it is particularly with reason
that the understanding is contrasted,
and it signifies then, not the faculty of concepts in general,
but the faculty of determinate concepts,
as if, as the prevailing opinion has it,
the concept were only a determinate.
When distinguished in this meaning
from the formal faculty of judgment and from formal reason,
the understanding is accordingly to be taken
as the faculty of the single determinate concept.
For the judgment and the syllogism or reason, as formal,
are themselves only a thing of the understanding,
since they are subsumed under the form
of the abstract determinateness of the concept.
Here, however, we are definitely not taking
the concept as just abstractly determined;
the understanding is therefore
to be distinguished from reason only
in that it is the faculty of the concept as such.

This universal concept that we now have to consider
contains the three moments of
universality, particularity, and singularity.
The difference and the determinations which the concept
gives itself in its process of distinguishing constitute
the sides formerly called positedness.
Since this positedness is in the concept
identical with being-in-and-for-itself,
each of the moments is just as much
the whole concept as it is determinate concept
and a determination of the concept.

It is at first pure concept,
or the determination of universality.
But the pure or universal concept is also
only a determinate or particular concept
that takes its place alongside the other concepts.
Because the concept is a totality,
and therefore in its universality
or pure identical self-reference
is essentially a determining and a distinguishing,
it possesses in itself the norm
by which this form of its self-identity,
in pervading all the moments
and comprehending them within,
equally determines itself immediately
as being only the universal
as against the distinctness of the moments.

Second, the concept is thereby posited
as this particular or determinate concept,
distinct from others.

Third, singularity is the concept reflecting itself
out of difference into absolute negativity.
This is at the same time the moment at which
it has stepped out of its identity
into its otherness and becomes judgment.

CHAPTER 2

Judgment

Judgment is the determinateness of the concept
posited in the concept itself.
The determinations of the concept,
or, what amounts to the same thing as shown,
the determinate concepts,
have already been considered on their own;
but this consideration was rather
a subjective reflection
or a subjective abstraction.
But the concept is itself this act of abstracting;
the positioning of its determinations over
against each other is its own determining.
Judgment is this positing of the determinate concepts
through the concept itself.

YS III.11

    sarva-arthata-ekagratayo kshayodayau cittasya samadhi-parinama

    The logical form of all judgments consists of the objective unity
    of the apperception of the concepts contained therein

    Ahamkara, the determinative power of judgment,
    second the faculty for the subsumption of the particular under the general

First, as immediate, judgment is the judgment of existence;
its subject is immediately an abstract, existent singular,
and the predicate is an immediate determinateness or property of it,
an abstract universal.

Second, as this qualitative character of
the subject and predicate is sublated,
the determination of the one begins
to shine reflectively in the other;
the judgment is now the judgement of reflection.

But this external combination passes over
into the essential identity of a substantial, necessary combination;
and so we have, third, the judgment of necessity.

Fourth, since in this essential identity
the difference of subject and predicate has become a form,
the judgment becomes subjective;
it entails the opposition of the concept and its reality
and the comparison of the two;
it is the judgment of the concept.

This emergence of the concept grounds
the transition of judgment into syllogistic inference.

CHAPTER 3

The syllogism

YS III.12

    tata punashantoditau tulya-pratyayau cittasya-ekagrata-parinama

    Syllogism

    Buddhi, reason, third the faculty for the determination
    of the particular from the general (for the derivation from principles)

The syllogism is the result of
the restoration of the concept in the judgment,
and consequently the unity and the truth of the two.
The concept as such holds its moments
sublated in this unity;
in judgment, the unity is an internal
or, what amounts to the same, an external one,
and although the moments are connected,
they are posited as self-subsisting extremes.
In the syllogism, the determinations of the concept
are like the extremes of the judgment,
and at the same time their determinate unity is posited.

First, the syllogism of existence,
in which the terms are thus
immediately and abstractly determined,
demonstrates internally that,
since like judgment it is
the connection of those terms,
these are not in fact abstract
but each contains in it the reference
connecting it to the others,
and the determination of the middle term is
not just a determinateness opposed to the
determinations of the extremes
but contains these extremes posited in it.

Through this dialectic,
the syllogism of existence becomes
the syllogism of reflection, the second syllogism.
Its terms are such that in each the other
shines essentially reflected in it,
or are posited as mediated,
as they are indeed supposed to be
in accordance with the nature of
syllogistic inference in general.

Third, inasmuch as this reflective shining
or this mediatedness is reflected into itself,
syllogism is determined as the syllogism of necessity,
one in which the mediating factor is
the objective nature of the fact.
As this syllogism determines the extremes
of the concept also as totalities,
it has attained the correspondence
of its concept (or the middle term)
and its existence (or the difference of the extremes).
It has attained its truth;
and with that it has stepped forth
out of subjectivity into objectivity.

SECTION II

Objectivity

In Book One of the Objective Logic,
abstract being was presented as
passing over into existence,
but at the same time as
retreating into essence.

In Book Two, essence shows itself as
determining itself as ground,
thereby stepping into concrete existence
and realizing itself as substance,
but at the same time
retreating into the concept.

At the present standpoint of our treatise,
objectivity has the meaning first of all of
the being in and for itself of the concept
that has sublated the mediation posited
in its self-determination,
raising it to immediate self-reference.
This immediacy is therefore itself
immediately and entirely pervaded by the concept,
just as its totality is immediately identical with its being.
But further, since the concept equally has to restore
the free being-for-itself of its subjectivity,
it enters with respect to objectivity
into a relation of purpose
in which the immediacy of the objectivity
becomes a negative for it,
something to be determined through its activity.
This immediacy thus acquires the other significance,
namely that in and for itself,
in so far as it stands opposed to the concept,
it is a nullity.

First, then, objectivity is in its immediacy.
Its moments, on account of the totality of all moments,
stand in self-subsistent indifference
as objects each outside the other,
and as so related they possess
the subjective unity of the concept
only as inner or as outer.
This is mechanism.

But, second, inasmuch as in mechanism that unity
reveals itself to be the immanent law of the objects,
their relation becomes one of non-indifference,
each specifically different according to law;
a connection in which the objects'
determinate self-subsistence is sublated.
This is chemism.

Third, this essential unity of the objects is
thereby posited as distinct from their self-subsistence.
It is the subjective concept,
but posited as referring in and for itself
to the objectivity, as purpose.
This is teleology.

Since purpose is the concept posited
as within it referring to objectivity,
and through itself sublating its defect
of being subjective,
the at first external purposiveness becomes,
through the realization of the purpose, internal.
It becomes idea.

CHAPTER 1

Mechanism

YS III.13

    etena bhutendriyesu dharma-laksana-vastha-parinama vyakhyata

Since objectivity is the totality of the concept
that has returned into its unity,
an immediate is thereby posited
which is in and for itself that totality,
and is also posited as such,
but in it the negativity of the concept has as yet
not detached itself from the immediacy of the totality;
in other words, the objectivity is not yet posited as judgment.
In so far as it has the concept immanent in it,
the difference of the concept is present in it;
but on account of the objective totality,
the differentiated moments are
complete and self-subsistent objects
that, consequently, even in connection
relate to one another as each standing on its own,
each maintaining itself in every combination as external.
This is what constitutes the character of mechanism,
namely, that whatever the connection that
obtains between the things combined,
the connection remains one that is alien to them,
that does not affect their nature,
and even when a reflective semblance
of unity is associated with it,
the connection remains nothing more than
composition, mixture, aggregate, etc.
Spiritual mechanism, like its material counterpart,
also consists in the things connected in the spirit
remaining external to one another and to spirit.
A mechanical mode of representation,
a mechanical memory, a habit, a mechanical mode of acting,
mean that the pervasive presence that is proper to spirit
is lacking in what spirit grasps or does.
Although its theoretical or practical mechanism
cannot take place without its spontaneous activity,
without an impulse and consciousness,
the freedom of individuality is still lacking in it,
and since this freedom does not appear in it,
the mechanical act appears as a merely external one.

CHAPTER 2

Chemism

YS III.14

    shantoditavyapadesya-dharmanupati dharmi

In objectivity as a whole
chemism constitutes the moment of judgment,
of the difference that has become objective,
and of process.

Since it already begins with
determinateness and positedness,
and the chemical object is
at the same time objective totality,
the course it follows next is
simple and perfectly determined
by its presupposition.

CHAPTER 3

Teleology

YS III.15

    kramanyatvam parinamanyatve hetu

Purpose has resulted as the third
to mechanism and chemism;
it is their truth.
Inasmuch as it still stands
inside the sphere of objectivity
or of the immediacy of the total concept,
it is still affected by externality as such
and has an objective world over
against it to which it refers.
From this side, mechanical causality,
to which chemism is also in general to be added,
still makes its appearance in this purposive connection
which is the external one,
but as subordinated to it
and as sublated in and for itself.

As regards the more precise relation,
the mechanical object is, as immediate totality,
indifferent to its being determined and consequently,
conversely, to its being a determinant.
This external determinateness has now
progressed to self-determination
and accordingly the concept that
in the object was only inner
or, which amounts to the same,
only outer, is now posited;
purpose is, in the first instance,
precisely this concept which is
external to the mechanical object.
And so for chemism also, purpose is the self-determining
which brings the external determinateness conditioning it
back to the unity of the concept.
We have here the nature of the subordination of
the two preceding forms of the objective process.
The other, which in those forms lies in the infinite progress,
is the concept posited at first as external to them,
and this is purpose;
not only is the concept their substance
but externality is for them also
an essential moment constituting their determinateness.

Thus mechanical or chemical technique,
because of its character of being externally determined,
naturally offers itself to the connection of purpose,
which we must now examine more closely.

SECTION III

The idea

First, the idea is the simple truth,
the identity of concept and objectivity as a
universal in which the opposition,
the presence of the particular,
is dissolved in its self-identical negativity
and is equality with itself.

Second, it is the connection of the subjectivity
of the simple concept, existing for itself,
and of the concept's objectivity which is distinguished from it;
the former is essentially the impulse to sublate this separation,
and the latter is indifferent positedness,
subsistence which in and for itself is null.
As this connection, the idea is
the process of disrupting itself into individuality
and into the latter's inorganic nature,
and of then bringing this inorganic nature again
under the controlling power of the subject
and back to the first simple universality.
The identity of the idea with itself is one with the process;
the thought that liberates actuality from
the seeming of purposeless mutability
and transfigures it into idea
must not represent this truth of actuality
as dead repose, as a mere picture, numb, without impulse and movement,
as a genus or number, or as an abstract thought;
the idea, because of the freedom which the concept has attained in it,
also has the most stubborn opposition within it;
its repose consists in the assurance and the certainty
with which it eternally generates that opposition
and eternally overcomes it, and in it rejoins itself.

But the idea is at first again only immediate or only in its concept;
the objective reality is indeed conformable to the concept
but has not yet been liberated into the concept,
and it does not concretely exist explicitly as the concept.
Thus the concept is indeed the soul,
but the soul is in the guise of an immediate,
that is, it is not determined as soul itself,
has not comprehended itself as soul,
does not have its objective reality within itself;
the concept is as a soul that is not yet fully animated.

Thus the idea is, first of all, life.
It is the concept which, distinct from its objectivity,
simple in itself, permeates that objectivity
and, as self-directed purpose, has its means within it
and posits it as its means, yet is immanent in this means
and is therein the realized purpose identical with itself.
The idea, on account of its immediacy, has singularity
for the form of its concrete existence.
But the reflection within it of its absolute process is
the sublating of this immediate singularity;
thereby the concept, which as universality is
in this singularity the inner,
transforms externality into universality,
or posits its objectivity as a self-equality.

Thus is the idea, in second place,
the idea of the true and the good,
as cognition and will.
It is at first finite cognition and finite will,
where the true and the good are still distinguished
and the two are at first only as a goal.
The concept has first liberated itself into itself,
giving itself only a still abstract objectivity for its reality.
But the process of this finite cognition and this finite action
transforms the initially abstract universality into totality,
whereby it becomes complete objectivity.
Or considered from the other side,
finite, that is, subjective spirit,
makes for itself the
presupposition of an objective world,
such a presupposition as life only has;
but its activity is the sublating of this presupposition
and the turning of it into something posited.
Thus its reality is for it the objective world,
or conversely the objective world is
the ideality in which it knows itself.

Third, spirit recognizes the idea as its absolute truth,
as the truth that is in and for itself:
the infinite idea in which cognizing and doing are equalized,
and which is the absolute knowledge of itself.

CHAPTER 1

Life

YS III.16

    parinama-traya-samyamad atitanagata-jnanam

First, life is therefore to be considered as a living individual
that is for itself the subjective totality
and is presupposed as indifferent to an objectivity
that stands indifferent over against it.

Second, it is the life-process of sublating its presupposition,
of positing as negative the objectivity indifferent to it,
and of actualizing itself as the power
and negative unity of this objectivity.
By so doing, it makes itself into the universal
which is the unity of itself and its other.

Third, consequently life is the genus-process,
the process of sublating its singularization
and relating itself to its objective existence
as to itself.
Accordingly, this process is
on the one hand the turning back to its concept
and the repetition of the first forcible separation,
the coming to be of a new individuality
and the death of the immediate first;
but, on the other hand, the withdrawing into itself
of the concept of life is the becoming of
the concept that relates itself to itself,
of the concept that exists for itself,
universal and free, the transition into cognition.

CHAPTER 2

The idea of cognition

YS III.17

    sabda-artha-pratyayanam itaretara-adhyasat sankaras
    tat-pravibhaga-samyamat sarva-bhuta-ruta-jnanam

YS III.18

    samskara-saksat-karanat purva-jati-jnanam

YS III.19

    pratyayasya para-citta-jnanam

YS III.20

    na ca tat salambanam tasya-avisayi-bhutatvat

Life is the immediate idea, or the idea as
its still internally unrealized concept.
In its judgment, the idea is cognition in general.

Initially, therefore, the idea is one extreme of a syllogism,
the concept that as purpose has itself
at first for its subjective reality;
the other extreme is the restriction
of the subjective, the objective world.
The two extremes are identical in that they are the idea.
Their unity is, first, that of the concept,
a unity which in the one extreme is only for itself
and in the other only in itself.
Second, it is reality, abstract in the one extreme
and in the other in its concrete externality.
This unity is now posited through cognition,
and, because the latter is the subjective idea
which as purpose proceeds from itself,
it is at first only a middle term.
The knowing subject, through the
determinateness of its concept
which is the abstract being-for-itself,
refers to an external world;
nevertheless, it does this in
the absolute certainty of itself,
in order to elevate its implicit reality,
this formal truth, to real truth.
It has the entire essentiality of
the objective world in its concept;
its process consists in positing for itself
the concrete reality of that world
as identical with the concept,
and conversely in positing the latter
as identical with objectivity.

Immediately, the idea of appearance is
the theoretical idea, cognition as such.
For to the concept that exists for itself,
the objective world immediately has
the form of immediacy or of being,
just as that concept is to itself
at first only the abstract concept of itself,
is still shut up within itself.
The concept is therefore only as form,
of which only its simple determinations
of universality and particularity are
the reality that it possesses within,
while the singularity or the determinate determinateness,
the content, is received by it from the outside.

CHAPTER 3

The absolute idea

YS III.21

    kaya-rupa-samyamat tad-grahya-sakti-stambhe caksu-prakasa-asamprayoge 'ntardhanam

YS III.22

    etena sabda-adi-antardhanam uktam

The absolute idea has shown itself to be
the identity of the theoretical and the practical idea,
each of which, of itself still one-sided, possesses the idea
only as a sought-for beyond and unattained goal;
each is therefore a synthesis of striving,
each possessing as well as not possessing the idea within it,
passing over from one thought to the other
without bringing the two together
but remaining fixed in the contradiction of the two.
The absolute idea, as the rational concept
that in its reality only rejoins itself,
is by virtue of this immediacy of its objective identity,
on the one hand, a turning back to life;
on the other hand, it has equally
sublated this form of its immediacy
and harbors the most extreme opposition within.
The concept is not only soul,
but free subjective concept
that exists for itself
and therefore has personality,
the practical objective concept
that is determined in and for itself
and is as person impenetrable, atomic subjectivity
but which is not, just the same, exclusive singularity;
it is rather explicitly universality and cognition,
and in its other has its own objectivity for its subject matter.
All the rest is error, confusion, opinion,
striving, arbitrariness, and transitoriness;
the absolute idea alone is being, imperishable life,
self-knowing truth, and is all truth.

In conclusion, there remains only this to be said of this idea,
that in it, in the first place,
the science of logic has apprehended its own concept.
In the sphere of being, at the beginning of its content,
its concept appears as a knowledge external to
that content in subjective reflection.
But in the idea of absolute cognition,
the concept has become the idea's own content.
The idea is itself the pure concept
that has itself as its subject matter
and which, as it runs itself as subject matter
through the totality of its determinations,
builds itself up to the entirety of its reality,
to the system of science,
and concludes by apprehending this
conceptual comprehension of itself,
hence by sublating its position
as content and subject matter
and cognizing the concept of science.
In second place, this idea is still logical;
it is shut up in pure thought,
the science only of the divine concept.
Its systematic exposition is of course itself a realization,
but one confined within the same sphere.
Because the pure idea of cognition is
to this extent shut up within subjectivity,
it is the impulse to sublate it,
and pure truth becomes as final result
also the beginning of another sphere and science.
It only remains here to indicate this transition.

The idea, namely, in positing itself
as the absolute unity of the pure concept and its reality
and thus collecting itself in the immediacy of being,
is in this form as totality:  nature.
This determination, however, is nothing that has become,
is not a transition, as was the case above
when the subjective concept in its totality becomes objectivity,
or the subjective purpose becomes life.
The pure idea into which the determinateness
or reality of the concept is itself
raised into concept is rather an
absolute liberation for which
there is no longer an immediate determination
which is not equally posited and is not concept;
in this freedom, therefore, there is
no transition that takes place;
the simple being to which the idea determines itself
remains perfectly transparent to it:
it is the idea that in its determination remains with itself.
The transition is to be grasped, therefore,
in the sense that the idea freely discharges itself,
absolutely certain of itself and internally at rest.
On account of this freedom, the form of
its determinateness is just as absolutely free:
the externality of space and time absolutely existing
for itself without subjectivity.
Inasmuch as this externality is only
in the abstract determinateness of being
and is apprehended by consciousness,
it is as mere objectivity and external life;
within the idea, however, it remains in
and for itself the totality of the concept,
and science in the relation of
divine cognition to nature.
But what is posited by this first resolve
of the pure idea to determine itself as external idea
is only the mediation out of which the concept,
as free concrete existence that from externality
has come to itself, raises itself up,
completes this self-liberation in the science of spirit,
and in the science of logic finds the highest concept of itself,
the pure concept conceptually comprehending itself.
