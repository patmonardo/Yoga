
A. THE SYLLOGISM OF EXISTENCE

1. The syllogism in its immediate form
has for its moments the determinations
of the concept as immediate.
Accordingly, these are the abstract determinacies of form,
such as have not yet been developed by mediation into concretion
but are only singular determinacies.
The first syllogism is thus the one which is strictly formal.
The formalism of syllogistic inference consists in
stopping short at the form of this first syllogism.
The concept, when partitioned into its abstract moments,
has singularity and universality for its extremes,
and itself appears as the particularity
that stands between them.
Because of their immediacy,
these determinacies only refer to themselves,
one and all a single content.
Particularity constitutes at first the middle term
by uniting within itself, immediately,
the two moments of singularity and universality.
Because of its determinateness,
on the one hand it is subsumed under the universal;
on the other hand, the singular
with respect to which it possesses universality
is subsumed under it.
This concretion is at first, however,
only a double-sidedness;
the middle term, because of the immediacy
that affects it in the immediate syllogism,
is a simple determinateness,
and the mediation which it constitutes is not as yet posited.
Now the dialectical movement of the syllogism of existence
consists in positing the moments of the mediation
that alone constitutes the syllogism.

a. First figure of the syllogism

S-P-U is the general schema of the determinate syllogism.
Singularity connects with universality through particularity;
the singular is not universal immediately
but by means of particularity;
and conversely, universality is likewise not singular immediately
but lowers itself to it through particularity.
These determinations stand over against each other as extremes
and are one in a third term which is diverse from them.
The two are both determinateness;
in this they are identical;
this, their universal determinateness, is particularity.
But they are no less extremes with respect to
this particularity than they are to each other,
for each is in its immediate determinateness.

The general meaning of this syllogism is that the singular,
which as such is infinite self-reference
and consequently would be only an inwardness,
emerges through the medium of particularity into existence,
into a universality wherein it no longer belongs just to itself
but stands in external conjunction;
conversely, since in its determinateness
the singular sets itself apart as particularity,
in this separation it is a concreted term
and, because of the self-reference of the determinateness,
it is a self-referring universal,
and consequently also a true singular;
in the extreme of universality the
singular has gone from externality into itself.
The objective significance of the syllogism is
in this first figure only superficially present at first,
for the determinations are not as yet posited in it
as the unity which constitutes the essence
of the syllogistic inference.
The syllogism is still something subjective
inasmuch as the abstract meaning which its terms have
has no being in and for itself
but is rather only in a subjective consciousness,
and is thus isolated.
Moreover, as we have seen, the relation of
singularity, particularity, and universality
is the necessary and essential form-relation of
the determinations of the syllogism;
the deficiency does not rest in the
determinateness of the form
but in that each single determination is not
at the same time richer under it.
Aristotle confined himself rather to the mere relation of
inherence by defining the nature of the syllogism as follows:
When three terms are so related to each other
that the one extreme is in the entire middle term,
and this middle term is in the entire other extreme,
then these two extremes are necessarily united in the conclusion.
What is here expressed is the repetition of
the equal relation of inherence
of the one extreme to the middle term,
and then again of this last to the other extreme,
rather than the determinateness of the three terms
to each other.
Now since the syllogistic inference rests
on this determinateness of the terms to each other,
it is immediately apparent that the other relations of terms
as are given by the other figures can have validity
as inferences of the understanding
only to the extent that they let themselves
be reduced to that original relation;
these other are not diverse species of figures
that stand alongside the first
but, on the one hand, to the extent that they are assumed
to be correct inferences,
they rest on the form of syllogistic inference in general;
and, on the other hand, to the extent that they deviate from it,
they are variant forms into which
the first abstract form necessarily passes over
and thereby further determines itself and becomes totality.
How this occurs, we must now see in greater detail.

S-P-U is thus the general schema of
the syllogism in its determinateness.
The singular is subsumed under the particular
and the particular under the universal;
therefore, the singular is also subsumed under the universal.
Or the particular inheres in the singular
and the universal in the particular;
therefore, the universal also inheres in the singular.
With respect to the side of the universal,
the particular is the subject;
with respect to that of the singular, it is predicate;
or as against the one it is singular,
as against the other it is universal.
Since both these determinations are united in it,
by virtue of this unity of determinations
the extremes are joined together.
The “therefore” appears as an inference
that has taken place in the subject
and derives from the subjective insight
into the relation of the two immediate premises.
Since subjective reflection expresses
the two connections of the middle
to the extremes as particular
and indeed immediate judgments or propositions,
the conclusion as the mediated connection is
of course also a particular proposition,
and the “hence” or the “therefore,” is
the expression that it is the one which is mediated.
But this “therefore” is not to be regarded
as a determination which is external to this proposition,
one that would have its ground and seat in subjective reflection,
but as grounded rather in the nature of the extremes themselves
whose connection is again enunciated as
a mere judgment or proposition only for the sake of,
and by virtue of, abstractive reflection,
but whose true connection is posited as the middle term.
“Therefore S is A”:
that this is a judgment is a merely subjective circumstance;
that it is not a merely subjective judgment,
that is, not a connection drawn through
the mere copula or the empty “is”
but one drawn rather through a determinate middle
which is replete with content,
that is precisely the meaning of the syllogistic inference.
For this reason, to regard the syllogism as
merely consisting of three judgments is
a formalistic view that ignores
the relation of the determinations
which alone is at issue in the inference.
It is altogether a merely subjective reflection
that splits the connection of
the terms into isolated premises
and a conclusion distinct from them:

All humans are mortal,
Gaius is a human
Therefore Gaius is mortal.

One is immediately seized by boredom the moment
one hears this inference being trotted out,
a boredom brought on by the futility of a form
that by means of separate propositions
gives the illusion of a diversity
which is immediately dissolved in the fact itself.
It is mostly because of this subjective attire
that the inference appears as a subjective expedient
in which reason or understanding take refuge
when they are incapable of immediate cognition.
The nature of things, the rational,
certainly does not operate in this way,
first by drawing up a major premise for itself,
the connection of some particularity to a subsistent universal;
then by discovering in a second moment the separate connection
of a singularity to the particularity,
out of which in a third and final moment
a new proportion comes to light.
This syllogistic inference from
one separate proposition to another is
nothing but a subjective form;
the nature of the fact is that
its various determinations are
united in a unity of essence.
This rationality is not an expedient;
on the contrary, in contrast to the
immediacy of the connection
that still obtains in judgment,
it is the objective element;
it is the prior immediacy of cognition
that rather is mere subjectivity,
in contrast to the syllogistic inference
which is the truth of the judgment.
All things are a syllogism,
a universal united through particularity with singularity;
surely not a whole made up of three propositions.

2. In the immediate syllogism of the understanding,
the terms have the form of immediate determinations;
we must now consider that syllogism from this side,
according to which the terms are content.
We may then regard it as qualitative,
just as we did the judgment of existence
which has the same side of qualitative determination.
The terms of this syllogism,
just like the terms of that judgment,
are accordingly singular determinacies,
for the determinateness is posited
because of its self-reference as indifferent to form
and hence as content.
The singular is some immediate concrete subject matter or other;
particularity, one of its determinacies, properties or relations;
universality, a yet more abstract, more singularized
determinateness in the particular.
Since the subject, as something immediately determined,
is as yet not posited in its concept,
its concretion is not reduced to its essential determinacies;
its self-referring determinateness is therefore
indeterminate, an infinite manifoldness.
In this immediacy, the singular
has an infinite multitude of determinacies
that belong to its particularity
and any may serve in a syllogism as the middle term for it.
Through each middle term, however,
the singular attaches to another universal;
through each of its properties it enters into
a different arrangement and context of existence.
Moreover, in comparison with the universal,
the middle term is also a concreted term;
it itself contains several predicates,
and through the same middle term
the singular can again attach to several universals.
In general, therefore, it is entirely accidental and arbitrary
which of the many properties of a thing is taken
for the purpose of connecting it with a predicate;
other middle terms are transitions to other predicates,
and even the same middle term may by itself be
the transition to different predicates,
for as a particular against the universal
it contains several determinations.
But not only is an indeterminate number of
syllogisms equally possible for a subject
and not only is any single syllogism
contingent as regards content,
but these syllogisms that concern the same subject
must also run into contradiction.
For difference as such,
which is at first an indifferent diversity,
is in essence equally opposition.
The concrete is no longer merely phenomenal
but is concrete through the unity in the concept of opposites
that have determined themselves as moments of the latter.
Now inasmuch as in the formal syllogism,
in keeping with the qualitative nature of the terms,
the concrete is taken according to
one of the single determinations that pertain to it,
the syllogistic inference assigns to it the predicate
corresponding to this middle term;
but inasmuch as from another side the
opposite determinateness is inferred,
the previous conclusion turns out to be false,
even though its premises and equally so its consequences
are by themselves quite correct.
If from the middle term that a wall was painted blue
it is concluded that it is blue,
this is a correct inference.
But, this conclusion notwithstanding,
the wall can be green if it was
also covered over with a yellow color,
a circumstance from which alone it would follow that the wall is yellow.
If from the senses as middle term it is concluded that
the human being is neither good nor bad,
for neither the one nor the other can be predicated of the senses,
the inference is correct;
yet the conclusion is wrong,
because to the human being, taken concretely,
spirituality also applies as the middle term.
From the middle term of the gravitation of the planets,
the satellites and comets towards the sun,
it follows correctly that these bodies fall into the sun;
but they do not fall into it,
because they are equally their own center of gravity
or, as it is said, are driven by the centrifugal force.
Likewise, from sociability as the middle term,
the community of goods among citizens can be inferred;
however, from individuality as the middle term,
if the term is pressed with equal abstractness,
there follows the dissolution of the state,
as for example it did follow for the German Empire
from adhering to that middle term.
It is only fair to hold that nothing is
as unsatisfactory as such a formal syllogism,
since which middle term is employed is
a matter of chance or arbitrariness.
No matter how elegantly a deduction is
run through inferences of this kind,
however fully its correctness is to be conceded,
all this still amounts to nothing,
for the possibility is still there
that other middle terms may be found
from which the opposite can be deduced with equal correctness.
Kant's antinomies of reason amount to nothing more than that
from a concept one of its determinations is
laid down as ground at one time,
and another determination at another time,
both with equal necessity.
The insufficiency and contingency of an inference
must not be blamed in these cases on the content,
as if they were independent of the form
and the latter alone were the concern of logic.
On the contrary, it lies in the form of the formal syllogism
that the content is such a one-sided quality;
the content is destined to this one-sidedness
because of the form's abstractness.
It is, namely, one single quality of the many qualities
or determinations of a concrete subject matter,
or of a concept, because according to the form it
is not supposed to be anything more than just
such an immediate, single determinateness.
The extreme of singularity is, as abstract singularity,
the immediate concrete, consequently
an infinite or indeterminate manifold;
the middle term is the equally abstract particularity,
consequently a single one of these manifold qualities,
and likewise the other extreme is the abstract universal.
It is therefore because of its form that the formal syllogism is
totally contingent as regards its content,
not indeed because to the syllogism it is accidental
whether this or that subject matter is subject to it
(logic abstracts from content),
but because, in so far as a subject is laid at its basis,
it is contingent which content determinations it will infer from it.

3. The determinations of the syllogism are
determinations of content inasmuch as
they are immediate and abstract determinations
reflected into themselves.
But their essence is to be, not immanently reflected and
mutually indifferent determinations,
but determinations of form,
and to this extent they are essentially connections.
These connections are,
first, those of the extremes to the middle term.
These are immediate connections,
the propositiones premise namely,
the connection of the particular to the universal,
the propositio major,
and that of the singular to the particular,
the propositio minor.
Second, there is the connection of the extremes to one
another, and this is the mediated connection, the conclusion.
The immediate connections, the premises, are
propositions or judgments in general,
and they contradict the nature of the syllogism,
for according to the latter the
different concept determinations should not be immediately connected
but also their unity should be posited;
the truth of the judgment is the syllogistic conclusion.
And there is all the more reason why the premises
cannot remain immediate connections as their content is made up of
immediately differentiated determinations
which, as such, are not in and for themselves identical,
unless the premises are identical propositions,
that is, empty tautologies that lead to nothing.

Accordingly, the normal expectation is
that the premises will be proved,
that is, that they ought likewise
to be exhibited as conclusions.
The two premises, therefore, yield two further syllogisms.
But these two new syllogisms together yield
four premises that require four new syllogisms;
these have eight premises whose eight conclusions yield
in turn sixteen conclusions for their sixteen premises,
and so on in a geometrical progression to infinity.

Thus we have again the progress to infinity
that occurred in the lower sphere of being
but we would not expect now
in the domain of the concept,
the domain of the absolute reflection
from the finite to the self,
the region of free infinity and truth.
It was shown in the sphere of being that
whenever the bad infinity that runs away
into a progression raises its head,
what we have is the contradiction of a qualitative being
and of an impotent ought that would transcend it;
the progression itself is the repeated demand that
there be unity that intervenes to confront the qualitative,
and the constant fall back into the limitation
which is inadequate to the demand.
Now in the formal syllogism the immediate connection
or the qualitative judgment is the basis,
and the mediation of the syllogism is
the higher truth posited over against it.
The infinite progression of the proof of the premises does
not resolve this contradiction
but only perpetually renews it
and is the repetition of one and the same original deficiency.
The truth of the infinite progression is
rather the sublation of it and of the form
which the progression itself has already determined as deficient.
This form is that of the mediation S-P-U.
The two connections, S-P and P-U are supposed to be mediated;
if this is done in the same manner,
only the deficient form S-P-U is replicated,
and so on to infinity.
With respect to S, P also has the
form determination of a universal;
and with respect to U that of a singular,
for these connections are as such judgments.
As such, they are in need of mediation;
but in that form of mediation,
only the relation that was supposed
to be sublated comes up again.

The mediation must therefore occur in some other way.
For the mediation of P-U, there is S available;
hence the mediation must be given the shape of P-S-U.
To mediate S-P, there is U available;
accordingly, this mediation becomes S-U-P.

If this transition is examined more closely in light of its concept,
then, as was shown earlier, the mediation of the formal syllogism is
in the first instance contingent according to content.
The immediate singular has in its determinacies
an indeterminate number of middle terms,
and these have in general equally as many determinacies in turn;
it is, therefore, entirely a matter of external arbitrary choice,
or in general of an external circumstance and accidental determination,
as to which universal the subject of the syllogism should be annexed.
As regards its content, therefore, the mediation is
not anything necessary, nor universal;
it is not grounded in the concept of the fact
but the basis of the inference is something external to it,
that is, the immediate;
but of the determinations of the concept,
it is the singular which is the immediate.

With respect to form, the mediation likewise
presupposes the immediacy of connection;
the mediation itself is thus mediated,
mediated indeed by means of the immediate,
that is, the singular.
More precisely, the singular has become a mediating term
through the conclusion of the first syllogism.
That conclusion is S-U;
the singular is thereby posited as a universal.
In one premise, that is, the minor S-P,
it is already as a particular;
consequently, the singular is that
in which these two determinations are united.
Or the conclusion expresses in and for itself
the singular as a universal,
and it does it, not in any immediate manner, but mediatedly,
hence as a necessary connection.
The simple particularity was the middle term;
in the conclusion, this particularity is posited
as developed as the connection of singular and universality.
But the universal is still a qualitative determinateness,
the predicate of the singular;
in being determined as universal,
the singular is posited as the universality of the extremes
or as the middle;
it is for itself the extreme of singularity,
but since it is now determined as a universal,
it is at the same time the unity of the two extremes.

b. The second figure: P-S-U

1. The truth of the first qualitative syllogism is
that something is not in and for itself united
to a qualitative determinateness which is a universal,
but is united to it by means of a contingency
or in a singularity.
The subject of the syllogism has not returned
in such a quality to its concept
but is conceived only in its externality;
the immediacy constitutes the basis of
the connection and hence the mediation;
to this extent, the singular is in truth the middle.

But further, the syllogistic connection is
the sublation of the immediacy;
the conclusion is a connection drawn not immediately
but through a third term;
therefore, it contains a negative unity;
therefore, the mediation is now determined
as containing a negative moment within it.
In this second syllogism, the premises are: P-S and S-U;
only the firstof these premises is still an immediate one;
the second, S-U, is already mediated,
namely through the first syllogism;
the second syllogism thus presupposes the first just as, conversely,
the first presupposes the second.
The two extremes are here determined,
the one as against the other, as particular and universal.
The latter thus retains its place; it is predicate.
But the particular has exchanged places;
it is subject or is posited in the
determination of the extreme of singularity,
just as the singular is posited with
the determination of the middle term or of particularity.
The two no longer are, therefore, the abstract immediacies
which they were in the first syllogism.
However, they are not yet posited as concrete somethings;
in standing in the place of the other, each is thereby posited;
in its own determination and at the same time,
although only externally, into that of the other.

The determinate and objective meaning of this syllogism is
that the universal is not in and for itself a determinate particular
(it is rather the totality of its particulars)
but that it is one of its species
through the mediation of singularity;
the rest of its species are excluded from
it by the immediacy of externality.
Likewise the particular is not for its part immediately,
and in and for itself, the universal;
the negative unity is rather what removes
the determinateness from it
and thereby raises it to universality.
The singularity thus relates to the particular negatively
in so far as it is supposed to be its predicate;
it is not the predicate of the particular.

2. But the terms are at first
still immediate determinacies;
they have not advanced of their own
to any objective signification;
the positions which two of them
have exchanged and now occupy is the form,
and this is as yet only external to them.
Therefore they are still, as in the first syllogism,
each a content indifferent as such to the other,
two qualities linked together, not in and for themselves,
but through the mediation of an accidental singularity.

The syllogism of the first figure was the immediate syllogism,
or again, the syllogism in so far as its concept is
an abstract form that has not yet realized itself
in all its determinations.
The transition of this pure form into another figure is
on the one hand the beginning of the realization of the concept,
in that the negative moment of the mediation,
and thereby one further determinateness of the form,
is posited in the originally immediate,
qualitative determinateness of the terms.
But, on the other hand, this is at the same time
an alteration of the pure form of the syllogism;
the latter no longer conforms to it fully,
and the determinateness posited in its terms is
at variance with that original form determination.
In so far as it is regarded as only a subjective syllogism
that runs its course in external reflection,
we can then take it as a species of syllogistic inference
that should conform to the genus,
namely the general schema S-P-U.
But it does not at the moment conform to it;
its two premises are P-S or S-P and S-U;
the middle term is in both cases the one which is subsumed
or is the subject in which the two other terms thus inhere,
is not therefore a middle term
that in one case would subsume or be predicate,
and in the other would be subsumed or be subject,
or a middle in which one of the terms
would inhere but would itself inhere in the other.
The true meaning of this syllogism's lack of conformity to
the general form of the syllogism is that
the latter has passed over into it,
for its truth consists in being
a subjective, contingent conjoining of terms.
If the conclusion in this second figure is correct
(that is, without recurring to the restriction,
to which we shall presently turn,
that makes of it something indeterminate),
then it is correct because it is so on its own,
not because it is the conclusion of this syllogism.
But the same is the case for the conclusion of the first figure;
it is this, the truth of that first figure,
which is posited by the second.
On the view that the second figure is only one species,
we overlook the necessary transition of
the first figure into this second
and stop short at the first as the true form.
Hence, if in the second figure
(which from ancient custom is referred to,
without further ground, as the third)
we are equally supposed to find
a correct syllogism in this subjective sense,
this syllogism would have to be commensurate with the first;
consequently, since the one premise S-U has the relation of
the subsumption of the middle term under one extreme,
then it would have to be possible for the other premise S-P
to receive the opposite relation to that which it has,
and for P to be subsumed under S.
But such a relation would be the sublation of
the determinate judgment S is P,
and could only occur in an indeterminate judgment, a particular judgment;
consequently, the conclusion in this figure can only be particular.
But the particular judgment, as we remarked above,
is positive as well as negative,
a conclusion, therefore, to which no great value can be ascribed.
Since the particular and universal are also the extremes,
and are immediate determinacies indifferent to each other,
their relation itself is indifferent;
each can be the major or the minor term,
indifferently the one or the other,
and consequently either premise can also
be taken as major or minor.

3. Since the conclusion is positive as well as negative,
it is a connection which for that reason is
indifferent to these determinacies,
hence a universal connection.
More precisely, the mediation of the first syllogism
was implicitly a contingent one;
in the second syllogism, this contingency is posited.
Consequently, the mediation is self-sublating;
it has the determination of singularity and immediacy;
what this syllogism joins together must,
on the contrary, be in itself and immediately identical,
for that mediating middle, the immediate singularity,
is an infinitely manifold and external determining.
Posited in it, therefore, is rather the self-external mediation.
The externality of singularity, however, is universality;
that mediation by means of the immediate singular
points beyond itself to the mediation which is the other than it,
one which therefore occurs by means of the universal.
In other words, what is supposed to be united by means of
the second syllogism, must be immediately conjoined;
the immediacy on which it is based
does not allow any definite conclusion.
The immediacy to which this syllogism points is
the opposite of its own:
it is the sublated first immediacy of being,
therefore the immediacy reflected into itself
or the abstract universal existing in itself.

From the standpoint of the present consideration,
the transition of this syllogism was like
the transition of being an alteration,
for its base is qualitative;
it is the immediacy of singularity.
But according to the concept,
singularity conjoins the particular and the universal
by sublating the determinateness of the particular
and this is what presents itself as the contingency
of this syllogistic inference.
The extremes are not conjoined by the specific connective
which they have in the middle term;
this term is not, therefore, their determinate unity,
and the positive unity that yet pertains to it is abstract universality.
But inasmuch as the middle term is posited in this determination
which is its truth, we have another form of the syllogism.

c. The third figure: S-U-P

1. This third syllogism no longer has any single immediate premise;
the connection S-U has been mediated by the first syllogism;
the connection P-U by the second.
It thus presupposes both these syllogisms;
but conversely it is presupposed by them,
just as in general each presupposes the other two.
In this third figure, therefore, it is the determination of the syllogism
as such that is brought to completion.
This reciprocal mediation means just this,
that each syllogism, although for itself a mediation,
does not possess the totality of mediation
but is affected by an immediacy
whose mediation lies outside it.

Considered in itself, the syllogism S-U-P is
the truth of the formal syllogism;
it expresses the fact that
its mediating middle is the abstract universal
and that the extremes are not contained in it
according to their essential determinateness
but only according to their universality,
that precisely that is not conjoined in it,
which was supposed to be mediated.
Posited here, therefore, is that
wherein the formalism of the syllogism consists;
that its terms have an immediate content
which is indifferent towards the form,
or, what amounts to the same,
that they are such form determinations
as have not yet reflected themselves
into determinations of content.

2. The middle of this syllogism is indeed the unity of the extremes,
but a unity in which abstraction is made from their determinateness,
the indeterminate universal.
But in so far as this universal is at the same time distinguished
from the extremes as the abstract from the determinate,
it is itself also a determinate as against them,
and the whole is a syllogism
whose relation to its concept needs examining.
As the universal, the middle term is
with respect to both its extremes
the term that subsumes or the predicate,
not a term for once also subsumed or the subject.
Now as a species of syllogism, it ought to conform to the latter,
and this can only happen on condition that,
inasmuch as the one connection S-U
already possesses the appropriate relation,
the other connection P-U contains it too.
This occurs in a judgment in which
the relation of subject and predicate is
an indifferent one, in a negative judgment.
Thus does the syllogism become legitimate,
but the conclusion is necessarily negative.

Consequently, also indifferent is now
which of the two determinations
of this proposition is taken as predicate or subject,
and whether the determination is taken in the syllogism as
the extreme of singularity or the extreme of particularity,
hence as the minor or major term.
Since on the usual assumption
which of the premises is supposed to be
the major or the minor depends on this distinction,
this too has now become a matter of indifference.
This is the ground of the customary
fourth figure of the syllogism
which was unknown to Aristotle
and has to do with an entirely void
and uninteresting distinction.
In it the immediate position of the terms is
the reverse of their position in the first figure;
since from the point of view of the formal treatment of judgment
the subject and predicate of the negative conclusion
do not have the determinate relation of subject and predicate,
but each can take the place of the other,
it is a matter of indifference which term is taken
as subject and which as predicate;
and just as indifferent is therefore
which premise is taken as the major
and which as the minor.
This indifference, to which the determination of particularity
also contributes
(especially if it is noted that this particularity
can be taken in a comprehensive sense),
makes of this fourth figure something totally idle.

3. The objective significance of the syllogism
in which the universal is the middle is that the mediating term,
as the unity of the extremes, is essentially a universal.
But since the universality is at first only qualitative or abstract,
the determinateness of the extremes is not contained in it;
their being conjoined in the conclusion,
if the conjunction is to take place,
must likewise have its ground in a mediation
that lies outside this syllogism
and is, with respect to the latter,
just as contingent as it is in
the preceding forms of the syllogism.
But now, since the universal is determined as the middle term,
and since the determinateness of the extremes
is not contained in this middle,
the latter is posited as one
which is wholly indifferent and external.
It is here, by virtue indeed of a bare abstraction,
that a fourth figure of the syllogism arose in the first place,
namely the figure of the relationless syllogism, U-U-U,
which abstracts from the qualitative differentiation of the terms
and therefore has their merely external unity,
their equality, for its determination.

d. The fourth figure: U-U-U, or the mathematical syllogism

1. The mathematical syllogism goes like this:
if two things or two determinations are equal to a third,
then they are equal to each other.
The relation of inherence or subsumption of terms is done away with.
A “third” is in general the mediating term;
but this third has absolutely no determination as against the extremes.
Each of the three terms can therefore be the mediating term
just as well as any other.
Which is needed for the job,
which of the three connections are
therefore to be taken as immediate,
and which as mediated,
depends on external circumstances and other conditions,
namely which two of the three are immediately given.
But this determination does not concern the syllogism
and is wholly external.

2. The mathematical syllogism ranks in mathematics as an axiom,
as a first self-explanatory proposition
which is neither capable nor in need of proof,
i.e of any mediation which neither presupposes anything else
nor can be derived from anything else.
If we take a closer look at this prerogative
that the proposition claims, of being immediately self-evident,
we find that it lies in its formalism,
in the fact that it abstracts from
every qualitative diversity of determinations
and only admits their quantitative equality or inequality.
But for this very reason it is not without presupposition or mediation;
the quantitative determination,
which alone comes into consideration in it,
is only by virtue of the abstraction from qualitative differentiation
and from the concept determinations.
Lines, figures, posited as equal to each other,
are understood only according to their magnitude.
A triangle is posited as equal to a square,
not however as triangle to square
but only according to magnitude, etc.
Nor does the concept and its determinations enter into this syllogism;
there is in it, therefore, no conceptual comprehension at all;
the understanding is also not faced here by even
the formal, abstract determinations of the concept.
The self-evidence of this syllogism rests,
therefore, solely on the indigence and abstractness
of its mode of thought.

3. But the result of the syllogism of existence is not
just this abstraction from all determinateness of the concept;
the negativity of the immediate and abstract determinations
that emerged from it has yet another positive side,
namely that in the abstract determinateness
its other has been posited
and the determinateness has thereby become concrete.

In the first place, the syllogisms of existence
all have one another for presupposition,
and the extremes conjoined in the conclusion are truly conjoined,
in and for themselves, only inasmuch as they are otherwise
united by an identity grounded elsewhere;
the middle term, as constituted in the syllogisms we have examined,
ought to be the conceptual unity of these syllogisms
but is in fact only a formal determinateness
that is not posited as their concrete unity.
But what is thus presupposed by each and
every of these mediations is not merely
a given immediacy in general,
as is the case for the mathematical syllogism,
but is itself a mediation,
namely of each of the other two syllogisms.
Therefore, what is truly present here is not
a mediation based on a given immediacy,
but a mediation based on mediation.
And this mediation is not quantitative,
not one that abstracts from the form of mediation,
but is rather a self-referring mediation,
or the mediation of reflection.
The circle of reciprocal presupposing
which these syllogisms bring to closure is
the turning back of this presupposing into itself;
a presupposing that in this turning back forms a totality,
and has the other to which every single syllogism refers,
not outside by virtue of abstraction,
but included within the circle.

Further, from the side of the single determinations of form
it has been shown that in this whole of formal syllogisms
each single determination has in turn occupied
the place of the middle term.
As immediate, this term was determined as particularity;
thereupon, through dialectical movement
it determined itself as singularity and universality.
Likewise did each of these determinations occupy the
places of both of the two extremes.
The merely negative result is the dissolution of
the qualitative determinations of form
into the merely quantitative, mathematical syllogism.
But what we truly have here is the positive result,
namely that mediation occurs,
not through any single qualitative determinateness of form,
but through the concrete identity of the determinacies.
The deficiency and formalism of the three figures of the syllogism
just considered consists precisely in this,
that one such single determinateness was
supposed to constitute the middle term in it.
Mediation has thus determined itself as
the indifference of the immediate or abstract determinations of form
and the positive reflection of one into the other.
The immediate syllogism of existence has thereby
passed over into the syllogism of reflection.

B. THE SYLLOGISM OF REFLECTION

The course of the qualitative syllogism has sublated
the abstractness of its terms;
the syllogistic term has thus posited itself as
a determinateness in which also
the other determinateness shines reflectively.
Besides the abstract terms,
there is also present in the syllogism
the connection of the terms,
and in the conclusion this connection is
posited as one which is mediated and necessary;
in truth, therefore, each determinateness is posited,
not singly by itself,
but with reference to the others,
as concrete determinateness.

The middle term was the abstract particularity,
an isolated simple determinateness,
and was a middle only externally
and relative to the self-subsisting extremes.
This term is now posited as the totality of the determinations;
thus it is the posited unity of the extremes;
but this unity is at first that of a reflection
embracing the extremes within itself;
an embracing which, as a first sublating of immediacy
and a first connecting of the determinations,
is not yet the absolute identity of the concept.

The extremes are the determinations of
the judgment of reflection, singularity proper,
and universality as a determination of relation,
or a reflection that embraces a manifold within itself.
But, as was shown in connection with the judgment of reflection,
the singular subject also contains,
besides the mere singularity that belongs to form,
determinateness as universality absolutely reflected into itself,
as presupposed, that is, here still immediately assumed, genus.

From this determinateness of the extremes,
which belongs to the course of the determination of the judgment,
there results the more precise content of the middle,
which is what counts most in the syllogism,
for it is the middle that distinguishes
the syllogism from judgment.
The middle contains
(1) singularity;
(2) but singularity expanded into universality, as an “all”;
(3) the universality that lies at the basis,
uniting singularity and abstract universality in itself, the genus.
The syllogism of reflection is thus the first
to possess genuine determinateness of form,
for the middle is posited as the totality of determinations;
the immediate syllogism is by contrast indeterminate
because the middle is still only abstract particularity
in which the moments of its concept are not yet posited.
This first syllogism of reflection may be called
the syllogism of allness.

a. The syllogism of allness

1. The syllogism of allness is
the syllogism of the understanding in its perfection
but more than that it is not yet.
That the middle in it is not abstract particularity
but is developed into its moments
and is therefore concrete,
is indeed an essential requirement of the concept.
But at first the form of the allness
gathers the singular into universality only externally,
and conversely the singular behaves in the universality
still as an immediate that subsists on its own.
The negation of the immediacy of the determinations
which was the result of the syllogism of existence is
only the first negation, not yet the negation of the negation,
or absolute immanent reflection.
The singular determinations that
the universality of reflection holds within still lie,
therefore, at the basis of that universality;
in other words, allness is not yet the universality of the concept,
but the external universality of reflection.

The syllogism of existence was contingent
because its middle term was
one single determinateness of the concrete subject
and as such admitted of a multitude of other such middle terms,
and consequently the subject could be conjoined
in conclusion with an indeterminate number of other predicates,
with opposite predicates as well.
But since the middle term now contains singularity
and is thereby itself concrete,
only a predicate that concretely belongs to
the subject can be attached to the latter by means of it.
For instance, if from the middle term “green”
the conclusion is made to follow that a painting is pleasing,
because green is pleasing to the eye,
or if a poem, a building, etc., is said to be beautiful
because it possesses regularity,
the painting, the poem, the building, etc.,
may nonetheless still be ugly on
account of other determinations
from which this predicate “ugly” might be deduced.
By contrast, when the middle term has the determination of allness,
it contains the green, the regularity, as a concreted term
which for that very reason is not
the abstraction of a mere green, a mere regular, etc.;
only predicates commensurate with concrete totality
may now be attached to this concreted term.
In the judgment, “what is green or regular is pleasing,”
the subject is only the abstraction of green, regularity;
in the proposition, “all things green or regular are pleasing,”
the subject is on the contrary all actual concrete things
that are green or regular things, therefore,
that are intended as concreted with all the properties
that they may also have besides the green or the regularity.

2. However, this very reflective perfection
of the syllogism makes of it a mere illusion.
The middle term has the determinateness of “all,”
to which there is immediately attached in the major
the predicate which in the conclusion is
then conjoined with the subject.
But the “all” is “all singulars”;
in it, therefore, the subject already possesses
that predicate immediately;
it does not first obtain it
by means of the syllogistic inference.
Or again, the subject obtains a predicate as
a consequence through the conclusion;
but the major premise already contains this conclusion in it;
therefore the major premise is not correct on its own account,
or is not an immediately presupposed judgment,
but itself already presupposes the conclusion
of which it should be the ground.

In the much cited syllogism:

All humans are mortal,
Now Gaius is a human,
Therefore Gaius is mortal,

the major premise is correct only because
and to the extent that the conclusion is correct;
were Gaius by chance not mortal,
the major premise would not be correct.
The proposition which was supposed to be the conclusion
must be correct on its own, immediately,
for otherwise the major premise would not include all singulars;
before the major premise can be accepted as correct,
the antecedent question is whether the conclusion
may not be a counter-instance of it.

3. It followed from the concept of the syllogism,
with regard to the syllogism of existence,
that the premises, as immediate, contradicted the conclusion,
that is to say, contradicted the mediation
that the concept of the syllogism requires;
that the first syllogism thus presupposed other syllogisms,
and conversely these presupposed the first.
In the syllogism of reflection this result
is posited in the syllogism itself:
the major premise presupposes its conclusion,
for it contains the union of the singular with a
predicate that would have to be a conclusion first.
What we have here in fact can therefore be expressed by saying that
the syllogism of reflection is only an external,
empty reflective semblance of syllogistic inference;
that therefore the essence of the inference rests on
subjective singularity;
this singularity thus constitutes the middle term and
is to be posited as such:
singularity which is singularity as such and pos-
sesses universality only externally.
Or what has been shown on closer
inspection of the content of the syllogism of reflection is that the sin-
gular stands connected to its predicate immediately,
not by way of an inference,
and that the major premise,
the union of a particular with a universal,
or more precisely of a formal universal with a universal in itself,
is mediated through the connection of the singularity
that is present in the formal universal,
of singularity as allness.
But this is the syllogism of induction.

b. The syllogism of induction

1. The syllogism of allness comes under
the schema of the first figure, S-P-U;
the syllogism of induction under that of the second, U-S-P,
because it again has singularity for its middle term,
not abstract singularity but singularity as completed,
that is to say, posited with its opposite determination,
that of universality.
The one extreme is some predicate or other
which is common to all these singulars;
its connection with them makes up
the kind of immediate premises,
of which one was supposed to be the conclusion
in the preceding syllogism.
The other extreme may be the immediate genus,
as it is in the middle term of the preceding syllogism,
or in the subject of the universal judgment,
and which is exhausted in the collection of
singulars or also species of the middle term.
Accordingly, the syllogism has this configuration:

s
s
U -- P
s
s
ad
infinitum.

2. The second figure of the formal syllogism, U-S-P,
does not correspond to this schema,
because the S that constitutes the middle term
did not subsume or was not a predicate.
In induction this deficiency is eliminated;
here the middle term is “all singulars”;
the proposition, U-S, which contains
as the subject the objective universal
or the genus set apart as an extreme,
has a predicate which is
of at least equal extension  as the subject
and is consequently identical with it for external reflection.
Lion, elephant, etc., constitute the genus of quadruped;
the difference, that the same content is posited
once in singularity and again in universality,
is thus just an indifferent determination of form
an indifference which in the syllogism of reflection is
the posited result of the formal syllogism
and is posited here through the equality of extension.

Induction, therefore, is not the syllogism
of mere perception or of contingent existence,
like the second figure corresponding to it,
but the syllogism of experience;
of the subjective gathering together of singulars in the genus,
and of the conjoining of the genus with a universal determinateness on
the ground that the latter is found in all singulars.
It also has the objective significance
that the immediate genus has determined itself
through the totality of singularity as
a universal property and possesses its existence
in a universal relation or mark.
But the objective significance of this syllogism,
as it was of the others, is at first
only its inner concept,
and is not as yet posited in it.

3. On the contrary, induction is
essentially still a subjective syllogism.
The middle terms are the singulars in their immediacy,
the collecting of them into a genus through the allness
is an external reflection.
Because of the persisting immediacy of the singulars
and because of the externality that derives from it,
the universality is only completeness,
or rather, it remains a task.
In induction, therefore, there recurs
the progression into the bad infinity;
singularity ought to be posited as
identical with universality,
but since the singulars are equally posited as
immediate, the intended unity remains only a perpetual ought;
it is a unity of likeness;
the terms which are supposed to be identical are
at the same time supposed not to be identical.
The a, b, c, d, e, constitute the genus
only further on, in the infinite;
they do not yield a complete experience.
The conclusion of induction thus remains problematic.

But induction, by expressing that perception,
in order to become experience,
ought to be carried on to infinity,
presupposes that the genus is in and for itself
conjoined with its determinateness.
In this, it in fact rather presupposes
its conclusion as something immediate,
just as the syllogism of allness
presupposes the conclusion for one of its premises.
An experience that rests on induction is
assumed as valid even though
the perception is admittedly not complete;
it may be assumed, however,
that there is no counter-instance
to the experience only if
the latter is true in and for itself.
Inference by induction, therefore, is
based indeed on an immediacy,
but not on the immediacy on which it is supposed to be based,
not on a singularity that exists immediately,
but on one that exists in and for itself,
on the universal.
The fundamental character of induction is
that it is a syllogistic inference;
if singularity is taken as
the essential determination of the middle term,
but universality as only the external determination,
then the middle term would fall apart into two disjoined parts,
and there would be no inference;
this externality belongs rather to the extremes.
Singularity can only be a middle term
if immediately identical with the universality;
such a universality is in truth
objective universality, the genus.
The matter can also be viewed in this way:
universality is external but essential
to the determination of the singularity
which is at the basis of the middle term of induction;
such an external is just as much
immediately its opposite, the internal.
The truth of the syllogism of induction is therefore
a syllogism that has for its middle term a singularity
which is immediately in itself universality.
This is the syllogism of analogy.

c. The syllogism of analogy

1. This syllogism has the third figure
of the immediate syllogism, S-U-P,
for its abstract schema.
But its middle term is no longer
some single quality or other
but a universality
which is the immanent reflection of a concreted term
and is therefore its nature;
and conversely, since it is thus
the universality of a concreted term,
it is at the same time in itself this concreted term.
Here, therefore, a singular is the middle term,
but a singular taken in its universal nature;
there is moreover another singular, an extreme term,
which has the same universal nature as
the other which is the middle term.
For example:

The earth has inhabitants,
The moon is an earth,
Therefore the moon has inhabitants.

2. Analogy is all the more superficial,
the more the universal in which
the two extremes are united,
and in accordance with which the one extreme
becomes the predicate of the other,
is a mere quality or, since quality is
a matter of subjectivity,
is some distinctive mark or other
and the identity of the extremes is
therein taken as just a similarity.
But this kind of superficiality to which
a form of understanding or of reason is reduced
debased to the sphere of mere representation
should have no place in logic.
Also unacceptable is to present
the major premise of this syllogism
as though it should run:
“That which is similar to an object in one distinctive mark
is similar to it in other such marks as well.”
On this formulation, the form of the syllogism is
expressed in the shape of a content
while the empirical content,
the content properly so called,
is together relegated to the minor premise.
So, for example, could also the whole form of the first syllogism
be expressed as its major premise:
“That which is subsumed under another thing
in which a third thing inheres has
that third thing inhering in it too;
but now . . . etc.”
But what matters in the syllogism as such is
not the empirical content,
and to make its own form
the content of a major premise
makes just as little difference
as to take any other empirical content for that purpose.
Nothing of consequence follows for the syllogism of analogy
from a content that contains nothing
but the form peculiar to that syllogism,
just as nothing of consequence would have followed
for the first syllogism from having as its content
the form that makes the syllogism a syllogism.
What counts is always the form of the syllogism,
whether the latter has itself or something else
for its empirical content.
So the syllogism of analogy is
a form peculiarly its own,
and it is vacuous not to want to regard it as such
on the ground that that form could be made into
the content or matter of a major premise
whereas matter is no concern in logic.
What might tempt one to this view
in regard to the syllogism of analogy,
and perhaps in regard to the syllogism of induction too,
is that the middle term in them,
and also the extremes,
are more determined than they are in the
merely formal syllogism,
and therefore the determinations of form,
since they are no longer simple and abstract,
must also take on the appearance of a content determination.
But that the form determines itself to content is
first of all a necessary advance on the part of the formal side,
and therefore an advance that touches
the nature of the syllogism essentially;
secondly, such a content determination cannot, therefore,
be regarded as any other empirical content,
and abstraction cannot be made from it.

When we consider the syllogism of analogy
with its major premise expressed as above, namely,
“if two subject matters agree in one or more properties,
then a further property of one also belongs to the other,”
it may seem that this syllogism contains four terms,
the quaternio terminorum,
a circumstance that brings with it the difficulty of
how to bring analogy into the form of a formal syllogism.
There are two singulars;
for a third, a property immediately assumed as common,
and, for a fourth, the other properties
that one singular possesses immediately
but the other first comes to possess
only by means of the syllogism.
This is so because, as we have seen,
in the syllogism of analogy
the middle term is posited as singularity
but immediately also as the true
universality of the singularity.
In induction, the middle term is,
apart from the extremes,
an indeterminate number of singulars;
this syllogism, therefore, required the enumeration
of an infinite number of terms.
In the syllogism of allness the universality
in the middle term is still only
the external form determination of the allness;
in the syllogism of analogy, on the contrary,
it is as essential universality.
In the above example, the middle term, “the earth,”
is taken as something concrete which, in truth,
is just as much a universal nature or genus
as it is a singular.

From this aspect, the quaternio terminorum would not make analogy
an imperfect syllogism.
But it would make it so from another aspect;
for although the one subject has
the same universal nature as the other,
it is undetermined whether the determinateness,
which is inferred to pertain also to the second subject,
pertains to the first because of its nature in general
or because of its particularity;
for example, whether the earth has inhabitants
as a heavenly body in general
or only as this particular heavenly body.
Analogy is still a syllogism of reflection
inasmuch as singularity and universality
are united in its middle term immediately.
Because of this immediacy, the externality of
the unity of reflection is still there;
the singular is the genus only in itself, implicitly;
it is not posited in this negativity
by which its determinateness would be
the genus's own determinateness.
For this reason the predicate that belongs
to the singular of the middle term
is not already the predicate of the other singular,
even though the two singulars both belong to the one genus.

3. S-P (“the moon is inhabited”) is the conclusion;
but the one premise (“the earth is inhabited”) is likewise S-P;
in so far as S-P is supposed to be a conclusion,
it entails the requirement that that premise also be S-P.
This syllogism is thus in itself the demand
to counter the immediacy that it contains;
or again, it presupposes its conclusion.
One syllogism of existence has its presupposition
in the other syllogism of existence.
In the syllogisms just considered,
the presupposition has been moved into them,
because they are syllogisms of reflection.
Since the syllogism of analogy is therefore
the demand that it be mediated as against the immediacy
with which its mediation is burdened,
what it demands is the sublation of the moment of singularity.
Thus there remains for the middle term the objective universal,
the genus purified of immediacy.
In the syllogism of analogy the genus was
a moment of the middle term
only as immediate presupposition;
since the syllogism itself demands
the sublation of the presupposed immediacy,
the negation of singularity and hence the universal is
no longer immediate but posited.
The syllogism of reflection contained
the first negation of immediacy;
the second has now come on the scene,
and with it the external universality of reflection is
determined as existing in and for itself.
Regarded from the positive side,
the conclusion shows itself to be
identical with the premises,
the mediation to have rejoined its presupposition,
and what we have is thus an identity of
the universality of reflection
by virtue of which it becomes
a higher universality.

Reviewing the course of the syllogism of reflection,
we find that mediation is in general
the posited or concrete unity
of the form determinations of the extremes;
reflection consists in this positing of
the one determination in the other;
the mediating middle is thus allness.
But it is singularity that
proves to be the essential ground of mediation
while universality is only as a
n external determination in it, as completeness.
But universality is essential to the singular
if the latter is to be the conjoining middle term;
is therefore to be taken as an implicitly existing universal.
But the singular is not united with it in just this positive manner
but is sublated in it and is a negative moment;
thus the universal is the genus posited
as existing in and for itself,
and the singular as immediate is rather
the externality of the genus,
or it is an extreme.
The syllogism of reflection, taken in general,
comes under the schema P-S-U
in which the singular is still as such
the essential determination of the middle term;
but since its immediacy has been sublated,
the syllogism has entered under the formal schema S-U-P,
and the syllogism of reflection has thus passed over
into the syllogism of necessity.

C. THE SYLLOGISM OF NECESSITY

The mediating middle has now determined itself
(1) as simple determinate universality,
like the particularity in the syllogism of existence,
but (2) as objective universality,
that is to say, one that, like
the allness of the syllogism of reflection,
contains the whole determinateness of the different extremes;
this is a completed but simple universality,
the universal nature of the fact, the genus.
This syllogism is full of content,
because the abstract middle term of the syllogism of existence
has posited itself to be determinate difference,
in the way it is as the middle term of the syllogism of reflection,
but this difference has again reflected itself into simple identity.
This syllogism is for this reason the syllogism of necessity,
because its middle term is not any adventitious immediate content
but is the immanent reflection of the determinateness of the extremes.
These have their inner identity in the middle term,
whose content determinations are
the form determinations of the extremes.
Consequently, what differentiates the terms is a form
which is external and unessential
and the terms themselves are as moments of a necessary existence.
This syllogism is at first immediate and formal
in the sense that what holds the terms together is
the essential nature, as content,
and this content is in the distinguished terms only in different form,
and the extremes are by themselves only an unessential subsistence.
The realization of this syllogism is a matter
of determining it  in such a way that
the extremes are equally posited as this totality
which initially the middle term is,
and the necessity of the connection,
which is at first only the substantial content,
shall be a connection of the posited form.

a. The categorical syllogism

1. The categorical syllogism has
the categorical judgment
for one or for both of its premises.
Associated with this syllogism,
just as with that judgment,
is the more specific signification
that its middle term is
the objective universality.
Superficially, the categorical syllogism is
also taken for nothing more than
a mere syllogism of inherence.

Taken in its full import,
the categorical syllogism is
the first syllogism of necessity,
one in which a subject is
conjoined with a predicate
through its substance.
But when elevated to
the sphere of the concept,
substance is the universal,
so posited to be in and for itself
that it has for its form
or mode of being, not accidentality,
as it has in the relation specific to it,
but the determination of the concept.
Its differences are therefore
the extremes of the syllogism,
specifically universality and singularity.
This universality, as contrasted with the genus
that more closely defines the middle term,
is abstract or is a universal determinateness:
it is the accidentality of substance
summed up in a simple determinateness
which is, however, the substance's
essential difference, its specific difference.
Singularity, for its part, is the actual,
in itself the concrete unity of genus and determinateness
though here, in the immediate syllogism,
it is immediate singularity at first,
accidentality summed up in the form of
a subsistence existing for itself.
The connection of this extreme term to
the middle term constitutes a categorical judgment;
but since the other extreme term also, as just determined,
expresses the specific difference of the genus
or its determinate principle,
this other premise is also categorical.

2. This syllogism, as the first and
therefore immediate syllogism of necessity,
comes in the first instance under
the schema of the formal syllogism, S-P-U.
But since the middle term is
the essential nature of the singular
and not just one or other of its determinacies or properties,
and likewise the extreme of universality is
not any abstract universal,
nor just any singular quality either,
but is rather the universal determinateness of the genus,
its specific difference,
we no longer have the contingency of a subject
being conjoined with just any quality
through just any middle term.
Consequently, since
the connections of the extremes with the middle term
also do not have the external immediacy
that they have in the syllogism of existence,
we do not have coming into play
the demand for proof in the sense in which
it occurred in the case of that other syllogism
and led to an infinite progression.

Further, this syllogism does not presuppose
its conclusion for its premises,
as in the syllogism of reflection.
The terms, in keeping with the substantial content,
stand to one another in a connection of identity
that exists in and for itself;
we have here one essence running through the three terms;
an essence in which the determinations of
singularity, particularity, and universality
are only formal moments.
To this extent, therefore,
the categorical syllogism is no longer subjective;
in that connection of identity, objectivity begins;
the middle term is the identity,
full of content, of its extremes,
and these are contained in it in their
self-subsistence,
for their self-subsistence is
the said substantial universality
which is the genus.
The subjective element of the syllogism
consists in the indifferent subsistence
of the extremes with respect to the concept
or the middle term.

3. But there is still a subjective element in this syllogism,
for that identity is still the substantial identity
or content but is not yet identity of form at the same time.
The identity of the concept still is an inner bond
and therefore, as connection, still necessity;
the universality of the middle term is
solid, positive identity,
but is not equally the negativity of its extremes.
The immediacy of this syllogism,
which is not yet posited as what it is in itself,
is more precisely present in this way.
The truly immediate element
of the syllogism is the singular.
This singular is subsumed under its genus as middle term;
but subsumed under the same genus are also
an indeterminate number of many other singulars;
it is therefore contingent that only this singular
is posited as subsumed under it.
But further, this contingency does not belong
only to an external reflection
that finds the singular posited in
the syllogism to be contingent
by comparison with others;
on the contrary, it is because the singular
is itself connected to the middle term
as its objectivity universality
that it is posited as contingent,
as a subjective actuality.
From the other side,
because the subject is an immediate singular,
it contains determinations that are not
contained in the middle term as the universal nature;
it also has, therefore, a concrete existence
which is indifferent to the middle term,
determined for itself and with a content of its own.
Therefore, conversely, this other term also has
an indifferent immediacy and a concrete existence
distinct from the former.
The same relation also obtains between
the middle term and the other extreme;
for this too likewise has the determination of immediacy,
hence of a being which is contingent
with respect to the middle term.

Accordingly, what is posited
in the categorical syllogism are
on the one hand,
extremes that are so related to the middle term
that they have objective universality
or self-subsistent nature in themselves,
and are at the same time immediate actualities,
hence indifferent to one another.
On the other hand,
they are equally contingent,
or their immediacy is as sublated in their identity.
But this identity, because of the self-subsistence
and totality of the actuality,
is only formal, inner identity,
and the syllogism of necessity has thereby
determined itself to the hypothetical syllogism.

b. The hypothetical syllogism

1. The hypothetical judgment contains only
the necessary connection without
the immediacy of the connected terms.
“If A is, so is B”;
or, the being of A is also just as much
the being of an other, of the B;
with this, it is not as yet said either that A is, or that B is.
The hypothetical syllogism adds this immediacy of being:

If A is, so is B,
But A is,
Therefore B is.

The minor premise expresses by itself
the immediate being of the A.
But it is not only this that is added to the judgment.
The conclusion contains the connection of subject and predicate,
not as the abstract copula,
but as the accomplished mediating unity.
The being of the A is to be taken, therefore,
not as mere immediacy but essentially
as middle term of the syllogism.
This needs closer examination.

2. In the first place,
the connection of the hypothetical judgment is
the necessity or the inner substantial identity
associated with the external diversity of concrete existence;
an identical content lying internally as its basis.
The two sides of the judgment are both, therefore,
not an immediate being, but a being held in necessity,
hence one which is at the same time sublated
or only being as appearance.
The two behave, moreover, as sides of the judgment,
as universality and singularity;
the one, therefore, is the above content
as totality of determinations,
the other as actuality.
Yet it is a matter of indifference
which side is taken as universality
and which as singularity.
That is to say, inasmuch as
the conditions are still the inner,
abstract element of an actuality,
they are the universal,
and it is by being held together
in one singularity
that they step into actuality.
Conversely, the conditions are
a dismembered and dispersed appearance
that gains unity and meaning,
and a universally valid existence,
only in actuality.
The relation that is here being assumed
between the two sides of condition and conditioned
may however also be taken to be one of
cause and effect, ground and consequence.
This is a matter of indifference here.
The relation of condition, however,
corresponds more closely to the one that
obtains in the hypothetical judgment and syllogism
inasmuch as condition is essentially
an indifferent concrete existence,
whereas ground and cause are inherently a transition;
moreover, condition is a more universal condition
in that it comprehends both sides of the relation,
since effect, consequence, etc.,
are just as much the condition of cause and ground
as these are the condition of them.

Now A is the mediating being in so far as it is,
first, an immediate being, an indifferent actuality,
but, second, in so far as it is equally
inherently contingent, self-sublating being.
What translates the conditions into the actuality
of the new shape of which they are the conditions is
the fact that they are not being as an abstract immediacy,
but being according to its concept
becoming in the first instance,
but more determinedly
(since the concept is no longer transition)
singularity as self-referring negative unity.
The conditions are a dispersed material
awaiting and requiring application;
this negativity is the mediating means,
the free unity of the concept.
It determines itself as activity,
for this middle term is
the contradiction of objective universality,
or of the totality of the identical content
and the indifferent immediacy.
This middle term is no longer, therefore,
merely inner but existent necessity;
the objective universality contains its
self-reference as simple immediacy, as being.
In the categorical syllogism this moment is at first
a determination of the extremes;
but as against the objective universality of the middle term,
it determines itself as contingency,
hence as something which is only posited and also sublated,
something that has returned into the concept
or into the middle terms as unity,
a unity which is now in its objectivity also being.

The conclusion, “therefore B is,”
expresses the same contradiction,
that B exists immediately
but at the same time
through an other or as mediated.
According to its form,
it is therefore the same concept
that the middle term is,
distinguished from necessity only as the necessary,
in the totally superficial form of singularity
as contrasted with universality.
The absolute content of A and B is the same;
for ordinary representation,
they are two different names
for the same basic thing,
since representation fixes the appearances
of the diversified shape of existence
and distinguishes the necessary from its necessity;
but to the extent that necessity were to be separated from B,
the latter would not be the necessary.
What we have here, therefore, is
the identity of the mediating term and the mediated.

3. The hypothetical syllogism is the first
to display the necessary connection
as a connectedness through form or negative unity,
just as the categorical syllogism displays
it through positive unity,
the solid content, the objective universality.
But necessity merges with the necessary;
the form-activity of translating
the conditioning actuality into the conditioned
is in itself the unity into which
the determinacies of the oppositions
previously let free into indifferent existence are sublated,
and where the difference of A and B is an empty name.
The unity is therefore a unity reflected into itself,
and hence an identical content,
and is this content not only implicitly in itself
but, through this syllogism, it is also posited,
for the being of A is also not its own being
but that of B and vice versa,
and in general the being of the one
is the being of the other
and, as determined in the conclusion,
their immediate being or indifferent determinateness
is a mediated one;
therefore, their externality has been sublated,
and what is posited is their unity withdrawn into itself.
The mediation of the syllogism has thereby determined itself
as singularity, immediacy, and self-referring negativity,
or as a differentiating identity that retrieves itself
into itself out of this differentiation, as absolute form,
and for that very reason as objective universality,
self-identical existent content.
In this determination,
the syllogism is the disjunctive syllogism.

c. The disjunctive syllogism

As the hypothetical syllogism comes in general
under the schema of the second figure of
the formal syllogism, U-S-P,
so the disjunctive comes
under the schema of the third, S-U-P.
The middle term, however, is
a universality replete with form;
it has determined itself as totality,
as developed objective universality.
The middle term, therefore,
is universality as well as
particularity and singularity.
As that universality,
it is in the first place
the substantial identity of the genus,
but this identity is secondly one
in which particularity is included,
but again, included as equal to it
therefore as a universal sphere
that contains its total particularity,
the genus sorted out in its species,
an A which is B as well as C and D.
But particularization is differentiation
and as such equally the either-or of B, C, D negative unity,
the reciprocal exclusion of the determinations.
This excluding, moreover, is now not just reciprocal,
the determination not merely relative,
but is also just as much self-referring determination,
the particular as singularity to the exclusion of the others.

A is either B or C or D,
But A is B,
Therefore A is neither C nor D.
Or also:
A is either B or C or D,
But A is neither C nor D,
Therefore A is B.

A is subject not only in the two premises
but also in the conclusion.
It is a universal in the first premise
and in its predicate the universal sphere
particularized in the totality of its species;
in the second premise, it is as a determinate,
or as a species;
in the conclusion it is posited as the excluding,
singular determinateness.
Or again, in the minor it is already exclusive singularity,
and in the conclusion it is positively posited
as the determinate that it is.

Consequently, what as such appears to be meditated is
the universality of A with the singularity.
But the mediating means is this A
which is the universal sphere of its particularizations
and is determined as a singular.
What is posited in the disjunctive syllogism is thus
the truth of the hypothetical syllogism,
the unity of the mediator and the mediated,
and for that reason the disjunctive syllogism is
equally no longer a syllogism at all.
For the middle term which is posited in it
as the totality of the concept itself contains
the two extremes in their complete determinateness.
The extremes, as distinct from this middle term,
are only a positedness to which
there no longer accrues any proper determinateness
of its own as against the middle term.

If we consider the matter with narrower reference
to the hypothetical syllogism,
we find that there was in the latter
a substantial identity as the inner bond of necessity,
and a negative unity distinct from it,
namely the activity or the form
that translated one existence into another.
The disjunctive syllogism is in general
in the determination of universality,
its middle term is the A as genus and as perfectly determined;
also posited through this unity is the earlier inner content
and, conversely, the positedness
or the form is not the external negative unity
over against an indifferent existence
but is identical with that solid content.
The whole form determination of the concept is
posited in its determinate difference
and at the same time in the simple identity of the concept.

In this way the formalism of the syllogistic inference,
and consequently the subjectivity of the syllogism
and of the concept in general,
has sublated itself.
This formal or subjective factor consisted in that
the middle mediating the extremes is
the concept as an abstract determination
and is therefore distinct from
the terms whose unity it is.
In the completion of the syllogism,
where the objective universality is
equally posited as the totality
of the form determinations,
the distinction of mediating and mediated has
on the contrary fallen away.
That which is mediated is itself
an essential moment of what mediates it,
and each moment is the totality of what is mediated.

The figures of the syllogism
exhibit each determinateness of
the concept singly as the middle term,
a middle term which is at the same time
the concept as an ought,
the requirement that the mediating factor
be the concept's totality.
The different genera of the syllogism
exhibit instead the stages in the repletion
or concretion of the middle term.
In the formal syllogism the middle is
posited as totality only through all the determinacies,
but each singly, discharging the function of mediation.
In the syllogism of reflection,
the middle term is the unity
gathering together externally
the determinations of the extremes.
In the syllogism of necessity the middle
has determined itself as a unity
which is just as developed and total as it is
simple, and the form of the syllogism,
which consisted in the difference of
the middle term over against its extremes,
has thereby sublated itself.
With this the concept in general has been realized;
more precisely, it has gained
the kind of reality which is objectivity.
The first reality was that the concept,
in itself negative unity, partitions itself
and as judgment posits its determinations
in determinate and indifferent difference,
and in the syllogism it then sets itself over against them.
Since it is still in this way the inwardness
of this now acquired externality,
in the course of the syllogisms this externality is
equated with the inner unity;
the different determinations return into the latter
through the mediation that unites them at first in a third term,
and as a result the externality exhibits, in itself,
the concept which, for its part,
is no longer distinct from it as inner unity.
Conversely, however, that determinateness of the concept
which was considered as reality is equally a positedness.
For the identity of the concept's inwardness and externality
has been exhibited as the truth of the concept
not only in this result;
on the contrary, already in the judgment
the moments of the concept remain,
even in their reciprocal indifference,
determinations that have significance
only in their connection.
The syllogism is mediation,
the complete concept in its positedness.
Its movement is the sublation of this mediation
in which nothing is in and for itself,
but each thing is only through the meditation of an other.
The result is therefore an immediacy that has emerged
through the sublation of the mediation,
a being which is equally identical with mediation
and is the concept that has restored itself
out of, and in, its otherness.

This being is therefore a fact
which is in and for itself: objectivity.
