The doctrine of being

Being is the concept only as it is in itself.
Its determinations have being,
in their difference they are
others opposite one another,
and their further determination
(the form of the dialectical)
is a process of passing over into an other.
This progressive determination is
at once a matter of setting forth
and thereby unfolding the concept,
as it is in itself, and at the same time
the process of being entering into itself,
a deepening of it within itself.
The explication of the concept
in the sphere of being
becomes the totality of being,
precisely to the extent that
the immediacy of being
or the form of being as such
is sublated in the process.

Being itself as well as the subsequent determinations,
not only those of being but also
the logical determinations in general,
can be regarded as the definitions of the absolute,
as metaphysical definitions of God.
More specifically, only the first simple
determination within a given sphere,
and then the third,
which is the return from a difference
to the simple relation-to-itself,
can always be regarded in this way.
For, to define God metaphysically means
to express his nature in thoughts as such.
But logic comprises all thoughts as they are
while still in the form of thoughts.
By contrast, the second determinations,
making up a given sphere in its difference,
are the definitions of the finite.
But if the form of definitions were used,
this would entail envisaging a representational substratum.
For even the absolute, what is supposed
to express God in the sense and in the form of thought,
remains merely an intended thought,
a substratum that as such is indeterminate,
relative to its predicate as the determinate
and actual expression in thought.
Because the thought, the basic matter solely at issue here,
is contained only in the predicate,
the form of a proposition, like that subject,
is something completely superfluous.

A. QUALITY

a. Being

Pure being constitutes the beginning,
because it is pure thought as well as
the undetermined, simple immediate, and
the first beginning cannot be anything
mediated and further determined.

    All the doubts and reminders that might be raised against
    beginning the science with abstract, empty being
    take care of themselves through the simple consciousness of
    what is implied by the nature of a beginning.
    Being can be determined as 'I = I',
    as the absolute indifference or identity, etc.
    In the need to begin with something absolutely certain,
    the certainty of oneself,
    or with a definition
    or intuition of the absolutely true,
    these and other similar forms can be
    regarded as what must be the first.
    However, insofar as mediation is
    already present within each of these forms,
    they are not truly the first.
    Mediation means to have gone from a first to a second
    and to emerge from something differentiated.
    If 'I = I' or even the intellectual intuition is
    genuinely taken as simply the first,
    then in this pure immediacy
    it is nothing else but being, just as,
    conversely, pure being, insofar as it is
    no longer this abstract being,
    but being that contains mediation within itself,
    is pure thinking or intuiting.

    When being is expressed as a predicate of the absolute,
    this provides the first definition of the latter:
    the absolute is being.
    This is (in the thought) the absolutely first,
    most abstract, and most impoverished definition.
    It is the definition of the Eleatics,
    but at the same time also the familiar one
    that God is the sum total of all realities.
    The point is that one is supposed to abstract
    from the limitedness inherent in every reality,
    so that God is nothing but the real in all reality,
    the supremely real.
    Insofar as reality already contains a reflection,
    this idea is expressed more immediately in what
    Jacobi says about the God of Spinoza, namely
    that he is the principium of being in all existence.

Now this pure being is a pure abstraction
and thus the absolutely negative
which, when likewise taken immediately, is nothing.

    1. The second definition of the absolute,
    namely that it is nothing, followed from this.
    This conclusion is, indeed, entailed by saying
    that the thing-in-itself is the undetermined,
    utterly devoid of form and therefore of content.
    So, too, if it is said that God is simply
    the supreme being and nothing else,
    then he is being declared, as such,
    to be the very same negativity.
    The nothing that Buddhists make the principle of everything
    and the ultimate end and goal of everything is the same abstraction.

    2. When the opposition is expressed in this immediate way
    as one of being and nothing, it seems all too evident
    that it is null and void for one not to try to fix
    [upon some determinate sense of] being and
    to save it from this transition.
    In this respect, thinking the matter over is bound
    to fall prey to looking for a fixed determination for being
    through which it would be differentiated from nothing.
    For instance, one may take it to be what persists in all change,
    the infinitely determinable matter and so forth,
    or again, without thinking it through, to be
    any given individual concrete existence
    the next best sensory or spiritual entity.
    However, none of these further and more concrete determinations
    leave being as pure being, as it is here immediately in the beginning.
    It is nothing only in and because of this pure indeterminacy,
    something inexpressible;
    its difference from nothing is a mere opinion.
    We are concerned here exclusively with the
    consciousness of these beginnings,
    namely that they are nothing but these empty abstractions and
    that each of them is as empty as the other.
    The drive to find in being or in both a fixed meaning is
    the very necessity that expands being and nothing and
    gives them a true, concrete meaning.
    This development is the logical elaboration and
    the progression presented in what follows.
    The process of thinking them over that finds
    deeper determinations for them is the logical thinking
    by means of which these determinations produce themselves,
    not in a contingent but in a necessary manner.
    Each subsequent meaning they receive is therefore
    to be regarded only as a more specific determination and
    a truer definition of the absolute.
    Such a definition will then no longer be an empty abstraction
    like being and nothing, but rather something concrete in which
    both being and nothing are moments.
    The highest form of nothingness for itself would be freedom,
    but freedom is the negativity that deepens itself within itself
    to the point of the utmost intensity and is itself affirmation,
    and absolute affirmation at that.

Conversely, nothing, as this immediate, self-same [category],
is likewise the same as being.
The truth of being as well as of nothing is
therefore the unity of both;
this unity is becoming.

    1. The proposition 'Being and nothing are the same'
    appears to be such a paradoxical proposition
    for the representation or the understanding
    that one perhaps believes that it is not meant seriously.
    And indeed it is one of the hardest thoughts that
    thinking imposes upon itself,
    for being and nothing are the opposite
    in its complete immediacy, that is to say,
    without there already being posited in one of them
    a determination that would contain its relation to the other.
    And yet, they do contain this determination,
    as has been demonstrated in the previous section, namely,
    the determination that is the same in both.
    The deduction of their unity is thus entirely analytical,
    just as in general the whole progression in philosophizing
    (insofar as it is a methodical, a necessary progression)
    is nothing other than merely the positing of
    what is already contained in a concept.
    But as correct as the unity of being and nothing is,
    so it is also correct that they are absolutely different,
    that the one is not what the other is.
    However, since at this point the difference
    has not yet become determinate
    (for being and nothing are still what is immediate),
    how it bears on them is something that cannot be said,
    it is something merely meant.

    2. It does not require a great deal of wit to ridicule
    the proposition that being and nothing are the same,
    or rather to bring up absurdities with the false assurance
    that they are the consequences and applications of it;
    for example, that according to that proposition
    it would be the same whether my house, my assets,
    the air we breathe, this city, the sun,
    right, spirit, God are or not.
    For one thing, in examples such as these, particular purposes
    or the utility something has for me are surreptitiously introduced,
    and it is asked whether it makes no difference to me,
    if the useful thing exists or not.
    Philosophy is indeed just the doctrine that is meant
    to liberate man from an infinite number of finite purposes and goals,
    and to make him indifferent to them such that it is indeed
    all the same to him whether such things are or not.
    But generally speaking, as soon as we are talking about some contents,
    a connection is thereby posited with other concretely existing things,
    purposes, etc. that ate presupposed as valid, and
    it is then made dependent on such presuppositions,
    whether the being or not-being of a determinate content is
    the same or not.
    A difference full of content is surreptitiously substituted for the
    empty difference between being and nothing.
    But for another thing, purposes that are in themselves
    essential, absolute concrete existences and ideas are
    placed under the determination of being or not-being.
    Such concrete objects are something quite different
    from mere beings or not-beings;
    poor abstractions such as being and nothing
    (which are the poorest of all just because
    they are the determinations only of the beginning)
    are completely inadequate to the nature of those objects;
    a genuine content has long since transcended
    these abstractions themselves and their opposition.
    In general, if something concrete is surreptitiously substituted
    for being and nothing, the usual thing happens to this thoughtlessness,
    namely it entertains and talks about something
    quite different from what is at issue.
    And what is at issue here is merely abstract being and nothing.

    3. It can easily be said that one does not comprehend
    the unity of being and nothing.
    The concept of it, however, was stated in the preceding sections,
    and it is nothing over and above what has been stated.
    Comprehending it means nothing other than apprehending this.
    But by 'comprehending', something broader than
    the concept proper is understood.
    A more manifold, richer consciousness,
    a representation is demanded,
    with the result that a concept of this sort is
    put forward as a concrete case with which
    thinking in its ordinary routine would be more familiar.
    To the extent that the inability to comprehend
    expresses only that one is unaccustomed to
    holding onto abstract thoughts without any sensory input
    and to grasping speculative sentences,
    there is nothing further to be said than this,
    namely that philosophical knowledge is indeed
    of a different sort from the kind of knowledge
    one is accustomed to in ordinary life,
    as it also is from what reigns in other sciences.
    If, however, the inability to comprehend means only
    that one is unable to represent this
    unity of being and nothing to oneself,
    then this is in fact so little the case
    that to the contrary everybody possesses
    infinitely many representations of this unity.
    That one does not possess such representations
    can mean only that one fails to recognize the concept
    under discussion in any of those representations
    and that one does not know that they are examples of it.
    The example that comes most readily to mind is that of becoming.
    Everybody has a representation of becoming
    and will equally admit that it is one representation;
    further, that when one analyses it the determination of being,
    but also that of its absolute other, nothing, is contained therein;
    furthermore, that these two determinations exist
    undivided in this one representation,
    so that becoming is thereby the unity of being and nothing.
    Another example equally ready to hand is that of the beginning.
    The basic matter is not yet in its beginning,
    but the beginning is not merely its nothing either;
    rather being is already contained therein.
    The beginning is itself also a becoming,
    but it already expresses the relation to the further progression.
    If one wanted to follow the usual procedure of the sciences,
    one might start the Logic with the representation of
    the beginning thought in its purity,
    with the beginning qua beginning,
    and to analyse this representation.
    Perhaps one would then more easily accept
    as the result of this analysis
    that being and nothing show themselves
    as undivided in a single thought.

    4. In addition, we must further note that the expressions
    'Being and nothing are the same' or
    'the unity of being and nothing' and
    similarly all other such unities
    (e.g. that of subject and object, and so on)
    are rightly objectionable.
    The awkwardness and incorrectness lies in
    the fact that the unity is emphasized,
    and while the difference is indeed contained in it
    (because the unity posited is one of being and nothing, for instance),
    this difference is not simultaneously stated and acknowledged.
    Instead, it seems that one is merely
    abstracting illegitimately from it
    and not taking it into consideration.
    Indeed, a speculative determination cannot properly be
    expressed in the form of such a proposition:
    unity is supposed to be articulated in the difference
    that is simultaneously present and posited.
    As their unity, becoming is the true expression of
    the result of being and nothing.
    It is not only the unity of being and nothing,
    but the unrest in itself,
    the unity that as relation to itself is
    not merely immobile,
    but is within itself against itself
    on account of the difference of
    being and nothing contained in it.
    Existence is, by contrast, this unity, or
    becoming in this form of unity;
    this is why existence is one-sided and finite.
    It is as if the opposition had disappeared;
    It is contained in the unity only in itself,
    but not posited in the unity.

    5. Standing in contrast to the proposition
    that being is the transitioning into nothing
    and nothing the transitioning into being
    (this being the principle of becoming)
    is the proposition that 'Nothing comes from nothing'
    or 'something can only come from something',
    the proposition of the eternity of matter, pantheism.
    The ancients made the simple reflection
    that the proposition
    'something comes from something' or
    'nothing comes from nothing'
    does indeed sublate becoming.
    For that out of which something comes to be and
    that which comes to be are one and the same.
    There is nothing here but a proposition of
    the identity of the abstract understanding.
    It must strike one as curious, however,
    to see the propositions
    'nothing comes from nothing' or
    'something comes only from something'
    put forward quite naively even in our times
    with neither any awareness that
    they are the foundation of pantheism,
    nor any familiarity with the fact that the ancients
    considered these propositions quite exhaustively.

b. Existence

The being in becoming,
as one with nothing,
and the nothing that is
likewise one with being
are only vanishing [moments].
Due to its inner contradiction,
becoming collapses into the unity
in which both are sublated.
Its result is therefore existence.

    In connection with this initial example,
    we are once and for all to be reminded
    of what was stated in § 82 and in the Remark.
    What alone can ground a progression
    and a development in knowing is
    to hold on to the results in their truth.
    Suppose a contradiction is pointed up
    in any sort of object or concept
    (and there is simply nothing anywhere
    in which a contradiction, opposite determinations,
    could not and would not have to be pointed out,
    for the understanding's process of abstracting
    violently holds on to one determinacy,
    while striving to obscure and eliminate
    the consciousness of the other determinacy
    that is contained in it).
    When such a contradiction is recognized,
    the conclusion is usually drawn that
    'Therefore, the object is nothing',
    just as Zeno first demonstrated with regard to movement,
    namely that it contradicts itself
    and that therefore it does not exist,
    or as the ancients recognized the two kinds of becoming,
    namely coming-to-be and passing away,
    to be untrue determinations by stating that the One,
    the absolute, neither comes into being nor passes away.
    This kind of dialectic thus merely stops
    at the negative side of the result
    and abstracts from what is at the same time
    actually on hand, namely a determinate result,
    here a pure nothing, but a nothing that contains being
    and likewise a being that contains nothing within itself.
    Thus, existence is
    (1) the unity of being and nothing
    in which the immediacy of these determinations has disappeared
    and with it the contradiction in their relationship,
    a unity in which they are now only moments.
    (2) Since the result is the sublated contradiction,
    it is in the form of a simple unity with itself
    or itself as being, but a being with negation or determinateness.
    It is becoming posited in the form of one of its moments,
    that of being.

(a) Existence is being with a determinacy
that is immediate or that simply is: quality.
Existence qua reflected into itself
in this its determinacy is an existent, something.
The categories that develop in connection with existence
need to be specified in a summary fashion only.

As a determinacy that simply is over against
the negation that is contained in it but distinct from it,
quality is reality.
Negation, no longer as the abstract nothing
but as an existent and something,
is only the form in the latter,
it is as being-other.
Because this being-other is its own determination,
but at first distinct from it,
quality is being-for-another:
a breadth of existence, of something.
The being of quality as such,
as opposed to this relation to something other,
is being-in-itself.

(b) The being that is fastened onto as distinct from determinacy,
the being-in-itself, would be merely the empty abstraction of being.
In existence, determinacy is one with being,
and at the same time posited as negation, limit, barrier.
Being other is thus not something indifferent outside of it
but instead its own moment.
By virtue of its quality, something is,
first, finite and, second, alterable,
so that finitude and alterability belong to its being.

Something becomes an other,
but the other is itself a something,
hence it likewise becomes an other,
and so on and so forth ad infinitum.

This infinity is the bad or negative infinity in that
it is nothing but the negation of the finite, which, however,
re-emerges afresh and thus is just as much not sublated.
In other words, this infinity expresses only that
the finite ought to be sublated.
The progression to infinity stops short at expressing
the contradiction that is contained in the finite,
namely that it is something as well as its other
and that it is the perpetual continuance of
the alternation of these determinations
each of which brings about the other.

(c) What is in fact the case is that something becomes an other
and the other generally becomes something other.
In the relation to an other,
something is itself already an other opposite it.
Hence, since that into which it makes the transition is
entirely the same as that which makes the transition
(both have no further determination than this,
which is one and the same, the determination to be an other),
something comes together only with itself
in its transition into something other,
and this relation to itself in its transition
and in the other is the true infinity.
Or, considered negatively, what is altered is the other;
it becomes the other of the other.
In this way, being but as negation of negation is
re-established and is being-for-itself.

    The dualism that makes the opposition of the finite and the infinite
    insuperable fails to make the simple observation that in this manner
    the infinite is at once only one of the two, that it is thus made into
    merely one particular for which the finite is the other particular.
    Such an infinite that is only a particular, next to the finite which
    makes up its boundary and limit, is not what it is supposed to be;
    not the infinite, but merely finite.
    In such a relationship, where the finite is hither, the infinite thither,
    the one placed on this side, the other on the other side,
    the finite is accorded the same honour of subsisting and
    being self-standing that the infinite is.
    The being of the finite is made into an absolute being.
    In such a dualism, it stands firmly for itself.
    If it were touched by the infinite,
    so to speak, it would be annihilated.
    But it is supposed to be untouchable by the infinite.
    There is supposedly an abyss,
    an insurmountable chasm between the two,
    with the infinite remaining absolutely on the other side
    and the finite on this side.
    While one may believe that the assertion
    that the finite persists steadfastly opposite the infinite
    gets one beyond all metaphysics,
    it in fact stands squarely on the grounds of
    the most ordinary metaphysics of the understanding.
    The same thing happens here which is expressed
    by the infinite progression.
    At one moment, it is admitted that the finite
    does not exist in and for itself
    that it is not a self-standing actuality,
    not an absolute being,
    that it is only transitory.
    The next moment, this is immediately forgotten
    and the finite is represented as existing entirely
    over against the infinite,
    absolutely separated from it
    and exempted from annihilation,
    as self-standing and persisting for itself.
    While such thinking believes that it is elevating itself
    to the infinite in this manner, the opposite happens to it:
    it arrives at an infinite that is merely finite,
    and, instead of leaving the finite behind,
    permanently holds onto it, making it into an absolute.

    Based on these considerations concerning the emptiness of
    the understanding's opposition of the finite and the infinite
    (one may benefit from comparing Plato's Philebus with it),
    it is easy to lapse into the expression that therefore
    the infinite and the finite are one, that the true,
    true infinity, is determined and declared to be
    the unity of the infinite and the finite.
    It is true that phrasing the matter in such a way is
    in some sense correct, but it is equally skewed and false
    (as was mentioned earlier with regard to
    the unity of being and nothing).
    Furthermore, it invites the just reproach of having
    finitized the infinite, the reproach of a finite infinite.
    For in the above phrasing the finite appears as if untouched,
    it is not explicitly stated that the finite has been sublated.
    Or, when one reflects that the finite, in being posited as one with the
    infinite, could not indeed remain what it was outside this unity, and
    that it would suffer at least some modification in its determination
    (just as an alkali combined with acid loses some of its properties),
    then the same thing should happen to the infinite, which, as the
    negative, should have to be blunted by the other equally in turn.
    And this is indeed what happens to
    the abstract, one-sided infinite of the understanding.
    However, the true infinite does not behave merely like
    the one-sided acid, but instead preserves itself.
    Negation of negation is not a neutralization.
    The infinite is the affirmative,
    and only the finite is what is sublated.

In being-for-itself, the determination of
ideality has made its entry.
Existence, construed at first only
in terms of its being or
its affirmative nature, has reality.
Thus, too, finitude is at first
determined in terms of reality.
But the truth of the finite is rather its ideality.
Likewise, the infinite of the understanding,
which posited next to the finite is
itself merely one of the two finites,
is something untrue, something ideal.
This ideality of the finite is the
chief proposition of philosophy,
and every true philosophy is
for that reason idealism.
The only thing that matters is not to take as the infinite
what is at once made into something particular and finite
in the determination of it.
This is why we have drawn attention to
this distinction here at some length.
The fundamental concept of philosophy,
the true infinite, depends on this.
This distinction is taken care of
by the very simple, and therefore perhaps unremarkable,
but irrefutable reflections contained in this section.

c. Being-for-itself

(a) Being-for-itself as relation to itself is immediacy,
and as the relation of the negative to itself
it is a being that is for itself, the One:
what is in itself devoid of any distinction,
hence, what excludes the other from itself.

(b) The relationship of the negative to itself
is a negative relationship,
hence the distinguishing of the One from itself,
the repulsion of the One:
a positing of many Ones.
In accordance with the immediacy of
that which is a being-for-itself,
these many are beings,
and the repulsion of the Ones that have being
becomes in this respect their repulsion against each other
insofar as they are on hand, or a mutual excluding.

(c) Of the Many, however, one is what the others are;
each is a One as well as one of the Many.
They are therefore one and the same.
Or, considered in itself,
repulsion as the negative behaviour of
the many Ones to each other is
equally essentially their relation to each other.
And since those to which the One relates
in its repelling are Ones,
it relates to itself in them.
Thus repulsion is equally essentially attraction,
and the excluding One or being-for-itself sublates itself.
The qualitative determinacy that has reached in the One
its determinacy in-and-for-itself
has thus passed over into determinacy qua sublated,
into being as quantity.

    The atomistic philosophy is the standpoint on which
    the absolute determines itself as being-for-itself,
    as One, and as many Ones.
    Repulsion, which shows itself in the concept of the One,
    has also been assumed to be its fundamental force.
    Not, however, attraction but coincidence, something thoughtless, is
    supposed to bring them together.
    If one is fixated on the One as One,
    its coming together with others must indeed
    be regarded as something quite extrinsic.
    The void that is adopted as the other principle
    in addition to the atoms is repulsion itself,
    represented as the existing nothing in between the atoms.
    The more recent atomism
    (and physics continues to hold on to this principle)
    has given up atoms insofar as it focuses on
    small particles, the molecules.
    In this, it has drawn closer to sensory representation
    and abandoned thoughtful determination.
    Moreover, insofar as a force of attraction is set alongside
    the force of repulsion, the opposition has, it is true,
    been made complete, and the discovery of this so-called
    force of nature has been touted a lot.
    But the relationship of both to one another
    that constitutes what is concrete and true about them
    would need to be rescued from the state of cloudy confusion
    in which it has been left even in Kant's
    Metaphysical Foundations of the Natural Sciences.
    In recent times, the atomistic approach has become
    even more important in the political than in the physical sphere.
    According to this view, the will of the individual as such is
    the principle of the state.
    The attractive force is the particularity of the needs and inclinations,
    and the universal, the state itself, is [based on]
    the external relationship of the contract.
