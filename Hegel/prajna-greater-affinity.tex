
A. THE RELATION OF INDEPENDENT MEASURES

By measures we no longer mean now
merely immediate measures,
but measures that are self-subsistent
because they become within themselves
relations which are specified,
and in this being-for-itself
they are thus a something,
things that are physical
and at first material.

(a) However, the whole,
which is a relation of such measures
is at first itself immediate.
Thus the two sides,
which are as such independent measures,
have their subsistence in
things external to each other,
and are posited in combination externally.

(b) But the self-subsistent materialities are
what they qualitatively are only in virtue of
the quantitative determination that they have as measures,
and are posited in virtue of this same quantitative connection
with others as non-indifferent from them (the so-called affinity);
they are the members of a series of such quantitative relations.

(c) This indifferent multifarious relating
concludes at the same time by cutting itself off
as an exclusive being-for-itself:
the so-called elective affinity.

a. Combination of two measures

As a measure relation, something is in itself determined by
quanta to which further qualities accrue;
the something is the connection of such qualities.
One of them is the something's in-itselfness,
according to which it is something
that exists for itself, something material
(a weight, if taken intensively;
or, taken extensively, an aggregate of parts);
the other is however the externality of this itselfness
(something abstract and idealized, space).
These qualities are quantitatively determined,
and their mutual relation is what constitutes
the qualitative nature of the material something
the ratio of weight to volume, the specifically determined gravity.
The volume, the idealized element, is the one to be taken as unit,
whereas the intensive aspect,
which in quantitative determinateness
and in comparison with the other appears as extensive magnitude,
that is, an aggregate of ones existing for themselves, is the amount.
The purely qualitative relating of the two determinacies
of magnitude in a ratio of powers has disappeared,
because with the self-subsistence of the being-for-itself
(the material being) immediacy has come back,
and in this immediacy the determinateness
of magnitude is a quantum whose relation to the other side is likewise
determined in the ordinary exponent of a direct ratio.

This exponent is the specific quantum of the something,
but it is an immediate quantum and this is determined
(as is also the specific nature of the something)
only in being compared with other exponents of like ratios.
The exponent constitutes the specific way
in which the something is determined in itself,
the inner measure characteristic of it;
but since this, the something's measure, rests on a quantum,
it is also an external and indifferent determinateness,
and the something, despite its inner determination as measure,
is for this reason subject to alteration.
The other to which it can relate as alterable
is not an aggregate of matters,
not a quantum in general;
on the contrary, against these it holds out
its specific intrinsic determination of being;
rather, it is a quantum which is at the same time
it too the exponent of such a specific ratio.
Take two things of different inner measure that
stand connected and enter into composition,
say, two metals of different specific gravity.
(In what way the two must otherwise be homogeneous
in nature in order for the composition to be possible,
e.g. that we cannot be speaking here of
a metal combining with water, this is not at issue here).
Now, on the one hand each of the two measures preserves itself
in the alteration they will incur because of
the externality of quantum, for each is a measure;
but, on the other hand, this self-preservation is itself
a negative relating to the quantum, a specification of it,
and, since this quantum is the exponent of the ratio of measure,
the self-preservation is an alteration of
the measure itself and indeed a reciprocal specification.

According to its merely quantitative determination,
the compound would be the mere sum
of the two magnitudes of the one quality
and the two magnitudes of the other quality,
e.g. the sum of the two weights
and the two volumes in a compound of two matters
of different specific weights, so that not only
would the weight of the mixture remain equal to the said sum
but also the space which the mixture occupies
would remain equal to the sum of the two spaces.
But only the weight of the mixture is found
to be in fact the sum of the weights present prior to combination;
the side capable of addition is the one which,
as the one existing for itself,
has attained stability of being
and consequently a permanent immediate quantum;
the weight of matter, or, what amounts to the same thing from
a quantitative standpoint, the aggregate of material parts.
Alteration falls instead in the exponents,
for they are the expression of the qualitative side of the compound,
of its being-for-itself as relations of measure,
and, since the quantum as such is subject
to accidental alteration by an increase which is summed,
this being-for-itself proves at the same time to be
a negating factor with respect to this externality.
Since this immanent determining of the quantitative element
cannot appear, as we have seen, in the weight,
it turns up in the other quality
which is the idealized side of the relation.
It might be surprising to sense-perception
that, upon the mixing of two specifically diverse materials,
what results is an alteration of the total volume,
normally a diminution of it,
for it is space which gives stability to matters
otherwise existing outside one another.
But this stability, in face of the negativity
which is in the being-for-itself,
is that which has no existence in itself,
is the alterable as such.
In this manner, space is posited as what it is in truth,
an idealization.

But then, not only is one of the qualitative sides
posited as alterable but measure itself,
and so also the qualitative determinateness
of the something which is based on it,
has shown itself not to be something stable within
but, like quantum in general, to have
its determinateness in other measure-relations.

b. Measure as a series of measure-relations

1. If something united with an other were what it is
only by virtue of simple qualitative determination,
and the same also applied to the other,
then the two would only sublate themselves in being compounded.
But when something is in itself a measure-relation,
it is independent yet at the same time capable of
union with another which is also in itself a measure-relation.
For in being sublated in union, each preserves itself
through the permanence of its quantitative indifference
and it behaves as the specifying moment of a new measure-relation.
Its quality is enveloped in the quantitative element;
it is therefore equally indifferent towards
the exponent of the new measure
is itself only some quantum or other,
an external determinateness,
and its indifference is displayed by the fact
that the specifically determined something runs,
with other such measures,
into precisely the same kind of neutralizations
of reciprocal measure-relations;
the specific property of the something is
not expressed in the one measure-relation alone
which is formed by it and another something.

2. This combination with a plurality of others
which are in themselves likewise measures yields different ratios
that therefore have different exponents.
The measure which is independent has the exponent of its
being-determined-in-itself only in the comparison with other measures;
neutrality with the others constitutes,
however, its real comparison with them;
it is its comparison with them through itself.
The exponents of these ratios are however diverse,
and the independent measure consequently displays
its qualitative exponent as the series of
these different amounts of which it is the unit;
a series of specific ways of relating to others.
As one immediate quantum, the qualitative exponent
expresses one single relation.
The independent measure truly differentiates itself
in the characteristic series of exponents
which it, taken as a unit, forms with
other equally independent measures;
for each of these measures, also brought in connection with
any other and taken as a unit, forms another series.
The relation of such series among themselves now constitutes
the qualitative aspect of the independent measure.

Now inasmuch as this independent measure produces
a series of exponents with a series of such independent measures,
it appears at first, when compared with another
independent measure outside this series itself,
that it is differentiated from it by virtue of
the fact that this other independent measure produces
another series of exponents with the same measures opposite to it.
But in this way these two independent measures
would not be comparable,
not in so far as each is regarded as a unit
with respect to its exponents,
and the two series thus resulting on the basis
of this connection of unit and exponents
do not differ in any determinate way.
The two measures which, as independent,
are supposed to be compared,
are at first differentiated from each other only as quanta;
in order that their ratio be determined,
the ratio requires a unit which exists for itself
and is common to both.
This determinate unit is to be sought, as shown,
only where the specific existence of
the two independent measures to be compared
is to be found, namely in the ratio
which the exponents of the ratio of
the series have to each other.
But this ratio of the exponents is
itself a unit existing for itself
and in fact determinate only in so far as the members of
the series have it as a constant ratio between them;
in that way it can be their common unit.
In it alone, therefore, lies the possibility of comparing
the two independent measures which were assumed not
to neutralize each other but to be rather reciprocally indifferent.
Each, taken by itself outside the comparison,
is the unit of the ratios that it establishes
with the opposite members:
these are the amounts relative to this unit
and hence represent the series of exponents.
But conversely, this series is the unit
for the two independent measures which
when compared with each other are as quanta to each other;
as such they are themselves different amounts
of their just indicated unit.

But further, the measures which together with the two
or rather indefinitely many reciprocally opposing
and contrasting measures
yield the series of the exponents of
the ratio of these same measures,
are equally independent in themselves;
each is a specific something with its own relevant measure-relation.
To this extent, they are similarly to be taken each as a
unit so that they have a series of exponents in
the two or rather the indeterminate plurality of measures,
which are first named and compared only among themselves,
and these exponents are the comparative numbers of
precisely these named measures;
conversely, the comparative numbers of the independent measures
now taken singly are similarly the series of the
exponents for the members of the first series.
In this way, both sides are a series in which,
first, each number is simply a unit with respect to the
opposite series in which it has its specifically
determined being as a series of exponents;
second, each number is itself one of the exponents
for each member of the opposite series;
and, third, it is a comparative number for
the rest of the numbers of its series
and, as such an amount which belongs
to it also as an exponent,
it has its unit, determined for itself,
in the opposite series.

3. In this relation there is a return to the mode
in which quantum was posited as existing for itself,
namely as degree, as being simple
and yet having the magnitude of its determinateness
in a quantum that exists outside it,
one which is a circle of quanta.
In measure, however, this externality is not
merely a quantum and a circle of quanta,
but a series of numerical ratios,
and it is in the entirety of these that
the being-determined-for-itself of measure lies.
Just as it is the case in the being-for-itself of quantum as degree,
the nature of the independent measure has turned into this self-externality.
Its self-reference is at first an immediate relation,
and by that very fact its indifference to an other
lies only in the quantum.
Hence its qualitative side falls in this externality,
and its relating to the other becomes
that which constitutes the specific determination
of this independent measure.
Such a determination thus consists solely
in the quantitative mode of this relating,
and this mode is determined just as much by the other
as by the measure itself,
and this other is a series of quanta
while the measure, for its part,
is one such quantum.
But this connection in which two specific measures
specify themselves in a third thing (the exponent) further entails that,
in the exponent, the one measure has not gone over into the other;
that there is not, therefore, only one negation,
but that in the connection the two measures are
rather each negatively posited,
and since each preserves itself in it indifferently,
their negation is itself also negated.
This, their qualitative unity, is thus
an exclusive unit existing for itself.
It is only in this moment of exclusion
that the exponents, at first comparative numbers,
first gain in them the determinateness
truly specifying them reciprocally
and their difference thus becomes of qualitative nature.
But such a difference has a quantitative basis:
first, the independent measure relates to a plurality
of its qualitatively other side only because
in this relating it is at the same time indifferent;
second, this neutral connection is in virtue of
its quantitative basis not only alteration
but is now posited as a negation of the negation
and as exclusive unit.
Consequently, the affinity of an independent measure
with the plurality of measures of the other side is
no longer an indifferent connection but an elective affinity.

c. Elective affinity

The expression elective affinity used here
refers to a chemical relation,
as also do the preceding neutrality and affinity.
For in the chemical sphere a matter
receives its specific determinateness
essentially in the connection with its other;
it exists concretely only as this non-indifference.
Moreover, this specific connection is bound up with quantity
and is at the same time not only a connection with a single other,
but with a series of such non-indifferent matters
standing over and against it;
combinations with this series rest on a so-called affinity
with every member of the series,
even though in this indifference each member excludes every other.
We now have to consider this connection of opposite determinations.
But it is not just in the chemical sphere that specificity is
exhibited in a circle of combinations;
the meaning of a single note also depends on its relation to,
and combination with, another note and a series of other notes;
the harmony or disharmony in such a circle of combinations
constitutes its qualitative nature
which is at the same time based on quantitative ratios
that form a series of exponents,
and the ratios of the two specific ratios is
what each of the combined notes is within it.
The single note is the keynote of a system,
but at the same time also one member in
the system of every other note.
The harmonies are exclusive elective affinities
whose characteristic quality, however,
equally dissolves again in the externality of
a mere quantitative progression.
Where to find, however, the principle of a measure
for such affinities which among others and against others
(whether chemical or musical or whatever)
are elective affinities,
more will be said in the following Remark
in connection with chemical affinity;
but this is a question of a higher order
which is very closely bound up with
the specific nature of what is strictly qualitative
and belongs to particular parts of the concrete science of nature.

Inasmuch as the member of a series has its qualitative unity
in the way it relates to the whole of an opposite series,
the members of which differ from each other only by virtue of
the quantum required for being neutralized with that member,
the more specific determinateness in this manifold affinity
is likewise only quantitative.
In elective affinity which is an exclusive, qualitative connection,
the relating escapes this quantitative difference.
The next determination which offers itself is this:
that in accordance with the difference in the number,
hence the extensive magnitudes,
of the members of one side required
for the neutralization of a member in another side,
the elective affinity of this member would be directed
to the members of the first series with all of which it has affinity.
Thus transformed, the exclusion that would thereby be established
in the form of a firmer bonding against other possibilities of combination
would appear in a proportionally greater intensity,
in keeping with the previously demonstrated identity
of the forms of extensive and intensive magnitude,
inasmuch as the determinateness of magnitude is in both one and the same.
However, this sudden conversion of the one-sided form
of extensive magnitude also into its other,
the intensive form, changes nothing in so far as
the nature of the fundamental determination is concerned,
which is one and the same quantum.
In fact, therefore, no exclusiveness would thereby be posited,
but instead of one single bonding
a combination of any number of members
could just as well take place,
provided that the portions of such members
entering into the combination corresponded to
the required quantum proportionately
to the ratios between them.

But the combination which we have also called neutralization
is not only the form of intensity;
the exponent is essentially a determination of measure
and, as such, exclusive.
On this side of exclusive relating,
numbers have lost their continuity
and their tendency to combine;
their relating is one of more or less,
and this acquires a negative character;
and the advantage that one exponent has over another
does not remain confined to the determinateness of magnitude.
Also equally present is the other side
which makes it again a matter of indifference
to a moment of measure whether it receives its
neutralizing quantum from several opposite moments,
from each according to the determinateness
that specifies it as against an other;
the exclusive, negative relating is at the same time
vulnerable to this incursion of the quantitative side.
What is posited here is thus the sudden conversion of an indifferent,
merely quantitative relating into a qualitative one,
and conversely the transition of a specifically determinate being
into merely external relation,
a series of relations, sometimes of a merely quantitative nature,
sometimes specific relations and measures.

B. NODAL LINES OF MEASURE-RELATIONS

The last determination of the measure-relation was
that, as specific, it is exclusive;
exclusiveness accrues to neutrality
as a negative unity of the distinct moments.
For this unity existing for itself,
for this elective affinity,
no further principle has become available
specifying its connection with other neutralities;
the specification resides only in
the quantitative determination of affinity in general,
according to which it is specific amounts that neu-
tralize themselves
and consequently stand opposed to other relative elective
affinities of their moments.
But further, because the basic determination is quantitative,
the exclusive elective affinity continues also
into these opposite neutralities,
and this continuity is not just
an external comparative connection of the different ratios
of the neutralities but the neutrality,
as such, has an element of separability in it,
for it is as self-subsisting somethings
that the moments from whose union
it has come to be enter into connection
with one or the other of the opposite series:
they enter into the connection indifferently,
even though they combine in different specific determinate amounts.
This measure, internally based on such a relation,
is for this reason affected by an indifference of its own;
it is something which is external within
and alterable in its reference to itself.

This self-reference of the measure-relation differs
from the externality and alterability
which belong to its quantitative side.
In contrast to these, its self-reference is
an existent qualitative foundation,
a permanent, material substrate
which, since it is also the continuity of
the measure with itself in its externality,
would have to contain in its quality the said
principle of specification of this externality.

Now in more developed form, as external
to itself in its being-for-itself,
the exclusive measure repels itself from itself
and posits itself both as another merely quantitative relation
and as another relation which is as such
at the same time another measure;
it is determined as in itself a specifying unity
which produces measure-relations within itself.
These relations differ from affinities of
the kind mentioned above in which
an independent measure relates itself
to independent measures of another quality
and to a series of these.
They occur in one and the same substrate
within the same moments of the neutrality;
the self-repelling measure takes on the determination
of other only qualitatively diverse relations,
and these relations likewise form affinities and measures,
alternating with those that remain only quantitative diversities.
They form in this way a nodal line of measures
on a scale of more and less.

Here we have a measure-relation,
self-subsistent reality qualitatively distinguished from others.
Such a being-for-itself, since it essentially is
at the same time a relation of quanta,
is open to externality and quantitative alteration;
it has a margin within which it remains indifferent to this
alteration and does not alter its quality.
But there comes a point in the quantitative alteration
at which this quality alters and the quantum shows
itself to be specifying,
so that the altered quantitative relation is
suddenly turned into a measure
and thereby into a new quality,
a new something.
The relation that has now replaced the first
is determined by the latter,
both because of the qualitative sameness
of the moments which are in affinity
and because of the quantitative continuity.
But since the distinction falls on
the side of this quantitative moment,
the new something stands indifferently
related to the preceding one;
the difference between the two is
the external one of quantum.
The new something has not therefore
emerged out of the preceding one;
it has emerged rather immediately from itself, that is,
from the internal specifying unity
which has not yet entered into existence.
The new quality or the new something is
subjected to the same progression of alteration,
and so on, into infinity.

Inasmuch as the advance from a quality proceeds
in the steady continuity of quantity,
the ratios approaching the one qualifying point are distinguished,
quantitatively considered, by a more or less.
In this respect, the alteration is a gradual one.
But the gradualness concerns merely the
externality of the alteration, not its qualitative moment;
the preceding quantitative relation,
though infinitely near to the succeeding one,
is still another qualitative existence.
From the qualitative side, therefore,
the gradual, merely quantitative progression
which has no limits in itself,
is absolutely interrupted;
and since in its merely quantitative connection
the newly emerging quality is with respect to
the vanishing one an indeterminate other,
one which is indifferent to it,
the transition is a leap;
the two are posited as wholly external to each other.
It is a favorite practice to try to make an alteration
conceptually comprehensible by the gradualness of
the transition leading up to it;
but gradualness is rather alteration precisely
as merely indifferent, the opposite of a qualitative alteration.
Rather, in gradualness the connecting link between two realities
(be they states or self-subsistent things)
is sublated, what is posited is that
neither reality is the limit of the other,
but that each is absolutely external to the other.
Thus, the very point necessary for the conceptual comprehension
of the alteration is missed,
although little enough is required for that purpose.

C. THE MEASURELESS

The exclusive measure, even in its realized being-for-itself,
remains affected by the moment of quantitative existence
and is therefore susceptible to the movement up and down
the quantum scale of varying ratios.
Something, or a quality, based on such a ratio
is driven beyond itself into the measureless,
collapsing under the mere mutation of its magnitude.
Magnitude is that constitutional aspect
where an existence can be caught up
in an apparently harmless entanglement
and be destroyed by it.

The abstract measureless is the quantum in general
inasmuch as it lacks internal order
and is only indifferent determinateness
which does not alter measure.
In the nodal lines of measures this determinateness is
at the same time  posited as specifying;
the abstract measureless sublates itself
into qualitative determinateness;
the new measure-relation into which
the original passes over is measureless
with respect to the latter
but, in it, it equally is a quality that exists for itself;
what is thus posited is the alternation of specific existences
with one another and equally of them with relations
that still remain merely quantitative,
an alternation ad infinitum.
Therefore, present in this transition is
both the negation of the specific relations
and the negation of the quantitative progression;
this is the infinite existing for itself.
In the sphere of existence, the qualitative infinite
was the irruption of the infinite into the finite,
the immediate transition and vanishing of
the “this here” into its “beyond there.”
In contrast to it, the quantitative infinite is
in its very determinateness the continuity of quantum,
its continuing beyond itself.
The qualitatively finite becomes infinite;
the quantitatively finite is its beyond in it:
it points beyond itself.
But this infinite of the specification of measure posits
both the qualitative and the quantitative as
each sublating itself into the other,
and it thereby posits their first immediate unity,
which is measure in general, as returned into itself
and consequently as itself posited.
The transition of the qualitative,
of one specific concrete existence into another,
is such that what happens is only
an alteration of magnitude determinateness;
the alteration of the qualitative
as such into the qualitative is thus posited as
an external and indifferent alteration,
as a coming together with itself;
the quantitative, for its part,
sublates itself by suddenly turning into the qualitative,
that is, a being which is determined in-and-for-itself.
This unity which thus continues in itself
in its alternating measures is
the self-subsistent matter that truly persists,
the fact.

What we have here is

(a) one and the same substantial matter
which is posited as the perennial substrate
of its differentiations.
This detaching of being from its determinateness
already begins in quantum in general;
something is great indifferently as
against the determinateness it has.
In measure, the persisting matter is
itself already in itself the unity of
the qualitative and the quantitative,
the two moments into which the general
sphere of being is distinguished,
each as the beyond of the other;
in this way the perennial substrate
begins to possess in it the determination
of an existent infinity.

(b) This self-sameness of the substrate is posited
in that the qualitative independent measures
into which the determining unit is dispersed
consist of only quantitative differences,
so that the substrate persists
while being internally distinguished.

(c) In the infinite progression of the nodal series
there is posited the continuation of the qualitative
into the quantitative advance as into an indifferent alteration,
but equally too, there is posited the negation
of the qualitative contained in the progression and consequently,
at the same time, of the merely quantitative externality.
The quantitative pointing beyond itself to an other
which is itself quantitative perishes
with the emergence of a measure-relation,
of a quality, and the qualitative transition is sublated
in the very fact that the new quality is
itself only a quantitative relation.
This reciprocal transition into the other
of the qualitative and the quantitative moments
occurs on the basis of their unity,
and the meaning of this process is only
the existence which is the demonstration or the positing
that such a substrate does underlie the process
and is the unity of its moments.

In the series of indepedent measure relations
the one-sided members of the series are
immediate qualitative somethings
(specific gravities or chemical materials,
bases or alkalis, acids, for instance),
and their neutralizations
(by which we must understand here also the combinations
of materials of different specific gravities)
are in turn self-subsistent
and themselves exclusive measure-relations,
mutually indifferent totalities existing for themselves.
Such relations are now determined
only as nodes of one and the same substrate.
The measures and the self-subsistent forms posited
with them are consequently demoted to states.
Alteration is only the mutation of a state,
and that which passes over is posited
as remaining the same in the mutation.

In reviewing the progressive determination
which measure has gone through,
the moments of the progression can be summed up as follows.
Measure is at first itself
the immediate unity of quality and quantity
in the form of a common quantum
which is, however, specific.
As a determinateness of quantity
which does not refer to an other but refers to itself,
measure is thus essentially a ratio.
Consequently, it also holds its moments
as sublated and unseparated in itself;
as is always the case in a concept,
the difference in the ratio is such
that each of its moments is itself the unity of
the qualitative and the quantitative moment.
The difference is thus real,
and it yields a multitude of measure-relations
which, as formal totalities,
are each self-subsistent in themselves.
The series which form the two sides of these ratios
are for each single member the same constant ordering
by which the member, as belonging to the one side,
relates itself to the whole series standing opposite it.
This ordering, although a mere order,
a still wholly external unity,
demonstrates itself to be,
as the immanently specifying unity of a measure
which has a determinate being-for-itself,
distinguished from its specifications;
but this specifying principle is not yet the free concept,
which alone gives its differences an immanent determination;
the principle is at first, rather, only a substrate,
a matter, for whose differences, in order that they be totalities,
that is, that they should have in themselves the nature
of their stable, self-equal substrate,
there is only available the external quantitative determination
which shows itself at the same time to be a qualitative differentiation.
In this self-unity of the substrate
the determination of measure is a sublated determination,
its quality an external state determined by quantum.
This course is equally the progressive determination
of measure and its demotion to a moment.
