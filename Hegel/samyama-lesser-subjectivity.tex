The doctrine of the concept

The concept is the free [actuality],
as the substantial power that is for itself,
and it is the totality,
since each of the moments is the whole that it is,
and each is posited as an undivided unity with it.
So, in its identity with itself,
it is what is determinate in and for itself.

The way the concept proceeds is
no longer passing over or shining in an other.
It is instead development
since what are differentiated are
at the same time immediately posited as
identical with one another and with the whole,
each being the determinacy that it is
as a free being of the whole concept.

The doctrine of the concept is divided into the doctrine of

(1) the subjective or formal concept,
(2) the concept as determined to immediacy,
    or the objectivity,
(3) the idea,
    the subject-object,
    the unity of the concept and objectivity,
    the absolute truth.

Ordinary logic apprehends only matters in themselves
that surface here as a part of the third part of the whole
and, in addition, the so-called 'laws of thinking' (that surfaced earlier)
and, in applied logic, some from the sort of knowing bound up with material
that is still psychological, metaphysical, and otherwise empirical,
since those forms of thinking in the end no longer sufficed for it.
Nonetheless, this science thereby lost any solid orientation.
Moreover, those forms that pertain at least to the genuine domain of logic
are taken merely as determinations of conscious thinking
and, indeed, conscious thinking at the level
merely  of the understanding, not of reason.

The preceding logical determinations,
the determinations of being and essence,
are not mere determinations of thought, to be sure.
In their process of passing over (the dialectical moment),
and in their return into themselves and in their totality,
they have proven themselves to be concepts.
But they are merely determinate concepts,
concepts in themselves or, what is the same,
concepts for us since the other
(into which each determination passes over
or in which it shines and is accordingly something relative)
is determined not as something particular.
Nor is the third factor determined as
something individual or as a subject,
which is to say that the identity of the determination is
not posited in the determination opposite it,
that its freedom is not posited,
since it is not universality.
What is usually understood by 'concepts' are
determinations of understanding,
even merely universal representations,
hence, in general, finite determinations.

The logic of the concept is
usually understood as a merely formal science,
revolving around the form as such of
the concept, the judgment, and the syllogism,
but not at all around whether something is true;
this depends, to the contrary, completely on the content alone.
Were the logical forms of the concept actually dead, ineffective,
and  indifferent receptacles of representations or thoughts,
then familiarity with them would be a historical record that
is quite superfluous and dispensable for the truth.
In fact, however, as forms of the concept, they are, to the contrary,
the living spirit of the actual, and what is true of the actual is true
only by virtue of these forms, through them, and in them.
However, the truth of these forms for themselves,
let alone their necessary connection,
has never been considered and investigated until now.

A. THE SUBJECTIVE CONCEPT

a. The concept as such

The concept as such contains the moments of

universality
(as the free sameness with itself in its determinacy),
particularity
(as the determinacy in which
the universal remains the same as itself, unalloyed),
and individuality
(as the reflection-in-itself of
the determinacies of universality and particularity,
the negative unity with itself
that is the determinate in and for itself
and at the same time identical with itself or universal).

The individual is the same as the actual,
with the difference that the former has
gone forth from the concept
and is accordingly posited as universal,
as the negative identity with itself.
Because it is first only in itself
or immediately the unity of
the essence and concrete existence,
the actual can be productive.
But the individuality of the concept is
simply what produces and, indeed,
no longer as the cause with
the semblance of producing an other,
but as what produces its very self.
The individuality, however, is
not to be taken in the sense of
only immediate individuality in terms of which
we speak of individual things, human beings.
This determinate sense of individuality
surfaces first in the case of judgment.
While each moment of the concept is
itself the entire concept,
individuality, the subject, is
the concept posited as the totality.

The concept is what is utterly concrete
since the negative unity with itself
(as being-determined-in-and-for-itself
which is the individuality)
itself makes up its relation to itself, the universality.
To this extent, the moments of the concept
cannot be detached from one another;
the determinations of reflection are supposed
to be grasped and to be valid each for itself,
detached from the opposed determination.
Since, however, their identity is posited in the concept,
each of its moments can be immediately grasped only
on the basis of and with the others.

Taken in an abstract sense,
universality, particularity, and individuality
are the same as identity, difference, and ground.
But the universal is what is identical with itself
explicitly in the sense that at the same time
the particular and the individual are contained in it.
Furthermore, the particular is what has been
differentiated or the determinacy,
but in the sense that it is
universal in itself and as an individual.
Similarly, the individual has
the meaning of being the subject,
the foundation which contains
the genus and species in itself
and is itself substantial.
This is the posited inseparability of
the moments in their difference,
the clarity of the concept in which
no difference interrupts or obscures the concept,
but in which each difference is instead equally transparent.

There is nothing said more commonly
than that the concept is something abstract.
This is correct in part insofar as
its element is thinking generally
and not the empirically concrete sphere of the senses,
in part insofar as it is not yet the idea.
In this respect, the subjective concept is still formal,
yet not at all as if it should respectively have or acquire
some other content than itself.
As the absolute form itself,
the concept is every determinacy,
but as it is in its truth.
Thus, although the concept is
at the same time abstract,
it is what is concrete
and, indeed, the absolutely concrete,
the subject as such.
The absolutely concrete is the spirit,
the concept insofar as it concretely exists as concept,
differentiating itself from its objectivity
which, despite the differentiating,
remains the concept's own objectivity.
Everything else concrete, as rich as it may be,
is not so inwardly identical with itself
and, for that reason, in itself not as concrete,
least of all what one commonly understands
by the concrete, a manifold externally held together.
What are also called concepts
and, to be sure, determinate concepts,
e.g. human being, house, animal, and so forth, are
simple determinations and abstract representations,
abstractions that, taking only the factor of universality
from the concept while omitting
the particularity and individuality,
are thus not developed in themselves
and accordingly abstract precisely from the concept.

The moment of individuality first posits
the moments of the concept as differences,
since it is the concept's negative reflection-in-itself.
Thus it is initially the free differentiating of
the concept as the first negation,
by means of which the determinacy of the concept is posited,
but posited as particularity.
That is to say, first, that the moments differentiated have
the determinacy of conceptual moments only opposite one another
and, second, that their identity
(that the one is the other)
is equally posited.
This posited particularity of
the concept is the judgment.

The usual species of clear, distinct, and adequate concepts pertain,
not to the concept, but to psychology insofar as,
by 'clear and distinct concepts', representations are meant,
where 'clear' means an abstract, simply determinate representation
and 'distinct' the sort of representation in which a distinguishing mark,
i.e. some sort of determinacy has been singled out
as a sign for subjective knowing.
Nothing is so much the distinguishing mark of the externality and
decay of logic than the cherished category of the distinguishing mark.
The adequate concept is more of a play on the concept,
indeed even the idea, but still expresses nothing
but the formal aspect of the agreement of a concept
or even a representation with its object, some external thing.
Underlying the so-called subordinate and coordinate concepts is
[a] the concept-less difference between the universal and the particular
as well as [b] their relatedness in an external reflection.
An enumeration of species of contrary and contradictory,
affirmative, negative concepts and so forth is, moreover,
nothing other than a process of arbitrarily reading off
determinacies of thought that for their part belong to
the sphere of being or essence,
where they have already been considered,
and that have nothing to do
with the determinacy of the concept itself as such.
The genuine differences of the concept,
the universal, particular, and individual,
constitute species of the concept,
if at all only insofar as they are held apart
from one another by external reflection.
The immanent differentiating and determining of
the concept is on hand in the judgment,
since the judging is the determining of the concept.

b. The judgment

The judgment is the concept in its particularity as
the differentiating relation of its moments,
which are posited as being for themselves
and, at the same time, as
identical with themselves,
not with one another.

In the case of a judgment one usually thinks first of
the self-sufficiency of the extremes, subject and predicate,
such that the subject is a thing or a determination for itself
and the predicate, too, is a universal determination
outside that subject, in my head somehow.
I then bring the predicate together with the subject
and, by this means, I judge.
However, since the copula 'is' asserts the predicate of the subject,
that external, subjective subsuming is sublated in turn
and the judgment is taken as a determination of the object itself.
The etymological meaning of 'judgment' in our language is
profounder and expresses the unity of the concept as what comes first
and its differentiation as the original division
that the judgment truly is.

The abstract judgment is the sentence:
'the individual is the universal'.
These are the determinations
that the subject and predicate
first have opposite one another,
in that the moments of the concept are taken
in their immediate determinacy or first abstraction.
(The sentences 'the particular is the universal' and
'the individual is the particular' belong to
the further determination of the judgment.)
It has to be viewed as an amazing lack of attentiveness that
in the logic books there is nowhere to be found [acknowledgment of]
the fact that in each judgment one is articulating a sentence such as
'the individual is the universal'
or, even more determinately,
'the subject is the predicate'
(e.g. 'God is absolute spirit').
To be sure, the determinations, individuality and universality,
subject and predicate, are also distinct,
but on that account, nonetheless,
the completely universal fact remains that
each judgment asserts them as identical.

The copula 'is' comes from the concept's nature, namely,
to be identical with itself in its externalization.
The individual and the universal, as its moments,
are the sort of determinacies that cannot be isolated.
The earlier determinacies of reflection, in their relationships,
are equally related to one another,
but their connection is only that of having, not being,
the identity posited as such or the universality.
For this very reason, the judgment is
the true particularity of the concept,
since it is the determinacy or differentiation of the same,
a differentiation that, however, remains the universality.

Judgment is usually taken in the subjective sense
as an operation and form that surfaces
merely in self-conscious thinking.
This difference, however, is not yet on hand
in the logical [sphere, where]
judgment is supposed to be taken
in the completely universal sense:
all things are a judgment;
they are individuals which are a universality
or inner nature in themselves,
or a universal that is individuated.
The universality and individuality
distinguish themselves in them [the things]
but are at the same time identical.

The sense of the judgment that is supposed to be merely subjective,
as if it were I who attributes a predicate to a subject,
is contradicted by the objective expression of the judgment:
'the rose is red', 'gold is metal', and so forth;
I do not first attribute something to them.
Judgments are different from sentences;
the latter contain the determination of the subjects
that does not stand in a connection of universality with them,
a condition, an individual action, and the like;
'Caesar was born in Rome in such and such a year,
conducted the war in Gaul for ten years, crossed the Rubicon,
and so forth' are sentences, not judgments.
There is, furthermore, something quite empty
in saying that sentences of the sort, e.g.
'I slept well last night' or even 'Present arms!'
can be put into the form of judgments.
A sentence like 'a carriage is passing by' would be a judgment
and, to be sure, a subjective one only if it could be doubted
whether what is moving by is a carriage,
whether it is the object that is moving and not
the standpoint from which we are observing it;
where what then matters is finding the determination
for a representation not properly determined yet.

The standpoint of the judgment is finitude, and from this standpoint the
finitude of things consists in the fact that they are a judgment,
that their existence and their universal nature (their body and their soul) are
certainly unified (otherwise the things would be nothing), but that these,
their moments, are both already diverse and generally able to be separated.

In the abstract judgment 'the individual is the universal',
the subject relates itself negatively to itself and, as such,
is the immediately concrete, while the predicate is, by contrast,
the abstract, indeterminate, the universal.
But since they are joined by 'is', the predicate in its universality
must also contain the determinacy of the subject
and it [that determinacy] is the particularity
and the latter is the posited identity of the subject and predicate.
As thus indifferent to this difference of form, it is the content.

Only in the predicate does the subject
have its explicit determinacy and content;
hence, taken by itself it is a mere representation or a bare name.
In the judgments 'God is the supremely real' and so forth
or 'the absolute is identical with itself' and so forth,
'God' and 'absolute' are mere names.
What the subject is, is first said in the predicate.
What it might otherwise also be as something concrete
does not matter to this judgment.

As far as the more precise determinacy of subject and predicate is concerned,
the former, as the negative relation to itself, is the
underlying fixity in which the predicate has its subsistence
and is in an ideal way (it inheres in the subject).
Moreover, since the subject is generally and immediately concrete,
the determinate content of the predicate is only one of the many
determinacies of the subject and the latter is richer and broader than the predicate.

Conversely, the predicate, as the universal subsisting for itself and
indifferent to whether this subject is or not, goes beyond the subject,
subsumes the subject under it, and is, for its part, broader than the subject.
The determinate content of the predicate (see preceding section)
alone makes up the identity of both.

Subject, predicate, and the determinate content or the identity [of them]
are initially posited in the judgment, in their relation,
as themselves diverse, falling outside one another.
But in themselves, in terms of the concept, they are identical,
since the concrete totality of the subject is this,
not to be some sort of indeterminate manifold,
but instead individuality alone,
the particular and universal in an identity,
and precisely this unity is the predicate.
In the copula, furthermore, the identity of
the subject and predicate is of course posited
but initially only as the abstract 'is'.
In keeping with this identity, the subject is also
to be posited in the determination of the predicate,
by means of which the latter also acquires
the determination of the subject
and the copula is fulfilled.
This is the further determination of the judgment,
by means of the copula full of content, into the syllogism.
But first, in terms of the judgment,
there is the further determination of it,
the determining of the initially abstract,
sensory universality into a set of all, genus, and species
and into the developed universality of the concept.

Only knowledge of the further development of the judgment
gives a context as well as a sense to what are customarily
put forward as species of judgment.
In addition to appearing completely contingent,
the usual enumeration is superficial and
even barren and wild in the presentation of the differences.
In part, the manner in which positive, categorical, and assertoric
judgments are differentiated is generally pulled out of the air
and in part it remains undetermined.
The various judgments should be considered as
following necessarily from one another
and as a further determining of the concept,
since the judgment is nothing other
than the determinate concept.

In relation to the two previous spheres of being and essence,
the determinate concepts, qua judgments, are reproductions of
these spheres, but posited in the simple relation of the concept.

(a) Qualitative judgment

The immediate judgment is the judgment of existence:
the subject posited in a universality, as its predicate,
which is an immediate (thus sensory) quality.
(1) Positive judgment: the individual is a particular.
But the individual is not a particular;
more precisely, such an individual quality
does not correspond to the concrete nature of the subject;
(2) negative judgment.

It is one of the most essential logical prejudices
that such qualitative judgments as 'the rose is red'
or 'the rose is not red' can contain truth.
They can be correct, i.e. in the limited sphere of
perception, finite representing, and thinking.
This depends upon the content,
which is just as much a finite content,
untrue for itself.
But the truth rests solely on the form, i.e.
the posited concept
and the reality corresponding to it;
but such truth is not at hand in the qualitative judgment.

In this as first negation there still remains
the relation of the subject to the predicate,
which is thereby something relatively universal,
the determinacy of which has only been negated
('the rose is not red' entails that it still has colour;
immediately another [colour] which, however,
would only be a positive judgment in turn).
The individual, however, is also not a universal.

(3) Hence,
(aa) the judgment collapses in itself
into the empty identical relation:
the individual is the individual - identical judgment;
and (bb) it collapses into itself as the present,
complete inadequacy of the subject and predicate:
a so-called infinite judgment.

Examples of the latter are
'the spirit is no elephant',
'a lion is no table',
and so forth,
sentences that are correct
but as nonsensical as the identical sentences
'a lion is a lion',
'the spirit is spirit'.
These sentences are, to be sure,
the truth of the immediate,
so-called qualitative judgment,
but not judgments at all,
and they can only surface in a subjective thinking
that can fix upon an untrue abstraction.
Objectively considered, they express
the nature of beings
or sensory things, namely,
that they collapse into an empty identity
and into a fulfilled relation that is, however,
the qualitative otherness of what is related,
their complete inadequacy.

(b) The judgment of reflection

The individual, posited as individual
(reflected in itself) in the judgment,
has a predicate, opposite which the subject,
relating itself to itself,
remains at the same time an other.
In the concrete existence the subject is
no longer immediately qualitative,
but is instead in a connection with
and joined to an other, an external world.
The universality has acquired hereby
the meaning of this relativity.
(For example, useful, dangerous;
weight, acidity, then drive, and so forth.)

(1) The subject, the individual as individual
(in the singular judgment), is a universal.
(2) In this relation it is elevated above its singularity.
This expansion is an external one, the subjective reflection,
at first the indeterminate particularity
(in the particular judgment
which is, immediately, negative as well as positive;
the individual is in itself divided,
it relates itself in part to itself,
in part to another).
(3) Some are the universal,
so the particularity is expanded to universality;
or this universality, determined by the individuality of the subject,
is the set of all (commonality, the usual universality-of-reflection).

By the fact that the subject is likewise determined as universal,
the identity of it and the predicate is posited as indifferent,
as is, thanks to this, the determination of the judgment itself.
This unity of the content as the universal identical with
the subject's negative reflection-in-itself makes
the relation of the judgment a necessary relation.

(c) Judgment of necessity

The judgment of necessity as the identity of the content in its difference

(1) contains within the predicate in part the substance or nature of the
subject, the concrete universal - the genus; in part, since this universal
equally contains in itself the determinacy as negative, the excluding essential
determinacy - the species; - categorical judgment.

(2) In keeping with their substantiality, the two sides acquire the form
of self-sufficient actuality, the identity of which is only an inner identity,
and with that the actuality of the one is at the same time not its actuality,
but instead the being of the other; ~ hypothetical judgment.

(3) At the same time, in this externalization of the concept, the inner
identity is posited and so the universal is the genus that is identical with
itself in its excluding individuality. The judgment which has this universal
on both sides of it, the one time as such, the other time as the sphere of
its self-excluding particularization - the either/or of which just as much
as the as well as is the genus - is the disjunctive judgment. With this, the
universality at first as genus and then also as the scope of its species is
determined and posited as a totality.

(d) The judgment of the concept

The judgment of the concept has the concept,
the totality in simple form, for its content,
the universal with its complete determinacy.
The subject is (1) initially an individual
that has, as its predicate, the reflection of
the particular existence on its universal,
the agreement or lack of agreement
of these two determinations:
good, true, correct, and so forth:
assertoric judgment.

In ordinary life, too, one only calls it judging
when a judgment is of this sort, e.g.
the judgment whether an object, action, and so forth is
good or bad, true, beautiful, and so forth.
One will not ascribe a power of judgment to someone [simply]
for knowing, for example, how to make positive or negative judgments
such as 'this rose is red', 'this painting is red, green, dusty',
and so forth.

Even in philosophy, through the principle of
immediate knowing and believing,
the assertoric judgment has been made into
the sole and essential form of the doctrine
(despite the fact that in society the
assertoric judgment counts as improper,
when someone claims that it is supposed
to be valid by itself).
In the so-called philosophical works
that maintain that principle,
one can read hundreds upon hundreds of
assurances about reason, knowing, thinking,
and so forth, which seek to gain credence for themselves
through endless repetitions of one and the same point,
since external authority no longer counts for much.

In what is at first the immediate subject
of the assertoric judgment,
this judgment does not contain that relation
of the particular and the universal.
that is expressed in the predicate.
This judgment is thus merely a subjective particularity
and the opposite assurance stands over against it
with the same right or, rather, the same lack of right.
It is thus (2) at the same time only a problematic judgment.
But (3) [insofar as] the objective particularity
is posited in the subject,
its particularity as the constitution of its existence,
the subject then expresses the relation of
that particularity to its constitution,
i.e. to its genus and, with this,
expresses what (see preceding section)
makes up the content of the predicate
(this, the immediate individuality, house, genus,
so and so constituted, particularity, is good or bad):
apodictic judgment.
All things are a genus (their determination and purpose)
in one individual actuality with a particular constitution;
and their [i.e. all things'] finitude consists in the fact that
their particular [character] may or may not be adequate to the universal.

In this way, subject and predicate are
each themselves the entire judgment.
The immediate constitution of the subject
shows itself at first as the mediating ground
between the individuality of the actual and its universality,
as the ground of the judgment.
What has in fact been posited is
the unity of the subject and the predicate,
as the concept itself;
it is the fulfilment of the empty 'is', the copula,
and since its moments are at the same time
differentiated as subject and predicate,
it is posited as their unity,
as the relation mediating them:
the syllogism.

c. The syllogism

The syllogism is the unity of the concept and the judgment;
it is the concept as the simple identity
(into which the judgment's differences of form have gone back),
and [it is] judgment insofar as it is posited
at the same time in reality,
namely, in the difference of its determinations.
The syllogism is what is rational
and everything rational.

The syllogism tends to be put forward usually
as the form of the rational,
but as a subjective form and without pointing up
any sort of connection between it
and any other rational content, e.g.
a rational grounding principle, a rational action, idea, and so forth.
In general, there is much and frequent talk of reason
and appeal is made to it without indicating what it is,
what its determinacy is and without giving the slightest thought
to what inferring via syllogism is.
In fact, formally inferring via syllogism is
the rational in such a non-rational manner,
that it has nothing to do with a rational basic content.
Since, however, such a content can be rational
only through the determinacy through which thinking is reason,
it can be rational only through the form which the syllogism is.
This, however, is nothing else than
the posited, (at first formally) real concept,
as this section expresses.
The syllogism is, on account of this,
the essential ground of everything true;
and the definition of the absolute is from now on
that it is the syllogistic inference,
or, articulated in the form of a sentence,
it is this determinacy:
'everything is a syllogism'.
Everything is a concept,
and its existence is the difference of its moments,
so that its universal nature provides itself
with external reality through particularity
and, by this means and as negative reflection-in-itself,
makes itself something individual.
Or conversely, the actual is an individual
that by means of particularity
elevates itself into universality
and makes itself identical with itself.
The actual is one, but [it is] similarly
the segregation of the moments of the concept,
and the syllogism is the cyclical course
taken by the mediation of its moments,
a course through which it posits itself as one.

The immediate syllogism is such that
the determinations of the concept stand opposite one another
in an external connection as abstract determinations,
so that the two extremes [are] the individuality and universality,
but the concept, as the middle joining the two together,
is likewise only the abstract particularity.
The extremes are accordingly posited
as subsisting for themselves,
as indifferent to one another as
they are to the middle [term that joins them].
This syllogism is thus rational but non-conceptual;
it is the formal syllogism of the understanding.
In it the subject is joined together with another determinacy;
or through this mediation the universal subsumes a subject external to it.
In a rational syllogism, by contrast, the subject joins itself
together with itself by means of this mediation.
It is only a subject in this way,
or the subject is only in itself
the syllogism of reason.

In the following consideration,
the syllogism of the understanding is expressed
in terms of its ordinary, usual meaning, [namely,]
in the subjective manner attributed to it
in the sense that we make such syllogistic inferences.
In fact it is only a subjective inferring via syllogism,
though this has equally the objective meaning
that it expresses only the finitude of things,
but in the determinate manner that the form has attained here.
With respect to finite things, subjectivity as thinghood,
separable from its properties, its particularity, is
equally separable from its universality insofar as this is
the mere quality of the thing
and its external connection with other things
as its genus and concept.

(a) Qualitative syllogism

The first syllogism is the syllogism of existence
or the qualitative syllogism,
as it was portrayed in the previous section,
(1) I - P - U
[individuality, particularity, universality]
that a subject as individual is joined together, through a quality,
with some universal determinacy.

That the subject (terminus minor) has even further determinations than
that of individuality, similarly that the other extreme
(the predicate of the conclusion, the terminus maior) is
further determined than being merely a universal,
does not come into consideration here;
only the forms through which they constitute the syllogism [come into consideration].

This syllogism is (a) completely contingent with respect to its determi-
nations since the middle, as an abstract particularity, is merely any sort of
determinacy of the subject, of which, as something immediate and thus
empirically concrete, it has several. Hence, it can be joined together just as
much with many sorts of other universalities, just as an individual particu-
larity in turn can also have several diverse determinacies in itsel£ Thus, the
subject can be related to different universals by means of the same medius
terminus [middle term].

It is more that formally inferring has gone out of fashion than that
its incorrectness has been detected and that the lack of its use has
been justified on that basis [i.e. its incorrectness). This and the
following section indicate the vacuousness of such
inferring for the truth.

By means of such syllogisms (according to the side indicated in
the section), the most diverse sorts of things can be proved, as it is
said. The only thing required is to take up the medius terminus from
which the transition to the desired determination can be made. Yet,
with a different medius terminus, something else, even something
opposite, may be proven. - The more concrete an object is, the more
sides it has that inhere in it and can serve as medii termini. Which of
these sides is more essential than the other must depend again on the
sort of inferring that fixes upon the individual determinacy and can
likewise easily find for it a side and a respect in terms of which it can
be rendered important and necessarily valid.

(b) This syllogism is equally contingent on account of
the form of the relation in it.
According to the concept of the syllogism,
the true is the relation of differentiated entities,
through a middle that is their unity.
The relations of the extremes to the middle
(the so-called premises, the major and the minor)
are, however, immediate relations.

This contradiction of the syllogism expresses itself again through an
infinite progression as the demand that each of the premises
likewise be proven by means of a syllogism; since this syllogism,
however, has two immediate premises of the same sort, this demand
then repeats itself and, indeed, as a demand constantly doubling
itself, ad infinitum.

What here (on account of the empirical importance) has been noted as
a deficiency of the syllogism, to which in this form absolute correctness
is ascribed, must of itself sublate itself in the
further determination of the syllogism. Here, within the sphere of the
concept as well as in the judgment, the opposite determinacy is not merely
in itself on hand, but instead it is posited, and hence, for the further
determination of the syllogism, it is only necessary to take up what is
posited each time by it itself.

By means of the immediate inference (I - P - U), the individual is
mediated with the universal and posited as universal in this conclusion.
By this means, the individual as subject, thus itself as universal,
is now the unity of the two extremes and the mediating factor, which results
in the second figure of the syllogism ((2) U -I -P). This expresses the truth of
the first figure, [namely] that the mediation took place in the individuality
and accordingly is something contingent.

The second figure joins the universal with the particular (i.e. the universal
that emerges from the previous conclusion is determined by the individ-
uality, and accordingly occupies the position of the immediate subject).
As a result, the universal is posited as particular, via the conclusion [of the
second figure], thus as the factor mediating the extremes, the positions of
which are now taken by the others in the third figure of the syllogism
((3) P-U-I).

The so-called figures of the syllogism
(Aristotle rightly acknowledges only three of them;
the fourth is a superficial, indeed fatuous, addition of the moderns)
are placed next to one another in the standard treatment of them,
without the slightest thought being given to showing their necessity,
even less their meaning and their value.
For this reason it is no wonder, if the figures have later been
treated as an empty formalism. They have, however, a very basic
[methodical] sense that rests upon the necessity that each moment as
a determination of the concept becomes itself the whole and the
mediating ground. - What determinations the sentences otherwise
have, whether they may be universal and so forth or negative,
in order to bring about a correct inference, this is a mechanical
investigation that has rightly come to be forgotten on account of
its concept-less mechanism and its lack of inner meaning. - One
can appeal least of all to Aristotle for the importance of such an
investigation and of syllogism at the level of understanding. To be
sure, he described these like countless other forms of the spirit and
nature, and he both investigated and presented their determinacy.
But in his metaphysical concepts as well as in the concepts of the
natural and the spiritual, he was far from intending to make the
form of the syllogism at the level of the understanding a foundation
and criterion, so far that probably not a single one of these concepts
would have been able to arise or be left standing if it were supposed
to be subjected to the laws of understanding. In the considerable
amount of descriptive and sensible detail that
Aristotle, after his fashion, brings together, the speculative concept is
invariably what dominates for him and he does not allow that
inferring at the level of mere understanding, the inferring that he
first outlined in so determinate a fashion, to enter into this sphere.

Since each moment has run through the position of the middle and the
extremes, their determinate difference relative to one another has sublated
itself and, in this form where there is no difference between its moments, the
syllogism first has the external identity of the understanding, the equality,
as its relation the quantitative or mathematical syllogism. If
two things are equal to a third, then they are equal to one another.

By this means, it has come about with respect to the form (1) that each
moment received the determination and position of the middle, hence
the whole in general, and with this it has lost the one-sidedness of its
abstraction in itself, (2) that the mediation (§ 185) has been
completed, also only in itself, namely, only as a circle of mediations that
mutually presuppose one another. In the first figure (I P U), the two
premises, I - P and P - U, are still unmediated; the former being mediated
in the third, the latter in the second figure. But each of these two figures
equally presupposes the two other figures to mediate their premises.

In keeping with this, the mediating unity of the concept is no longer
to be posited only as an abstract particularity, but instead as the developed
unity of individuality and universality and, indeed, at first as the reflected
unity of these determinations, the individuality determined at the same time
as universality. This sort of middle yields the syllogism of reflection.

(b) Syllogism of reflection

The middle is in the first place (1) not alone the abstract, particular deter-
minacy of the subject, but instead at the same time as all individual concrete
subjects, to which that determinacy as only one among others accrues. As
such, the middle yields the syllogism of the set of all.
The major premise (the subject of which is the particular determinacy, the
terminus medius, as the set of all) presupposes the conclusion, of which it
is supposed to be the presupposition. It thus rests upon (2) induction, the
middle of which is the complete set of the individuals as such, a,
b, c, d, and so forth. Since, however, the immediate empirical individuality
is different from the universality and, for that reason, cannot ensure any
completeness, the induction rests upon (3) analogy, the middle of which
is an individual but in the sense of its essential universality, its genus or
essential determinacy. - The first syllogism refers, for its mediation, to
the second and the second to the third; but the latter equally demands a
universality or the individuality as genus after the forms of the external
relation of individuality and universality have been run through in the
figures of the syllogism of reflection.

By means of the syllogism of the set of all, some improvement is made
relative to the deficiency of the basic form of the inference at the level
of the understanding (pointed out in § 184). But the improvement is
only such that a new deficiency arises, namely, that the major premise
presupposes as an accordingly immediate sentence what was supposed to
be the conclusion. - 'All human beings are mortal, therefore Gajus is mortal',
'all metals are electric conductors, therefore, for example, copper is, too'. In
order to be able to assert those major premises that are supposed to express
the set of all of the immediate individuals and to be essentially empirical
sentences, it is required that already previously the sentences about the
individual Gajus, the individual copper are confirmed for themselves as
correct. - Everyone rightly notices not merely the pedantry, but the vapid
formalism of such syllogisms as 'all humans are mortal, but
now Gajus is human, and so forth'.

(c) Syllogism of necessity

As far as the merely abstract determinations of this syllogism are concerned,
it has the universal as the middle term, just as the syllogism of reflection
has the individuality as the middle term,
the latter in terms of the second figure, the former in terms of the third;
the universal posited as essentially determined in itself.
(1) At first, the particular in the sense of the determinate genus or species is
the mediating determination - in the categorical syllogism;
(2) the individual in the sense of immediate being that is
both mediated and mediating in the hypothetical syllogism;
(3) the mediating universal is also posited as the totality of
its particularizations and as an individual particular,
 an exclusive individuality - in the disjunctive syllogism;
so that one and the same universal is in these determinations as
merely in forms of difference.

The syllogism has been taken in terms of the differences contained in it and
the universal result of the course of those differences is that
these differences and the concept's manner of being-outside-itself
sublate themselves in it.
Indeed, (1) each of the moments themselves has demonstrated itself to be
the totality of the moments and, hence, to be the entire syllogism;
thus, they are in themselves identical
(2) The negation of their differences and their mediation constitutes the
manner of being-for-itself, such that it is one and the same universal
that is in these forms and is accordingly also posited as their identity.
In this ideality of the moments, inferring acquires the determination
of essentially containing the negation of the determinacies by means of
which it runs its course and, with this, the determination of being a
mediation by way of sublating the mediation and a manner of joining
the subject together, not with another, but with the sublated other,
with itself.

This realization of the concept, in which
the universal is this singular totality that has returned into itself
and whose differences are equally this totality,
which has determined itself to be an immediate unity by sublating mediation,
this realization of the concept is - the object.

At first glance, this transition from the subject,
from the concept in general, and, more precisely,
from the syllogism into the object,
may seem strange, especially if one
regards the syllogism only at the level of the understanding and
regards inferring only as an act of consciousness.
At the same time, it cannot be our task to want
to make this transition plausible to representation.
It is only possible to recall whether our customary representation
of what is called an 'object' corresponds roughly to
what constitutes the determination of the object here.
By 'object', however, one tends to understand not merely
an abstract entity or concretely existing thing or
something in general actual, but instead something
concretely and completely self-sufficient in itself;
this completeness is the totality of the concept.
That the object is also something standing
opposite and external to another,
this will be determined subsequently insofar as
it posits itself in opposition to the subjective.
Here, as that into which the concept has passed over from its mediation,
it is at first only an immediate, neutral object,
just as the concept is determined to be the subjective
only in the subsequent opposition.

Furthermore, the object in general is also the one whole,
in itself as yet indeterminate, the objective world in general,
God, the absolute object.
But the object equally has difference within it, breaking
down in itself into an indeterminate manifold (as objective world)
and each of these individuated entities is also an object, an existence,
in itself concrete, complete, self-sufficient.

As objectivity has been compared with being. concrete existence
and actuality, so too the transition to concrete existence
and actuality (since being is the first, completely abstract
immediacy) is to be compared with the transition to objectivity.
The ground from which concrete existence emerges, the relationship
of reflection that sublates [and elevates] itself to actuality ,
these are nothing other than the concept,
posited in a still imperfect way. Or they are only abstract sides of it,
the ground being the unity of it merely in the form of essence ,
the relationship merely the relation of sides that are
supposed to be real, reflected only in themselves. The concept is the
unity of both and the object is not only the unity befitting an
essence but the unity in itself universal, containing not only real
differences but these differences as totalities in itself

It is clear, moreover, that in all these transitions, it is a matter
of more than merely showing the inseparability of the concept or
thinking from being. It has been frequently noted that being is
nothing more than the simple relation to itself and that this
impoverished determination is contained, without further ado, in
the concept or even in thinking. The point of these transitions is not
to take up determinations only insofar as they are contained [in the
concept] (as happens even in the ontological argument for the
existence of God based upon the principle that being is one of the
realities). The point is instead to take the concept as it is prima facie
supposed to be determined for itself as concept, with which this
distant abstraction of being or even objectivity still has nothing to
do, and to see whether, in its determinacy solely as the determinacy
of the concept, it passes over into a form which differs from the
determinacy inherent in the concept and appears within it.

If the product of this transition, the object, is placed in relation to
the concept that has disappeared in the transition in its peculiar
form, then the result can be correctly expressed in such a way that in
itself concept, or also, if one prefers, subjectivity, and the object are
the same. However, it is equally correct that they are diverse. Since one
is as correct as the other, then by the same token one is as incorrect
as the other. Such a manner of expression is incapable of presenting
their true relationship_ The expression in itse/fhere is an abstractum
and even more one-sided than the concept itself, the one-sidedness
of which in general sublates itself in that it elevates
itself to the object, the opposite one-sidedness.
Thus, that in itself must also determine itself to being-for-itself through the
negation of itself.
As is everywhere the case, the speculative identity is
not that trivial identity that concept and object are in themselves
dentical- an observation that has been repeated often enough, but
could not be repeated enough, if the aim is to put an end to the
shallow and completely malicious misunderstanding of this identity.
Understandably, however, this is not something that can be hoped for.

Moreover, if that unity is taken quite generally,
without recalling the one-sided form of its being-in-itself,
then it is, as is familiar to many,
what is presupposed in the ontological proof of God's existence
and, indeed, presupposed as what is most perfect.
The utterly remarkable thought of this proof
first occurs to Anselm who, to be sure,
begins by discussing merely whether a content is
only in our thinking.
These, in brief, are his words:
'Certeid, quo maius cogitari nequit,
non potest esse in intellectu solo. Si enim vel in solo
intellectu est, potest cogitari esse et in re: quod maius est. 5i ergo id,
quo maius cogitari non potest, est in solo intellectll; id ipsum, quo
maius cogitari non potest, est, quo maius cogitari potest. Sed certe
hoc esse non potest.'
[Certainly that, of which nothing greater can be thought,
cannot be in the intellect alone.
For if it is in the intellect alone,
it can be thought to be in some thing as well:
which is greater.
If, therefore, that of which a greater cannot be thought is
in the intellect alone, then it is possible to think something greater
than that of which nothing greater can be thought. But this certainly
cannot be the case.]
In terms of the determinations we are
standing among here, finite things are such that their objectivity is
not in agreement with the thought of them, i.e. their universal
determination, their genus, and their purpose. Descartes and
Spinoza, among others, have articulated this unity more objectively,
whereas the principle of immediate certainty or belief takes them
more in the subjective manner of Anselm, namely, that the
determination of God's being is inseparably bound up with the
representation of God in our consciousness. If the principle of this
belief also takes up the representations of external, finite
things into the inseparability of the consciousness of them and
their being, because in the intuition they are bound up with the
determination of concrete existence, then this is indeed correct. But
it would be the greatest thoughtlessness if that were supposed to
mean that in our consciousness concrete existence is bound up in
the same way with the representation of finite things as with the
representation of God. It would be forgotten that finite things are
mutable and transient, i.e. that concrete existence is only bound up
with them in a transitory manner, that this bond is not eternal, but
separable. For this reason, in the course of relegating the sort of
connection that obtains in the case of finite things,
Anselm rightly declared that alone to be the perfect being that is not
merely in a subjective but at the same time in an objective manner.
None of the condescension shown towards the so~called 'ontological'
proof and against this Anselmian determination of the perfect is of
any help, since it lies in every innocent, common sense just as much
as it returns in every philosophy, even against one's will and better
judgment, as in the principle of immediate belief.

But the deficiency in Anselm's argument
(a deficiency, moreover, that Descartes and Spinoza
as well as the principle of immediate knowing share with it)
is that this unity, articulated as the most perfect being
or subjectively as true knowing,
is presupposed, i.e. it is only assumed
as something that is in itself.
The diversity of the two determinations is
immediately opposed to this entity
which is accordingly abstract
(something that has long since been held against Anselm).
That is, in fact, to say that the
representation and concrete existence of the finite
are opposed to the infinite since, as noted earlier,
the finite is the sort of objectivity
that is at the same time not adequate
to the purpose, to its essence and concept,
the sort of objectivity that differs from it
or that it is the sort of the representation,
the sort of subjective entity that
does not involve concrete existence.
This objection and contrast is
sublated only by demonstrating
that the finite is something untrue,
that these determinations are,
for themselves, one-sided and vacuous
and that the identity, accordingly,
is one into which they themselves pass over
and in which they are reconciled.
