SECTION III

Actuality

Actuality is the unity of essence and concrete existence;
in it, shapeless essence and unstable appearance
(subsistence without determination
and manifoldness without permanence)
have their truth.
Although concrete existence is the immediacy
that has proceeded from ground,
it still does not have form explicitly posited in it;
inasmuch as it determines and informs itself, it is appearance;
and in developing this subsistence that otherwise only is
a reflection-into-other into an immanent reflection,
it becomes two worlds, two totalities of content,
one determined as reflected into itself
and the other as reflected into other.
But the essential relation exposes
the formality of their connection,
and the consummation of the latter is
the relation of the inner and the outer
in which the content of both is equally
only one identical substrate
and only one identity of form.
This identity has come about also in regard to form,
the form determination of their difference is sublated,
and that they are one absolute totality is posited.

This unity of the inner and outer is absolute actuality.
But this actuality is, first, the absolute as such
(in so far as it is posited as a unity
in which the form has sublated itself)
making itself into the empty or external
distinction of an outer and inner.
Reflection relates to this absolute
as external to it;
it only contemplates it
rather than being its own movement.
But it is essentially this movement
and is, therefore, as the absolute's
negative turning back into itself.

Second, it is actuality proper.
Actuality, possibility, and necessity constitute
the formal moments of the absolute,
or its reflection.

Third, the unity of the absolute
and its reflection is
the absolute relation,
or rather the absolute as
relation to itself, substance.

CHAPTER 1

The absolute

The simple solid identity of the absolute is indeterminate,
or rather, every determinateness of essence and concrete existence,
or of being in general as well as of reflection,
has dissolved itself into it.
Accordingly, the determining of what is
the absolute appears to be a negating,
and the absolute itself appears only as
the negation of all predicates, as the void.
But since it must equally be spoken of
as the position of all predicates,
it appears as the most formal of contradictions.
In so far as that negating and this positing
belong to external reflection,
what we have is a formal, unsystematic dialectic
that has an easy time picking up
a variety of determinations here and there,
and is just as at ease demonstrating, on the one hand,
their finitude and relativity, as declaring, on the other,
that the absolute, which it vaguely envisages as totality,
is the dwelling place of all determinations,
yet is incapable of raising
either the positions or the negations
to a true unity.
The task is indeed to demonstrate what the absolute is.
But this demonstration cannot be either
a determining or an external reflection
by virtue of which determinations
of the absolute would result,
but is rather the exposition of the absolute,
more precisely the absolute's own exposition,
and only a displaying of what it is.

A. THE EXPOSITION OF THE ABSOLUTE

The absolute is not just being, nor even essence.
The former is the first unreflected immediacy;
the latter, the reflected immediacy;
further, each is explicitly a totality,
but a determinate totality.
Being emerges in essence as concrete existence,
and the connection of being and essence develops
into the relation of inner and outer.
The inner is essence, but as a totality
whose essential determination is
to be referred to being and to be being immediately.
The outer is being, but with the essential determination of
being immediately connected with reflection
and, equally, in a relationless identity with essence.
The absolute itself is the absolute unity of the two;
it is that which constitutes in general
the ground of the essential relation
which, as only relation, has yet
to return into this its identity
and whose ground is not yet posited.

It follows that the determination of
the absolute is to be absolute form,
but at the same time not as an identity
whose moments only are simple determinacies,
but, on the contrary, as an identity
whose moments are each explicitly the totality
and hence, indifferent with respect to the form,
the complete content of the whole.
But, conversely, the absolute is absolute content
in such a way that this content,
which is as such indifferent plurality,
explicitly has the negative connection of form
by virtue of which its manifold is
only one substantial identity.

Thus the identity of the absolute is
for this reason absolute identity,
because each of its parts is itself the whole
or each determinateness is the totality, that is,
because determinateness has become as such
a thoroughly transparent reflective shine,
a difference that has disappeared in its positedness.
Essence, concrete existence, the world existing in itself,
whole, parts, force:
these reflected determinations appear to representation
as true being valid in and for itself;
but against them the absolute is the ground
into which they have foundered.
Because in the absolute the form is
now only simple self-identity,
the absolute does not determine itself,
for the determination is a difference of form
which is valid as such from the start.
But because the absolute at the same time contains
every difference and form determination in general,
or because it is itself absolute form and reflection,
the difference of content must also come into it.
But the absolute itself is the absolute identity;
to be this identity is its determination,
for the manifoldness of the world-in-itself
and of the phenomenal world has all been sublated in it.
In the absolute itself there is no becoming,
since the absolute is not being;
nor does the absolute determine itself reflectively,
for it is not the essence which determines itself only inwardly;
and it also does not externalize itself,
for it is the identity of inner and outer.
But in this way the movement of reflection
stands over against its absolute identity.
The movement is sublated in this identity
and is thus only its inner;
but consequently its outer.

At first, therefore, the movement consists only
in sublating its act in the absolute.
It is the beyond of the manifold differences
and determinations and of their movement,
a beyond that lies at the back of the absolute.
It is thus the negative exposition of the absolute
earlier alluded to.
In its true presentation, this exposition is
the preceding whole of the logical movement
of the spheres of being and essence,
the content of which has not been gathered in
from outside as something given and contingent;
nor has it been sunk into the abyss of the absolute
by a reflection external to it;
on the contrary, it has determined itself within it
by virtue of its inner necessity,
and, as being's own becoming
and as the reflection of essence,
has returned into the absolute
as into its ground.

But this exposition has itself also a positive side,
for in foundering to the ground the finite demonstrates
that its nature is to be referred to the absolute,
or to contain the absolute within.
However, this side is not as much
the positive exposition of the absolute
as it is rather the exposition of the determinations,
namely that these have the absolute for their abyss,
but also for their ground,
or that that which imparts subsistence to them,
to their reflective shine, is the absolute itself.
Being as shine is not nothing but reflection,
reference to the absolute;
or it is a shine inasmuch as
that which shines in it is the absolute.
This positive exposition thus halts the finite
just before its disappearing:
it considers it an expression and
a copy of the absolute.
But this transparency of the finite
that lets only the absolute transpire through it
ends up in complete disappearance,
for there is nothing in the finite
which would retain for it a difference
over against the absolute;
as a medium, it is absorbed by
that through which it shines.

This positive exposition of the absolute is
therefore itself only a reflective shine,
for the true positive, that which contains
the exposition and the expounded content,
is the absolute itself.
Whatever the further determinations that may occur,
the form in which the absolute reflectively shines is
a nullity which the exposition gathers up from outside
and in which it gains for itself
a starting point for its activity.
Any such determination has in the absolute,
not its beginning but its end.
This expository process, therefore,
though it is an absolute act
because of its reference to
the absolute into which it returns,
is not so at its starting point
which is a determination
external to the absolute.

But in actual fact
the exposition of the absolute
is the absolute's own doing,
an act that begins from itself
and arrives at itself.
The absolute, only as absolute identity,
is absolute in a determined guise,
that is, as identical absolute;
it is posited as such by reflection
over against opposition and manifoldness;
or it is only the negative of
reflection and determination in general.
It is not just the exposition of the absolute
which is therefore something incomplete,
but this absolute itself
which is only arrived at.
Or again, the absolute
which is only as absolute identity
is only the absolute of an external reflection.
It is, therefore, not the absolutely absolute
but the absolute in a determination,
or it is attribute.

But the absolute is not attribute just because
it is the subject matter of an external reflection
and is consequently something determined by it.
Or, reflection is not only external to it;
but, precisely because it is external to it,
it is immediately internal to it.
The absolute is absolute only because
it is not abstract identity
but is the identity of being and essence,
or the identity of the inner and the outer.
It is therefore itself the absolute form
that makes it reflectively shine within itself
and determines it as attribute.

B. THE ABSOLUTE ATTRIBUTE

The expression which we have used, “the absolute absolute,”
denotes the absolute which in its form
has returned back into itself
or whose form is equal to its content.
The attribute is just the relative absolute,
a combination which only signifies the absolute
in a form determination.
For at first, before its complete exposition,
the form is only internally
or, which is the same, only externally;
it is at first determinate form in general
or negation in general.
But because form is at the same time
as the form of the absolute,
the attribute is the whole content of the absolute;
it is the totality which earlier appeared as a world,
or as one of the sides of the essential relation,
each of which is itself the whole.
But both worlds, the phenomenal world
and the world that exists in and for itself,
were supposed to be opposed
to each other in their essence.
Each side of the essential relation was
indeed equal to the other:
the whole as much as the parts,
the expression of force the same content
as force itself,
and the outer everywhere the same as the inner.
But these sides were at the same time
supposed each to have still
an immediate subsistence of its own,
the one side as existent immediacy
and the other as reflected immediacy.
In the absolute, on the contrary,
these different immediacies have been
reduced to a reflective shine,
and the totality that the attribute is
is posited as its true and single subsistence,
while the determination in which it is
is posited as unessential subsistence.

The absolute is attribute because,
as simple absolute identity,
it is in the determination of identity;
now to the determination as such
other determinations can be attached,
for instance, also that there are several attributes.
But because absolute identity has only this meaning,
that not only all determinations have been sublated
but that reflection itself has also sublated itself,
all determinations are thus posited in it as sublated.
Or the totality is posited as absolute totality.
Or again, the attribute has the absolute
for its content and subsistence
and, consequently, its form determination
by which it is attribute is also posited,
posited immediately as mere reflective shine;
the negative is posited as negative.
The positive reflective shine that
the exposition gives itself through the attribute
in that it does not take the finite in its limitation as
something that exists in and for itself
but dissolves its subsistence into the absolute
and expands it into attribute;
sublates precisely this, that the attribute is attribute;
it sinks it and its differentiating act
into the simple absolute.

But since reflection thus reverts
from its differentiating act
only to the identity of the absolute,
it has not at the same time
left its externality behind
and has not arrived at the true absolute.
It has only reached the
indeterminate, abstract identity,
which is to say, the identity
in the determinateness of identity.
Or, since reflection determines the
absolute into attribute as inner form,
this determining is something
still distinct from externality;
the inner determination does not
penetrate the absolute;
the attribute's expression,
as something merely posited,
is to disappear into the absolute.

The form by virtue of which
the absolute would be attribute,
whether it is taken as outer or inner,
is therefore posited as something null in itself,
an external reflective shine,
or a mere way and manner.

C. THE MODE OF THE ABSOLUTE

The attribute is first
the absolute in simple self-identity.
Second, it is negation,
a negation which is as such
formal immanent reflection.
These two sides constitute at first
the two extremes of the attribute,
the middle term of which is the attribute itself,
since it is both the absolute and the determinateness.
The second of these extremes is the negative as negative,
the reflection external to the absolute.
Or inasmuch as the negative is
taken as the inner of the absolute
and its own determination is to posit itself as mode,
it is then the self-externality of the absolute,
the loss of itself in the
changeability and contingency of being,
its having passed over into its opposite
without turning back into itself,
the manifoldness of form and content
determinations that lacks totality.

But the mode, the externality of the absolute, is not just this.
It is rather externality posited as externality,
a mere way and manner,
hence the reflective shine as reflective shine,
or the reflection of form into itself;
hence, the self-identity which is the absolute.
In actual fact, therefore, the absolute is first posited
as absolute identity only in the mode;
it is what it is, namely self-identity,
only as self-referring negativity,
as reflective shining which is posited as reflective shining.

Hence, in so far as the exposition of the absolute
begins from its absolute identity
and passes over to the attribute
and from there to the mode,
it has therein exhaustively run through its moments.

But first, in this course it does not just behave
negatively towards these determinations;
its act is rather the reflective movement itself,
and it is only as such a movement that
the absolute truly is absolute identity.

Second, the exposition does not thereby deal with mere externality,
and the mode is not only the most external externality.
Rather, since the mode is reflective shine as shine,
it is an immanent turning back, the self-dissolving reflection,
and it is in being this reflection that
the absolute is absolute being.

Third, the reflective act of exposition seems to begin
from its own determinations and from something external,
to take up the modes or even the determinations of the attribute
as if they were found outside the absolute
and its contribution were only to reduce them
to undifferentiated identity.
But it has in fact found the determinateness
from which it begins in the absolute itself.
For as first undifferentiated identity,
the absolute is itself only the determinate absolute,
or attribute, because it is the unmoved,
still unreflected absolute.
This determinateness, since it is determinateness,
belongs to the reflective movement,
and it is through this movement alone
that the absolute is determined as the first identity;
through it alone that it has absolute form
and does not just exist as self-equal
but posits itself as self-equal.

Accordingly the true meaning of mode is that
it is the absolute's own reflective movement;
it is a determining by virtue of which
the absolute would become, not an other,
but what it already is;
a transparent externality
which is a pointing to itself;
a movement out of itself,
but in such a way that being outwardly is
just as much inwardness,
and consequently equally a positing
which is not mere positedness
but absolute being.

When therefore one asks for a content of the exposition,
for what the absolute manifests,
the reply is that the distinction of form and content
in the absolute has been dissolved;
or that just this is the content of the absolute,
that it manifests itself.
The absolute is the absolute form
which in its diremption of itself is
utterly identical with itself,
is the negative as negative
or the negative that rejoins itself
and in this way alone is the absolute self-identity
which equally is indifferent towards its distinctions
or is absolute content.
The content is therefore only this exposition itself.

As this self-bearing movement of exposition,
as a way and manner which is
its absolute identity with itself,
the absolute is expression,
not of an inner,
nor over against an other,
but simply as absolute manifestation
of itself for itself.
Thus it is actuality.

CHAPTER 2

Actuality

The absolute is the unity of inner and outer
as a first implicitly existent unit.
The exposition appeared as an external reflection
which, for its part, has the immediate
as something it has found,
but it equally is its movement
and the reference connecting it to the absolute
and, as such, it leads it back to the latter,
determining it as a mere “way and manner.”
But this “way and manner” is the
determination of the absolute itself,
namely its first identity
or its mere implicitly existent unity.
And through this reflection, not only is
that first in-itself posited as essenceless determination,
but, since the reflection is negative self-reference,
it is through it that the in-itself becomes
a mode in the first place.
It is this reflection that,
in sublating itself in its determinations
and as a movement which as such turns back upon itself,
is first truly absolute identity
and, at the same time, the determining of
the absolute or its modality.
The mode, therefore, is the externality of the absolute,
but equally so only its reflection into itself;
or again, it is the absolute's own manifestation,
so that this externalization is its immanent reflection
and therefore its being in-and-for-itself.

So, as the manifestation that it is nothing,
that it has no content, save to be
the manifestation of itself,
the absolute is absolute form.
Actuality is to be taken as
this reflected absoluteness.
Being is not yet actual;
it is the first immediacy;
its reflection is therefore becoming
and transition into an other;
or its immediacy is not being-in-and-for-itself.
Actuality also stands higher than concrete existence.
It is true that the latter is the immediacy
that has proceeded from ground and conditions,
or from essence and its reflection.
In itself or implicitly, it is therefore
what actuality is, real reflection;
but it is still not the posited unity of reflection and immediacy.
Hence concrete existence passes over into appearance
as it develops the reflection contained within it.
It is the ground that has foundered to the ground;
its determination, its vocation, is to restore this ground,
and therefore it becomes essential relation,
and its final reflection is that its
immediacy be posited as immanent reflection and conversely.
This unity, in which concrete existence
or immediacy and the in-itself,
the ground or the reflected, are simply moments,
is now actuality.
The actual is therefore manifestation.
It is not drawn into
the sphere of alteration by its externality,
nor is it the reflective shining of itself in an other.
It just manifests itself,
and this means that in its externality,
and only in it, it is itself, that is to say,
only as a self-differentiating and self-determining movement.

Now in actuality as this absolute form,
the moments only are as sublated or formal, not yet realized;
their differentiation thus belongs at first to external reflection
and is not determined as content.

Actuality, as itself immediate form-unity of inner and outer,
is thus in the determination of immediacy
as against the determination of immanent reflection;
or it is an actuality as against a possibility.
The connection of the two to each other is the third,
the actual determined both as being reflected into itself
and as this being immediately existing.
This third is necessity.

But first, since the actual and the possible
are formal distinctions,
their connection is likewise only formal,
and consists only in this,
that the one just like the other
is a positedness, or in contingency.

Second, because in contingency
the actual as well as the possible
are a positedness,
because they have retained their determination,
real actuality now arises,
and with it also real possibility
and relative necessity.

Third, the reflection of relative necessity
into itself yields absolute necessity,
which is absolute possibility and actuality.

A. CONTINGENCY OR FORMAL ACTUALITY, POSSIBILITY, AND NECESSITY

1. Actuality is formal inasmuch as, as a first actuality,
it is only immediate, unreflected actuality,
and hence is only in this form determination
but not as the totality of form.
And so it is nothing more than a being,
or concrete existence in general.
But because by essence it is not mere concrete existence
but is the form-unity of the in-itselfness
or inwardness and externality,
it immediately contains in-itselfness or possibility.
What is actual is possible.

2. This possibility is actuality reflected into itself.
But this reflectedness, itself a first, is equally something formal
and consequently only the determination of self-identity
or of the in-itself in general.

But because the determination is here totality of form,
this in-itself is determined as sublated
or essentially only with reference to actuality;
as the negative of actuality, it is posited as negative.
Possibility entails, therefore, two moments.
It has first the positive moment
of being a being-reflected-into-itself.
But this being-reflected-into-itself,
since in the absolute form it is reduced to a moment,
no longer has the value of essence but has rather
the negative meaning that possibility is (in a second moment)
something deficient, that it points to an other, to actuality,
and is completed in this other.

According to the first, merely positive side,
possibility is therefore the mere
form determination of self-identity,
or the form of essentiality.
As such it is the relationless, indeterminate
receptacle of everything in general.
In this formal sense of possibility,
everything is possible
that does not contradict itself;
the realm of possibility is therefore
limitless manifoldness.
But every manifold is determined in itself
and as against an other:
it possesses negation within.
Indifferent diversity passes over
as such into opposition;
but opposition is contradiction.
Therefore, all things are
just as much contradictory
and hence impossible.

When we therefore say of something
that “it is possible,”
this purely formal assertion is
just as superficial and empty
as the principle of contradiction,
and any content that we put into it,
“A is possible,” says no more than “A is A.”
Left undeveloped, this content has
the form of simplicity;
only after being resolved
into its determinations,
does difference emerge within it.
To the extent that we stop at that
simple for the content remains
something self-identical
and hence a possible.
But we do not say anything by it,
just as we do not with the principle of identity.

Yet the possible amounts to more
than just the principle of identity.
The possible is reflected immanent reflectedness;
or the identical simply as a moment of the totality,
hence also as determined not to be in itself;
it therefore has the second determination of being only a possible
and the ought-to-be of the totality of form.
Without this ought-to-be, possibility is essentiality as such;
but the absolute form entails this,
that essence itself is only a moment
and that it has no truth without being.
Possibility is this mere essentiality,
but so posited as to be only a moment,
to be disproportionate with respect to the absolute form.
It is the in-itself, determined as only a posited
or, equally, as not to be in itself.

Internally, therefore, possibility is contradiction,
or it is impossibility.

This finds expression at first in this way,
that possibility as form determination
posited as sublated possesses a content in general.
As possible, this content is an in-itself
which is at the same time something sublated
or an otherness.
But because this content is only a possible,
an other opposite to it is equally possible.
“A is A”; then, too, “not-A is not-A.”
These two statements each express
the possibility of its content determination.
But, as identical statements,
they are indifferent to each other;
that the other is also added,
is not posited in either.
Possibility is the connection comparing the two;
as a reflection of the totality,
it implies that the opposite also is possible.
It is therefore the ground for drawing the connection that,
because A equals A, not-A also equals not-A;
entailed in the possible A there is also the possible not-A,
and it is this reference itself connecting them
which determines both as possible.

But this connection, in which
the one possible also contains its other,
is as such a contradiction that sublates itself.
Now, since it is determined to be reflective
and, as we have just seen, reflectively self-sublating,
it is also therefore an immediate
and it consequently becomes actuality.

3. This actuality is not the first actuality
but reflected actuality,
posited as unity of itself and possibility.
What is actual is as such possible;
it is in immediate positive identity with possibility;
but the latter has determined itself as only possibility;
consequently the actual is also determined as only a possible.
And because possibility is immediately contained in actuality,
it is immediately in it as sublated, as only possibility.
Conversely, actuality which is in unity with possibility
is only sublated immediacy;
or again, because formal actuality is only immediate first actuality,
it is only a moment, only sublated actuality, or only possibility.

With this we also have a more precise expression of
the extent to which possibility is actuality.
Possibility is not yet all actuality;
there has been no talk yet of real and absolute actuality.
It is still only the possibility as it first presented itself,
namely the formal possibility that has determined itself
as being only possibility and hence the formless actuality
which is only being or concrete existence in general.
Everything possible has therefore in general
a being or a concrete existence.

This unity of possibility and actuality is contingency.
The contingent is an actual which is at the same time
determined as only possible,
an actual whose other or opposite equally is.
This actuality is, therefore, mere being or concrete existence,
but posited in its truth as having the value
of a positedness or a possibility.
Conversely, possibility is immanent reflection
or the in-itself posited as positedness;
what is possible is an actual in this sense of actuality,
that it has only as much value as contingent actuality;
it is itself something contingent.

The contingent thus presents these two sides.
First, in so far as it has possibility immediately in it,
or, what is the same, in so far as
this possibility is sublated in it,
it is not positedness, nor is it mediated,
but is immediate actuality; it has no ground.
Because this immediate actuality pertains also to the possible,
the latter is determined no less than the actual as contingent
and is likewise groundless.

But, second, the contingent is the actual
as what is only possible, or as a positedness;
thus the possible also, as formal in-itself, is only positedness.
Consequently, the two are both not in and for themselves
but have their immanent reflection in an other,
or they do have a ground.

The contingent thus has no ground because it is contingent;
and for that same reason it has a ground, because it is contingent.

It is the posited, immediate conversion of inner and outer,
or of immanently-reflected-being and being,
each into the other posited,
because possibility and actuality
both have this determination in them
by being moments of the absolute form.
So actuality, in its immediate unity with possibility,
is only concrete existence and is determined as groundless,
something only posited or only possible;
or, as reflected and determined over against possibility,
it is separated from possibility,
from immanent reflectedness,
and then, too, is no less
immediately only a possible.
Likewise possibility, as simple in-itself,
is something immediate,
only an existent in general;
or, opposed to actuality,
it equally is an in-itself
without actuality, only a possible,
but, for that very reason,
again only a concrete,
not immanently reflected,
existence in general.

This absolute restlessness of the becoming of
these two determinations is contingency.
But for this reason, because each determination
immediately turns into the opposite,
in this opposite each equally rejoins itself,
and this identity of the two,
of each in the other,
is necessity.

The necessary is an actual;
as such it is immediate, groundless;
but it equally has its actuality
through an other or in its ground
and is at the same time the positedness of this ground
and its reflection into itself;
the possibility of the necessary is a sublated one.
The contingent is therefore necessary
because the actual is determined as a possible;
its immediacy is consequently sublated
and is repelled into the ground or the in-itself,
and into the grounded, equally because its possibility,
this ground-grounded-connection,
is simply sublated and posited as being.
What is necessary is,
and this existent is itself the necessary.
At the same time it is in itself;
this immanent reflection is an other than that immediacy of being,
and the necessity of the existent is an other.
Thus the existent is not the necessary;
but this in-itself is itself only positedness;
it is sublated and itself immediate.
And so actuality, in that from which it is distinguished,
in possibility, is identical with itself.
As this identity, it is necessity.

B. RELATIVE NECESSITY OR REAL ACTUALITY, POSSIBILITY, AND NECESSITY

1. The necessity which has resulted is formal
because its moments are formal,
that is, simple determinations which are a totality
only as an immediate unity,
or as an immediate conversion of the one into the other,
and thus lack the shape of self-subsistence.
The unity in this formal necessity is therefore simple at first,
and indifferent to its differences.
As the immediate unity of the form determinations,
this necessity is actuality,
but an actuality which, since its unity is now determined as indifferent
to the difference of the form determinations, has a content.
This content as an indifferent identity contains the form
also as indifferent that is, as a mere variety of determinations,
and is a manifold content in general.
This actuality is real actuality.

Real actuality is as such at first
the thing of many properties,
the concretely existing world;
but it is not the concrete existence
that dissolves into appearance
but, as actuality, it is at the same time
an in-itself and immanent reflection;
it preserves itself in the manifoldness of mere concrete existence;
its externality is an inner relating only to itself.
What is actual can act;
something announces its actuality by what it produces.
Its relating to an other is the manifestation of itself,
and this manifestation is
neither a transition
(the immediate something refers to the other in this way)
nor an appearing
(in this way the thing only is in relation to an other);
it is a self-subsistent which has its immanent reflection,
its determinate essentiality, in another self-subsistent.

Now real actuality likewise has possibility immediately present in it.
It contains the moment of the in-itself;
but, since it is in the first instance only immediate unity,
it is in one of the determinations of form
and hence distinguished, as immediate existent,
from the in-itself or possibility.

2. This possibility, as the in-itself of real actuality,
is itself real possibility, at first the in-itself full of content.
Formal possibility is immanent reflection only as abstract identity,
the absence of contradiction in a something.
But when we delve into the determinations,
the circumstances, the conditions of a fact
in order to discover its possibility,
we do not stop at this formal possibility
but consider its real possibility.

This real possibility is itself immediate concrete existence,
but no longer because possibility as such, as a formal moment,
is immediately its opposite, a non-reflected actuality,
but because this determination pertains to it
by the very fact of being real possibility.
The real possibility of a fact is therefore
the immediately existent manifoldness of
circumstances that refer to it.

This manifoldness of existence is therefore indeed
both possibility and actuality,
but their identity is at first only the content
which is indifferent to these form determinations;
they therefore constitute the form,
determined as against their identity.
Or the immediate real actuality, because it is immediate,
is determined as against its possibility;
as this determinate and hence reflected actuality,
it is real possibility.
This real possibility is now indeed the posited whole of the form,
but of the form in the determinateness of actuality as formal
or immediate and equally of possibility as the abstract in-itself.
This actuality, therefore, which constitutes the possibility of a fact,
is not its own possibility but the in-itself of an other actual;
itself, it is the actuality that ought to be sublated,
the possibility as only possibility.
Real possibility thus constitutes the totality of conditions,
a dispersed actuality which is not reflected into itself
but is determined to be the in-itself of an other
and intended in this determination to return to itself.

What is really possible is, therefore,
something formally identical according to its in-itself,
free of contradiction because of its simple content determination;
but, as self-identical, this something must also not contradict
itself according to its developed and differentiated circumstances
and all else connected with it.
But, secondly, because it is manifold in itself
and in manifold connection with others,
and variety inherently passes over into opposition,
it is contradictory.
Whenever a possibility is in question,
and the issue is to demonstrate its contradiction,
one need only fasten on to the multiplicity that it contains as content
or as its conditioned concrete existence,
and from this the contradiction will easily be discovered.
And this contradiction is not just a function of comparing;
on the contrary, the manifold of concrete existence is in itself this,
to sublate itself and to founder to the ground:
in this it explicitly has the determination of
being only a possibility.
Whenever all the conditions of a fact are completely present,
the fact is actually there;
the completeness of the conditions is
the totality as in the content,
and the fact is itself this content determined
as being equally actual as possible.
In the sphere of the conditioned ground,
the conditions have the form
(that is, the ground or the reflection that stands on its own)
outside them,
and it is this form that makes them moments
of the fact and elicits concrete existence in them.
Here, on the contrary, the immediate actuality is
not determined to be condition by virtue of
a presupposing reflection,
but the supposition is rather that the immediate actuality is
itself the possibility.

In self-sublating real possibility,
it is a twofold that is now sublated;
for this possibility is itself
the twofold of actuality and possibility.
(1) The actuality is formal, or is a concrete existence
which appeared to subsist immediately,
and through its sublating becomes reflected being,
the moment of an other,
and thus comes in possession of the in-itself.
(2) That concrete existence was also determined
as possibility or as the in-itself, but of an other.
As it sublates itself, this in-itself of the other is
also sublated and passes over into actuality.
This movement of self-sublating real possibility
thus produces the same moments that are already present,
but each as it comes to be out of the other;
in this negation, therefore, the possibility
is also not a transition but a self-rejoining.
In formal possibility, if something was possible,
then an other than it, not itself, was also possible.
Real possibility no longer has such an other over against it,
for it is real in so far as it is itself also actuality.
Therefore, as its immediate concrete existence,
the circle of conditions, sublates itself,
it makes itself into the in-itselfness which it already is,
namely the in-itself of an other.
And conversely, since its moment of in-itselfness
thereby sublates itself at the same time,
it becomes actuality, hence the moment
which it likewise already is.
What disappears is consequently this,
that actuality was determined as the possibility
or the in-itself of an other,
and, conversely, the possibility as an actuality
which is not that of which it is the possibility.

3. The negation of real possibility is thus its self-identity;
inasmuch as in its sublating it is thus within itself
the recoiling of this sublating, it is real necessity.

What is necessary cannot be otherwise;
but what is only possible can be,
for possibility is the in-itself
which is only positedness
and hence essentially otherness.
Formal possibility is this identity
as transition into the other as such;
but real possibility, since it has
the other moment of actuality within it,
is already itself necessity.
Hence what is really possible can no longer be otherwise;
under the given conditions and circumstances,
nothing else can follow.
Real possibility and necessity are, therefore,
only apparently distinguished;
theirs is an identity that does not first come to be
but is already presupposed at their base.
Real possibility is therefore a connection full of content,
for the content is that identity, existing in itself,
which is indifferent to form.

But this necessity is at the same time relative.
For it has a presupposition from which it begins;
it takes its start from the contingent.
For the real actual is as such the determinate actual,
and first has its determinateness as immediate being
in that it is a multiplicity of concretely existing circumstances;
but this immediate being as determinateness is also the negative of
itself, is an in-itself or possibility and so real possibility.
As this unity of the two moments, it is the totality of form,
but a totality which is still external to itself;
it is the unity of possibility and actuality in such a way that
(1) the manifold concrete existence is possibility immediately or positively:
it is a possible, something self-identical as such, because it is an actual;
(2) inasmuch as this possibility of concrete existence is posited,
it is determined as only possibility,
as the immediate conversion of actuality into its opposite, or as contingency.
Hence this possibility which immediate actuality has within
in so far as it is condition, is only the in-itself
or the possibility of an other.
Because this in-itself, as shown, sublates itself and this positedness
is itself posited, real possibility becomes indeed necessity;
but this necessity thus begins from that unity of the possible and the actual
which is not yet reflected into itself;
this presupposing and the movement which turns back
unto itself are still separate;
or necessity has not yet determined itself
out of itself into contingency.

The relativity of real possibility is manifested in the content
by the fact that the latter is at first only
the identity indifferent to form,
is therefore distinct from it
and a determinate content in general.
A necessary reality is for this reason any limited actuality
which, because of its limitation, is in some other respect
also only something contingent.

In actual fact, therefore, real necessity is
in itself also contingency.
This first becomes apparent because real necessity,
although something necessary according to form,
is still something limited according to content,
and derives its contingency through the latter.
But this contingency is to be found also
in the form of real necessity because, as shown,
real possibility is the necessary only in itself,
but as posited it is the mutual otherness of
actuality and possibility.
Real necessity thus contains contingency;
it is the turning back into itself from the restless
being-the-other-of-each-other of actuality and possibility,
but not the turning back from itself to itself.
In itself, therefore, we have here the unity
of necessity and contingency;
this unity is to be called absolute actuality.

C. ABSOLUTE NECESSITY

Real necessity is determinate necessity;
formal necessity does not yet have
any content and determinateness in it.
The determinateness of necessity consists in
its having its negation, contingency, within it.
This is how it has shown itself to be.

But in its first simplicity this determinateness is actuality;
determinate necessity is therefore immediate actual necessity.
This actuality which is itself as such necessary,
since it contains necessity as its in-itself,
is absolute actuality;
an actuality which can no longer be otherwise,
or its in-itself is not possibility but necessity itself.

But because this actuality is posited to be absolute,
that is to say, to be itself
the unity of itself and possibility,
it is consequently only an empty determination,
or it is contingency.
This emptiness of its determination makes it
into a mere possibility,
one which can just as well be an other
and is determined as possibility.
But this possibility is itself absolute possibility,
for it is precisely the possibility of being
equally determined as possibility and actuality.
For this reason, because it is this indifference towards itself,
it is posited as empty, contingent determination.

Thus real necessity not only contains contingency implicitly,
but the latter also becomes in it;
but this becoming, as externality,
is itself only the in-itself of the necessity,
because it is only an immediate determinateness.
But it is not only this but the necessity's own becoming
or the presupposition which it had is its own positing.
For as real necessity, it is the sublatedness
of actuality into possibility
and of possibility into actuality;
because it is this simple conversion of
one of these moments into the other,
it is also their positive unity,
for in the other each rejoins itself.
And so it is actuality, yet an actuality
which is nothing but this rejoining of form with itself.
Its negative positing of these moments is
thereby itself the presupposing
or the positing of itself as sublated,
or the positing of immediacy.

But it is precisely in this positing that
this actuality is determined as the negative;
it rejoins itself from the actuality which was real possibility;
this new actuality thus comes to be only out of its in-itself,
out of the negation of itself.
Consequently, it is at the same time
immediately determined as possibility,
as mediated by virtue of its negation.
But accordingly, this possibility is
immediately nothing but this mediating
in which the in-itself,
namely the possibility itself and the mediating,
both in the same manner, are positedness.
Thus it is necessity which is
equally the sublating of this positedness,
or the positing of immediacy and of the in-itself,
just as in this very sublating
it is the determining of it as positedness.
It is necessity itself, therefore,
that determines itself as contingency:
in its being it repels itself from itself,
in this very repelling has only returned to itself,
and in this turning back which is its being
has repelled itself from itself.

Thus has form pervaded in its realization all its distinctions;
it has made itself transparent
and, as absolute necessity,
is only this simple self-identity of
being in its negation, or in essence.
The distinction itself of content and form
has thus equally vanished;
for that unity of possibility in actuality
and actuality in possibility is the form
which in its determinateness or in positedness is
indifferent towards itself:
it is the fact full of content
on which the form of necessity
externally ran its course.
But necessity is thus this
reflected identity of the two determinations
as indifferent to them,
and hence the form determination of the in-itself
as against the positedness,
and this possibility constitutes the limitation of
the content which real necessity had.
The resolution of this difference is
however the absolute necessity
whose content is this difference
which in this necessity penetrates itself.

Absolute necessity is therefore the truth
in which actuality and possibility in general
as well as formal and real necessity return.
As we have just seen, it is being which in its negation,
in essence, refers itself to itself and is being.
It is equally simple immediacy or pure being
and simple immanent reflection or pure essence;
it is this, that the two are one and the same.
The absolutely necessary only is because it is;
it otherwise has neither condition nor ground.
But it equally is pure essence,
its being the simple immanent reflection;
it is because it is.
As reflection, it has a ground and a condition
but has only itself for this ground and condition.
It is in-itself, but its in-itself is its immediacy,
its possibility is its actuality.
It is, therefore, because it is;
as the rejoining of being with itself,
it is essence;
but because this simple is
equally immediate simplicity,
it is being.

Absolute necessity is thus
the reflection or form of the absolute,
the unity of being and essence,
simple immediacy which is absolute negativity.
On the one hand, therefore, its differences are
not like the determinations of reflection
but an existing manifoldness,
a differentiated actuality in the shape of others
independently subsisting over against each other.
On the other hand, since its connection is
that of absolute identity,
it is the absolute conversion of
its actuality into its possibility
and its possibility into its actuality.
Absolute necessity is therefore blind.
On the one hand, the two different terms
determined as actuality and possibility have
the shape of immanent reflection as being;
they are therefore free actualities,
neither of which reflectively shines in the other,
nor will either allow in it
a trace of its reference to the other;
grounded in itself, each is inherently necessary.
Necessity as essence is concealed in this being;
the reciprocal contact of these actualities appears
therefore, as an empty externality;
the actuality of the one in the other is
the possibility which is only possibility, contingency.
For being is posited as absolutely necessary,
as the self-mediation which is
the absolute negation of mediation-through-other,
or being which is identical only with being;
consequently, an other that has actuality in being,
is therefore determined as something merely possible,
as empty positedness.

But this contingency is rather absolute necessity;
it is the essence of those free, inherently necessary actualities.
This essence is averse to light,
because there is no reflective shining
in these actualities, no reflex:
because they are grounded purely in themselves,
are shaped for themselves,
manifest themselves only to themselves;
because they are only being.
But their essence will break forth in them
and will reveal what it is and what they are.
The simplicity of their being,
their resting just on themselves,
is absolute negativity;
it is the freedom of their reflectionless immediacy.
This negative breaks forth in them because being,
through this same negativity which is its essence,
is self-contradiction;
it will break forth against this
being in the form of being,
hence as the negation of those actualities,
a negation absolutely different from their being;
it will break forth as their nothing,
as an otherness which is just as free towards them
as their being is free.
Yet this negative was not to be missed in them.
In their self-based shape
they are indifferent to form,
are a content and consequently
different actualities
and a determinate content.
This content is the mark
that necessity impressed upon them
by letting them go free as absolutely actual;
for in its determination it is
an absolute turning back into itself.
It is the mark to which necessity appeals
as witness to its right,
and, overcome by it,
the actualities now perish.
This manifestation of what
determinateness is in its truth,
that it is negative self-reference,
is a blind collapse into otherness;
in the sphere of immediate existence,
the shining or the reflection
that breaks out in it is a becoming,
a transition of being into nothing.
But, conversely, being is equally essence,
and becoming is reflection or a shining.
Thus the externality is its inwardness;
their connection is one of absolute identity;
and the transition of the actual into the possible,
of being into nothing, is a self-rejoining;
contingency is absolute necessity;
it is itself the presupposing of that
first absolute actuality.

This identity of being with itself
in its negation is now substance.
It is this unity as in its negation
or as in contingency;
and so, as relation to itself, it is substance.
The blind transition of necessity is
rather the absolute's own exposition,
its movement in itself which, in its externalization,
reveals itself instead.

CHAPTER 3

The absolute relation

Absolute necessity is not so much the necessary,
even less a necessary, but necessity:
being simply as reflection.
It is relation because it is a distinguishing
whose moments are themselves
the whole totality of necessity,
and therefore subsist absolutely,
but do so in such a way that
their subsisting is one subsistence,
and the difference only the reflective shine of
the movement of exposition,
and this reflective shine is the absolute itself.
Essence as such is reflection or a shining;
as absolute relation, however, essence is the
reflective shine posited as reflective shine,
one which, as such self-referring, is absolute actuality.
The absolute, first expounded by external reflection,
as absolute form or as necessity now expounds itself;
this self-exposition is its self-positing,
and is only this self-positing.
Just as the light of nature is not a something,
nor is it a thing, but its being is rather only its shining,
so manifestation is self-identical absolute actuality.

The sides of the absolute relation
are not, therefore, attributes.
In the attribute the absolute reflectively shines
only in one of its moments,
as in a presupposition that
external reflection has simply assumed.
But the expositor of the absolute is the absolute necessity
which, as self-determining, is identical with itself.
Since this necessity is the reflective shining
posited as reflective shining, the sides of this relation,
because they are as shine, are totalities;
for as shine, the differences are
themselves and their opposite,
that is, they are the whole;
and, conversely, they thus are only shine
because they are totalities.
Thus this distinguishing,
this reflecting shining of the absolute,
is only the identical positing of itself.

This relation in its immediate concept is
the relation of substance and accidents,
the immediate internal disappearing and becoming
of the absolute reflective shine.
If substance determines itself as a being-for-itself over
against an other or is absolute relation as something real,
then we have the relation of causality.
Finally, when this last relation passes over into
reciprocal causality by referring itself to itself,
we then have the absolute relation also posited
in accordance with the determination it contains;
this posited unity of itself in its determinations,
which are posited as the whole itself
and consequently equally as determinations,
is then the concept.
