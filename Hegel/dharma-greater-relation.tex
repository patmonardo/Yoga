A. THE RELATION OF WHOLE AND PARTS

First, the essential relation contains the self-subsistence
of concrete existence reflected into itself;
it is then the simple form whose determinations are
indeed also concrete existences,
but they are posited at the same time,
moments held in the unity.
This self-subsistence reflected into itself is
at the same time reflection into its opposite,
namely the immediate self-subsistence,
and its subsistence is this identity with its opposite
no less than its own self-subsistence.
Second, the other side is
thereby also immediately posited.
This is the immediate self-subsistence
which, determined as the other,
is in itself a multifarious manifold,
but in such a way that this manifold also essentially
has within it the reference of the other side,
the unity of the reflected self-subsistence.
That one side, the whole, is the self-subsistence
that constitutes the world existing in and for itself;
the other side, the parts,
is the immediate concrete existence
which was the world of appearance.
In the relation of whole and parts,
the two sides are these self-subsistences
but in such a way that each has the other
reflectively shining in it
and, at the same time, only is as the identity of both.
Now because the essential relation is
at first only the first, immediate relation,
the negative unity and the positive self-subsistence
are bound together by the “also”;
the two sides are indeed both posited as moments,
but equally so as concretely existing self-subsistences.
Their being posited as moments is henceforth so distributed
that the whole, the reflected self-subsistence, is
as concrete self-existent first,
and the other, the immediate, is in it as a moment.
The whole constitutes here the
unity of the two sides, the substrate,
and the immediate concrete existence is as positedness.
Conversely, on the other side which is the side of the parts,
the immediate and internally manifold concrete existence is
the self-subsistent substrate;
the reflected unity, the whole,
is on the contrary only external reference.

2. This relation thus contains
the self-subsistence of the sides,
and their sublatedness no less,
and the two simply in one reference.
The whole is the self-subsistent;
the parts are only moments of this unity,
but they are also equally self-subsistent
and their reflected unity is only a moment;
and each is, in its self-subsistence,
simply the relative of an other.
This relation is within it, therefore,
immediate contradiction, and it sublates itself.

On closer inspection,
the whole is the reflected unity
that stands independently on its own;
but this subsistence that belongs to it
is equally repelled by it;
it is thus self-externalized;
it has its subsistence in its opposite,
in the manifold immediacy, the parts.
The whole thus consists of the parts,
and apart from them it is not anything.
It is therefore the whole relation
and the self-subsistent totality,
but, for precisely this reason,
it is only a relative,
for what makes it a totality is
rather its other, the parts;
it does not have its subsistence
within it but in its other.

The parts, too, are likewise the whole relation.
They are the immediate as against
the reflected self-subsistence,
and do not subsist in the whole
but are for themselves.
Further, they have this whole within them as their moment;
the whole constitutes their connecting reference;
without the whole there are no parts.
But because they are the self-subsistent,
this connection is only an external moment
with respect to which they are
in and for themselves indifferent.
But at the same time the parts,
as manifold concrete existence, collapse together,
for this concrete existence is reflectionless being;
they have their self-subsistence
only in the reflected unity
which is this unity as well as
the concrete existent manifoldness;
this means that they have
self-subsistence only in the whole,
but this whole is at the same time the
self-subsistence which is the other to the parts.

The whole and the parts thus
reciprocally condition each other;
but the relation here considered is
at the same time higher than the
reference of conditioned and condition
to each other as earlier determined.
Here this reference is realized, that is to say,
it is posited that the condition is
the essential self-subsistence of the conditioned
in such a manner that it is presupposed by the latter.
The condition as such is only the immediate,
and it is only implicitly presupposed.
But the whole, through the condition of the parts,
itself immediately entails that it, too,
is only in so far as it has
the parts for presupposition.
Thus, since both sides of the relation
are posited as conditioning each other reciprocally,
each is on its own an immediate self-subsistence,
but their self-subsistence is equally
mediated or posited through the other.
The whole relation, because of this reciprocity,
is the turning back of the conditioning into itself,
the non-relative, the unconditioned.

Now inasmuch as each side of the relation has
its self-subsistence not in it but in its other,
what we have is only one identity of the two
in which they are both only moments;
but inasmuch as each is self-subsistent on its own,
the two are two self-subsistent concrete existences
indifferent to each other.

In the first respect, that of
the essential identity of the two sides,
the whole is equal to the parts
and the parts are equal to the whole.
Nothing is in the whole which is not in the parts,
and nothing is in the parts which is not in the whole.
The whole is not an abstract unity
but the unity of a diversified manifoldness;
but this unity within which the manifold is
held together is the determinateness
by virtue of which the latter is the parts.
The relation has, therefore, an indivisible identity
and only one self-subsistence.

But further, the whole is equal to the parts
but not to them as parts;
the whole is the reflected unity
whereas the parts constitute
the determinate moment
or the otherness of the unity
and are the diversified manifold.
The whole is not equal to them
as this self-subsistent diversity
but to them together.
But this, their “together,” is
nothing else but their unity,
the whole as such.
In the parts, therefore, the whole
is only equal to itself,
and the equality of it and the parts expresses
only this tautology,
namely that the whole as whole is equal
not to the parts but to the whole.

Conversely, the parts are equal to the whole;
but because, as parts,
they are the moment of otherness,
they are not equal to it as the unity,
but in such a way that one of the whole's
manifold determinations maps over a part,
or that they are equal to the whole as manifold,
and this is to say that they are equal to it
as an apportioned whole, that is, as parts.
Here we thus have the same tautology,
that the parts as parts are equal
not to the whole as such
but, in the whole, to themselves.

The whole and the parts thus
fall indifferently apart;
each side refers only to itself.
But, as so held apart, they destroy themselves.
The whole which is indifferent towards the parts is
abstract identity, undifferentiated in itself.
Identity is a whole only inasmuch
as it is differentiated in itself,
so differentiated indeed that the manifold
determinations are reflected into themselves
and have immediate self-subsistence.
And the identity of reflection has
shown through its movement
that it has this reflection
into its other for its truth.
In just the same way are the parts,
as indifferent to the unity of the whole,
only the unconnected manifold,
the inherently other which, as such,
is the other of itself and only sublates itself.
This self-reference of each of the two sides
is their self-subsistence;
but this self-subsistence which
each side has for itself is rather
the negation of their respective selves.
Each side has its self-subsistence, therefore,
not within but in the other side;
this other, which constitutes the subsistence,
is its presupposed immediate which is supposed
to be the first and its starting point;
but this first of each side is itself
only a first which is not first
but has its beginning in its other.

The truth of the relation consists
therefore in the mediation;
its essence is the negative unity
in which both the reflected
and the existent immediacy
are equally sublated.
The relation is the contradiction
that returns to its ground,
into the unity which,
as turning back,
is reflected unity
but which, since it has
equally posited itself as sublated,
refers to itself negatively
and makes itself into existent immediacy.
But this unity's negative reference,
in so far as it is a first and an immediate,
only is as mediated by its other
and equally as posited.
This other, the existent immediacy,
is equally only as sublated;
its self-subsistence is a first,
but only in order to disappear,
and it has an existence
which is posited and mediated.

Determined in this way, the relation is
no longer one of whole and parts.
The previous immediacy of its sides has
passed over into positedness and mediation.
Each side is posited, in so far as it is immediate,
as self-sublating and as passing over into the other;
and, in so far as it is itself negative reference,
it is at the same time posited as conditioned
through the other, as through its positive.
And the same applies to the immediate transition of each;
it is equally a mediation, a sublating
which is posited through the other.
Thus the relation of whole and parts has passed over
into the relation of force and its expressions.

B. THE RELATION OF FORCE AND ITS EXPRESSION

Force is the negative unity into which
the contradiction of whole and parts has resolved itself;
it is the truth of that first relation.
That of whole and parts is the thoughtless relation
which the understanding first happens to come up with;
or, objectively speaking, it is a dead mechanical aggregate
that indeed has form determinations
and brings the manifoldness of
its self-subsisting matter together into one unity;
but this unity is external to the manifoldness.
But the relation of force is the higher immanent turning back
in which the unity of the whole that made up the connection
of the self-subsisting otherness ceases to be something external
and indifferent to this manifoldness.

In the essential relation as now determined,
the immediate and the reflected self-subsistence are
now posited in that manifoldness as sublated or as moments,
whereas in the preceding relation they were self-subsisting
sides or extremes.

In this there is contained,
first, that the reflected unity
and its immediate existence,
in so far as they are both first and immediate,
sublate themselves and pass over into their other:
the former, force, passes over into its expression,
and what is expressed is a disappearing something
that returns into force as its ground
and only exists as supported and posited by it.

Second, this transition is not only
a becoming and a disappearing
but is rather negative reference to itself;
that is, that which alters its determination is
in this altering reflected-into-itself and preserves itself;
the movement of force is not
as much a transition as a translation,
and in this alteration posited through itself
it remains what it is.

Third, this reflected, self-referring unity is
itself also sublated and a moment;
it is mediated through its other
and it has this as condition;
its negative self-reference,
which is a first
and begins the movement of
the transition out of itself,
has equally a presupposition
by which it is solicited,
and an other from which it begins.

a. The conditionedness of force

Considered in its closer determinations,
force contains, first, the moment of existing immediacy;
it itself is determined over against this immediacy
as negative unity.
But this unity,
in the determination of immediate being,
is an existing something.
This something appears as a first, since as an
immediate it is negative unity;
force, on the contrary,
since it is a reflected something,
appears as positedness
and to this extent as pertaining to
the existing thing or to a matter.
Not that force is the form of this thing
and the thing is determined by it;
on the contrary, the thing is as
an immediate indifferent to it.
As so determined, there is no ground in
the thing for having a force;
force, on the other hand,
since it is the side of positedness,
presupposes the thing essentially.
If it is therefore asked,
how the thing or matter happens to have a force,
the latter appears as externally connected to it
and impressed upon the thing by some alien power.

As this immediate subsistence,
force is a quiescent determinateness of
the thing in general;
not anything that expresses itself
but something immediately external.
Hence force is also designated as matter,
and instead of a magnetic force,
and an electric force,
and other such forces,
a magnetic matter,
an electric matter,
and so on, are assumed;
or again, instead of
the renowned force of attraction,
a fine ether is assumed
that holds everything together.
These are the matters,
which we considered above,
into which the inert, powerless
negative unity of the thing dissolved itself.
But force contains immediate
concrete existence as a moment,
one which, though a condition,
is transient and self-sublating;
it contains it, therefore,
not as a concretely existing thing.
Further, it is not negation as determinateness,
but negative unity reflected into itself.
Consequently, the thing where the force was
supposed to be no longer has any significance here;
the force itself is rather
the positing of the externality
that appears as concrete existence.
It also no longer is, therefore,
merely a determinate matter;
such self-subsistence has long since passed over
into positedness and appearance.

Second, force is the unity of
reflected and immediate subsistence,
or of form-unity and external self-subsistence.
It is both in one;
it is the contact of sides of
which one is in so far as the other is not,
self-identical positive reflection and negated reflection.
Force is thus self-repelling contradiction;
it is active;
or it is self-referring negative unity
in which the reflected immediacy
or the essential in-itselfness is
posited as being only as sublated or as a moment,
and consequently, in so far as it distinguishes itself
from immediate concrete existence,
as passing over into it.
Force, as the determination of
the reflected unity of the whole,
is thus posited as  becoming
concretely existent external manifoldness
from out of itself.

But, third, force is activity at first only
in principle and immediately;
it is reflected unity,
and just as essentially the negation of it;
inasmuch as it differs from this unity,
but is only the identity of itself and its negation,
it essentially refers to this identity as
an immediacy external to it
and one which it has as
presupposition and condition.

Now this presupposition is not
a thing standing over against it;
in force any such indifferent
self-subsistence is sublated;
as the condition of force,
the thing is a self-subsistent other to it.
But because it is not a thing,
and the self-subsistent immediacy has
on the contrary attained here the
determination of self-referring negative unity,
the self-subsistent other is itself a force.
The activity of force is conditioned
through itself as through an other to itself,
through a force.

Accordingly, force is a relation
in which each side is the same as the other.
They are forces that stand in relation,
and refer to each other essentially.
Further, they are different at first only in general;
the unity of their relation
is at first one which is internal and exists only implicitly.
The conditionedness of a force through another force is
thus the doing of the force itself in itself;
that is, the force is at first a positing act as pre-supposing,
an act that only negatively refers to itself;
the other force still lies beyond its positing activity,
namely the reflection that in its determining
immediately returns into itself.

b. The solicitation of force

Force is conditioned because
the moment of immediate concrete existence
which it contains is something only posited,
but, because it is at the same time an immediate,
is posited as something presupposed in which
the force negates itself.
Accordingly, the externality
which is present to force is its
own activity of presupposing posited
at first as another force.
This presupposing is moreover reciprocal.
Each of the two forces contains the unity
reflected into itself as sublated
and is therefore a presupposing;
it posits itself as external;
this moment of externality is its own;
but since it is equally a unity reflected into itself,
it posits that externality at the same time
not within itself but as another force.

But the external as such is self-sublating;
further, the activity that reflects
itself into itself essentially refers to
that externality as to its other,
but equally to it as to something
which is null in itself and identical with it.
Since the presupposing activity is
equally immanent reflection,
it sublates that external negation,
and posits it as something external to it,
or as its externality.
Thus force, as conditioning,
is reciprocally a stimulus for the other force
against which it is active.
The attitude of each force is not one
of passive determination,
as if something other than it were
thereby being elicited in it;
the stimulus rather only solicits it.
The force is within it the negativity of itself,
the repelling of itself from itself is its own positing.
Its act, therefore, consists in sublating
the externality of the stimulus,
reducing it to just a stimulus
and positing it as its own repelling
of itself from itself,
as its own expression.

The force that expresses itself is thus
the same as what was at first a presupposing activity,
that is, one which makes itself external;
but, as self-expressive,
force also negates externality
and posits it as its own activity.
Now in so far as in this examination
we start from force as the negative unity of itself,
and consequently as presupposing reflection,
this is the same as when,
in the expression of force,
we start from the soliciting stimulus.
Thus force is in its concept at first
determined as self-sublating identity,
and in its reality one of the two forces
is determined as soliciting
and the other as being solicited.
But the concept of force is as such
the identity of positing and presupposing reflection,
or of reflected and immediate unity,
and each of these determinations is simply a moment, in unity,
and consequently is as mediated through the other.
But, equally so, there is nothing in the two forces
thus alternately referring to each other that determines
which would be the soliciting and which the solicited,
or rather, both of these form determinations
belong to each in equal manner.
And this identity is not just one of
external comparison but an essential unity of the two.
Thus one force is determined first as soliciting
and the other as being solicited;
these determinations of form appear in this guise as two differences
present in the forces immediately.
But they are essentially mediated.
The one force is solicited;
this stimulus is a determination posited in it from outside.
But the force is itself a presupposing;
it essentially reflects into itself
and sublates the fact that
the stimulus is something external.
That it is solicited is thus its own doing,
or, it is through its own determining
that the other force is an other force in general
and the one soliciting.
The soliciting force refers to the other negatively
and so sublates its externality and is positing;
but it is this positing only on the presupposition
that it has an other over against it;
that is to say, it is itself soliciting only to
the extent that it has an externality in it,
and hence to the extent that it is solicited.
Or it is soliciting only to the extent that
it is solicited to be soliciting.
And so, conversely, the first is solicited only
to the extent that it itself solicits
the other to solicit it, that is, the first force.
Each thus receives the stimulus from the other;
but the stimulus that each delivers as
active consists in receiving
a stimulus from the other;
the stimulus which it receives is solicited by itself.
Both, the given and the received stimulus,
or the active expression and the passive externality,
are each, therefore, nothing immediate but are mediated:
indeed, each force is itself the determinateness
which the other has over against it,
is mediated through this other,
and this mediating other is again
its own determining positing.

This then that a force happens to incur
a stimulus through another force;
that it therefore behaves passively
but then again passes over from
this passivity into activity,
this is the turning back of force into itself.
Force expresses itself.
The external expression is a reaction
in the sense that it posits
the externality as its own moment
and thus sublates its having been
solicited through an other force.
The two are therefore one:
the expression of the force
by virtue of which the latter,
through its negative activity
which is directed at itself,
imparts a determinate being-for-other to itself;
and the infinite turning in
this externality back to itself,
so that there it only refers to itself.
The presupposing reflection,
to which belong the conditionedness and the stimulus,
is therefore immediately also the reflection
that returns into itself,
and the activity is essentially reactive,
against itself.
The positing of the stimulus
or the external is itself
the sublation of it,
and, conversely, the sublation of the stimulus is
the positing of the externality.

c. The infinity of force

Force is finite inasmuch as its moments
still have the form of immediacy.
In this determination its presupposing
and its self-referring reflection are different:
the one appears as an external self-subsisting force
and the other as passively referring to it.
Force is thus still conditioned according to form,
and according to content likewise still restricted,
for a determinateness of form still entails a restriction of content.
But the activity of force consists in expressing itself;
that is, as we have seen, in sublating the externality and
determining it as that in which
it is identical with itself.
What force truly expresses, therefore,
is that its reference to an other
is its reference to itself;
that its passivity consists in its activity.
The stimulus by virtue of which
it is solicited to activity is its own soliciting;
the externality that comes to it is nothing immediate
but something mediated by it,
just as its own essential
self-identity is not immediate
but is mediated by virtue of its negation.
In brief, force expresses this,
that its externality is identical with its inwardness.

C. RELATION OF OUTER AND INNER

1. The relation of whole and parts is the immediate relation;
in it, therefore, reflected and existent immediacy have
a self-subsistence of their own.
But now, since they stand in essential relation,
their self-subsistence is their negative unity,
and this is now posited in the expression of force;
the reflected unity is essentially a becoming-other,
the unity's translation of itself into externality;
but this externality is just as immediately
taken back into that unity;
the difference of the self-subsisting forces sublates itself;
the expression of force is only a mediation
of the reflected unity with itself.
What is present is only an empty
and transparent difference, a reflective shine,
but this shine is the mediation
which is precisely the independent subsistence.
What we have is not just opposite determinations
openly sublating themselves,
and their movement is not only a transition;
rather, what we have is both that
the immediacy from which the start
and the transition into otherness
were made is itself only posited,
and that, consequently, each of the determinations is
already in its immediacy the unity with its other,
so that the transition equally is
a self-positing turning back into itself.

The inner is determined as
the form of reflected immediacy
or of essence over against
the outer as the form of being;
the two, however, are only one identity.
This identity is, first,
the sustaining unity of the two
as substrate replete of content,
or the absolute fact with respect to which
the two determinations are indifferent, external moments.
To this extent, it is content and totality,
a totality which is an inner
that has equally become an outer
but, in this outer, is not something-that-has-become
or something-that-has-been-left-behind but is self-equal.
The outer, in this determination, is not only
equal to the inner according to content
but the two are rather only one fact.
But this fact, as simple identity with itself,
is different from its form determinations,
or these determinations are external to it;
it is itself, therefore, an inner
which is different from its externality.
But this externality consists in the two determinations,
the inner and the outer, both constituting it.
But the fact is itself nothing other
than the unity of the two.
Again, therefore, the two sides are
the same according to content.
But in the fact they are as self-penetrating identity,
as substrate full of content.
But in the externality, as forms of the fact,
they are indifferent to that identity
and consequently each is indifferent to the other.

2. They are in this wise the different form determinations
that have an identical substrate, not in them but in an other.
These are determinations of reflection
which are each for itself:
the inner, as the form of immanent reflection,
the form of essentiality;
the outer, as the form instead of
immediacy reflected into an other,
or the form of unessentiality.
But the nature of relation has shown that
these determinations constitute just one identity alone.
In its expression force is a determining
which is one and the same as presupposing
and as returning into itself.
Inasmuch as the inner and the outer are
considered as determinations of form,
they are, therefore,
first, only the simple form itself,
and, second, because in this form they are
at the same time determined as opposite,
their unity is the pure abstract determination
in which the one is immediately the other,
and is this other because it is the one that it is.
Thus the inner is immediately only the outer,
and it is this determinateness of externality
for the reason that it is the inner;
conversely, the outer is only an inner
because it is only an outer.
In other words, since the unity of form
holds its two determinations as opposites,
their identity is only this transition,
and is in this transition only the other of both,
not their identity replete with content.
Or this holding fast to form is in general
the side of determinateness.
What is determined according to this side is
not the real totality of the whole
but the totality or the fact itself
only in the determinacy of form;
since this unity is simply the coincidence
of two opposed determinations,
then when one of them is taken first
(it is indifferent which),
it must be said of the substrate or the fact
that it is for this reason just as essentially
in the other determinateness,
but also only in the other,
just as it was first said
that it is only in the first.

Thus something which is at first only an inner,
is for just that reason only an outer.
Or conversely something which is only an outer,
is for that reason only an inner.
Or if the inner is determined as essence
but the outer as being,
then inasmuch as a fact is only in its essence,
it is for that very reason only an immediate being;
or a fact which only is, is for that very reason
as yet only in its essence.
Outer and inner are determinateness
so posited that each, as a determination,
not only presupposes the other
and passes over into it as its truth,
but, in being this truth of the other,
remains posited as determinateness
and points to the totality of both.
The inner is thus the completion
of essence according to form.
For in being determined as inner,
essence implies that it is deficient
and that it is only with reference to
its other, the outer;
but this other is not just being,
or even concrete existence,
but is the reference to essence or the inner.
What we have here is not just
the reference of the two to each other,
but the determining element of absolute form,
namely that each term is immediately its opposite,
and each is their common reference
to a third or rather to their unity.
Their mediation, however, still misses
this identical substrate that contains them both;
their reference is for this reason
the immediate conversion of the one into the other,
and this negative unity tying them together is
the simple point empty of content.

3. The first of the identities considered,
the identity of inner and outer,
is the substrate which is indifferent
to the difference of these determinations
as to a form external to it,
or the identity is as content.
The second is the unmediated identity of their difference,
the immediate conversion of each into its opposite,
or it is inner and outer as pure form.
But both these identities are only
the sides of one totality,
or the totality itself is only the
conversion of the one identity into the other.
The totality, as substrate and content,
is this immediacy reflected into itself
only through the presupposing reflection
of form that sublates their difference
and posits itself as indifferent identity,
as reflected unity over against it.
Or again, the content is the form itself
in so far as the latter determines itself as difference
and makes itself into one side of this difference as externality,
but into the other side as an immediacy
which is reflected into itself,
or into an inner.

It follows that, conversely,
the differences of form,
the inner and the outer,
are each posited as the totality
within it of itself and its other;
the inner, as simple identity
reflected into itself, is immediacy
and hence, no less than essence,
being and externality;
and the external,
as the manifold and determined being,
is only external,
that is, is posited as unessential
and as having returned into its ground,
therefore as inner.
This transition of each into the other is
their immediate identity, as substrate,
but also their mediated identity, that is,
each is what it is in itself,
the totality of the relation,
precisely through its other.
Or, conversely, the determinateness of
either side is mediated through
the determinateness of the other
because each is in itself the totality;
the totality thus mediates itself with itself
through the form or the determinateness,
and the determinateness mediates
itself with itself through its simple identity.

Therefore, what something is, that it is
entirely in its externality;
its externality is its totality
and equally so its unity reflected into itself.
Its appearance is not only reflection-into-other
but immanent reflection,
and its externality is therefore
the expression of what it is in itself;
and since its content and its form
are thus absolutely identical,
it is, in and for itself, nothing but this:
to express itself.
It is the revealing of its essence,
and this essence, accordingly,
consists simply in being self-revealing.

The essential relation, in this identity
of appearance with the inner or with essence,
has determined itself as actuality.
