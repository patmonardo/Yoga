
DIVISION

The concept, as considered so far,
has demonstrated itself to be
the unity of being and essence.
Essence is the first negation of being,
which has thereby become reflective shine;
the concept is the second negation,
or the negation of this negation,
and is therefore being
which has been restored once more,
but as in itself the infinite mediation
and negation of being.
In the concept, therefore,
being and essence no longer have
determination as being and essence,
nor are they only in such a unity
in which each would reflectively shine in the other.
Consequently, the concept does not differentiate
itself into these determinations.
The concept is the truth of the substantial relation
in which being and essence attain their perfect
self-subsistence and determination each through the other.
The truth of substantiality proved
to be the substantial identity,
an identity that equally is,
and only is, positedness.
Positedness is determinate existence and differentiation;
in the concept, therefore, being-in-and-for-itself
has attained a true existence adequate to it,
for that positedness is itself being-in-and-for-itself.
This positedness constitutes the difference
of the concept in the concept itself;
and because the concept is
immediately being-in-and-for-itself,
its differences are themselves the whole concept,
universal in their determinateness
and identical in their negation.

This is now the concept itself of the concept,
but at first only the concept of the concept
or also itself only concept.
Since the concept is being-in-and-for-itself
by being a positedness, or is absolute substance,
and substance manifests the necessity of
distinct substances as an identity,
this identity must itself posit what it is.
The moments of the movement of the substantial relation
through which the concept came to be
and the reality thereby exhibited are
only in the transition to the concept;
that reality is not yet the
concept's own determination,
one that has emerged out of it;
it fell in the sphere of necessity
whereas the reality of the concept
can only be its free determination,
a determinate existence in which
the concept is identical with itself
and whose moments are themselves concepts
posited through the concept itself.

YS III.1

    desa-bandha cittasya dharana

At first, therefore, the concept is
only implicitly the truth;
because it is only something inner,
it is equally only something outer.
It is at first simply an immediate
and in this shape its moments have
the form of immediate, fixed determinations.
It appears as the determinate concept,
as the sphere of mere understanding.
Because this form of immediacy is an existence
still inadequate to the nature of the concept,
for the concept is free and only refers to itself,
it is an external form in which the concept
does not exist in-and-for-itself,
but can only count as something posited or subjective.
The shape of the immediate concept
constitutes the standpoint that makes
of the concept a subjective thinking,
a reflection external to the subject matter.
This stage constitutes, therefore, subjectivity,
or the formal concept.
Its externality is manifested in
the fixed being of its determinations
that makes them come up each by itself,
isolated and qualitative,
and each only externally referred to the other.
But the identity of the concept,
which is precisely their inner or subjective essence,
sets them in dialectical movement,
and through this movement their singleness is sublated
and with it also the separation of
the concept from the subject matter,
and what emerges as their truth is
the totality which is the objective concept.

YS III.2

    tatra pratyaya-eka-tanata dhyanam

Second, in its objectivity the concept is
the fact itself as it exists in-and-for-itself.
The formal concept makes itself into the fact
by virtue of the necessary determination of its form,
and it thereby sheds the relation
of subjectivity and externality
that it had to that matter.
Or, conversely, objectivity is the real concept
that has emerged from its inwardness
and has passed over into existence.
In this identity with the fact,
the concept thus has an existence
which is its own and free.
But this existence is still a freedom
which is immediate and not yet negative.
Being at one with the subject matter,
the concept is submerged into it;
its differences are objective
determinations of existence
in which it is itself again the inner.
As the soul of objective existence,
the concept must give itself the form of subjectivity
that it immediately had as formal concept;
and so, in the form of the free concept
which in objectivity it still lacked,
it steps forth over against that objectivity
and, over against it, it makes therein the identity with it,
which as objective concept it has in and for itself,
into an identity that is also posited.

YS III.3

    tad evartha-matra-nirbhasa svarupa-sunyam iva samadhi

In this consummation in which
the concept has the form of freedom
even in its objectivity,
the adequate concept is the idea.
Reason, which is the sphere of the idea,
is the self-unveiled truth
in which the concept attains
the realization absolutely adequate to it,
and is free inasmuch as in this real world,
in its objectivity, it recognizes its subjectivity,
and in this subjectivity recognizes that objective world.

SECTION I

Subjectivity

YS III.4

    trayam ekatra samyama

YS III.5

    taj-jayat prajna-aloka

YS III.6

    tasya bhumisu viniyoga

YS III.7

    trayam antar-angam purvebhya

YS III.8

    tad api bahir-angam nirbijasya

The concept is, to start with, formal,
the concept in its beginning
or as the immediate concept.
In this immediate unity,
its difference or its positedness
is, first, itself initially simple
and only a reflective shine,
so that the moments of the difference
are immediately the totality of the concept
and only the concept as such.

But, second, because it is absolute negativity,
the concept divides and posits itself
as the negative or the other of itself;
yet, because it is still immediate concept,
this positing or this differentiation is
characterized by the reciprocal
indifference of its moments,
each of which comes to be on its own;
in this division the unity of the concept is
still only an external connection.
Thus, as the connection of its moments
posited as self-subsisting and indifferent,
the concept is judgment.

Third, although the judgment contains
the unity of the concept that has been lost
in its self-subsisting moments,
this unity is not posited.
It will become posited by virtue of
the dialectical movement of the judgment
which, through this movement,
becomes syllogistic inference,
and this is the fully posited concept,
for in the inference the moments of
the concept as self-subsisting extremes
and their mediating unity are both equally posited.

But since this unity itself, as unifying middle,
and the moments, as self-subsisting extremes,
stand at first immediately opposite one another,
this contradictory relation that occurs
in the formal inference sublates itself,
and the completeness of the concept passes over
into the unity of totality;
the subjectivity of the concept
into its objectivity.

CHAPTER 1

The concept

YS III.9

    vyutthana-nirodha-samskarayor abhibhava-pradur-bhavau
    nirodha-kshana-cittanvayo nirodha-parinama

    The Concept as such

    Manas, the understanding, first the faculty
    for the cognition of the general (of rules)

YS III.10

    tasya prasanta-vahita samskarat

The faculty of concepts is normally
associated with the understanding,
and the latter is accordingly distinguished
from the faculty of judgment
and from the faculty of syllogistic inferences
which is formal reason.
But it is particularly with reason
that the understanding is contrasted,
and it signifies then, not the faculty of concepts in general,
but the faculty of determinate concepts,
as if, as the prevailing opinion has it,
the concept were only a determinate.
When distinguished in this meaning
from the formal faculty of judgment and from formal reason,
the understanding is accordingly to be taken
as the faculty of the single determinate concept.
For the judgment and the syllogism or reason, as formal,
are themselves only a thing of the understanding,
since they are subsumed under the form
of the abstract determinateness of the concept.
Here, however, we are definitely not taking
the concept as just abstractly determined;
the understanding is therefore
to be distinguished from reason only
in that it is the faculty of the concept as such.

This universal concept that we now have to consider
contains the three moments of
universality, particularity, and singularity.
The difference and the determinations which the concept
gives itself in its process of distinguishing constitute
the sides formerly called positedness.
Since this positedness is in the concept
identical with being-in-and-for-itself,
each of the moments is just as much
the whole concept as it is determinate concept
and a determination of the concept.

It is at first pure concept,
or the determination of universality.
But the pure or universal concept is also
only a determinate or particular concept
that takes its place alongside the other concepts.
Because the concept is a totality,
and therefore in its universality
or pure identical self-reference
is essentially a determining and a distinguishing,
it possesses in itself the norm
by which this form of its self-identity,
in pervading all the moments
and comprehending them within,
equally determines itself immediately
as being only the universal
as against the distinctness of the moments.

Second, the concept is thereby posited
as this particular or determinate concept,
distinct from others.

Third, singularity is the concept reflecting itself
out of difference into absolute negativity.
This is at the same time the moment at which
it has stepped out of its identity
into its otherness and becomes judgment.

CHAPTER 2

Judgment

YS III.11

    sarva-arthata-ekagratayo kshayodayau cittasya samadhi-parinama

    The logical form of all judgments consists of the objective unity
    of the apperception of the concepts contained therein

    Judgment

    Ahamkara, the determinative power of judgment,
    second the faculty for the subsumption of the particular under the general

Judgment is the determinateness of the concept
posited in the concept itself.
The determinations of the concept,
or, what amounts to the same thing as shown,
the determinate concepts,
have already been considered on their own;
but this consideration was rather
a subjective reflection
or a subjective abstraction.
But the concept is itself this act of abstracting;
the positioning of its determinations over
against each other is its own determining.
Judgment is this positing of the determinate concepts
through the concept itself.

First, as immediate, judgment is the judgment of existence;
its subject is immediately an abstract, existent singular,
and the predicate is an immediate determinateness or property of it,
an abstract universal.

Second, as this qualitative character of
the subject and predicate is sublated,
the determination of the one begins
to shine reflectively in the other;
the judgment is now the judgement of reflection.

But this external combination passes over
into the essential identity of a substantial, necessary combination;
and so we have, third, the judgment of necessity.

Fourth, since in this essential identity
the difference of subject and predicate has become a form,
the judgment becomes subjective;
it entails the opposition of the concept and its reality
and the comparison of the two;
it is the judgment of the concept.

This emergence of the concept grounds
the transition of judgment into syllogistic inference.

CHAPTER 3

The syllogism

YS III.12

    tata punashantoditau tulya-pratyayau cittasya-ekagrata-parinama

    Syllogism

    Buddhi, reason, third the faculty for the determination
    of the particular from the general (for the derivation from principles)

The syllogism is the result of
the restoration of the concept in the judgment,
and consequently the unity and the truth of the two.
The concept as such holds its moments
sublated in this unity;
in judgment, the unity is an internal
or, what amounts to the same, an external one,
and although the moments are connected,
they are posited as self-subsisting extremes.
In the syllogism, the determinations of the concept
are like the extremes of the judgment,
and at the same time their determinate unity is posited.

First, the syllogism of existence,
in which the terms are thus
immediately and abstractly determined,
demonstrates internally that,
since like judgment it is
the connection of those terms,
these are not in fact abstract
but each contains in it the reference
connecting it to the others,
and the determination of the middle term is
not just a determinateness opposed to the
determinations of the extremes
but contains these extremes posited in it.

Through this dialectic,
the syllogism of existence becomes
the syllogism of reflection, the second syllogism.
Its terms are such that in each the other
shines essentially reflected in it,
or are posited as mediated,
as they are indeed supposed to be
in accordance with the nature of
syllogistic inference in general.

Third, inasmuch as this reflective shining
or this mediatedness is reflected into itself,
syllogism is determined as the syllogism of necessity,
one in which the mediating factor is
the objective nature of the fact.
As this syllogism determines the extremes
of the concept also as totalities,
it has attained the correspondence
of its concept (or the middle term)
and its existence (or the difference of the extremes).
It has attained its truth;
and with that it has stepped forth
out of subjectivity into objectivity.

SECTION II

Objectivity

In Book One of the Objective Logic,
abstract being was presented as
passing over into existence,
but at the same time as
retreating into essence.

In Book Two, essence shows itself as
determining itself as ground,
thereby stepping into concrete existence
and realizing itself as substance,
but at the same time
retreating into the concept.

Of the concept, we have now first shown
that it determines itself as objectivity.
It should be obvious that this latter transition is
essentially the same as the proof from the concept,
that is to say, from the concept of God to his existence,
that was formerly found in Metaphysics,
or the so-called ontological proof.
Equally well known is that
Descartes's sublimest thought,
that God is that whose concept
includes his being within itself,
after having degenerated into
the bad form of the formal syllogism,
namely into the form of the said proof,
finally succumbed to the Critique of Reason
and to the thought that existence
cannot be extracted from the concept.
Some elucidations concerning this proof
have already been made earlier.
In Volume I, pp. 47 ff.,
where being has vanished into
its closest opposite, non-being,
and becoming has shown itself
to be the truth of both,
attention was called to the
confusion that arises in the case
of a determinate existence
when we concentrate, not on its being,
but on its determinate content,
and then imagine if we compare this determinate content
(e.g. one hundred dollars) with another determinate content
(e.g. the context of my perception, of my financial situation)
and discover that it makes indeed a difference
whether the one content is added to the other or not;
that we are dealing with the distinction of being and non-being,
or even the distinction of being and the concept.
Further, in the same Volume
on pp. 64ff. and on p. 289 of Volume II,
the definition of a sum-total of all reality
which occurs in the ontological proof was elucidated.
But the essential subject matter of that proof,
the connectedness of concept and existence,
is the concern of the treatment of the concept just concluded
and of the entire course that the latter traverses
in determining itself to objectivity.
The concept, as absolutely self-identical negativity,
is self-determining;
it was noted that the concept,
in resolving itself into judgment in singularity,
already posits itself as something real, an existent;
this still abstract reality completes itself in objectivity.

Now it might appear that the transition
from the concept into objectivity
is quite another thing than the transition
from the concept of God to God's existence.
But, on the one hand, it must be borne in mind
that the determinate content, God,
makes no difference in a logical progression,
and that the ontological proof is
only one application of this logical progression
to that particular content.
On the other hand, it is essential to be
reminded of the remark made above that
the subject obtains determinateness
and content only in its predicate;
that prior to the predicate,
whatever that content might otherwise be for
feeling, intuition, and representation,
so far as conceptual cognition is concerned
it is only a name;
but in the predicate, with determinateness,
there begins at the same time
the process of realization in general.
The predicates, however, must be grasped as
themselves still confined within the concept,
hence as something subjective with which
no move to existence has yet been made;
even for this reason, in judgment the realization of
the concept is certainly not completed yet.
But there is the further reason that
the mere determination of a subject matter through predicates,
without this determination being at the same time
the realization and objectification of the concept,
remains something so subjective that it is not even
a true cognition and determination of the
concept of the subject matter;
“subjective” in the sense of abstract reflection
and non-conceptual representation.
God as living God, and better still as absolute spirit,
is only recognized in what he does.
Humankind were directed early to recognize God in his works;
only from these can the determinations proceed
that can be called his properties,
and in which his being is also contained.
It is thus the conceptual comprehension of God's activity,
that is to say, of God himself,
that recognizes the concept of God in his being
and his being in his concept.
Being by itself, or even existence,
are such a poor and restricted determination,
that the difficulty of finding them in the concept
may well be due to not having considered
what being or existence themselves are.
Being as entirely abstract, immediate self-reference,
is nothing but the abstract moment of the concept;
it is its moment of abstract universality
that also provides what is required of being,
namely that it be outside the concept,
for inasmuch as universality is a moment of the concept,
it is also its difference or the abstract judgment
wherein the concept opposes itself to itself.
The concept, even as formal, already immediately contains
being in a truer and richer form,
in that, as self-referring negativity,
it is singularity.

But of course the difficulty of finding
being in the concept in general,
and equally so in the concept of God,
becomes insuperable if we expect
being to be something that
we find in the context of external experience
or in the form of sense-perception,
like the one hundred dollars
in the context of my finances,
as something graspable only by hand,
not by spirit, essentially visible to
the external and not the internal eye;
in other words, if the name of
being, reality, truth, is given
to that which things possess
as sensuous, temporal, and perishable.
The consequence of a philosophizing
that in regard to being fails
to rise above the senses is
that, in regard to the concept,
it also fails to let go of
merely abstract thought;
such thought stands opposed to being.

The customary practice of regarding
the concept as something just as one-sided
as abstract thought will already stand in the way
of accepting what has just been suggested,
namely, that we regard the transition of
the concept of God to his being as
an application of the logical course
of objectification of the concept presented above.
Yet if it is granted, as it commonly is,
that the logical element, as the formal element,
constitutes the form for the cognition
of every determinate content,
then that application at least
would have to be conceded,
unless even at the opposition of
concept and objectivity in general
one stops short at the untrue concept
and an equally untrue reality as an ultimate.
But in the exposition of the pure concept
it was further indicated that the latter is
the absolute divine concept itself.
In truth, therefore, what takes place is
not a relation of application
but the immediate display in the logical course
of God's self-determination as being.
But on this point it is to be remarked
that inasmuch as the concept is to be
presented as the concept of God,
it ought be apprehended as it is
when already taken up in the idea.
The said pure concept passes through
the finite forms of the judgment
and the syllogism precisely
because it is not yet posited
in and for itself as one with objectivity,
but is conceived rather only in
the process of becoming that objectivity.
The latter, too, is not yet the divine concrete existence,
not yet the reality reflectively shining in the idea.
And yet objectivity is just that much richer and higher
than the being or existence of the ontological proof,
as the pure concept is richer and
higher than that metaphysical vacuum
of the sum-total of all reality.
But I reserve for another occasion
the task of elucidating in greater detail
the manifold misunderstanding brought
upon the ontological proof of God's existence,
and also on the rest of the other so-called proofs,
by logical formalism.
We shall also elucidate Kant's critique of such proofs
in order to establish their true meaning
and thus restore the thoughts on which they are based
to their worth and dignity.

We have previously called attention to
the several forms of immediacy
that have already come on the scene,
but in different determinations.
In the sphere of being, immediacy is
being itself and existence;
in the sphere of essence,
it is concrete existence
and then actuality and substantiality;
in the sphere of the concept,
besides being immediacy as abstract universality,
it is now objectivity.
These expressions, when the exactitude
of philosophical conceptual distinctions is
not at stake, may be used as synonymous;
but the determinations are derived
from the necessity of the concept.
Being is as such the first immediacy,
and existence is the same immediacy
with a first determinateness.
Concrete existence, along with the thing,
is the immediacy that proceeds from ground,
from the self-sublating mediation
of the simple reflection of essence.
But actuality and substantiality are
the immediacy that proceeds from
the sublated difference of the still unessential
concrete existence as appearance and its essentiality.
Finally, objectivity is the immediacy
as which the concept has determined itself
by the sublation of its abstraction and mediation.
It is the privilege of philosophy to choose
such expressions from the language of ordinary life,
which is made for the world of imaginary representations,
as seem to approximate the determinations of the concept.
There is no question of demonstrating
for a word chosen from ordinary life
that in ordinary life too the same concept is associated
with that for which philosophy uses it,
for ordinary life has no concepts,
only representations of the imagination,
and to recognize the concept in what is otherwise
mere representation is philosophy itself.
It must therefore suffice if representation,
for those of its expressions
that philosophy uses for its definitions,
has only some rough approximation
of their distinctive difference;
it may also be the case that in these expressions
one recognizes pictorial adumbrations
which, as approximations, are close indeed to the
corresponding concepts.
One will be hard pressed, perhaps,
to concede that something can be
without actually existing;
but at least nobody will mistake, for instance,
being as the copula of the judgment
for the expression “to exist actually,”
and nobody will say that
“this article exists dear, suitable, etc.,”
“gold exists a metal or metallic,”
instead of “this article is dear, suitable, etc.,”
“gold is a metal.”
And surely it is common to distinguish
being from appearing,
appearance from actuality,
as also being as contrasted to actuality,
and still more all these expressions from objectivity.
But even if such expressions were used synonymously,
philosophy would in any case have the freedom
to take advantage of such empty superfluity of language
for the purpose of its distinctions.

Mention was made in connection with the apodictic judgment
where judgment attains completion
and the subject thus loses its determinateness
as against the predicate,
of the double meaning of subjectivity originating from it,
namely the subjectivity of the concept
and equally so of the externality
and contingency confronting the concept.
A similar objectivity also appears for the double meaning,
of standing opposed to the self-subsistent concept
yet of also existing in and for itself.
In the former sense, the object stands opposed
to the “I = I” which in subjective idealism
is declared to be the absolute truth.
It is then the manifold world in its immediate existence
with which the “I” or the concept is engaged in endless struggle,
in order, by the negation of the inherently nullity of this other,
to give to its first certainty of being a self,
the actual truth of its equality with itself.
In a broader sense, it means a subject matter in general
for whatever interest or activity of the subject.

In the opposite sense, however,
the objective signifies that which exists in and for itself,
without restriction and opposition.
Rational principles, perfect works of art, etc.,
are said to be objective to the extent
that they are free and above every accidentality.
Although rational principles, whether theoretical or ethical,
only belong to the sphere of the subjective, to consciousness,
this aspect of the latter of existing in and for itself
is nonetheless called objective;
the cognition of truth is made to rest on
the cognition of the object as free of
any addition by subjective reflection,
and right conduct on the adherence to objective laws,
such as are not of subjective origin
and are immune to arbitrariness
and to treatment that would compromise their necessity.

At the present standpoint of our treatise,
objectivity has the meaning first of all of
the being in and for itself of the concept
that has sublated the mediation posited
in its self-determination,
raising it to immediate self-reference.
This immediacy is therefore itself
immediately and entirely pervaded by the concept,
just as its totality is immediately identical with its being.
But further, since the concept equally has to restore
the free being-for-itself of its subjectivity,
it enters with respect to objectivity
into a relation of purpose
in which the immediacy of the objectivity
becomes a negative for it,
something to be determined through its activity.
This immediacy thus acquires the other significance,
namely that in and for itself,
in so far as it stands opposed to the concept,
it is a nullity.

First, then, objectivity is in its immediacy.
Its moments, on account of the totality of all moments,
stand in self-subsistent indifference
as objects each outside the other,
and as so related they possess
the subjective unity of the concept
only as inner or as outer.
This is mechanism.

But, second, inasmuch as in mechanism that unity
reveals itself to be the immanent law of the objects,
their relation becomes one of non-indifference,
each specifically different according to law;
a connection in which the objects'
determinate self-subsistence is sublated.
This is chemism.

Third, this essential unity of the objects is
thereby posited as distinct from their self-subsistence.
It is the subjective concept,
but posited as referring in and for itself
to the objectivity, as purpose.
This is teleology.

Since purpose is the concept posited
as within it referring to objectivity,
and through itself sublating its defect
of being subjective,
the at first external purposiveness becomes,
through the realization of the purpose, internal.
It becomes idea.

CHAPTER 1

Mechanism

YS III.13

    etena bhutendriyesu dharma-laksana-vastha-parinama vyakhyata

    Mechanism

Since objectivity is the totality of the concept
that has returned into its unity,
an immediate is thereby posited
which is in and for itself that totality,
and is also posited as such,
but in it the negativity of the concept has as yet
not detached itself from the immediacy of the totality;
in other words, the objectivity is not yet posited as judgment.
In so far as it has the concept immanent in it,
the difference of the concept is present in it;
but on account of the objective totality,
the differentiated moments are
complete and self-subsistent objects
that, consequently, even in connection
relate to one another as each standing on its own,
each maintaining itself in every combination as external.
This is what constitutes the character of mechanism,
namely, that whatever the connection that
obtains between the things combined,
the connection remains one that is alien to them,
that does not affect their nature,
and even when a reflective semblance
of unity is associated with it,
the connection remains nothing more than
composition, mixture, aggregate, etc.
Spiritual mechanism, like its material counterpart,
also consists in the things connected in the spirit
remaining external to one another and to spirit.
A mechanical mode of representation,
a mechanical memory, a habit, a mechanical mode of acting,
mean that the pervasive presence that is proper to spirit
is lacking in what spirit grasps or does.
Although its theoretical or practical mechanism
cannot take place without its spontaneous activity,
without an impulse and consciousness,
the freedom of individuality is still lacking in it,
and since this freedom does not appear in it,
the mechanical act appears as a merely external one.

CHAPTER 2

Chemism

YS III.14

    shantoditavyapadesya-dharmanupati dharmi

In objectivity as a whole
chemism constitutes the moment of judgment,
of the difference that has become objective,
and of process.

Since it already begins with
determinateness and positedness,
and the chemical object is
at the same time objective totality,
the course it follows next is
simple and perfectly determined
by its presupposition.

CHAPTER 3

Teleology

YS III.15

    kramanyatvam parinamanyatve hetu

Where there is the perception of a purposiveness,
an intelligence is assumed as its author;
required for purpose is thus the concept's
own free concrete existence.
Teleology is above all contrasted with mechanism,
in which the determinateness posited in the object,
being external, is one that gives no sign of self-determination.
The opposition between causæ efficientes and causæ finales,
between merely efficient and final causes,
refers to this distinction, just as,
at a more concrete level, the enquiry whether the absolute essence
of the world is to be conceived as blind mechanism
or as an intelligence that determines itself
in accordance with purposes also comes down to it.
The antinomy of fatalism, along with determinism,
and freedom is equally concerned with
the opposition of mechanism and teleology;
for the free is the concept in its concrete existence.

Earlier metaphysics has dealt with these concepts
as it dealt with others.
It presupposed a certain picture of the world
and strived to show that one or the other concept
of causality was adequate to it,
and the opposite defective because
not explainable from the presupposed picture,
all the while not examining the concept of
mechanical cause and that of purpose to see
which possesses truth in and for itself.
If this is established independently, it may turn out
that the objective world exhibits mechanical and final causes;
its actual existence is not the norm of what is true,
but what is true is rather the criterion for deciding
which of these concrete existences is its true one.
Just as the subjective understanding exhibits also errors in it,
so the objective world exhibits also aspects and stages of truth
that by themselves are still one-sided, incomplete,
and only relations of appearances.
If mechanism and purposiveness stand opposed to each other,
then by that very fact they cannot
be taken as indifferent concepts,
as if each were by itself a correct concept
and had as much validity as the other,
the only question being where
the one or the other may apply.
This equal validity of the two rests
only on the fact that they are,
that is to say, that we have them both.
But since they do stand opposed,
the necessary first question is,
which of the two concepts is the true one;
and the higher and truly telling question is,
whether there is a third which is their truth,
or whether one of them is the truth of the other.
But purposive connection has proved
to be the truth of mechanism.
Regarding chemism, what came under it
can be taken together with mechanism,
for purpose is the concept in free concrete existence,
and the concept's state of unfreedom,
its being sunk into externality,
stands opposed to it in any form.
Both, mechanism as well as chemism, are
therefore included under natural necessity:
mechanism, because in it the concept
does not exist in the object concretely,
for as mechanical the latter lacks self-determination;
chemism, either because the concept has in it
a one-sided concrete existence in a state of tension,
or because, emerging as the unity that creates
in the neutral object a tension of extremes,
it is external to itself in so far as it sublates this divide.

The closer the teleological principle is associated
with the concept of an extra-mundane intelligence,
and the more it has therefore enjoyed the favor of piety,
all the more it has seemed to depart from
the true investigation of nature,
which aims at a cognition of the properties of nature
not as extraneous, but as immanent determinacies,
and accepts only such cognition
as a valid conceptual comprehension.
Since purpose is the concept itself in its concrete existence,
it may seem strange that a cognition of objects
based on their concept rather appears as
an unjustified trespass into a heterogeneous element,
whereas mechanism, for which the determinateness
of an object is posited in it externally and by an other,
is accepted as a more immanent view of things than teleology.
Of course mechanism, at least the ordinary unfree mechanism,
and chemism as well, must be regarded as an immanent principle
in so far as the externally determining object
is itself again just another such object,
externally determined and indifferent to its being determined,
or, in the case of chemism, in so far as the other object
must likewise be one that is chemically determined;
in general, in so far as an essential moment
of the totality always lies in something external.
These principles remain confined, therefore,
within the same natural form of finitude;
but although they do not wish to transcend the finite
and, as regards appearances, lead only to finite causes
that themselves demand further causes,
they nonetheless equally expand themselves,
partly into a formal totality in the concept
of force, cause, or of such determinations of reflection
that are supposed to signify originariness, and partly,
through the medium of abstract universality,
also into a sum total of forces,
a whole of reciprocal causes.
Mechanism thus reveals itself to be a striving for totality
by the very fact that it seeks to comprehend nature by itself
as a whole that has no need of an other for its concept;
a totality that is not found in purpose
and the extra-mundane intelligence associated with it.

Now purposiveness presents itself from the first
as something of a generally higher nature,
as an intelligence that externally determines
the manifoldness of objects through a unity
that exists in and for itself,
so that the indifferent determinacies of the objects
become essential by virtue of this connection.
In mechanism they become so through the mere form of necessity
that leaves their content indifferent,
for they are supposed to remain external
and only the understanding as such is
expected to find satisfaction by recognizing
its principle of union, the abstract identity.
In teleology, on the contrary, the content becomes important,
for teleology presupposes a concept,
something determined in and for itself
and consequently self-determining,
and has therefore extracted from the connection of
differences and their reciprocal determinateness, from the form,
a unity that is reflected into itself,
something that is determined in and for itself
and is consequently a content.
But if this content is otherwise finite and insignificant,
then it contradicts what it is supposed to be,
for according to its form purpose is
a totality infinite within itself;
especially when the activity operating in accordance
with it is assumed to be an absolute will and intelligence.
For this reason has teleology drawn the
reproach of triviality so much upon itself,
for the purposes that it has espoused are,
as the case may be, more important or more trivial [than the content],
and it was inevitable that the connection of purposiveness
in objects would so often appear just a frivolity,
since it appears external and therefore contingent.
Mechanism, on the contrary, leaves to the determinacies of the objects,
as regards their content, their status as accidents indifferent to the object,
and these determinacies are not supposed to have,
whether for the objects or the subjective understanding,
any value higher than that.
This principle, combined with external necessity,
yields therefore a consciousness of infinite freedom
that contrasts with teleology,
which sets up as something absolute bits of its content
that are trivial and even contemptible,
where the more universal thought can only
find itself infinitely constricted,
even to the point of feeling disgust.

The formal disadvantage from which
this teleology immediately suffers
is that it only goes as far as external purposiveness.
The content of concept, since the latter is
thereby posited as something formal,
is for teleology also externally given to it
in the manifoldness of the objective world
in those very determinacies that
are also the content of mechanism,
but are there as something external and accidental.
Because of this commonality of content,
only the form of purposiveness constitutes by itself
the essential element of the teleological.
In this respect, without as yet considering
the distinction between external and internal purposiveness,
the connection of purpose in general has
proven itself to be the truth of mechanism.
Teleology possesses in general the higher principle,
the concept in its concrete existence,
which is in and for itself the infinite and absolute;
a principle of freedom which, utterly certain of its self-determination,
is absolutely withdrawn from the external determining of mechanism.

One of Kant's greatest services to philosophy was
in drawing the distinction between relative or
external purposiveness and internal purposiveness;
in the latter he opened up the concept of life, the idea,
and with that he positively raised philosophy
above the determinations of reflection
and the relative world of metaphysics,
something that the Critique of Reason does
only imperfectly, ambiguously, and only negatively.
We have remarked that the opposition of teleology and mechanism is
first of all the general opposition of freedom and necessity.
Kant treated the opposition in this form, among the antinomies of reason,
namely, as the third conflict of the transcendental ideas.
I cite his exposition, to which reference was made earlier,
very briefly because its essential point is so simple
that it does not need extensive explanation,
and moreover, the peculiarities of Kant's antinomies
have been elucidated in greater detail elsewhere.

The thesis of the antinomy now in question runs thus:
Causality according to the laws of nature is
not the only one from which the appearance of
the world can exhaustively be derived.
For their explanation, it is necessary to
assume yet another causality through freedom.

The antithesis: There is no freedom,
but everything in the world happens
solely according to laws of nature.

As in the other antinomies, the proof starts off apagogically
by assuming the opposite of each thesis;
secondly, in order to show the contradiction of this assumption,
its opposite (which is then the proposition to be proved)
is assumed in turn and presupposed as valid.
This whole roundabout proof could therefore be spared,
for the proof consists in nothing but the assertoric
assertion of the two opposite propositions.

Thus to prove the thesis we should first assume
that there is no other causality than
that according to the laws of nature, that is,
according to the necessity of mechanism in general,
chemism being included.
This proposition contradicts itself,
because the law of nature consists just in this,
that nothing happens without a cause sufficiently determined a priori,
a cause that would have to contain an absolute spontaneity within it,
that is, the assumption opposed to the thesis is contradictory,
for the reason that it contradicts the thesis.

In support of the proof of the antithesis
we should assume that there is a freedom
as a particular kind of causality
for absolutely initiating a situation,
and together with it also a series of
consequences following upon it.
But now, since such a beginning presupposes
a situation that has no causal link with the one preceding it,
it contradicts the law of causality that alone makes the unity
of experience and experience in general possible;
that is, the assumption of the freedom
that is opposed to the antithesis cannot be made,
for the reason that it contradicts the antithesis.

We find in essence the same antinomy in
the Critique of the Teleological Judgement as
the opposition between the proposition that
every generation of material things happens
according to merely mechanical laws,
and the proposition that some cases
of generation of material things
are not possible according to such laws.
Kant's resolution of this antinomy is the same
as the general resolution of the rest,
namely that reason cannot prove
either the one or the other proposition
because we cannot have a priori
any determining principle of the possibility of things
according to merely empirical laws of nature;
further, that therefore the two propositions must
be regarded not as objective propositions but as subjective maxims;
that I ought to reflect on the events of nature every time
according to the principle of the mechanism of nature alone,
but that this does not prevent, when occasion permits,
following up certain natural forms in accordance with another maxim,
namely in accordance with the principal of final causes,
as if now these two maxims, which moreover are supposed to be
necessary only for human reason,
did not stand in the same opposition as
the two propositions in antinomy.
Missing in all this, as we remarked above,
is the one thing that alone is of philosophical interest,
namely the investigation of which of the two principles
has truth in and for itself.
On this standpoint, it makes no difference
whether the principles should be regarded as objective,
which means here, as externally existing
determinations of nature,
or as mere maxims of a subjective cognition;
what is subjective here is rather the
contingent cognition that applies one or the other maxim
as occasion demands, indeed, according to whether
it deems them fitting for given objects,
but for the rest does not ask about the truth
of these determinations themselves,
whether they both are determinations
of the objects or of cognition.

However unsatisfactory is for this reason Kant's discussion
of the teleological principle with respect to its essential viewpoint,
still worthy of note is the place that Kant assigns to it.
Since he ascribes it to a reflective faculty of judgment,
he makes it into a mediating link between
the universal of reason and the singular of intuition;
further, he distinguishes this reflective judgment
from the determining judgment, the latter one
that merely subsumes the particular under the universal.
Such a universal that only subsumes is an abstraction
that becomes concrete only in an other, in the particular.
Purpose, on the contrary, is the concrete universal
containing within itself the moment
of particularity and of externality;
it is therefore active and the impulse
to repel itself from itself.
The concept, as purpose, is of course an objective judgment
in which one determination, the subject,
namely the concrete concept, is self-determined,
while the other is not only a predicate but external objectivity.
But for that reason the connection of purpose is not
a reflective judgment that considers external objects
only according to a unity,
as though an intelligence had given them to us
for the convenience of our faculty of cognition;
on the contrary, it is the truth that exists in
and for itself and judges objectively,
determining the external objectivity absolutely.
The connection of purpose is therefore more than judgment;
it is the syllogism of the self-subsistent free concept
that through objectivity unites itself with itself in conclusion.

Purpose has resulted as the third
to mechanism and chemism;
it is their truth.
Inasmuch as it still stands
inside the sphere of objectivity
or of the immediacy of the total concept,
it is still affected by externality as such
and has an objective world over
against it to which it refers.
From this side, mechanical causality,
to which chemism is also in general to be added,
still makes its appearance in this purposive connection
which is the external one,
but as subordinated to it
and as sublated in and for itself.
As regards the more precise relation,
the mechanical object is, as immediate totality,
indifferent to its being determined and consequently,
conversely, to its being a determinant.
This external determinateness has now
progressed to self-determination
and accordingly the concept that
in the object was only inner
or, which amounts to the same,
only outer, is now posited;
purpose is, in the first instance,
precisely this concept which is
external to the mechanical object.
And so for chemism also, purpose is the self-determining
which brings the external determinateness conditioning it
back to the unity of the concept.
We have here the nature of the subordination of
the two preceding forms of the objective process.
The other, which in those forms lies in the infinite progress,
is the concept posited at first as external to them,
and this is purpose;
not only is the concept their substance
but externality is for them also
an essential moment constituting their determinateness.
Thus mechanical or chemical technique,
because of its character of being externally determined,
naturally offers itself to the connection of purpose,
which we must now examine more closely.

SECTION III

The idea

The idea is the adequate concept,
the objectively true,
or the true as such.
If anything has truth,
it has it by virtue of its idea,
or something has truth only
in so far as it is idea.
The expression “idea” has otherwise also often been used
in philosophy as well as in ordinary life for “concept,”
or even for just a “representation.”
To say that I still have no idea of
this lawsuit, this building, this region,
means nothing more than I still have no representation of it.
It is Kant who reclaimed the expression “idea”
for the “concept of reason.”
Now according to Kant the concept of reason
should be the concept of the unconditional,
but a concept which is transcendent with respect to appearances,
that is, one for which no adequate empirical use can be made.
The concepts of reason are supposed to serve
for the comprehension of perceptions,
those of the understanding for the understanding of them.
In fact, however, if these last
concepts of the understanding are truly concepts,
then they are comprehensions, which means concepts;
they will make comprehending possible,
and an understanding of perceptions
through concepts of the understanding
will be a comprehending.
But if understanding is only
the determining of perceptions by categories
such as whole and parts, force, cause, and the like,
then it signifies only a determining by means of reflection,
just as by understanding one may mean only the determinate
representation of a fully determined sensuous content;
as when someone is being shown the way,
that at the end of the wood he must turn left,
and he replies “I understand,”
understanding means nothing more than a grasp
in pictorial representation and in memory.
“Concept of reason,” too, is a somewhat clumsy expression;
for the concept is in general something rational,
and in so far as reason is distinguished
from the understanding and the concept as such,
it is the totality of the concept and objectivity.
The idea is the rational in this sense;
it is the unconditioned,
because only that has conditions
which essentially refers to an objectivity
that it does not determine itself
but which still stands over against it
in the form of indifference and externality,
just as the external purpose had conditions.

If we now reserve the expression “idea”
for the objective or real concept
and we distinguish it from the concept itself
and still more from mere representation,
then we must also even more definitely reject
that estimate of it according to which
the idea is something with no actuality,
and true thoughts are accordingly said to be only ideas.
If thoughts are something merely subjective and contingent,
then they certainly have no further value;
but in this they do not stand lower
than the temporal and contingent actualities
which likewise have no further value than
that of accidentalities and appearances.
But if, on the contrary, the idea is
supposed not to have the value of truth
because in regard to appearances it is transcendent,
because no congruent object can be given for it
in the world of the senses,
then this is indeed an odd misunderstanding,
for objective validity is being denied to it
on the ground that it lacks precisely what makes
of appearances the untrue being of the objective world.
In regard to the practical ideas, Kant recognizes that
“nothing can be more harmful and unworthy of
a philosopher than the vulgar appeal to experience,
which supposedly contradicts the idea.
Any such alleged contradiction would not be there at all
if, for example, political institutions were set up
at the right time in accordance with ideas,
and if crude concepts, crude just
because they are drawn from experience,
had not usurped the place of ideas
thus thwarting all good intentions.”
Kant regards the idea as something necessary,
the goal which, as the archetype,
we must strive to set up as a maximum
and to which we must bring actuality
as it presently stands ever closer.

But since the result now is that the idea is
the unity of the concept and objectivity, the true,
we must not regard it as just a goal which is to be approximated
but itself remains always a kind of beyond;
we must rather regard everything as being actual
only to the extent that it has the idea
in it and expresses it.
It is not just that the subject matter,
the objective and the subjective world,
ought to be in principle congruent with the idea;
the two are themselves rather the congruence of concept and reality;
a reality that does not correspond to the concept is mere appearance,
something subjective, accidental, arbitrary, something which is not the truth.
When it is said that there is no subject matter
to be found in experience which is perfectly congruent with the idea,
the latter is opposed to the actual as a subjective standard;
but there is no saying what anything actual
might possibly be in truth, if its concept is not in it
and its objectivity does not measure up to this concept;
it would then be a nothing.
Indeed, the mechanical and the chemical object,
like a subject devoid of spirit
and a spirit conscious only of finitude and not of its essence,
do not, according to their various natures,
have their concept concretely existing in them in
its own free form.
But they can be something at all true only in so far as
they are the union of their concept and reality, of their soul and their body.
Wholes like the state and the church cease to exist in concreto when the
unity of their concept and their reality is dissolved; the human being, the
living thing, is dead when soul and body are parted in it; dead nature, the
mechanical and the chemical world
that is, when “the dead” is taken
to mean the inorganic world,
for the expression would otherwise have no positive meaning at all
this dead nature, then, if it is separated into its
concept and its reality,
is nothing but the subjective abstraction of
a thought form and a formless matter.
Spirit that were not idea,
not the unity of the concept with itself,
not the concept that has the concept itself as its reality,
would be dead spirit, spiritless spirit, a material object.

Since the idea is the unity of the concept and reality,
being has attained the significance of truth;
it now is, therefore, only what the idea is.
Finite things are finite because,
and to the extent that,
they do not possess the reality of
their concept completely within them
but are in need of other things for it
or, conversely, because they are presupposed as objects
and consequently the concept is in them as an external determination.
The highest to which they attain on
the side of this finitude is external purposiveness.
That actual things are not congruent with the idea
constitutes the side of their finitude, of their untruth,
and it is according to this side that they are objects,
each in accordance with its specific sphere,
and, in the relations of objectivity, determined
as mechanical, chemical, or by an external purpose.
That the idea has not perfectly fashioned their reality,
that it has not completely subjugated it to the concept,
the possibility of that rests on the fact that
the idea itself has a restricted content;
that, as essentially as it is
the unity of the concept and reality,
just as essentially it is also their difference;
for only the object is the immediate unity, that is,
the unity that only exists in itself.
But if a subject matter, say the state,
did not at all conform to its idea,
that is to say, if it were not rather the idea of the state;
if its reality, which is the self-conscious individuals,
did not correspond at all to the concept,
its soul and body would have come apart;
the soul would have taken refuge in the secluded regions of thought,
the body been dispersed into singular individualities.
But because the concept of the state is essential
to the nature of these individualities,
it is present in them as so mighty an impulse
that they are driven to translate it into reality,
be it only in the form of external purposiveness,
or to put up with it as it is,
or else they must needs perish.
The worst state, one whose reality least corresponds to the concept,
in so far as it still has concrete existence, is yet idea;
the individuals still obey the power of a concept.

But the idea has not only
the general meaning of true being,
of the unity of concept and reality,
but also the more particular one
of the unity of subjective concept and objectivity.
For the concept is as such itself
already the identity of itself and reality;
for the indeterminate expression “reality”
means nothing but determinate being,
and this the concept possesses
in its particularity and singularity.
Objectivity, moreover, is likewise the total concept
that has withdrawn into identity with itself out of its determinateness.
In the subjectivity of the concept,
the determinateness
or the difference of the latter is
a reflective shine which is immediately sublated,
withdrawn into being-for-itself
or into negative unity,
an inhering predicate.
But in this objectivity the determinateness is
posited as immediate totality, as external whole.
Now the idea has shown itself to be the concept
liberated again into its subjectivity
from the immediacy into which it has sunk in the object;
it is the concept that distinguishes itself from its objectivity
but an objectivity which is no less determined by it
and possesses its substantiality only in that concept.
This identity has therefore rightly been
designated as a subject-object,
for it is just as well the formal
or subjective concept
as it is the object as such.
But this is a point that needs further precision.
The concept, inasmuch as it has truly attained its reality,
is this absolute judgment whose subject distinguishes itself
as self-referring negative unity from its objectivity
and is the latter's being-in-and-for-itself;
but it refers to it essentially through itself
and is, therefore, self-directed purpose and impulse.
For this very reason, however,
the subject does not possess objectivity immediately in it
(it would then be only the totality of the object as such,
a totality lost in the objectivity)
but is the realization of the purpose,
an objectivity posited by virtue of
the activity of the purpose,
one which, as positedness, has its subsistence and its form only as
permeated by its subject.
As objectivity, it has the moment of the externality
of the concept in it and is in general,
therefore, the side of finitude,
of alteration and appearance;
but this side retreats into the negative unity of
the concept and there it perishes;
the negativity whereby its indifferent
externality of being manifests itself as unessential
and as a positedness is the concept itself.
Despite this objectivity, the idea is therefore
absolutely simple and immaterial,
for the externality has being only
as determined by the concept
and as taken up into its negativity;
in so far as it exists as indifferent externality,
it is not only abandoned to mechanism in general
but exists only as the transitory and untrue.
Thus although the idea
has its reality in a materiality,
the latter is not an abstract being standing
over against the concept
but, on the contrary, it exists only as becoming,
as simple determinateness of the concept
by virtue of the negativity of the indifferent being.

This yields the following closer determinations of the idea.

First, the idea is the simple truth,
the identity of concept and objectivity as a
universal in which the opposition,
the presence of the particular,
is dissolved in its self-identical negativity
and is equality with itself.

Second, it is the connection of the subjectivity
of the simple concept, existing for itself,
and of the concept's objectivity which is distinguished from it;
the former is essentially the impulse to sublate this separation,
and the latter is indifferent positedness,
subsistence which in and for itself is null.
As this connection, the idea is
the process of disrupting itself into individuality
and into the latter's inorganic nature,
and of then bringing this inorganic nature again
under the controlling power of the subject
and back to the first simple universality.
The identity of the idea with itself is one with the process;
the thought that liberates actuality from
the seeming of purposeless mutability
and transfigures it into idea
must not represent this truth of actuality
as dead repose, as a mere picture, numb, without impulse and movement,
as a genus or number, or as an abstract thought;
the idea, because of the freedom which the concept has attained in it,
also has the most stubborn opposition within it;
its repose consists in the assurance and the certainty
with which it eternally generates that opposition
and eternally overcomes it, and in it rejoins itself.

But the idea is at first again only immediate or only in its concept;
the objective reality is indeed conformable to the concept
but has not yet been liberated into the concept,
and it does not concretely exist explicitly as the concept.
Thus the concept is indeed the soul,
but the soul is in the guise of an immediate,
that is, it is not determined as soul itself,
has not comprehended itself as soul,
does not have its objective reality within itself;
the concept is as a soul that is not yet fully animated.

Thus the idea is, first of all, life.
It is the concept which, distinct from its objectivity,
simple in itself, permeates that objectivity
and, as self-directed purpose, has its means within it
and posits it as its means, yet is immanent in this means
and is therein the realized purpose identical with itself.
The idea, on account of its immediacy, has singularity
for the form of its concrete existence.
But the reflection within it of its absolute process is
the sublating of this immediate singularity;
thereby the concept, which as universality is
in this singularity the inner,
transforms externality into universality,
or posits its objectivity as a self-equality.

Thus is the idea, in second place,
the idea of the true and the good,
as cognition and will.
It is at first finite cognition and finite will,
where the true and the good are still distinguished
and the two are at first only as a goal.
The concept has first liberated itself into itself,
giving itself only a still abstract objectivity for its reality.
But the process of this finite cognition and this finite action
transforms the initially abstract universality into totality,
whereby it becomes complete objectivity.
Or considered from the other side,
finite, that is, subjective spirit,
makes for itself the
presupposition of an objective world,
such a presupposition as life only has;
but its activity is the sublating of this presupposition
and the turning of it into something posited.
Thus its reality is for it the objective world,
or conversely the objective world is
the ideality in which it knows itself.

Third, spirit recognizes the idea as its absolute truth,
as the truth that is in and for itself:
the infinite idea in which cognizing and doing are equalized,
and which is the absolute knowledge of itself.

CHAPTER 1

Life

YS III.16

    parinama-traya-samyamad atitanagata-jnanam

The idea of life has to do with
a subject matter so concrete,
and if you will so real,
that in dealing with it one may seem
according to the common notion of logic
to have overstepped its boundaries.
Needless to say, if the logic were to contain
nothing but empty, dead forms of thought,
then there could be no talk in it at all
of such a content as the idea, or life, are.
But if the subject matter of logic is the absolute truth,
and truth as such lies essentially in cognition,
then cognition at least would have to come in for consideration.
It is common practice to have the so-called pure logic
be followed by an applied logic,
a logic that has to do with concrete cognition,
quite apart from all the psychology and anthropology
that is commonly deemed necessary to interpolate into logic.
But the anthropological and psychological side of
cognition is concerned with the form in which cognition appears
when the concept does not as yet have an objectivity equal to it,
that is, when it does not have itself as object.
The part of the logic that deals with this
concrete cognition does not belong to applied logic as such;
if it did, then every science would have to be dragged into logic,
for each is an applied logic in so far as
it consists in apprehending its subject matter
in forms of thought and of concepts.
The subjective concept has presuppositions
that are exhibited in psychological, anthropological, and other forms.
But the presuppositions of the pure concept
belong in logic only to the extent that
they have the form of pure thoughts, of abstract essentialities,
such as are the determinations of being and essence.
The same goes for cognition,
which is the concept's comprehension of itself:
no other shape of its presupposition
but the one which is itself idea
is to be dealt with in the logic;
this, however, is a presupposition
which is necessarily treated in logic.
This presupposition is now the immediate idea;
for while cognition is the concept,
in so far as the latter exists for itself
but as a subjectivity referring to an objectivity,
then the concept refers to the idea
as presupposed or as immediate.
But the immediate idea is life.

To this extent the necessity of considering
the idea of life in logic would be based on
the necessity, itself recognized in other ways,
of treating the concrete concept.
But this idea has arisen through the concept's own necessity;
the idea, that which is true in and for itself,
is essentially the subject matter of the logic;
since it is first to be considered in its immediacy,
so that this treatment be not
an empty affair devoid of determination,
it is to be apprehended and cognized
in this determinateness in which it is life.
A comment may be in order here to
differentiate the logical view of life
from any other scientific view of it,
though this is not the place to concern
ourselves with how life is treated in non-philosophical sciences
but only with how to differentiate logical life as idea
from natural life as treated in the philosophy of nature,
and from life in so far as it is bound to spirit.
As treated in the philosophy of nature,
as the life of nature and to that extent
exposed to the externality of existence,
life is conditioned by inorganic nature
and its moments as idea are a manifold of actual shapes.
Life in the idea is without such presuppositions,
which are in shapes of actuality;
its presupposition is the concept
as we have considered it,
on the one hand as subjective,
and on the other hand as objective.
In nature life appears as the highest stage
that nature's externality can attain
by withdrawing into itself
and sublating itself in subjectivity.
It is in logic the simple in-itselfness
which in the idea of life has attained
the externality truly corresponding to it;
the concept that came on the scene earlier as
a subjective concept is the soul of life itself;
it is the impulse that gives itself reality
through a process of objectification.
Nature, as it reaches this idea
starting from its externality, transcends itself;
its end is not its beginning but is for it
as a limit in which it sublates itself.
Similarly, in the idea of life
the moments of life's reality do not
receive the shape of external actuality
but remain enveloped in conceptual form.

In spirit, however, life appears
both as opposed to it and as posited as at one with it,
in a unity reborn as the pure product of spirit.
For life is here to be taken generally in
its proper sense as natural life,
for what is called the life of spirit as spirit,
is spirit's own peculiar nature
that stands opposed to mere life;
just as we speak of the nature of spirit,
even though spirit is nothing natural
but stands rather in opposition to nature.
Thus life as such is for spirit in one respect a means,
and then spirit holds it over against itself;
in another respect, spirit is an individual,
and then life is its body;
in yet another respect, this unity of spirit
and its living corporeality is
born of spirit into ideality.
None of these connections of life to spirit
concerns logical life,
and life is to be considered here
neither as the instrument of a spirit,
nor as a living body,
nor again as a moment of the ideal and of beauty.
In both cases, as natural life and as referring to spirit,
life obtains a determinateness from its externality,
in one case through its presuppositions,
such as are other formations of nature,
and in the other case through the
purposes and the activity of spirit.
The idea of life by itself is free
from both the conditioning objectivity
presupposed in the first case
and the reference to subjectivity
of the second case.

Life, considered now more closely in its idea,
is in and for itself absolute universality;
the objectivity which it possesses is
throughout permeated by the concept,
and this concept alone it has as substance.
Whatever is distinguished as part,
or by some otherwise external reflection,
has the whole concept within it;
the concept is the soul omnipresent in it,
a soul which is simple self-reference
and remains one in the manifoldness
that accrues to the objective being.
This manifoldness, as self-external objectivity,
has an indifferent subsistence which in space and time,
if these could already be mentioned here,
is a mutual externality of entirely
diverse and atomistic matters.
But externality is in life at the same time
as the simple determinateness of its concept;
thus the soul flows omnipresently in this manifold
but remains at the same time the simple oneness
of the concrete concept with itself.
That way of thinking that clings to the determinations of
reflective relations and of the formal concept,
when it comes to consider life,
the unity of its concept in the externality of objectivity,
the absolute multiplicity of atomistic matter,
finds that all its thoughts are absolutely of no avail;
the omnipresence of the simple in the manifold externality
is for reflection an absolute contradiction
and also, since it cannot at the same time avoid
witnessing this omnipresence in the perception of life
and must therefore grant the actuality of this idea,
an incomprehensible mystery
for reflection does not grasp the concept,
nor does it grasp it as the substance of life.
But this simple life is not only omnipresent;
it is the one and only subsistence
and immanent substance of its objectivity;
but as subjective substance it is impulse,
more precisely the specific impulse of particular difference,
and no less essentially the one and universal impulse
of the specific that leads its particularization
back to unity and holds it there.
Only as this negative unity of its
objectivity and particularization is life self-referring,
life that exists for itself, a soul.
As such, it is essentially a singular that refers to
objectivity as to an other, an inanimate nature.
The originative judgment of life consists therefore in this,
that it separates itself off as individual subject from the objective
and, since it constitutes itself as the negative unity of the concept,
makes the presupposition of an immediate objectivity.

First, life is therefore to be considered as a living individual
that is for itself the subjective totality
and is presupposed as indifferent to an objectivity
that stands indifferent over against it.

Second, it is the life-process of sublating its presupposition,
of positing as negative the objectivity indifferent to it,
and of actualizing itself as the power
and negative unity of this objectivity.
By so doing, it makes itself into the universal
which is the unity of itself and its other.

Third, consequently life is the genus-process,
the process of sublating its singularization
and relating itself to its objective existence
as to itself.
Accordingly, this process is
on the one hand the turning back to its concept
and the repetition of the first forcible separation,
the coming to be of a new individuality
and the death of the immediate first;
but, on the other hand, the withdrawing into itself
of the concept of life is the becoming of
the concept that relates itself to itself,
of the concept that exists for itself,
universal and free, the transition into cognition.

CHAPTER 2

The idea of cognition

YS III.17

    sabda-artha-pratyayanam itaretara-adhyasat sankaras tat-pravibhaga-samyamat sarva-bhuta-ruta-jnanam

YS III.18

    samskara-saksat-karanat purva-jati-jnanam

YS III.19

    pratyayasya para-citta-jnanam

YS III.20

    na ca tat salambanam tasya-avisayi-bhutatvat

Life is the immediate idea, or the idea as
its still internally unrealized concept.
In its judgment, the idea is cognition in general.

The concept is for itself as concept
inasmuch as it freely and concretely exists
as abstract universality or a genus.
As such, it is its pure self-identity
that internally differentiates itself
in such a way that the differentiated is
not an objectivity but is rather
equally liberated into subjectivity
or into the form of simple self-equality;
consequently, the object facing
the concept is the concept itself.
Its reality in general is the form of its existence;
all depends on the determination of this form;
on it rests the difference between
what the concept is in itself, or as subjective,
and what it is when immersed in objectivity,
and then in the idea of life.
In this last, the concept is indeed distinguished
from its external reality and posited for itself;
however, this being-for-itself which it now has,
it has only as an identity that refers to itself
as immersed in the objectivity subjugated to it,
or to itself as indwelling, substantial form.
The elevation of the concept above life consists in this,
that its reality is the concept-form liberated into universality.
Through this judgment the idea is doubled,
into the subjective concept whose reality is the concept itself,
and the objective concept which is as life.
Thought, spirit, self-consciousness, are determinations
of the idea inasmuch as the latter has itself
as the subject matter, and its existence, that is, the determinateness
of its being, is its own difference from itself.

The metaphysics of the spirit or,
as was more commonly said in the past, of the soul,
revolved around the determinations
of substance, simplicity, immateriality.
These were determinations for which
spirit was supposed to be the ground,
but as a subject drawn from empirical consciousness,
and the question then was which predicates
agreed with the perceived facts.
But this was a procedure that could go
no further than the procedure of physics,
which reduces the world of appearance
to general laws and determinations of reflection,
for it is spirit still as phenomenal
that is taken as the foundation.
In fact, in so far as scientific stringency goes,
it also had to fall short of physics.
For not only is spirit infinitely richer than nature;
since its essence is constituted by
the absolute unity in the concept of opposites,
and in its appearance, therefore,
and in its connection with externality,
it exhibits contradiction at its most extreme form,
it must be possible to adduce an experience in support of
each of the opposite determinations of reflection,
or, starting from experiences, to proceed by way
of formal inference to the opposite determinations.
Since the predicates immediately drawn from
the appearances still belong to empirical psychology,
so far as metaphysical consideration goes,
all that is in truth left are
the entirely inadequate determinations of reflection.
In his critique of rational psychology, Kant insists that,
since this metaphysics is supposed to be a rational science,
the least addition of anything drawn from perception
to the universal representation of self-consciousness
would alter it into an empirical science,
thus compromising its rational purity
and its independence from all experience.
Accordingly, all that is left on this view is
the simple representation “I,”
a representation entirely devoid of content,
of which one cannot even say that it is a concept,
but must say that it is a mere consciousness,
one that accompanies every concept.
Now, as Kant argues further, this “I,”
or, if you prefer, this “it” (the thing) that thinks,
takes us no further than the representation
of a transcendental subject of thoughts = x,
a subject which is known only through
the thoughts that are its predicates,
and of which, taken in isolation,
we cannot ever have the least concept.
This “I” has the associated inconvenience that,
as Kant expresses it, in order to judge anything about it,
we must every time already make use of it,
for it is not so much one representation
by which a particular object is distinguished,
as it is rather a form of representation in general,
in so far as representation can be said to be cognition.
Now the paralogism that rational psychology incurs,
as Kant expresses it, consists in this:
that modes of self-consciousness in thinking are
converted into concepts of the understanding,
as if they were the concepts of an object;
that that “I think” is taken to be a thinking being, a thing-in-itself;
that in this way, because I am present
in consciousness always as a subject,
am indeed as a singular subject,
identical in all the manifoldness of representation,
and distinguishing myself from this manifoldness as external to it,
the illegitimate inference is thereby drawn that I am a substance,
and a qualitatively simple being on top of that,
and a one, and a being that concretely exists
independently of the things of space and time.

I have cited this position in some detail
because one can clearly recognize in it
both the nature of the former metaphysics of the soul
and also, more to the point, of the Critique that put an end to it.
The former was intent on determining the abstract essence of the soul;
it went about this starting from observation,
and then converting the latter's empirical generalizations,
and the determination of purely external reflection
attaching to the singularity of the actual,
into the form of the determinations of essence just cited.
What Kant generally has in mind here is
the state of the metaphysics of his time
which, as a rule, stayed at these one-sided determinations
with no hint of dialectic;
he neither paid attention to, nor examined,
the genuinely speculative ideas of older
philosophers on the concept of spirit.
In his critique of those determinations he then
simply abided by the Humean style of skepticism;
that is to say, he fixes on how the “I” appears in self-consciousness,
but from this “I,” since it is its essence
(the thing in itself) that we want to cognize,
he removes everything empirical;
nothing then remains but this appearance of the “I think”
that accompanies all representations
and of which we do not have the slightest concept.
It must of course be conceded that,
as long as we are not engaged in comprehending
but confine ourselves to a simple, fixed representation or to a name,
we do not have the slightest concept of the “I,”
or of anything whatever, not even of the concept itself.
Peculiar indeed is the thought
(if one can call it a thought at all)
that I must make use of the “I” in order to judge the “I.”
The “I” that makes use of self-consciousness
as a means in order to judge:
this is indeed an x of which,
and also of the relation involved in this “making use,”
we cannot possibly have the slightest concept.
But surely it is laughable to label the nature of this self-consciousness,
namely that the “I” thinks itself,
that the “I” cannot be thought without the “I” thinking it,
an awkwardness and, as if it were a fallacy, a circle.
The awkwardness, the circle, is in fact the relation
by which the eternal nature of self-consciousness
and of the concept is revealed in
immediate, empirical self-consciousness,
is revealed because self-consciousness is
precisely the existent and therefore
empirically perceivable pure concept;
because it is the absolute self-reference that,
as parting judgment, makes itself into an intended object
and consists in simply making itself thereby into a circle.
This is an awkwardness that a stone does not have.
When it is a matter of thinking or judging,
the stone does not stand in its own way;
it is dispensed from the burden of making
use of itself for the task;
something else outside it must shoulder that effort.

The defect, which these surely barbarous notions place
in the fact that in thinking the “I”
the latter cannot be left out as a subject,
then also appears the other way around,
in that the “I” occurs only as the subject of consciousness,
or in that I can use myself only as a subject,
and no intuition is available by which the “I” would be given as an object;
but the concept of a thing capable of existence only as a subject
does not as yet carry any objective reality with it.
Now if external intuition as determined in
time and space is required for objectivity,
and it is this objectivity that is missed,
it is then clear that by objectivity is meant only sensuous reality.
But to have risen above such a reality is precisely
the condition of thinking and of truth.
Of course, if the “I” is not grasped conceptually
but is taken as a mere representation,
in the way we talk about it in everyday consciousness,
then it is an abstract self-determination,
and not the self-reference that has itself
for its subject matter.
Then it is only one of the extremes,
a one-sided subject without its objectivity;
or else just an object without subjectivity,
which it would be were it not for the awkwardness just touched upon,
namely that the thinking subject will not
be left out of the “I” as object.
But as a matter of fact this awkwardness is
already found in the other determination,
that of the “I” as subject;
the “I” does think something,
whether itself or something else.
This inseparability of the two forms
in which the “I” opposes itself to itself
belongs to the most intimate nature of its concept
and of the concept as such;
it is precisely what Kant wants to keep away
in order to retain what is only a representation
that does not internally differentiate itself
and consequently, of course, is void of concept.
Now this kind of conceptual void may well oppose itself
to the abstract determinations of reflection
or to the categories of the previous metaphysics,
for in one-sidedness it stands at the same level with them,
though these are in fact on a higher level of thought;
but it appears all the more lame and empty
when compared with the profounder ideas
of ancient philosophy concerning
the concept of the soul or of thinking,
as for instance the truly speculative ideas of Aristotle.
If the Kantian philosophy subjected the
categories of reflection to critical investigation,
all the more should it have investigated the abstraction
of the empty “I” that he retained,
the supposed idea of the thing-in-itself.
The experience of the awkwardness complained of is
itself the empirical fact in which
the untruth of that abstraction finds expression.

The Kantian critique of rational psychology
only refers to Mendelssohn's proof of the persistence of the soul,
and I now also cite its refutation of that proof
because of the oddness of what it adduces against it.
Mendelssohn's proof is based on the simplicity of the soul,
by virtue of which it is supposed to be incapable
of alteration in time, of transition into an other.
Qualitative simplicity is in general the
form of abstraction earlier considered;
as qualitative determinateness, it was investigated
in the sphere of being
and it was then proved that the qualitative,
which is as such abstractly self-referring determinateness,
is precisely for that reason dialectical,
mere transition into an other.
In the case of the concept, however, it was shown that,
when considered in connection with persistence,
indestructibility, imperishableness,
it is that which exists for itself,
which is eternal, just because it is
not abstract but concrete simplicity
because it is not a determinateness
that refers to itself abstractly
but is the unity of itself and its other,
and it cannot therefore pass over into this other
as if it thereby altered in it;
it cannot precisely because it is itself the other,
the determinateness, and hence in this
passing over it only comes to itself.
Now the Kantian critique opposes to this
qualitative determination of the unity of the concept
a quantitative one.
As it says, although the soul is not
a manifold of reciprocally external parts
and contains no extensive magnitude,
yet consciousness has a degree,
and the soul, like every concretely existing being,
is an intensive magnitude;
with this magnitude, however, there is posited
the possibility of a transition into nothing through gradual vanishing.
Now what is this refutation but the application to spirit
of a category of being, of intensive magnitude,
a determination that has no truth in itself
but on the contrary is sublated in the concept?

Metaphysics even one that restricted itself
to the fixed concepts of the understanding
without rising to speculation,
to the nature of the concept and of the idea,
did have for its aim the cognition of truth;
it did probe its subject matter to ascertain
whether they were something true or not,
whether substances or phenomena.
The triumph of the Kantian critique
over this metaphysics consists, on the contrary,
in side-lining any investigation
that would have truth for its aim and this aim itself;
it simply does not pose the one question
which is of interest,
namely whether a determinate subject,
in this case the abstract “I” of representation,
has truth in and for itself.
But to stay at appearances
and at the mere representations
of ordinary consciousness is to give up
on the concept and on philosophy.
Anything beyond that is branded
by the Kantian critique as high-flown,
something to which reason has no claim.
As a matter of fact, the concept does fly high,
rising above what has no concept,
and the immediate justification
for going beyond it is, for one thing, the concept itself,
and, for another, on the negative side,
the untruth of appearance and of representation,
and also of such abstractions as the thing-in-itself
and the said “I” which is not supposed to be an object to itself.

In the context of this logical exposition,
it is from the idea of life that
the idea of spirit has emerged,
or what is the same thing, that has demonstrated
itself to be the truth of the idea of life.
As this result, the idea possesses its truth in and for itself,
with which one may then also compare the empirical reality
or the appearance of spirit to see how far it accords with it.
We have seen regarding life that it is the idea,
but at the same time it has shown itself
not to be as yet the true presentation
or the true mode of its existence.
For in life, the reality of the idea is singularity;
universality or the genus is the inwardness.
The truth of life as absolute negative unity consists,
therefore, in this:
to sublate the abstract or, what is the same,
the immediate singularity,
and as identical to be self-identical,
as genus, to be self-equal.
Now this idea is spirit.
In this connection, we may further remark
that spirit is here considered in the form
that pertains to this idea as logical.
For the idea also has other shapes
which we may now mention in passing;
in these it falls to the concrete sciences
of spirit to consider it, namely
as soul, consciousness, and spirit as such.

The name “soul” was used formerly to mean
singular finite spirit in general,
and rational or empirical psychology was
supposed to be synonymous with doctrine of spirit.
The expression, “soul,” evokes an image of it
as if it were a thing like other things.
One enquires regarding its seat,
the spatial location from which its forces operate;
still more, how this thing can be imperishable,
subjected to the conditions of temporality yet exempt
from alteration in it.
The system of monads elevates matter by making all
of it in principle a soul;
on this way of representing it, the soul is an
atom like the atoms of matter;
the atom that rises from a cup of coffee as vapor
is capable in favorable circumstances of developing into a soul;
only the greater obscurity of its ideation distinguishes it
from the kind of thing that is manifestly soul.
The concept that is for itself is
necessarily also in immediate existence;
in this substantial identity with life,
immersed in its externality,
the concept is the subject matter of anthropology.
But even anthropology would find alien a metaphysics
in which this form of immediacy is made into a soul-thing,
into an atom like the atoms of matter.
To anthropology must be left only that obscure region where spirit,
under influences which were once called sidereal and terrestrial,
lives as a natural spirit in sympathy with nature
and has presentiments of the latter's alterations
in dreams and presentiments,
and indwells the brain, the heart, the liver, and so forth.
To the liver, according to Plato,
God gave the gift of prophesy
above which the self-conscious human is exalted,
so that even the irrational part of the soul
would be provided for by his bounty
and made to share in higher things.
To this irrational side belongs further the
behavior of figurative representation,
and of higher spiritual activity in so far
as the latter is subject to the play
of an entirely corporeal constitution,
of external influences and particular circumstances.

This lowest of the concrete shapes
in which spirit is sunk into materiality has
the one immediately superior to it in consciousness.
In this form the free concept,
as the “I” existing for itself,
is withdrawn from objectivity,
but it refers to the latter as its other,
a subject matter that confronts it.
Since spirit is here no longer as soul,
but, in the certainty that it has of itself,
the immediacy of being has for it the significance
rather of a negative, its identity with itself
in the objectivity confronting it is at
the same time still only a reflective shining,
for that objectivity still also has the
form of a being that exists in itself.
This stage is the subject matter
of the phenomenology of spirit,
a science that stands midway between the science
of the natural spirit and of the spirit as such.
It considers spirit as it exists for itself,
but at the same time as referring to its other,
an other which, as we have just said, is
thereby determined both as an object
existing in itself and as a negative.
The science thus considers spirit as appearing,
as exhibiting itself in its contrary.

But the higher truth of this form is spirit for itself.
For this spirit, the subject matter which for consciousness
exists in itself has the form of its own determination,
the form of representation in general;
this spirit, which acts on the subject matter's determinations
as on its own, on feelings, on representations and thought,
is thus infinite in itself and in its form.
The consideration of this stage belongs
to the doctrine of spirit proper,
which would embrace the subject matter
of ordinary empirical psychology
but which, in order to be the science of spirit,
must not go about its work empirically
but must be conceived scientifically.
At this stage spirit is finite spirit in so far as
the content of its determinateness is an immediately given content;
the science of this finite spirit has
to display the course along which
it liberates itself from this determinateness
and goes on to grasp its truth, the infinite spirit.

The idea of spirit which is the subject matter of logic
already stands, on the contrary, inside pure science;
it has no need, therefore, to observe spirit's
tracing that course, to see how it gets entangled
with nature, with immediate determinateness, with matter,
or in other words with pictorial representation;
this is what the other three sciences investigate.
The idea of spirit has this course already behind it,
or what is the same, it has it rather ahead of it
behind in so far as logic is taken as the final science;
ahead in so far as it is taken as the first science
from which the idea first passes over into nature.
In the logical idea of spirit, therefore,
the “I” is from the start in the way it has emerged
from the concept of nature as the truth of nature,
the free concept which in its judgment is itself the subject matter
confronting it, the concept as its idea.
Also in this shape, however, the idea is still not consummated.

Although the idea is indeed the free concept
that has itself as its subject matter,
it is nonetheless immediate,
and just because it is immediate,
it is still the idea in its subjectivity,
and hence in its finitude in general.
It is the purpose that ought to realize itself,
or the absolute idea itself still in its appearance.
What the idea seeks is the truth,
this identity of the concept itself and reality;
but at first it only seeks it;
for it is here as it is at first,
still something subjective.
Consequently, although the subject matter
that is for the concept is here also a given subject matter,
it does not enter into the subject as affecting it,
or as confronting it with a constitution of its own
as subject matter, or as a pictorial representation;
on the contrary, the subject transforms it
into a conceptual determination;
it is the concept which is the active principle in it
which therein refers itself to itself,
and, by thus giving itself its reality in the object, finds truth.

Initially, therefore, the idea is one extreme of a syllogism,
the concept that as purpose has itself
at first for its subjective reality;
the other extreme is the restriction
of the subjective, the objective world.
The two extremes are identical in that they are the idea.
Their unity is, first, that of the concept,
a unity which in the one extreme is only for itself
and in the other only in itself.
Second, it is reality, abstract in the one extreme
and in the other in its concrete externality.
This unity is now posited through cognition,
and, because the latter is the subjective idea
which as purpose proceeds from itself,
it is at first only a middle term.
The knowing subject, through the
determinateness of its concept
which is the abstract being-for-itself,
refers to an external world;
nevertheless, it does this in
the absolute certainty of itself,
in order to elevate its implicit reality,
this formal truth, to real truth.
It has the entire essentiality of
the objective world in its concept;
its process consists in positing for itself
the concrete reality of that world
as identical with the concept,
and conversely in positing the latter
as identical with objectivity.

Immediately, the idea of appearance is
the theoretical idea, cognition as such.
For to the concept that exists for itself,
the objective world immediately has
the form of immediacy or of being,
just as that concept is to itself
at first only the abstract concept of itself,
is still shut up within itself.
The concept is therefore only as form,
of which only its simple determinations
of universality and particularity are
the reality that it possesses within,
while the singularity or the determinate determinateness,
the content, is received by it from the outside.

CHAPTER 3

The absolute idea

YS III.21

    kaya-rupa-samyamat tad-grahya-sakti-stambhe caksu-prakasa-asamprayoge 'ntardhanam

YS III.22

    etena sabda-adi-antardhanam uktam

The absolute idea has shown itself to be
the identity of the theoretical and the practical idea,
each of which, of itself still one-sided, possesses the idea
only as a sought-for beyond and unattained goal;
each is therefore a synthesis of striving,
each possessing as well as not possessing the idea within it,
passing over from one thought to the other
without bringing the two together
but remaining fixed in the contradiction of the two.
The absolute idea, as the rational concept
that in its reality only rejoins itself,
is by virtue of this immediacy of its objective identity,
on the one hand, a turning back to life;
on the other hand, it has equally
sublated this form of its immediacy
and harbors the most extreme opposition within.
The concept is not only soul,
but free subjective concept
that exists for itself
and therefore has personality,
the practical objective concept
that is determined in and for itself
and is as person impenetrable, atomic subjectivity
but which is not, just the same, exclusive singularity;
it is rather explicitly universality and cognition,
and in its other has its own objectivity for its subject matter.
All the rest is error, confusion, opinion,
striving, arbitrariness, and transitoriness;
the absolute idea alone is being, imperishable life,
self-knowing truth, and is all truth.

It is the sole subject matter and content of philosophy.
Since it contains all determinateness within it,
and its essence consists in returning
through its self-determination
and particularization back to itself,
it has various shapes,
and the business of philosophy
is to recognize it in these.
Nature and spirit are in general
different modes of exhibiting its existence,
art and religion its different modes of apprehending itself
and giving itself appropriate existence.
Philosophy has the same content and the same
purpose as art and religion,
but it is the highest mode of apprehending the absolute idea,
because its mode, that of the concept, is the highest.
Hence it seizes those shapes
of real and ideal finitude,
as well of infinity and holiness,
and comprehends them and itself.
The derivation and cognition of these particular modes are
now the further business of the particular philosophical sciences.
Also the logicality of the absolute idea
can be called a mode of it;
but mode signifies a particular kind,
a determinateness of form,
whereas the logicality of the idea is
the universal mode in which
all particular modes are sublated and enveloped.
The logical idea is the idea itself in its pure essence,
the idea which is enclosed in
simple identity within its concept
and in reflective shining has as yet
to step into a form-determinateness.
The Logic thus exhibits the
self-movement of the absolute idea
only as the original word,
a word which is an utterance,
but one that in being externally uttered
has immediately vanished again.
The idea is, therefore, only in this
self-determination of apprehending itself;
it is in pure thought, where difference is not yet otherness,
but is and remains perfectly transparent to itself.
The logical idea thus has itself,
as the infinite form, for its content,
form that constitutes the opposite of content
inasmuch as the latter is the form determination
that has withdrawn into itself
and has been so sublated in identity
that this concrete identity stands over against
the identity developed as form;
the content has the shape of an other
and of something given as against the form
that as such stands simply in reference,
and whose determinateness is posited
at the same time as reflective shine.
More exactly, the absolute idea itself has
only this for its content,
namely that the form determination is
its own completed totality, the pure content.
Now the determinateness of the idea
and the entire course traversed by this determinateness
has constituted the subject matter of the science of logic,
and out of this course the absolute idea has come forth for itself;
thus to be for itself, however, has shown itself to amount to this,
namely that determinateness does not
have the shape of a content,
but that it is simply as form,
and that accordingly the idea is
the absolutely universal idea.
What is left to be considered here, therefore,
is thus not a content as such,
but the universal character of its form
that is, method.

Method may appear at first to be just
the manner in which cognition proceeds,
and this is in fact its nature.
But as method this manner of proceeding is
not only a modality of being determined in and for itself;
it is a modality of cognition,
and as such is posited as determined
by the concept and as form,
since form is the soul of all objectivity
and all otherwise determined content has its truth in form alone.
If the content is again assumed as given to the method
and of a nature of its own, then method, so understood,
is just like the logical realm in general,
a merely external form.
But against this assumption appeal can be made,
not only to the fundamental concept of what constitutes logic,
but to the entire logical course
in which all the shapes of a given content
and of objects came up for consideration.
This course has shown the transitoriness
and the untruth of all such shapes;
also that no given object is capable of being the foundation
to which the absolute form would relate
as only an external and accidental determination;
that, on the contrary, it is the absolute form
that has proved itself to be the absolute foundation
and the ultimate truth.
For this course the method has resulted as
the absolutely self-knowing concept,
as the concept that has the absolute,
both as subjective and objective,
as its subject matter,
and consequently as the pure correspondence
of the concept and its reality,
a concrete existence that is the concept itself.

Accordingly, what is to be considered as method here is
only the movement of the concept itself.
We already know the nature of this movement,
but it now has, first, the added significance
that the concept is all,
and that its movement is the universal absolute activity,
the self-determining and self-realizing movement.
The method is therefore to be acknowledged as the universal,
internal and external mode, free of restrictions,
and as the absolutely infinite force to which
no object that may present itself as something external,
removed from reason and independent of it,
could offer resistance,
or be of a particular nature opposite to it,
and could not be penetrated by it.
It is therefore soul and substance,
and nothing is conceived and known in its truth
unless completely subjugated to the method;
it is the method proper to each and every fact
because its activity is the concept.
This is also the truer meaning of its universality;
according to the universality of reflection,
it is taken only as the method for all things;
but according to the universality of the idea,
it is both the manner of cognition,
of the concept subjectively aware of itself,
and the objective manner, or rather the
substantiality of things, that is,
of concepts as they first appear as others
to representation and reflection.
It is therefore not only the highest force of reason,
or rather its sole and absolute force,
but also reason's highest and sole impulse
to find and recognize itself through itself in all things.
Second, here we also have the distinction of
the method from the concept as such,
the particularization of the method.
As the concept was considered for itself,
it appeared in its immediacy;
the reflection, or the concept considering it,
fell on the side of our knowledge.
The method is this knowledge itself,
for which the concept is not only as subject matter
but is as its own subjective act,
the instrument and the means of cognitive activity,
distinct from this activity
and yet the activity's own essentiality.
In cognition as enquiry,
the method likewise occupies the position of an instrument,
as a means that stands on the side of the subject,
connecting it with the object.
The subject in this syllogism is one extreme,
the object is the other,
and in conclusion the subject unites
through its method with the object
without however uniting with itself there.
The extremes remain diverse,
because subject, method, and object are not
posited as the one identical concept;
the syllogism is therefore always the formal syllogism;
the premise in which the subject posits the form
on its side as its method is an immediate determination
and contains therefore the determinations of the form
as we have seen, of definition, division, and so forth
as matters of fact found ready-made in the subject.
In true cognition, on the contrary, method is
not only an aggregate of certain determinations,
but the determinateness in-and-for-itself of the concept,
and the concept is the middle term only
because it equally has the significance of the objective;
in the conclusion, therefore, the objective does not attain
only an external determinateness by virtue of the method,
but is posited rather in its identity with the subjective concept.

1. Accordingly, what constitutes the method are
the determinations of the concept itself and their connections,
and these we must now examine in the significance
that they have as determinations of the method.
In this, we must begin from the beginning.
We spoke of this beginning at the very beginning of the Logic,
and also in connection with subjective cognition,
and we showed that, when not performed arbitrarily
and in the absence of categorial sensitivity,
though it may seem to present many difficulties,
it is nevertheless of an extremely simple nature.
Because it is the beginning, its content is an immediate,
but one that has the meaning and the form of abstract universality.
Or be it a content of being, or of essence or of the concept,
inasmuch as it is something immediate, it is assumed,
found in advance, assertoric.
But first of all it is not an immediate of
sense-intuition or of representation,
but of thought, which because of its immediacy can
also be called a supersensuous, inner intuiting.
The immediate of sense-intuition is a manifold and a singular.
Cognition, on the contrary, is a thinking that conceptualizes;
its beginning, therefore, is also only in the element of thought,
a simple and a universal.
We spoke of this form earlier, in connection with definition.
At the beginning of finite cognition universality is likewise
recognized as an essential determination,
but only as thought and concept
determination in opposition to being.
In fact this first universality is an immediate universality,
and for that reason it has equally the significance of being,
for being is precisely this abstract self-reference.
Being has no need of further derivation,
as if it came to the abstract element of definition
only because taken from the intuition of the senses or elsewhere,
and in so far as it can be pointed at.
This pointing and deriving involve a mediation
that is more than a mere beginning,
and is a mediation of a kind that does not
belong to the comprehension of thought,
but is rather the elevation of representation,
of empirical and ratiocinative consciousness,
to the standpoint of thinking.
According to the currently accepted opposition
of thought, or concept, and being,
it passes as a very important truth that
no being belongs as yet to thought as thought,
and that being has a ground of its own independent of thought.
But the simple determination of being is in itself so poor that,
if for that reason alone, not much fuss ought to be made about it;
the universal is immediately itself this immediate
because, as abstract, it is also the
abstract self-reference which is being.
In fact, the demand that being should be exhibited has
a further, inner meaning in which more is at issue
than just this abstract determination;
implied in it is the demand for the realization of the concept,
a realization that is missing at the beginning itself
but is rather the goal and the business of the
entire subsequent development of cognition.
Further, inasmuch as the content of the beginning is
to be justified and authenticated as something true or correct
by being exhibited in inner or outer perception,
it is no longer the form of universality as such
that is meant, but its determinateness,
about which more in a moment.
The authentication of the determinate content
with which the beginning is made seems to lie behind it,
but is in fact to be regarded as an advance,
in so far as it is a matter of conceptual cognition.

The beginning, therefore, has for the method no other determinateness
than that of being the simple and universal;
this is precisely the determinateness that makes it deficient.
Universality is the pure, simple concept,
and the method, as the consciousness of this concept,
is aware that universality is only a moment
and that in it the concept is still not determined in and for itself.
But with this consciousness that would want to carry
the beginning further only for the sake of method,
the method is only a formal procedure posited in external reflection.
Where the method, however, is the objective and immanent form,
the immediate character of the beginning must be
a lack inherent in the beginning itself,
which must be endowed with the
impulse to carry itself further.
But in the absolute method the universal
has the value not of a mere abstraction
but of the objective universal, that is,
the universal that is in itself the concrete totality,
but a totality as yet not posited, not yet for itself.
Even the abstract universal is as such,
when considered conceptually, that is, in its truth,
not just anything simple, but is, as abstract,
already posited afflicted by a negation.
For this reason also there is nothing so simple and so abstract,
be it in actuality or in thought, as is commonly imagined.
Anything as simple as that is a mere presumption
that has its ground solely in the lack of
awareness of what is actually there.
We said earlier that the beginning is
made with the immediate;
the immediacy of the universal is the same as
what is here expressed as the in-itself
that is without being-for-itself.
One may well say, therefore, that every
beginning must be made with the absolute,
just as every advance is only the exposition of it,
in so far as implicit in existence is the concept.
But because the absolute exists
at first only implicitly, in itself,
it equally is not the absolute
nor the posited concept,
and also not the idea,
for the in-itself is only
an abstract, one-sided moment,
and this is what they are.
The advance is not, therefore, a kind of superfluity;
this is what it would be if that which is
at the beginning were already the absolute;
the advance consists rather in this,
that the universal determines itself
and is the universal for itself,
that is, equally a singular and a subject.
Only in its consummation is it the absolute.

It may also be mentioned that a beginning
which is in itself a concrete totality may as such
also be free and its immediacy have
the determination of an external existence;
the germ of anything living,
and subjective purpose in general,
have shown themselves to be such beginnings;
hence both are themselves impulses.
The non-spiritual and inanimate, on the contrary,
are the concrete concept only as real possibility;
cause is the highest stage in which the concrete concept has,
as the beginning in the sphere of necessity,
an immediate existence;
but it is not yet a subject
that maintains itself as such
in the course of its effective realization.
The sun, for instance,
and in general all things inanimate,
are determinate concrete existences
in which real possibility remains an inner totality;
the moments of the latter are not posited in them
in subjective form and therefore,
in so far as they are realized,
they attain concrete existence
through other corporeal individuals.

2. The concrete totality which makes the beginning
possesses as such, within it, the beginning of
the advance and development.
As concrete, it is differentiated in itself,
but because of its initial immediacy, this first
differentiation is to start with a diversity.
However, as self-referring universality, as subject,
the immediate is also the unity of this diversity.
This reflection is the first stage of the forward movement
the emergence of non-indifference, judgment, and determining in general.
Essential is that the method find, and recognize,
the determination of the universal within it.
Whatever in this abstractive generation of the universal
is left out of the concrete is then picked up, still externally,
by the finite cognition of the understanding.
This is how the latter operates.
The absolute method, on the contrary, does not behave
in this manner of external reflection
but takes the determinate from its subject matter,
for it is itself its immanent principle and its soul.
This is what Plato demanded of cognition,
that it should consider things in and for themselves;
on the one hand, that it should consider them in their universality;
on the other hand, that it should not stray away from them
while it grasps at circumstances, examples, and comparisons,
but, on the contrary, should keep only them in view before it
and bring to consciousness what is immanent in them.
To this extent the method of absolute cognition is analytic.
That the method finds the further determinations
of its initial universal simply and solely in this universal,
constitutes the concept's absolute objectivity,
of which the method is the certainty.
Equally so, however, is the method synthetic,
for its subject matter,
while immediately determined as the simple universal,
through the determinateness which it has
in its very immediacy and universality,
proves to be an other.
Yet this connection in diversity
that the subject matter is thus in itself,
is no longer a synthesis as understood in finite cognition;
the no less thoroughly analytic determination of the subject matter,
the fact that the connection is within the concept,
already distinguishes it fully from the latter synthesis.

This no less synthetic than analytic moment of the judgment
through which the initial universal determines itself
from within itself as the other of itself
is to be called the dialectical moment.
Dialectic is one of those ancient sciences
that have been the most misjudged in
the metaphysics of the moderns,
and in general also by popular philosophy,
both ancient and recent.
Diogenes Laërtius said of Plato that,
just as Thales was the founder of natural philosophy
and Socrates of moral philosophy,
so Plato was the founder of the third of
the sciences that belong to philosophy,
of dialectic, a contribution for which
he was highly esteemed by the ancients
but that often goes quite unnoticed
by those who have the most to say about him.
Dialectic has often been regarded as an art,
as if it rested on a subjective talent
and did not belong to the objectivity of the concept.
What shape it received in Kantian philosophy,
and with what result, has already been
indicated in representative examples of that philosophy's view.
It must be regarded as an infinitely important step
that dialectic is once more being
recognized as necessary to reason,
although the result that must be drawn from it
is the opposite than Kant drew.

When dialectic is not presented, as it generally is,
as something incidental, it usually assumes
the following more precise form.
It is shown of a subject matter or other
(for instance: world, movement, point, and so on)
that a certain determination accrues to it
(for instance, in the order of the just mentioned examples:
finitude in space or time, being at this place,
absolute negation of space),
also the opposite determinations can then
just as necessarily be shown to accrue to it
(for example: infinity in space and time,
not-being at this place, reference to space and hence spatiality).
The older Eleatic school directed
its dialectic especially against motion;
Plato commonly did it against accepted
notions and concepts of his time,
in particular those of the Sophists,
but also against the pure categories
and the determinations of reflection;
the later and more sophisticated form of skepticism
extended it not only to the immediate so-called facts of consciousness
and the maxims of ordinary life,
but also to all scientific concepts.
Now the conclusion drawn from this kind of dialectic is
in general the contradiction and nullity of the asserted claims.
But this can happen in two ways either in the objective sense,
that the subject matter that thus contradicts itself
internally cancels itself and is a non-thing
(this was, for instance, the conclusion of the Eleatics,
who denied the truth of the world, of movement, of the point);
or in the subjective sense, that cognition is deficient.
Now understood in this last subjective sense,
the conclusion may be taken in two further ways.
It may mean that it is this dialectic itself
that generates the artifice of an illusion.
This is the common view of the so-called healthy
common sense that takes its stand on the evidence of the senses
and on customary notions and claims,
at times quietly, like Diogenes the cynic did,
who demonstrated the vacuity of the dialectic of motion
by silently walking up and down;
but often by getting itself all worked up,
declaring that dialectic is mere foolery or,
when important ethical matters are at issue,
the criminal attempt at unsettling essentially solid norms
and providing excuses for the wicked,
a view we see directed in the Socratic dialectic
against that of the Sophists, with an ire that,
turned into the opposite direction,
even cost Socrates his life.
As for the vulgar refutation that opposes to thinking,
as Diogenes did, sensuous consciousness
and in this latter believes that it finds the truth,
this we must leave to itself;
but in so far as dialectic sublates ethical determinations,
we must have confidence in reason that it will
know how to reinstate them,
but reinstate them in their truth
and in the consciousness of their right,
though also of their limitations.
Yet another view is that the
result of subjective nullity has
nothing to do with dialectic itself,
but that it affects the cognition
against which it is directed
and, in the view of skepticism
and likewise of the Kantian philosophy,
cognition in general.

The fundamental prejudice here is that
dialectic has only a negative result,
a point about which more in a moment.
First, regarding the said form in which dialectic
usually makes its appearance,
it is to be observed that according to that form
the dialectic and its result affect
a subject matter which is previously assumed
or also the subjective cognition of it,
and declare either the latter or the subject matter
to be null and void,
while, on the contrary, no attention is given to
the determinations which are exhibited in the subject matter
as in a third thing and presupposed as valid for themselves.
To have called attention to this uncritical procedure has
been the infinite merit of the Kantian philosophy,
and in so doing to have given the impetus to the restoration
of logic and dialectic understood as the examination of
thought determinations in and for themselves.
The subject matter, as it is apart from thought and conceptualization,
is a picture representation or also a name;
it is in the determinations of thought and of the concept
that it is what it is.
In fact, therefore, everything rests on these determinations;
they are the true subject matter and content of reason,
and anything else that might be understood by subject matter and content
in distinction from them has meaning only through them and in them.
It must not therefore be taken as the fault of a subject matter
or of the cognition that these determinations,
because of what they are and the way they are externally joined,
prove to be dialectical.
On this assumption, the subject matter and the cognition
are imagined to be a subject on which
the determinations are brought to bear,
in the form of predicates, properties,
or self-subsistent universals,
as fixed and independently correct,
so that these determinations are
brought into dialectical relations
and incur contradiction only by
extraneous and contingent conjunction
in and by a third thing.
But this kind of external and fixed
subject of imagination and understanding,
and also these abstract determinations,
cannot be regarded as ultimates,
as secure and permanent substrates.
On the contrary, they are to be regarded as themselves immediate,
precisely the kind of presuppositions and starting points which,
as we have shown above, must succumb to dialectic in and for themselves,
because they are to be taken as in themselves the concept.
The same applies to all oppositions that are assumed as fixed,
as for example the finite and the infinite, the singular and the universal.
These are not in contradiction through some external conjoining;
on the contrary, as an examination of their nature shows,
they are a transition in and for themselves;
the synthesis and the subject in which they appear is
the product of their concept's own reflection.
If a consideration that avoids
the concept stops short at their external relation,
isolates them and leaves them as fixed presuppositions,
it is the concept that,  on the contrary,
will fix its sight on them,
move them as their soul and bring out their dialectic.

Now this is the very standpoint indicated above
from which a universal prius,
considered in and for itself,
proves to be the other of itself.
Taken quite generally, this determination
can be taken to mean that what is at first immediate
is therewith posited as mediated, as referred to an other,
or that the universal is posited as a particular.
The second universal that has thereby arisen is
thus the negative of that first
and, in view of subsequent developments, the first negative.
From this negative side, the immediate has perished in the other;
but the other is essentially not an empty negative,
the nothing which is normally taken to be the result of dialectic,
but is rather the other of the first,
the negative of the immediate;
it is therefore determined as the mediated,
contains as such the determination of the first in it.
The first is thus essentially preserved and contained also in the other.
To hold fast to the positive in its negative,
to the content of the presupposition in the result,
this is the most important factor in rational cognition;
what is more, it takes only the simplest of reflections
to be convinced of the absolute truth and necessity of this requirement,
and as for examples of proofs that testify to this,
the whole Logic consists of such proofs.

So what we now have, taken first or also immediately, is the mediated,
also a simple determination,
for the first has perished in it,
and only the second is therefore at hand.
Now since the first is contained in the second,
and this second is the truth of the first,
this unity of the two can be expressed
in the form of a proposition in which
the immediate is placed as the subject
but the mediated as its predicate;
for example, “the finite is infinite,”
“one is many,” “the singular is the universal.”
The inadequacy of the form of such
propositions and judgments is however obvious.
In connection with judgment it was shown that its form in general,
and most of all the immediate form of the positive judgment,
is incapable of holding within its grasp
the speculative content and the truth.
Its closest complement, the negative judgment,
would have to be brought in at least in equal measure.
In judgment the first, as subject, conveys the
reflective semblance of an independent subsistence,
whereas it is in fact sublated in the predicate as in its other;
this negation is indeed contained in
the content of the above propositions,
but their positive form contradicts the content;
consequently, what is contained in them is not posited
whereas this was precisely the intent
behind the use of a proposition.

The second determination, the negative or mediated determination,
is moreover at the same time the one that mediates.
At first it may be taken as a simple determination,
but its truth is that it is a reference or relation;
for it is the negative, but the negative of the positive,
and it includes this positive within itself.
It is the other, therefore, not of a one to which it is indifferent;
in that case it would not be an other, nor a reference or relation.
It is rather the other in itself, the other of an other;
hence it includes its own other within itself
and is consequently the contradiction, the posited dialectic, of itself.
Because the first or the immediate is the concept in itself or implicitly,
and therefore is the negative also only implicitly,
the dialectical moment in it consists in the positing
of the difference that is implicitly contained in it.
The second is on the contrary itself the determinate,
the difference or relation;
hence the dialectical moment consists in its case
in the positing of the unity contained within it.
For this reason, if the negative, the determinate, relation, judgment,
and all the determinations falling under this second moment,
do not appear by themselves already as contradiction,
as dialectical, this is solely a defect on the part
of thinking that fails to bring its thoughts together.
For the material, the opposed determinations in one connection,
are already posited, already present for thought.
But formal thinking makes identity its law,
lets the contradictory content that it has before it
fall into the sphere of representation, in space and time,
where the contradictory is held in external moments,
next to and following each other,
parading before consciousness without reciprocal contact.
The firm principle that formal thinking lays down for itself here
is that contradiction cannot be thought.
But in fact the thought of contradiction is
the essential moment of the concept.
Formal thought does in fact think it,
only it at once looks away from it
and stating its principle it only passes over
from it into abstract negation.

Now the negativity just considered constitutes
the turning point of the movement of the concept.
It is the simple point of the negative self-reference,
the innermost source of all activity,
of living and spiritual self-movement;
it is the dialectical soul which everything true possesses
and through which alone it is true;
for on this subjectivity alone rests
the sublation of the opposition
between concept and reality,
and the unity which is truth.
The second negative at which we have arrived,
the negative of the negative,
is this sublating of contradiction,
and it too, just like contradiction,
is not an act of external reflection;
for it is on the contrary the innermost,
objective moment of the life of spirit
by virtue of which a subject is a person, is free.
The self-reference of the negative is to be regarded
as the second premise of the entire syllogism.
If the terms analytic and synthetic are used as opposites,
the first premise may be regarded as the analytic moment,
for in it the immediate relates to its other immediately
and therefore passes over, or rather has passed over,
into it though this connection, as already remarked,
is for this very reason also synthetic,
for it is its other that it passes over into.
The second premise considered here
may be defined as synthetic,
because it is the connection of
the differentiated, as differentiated,
to that from which it is differentiated.
Just as the first premise is the moment
of universality and communication,
so is the second determined by singularity;
a singularity which in referring to the other is
at first exclusive, for itself, and different.
The negative appears as the mediating factor,
because it holds itself and the immediate
of which it is the negation within itself.
In so far as these two determinations are taken
as referring to each other externally
in some relation or other,
the negative is only the formal mediating factor;
but, as absolute negativity,
the negative moment of absolute mediation is
the unity which is subjectivity and soul.

In this turning point of the method,
the course of cognition returns at the
same time back into itself.
This negativity is as self-sublating contradiction
the restoration of the first immediacy, of simple universality;
for the other of the other, the negative of the negative,
is immediately the positive, the identical, the universal.
In the whole course, if one at all cares to count,
this second immediate is third to the first immediate and the mediated.
But it is also third to the first or formal negative
and to the absolute negativity or second negative;
now in so far as that first negative is already the second
term, the term counted as third can also be counted as fourth,
and instead of a triplicity, the abstract form
may also be taken to be a quadruplicity;
in this way the negative or the difference
is counted as a duality.
The third or the fourth is in general
the unity of the first and the second moment,
of the immediate and the mediated.
That it is this unity,
or that the entire form of the method is a triplicity,
is indeed nothing but the merely superficial,
external side of cognition;
but to have also demonstrated this superficiality,
and to have done it in the context of a specific application
(for the abstract form of number has been around for a long time,
as is well known, but without conceptual comprehension
and therefore without any result)
is again to be regarded as an infinite merit of the Kantian philosophy.
The syllogism, or the threefold, has always been recognized
to be the universal form of reason;
but it has had in general the value of a wholly external form
that does not determine the nature of the content;
moreover, since in its formalism it gets caught up
in the understanding's determination of mere identity,
it lacks the essential dialectical moment of negativity;
and yet this moment enters into the triplicity of the determinations,
because the third term is the unity of the two first determinations
and these, since they are diverse, can be in unity only as sublated.
Formalism, it is true, has also seized hold of triplicity,
attending to its empty schema;
the shallow nonsense and the barrenness of
the so-called construction of modern philosophy,
that consists in nothing but fastening that formal schema
everywhere for the sake of external order,
with no concept or immanent determination,
has rendered that form tedious and has given it a bad name.
Yet the insipidity of this use cannot rob it of its inner worth,
and the fact that the shape of reason was discovered,
albeit without conceptual comprehension at first,
is always to be highly valued.

Now, on closer examination, the third is the immediate,
but the immediate through sublation of mediation,
the simple through the sublating of difference,
the positive through the sublating of the negative;
it is the concept that has realized itself through its otherness,
and through the sublating of this reality has rejoined itself
and has restored its absolute reality,
its simple self-reference.
This result is therefore the truth.
It is just as much immediacy as mediation
though these forms of judgments,
that the third is immediacy and mediation,
or that it is the unity of the two,
are not capable of grasping it,
for it is not a dormant third
but, exactly like this unity,
self-mediating movement and activity.
Just as that with which we began was the universal,
so the result is the singular, the concrete, the subject;
what the former is in itself, the latter is now equally for itself:
the universal is posited in the subject.
The two first moments of triplicity are abstract,
untrue moments that are dialectical for that very reason,
and through this their negativity make themselves into the subject.
For us at first, the concept itself is
both the universal that exists in itself
and the negative that exists for itself,
and also the third term that exists in and for itself,
the universal that runs through all the moments of the syllogism;
but this third is the conclusion in which
the concept mediates itself with itself through its negativity
and is thereby posited for itself as the universal
and the identity of its moments.

Now this result, as the whole that has withdrawn into itself
and is identical with itself, has given itself again
the form of immediacy.
Consequently, it is now itself all that
the starting point had determined itself to be.
As simple self-reference it is a universal,
and in this universal the negativity that
constituted its dialectic and mediation has
likewise withdrawn into simple determinateness,
which can again be a beginning.
It may seem at first that this cognition of the result
is an analysis of it
and would therefore have to dissect these determinations again,
and the course that it went through in order to come to be
the course that we have examined.
But if the subject matter were in fact treated analytically in this manner,
it would belong to that stage of the idea considered above,
a mode of cognition that searches for its subject matter
and only states of it what it is,
without the necessity of its concrete identity and of its concept.
But the method of truth that comprehends the subject matter,
though analytic as we have seen,
since it remains strictly within the concept,
is however equally synthetic,
for through the concept the subject matter is
determined  as dialectical and as other.
On the new foundation that the result
has now constituted as the subject matter,
the method remains the same as in the preceding subject matter.
The difference concerns solely the status of the foundation as such;
although it is certainly still a foundation,
its immediacy is only form,
since it was a result as well;
hence its determinateness as content is
no longer something merely taken up
but is deduced and proved.

It is here that the content of cognition first enters
as such into the circle of consideration,
because as deduced it now belongs to the method.
The method itself expands with this moment into a system.
With respect to content, the beginning has to be
for the method at first wholly indeterminate;
to this extent the method appears as the merely formal soul,
for which and by which the beginning was determined
simply and solely only according to form,
that is to say, as the immediate and universal.
In the course of the movement we have indicated,
the subject matter has received a determinateness for itself
and this determinateness is a content,
for the negativity that has withdrawn
into simplicity is the sublated form,
and stands as simple determinateness over against its development,
and in the first instance against its very opposition to universality.

Now since this determinateness is the
proximate truth of the indeterminate beginning,
it denounces the incompleteness of the latter,
and it also denounces the method itself
which, starting from that beginning, was only formal.
This can now be expressed as the henceforth
determinate demand that the beginning,
since as against the determinateness of the result
it is itself something determinate,
ought to be taken not as immediate,
but as mediated and deduced.
This may appear as the demand for
an infinite retrogression in proof and deduction;
just as from the newly obtained beginning a result
likewise emerges as the method runs its course,
so that the movement would roll on forwards to infinity as well.

It has been repeatedly shown that
the infinite progression as such belongs
to a reflection void of concept;
the absolute method, which has the concept
for its soul and content, cannot lead into it.
Even such beginnings as being, essence, universality,
may seem at first to be of the kind that possess
the full universality and complete absence of content
that is required for an entirely formal beginning,
such as the beginning is supposed to be,
and therefore not to require or allow,
as absolutely first beginnings, further regress.
Since they refer purely to themselves,
they are immediate and indeterminate,
and so they do not of course have in them the difference
which is straightaway posited in some other beginning
between the universality of its form and its content.
But the very indeterminacy which these logical beginnings have
as their sole content is what constitutes their determinateness;
this determinateness consists in their negativity,
as sublated mediation;
the particularity of this negativity gives
a particularity also to their indeterminacy,
and it is by virtue of it that
being, essence, and universality, are differentiated.
Now the determinateness that accrues to them
when taken for themselves is their immediate determinateness,
and this is just as immediate as that of any content
and in need, therefore, of derivation;
for the method it is a matter of indifference
whether the determinateness is taken
as determinateness of form or of content.
That it gives itself a determination by
the first of its results does not mean that,
in fact, it is thereby set on a new footing;
it remains neither more nor less formal than before.
For since the method is the absolute form,
the concept that knows itself and everything as concept,
there is no content that would stand out over against it
and determine it as a one-sided external form.
Hence, just as the lack of content of the said beginnings
does not make them absolute beginnings,
so too it is not the content that would as such lead
the method into the infinite progress forwards or backwards.
In one respect, the determinateness that the method generates for itself
in its result is the moment through which it is self-mediation
and converts the immediate into a mediated beginning.
But conversely, it is through that determinateness
that this mediation of the method runs its course;
it goes through a content, as through a seeming other of itself,
back to its beginning, in such a way that it does not merely
restore that beginning, albeit as determinate,
but that the result is equally the sublated determinateness,
and hence also the restoration of the first immediacy in which it began.
This it accomplishes as a system of totality.
We now have to consider it in this determination.

The determinateness which was the result is,
as we have shown, itself a new beginning
because of the form of simplicity
into which it has withdrawn;
since this beginning is distinguished
from the one preceding it by this very determinateness,
cognition rolls onwards from content to content.
First of all, this forward movement determines itself
in that it begins from simple determinacies,
and the following become ever richer and more concrete.
For the result contains its beginning and its course
has enriched it with a new determinateness.
The universal constitutes the foundation;
the advance is not to be taken, therefore,
as a flowing from other to other.
In the absolute method,
the concept maintains itself in its otherness,
the universal in its particularization,
in judgment and reality;
at each stage of further determination,
the universal elevates the whole mass of its preceding content,
not only not losing anything through its dialectical advance,
or leaving it behind, but, on the contrary,
carrying with itself all that it has gained,
inwardly enriched and compressed.

This expansion may be regarded as
the moment of content,
and in the whole as the first premise;
the universal is communicated to the wealth of content,
is immediately received in it.
But the relation has also a second,
negative or dialectical side.
The enrichment proceeds in the
necessity of the concept,
it is contained by it,
and every determination is
a reflection into itself.
Each new stage of exteriorization,
that is, of further determination,
is also a withdrawing into itself,
and the greater the extension,
just as dense is the intensity.
The richest is therefore
the most concrete and the most subjective,
and that which retreats to the simplest depth is
the mightiest and the most all-encompassing.
The highest and most intense point is
the pure personality that,
solely by virtue of the
absolute dialectic which is its nature,
equally embraces and holds everything within itself,
for it makes itself into the supremely free,
the simplicity which is
the first immediacy and universality.

It is in this manner that each step of the advance
in the process of further determination,
while getting away from the indeterminate beginning,
is also a getting back closer to it;
consequently, that what may at first appear to be different,
the retrogressive grounding of the beginning
and the progressive further determination of it,
run into one another and are the same.
The method, which thus coils in a circle,
cannot however anticipate in a temporal development
that the beginning is as such already something derived;
sufficient for an immediate beginning is that it be simple universality.
Inasmuch as this is what it is, it has its complete condition;
and there is no need to deprecate the fact that
it may be accepted only provisionally and hypothetically.
Whatever might be adduced against it about
the limitations of human cognition;
about the need to reflect critically
on the instrument of cognition
before getting to the fact itself;
all these are themselves presuppositions,
concrete determinations that as such carry
with them the demand for mediation and grounding.
Therefore, since they formally have no advantage
over beginning with the fact itself as they protest against,
and, because of their more concrete content,
are on the contrary all the more in need of derivation,
singling them out for special attention
is to be considered as empty presumption.
They have an untrue content, for they make into something
incontestable and absolute what is known to be finite and untrue,
namely a restricted cognition determined as form and instrument
in opposition to its content;
this untrue cognition is itself also the form,
the retroactive search for grounds.
The method of truth also knows that the beginning is incomplete,
because it is a beginning;
but at the same time it knows that
this incompleteness is necessary,
because truth is but the coming-to-oneself
through the negativity of immediacy.
The impatience that would merely transcend the determinate
be it called beginning, object, the finite,
or in whatever other form it is otherwise taken
in order that one would find oneself immediately in the absolute,
has nothing before it as cognition but the empty negative, the abstract infinite.
Or what it has before it is a presumed absolute,
presumed because not posited, not comprehended;
comprehended it will be only through the mediation of cognition,
of which the universal and immediate are a moment,
and as for the truth itself,
it resides only in the extended course of mediation and at the end.
To meet the subjective need and the impatience
that come with not knowing,
one may well provide an overview of the whole in advance
by means of a division for reflection that,
in the manner of finite cognition,
gives the particular of the universal as already there, to be waited
for as the science progresses.
Yet this affords nothing more than a picture
for representation;
for the true transition from the universal
to the particular and to the whole
which is determined in and for itself
and in which that first universal is in truth
itself again a moment;
this transition is alien to the division of reflection
and is the exclusive mediation of science itself.

By virtue of the nature of the method just indicated,
the science presents itself as a circle that winds around itself,
where the mediation winds the end back to the beginning
which is the simple ground;
the circle is thus a circle of circles,
for each single member ensouled by the method
is reflected into itself so that, in returning to the beginning
it is at the same time the beginning of a new member.
Fragments of this chain are the single sciences,
each of which has a before and an after
or, more accurately said,
has in possession only the before
and in its conclusion points to its after.

So the logic also has returned in the absolute idea
to this simple unity which is its beginning;
the pure immediacy of being,
in which all determination appears at first
as extinguished or removed by abstraction,
is the idea that through mediation,
that is, the sublation of mediation,
has come to the likeness corresponding to it.
The method is the pure concept
that only relates to itself;
it is, therefore, the simple self-reference which is being.
But it now is also the fulfilled concept,
the concept that comprehends itself conceptually,
being as the concrete and just as absolutely intensive totality.

In conclusion, there remains only this to be said of this idea,
that in it, in the first place,
the science of logic has apprehended its own concept.
In the sphere of being, at the beginning of its content,
its concept appears as a knowledge external to
that content in subjective reflection.
But in the idea of absolute cognition,
the concept has become the idea's own content.
The idea is itself the pure concept
that has itself as its subject matter
and which, as it runs itself as subject matter
through the totality of its determinations,
builds itself up to the entirety of its reality,
to the system of science,
and concludes by apprehending this
conceptual comprehension of itself,
hence by sublating its position
as content and subject matter
and cognizing the concept of science.
In second place, this idea is still logical;
it is shut up in pure thought,
the science only of the divine concept.
Its systematic exposition is of course itself a realization,
but one confined within the same sphere.
Because the pure idea of cognition is
to this extent shut up within subjectivity,
it is the impulse to sublate it,
and pure truth becomes as final result
also the beginning of another sphere and science.
It only remains here to indicate this transition.

The idea, namely, in positing itself
as the absolute unity of the pure concept and its reality
and thus collecting itself in the immediacy of being,
is in this form as totality:  nature.
This determination, however, is nothing that has become,
is not a transition, as was the case above
when the subjective concept in its totality becomes objectivity,
or the subjective purpose becomes life.
The pure idea into which the determinateness
or reality of the concept is itself
raised into concept is rather an
absolute liberation for which
there is no longer an immediate determination
which is not equally posited and is not concept;
in this freedom, therefore, there is
no transition that takes place;
the simple being to which the idea determines itself
remains perfectly transparent to it:
it is the idea that in its determination remains with itself.
The transition is to be grasped, therefore,
in the sense that the idea freely discharges itself,
absolutely certain of itself and internally at rest.
On account of this freedom, the form of
its determinateness is just as absolutely free:
the externality of space and time absolutely existing
for itself without subjectivity.
Inasmuch as this externality is only
in the abstract determinateness of being
and is apprehended by consciousness,
it is as mere objectivity and external life;
within the idea, however, it remains in
and for itself the totality of the concept,
and science in the relation of
divine cognition to nature.
But what is posited by this first resolve
of the pure idea to determine itself as external idea
is only the mediation out of which the concept,
as free concrete existence that from externality
has come to itself, raises itself up,
completes this self-liberation in the science of spirit,
and in the science of logic finds the highest concept of itself,
the pure concept conceptually comprehending itself.
