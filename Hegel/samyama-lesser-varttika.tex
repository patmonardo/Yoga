
The doctrine of the concept

The standpoint of the concept is in general that of absolute idealism,
and philosophy is knowing conceptually.
It is conceptual knowing insofar as everything that ordinary consciousness
regards as an entity, and in its immediacy as independent, is known merely as
an ideal moment in it.
In logic at the level of the understanding the concept is
usually considered as a mere form of thinking and,
more precisely, as a universal representation.
The claim, so often repeated from the side of sentiment and the heart,
that concepts as such are something dead, empty, and abstract, refers to
this low-level construal of the concept.
Meanwhile, just the opposite holds and the concept is instead the principle of
all life and thereby, at the same time, something absolutely concrete.
That such is the case has emerged as the result of the entire logical movement
up to this point and hence does not need first to be proven here.
As far as the opposition of form and content is concerned in this connection,
namely, with respect to the concept as allegedly merely formal,
this opposition, like all the other oppositions held fast by reflection, is already
behind us as something overcome dialectically, that is to say through itself, and it
is precisely the concept which contains all the earlier determinations of thinking as
sublated determinations in itself.
To be sure, the concept needs to be considered as form,
but only as infinite, fecund form that encompasses the fullness of all content
within itself and at the same time releases it from itself.
By the same token, the concept may also be called 'abstract',
if by 'concrete' one understands what presents itself to the senses as concrete
what can be perceived in any immediate way at all.
We cannot grasp the concept as such with our hands and, when it comes to
the concept, we generally have to take leave of seeing and hearing.
Nonetheless, the concept is at the same time, as already noted, the absolutely concrete, and
indeed is so insofar as it contains in itself being and essence, and accordingly
contains the entire richness of these two spheres in an ideal unity.
If, as previously noted, the diverse stages of the logical idea can be considered as a
series of definitions of the absolute, then the definition of the absolute
that is the result for us here is that the absolute is the concept.
To be sure, one must in this case then construe the concept in a sense different from and higher than occurs in
logic at the level of the understanding, for which the concept is regarded merely
as a form of our subjective thinking, a form devoid of content in itself.
 In light of this, there is only one question that could still be raised.
If in speculative logic 'concept' has a meaning completely different from the one that would otherwise
be ordinarily associated with the expression, why is what is completely different in
this sense nonetheless called the 'concept' here, when doing so
occasions misunderstanding and confusion?
The reply to such a question would be that,
however great the distance berween the concept of formal logic and
the speculative concept, it still turns out, on closer inspection, that the profounder
meaning of the concept is by no means as alien to the ordinary use of language as
might at first seem to be the case.
One speaks of the derivation of a content, such
as, for example, the derivation of legal determinations concerning property from
the concept of property, and one speaks also conversely of tracing such a content
back to the concept. With this, however, it is recognized that the concept is not
merely a form devoid of content in itself, since, on the one hand, there would be
nothing to derive from the latter and, on the other, in tracing a given content back
to the empty form of the concept, the content would not only be robbed of its
determinacy; it would also not be known.

Passing over into an other is the dialectical process in the sphere of being and
the process of shining in an other within the sphere of essence.
The movement of the concept is, by contrast, the development, by means of which
that alone is posited that is already on hand in itself in nature it is the organic life,
which corresponds to the stage of the concept.
Thus, for example, the plant develops itself out of its seed.
This seed contains the entire plant in itself already,
but in an ideal manner and so one should not construe its development as if the various
parts of the plant, root, stem, leaves, and so forth were already really in the seed yet
merely in utterly miniature fashion.
This is the so-called 'Chinese box hypothesis',
the deficiency of which consists in the fact that what is only on hand initially
in an ideal manner is considered as already concretely existing.
What is right in this hypothesis is, by contrast, this:
that the concept, in its process, remains with itself and
that nothing new is posited by this means with respect to the content.
Instead only an alteration of form is brought forth.
It is then, too, this nature of the concept
(that of demonstrating itself in its process as self-development)
that one has one's eyes on when one speaks of ideas innate to human beings or
considers all learning, as Plato did, merely as recollection.
Yet this likewise should not be understood as if what makes up the content of
the consciousness educated by instruction were already on hand previously in
the same consciousness in the specific way that that content unfolds.
The movement of the concept is to be considered, as it were, merely as a play;
the other posited by it is in fact not an other.
In the Christian religious doctrine, this is articulated in such a way that God
not only created a world that as an other stands over against him,
but also that he has, from all eternity, produced a son in whom he is with himself as spirit.

A. THE SUBJECTIVE CONCEPT

When there is talk of concepts, one usually has in view an abstract
universality and the concept would then also be customarily defined as a universal
representation. One accordingly speaks of colour, plant, animal, and so forth,
and these concepts are supposed to arise by way of the fact that, in the process
of leaving aside the particular factor through which the diverse colours, plants,
animals, and so forth are distinguished from one another, we hold fast to what
is common to them. This is the manner in which the understanding construes
the concept and it is right for sentiment to declare such concepts to be
hollow and empty, mere schemata and shadows. But the universal factor of the
concept is not merely something common, opposite which the particular has its
standing for itself. Instead the universal factor is the process of particularizing
(specifying) itself and remaining in unclouded clarity with itself in its other. It
is of the most enormous importance as much for knowing as for our practical
comportment that the merely common is not confused with the truly universal
factor, the universal. All the reproaches that tend to
be raised from the standpoint of sentiment against thinking in general, and then,
more particularly, against philosophical thinking, are grounded in this confusion,
as is the otten-repeated claim about the dangerousness of thinking, allegedly driven
to extremes. Moreover, in its true and encompassing meaning, the universal is a
thought, of which it has to be said that it cost millennia before entering in to human
consciousness and which attained full recognition only through Christendom. The
Greeks, who were otherwise so highly cultivated, knew neither God in his true
universality nor even the human being. The Greek gods were only the particular
powers of the spirit, and the universal God, the God of nations, was still the
hidden God for the Athenians. So, too, for the Greeks there was an absolute
chasm between them and the barbarians, and the human being as such was not
yet recognized in his infinite worth and his infinite justification. When, indeed,
the question has been posed why slavery has disappeared in modern Europe,
first the one and then the other particular circumstance is cited to explain this
phenomenon.
The true reason why there are no longer slaves in Christian Europe
is to be sought in nothing other than the principle of Christendom itself.
The Christian religion is the religion of absolute freedom, and only for the Christian is
the human being as such valid, in his infiniteness and universality.
What the slave lacks is the recognition of his personhood;
the principle of personhood, however, is the universality.
The master regards the slave not as a person but as a basic matter
devoid of a self, and the slave himself does not count as an 'I'; instead,
the master is his '1'.
The previously mentioned difference between the merely
common and the truly universal is articulated in Rousseau's well-known Contract
social in a quite fitting manner where it is said that the laws of a state would have
to proceed from the universal will (volonte generale) without, however, needing
at all to be the will of all (volonte de tous).
In relation to the theory of the state,
Rousseau would have accomplished something more thorough, had he always
kept this distinction in mind.
The universal will is the concept of the will and the
laws are the particular determinations of the will, grounded in this
concept.

In regard to the usual discussion in logic [operating] at
the level of the understanding, about the emergence and formation of concepts,
it remains to be noted that we do not form the concepts at all and
that the concept in general is not to be considered something that has a genesis at all.
To be sure, the concept is not merely being or the immediate;
instead, mediation is also part of it.
However, this mediation lies in the concept itself,
and the concept is what mediates itself through itself and with itself.
It is wrong to assume, first that there are objects
which form the content of our representations and
then our subjective activity comes along behind them,
forming the concepts of objects by means of the earlier mentioned operation of
abstracting and gathering together what is common to the objects.
On the contrary, the concept is what is truly first and
the things are what they are, thanks to the activity of
the concept dwelling in them and revealing itself in them.
In our religious consciousness this surfaces in such a way that we say,
'God created the world out of nothing' or, to put it otherwise,
'the world and finite things have gone forth out of
the fullness of divine thoughts and divine decrees'.
In this manner it is recognized that the thought and, more precisely,
the concept is the infinite form or the free, creative activity,
which is not in need of some stuff on hand outside itself,
in order to realize itself.

b. Judgment

The judgment is customarily regarded as
a combination of concepts and, indeed, diverse sorts of concepts.
What is right in this construal is this,
that the concept forms the presupposition of the judgment and
makes its appearance in the judgment in the form of the difference.
But it is wrong to speak of diverse sorts of concepts,
for the concept, although concrete, is still essentially one and
the moments contained in it are not to be considered as diverse sorts.
Moreover, it is equally false to speak of a combination of
the sides of the judgment since, when there is talk of a combination,
then what are combined are thought of as being on hand for themselves
even apart from the combination.
This external construal is evident then in an even more determinate fashion if it is said of a
judgment that it comes about by virtue of the fact that a predicate is attributed to
a subject.
In this connection the subject counts as something obtaining externally
for itself and the predicate as something occurring in our head.
Meanwhile, the copula 'is' already contradicts this representation.
If we say 'this rose is red' or 'this painting is beautiful',
what is thereby said is not that it is we who in some external
fashion make the rose red or the painting beautiful,
but instead that these are the objects' own determinations.
A further deficiency of the usual way of construing
judgment (usual in formal logic) consists in the fact that, as a consequence of this
construal, the judgment generally appears as something merely contingent and the
progression from concept to judgment is not demonstrated. The concept as such,
however, is not something in itself stagnant [verharrena'J, devoid of process, as the
understanding thinks. To the contrary, as infinite form, it is absolutely active, as
it were, the punctum saliem of all vitality, and accordingly differentiates itself from
itself. This diremption posited by the concept's own activity, the diremption of the
concept into the difference between its moments, is the judgment, the meaning
of which is accordingly to be construed as the panicuiariZ4tion of the concept.
In itself, the concept is, to be sure, already the particular but, in the concept as
such, the particular is not yet posited, but is instead still in transparent unity with
the universal. Thus, for example, as earlier noted (§ 160 Addition), the seed of
a plant already contains the particular factor of the root, of the branches, of the
leaves, and so forth. But this particular factor is at first only on hand in itselfand is
only posited in that the seed discloses itself, something which is to be considered
the judgment of the plant. This example can also serve to draw notice to the
fact that neither the concept nor the judgment are merely occurrences in our
head and are not fashioned merely by us. The concept is something that dwells
within the things themselves, by means of which they are what they are, and to
comprehend [begreift'n] an object means accordingly to become conscious of its
concept [BegriffJ. If we then take the next step to judging the object, it is not our
subjective doing that accounts for attributing this or that predicate to the object.
Instead we consider the object in the determinacy posited by its concept.

If one says: 'The subject is that of which something is asserted and the
predicate is what is asserted of it', then this is to say something quite trivial.
One learns nothing more precise about the difference between the two by this means.
As far as the thought of the subject is concerned, it is initially the individual and
the predicate the universal. In the further development of the judgment, it then
happens that the subject does not remain merely the immediately individual and
the predicate merely the abstract universal. Subject and predicate then also acquire
a [new] meaning, the former that of the particular and universal, the lattet that
of the particular and individual. This exchange in the meaning of the two sides
of the judgment is what takes place under the two designations of 'subject' and
'predicate'.

The various species of judgment are to be
construed not merely as an empirical manifold,
but instead as a totality determined by thinking.
One of Kant's great services is to have provided some validation for this demand.
Kant divided judgments, according to the schema of his table of categories,
into judgments of quality, quantity, relation, and modality.
Although this division set up by Kant cannot be recognized as adequate
(in part because of the merely formal application of the schema of these categories,
in part also because of their content),
underlying this division, nevertheless, is the genuine intuition that
it is the universal forms of the logical idea itself
through which the diverse species of judgment are determined.
Accordingly, we initially obtain three main species of judgment,
which correspond to the stages of being, essence, and concept.
The second of these main species is then doubled in turn,
corresponding to the character of essence as the stage of difference.
The inner ground of this systematic [character] of
the judgment is to be sought in the fact that,
since the concept is the ideal unity of being and essence,
its unfolding, as it comes about in the judgment, also
has to reproduce initially these two stages in a transformation that
conforms to the concept, while it itself, the concept,
demonstrates itself to be the determining factor for the genuine judgment.
The various species of judgment are to be considered,
not as standing next to one another with the same value but
instead as forming a sequence of stages, whose differences
rest upon the logical meaning of the predicate.
This sort of consideration is also already at hand in
ordinary consciousness to the extent that one does not hesitate to ascribe a very
slight capacity for judgment to those only used to making such judgments like
'this wall is green', 'this stove is hot', and so forth.
At the same time, by contrast, it will be said that someone truly understands
how to judge only if his judgments concern whether a certain artwork is beautiful,
an action is good, and the like.
In judgments of the first-mentioned species, the content forms merely an abstract
quality and the immediate perception suffices to decide on its presence, whereas,
by contrast, if it is said that an artwork is beautiful or that an action is good,
the objects named are compared with what they ought to be, i.e. with their concept.

c. Syllogism

Like the concept and the judgment, the syllogism also tends to be
regarded merely as a form of our subjective thinking and, in keeping with this
tendency, it is said that the syllogism is the justification of the
judgment. Now, to be sure, the judgment points to the syllogism, but it is not
merely our subjective doing through which this progression comes about. Instead
it is the judgment itself that posits itself as syllogism and, in doing so, returns to
the unity of the concept. More precisely, it is the apodictic judgment that forms
the transition to the syllogism. In the apodictic judgment we have an individual
that relates itself, thanks to its constitution, to its universal, i.e. its concept. The
particular appears here as the mediating middle between the individual and the
universal and this is the basic form of the syllogism, the further development of
which, formally construed, consists in the fact that the individual and the universal
also occupy this place, by means of which the transition from subjectivity to
objectivity is then formed.

In keeping with the construal of the syllogism, mentioned above, as
the form of the rational, reason itself has been defined as the capacity to make
syllogistic inferences and understanding, by contrast, as the capacity to form
concepts. Underlying these definitions is a representation of the spirit as the mere
sum of powers or capabilities lying next to one another. Apart from this superficial
representation, what is to be noted about this combination of the understanding
with the concept and reason with the syllogism is that just as little as the concept
is to be regarded merely as a determination of the understanding, so, too, the
syllogism is to be regarded without further ado as rational. On the one hand, what
is usually treated in formal logic in the doctrine of the syllogism is in fact nothing
other than the mere syllogism of the understanding, which in no way deserves the
honour of counting as the form of the rational, indeed, as the rational itself. On
the other hand, the concept as such is so little merely a form of undersranding that
it is rather the understanding in the mode of abstracting alone, through which the
concept is demoted to this level. In accordance with this, there is also a tendency to
distinguish mere concepts of the understanding [Vmtandesbegriffi] and concepts
of reason [Vemunftbegriffi] , which is nevertheless not to be understood as though
there were two distinct species of concepts but instead much more so that it is
our doing either to stand pat merely with the negative and abstract form of the
concept or to construe it, in keeping with its true nature, as at the same time
positive and concrete. Thus, for example, the concept of freedom, insofar as it is a
mere concept of the understanding, is freedom considered as the abstract opposite
of necessity, while the true and rational concept of freedom contains in itself
necessity as sublated. Similarly, the definition of God put forward by so-called
deism, is the concept of God insofar as it is a mere concept of the understanding,
while by contrast the Christian religion, knowing [wissm} God as the triune God,
contains the rational concept of God.

B. THE OBJECT

Construing the absolute (God) as the object and not moving beyond
such a construal is in general the standpoint of superstition and slavish fear, as
Fichte above all has rightly emphasized in recent times. To be sure, God is the
object, and indeed the object without qualification, opposite which our particular
(subjective) opinions and wants have no truth and no validity. But precisely as
the absolute object, God does not stand like some sinister and inimical power
over against subjectivity. Instead God contains subjectivity as an essential factor
within himself This point is formulated in the teachings of the Christian religion,
when it is said that God wants for all human beings to be helped and wants all
of them to be blessed. That human beings are helped, that they are blessed, this
happens by virtue of the fact that they attain consciousness of their unity with
God and God ceases to be for them a mere object arid thereby just an object
of fear and terror, as was the case for the religious consciousness of
the Romans in particular. If, furthermore, in the Christian religion, God is known
as love, and indeed insofar as he revealed himself to humanity in his
Son, who is one with him, and did so as this individual human being, by this
means redeeming humanity, this says likewise that the opposition of objectivity
and subjectivity is in itself overcome and the basic matter for us is to participate
in this redemption by letting go of our immediate subjectivity (taking off the old
Adam) and becoming conscious of God as our true and essential self. - Now,
just as religion and the religious culture consists in overcoming the opposition
of subjectivity and objectivity, so too science, and more precisely philosophy, has
no other task than to overcome this opposition through thinking. In the case of
knowing, what generally needs to be done is to strip away the alien ness of the
objective world standing over against us, to find our way into it, as one says,
which amounts to saying that we need to trace the objective [dimension] back
to the concept which is our innermost self. From the previous discussion it can
be gathered how wrong it is to consider subjectivity and objectivity a rigid and
abstract opposition. Both are utterly dialectical. In keeping with its own activity,
without needing any external material or stuff, the concept which at first is only
subjective proceeds to objectify itself, and so too the object is not something
immovable and devoid of process, but instead is the process of proving itself to
be at the same time the subjective [dimension] that forms the progression to the
idea. What happens to anyone who is not familiar with the determinations of
subjectivity and objectivity, preferring to hold fast to them in their abstraction, is
that these abstract determinations slip through his fingers before he lays hold on
them and he says precisely the opposite of what he wanted to say.

Objectivity contains the three forms: mechanism, chemism, and the relation of purpose.
The mechanically determined object is the immediate, indifferent object.
It contains difference, to be sure, but the diverse [elements]
behave indifferently towards one another and the combination of them is only
external to them.
In chemism, by contrast, the object demonstrates itself to be essentially different,
such that the objects are what they are only through their
relation to one another and the difference constitutes their quality.
The third form of objectivity, the teleological relationship,
is the unity of mechanism and chemism.
The purpose is again, like the mechanical object, a totality enclosed within itself,
yet enriched by the principle of difference that emerged in chemism,
and so it [the purpose] refers to the object standing over against it.
It is the realization of the purpose, then, that forms the transition to the idea.

a. Mechanism

Mechanism, as the first form of objectivity, is also that very category
that first presents itself to reflection in observation of the objective
world and a category from which this observation quite frequently does not budge.
Nevertheless, this is a superficial manner of observation, lacking in thought,
insufficient for making do either in relation to nature or even less in relation
to the spiritual world. In nature only the completely abstract relationships of
matter (insofar as it remains locked up in itself) are subject to mechanism. By
contrast, even the phenomena and processes of the so-called 'physical domain'
in the narrower sense of the word (for example, the phenomena of light, heat,
magnetism, electricity, and so forth) cannot be explained in a merely mechanical
manner (i.e. through pressure, impulse, displacement of parts, etc.). Even more
unsatisfactory is the application and transference of this category to the domain
of organic nature, insomr as it is a matter of conceiving what is specific to it:
for example, the nourishment and growth of plants or even animal sensation.
In any case it must be regarded as a quite essential deficiency, indeed, the chief
deficiency of modern research into nature that, even where it is a matter of
completely different and higher categories than those of mere mechanism, it
nevertheless stubbornly dings to the latter, contradicting what presents itself to
an unprejudiced observation [Anschauung], and by this means blocks the path to
an adequate knowledge of nature. - Next, with regard to the formations of the
spiritual world, here too in the consideration of them the mechanical perspective
has been unduly promoted in various ways. This is the case, for example, if it is
said that a human being consists of body and soul. In this assertion these two count
as subsisting each for themselves and as being combined with one another only
externally. It also happens when the soul is regarded as a mere complex of forces
and faculties, subsisting self-sufficiently next to one another. - Thus, on the one
hand, the mechanical manner of observation must be rejected out of hand where
it comes on the scene with the pretension of occupying the position of conceptual
knowing in general and making mechanism the absolute category. Yet, on the
other hand, mechanism's legitimacy and meaning as a universal, logical category
must also be expressly vindicated, and accordingly by no means should it be
limited merely to the domain of nature from which this category's name is taken.
Thus, there is nothing to object to if attention is paid to mechanical actions (e.g.
those of weight, lever, and so forth) even outside the realm of genuine mechanics,
particularly in physics and in physiology. Only it should not be overlooked thereby
that within these domains the laws of mechanism are no longer the decisive ones,
but make their appearance only, as it were, in a subservient position. Immediately
linked to this point is then the further remark that where the higher functions in
nature, namely, the organic functions, suffer a disturbance or hindrance in one
way or another in their normal effectiveness, the otherwise subordinate mechanism
immediately emerges as dominating. Thus, for example, someone suffering from
a weak stomach has a sensation of pressure in the stomach after consuming a
modest quantity of certain foods, while others whose digestive organs are healthy
remain free of this sensation, despite having consumed the same thing. This is
also the case with the general feeling of heaviness in the arms and legs when
the body is in a sickJy mood. - Even in the domain of the spiritual world,
mechanism has its place, albeit a place that is likewise merely subordinate. One
speaks rightly of mechanical memory and of all sorts of mechanical activities such
as, for example, reading, writing, playing music, and so forth. More precisely in
this connection, as far as memory is concerned, a mechanical manner of behaving
is even inherent in its essence; a circumstance that has often been overlooked
by modern pedagogy, to the great detriment of the education of youth, in a
mistaken zeal for freedom of the intelligence. Nevertheless, someone would prove
to be a bad psychologist if, in order to fathom the nature of memory, he were
to take flight to mechanics and apply its laws without further ado to the soul.
The mechanical dimension of memory precisely consists solely in the fact that
here certain signs, sounds, and so forth are construed in their merely external
combination and then reproduced in this combination, without it being necessary
thereby to attend explicitly to their meaning and inner combination. In order to
recognize this connection with mechanical memory, no further study of mechanics
is needed, and from this study there is nothing to be gained for psychology
as such.

b. Chemism

Chemism is a category of objectivity that as a rule does not tend
to be stressed particularly.
Instead it is usually taken together with mechanism as one and,
in this manner of taking them together, under the common title
'mechanistic relationship', it is opposed to the relationship of purposiveness.
The motivation for this is to be sought in the fact that
mechanism and chemism have, indeed, this in common:
each is initially the concretely existing concept only in itself,
whereas the purpose, by contrast, is to be regarded as
the concept existing concretely for itself.
Nonetheless, mechanism and chemism also differ from one another very specifically,
namely, in the way that the object, in the form of mechanism, is
initially only an indifferent relation to itself, whereas
the chemical object, by contrast, demonstrates itself to be related straightaway to an other.
Now, to be sure, even in the case of mechanism, as it develops,
relations to an other are already emerging.
But the relation of the mechanical objects to one another is only
an external relation initially, such that the objects related
to one another are left with the semblance of self-sufficiency.
Thus, for example, in nature the various heavenly bodies that
form the solar system are connected by their movements and,
by this means, demonstrate that they are related to one another.
Yet motion, as the unity of space and time, is nothing but
an utterly external and abstract relation,
and so it seems as though the heavenly bodies,
related in such an external manner to one another,
would be and even remain what they are without this relation
that they have to one another.
In the case of chemism, by contrast, things behave otherwise.
Chemically differentiated objects are explicitly what they are,
only through their difference, and are thus the absolute drive
to integrate themselves through and with one another.

The chemical process is still a finite, conditioned process.
The concept as such is as yet only the inner dimension of this process and
does not yet come into concrete existence in its being-for-itself.
In the neutral product, the process is extinguished and
what stirs things up falls outside the process.

The transition from chemism to the teleological relationship is entailed
by the fact that the two forms of the chemical process reciprocally
sublate one another.
In this way it comes about that the concept, initially only present in
itself in chemism and in mechanism, becomes free, and the concept,
thus existing concretely for itself, is the purpose.

c. Teleology

When speaking of purpose, one usually has one's eye only on external
purposiveness. In this manner of considering things, they do not count as bearing
their determination in themselves. Instead they count merely as means that are
used and used up to realize some purpose lying outside them. This is in general
the viewpoint of utility, which formerly played a great role in the sciences as well,
but then deservedly came to be discredited, and recognized to be insufficient for
true insight into the nature of things. To be sure, justice must be done to finite
things as such inasmuch as they are to be considered to be other than ultimate
and to point beyond themselves. This negativity of finite things, however, is their
own dialectic and, in order to know this, one first has to get involved with their
positive content. Moreover, what is at stake in the case of the teleological manner
of consideration is the well-intended interest of pointing out the wisdom of God
announcing itself in nature. To this extent, accordingly, it should be noted that,
with this search for purposes that things serve as means, one does not get beyond
the finite and easily lapses into meagre reflections, as, for example, when not only
is the grapevine considered from the viewpoint of the familiar use that it affords
human beings, but even the cork tree is so considered in relation to the stopper
that is cut from its bark in order to seal the wine bottle. In former times, entire
books have been written in this vein and it is easy to establish that neither the true
interest of religion nor that of science can be advanced in this way. The external
purposiveness stands immediately before the idea, but sometimes what thus stands
on the threshold is precisely more insufficient than anything else.

The development of the purpose into the idea comes about by way
of three steps: first, that of the subjective purpose; second, that of the purpose
bringing itself about; and third, that of the purpose that has
brought itself about. At the outset we have the subjective purpose and this, as
the concept being for itself, is itself the totality of the conceptual moments. The
first of these moments is that of the universality identical with itself: as it were, the
neutral first water in which everything is contained but not yet separated out. The
second is then the particularization of this universal, through which it receives a
determinate content. But since this determinate content is posited by the activity
of the universal, the latter then returns to itself by means of that content and
joins itself together with itself [schliejft sich mit sich selbst zusammen]. Accordingly,
when we set a purpose in front of us, we also say that we decide [beschliejfenl
on something and accordingly consider ourselves at the outset to be, as it were,
open and amenable to this or that determination. Similarly then, however, it is
also said that 'one has resolved [entschlossen] to do something', which expresses
that the subject has emerged from his inwardness, i.e. his being only for himself,
and let himself in for the objectivity standing opposite him: This then yields the
progression from the merely subjective purpose to the purposive activity directed
outwards.

The process of carrying out the purpose is
the mediated manner of realizing the purpose;
just as necessary, however, is the immediate realization of it.
The purpose seizes the object immediately
because it is the power over the object,
because in it the particularity is contained and,
in the latter, the objectivity is also contained.
The living entity has a body;
the soul takes control of it
and has immediately objectified itself in it.
The human soul has a great deal to do in making its corporeal condition a means.
A human being must first take possession of his body, as it were,
so that it may be the instrument of his soul.

Reason is as cunning as it is powerful. The cunning consists generally
in the activity of mediating, which, by letting the objects, in keeping with their
own nature, act on one another and wear themselves out on one another, without
meddling immediately in this process, achieves its purpose alone. In this sense, one
can say that the divine providence, over against the world and its process, behaves
as the absolute cunning. God gives free rein to human beings with their particular
passions and interests and, by this means, what comes about is the accomplishment
of his aims which are different from what was pursued by those of whom he makes
use in the process.

The finitude of the purpose consists in the fact that, in the course
of its realization, the material applied as means to it is only subsumed under it
externally and made to conform to it. But, now, in fact the object in itself is the
concept and because the concept, as purpose, is realized therein, this is only the
manifestation of its own inner dimension. The objectivity is thus as it were only a
hull under which the concept lies hidden. Within the finite, we cannot experience
it or see that the purpose is truly attained. To accomplish the infinite purpose is
thus merely to sublate the illusion that it is not yet accomplished. The
good, the absolute good, brings itself to completion in the world eternally and the
result is that it is already brought to completion in and for itself, without needing
first to wait for us. It is this illusion in which we live and at the same time it alone
is the activating principle upon which the interest of the world rests. The idea in
its process fabricates that illusion for itself, positing an other opposite itself, 'and
its action consists in sublating this illusion. Truth emerges only from this error
and herein lies the reconciliation with error and with finitude. Otherness or error,
as something sublated, is itself a necessary moment of the truth, the truth which
only is by making itself its own result.

C. THE IDEA

By truth, one understands at first that I know how something is.
Yet this is truth only in relation to consciousness or the formal truth, mere correctness.
In contrast to this, truth in the deeper sense consists in this,
that objectivity is identical with the concept.
 It is truth in this deeper sense that is
at stake if, for example, one is speaking of a true state or of a true work of art.
These objects are true if they are what they should be, that is to
say, if their reality corresponds to their concept.
So construed, the untrue is the
same as what is otherwise also called 'the bad'.
A bad human being is one who is not truly human, i.e. a human being who does not
behave in keeping with the concept or determination of a human being.
 Nothing, meanwhile, can subsist
utterly without the identity of the concept and reality. Even something bad and
untrue is only insofar as its reality still behaves somehow in conformity with its
concept. Something thoroughly bad or at odds with the concept is, precisely for
this reason, something collapsing in itself. It is the concept alone through which
things have their standing in the world; that is to say, in the language of religious
representation, things are what they are only by virtue of the divine and thereby
creative thought dwelling within them. - When speaking of the idea, one must
not imagine something remote and other-worldly by this. The idea is instead what
is thoroughly present, and so too it is to be found in every consciousness, even
if muddled and stunted. We represent the world to ourselves as an enormous
totality created by God and, indeed, such that God has revealed himself to us in
it. So too we regard the world as governed by divine providence and herein lies the
fact that the asundered character of the world is eternally led back to
the unity out of which it went forth and, in keeping with that unity, is preserved.
From time immemorial in philosophy, it has been about nothing other than
thoughtfully knowing the idea, and underlying everything that deserves the name
'philosophy' has been the consciousness of an absolute unity of what holds for
the understanding only in its separation. - The proof that the idea is the truth
is not something to be demanded only now; the entire foregoing elaboration and
development of thinking contains this proof The idea is the result of the course
that this has taken, a course that is, nevertheless, not to be understood as -if it
were something only mediated, that is to say, mediated by something other than
itself The idea is instead its own result and, as such, just as much immediate as
mediated. The stages considered so far, those of being and essence and equally
of the concept and objectivity, are not something fixed and resting on themselves
with regard to this difference among them. Instead they have been demonstrated
to be dialectical and their truth is only that of being moments of the idea.

The idea, as a process, runs through three stages in its development.
The first form of the idea is life, the idea in the form of immediacy.
The second form is then that of the mediation or the difference, and
this is the idea as knowing which appears in the twofold shape of
the theoretical and the practical idea.
The process of knowing has, as its result,
the restoration of the unity, enriched by the difference, and
this yields the third form of the hereby absolute idea,
the final stage of the logical process that proves itself to be at once
the truly first and the only entity that is through itself alone.

a. Life

The individual members of the body are what they are only by means
of their unity and in relation to it. Thus, for example, a hand that is severed
from the body is a hand only in name, but not in reality, as
Aristotle already noted. From the standpoint of the understanding, life is usually
regarded as a mystery and generally as incomprehensible. In this way, meanwhile,
the understanding merely confesses its finitude and vacuousness. Life is, in fact, so
little something incomprehensible that in it we are confronted with the concept
itself and, more precisely, the immediate idea existing concretely as a concept.
With this, then, the deficiency of life is also at once articulated. This deficiency
consists in the fact that here concept and reality do not truly correspond to one
another. The concept of life is the soul and this concept has the body for its reality.
The soul is, as it were, poured into its corporality and thus the former is only
sensing and feeling but not yet freely being-for-itself. The process of
life consists then in overcoming the immediacy in which it is still caught up, and
this process (which is itself in turn threefold) has as its result the idea in the form
of the judgment, i.e. the idea as knowing.

The process of living that is internal to it has in nature the threefold
form of sensibility, irritability, and reproduction. As sensibility, the living is imme-
diately a simple relation to itself, the soul that is everywhere present, in its body,
the external juxtapositions of which have no truth for it. As irritability, the living
appears divided in itself and, as reproduction, the living is constantly reproducing
itself from the inner difference of its members and organs. The living is only as
this continually self-renewing process within itself.

The living stands over against an inorganic nature towards which it
behaves as its power and which it assimilates to itself. The result of this process
is not, as in the case of the chemical process, a neutral product in which the self-
sufficiency of both sides standing opposite one another is sublated. Instead, the
living demonstrates itself to be something that reaches over and beyond its other
which is incapable of withstanding its power. The
inorganic nature that is subjugated by the living endures this because it is in itself
the same as life is for itself Hence, in the other, the living is merely connecting
with itself When the soul has fled the body, the play of the elementary powers
of objectivity commences. These powers are, so to speak, continually poised to
initiate their process in the organic body, and life is the constant battle against
them.

What is alive dies because it is the contradiction of being in itself
the universal, the genus, and yet existing concretely and immediately only as
individual. In death, the genus demonstrates itself to be the power over the
immediately individual. - For the animal, the process of the genus is the highest
point of its condition of being alive. But the animal does not manage to be for
itself in its genus, succumbing instead to the latter's power. What is immediately
alive mediates itself with itself in the process of the genus and thus elevates itself
above its immediacy, only to sink back down to that same immediacy again and
again. In this way, life runs its course at first merely into the bad infinity of the
progression ad infinitum. What, meanwhile, in keeping with the concept, comes
about through the process of life is the sublation and overcoming of the immediacy
in which the idea as life is still ensnared.

b. Knowing

The finitude of knowing lies in the presupposition of a world already found before it,
and in the process the knowing subject appears as a tabula rasa.
This representation of things has been ascribed to Aristotle, although no one is
more removed from this external way of construing knowing than Aristotle.
This knowing does not yet know itself as the activity of the concept,
something which it is only in itself, but not for itself.
Its behaviour appears to it as something passive, yet it is in fact active.

It is customary to speak of analytic and synthetic method as though
following the one or the other were a mere matter of our whim. Yet this is in
no way the case. Instead, which of the two methods to apply - both of which
result from the concept of finite knowing depends upon the form of the objects
themselves that are to be known. Knowing is at the outset analytical.
The object has for it the shape of an isolated individual
and the activity of analytic knowing aims at tracing the individual lying before
it back to a universal. Here thinking, has the meaning of abstraction or formal
identity only. This is the standpoint on which Locke and all empiricists stand.
Many say that knowing can do nothing further than analyse the given, concrete
objects into their abstract elements and then consider the latter in
isolation. It is immediately evident, meanwhile, that this is to turn things upside
down and that the sort of knowing that wants to take things as they are thereby
falls into self-contradiction. Thus, for example, the chemist brings a piece of meat
to his test-tube, breaks it down in a variety of ways, and then says that he has found
that it consists of nitrogen, carbon, hydrogen, and so on. However, these abstract
bits of material are then no longer meat. Something similar is the case when the
empirical psychologist analyses an action into the diverse sides which it presents
for consideration and then clings to them in abstraction from one another. In this
case, the analytically treated object is regarded, as it were, as an onion
from which one peels one skin after the other.

The movement of the synthetic method is the inversion of the analytic method.
While the latter advances by going from the individual as its starting point
to the universal, in the former case the universal (as definition) forms the point of
departure instead, and there is a progression from it through the particularization
(in the division) to the individual (the theorem). With this, the synthetic method
demonstrates itself to be the development of the moments of the concept in the
object.

The definition itself contains the three moments of the concept:
the universal as the proximate genus (genus proximum),
the particular as the determinacy of the genus (qualitas speciftca),
and the individual as the defined object itself.
With respect to definition, the question immediately
arises 'where does it comes from?' and this question is generally to be answered
by noting that definitions arise on the analytic path.
With this answer, however, the dispute about the correctness of the definitions put forward immediately
presents itself. For it is a matter here of rhe perceptions that formed one's point
of departure and the kinds of viewpoint from which one looked. The richer the
object is that is to be defined, i.e. the more diverse sides it offers for
consideration, the more diverse the definitions given of it tend to be. Thus, for
example, there is an entire array of definitions of life, of the state, and so forth.
Geometry, by contrast, has an easy time making definitions since its object, space,
is such an abstract object. - Further, there is generally no necessity
on hand with respect to the content of the defined object.
One is supposed to accept that there is space, that there are plants, animals, and so
forth, and it is not a matter for geometry, botany, and so forth to point out the
necessity of the defined objects. On account of this circumstance, the
synthetic method is no more appropriate for philosophy than the analytic method
is, since philosophy has, before anything else, to justify to itself the necessity of
its objects. Nevertheless, the effort has been made over and over
to make use of the synthetic method in philosophy. Spinoza in particular begins
with definitions and says, for example, 'Substance is the causa sui.' He lays down
the most speculative themes in his definitions, but in the form of assurances.
The same holds for Schelling.

What is demanded of the division is that it be complete, and part of
this requirement is a principle or ground of the division that is so constituted that
the division based on it encompasses the entire scope of the domain designated
by the definition in general.
In the course of the division it is then necessary, in addition,
that it be done in such a way that its principle has been drawn from
the nature of the object itself that is divided up.
In this way the division is made naturally and not artificially, i.e. arbitrarily.
So, for example, in zoology in the division of mammals, the claws and teeth are used above all as the
ground of the division, and this is sensible since mammals themselves distinguish
themselves from one another through these parts of their bodies and the general
type of the diverse classes of them [i.e. mammals] are to be led back to this.
In general, the true division is to be regarded as determined by the concept.
To this extent it is initially threefold; but since the particularity presents itself
as something doubled, the division then progresses to something fourfold as well.
Trichotomies predominate in the sphere of the spirit and it is one of Kant's accomplishments
to have drawn attention to this circumstance.

The necessity that knowing attains through the proof is the opposite of
what forms for it its point of departure.
In its point of departure, knowing had a given and contingent content.
Now, however, at the conclusion of its movement,
it knows the content as a necessary one and
this necessity is mediated by the subjective activity.
So, too, subjectivity was at first completely abstract, a mere tabula rasa,
whereas it proves itself now, by contrast, to be determining.
Herein, however, lies the transition
from the idea of knowing to the idea of willing.
This transition consists then, more precisely, in the fact that
the universal is to be construed in its truth as subjectivity,
as the self-moving, active concept, positing determinations.

While what matters for intelligence is merely taking the world as it is,
the will, by contrast, is bent on making the world what it ought to be.
The immediate, what it finds before it, counts for the will, not as a fixed being,
but instead only as a semblance, as something in itself vacuous.
Here those contradictions come to the fore in which one stumbles around on the standpoint
of morality. This in general is the standpoint of the Kantian and even also the
Fichtean philosophy in a practical context. The good is supposed
to be realized; one has to work to produce it, and the will is only the good
acrivating itself. But then, were the world as it is supposed to be, the activity
of willing would fall by the wayside. Thus the will in itself requires that its
purpose also not be realized. This account correctly expresses the will's finitude.
But then we should not stand pat with this finitude, and it is the process of
willing itself through which this finitude and the contradiction contained in it
are sublated. The reconciliation consists in the fact that the will, in its result,
returns to the presupposition of knowing, that is to say, it consists in the unity of
the theoretical and pracrical idea. The will knows [weijSJ the purpose as its own
and the intelligence construes the world as the actual concept. This is the true
posture of rational knowing. What is vacuous and vanishing makes up only the
surface, not the genuine essence of the world. This is the concept, being in -and
for itself, and the world is thus itself the idea. The unsatisfied striving disappears
if we know that the final purpose of the world has been brought about and to the
same degree eternally brings itself about. This is generally the posture of the adult
man, while the youth believes that the whole world is in a bad way and out of it
a completely different world must be made. By contrast, religious consciousness
regards the world as governed by Divine Providence, and thus as corresponding
to what it ought to be. This correspondence of is and ought, meanwhile, is not
a frozen and inert correspondence; for the good, the final purpose of the world,
is only in that it produces itself again and again, and the difference between
the spiritual world and the natural world then consists in the fact that while
the latter constantly only returns into itself, a progression also takes place in the
former.

c. The absolute idea

The absolute idea is first the unity of the theoretical and the practical idea and,
by this means, at the same time the unity of the idea of life and the idea of knowing.
In knowing, we had the idea in the form of difference and
the process of knowing has presented itself to us as the overturning of
this difference and as the restoration of that unity which,
as such and in its immediacy, is first the idea of life.
The deficiency of [the concept of] life consists
in being at first only the idea insofar as it is in itself;
in contrast to this, but in just as one-sided a fashion,
knowing is only the idea insofar as it is for itself.
The unity and truth of these two is the idea insofar as
it is in and for itself and, thereby, absolute.
Up to now we have had for our object
the idea in the development through its diverse stages;
now, however, the idea is objective with respect to itself.
This is the vonsis vonseos; what Aristotle already designated as the highest form of the idea.

When one speaks of the absolute idea, one can think that here finally
the substantive must come to the fore, that here everything must become clear.
One can, to be sure, vacuously spout on end about the absolute idea; the true
content, meanwhile, is nothing but the entire system, the development of which
we have considered up to this point. It can accordingly also be said that the absolute
idea is the universal, but the universal not merely as an abstract form opposite
which the particular content stands as something other than it. Instead it is the
absolute form, into which all determinations, the entire fullness of the content
posited by it, have gone back. In this respect, the absolute idea is comparable to
the old man who says the same religious sentences as the child does, but for the
old man they have the meaning of his entire life. Even if the child understands
the religious content, what validity that content has for him is still of the sort that
lies outside his entire life and world. The same holds then also for human life in
general and the occurrences that make up the content of it. All work is only aimed
at the goal, and if this is attained, then one is astonished at finding nothing else
than precisely this, what one wanted. The interest lies in the entire movement. If
a human being pursues his life, then the end can appear to him as quite limited,
but it is the entire decursus vitae [course of a life) that is encompassed in it. - Thus,
too, then the content of the absolute idea is the entire expanse of what we had
before us up until now. The final [point] is the insight that the entire unfolding
makes up the content and interest. - This is, furthermore, the philosophical view
that everything that appears limited, taken for itself, acquires its worth through
inhering in the whole and being a moment of the idea. Thus it is that we have
had the content and what we still have is the knowledge that the content
is the living development of the idea and this simple retrospective is contained
in the form. Each of the stages considered up to this point is an image of the
absolute, albeit in a limited manner at first, and so it drives irself on to the whole,
the unfolding of which is precisely what we have designated the method.

The philosophical method is as much analytic as it is synthetic,
yet not in the sense of a mere juxtaposition or a mere oscillation of
these two methods of finite knowing.
It is instead such that it contains them as sublated in itself
and accordingly behaves in each of its movements
both analytically and synthetically at the same time.
Philosophical thinking proceeds analytically insofar
as it merely takes up its object, the idea, giving the latter full play,
and as it were merely looking upon its movement and development.
To this extent, philosophizing is completely passive.
But philosophical thinking is then equally synthetic and
demonstrates itself to be the activity of the concept itself.
This requires, however, the strenuous effort of holding off on one's own notions
and particular opinions which are always trying to assert themselves.

In the progression of the idea, the beginning demonstrates itself to be
what it is in itself, namely,
something posited and mediated and not what simply and immediately is.
Only for immediate consciousness is nature the beginning point and the immediate,
and the spirit something mediated by nature.
In fact, however, nature is posited by the spirit and
the spirit itself makes nature its presupposition.

We have now returned to the concept of the idea with which we began.
This return to the beginning is at the same time a move forward.
What we began with was being, the abstract being,
and now we have the idea as being;
this idea insofar as it is, however, is nature.
