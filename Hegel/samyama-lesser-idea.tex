C. THE IDEA

The idea is the true in and for itself,
the absolute unity of the concept and objectivity.
Its ideal content is none other than
the concept in its determinations.
Its real content is only its exhibition,
an exhibition that it provides for itself
in the form of external existence
and, with this shape incorporated into
the concept's ideality and in its power,
the concept thus preserves itself in that exhibition.

The definition of the absolute,
that it is the idea, is itself absolute.
All previous definitions go back to this one.
The idea is the truth; for the truth is this,
that objectivity corresponds to the concept,
not that external things correspond to my representations;
these are only correct representations that I, this person, have.
In the idea it is not a matter of an indexical this,
it is a matter neither of representations nor of external things.
But everything actual, insofar as it is something true, is also the idea
and possesses its truth only through and in virtue of the idea.
The individual being is some side or other of the idea,
but for this still other actualities are needed,
actualities that likewise appear as obtaining particularly for themselves;
the concept is realized only in them together and in their relation.
The individual taken by itself does not correspond to its concept;
this limitation of its existence constitutes its finitude and its demise.

The idea itself is no more to be taken as an idea of something or other
than the concept is to be taken merely as a determinate concept.
The absolute is the universal idea and the one idea that, by judging,
particularizes itself into a system of determinate ideas:
ideas, however, that are only this,
the process of going back into the one idea, their truth.
On the basis of this judgment, the idea is at the outset only
the one, universal substance,
but its developed, true actuality is
that it is as subject and thus as spirit.

The idea is frequently taken for
something logical in a merely formal sense,
insofar as it does not have some concrete existence
as its point of departure and support.
One must leave such a view to the standpoints for which
the concretely existing thing and all further determinations
that have not yet penetrated to the idea still count as
so-called realities and true actualities.
Equally false is the representation of the idea
as though it were only something abstract.
It is this, of course, insofar as
everything untrue is consumed in it.
However, in itself it is essentially concrete
since it is the free concept,
the concept determining itself
and thereby determining itself as reality.
It would be something formally abstract
only if the concept that is its principle were taken as
the abstract unity and not as it is, namely,
as the negative return of it into itself
and as subjectivity.

The idea can be grasped as reason
(this is the genuine philosophical meaning of reason),
further as subject-object,
as the unity of the ideal and the real,
of the finite and the infinite,
of the soul and the body,
as the possibility that has
its actuality in itself,
as that the nature of which can
only be conceived as existing,
and so forth,
because in it [the idea]
all relationships of the understanding are contained,
but in their infinite return and identity in themselves.

The understanding makes easy work of pointing out
that everything said of the idea is self-contradictory.
This can be equally conceded to it or rather
it is already accomplished in the idea;
a work that is the work of reason and, of course,
not as easy as that of the understanding.
The understanding shows that the idea is self-contradictory because,
for example, the subjective dimension is only subjective
and the objective dimension, by contrast, is opposed to it;
because being is something completely different from
the concept and thus cannot be plucked from it;
similarly, because the finite is only finite
and precisely the opposite of the infinite,
and consequently is not identical with it
and so on for all determinations.
If the understanding thus shows that
the idea is self-contradictory,
the [science of] logic points out the opposite instead, namely,
that the subjective dimension
that is supposed to be merely subjective
lacks any truth, contradicts itself,
and passes over into its opposite,
as does the finite that is supposed to be merely finite,
the infinite that is supposed to be merely infinite, and so on.
By this means, the process of passing over into its opposite
and the unity in which the extremes are as something sublated,
as a shining or as moments,
reveals itself as their truth.

The understanding that tackles the idea
suffers from a twofold misunderstanding.
In the first place, it takes the extremes of the idea,
however they may be expressed, insofar as they are in their unity,
yet in the sense and determination proper to them
insofar as they are not in their concrete unity
but instead are still abstractions outside it.
The understanding mistakes no less the relation,
even if it is already posited explicitly.
In this way the understanding overlooks, for example,
the nature of the copula in a judgment,
which asserts of the individual, the subject,
that the individual is just as much
something not individual
but instead something universal.
In the second place, the understanding holds its reflection that
the idea that is identical with itself
contains the negative of itself
(that it contains the contradiction)
to be an external reflection,
that does not fall to the idea itself.
In fact, however, this is not a wisdom
proper to the understanding.
The idea is instead itself the dialectic that
eternally separates and distinguishes what is
identical with itself from the differentiated,
the subjective from the objective,
the finite from the infinite,
the soul from the body and,
only insofar as it does,
is it eternal creation,
eternally alive,
and eternal spirit.
Because it is thus itself the passing over
or rather the transposing of itself
into abstract understanding,
it is also eternally reason.
It [the idea] is the dialectic that
takes what is understandable in this [superficial] way,
including the diversity of its finite nature
and the false semblance of self-sufficiency of its productions
and renders it understandable in a recursive way
and leads it back to unity.
Since this twofold movement is
neither temporal nor separate and distinct in any way,
otherwise it would be again only abstract understanding,
it is the process of eternally intuiting itself in the other;
the concept that has carried itself out in its objectivity,
the object that is inner purposiveness, essential subjectivity.

The diverse ways of construing the idea,
as unity of the ideal and the real,
of the finite and infinite,
of identity and difference,
and so on, are more or less formal,
since they designate some sort of
stage of the determinate concept.
Only the concept itself is free and the truly universal;
in the idea its determinacy is thus equally only itself,
an objectivity into which it, as the universal,
continuously sets itself and in which it has only
its own determinacy, the total determinacy.
The idea is the infinite judgment,
each of the sides of which is
the self-sufficient totality and,
precisely by virtue of completing itself to this end,
has just as much passed over into the other.
None of the other determinate concepts is
this totality completed in its two sides,
except the concept itself and the objectivity.

The idea is essentially a process since its identity is
that of the absolute and free concept only insofar as
it is the absolute negativity and thus dialectical.
It is the course in which the concept
as the universality that is individuality
determines itself to be objectivity
and to be the opposite of objectivity,
and in which this externality
that has the concept as its substance leads
itself back into subjectivity
through its immanent dialectic.

Because the idea is (a) a process, the expression
'the unity of the finite and infinite,
of thinking and being, and so on',
as an expression for the absolute,
is false, as often noted.
For this unity expresses
an abstract, calmly enduring identity.
The expression is likewise false
because the idea is (b) subjectivity,
since that unity expresses the in itself,
the substantial dimension of the true unity.
The infinite thus appears as only
neutralized relative to the finite,
and so too the subjective relative to the objective,
thinking relative to being.
But in the negative unity of the idea
the infinite reaches over and beyond the finite,
as does thinking over being,
subjectivity over objectivity.
The unity of the idea is
subjectivity, thinking, infinity,
and hence it is essentially distinct
from the idea as substance
just as this overreaching subjectivity (thinking, infinity)
is to be distinguished from the one-sided subjectivity
(one-sided thinking, one-sided infinity)
to which it reduces itself in
judging and making determinations.

a. Life

The immediate idea is life.
The concept is realized as the soul in a body;
the soul is the immediate, self-referring universality of the body's externality,
just as much as it is the body's particularization, so that the body expresses
no other differences than the determination of the concept, and finally
it is the individuality as infinite negativity;
the dialectic of the body's objectivity,
[the factors of which are] outside one another,
an objectivity that is led back into subjectivity from the semblance of self-sufficient
subsistence, so that all members are reciprocally momentary means as
much as momentary purposes, while life, inasmuch as it is the inceptive
particularization, results in itself as the negative unity that is for itself and,
in the dialectic of embodiment, joins itself together only with itself.
Life is thus essentially a living entity and, with regard to its immediacy,
this individual living entity.
In this sphere, finitude has the determination that soul and body are separable,
on account of the immediacy of the idea;
this constitutes the mortality of the living.
But those two sides of the idea are diverse component parts
only insofar as it is dead.

The living is the syllogism, whose moments are systems and syllogisms
in themselves which, however, are active syllogisms, processes, and
in the subjective unity of the living, they are only one process.
The living is thus the process of its coming to closure together with itself,
that runs its course by means of three processes.

1. The first is the process of the living within itself,
in which it divides itself in itself and makes its
corporal condition its object, its inorganic nature.
For its part, this inorganic side, as the relatively external,
enters into the difference and opposition of its moments that reciprocally
surrender themselves, the one assimilating the other to itself, and
preserve themselves in the process of producing themselves.
This activity of the members, however, is only one activity of
the subject, the activity into which its productions go back,
so that through that activity only the subject is produced;
it merely reproduces itself.

2. But the judgment of the concept proceeds freely to release from itself
the objective dimension as a self-sufficient totality.
The negative relation of the living to itself, as immediate individuality,
presupposes an inorganic nature standing over against it.
Since this negative aspect of itself is just as much a moment of
the concept of the living itself, it is thus in the latter
(the at once concrete universality) as a lack.
The dialectic, through which the object as something in itself vacuous sublates itself,
is the activity of the living entity certain of itself that accordingly
preserves, develops, and objectifies itself in this process opposite an inorganic nature.

3. In the initial stage of its process,
the living individual behaves as
a subject and concept in itself.
Through its second stage, it assimilates
its external objectivity to itself
and thus posits in itself the real determinacy.
As a result, it is now in itself the genus,
substantial universality.
The particularization of the latter is
the relation of the subject to another subject of its genus
and the judgment is the relationship of the genus to
these determinate individuals standing opposite one another:
the difference of the sexes.

The process of the genus brings this [genus] to
the point of being-for-itself.
Because life is still the immediate idea,
the product of the process breaks down into two sides.
On the one side, the living individual in general,
at first presupposed as immediate,
emerges now as something mediated and produced.
On the other side, however, the living individuality that,
on account of its initial immediacy,
behaves negatively towards the universality,
perishes in this [universality] as the power.

By this means, however, the idea of life
has not only freed itself from just
any (particular) immediate 'this',
but from this initial immediacy altogether.
In this way, it comes to itself, to its truth,
entering into concrete existence
as the free genus for itself.
The death of the merely immediate,
individual living thing is the Spirit emerging.

b. Knowing

The idea concretely exists freely for itself
insofar as universality is the element
in which it exists concretely
or insofar as it is objectivity itself as the concept;
[that is to say,] the idea has itself for an object.
Its subjectivity, determined as universality,
is pure differentiating within it
intuiting that keeps itself in this identical universality.
But, as a differentiating in a determinate way,
it is the further judgment of thrusting itself as
a totality away from itself
and, indeed, initially
presupposing itself as the external universe.
These are two judgments that are in themselves identical
but not yet posited as identical.

The relation of these two ideas
that are identical in themselves
or as life is thus the relative relation
that makes up the determination of finitude in this sphere.
It is the relationship of reflection,
since the differentiation of the idea
in it [the idea] itself is only the first judgment,
the presupposing is not yet a positing,
and thus, for the subjective idea,
the objective dimension is the extant immediate world
or the idea as life in the appearance of individual concrete existence.
At the same time, insofar as this judgment is a pure differentiating
within it [the idea] itself (see the preceding section),
the idea is for itself both itself and its other.
Thus it is the certainty of being in itself
the identity of this objective world with it.
Reason comes to the world with the absolute faith
in its capacity to posit the identity
and elevate its certainty to truth,
and with the drive to posit as
also vacuous for it that opposition
that is in itself vacuous.

In general, this process is knowing.
In it, in one activity, the opposition,
the one-sidedness of subjectivity together with
the one-sidedness of objectivity,
is sublated in itself.
But this process of sublating takes place at the outset only in itself.
The process as such is thus itself immediately beset with the finitude
of this sphere and falls apart into the twofold, diversely posited
movement of the drive.
[In one respect,] it is the drive to sublate
the one-sidedness of the subjectivity of the idea
by taking up into itself the world that is,
taking it up into subjective representing and thinking,
and to fill out the abstract certainty of itself
with this objectivity as content, an objectivity that thus counts as true.
Conversely, it is the drive to sublate
the one-sidedness of the objective world
that here accordingly, by contrast, counts as a semblance,
a collection of contingencies and shapes vacuous in themselves,
and to determine and mould it through the inner dimension of the subjective,
that counts here as the objective, as what truly is.
The former is the drive of knowledge to truth, knowing as such,
the theoretical (activity);
the latter is the drive of the good to bring itself about, willing,
the practical activity of the idea.

(a) Knowing

The universal finitude of knowing that lies in the first judgment,
the presupposition of the opposition, which its very action contradicts,
specifies itself more precisely in its own idea in this direction,
that its moments receive the form of diversity from one another and,
since those moments are in fact complete, they come to stand in
the relationship of reflection, not of the concept, to one another.
The assimilation of the material as something given thus appears as
a way of taking it up into conceptual determinations that at the same time
remain external to it, determinations that likewise display themselves
opposite one another as diverse.
It is reason active as understanding.
The truth that this knowing comes to is thus likewise only finite;
the infinite truth of the concept is fixed as a goal that is only in itself,
something beyond this knowing.
But in its external action, it stands under the guidance of the concept,
and conceptual determinations make up the inner thread of the progression.

Because it presupposes what is differentiated as a being that is
found to be already on hand, standing opposite it
(the manifold facts of external nature or of consciousness),
finite knowing has (1) the formal identity or
the abstraction of universality as the form of its activity at the outset.
This activity thus consists in dissolving the given concrete dimension,
individuating its differences, and giving them the form of abstract universality;
or in leaving the concrete dimension as the ground and,
through abstraction from the particularities that seem inessential,
extracting a concrete universal, the genus or the force and the law.
Such is the analytic method.

This universality is (2) also a determinate one.
The activity here proceeds according to the moments of
the concept that, in finite knowing, is not in its infinity
but is the understandable , determinate concept instead.
Taking up the object into the forms of
the latter concept is the synthetic method.

(aa) Knowing initially puts the object into the form of
the determinate concept in general so that, by this means,
its genus and its universal determinacy are posited.
The respective object is the definition.

Its material and justification are procured by the analytic method.
The determinacy is, nevertheless, supposed to be only a characteristic,
that is to say, something to assist merely subjective knowing
that is external to the object.

(bb) The account of the second moment of the concept,
the determinacy of the universal as particularization,
is given by the division in terms of some sort of external aspect.

(cc) In the concrete individuality (such that the simple determinacy in
the definition is construed as a relationship),
the object is a synthetic relation of differentiated determinations - a theorem.
Because they are diverse, their identity is a mediated identity.
The process of supplying the material that constitutes
the middle members is the construction;
and the mediation itself, out of which the necessity of
that relation for knowing goes forth, is the proof.

The necessity which finite knowing produces in a proof
is initially an external necessity,
determined only for the subjective discernment.
But in the necessity as such, it has itself left behind its presupposition
and point of departure, the finding and givenness of its content.
The necessity as such is, in itself, the concept relating itself to itself.
The subjective idea has thus, in itself, come to what is determined in and for itself,
what is not given, and is thus immanent to it as the subject.
As such, it passes over into the idea of willing.

(b) Willing

The subjective idea - as what is determinate in and for itself,
the simple, self-same content - is the good.
Its drive of realizing itself inverts the relationship
that holds relative to the idea of the true, and is bent on determining,
in terms of its purpose, the world that it finds.
This willing is, on the one hand,
certain of the vacuousness of the presupposed object
but, on the other hand, as finite, it at the same time presupposes
both the purpose of the good as a merely subjective idea
and the independence of the object.

The finitude of this activity is thus the contradiction that, in the self-
contradicting determinations of the objective world, the purpose of the good
is both carried out and not carried out, and that it is posited as something
inessential just as much as something essential, as something actual and at
the same time as merely possible. This contradiction presents itself as the
endless progression in the actualization of the good, that is therein established
merely as an ought. Formally, however, this contradiction disappears in
that the activity sublates the subjectivity of the purpose and thereby the
objectivity, the opposition through which both are finite, and not only the
one-sidedness of this subjectivity but subjectivity in general; another such
subjectivity, that is to say, a new generation of the opposition, is not distinct
from what was supposed to be an earlier one. This return into itself is at
the same time the recollection of the content into itself, which
is the good and the identity in itself  of both
sides, - the recollection of the presupposition of the theoretical stance,
that the object is what is substantial in itself and true.

The truth of the good is, by this means, posited as
the unity of the theoretical and practical idea,
[the notion] that the good has been attained in and for itself,
that the objective world is thus in and for itself the idea precisely as
it [the idea] at the same time eternally posits itself as purpose and
through activity produces its actuality.
This life, having come back to itself from
the differentiation and finitude of knowing,
and having become identical with the concept
through the activity of the concept,
is the speculative or absolute idea.

c. The absolute idea

The idea as the unity of
the subjective and the objective idea is
the concept of the idea,
for which the idea as such is the object,
for which it is the object
an object into which
all determinations have gone together.
This unity is accordingly
the absolute and entire truth,
the idea thinking itself,
and here, indeed, as thinking,
as the logical idea.

The absolute idea is for itself,
since in it there is no transition or presupposing
and no determinacy at all that is not fluid and transparent;
it is the pure form of the concept
that intuits its content as itself.
It is content for itself
insofar as it is the ideal differentiating of
itself from itself,
and one side of what has been differentiated is
the identity with itself,
in which, however, the totality of the form is
contained as the system of the determinations of the content.
This content is the system of the logical.
Nothing remains here of the idea, as form,
but the method of this content,
the determinate knowledge of the validity of its moments.

The moments of the speculative method are

(a) The beginning, which is being or the immediate;
for itself for the simple reason that it is the beginning.
From the vantage point of the speculative idea, however,
it is the speculative idea's self-determining which,
as the absolute negativity or movement of the concept,
judges and posits itself as the negative of itself.

Being, which from the vantage point of the beginning as such
appears as abstract affirmation, is thus instead the negation,
positedness, being-mediated in general and being presupposed.
But as the negation of the concept that is
simply identical with itself in its otherness
and is the certainty of itself,
it is the concept not yet posited as concept
or, in other words,
it is the concept in itself.
For that reason, as the still undetermined concept,
i.e. the concept determined only in itself or immediately,
this being is just as much the universal.

(b) The progression is the posited judgment of the idea.
The immediate universal, as the concept in itself,
is the dialectic of reducing, within itself,
its immediacy and universality to a moment.
It is accordingly the negative [aspect] of the beginning
or the first [moment] posited in its determinacy;
it is for something,
the relation of what has been differentiated,
the moment of reflection.

The abstract form of the progression
within the stage of being is
[to be] an other and a passing over into an other;
in the stage of the essence,
it is the shining in something opposite;
in the stage of the concept,
it is the differentiated status of
the individual from the universality
which continues itself as such in
what is differentiated from it
and is as an identity with the latter.

In the second sphere, the concept at first
being in itself came to shine forth
and is thus in itself already the idea.
The development of this sphere becomes
the return to the first,
just as the development of the first sphere is
a transition into the second.
Only by means of this double movement is
justice done to the difference,
since each of the two differentiated factors,
each considered in itself,
completes itself so as to form the totality
and, in that totality,
puts itself into unity with the other.
Only the fact that both sublate the one-sidedness
in themselves prevents the unity from becoming one-sided.

The second sphere develops the relation of
what has been differentiated
into what the relation is at first,
namely a contradiction in the relation itself
in the infinite progression.

(c) This contradiction resolves itself into the end,
where the differentiated is posited as what it is in the concept.
It is the negative of the first,
and, as the identity with the latter,
the negativity of itself.
Hence, it is the unity
in which these first two,
as ideal and as moments,
are sublated, i.e. preserved at the same time.
The concept, starting out from its being-in-itself,
thus comes to a close with itself
by means of its difference
and the process of sublating that difference.
This concept is the realized concept,
the concept that contains the positedness
of its determinations in its being-for-itself;
it is the idea for which, as the absolutely first (in the method),
this end is at the same time
nothing more than the process
by which the semblance that
the beginning is something immediate
and it [the idea] a result vanishes;
in other words, this end is the knowledge
that the idea is the one totality.

In this way, the method is not an external form
but the soul and concept of the content,
from which it is distinguished only insofar as
the moments of the concept, even in themselves,
in their [respective] determinacy,
come to appear as the totality of the concept.
Insofar as this determinacy or the content, with the form,
leads itself back to the idea,
this idea exhibits itself as
the systematic totality which is only one idea,
the particular moments of which are in themselves this same idea
to the same extent that they bring forth
the simple being-for-itself of the idea
through the dialectic of the concept.
The science concludes in this way
by grasping the concept of itself
as the pure idea,
for which the idea is.

The idea, which is for itself,
considered in terms of this,
its unity with itself,
is the process of intuiting
and the idea insofar as it intuits is nature.
As intuiting, however,
the idea is posited by external reflection
in a one-sided determination of immediacy or negation.
Yet the absolute freedom of the idea is
that it does not merely pass over into life
or let life shine in itself as finite knowing,
but instead, in the absolute truth of itself,
resolves to release freely from itself
the moment of its particularity
or the first determining and otherness,
the immediate idea, as its reflection,
itself as nature.
