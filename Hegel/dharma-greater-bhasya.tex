BOOK TWO

The Doctrine of Essence

ESSENCE

The truth of being is essence.

IV.1
jati-antara-parinama prakrti-apurat

Being is the immediate.
Since the goal of knowledge is the truth,
what being is in and for itself,
knowledge does not stop at
the immediate and its determinations,
but penetrates beyond it on
the presupposition that
behind this being there still is
something other than being itself,
and that this background
constitutes the truth of being.
This cognition is a mediated knowledge,
for it is not to be found
with and in essence immediately,
but starts off from an other, from being,
and has a prior way to make,
the way that leads over and beyond being
or that rather penetrates into it.
Only inasmuch as knowledge recollects
itself into itself out of immediate being,
does it find essence through this mediation.
The German language has kept “essence” (Wesen)
in the past participle (gewesen) of the verb “to be” (sein),
for essence is past [but timelessly past] being.

When this movement is represented as a pathway of knowledge,
this beginning with being and the subsequent advance
which sublates being and arrives at essence as a mediated term
appears to be an activity of cognition external to being
and indifferent to its nature.

But this course is the movement of being itself.
That it is being's nature to recollect itself,
and that it becomes essence by virtue of this interiorizing,
this has been displayed in being itself.

If, therefore, the absolute was
at first determined as being,
now it is determined as essence.
Cognition cannot in general stop
at the manifold of existence;
but neither can it stop at being, pure being;
immediately one is forced to the reflection that
this pure being, this negation of everything finite,
presupposes a recollection and a movement
which has distilled immediate existence into pure being.
Being thus comes to be determined as essence,
as a being in which everything determined and finite is negated.
So it is simple unity, void of determination,
from which the determinate has been
removed in an external manner;
to this unity the determinate was
itself something external
and, after this removal,
it still remains opposite to it;
for it has not been sublated in itself but relatively,
only with reference to this unity.
We already noted above that if pure essence is defined
as the sum total of all realities,
these realities are equally subject to
the nature of determinateness and abstractive reflection
and their sum total is reduced to empty simplicity.
Thus defined, essence is only a product, an artifact.
External reflection, which is abstraction, only lifts
the determinacies of being out of what is left over as essence
and only deposits them, as it were, somewhere else,
letting them exist as before.
In this way, however, essence is neither in itself nor for itself;
it is by virtue of another, through external abstractive reflection;
and it is for another, namely for abstraction
and in general for the existent
which still remains opposite to it.
In its determination, therefore,
it is a dead and empty absence of determinateness.

As it has come to be here, however,
essence is what it is,
not through a negativity foreign to it,
but through one which is its own:
the infinite movement of being.
It is being-in-and-for-itself,
absolute in-itselfness;
since it is indifferent to
every determinateness of being,
otherness and reference to other have been sublated.
But neither is it only this in-itselfness;
as merely being-in-itself, it would be only
the abstraction of pure essence;
but it is being-for-itself just as essentially;
it is itself this negativity,
the self-sublation of otherness
and of determinateness.

Essence, as the complete turning back of being into itself,
is thus at first the indeterminate essence;
the determinacies of being are sublated in it;
it holds them in itself but without their being posited in it.
Absolute essence in this simple unity with itself has no existence.
But it must pass over into existence,
for it is being-in-and-for-itself;
that is to say, it differentiates
the determinations which it holds in itself,
and, since it is the repelling of itself from itself
or indifference towards itself, negative self-reference,
it thereby posits itself over against itself
and is infinite being-for-itself
only in so far as in thus
differentiating itself from itself
it is in unity with itself.
This determining is thus of another nature than
the determining in the sphere of being,
and the determinations of essence have another character
than the determinations of being.
Essence is absolute unity of being-in-itself and being-for-itself;
consequently, its determining remains inside this unity;
it is neither a becoming nor a passing over,
just as the determinations themselves are
neither an other as other nor references to some other;
they are self-subsisting but, as such,
at the same time conjoined in the unity of essence.
Since essence is at first simple negativity,
in order to give itself existence and then being-for-itself,
it must now posit in its sphere the determinateness
which it contains in principle only in itself.

Essence is in the whole what quality was in the sphere of being:
absolute indifference with respect to limit.
Quantity is instead this indifference in immediate determination,
limit being in it an immediate external determinateness;
quantity passes over into quantum;
the external limit is necessary to it and exists in it.
In essence, by contrast, the determinateness does not exist;
it is posited only by the essence itself,
not free but only with reference to
the unity of the essence.
The negativity of essence is reflection,
and the determinations are reflected,
posited by the essence itself
in which they remain as sublated.

Essence stands between being and concept;
it makes up their middle,
its movement constituting the transition
of being into the concept.
Essence is being-in-and-for-itself,
but it is this in the determination of being-in-itself;
for its general determination is that it emerges from being
or that it is the first negation of being.
Its movement consists in positing negation
or determination in being,
thereby giving itself existence
and becoming as infinite being-for-itself
what it is in itself.
It thus gives itself its existence
which is equal to its being-in-itself
and becomes concept.
For the concept is the absolute
as it is absolutely,
or in and for itself,
in its existence.
But the existence which essence gives to itself is
not yet existence as it is in and for itself
but as essence gives it to itself or as posited,
and hence still distinct from the existence of the concept.

First, essence shines within itself
or is reflection;
second, it appears;
third, it reveals itself.

In the course of its movement,
it posits itself in the following determinations:

I. As simple essence existing in itself,
remaining in itself in its determinations;

II. As emerging into existence,
or according to its concrete existence and appearance;

III. As essence which is one with its appearance,
as actuality.

SECTION I

Essence as Reflection Within

Essence issues from being;
hence it is not immediately in and for itself
but is a result of that movement.
Or, since essence is taken at first as something immediate,
it is a determinate existence to which another stands opposed;
it is only essential existence, as against the unessential.
But essence is being which has been sublated in and for itself;
what stands over against it is only shine.
The shine, however, is essence's own positing.

First, essence is reflection.
Reflection determines itself;
its determinations are a positedness
which is immanent reflection at the same time.
Second, these reflective determinations
or essentialities are to be considered.
Third, as the reflection of its immanent determining,
essence turns into foundation and passes over
into concrete existence and appearance.

CHAPTER 1

Shine

As it issues from being, essence seems to stand over against it;
this immediate being is, first, the unessential.

But, second, it is more than just the unessential;
it is being void of essence; it is shine.

Third, this shine is not something external,
something other than essence, but is essence's own shining.
This shining of essence within it is reflection.

IV.2
nimittam aprayojakam prakrtinam varana-bheda tu tata ksetrikavat

A. THE ESSENTIAL AND THE UNESSENTIAL

Essence is sublated being.

It is simple equality with itself
but is such as the negation of
the sphere of being in general.
And so it has immediacy over against it,
as something from which it has come to be but which has
preserved and maintained itself in this sublating.
Essence itself is in this determination
an existent immediate essence,
and with reference to it
being is only something negative,
nothing in and for itself:
essence, therefore, is a determined negation.
Being and essence relate to each other in this fashion
as against others in general which are mutually indifferent,
for each has a being, an immediacy,
and according to this being they stand in equal value.

But as contrasted with essence,
being is at the same time the unessential;
as against essence, it has the determination of something sublated.
And in so far as it thus relates to essence
as an other only in general,
essence itself is not essence proper
but is just another existence, the essential.

The distinction of essential and unessential has
made essence relapse into the sphere of existence,
for as essence is at first,
it is determined with respect to being
as an existent and therefore as an other.
The sphere of existence is thus laid out as foundation,
and that in this sphere being is being-in-and-for-itself,
is a further determination external to existence,
just as, contrariwise, essence is indeed being-in-and-for-itself,
but only over against an other, in a determinate respect.
Consequently, inasmuch as essential and unessential aspects are
distinguished in an existence from each other,
this distinguishing is an external positing,
a taking apart that leaves the existence itself untouched;
it is a separation which falls on the side of
a third and leaves undetermined
what belongs to the essential
and what belongs to the unessential.
It is dependent on some external standpoint or consideration
and the same content can therefore sometimes be considered
as essential, sometimes as unessential.

On closer consideration, essence becomes something
only essential as contrasted with an unessential
because essence is only taken,
is as sublated being or existence.
In this fashion, essence is only the first negation,
or the negation, which is determinateness,
through which being becomes only existence,
or existence only an other.
But essence is the absolute negativity of being;
it is being itself, but not being determined only as an other:
it is being rather that has sublated itself
both as immediate being
and as immediate negation,
as the negation which is affected by an otherness.
Being or existence, therefore, does not persist
except as what essence is,
and the immediate which still differs from essence is not just
an unessential existence but an immediate
which is null in and for itself;
it only is a non-essence, shine.

IV.3
nirmana-cittani-asmita-matra

B. SHINE

1. Being is shine.

The being of shine consists solely
in the sublatedness of being,
in being's nothingness;
this nothingness it has in essence,
apart from its nothingness,
apart from essence, it does not exist.
It is the negative posited as negative.

Shine is all that remains of the sphere of being.
But it still seems to have an immediate side
which is independent of essence
and to be, in general, an other of essence.
Other entails in general the two moments
of existence and non-existence.
Since the unessential no longer has a being,
what is left to it of otherness is
only the pure moment of non-existence;
shine is this immediate non-existence,
a non-existence in the determinateness of being,
so that it has existence only with reference to another,
in its non-existence;
it is the non-self-subsistent
which exists only in its negation.
What is left over to it is thus only
the pure determinateness of immediacy;
it is as reflected immediacy, that is,
one which is only by virtue of
the mediation of its negation
and which, over against this mediation, is
nothing except the empty determination
of the immediacy of non-existence.

Shine, the “phenomenon” of skepticism,
and also the “appearance” of idealism,
is thus this immediacy which is not
a something nor a thing in general,
not an indifferent being that would exist apart
from its determinateness and connection with the subject.
Skepticism did not permit itself to say “It is,”
and the more recent idealism did not permit itself
to regard cognitions as a knowledge of the thing-in-itself.
The shine of the former was supposed absolutely
not to have the foundation of a being:
the thing-in-itself was not supposed
to enter into these cognitions.
But at the same time skepticism allowed
a manifold of determinations for its shine,
or rather the latter turned out to have
the full richness of the world for its content.
Likewise for the appearance of idealism:
it encompassed the full range of these manifold determinacies.
So, the shine of skepticism and the appearance of idealism
do immediately have a manifold of determination.
This content, therefore, might well have no being as foundation,
no thing or thing-in-itself;
for itself, it remains as it is;
it is simply transposed from being into shine,
so that the latter has within itself those manifold
determinacies that exist immediately,
each an other to the other.
The shine is thus itself something immediately determined.
It can have this or that content;
but whatever content it has, it has not posited it
but possesses it immediately.
Idealism, whether Leibnizian, Kantian, Fichtean, or in any
other form, has not gone further than skepticism in this:
it has not advanced beyond being as determinateness.
Skepticism lets the content of its shine to be given to it;
the shine exists for it immediately,
whatever content it might have.
The Leibnizian monad develops its representations from itself
but is not their generating and controlling force;
they rise up in it as a froth, indifferent,
immediately present to each other and to the monad as well.
Likewise Kant's appearance is a given content of perception
that presupposes affections, determinations of the subject
which are immediate to each other and to the subject.
As for the infinite obstacle of Fichte's Idealism,
it might well be that it has no thing-in-itself for foundation,
so that it becomes a determinateness purely within the “I.”
But this determinateness that the “I” makes its own,
sublating its externality,
is to the “I” at the same time an immediate determinateness,
a limitation of the “I” which the latter may transcend
but which contains a side of indifference,
and on account of this indifference,
although internal to the “I,”
it entails an immediate non-being of it.

2. Shine thus contains an immediate presupposition,
an independent side vis-à-vis essence.
But the task, inasmuch as this shine is distinct from essence,
is not to demonstrate that it sublates itself
and returns into essence,
for being has returned into essence in its totality;
shine is the null as such.
The task is to demonstrate that the determinations which
distinguish it from essence are the determinations of essence itself;
further, that this determinateness of essence,
which shine is, is sublated in essence itself.

What constitutes the shine is
the immediacy of non-being;
this non-being, however, is nothing else than
the negativity of essence within essence itself.
In essence, being is non-being.
Its inherent nothingness is the
negative nature of essence itself.
But the immediacy or indifference
which this non-being contains is
essences's own absolute in-itself.
The negativity of essence is its self-equality
or its simple immediacy and indifference.
Being has preserved itself in essence inasmuch
as this latter, in its infinite negativity,
has this equality with itself;
it is through this that essence is itself being.
The immediacy that the determinateness has
in shine against essence is
thus none other than essence's own immediacy,
though not the immediacy of an existent
but rather the absolutely mediated
or reflective immediacy which is shine;
being, not as being, but only as
the determinateness of being as against mediation;
being as moment.

These two moments [nothingness but as subsisting],
and being but as moment;
or again, negativity existing in itself and reflected immediacy,
these two moments that are the moments of shine,
are thus the moments of essence itself;
it is not that there is a shine of being in essence,
or a shine of essence in being:
the shine in the essence is not the shine of an other
but is rather shine as such, the shine of essence itself.

Shine is essence itself in the determinateness of being.
Essence has a shine because it is determined within itself
and is therefore distinguished from its absolute unity.
But this determinateness is as determinateness
just as absolutely sublated in it.
For essence is what stands on its own:
it exists as self-mediating through a negation which it itself is.
It is, therefore, the identical unit of absolute negativity and immediacy.
The negativity is negativity in itself;
it is its reference to itself and thus immediacy in itself.
But it is negative reference to itself,
a self-repelling negating;
thus the immediacy existing in itself is
the negative or the determinate over against the negativity.
But this determinateness is itself absolute negativity
and this determining, which as determining immediately sublates itself,
is a turning back into itself.

Shine is the negative which has a being,
but in another, in its negation;
it is a non-self-subsisting-being
which is sublated within and null.
And so it is the negative which returns into itself,
the non-subsistent as such, internally non-subsistent.
This reference of the negative or
the non-subsistent to itself is
the immediacy of this non-subsistent;
it is an other than it;
it is its determinateness over against it,
or the negation over against the negative.
But this negation which stands over against the negative is
negativity as referring solely to itself,
the absolute sublation of the determinateness itself.

The determinateness that shine is in essence is,
therefore, infinite determinateness;
it is only the negative which coincides with itself
and hence a determinateness that, as determinateness,
is self-subsistence and not determined.
Contrariwise, the self-subsistence, as self-referring immediacy,
equally is just determinateness and moment,
negativity solely referring to itself.
This negativity which is identical with immediacy,
and thus the immediacy which is identical with negativity, is essence.
Shine is, therefore, essence itself,
but essence in a determinateness, in such a way, however,
that the determinateness is only a moment,
and the essence is the shining of itself within itself.

In the sphere of being, non-being arises over against being,
each equally an immediate, and the truth of both is becoming.
In the sphere of essence, we have the contrast
first of essence and the non-essential,
then of essence and shine,
the non-essential and the shine
being both the leftover of being.
But these two, and no less the
distinction of essence from them,
consist solely in this:
that essence is taken at first as an immediate,
not as it is in itself,
namely as an immediacy which is immediacy
as pure mediacy or absolute negativity.
This first immediacy is thus only the determinateness of immediacy.
The sublating of this determinateness of essence consists, therefore,
in nothing further than showing that the unessential is only shine,
and that essence rather contains this shine within itself.
For essence is an infinite self-contained movement
which determines its immediacy as negativity
and its negativity as immediacy,
and is thus the shining of itself within itself.
In this, in its self-movement,
essence is reflection.

IV.3
nirmana-cittani-asmita-matra

C. REFLECTION

Shine is the same as what reflection is;
but it is reflection as immediate.
For this shine which is internalized
and therefore alienated from its immediacy,
the German has a word from an alien language, “Reflexion.”

Essence is reflection, the movement of
becoming and transition that remains within itself,
wherein that which is distinguished is determined
simply and solely as the negative in itself, as shine.
In the becoming of being, it is being which lies
at the foundation of determinateness,
and determinateness is reference to an other.
Reflective movement is by contrast
the other as negation in itself,
a negation which has being only as self-referring.
Or, since this self-referring is precisely this negating of negation,
what we have is negation as negation,
negation that has its being in its being-negated, as shine.
Here, therefore, the other is not being with negation or limit,
but negation with negation.
But the first over against this other, the immediate or being,
is only this self-equality itself of negation,
the negated negation, the absolute negativity.
This self-equality or immediacy, therefore, is
not a first from which the beginning is made
and which would pass over into its negation;
nor is there an existent substrate which would
go through the moves of reflection;
immediacy is rather just this movement itself.

In essence, therefore, the becoming,
the reflective movement of essence,
is the movement from nothing to nothing
and thereby back to itself.
Transition or becoming sublates itself in its transition;
the other which comes to be in this transition is
not the non-being of a being but the nothingness of a nothingness,
and this, to be the negation of a nothingness, constitutes being.
Being is only as the movement of nothingness to nothingness,
and so it is essence;
and this essence does not have this movement in itself,
but the movement is rather the absolute shine itself,
the pure negativity which has nothing outside it
which it would negate but which rather negates only its negative,
the negative which is only in this negating.
This pure absolute reflection, which is the movement
from nothing to nothing, further determines itself.

It is, first, positing reflection.

Second, it takes as its starting point
the presupposed immediate,
and then it is external reflection.

Third, it sublates however this presupposition,
and because in the sublating of the presupposition
it presupposes at the same time,
it is determining reflection.

1. Positing reflection

Shine is a nothingness or a lack of essence.
But a nothingness or that which is void of essence
does not have its being in an other in which it shines,
but its being is its own equality with itself;
this conversion of the negative with itself has been determined
as the absolute reflection of essence.

This self-referring negativity is
therefore the negating of itself.
It is thus just as much
sublated negativity as it is negativity.
Or again, it is itself the negative
and the simple equality with itself or immediacy.
It consists, therefore, in being itself
and not being itself,
and the two in one unity.

Reflection is at first the movement of
the nothing to the nothing,
and thus negation coinciding with itself.
This self-coinciding is in general
simple equality with itself, immediacy.
But this falling together is not
the transition of negation into equality
as into a being other than it;
reflection is transition rather
as the sublating of transition,
for it is the immediate falling together
of the negative with itself.
And so this coinciding is, first,
self-equality or immediacy;
but, second, this immediacy is
the self-equality of the negative,
and hence self-negating equality,
immediacy which is in itself the negative,
the negative of itself:
its being is to be what it is not.

The self-reference of the negative is
therefore its turning back into itself;
it is immediacy as the sublating of the negative,
but immediacy simply and solely as this reference
or as turning back from a one,
and hence as self-sublating immediacy.
This is positedness,
immediacy purely as determinateness
or as self-reflecting.
This immediacy, which is only as
the turning back of the negative into itself,
is the immediacy which constitutes the determinateness of shine,
and from which the previous reflective movement seemed to begin.
But, far from being able to begin with this immediacy,
the latter first is rather as the turning back
or as the reflection itself.
Reflection is therefore the movement which,
since it is the turning back,
only in this turning is that
which starts out or returns.

It is a positing, inasmuch as it is
immediacy as a turning back;
that is to say, there is not an other beforehand,
one either from which or to which it would turn back;
it is, therefore, only as a turning back
or as the negative of itself.
But further, this immediacy is sublated negation
and sublated return into itself.
Reflection, as the sublating of the negative, is
the sublating of its other, of the immediacy.
Because it is thus immediacy as a turning back,
the coinciding of the negative with itself,
it is equally the negation of the negative as negative.
And so it is presupposing.
Or immediacy is as a turning back
only the negative of itself,
just this, not to be immediacy;
but reflection is the sublating
of the negative of itself,
coincidence with itself;
it therefore sublates its positing,
and inasmuch as it is in its positing
the sublating of positing, it is presupposing.
In presupposing, reflection determines the turning back
into itself as the negative of itself,
as that of which essence is the sublating.
It is its relating to itself,
but to itself as to the negative of itself;
only so is it negativity which abides with itself,
self-referring negativity.
Immediacy comes on the scene simply and solely
as a turning back and is that negative
which is the semblance of a beginning,
the beginning which the return negates.
The turning back of essence is therefore its self-repulsion.
Or inner directed reflection is essentially
the presupposing of that from which
the reflection is the turning back.

It is only by virtue of the sublating of its equality with itself
that essence is equality with itself.
Essence presupposes itself, and the sublating of
this presupposing is essence itself;
contrariwise, this sublating of its presupposition is
the presupposition itself.
Reflection thus finds an immediate before it
which it transcends and from which it is the turning back.
But this turning back is only the presupposing of
what was antecedently found.
This antecedent comes to be only by being left behind;
its immediacy is sublated immediacy.
The sublated immediacy is, contrariwise, the turning
back into itself,
essence that arrives at itself, simple being equal to itself.
This arriving at itself is thus the sublating of itself
and self-repelling, presupposing reflection,
and its repelling of itself from itself is
the arriving at itself.

It follows from these considerations that
the movement of reflection is to be taken as
an absolute internal counter-repelling.
For the presupposition of
the turning back into itself
[that from which essence arises,
essence being only as this coming back]
is only in the turning back itself.
Transcending the immediate from which reflection begins
occurs rather only through this transcending;
and the transcending of the immediate is
the arriving at the immediate.
The movement, as forward movement, turns immediately
around into itself and so is only self-movement:
a movement which comes from itself in so far as
positing reflection is presupposing reflection, yet,
as presupposing reflection, is simply positing reflection.

Thus is reflection itself and its non-being,
and only is itself by being the negative of itself,
for only in this way is the sublating of the negative
at the same time a coinciding with itself.

The immediacy which reflection,
as a process of sublating,
presupposes for itself is
simply and solely a positedness,
something in itself sublated
which is not diverse from
reflection's turning back into itself
but is itself only this turning back.
But it is at the same time determined as a negative,
as immediately in opposition to something,
and hence to an other.
And so is reflection determined.
According to this determinateness,
because reflection has a presupposition
and takes its start from the immediate as its other,
it is external reflection.

2. External reflection

Reflection, as absolute reflection,
is essence shining within,
essence that posits only shine,
only positedness, for its presupposition;
and as presupposing reflection,
it is immediately only positing reflection.
But external or real reflection
presupposes itself as sublated,
as the negative of itself.
In this determination, it is doubled.
At one time it is as what is presupposed,
or the reflection into itself which is the immediate.
At another time, it is as the reflection
negatively referring to itself;
it refers itself to itself as
to that its non-being.

External reflection thus presupposes a being,
at first not in the sense that
its immediacy is only positedness or moment,
but in the sense rather that
this immediacy refers to itself
and the determinateness is only as moment.
Reflection refers to its presupposition in such a way
that the latter is its negative,
but this negative is thereby sublated as negative.
Reflection, in positing, immediately sublates its positing,
and so it has an immediate presupposition.
It therefore finds this presupposition before it
as something from which it starts,
and from which it only makes its way back into itself,
negating it as its negative.
But that this presupposition is a negative
or a positedness is not its concern;
this determinateness belongs only to positing reflection,
whereas in the presupposing positedness
it is only as sublated.
What external reflection determines and posits in the immediate
are determinations which to that extent are external to it.
In the sphere of being, external reflection was the infinite;
the finite stands as the first,
as the real from which the beginning is made
as from a foundation that abides,
whereas the infinite is the reflection into itself
standing over against it.

This external reflection is the syllogism
in which the two extremes are
the immediate and the reflection into itself;
the middle term is the reference connecting the two,
the determinate immediate, so that one part of this connecting reference,
the immediate, falls to one extreme alone, and the other,
the determinateness or the negation, only to the other extreme.

But if one takes a closer look at what the external reflection does,
it turns out that it is, secondly, the positing of the immediate,
an immediate which thus becomes the negative or the determined;
but it is immediately also the sublating of this positing,
for it presupposes the immediate;
in negating, it is the negating of its negating.
But thereby it immediately is equally a positing,
the sublating of the immediate which is its negative;
and this negative, from which it seemed to begin
as from something alien,
only is in this its beginning.
In this way, the immediate is not only implicitly in itself
(that is, for us or in external reflection)
the same as what reflection is,
but is posited as being the same.
For the immediate is determined by reflection as
the negative of the latter or as the other of it,
but it is reflection itself which negates this determining.
The externality of reflection vis-à-vis
the immediate is consequently sublated;
its self-negating positing is its coinciding
with its negative, with the immediate,
and this coinciding is the immediacy of essence itself.
It thus transpires that external reflection is not external
but is just as much the immanent reflection of immediacy itself;
or that the result of positing reflection is
essence existing in and for itself.
External reflection is thus determining reflection.

3. Determining reflection

Determining reflection is in general
the unity of positing and external reflection.
This is now to be examined more closely.

1. External reflection begins from immediate being,
positing reflection from nothing.
In its determining, external reflection posits another in the
place of the sublated being, but this other is essence;
the positing does not posit its determination in the place of an other;
it has no presupposition.
But, precisely for this reason,
it is not complete as determining reflection;
the determination which it posits is consequently only a posited;
this is an immediate, not however as equal to itself
but as self-negating;
its connection with the turning back into itself is absolute;
it is only in the reflection-into-itself
but is not this reflection itself.

The posited is therefore an other,
but in such a manner that the self-equality
of reflection is retained;
for the posited is only as sublated,
as reference to the turning back into itself.
In the sphere of being, existence was the being
that had negation in it, and being was the immediate ground
and element of this negation which was,
therefore, itself immediate negation.
In the sphere of essence,
positedness is what corresponds to existence.
Positedness is equally an existence,
but its ground is being as essence
or as pure negativity;
it is a determinateness or a negation,
not as existent but immediately as sublated.
Existence is only positedness;
this is the principle of the essence of existence.
Positedness stands on the one side over against existence,
and over against essence on the other:
it is to be regarded as the means which conjoins
existence with essence and essence with existence.
If it is said, a determination is only a positedness,
the claim can thus have a twofold meaning,
according to whether the determination is such
in opposition to existence or in opposition to essence.
In either meaning, existence is taken for
something superior to positedness,
which is attributed to external reflection, to the subjective.
In fact, however, positedness is the superior, because, as posited,
existence is what it is in itself something negative,
something that refers simply and solely to the turning back into itself.
For this reason positedness is only a positedness
with respect to essence:
it is the negation of this turning back
as achieved return into itself.

2. Positedness is not yet a determination of reflection;
it is only determinateness as negation in general.
But the positing is now united with external reflection;
in this unity, the latter is absolute presupposing, that is,
the repelling of reflection from itself
or the positing of determinateness as its own.
As posited, therefore, positedness is negation;
but as presupposed, it is reflected into itself.
And in this way positedness is a determination of reflection.

The determination of reflection is distinct
from the determinateness of being, of quality;
the latter is immediate reference to other in general;
positedness also is reference to other,
but to immanently reflected being.
Negation as quality is existent negation;
being constitutes its ground and element.
The determination of reflection, on the contrary,
has for this ground immanent reflectedness.
Positedness gets fixed in determination precisely
because reflection is self-equality in its negatedness;
the latter is therefore itself reflection into itself.
Determination persists here, not by virtue of being
but because of its self-equality.
Since the being which sustains quality is
unequal to the negation, quality is
consequently unequal within itself,
and hence a transient moment which disappears in the other.
The determination of reflection is
on the contrary positedness as negation,
negation which has negatedness for its ground,
is therefore not unequal to itself within itself,
and hence essential rather than transient determinateness.
What gives subsistence to it is the self-equality of reflection
which has the negative only as negative,
as something sublated or posited.

Because of this reflection into themselves,
the determinations of reflection appear as
free essentialities, sublated in the void
without reciprocal attraction or repulsion.
In them the determinateness has become entranced
and infinitely fixed by virtue of the reference to itself.
It is the determinate which has subjugated its transitoriness
and its mere positedness to itself, that is to say,
has deflected its reflection-into-other into reflection-into- itself.
These determinations hereby constitute the determinate shine
as it is in essence, the essential shine.
Determining reflection is for this reason
reflection that has exited from itself;
the equality of essence with itself is
lost in the negation, and negation predominates.

Thus there are two distinct sides to the determination of reflection.
First, reflection is positedness, negation as such;
second, it is immanent reflection.
According to the side of positedness,
it is negation as negation,
and this already is its unity with itself.
But it is this unity at first only implicitly or in itself,
an immediate which sublates itself within, is the other of itself.
To this extent, reflection is a determining that abides in itself.
In it essence does not exit from itself;
the distinctions are solely posited,
taken back into essence.
But, from the other side, they are not posited
but are rather reflected into themselves;
negation as negation is equality with itself,
not in its other, not reflected into its non-being.

3. Now keeping in mind that the determination of reflection is
both immanently reflected reference and positedness as well,
its nature immediately becomes more transparent.
For, as positedness, the determination is negation as such,
a non-being as against another, namely,
as against the absolute immanent reflection or as against essence.
But as self-reference, it is reflected within itself.
This, the reflection of the determination,
and that positedness are distinct;
its positedness is rather the sublatedness of the determination
whereas its immanent reflectedness is its subsisting.
In so far as now the positedness is
at the same time immanent reflection,
the determinateness of the reflection is
the reference in it to its otherness.
It is not a determinateness that exists quiescent,
one which would be referred to an other
in such a way that the referred term
and its reference would be different,
each something existing in itself,
each a something that excludes its other
and its reference to this other from itself.
Rather, the determination of reflection is
within it the determinate side
and the reference of this determinate side as determinate,
that is, the reference to its negation.
Quality, through its reference, passes over into another;
its alteration begins in its reference.
The determination of reflection, on the contrary,
has taken its otherness back into itself.
It is positedness, negation which has however deflected
the reference to another into itself,
and negation which, equal to itself,
is the unity of itself and its other,
and only through this is an essentiality.
It is, therefore, positedness, negation,
but as reflection into itself it is at the same time
the sublatedness of this positedness,
infinite reference to itself.

CHAPTER 2

Foundation

The essentialities or the determinations of reflection

Reflection is determined reflection;
accordingly, essence is determined essence, or it is essentiality.

Reflection is the shining of essence within itself.

Essence, as infinite immanent turning back is
not immediate simplicity, but negative simplicity;
it is a movement across moments that are distinct,
is absolute mediation with itself.
But in these moments it shines;
the moments are, therefore, themselves
determinations reflected into themselves.

First, essence is simple self-reference, pure identity.
This is its determination, one by which it is rather
the absence of determination.

Second, the specifying determination is difference,
difference which is either external or indefinite,
diversity in general, or opposed diversity or opposition.

Third, as contradiction this opposition is reflected into itself
and returns to its foundation.

IV.4
pravrtti-bhede prayojakam cittam ekam anekesam

A. IDENTITY

1. Essence is simple immediacy as sublated immediacy.
Its negativity is its being;
it is equal to itself in its absolute negativity
by virtue of which otherness and reference to other
have as such simply disappeared into pure self-equality.
Essence is therefore simple self-identity.

This self-identity is the immediacy of reflection.
It is not that self-equality which being is, or also nothing,
but a self-equality which, in producing itself as unity,
does not produce itself over again, as from another,
but is a pure production, from itself and in itself,
essential identity.

It is not, therefore, abstract identity
or an identity which is the result
of a relative negation preceding it,
one that separates indeed
what it distinguishes from it
but, for the rest, leaves it existing outside it,
the same after as before.

Being, and every determinateness of being,
has rather sublated itself not relatively,
but in itself, and this simple negativity,
the negativity of being in itself,
is the identity itself.

In general, therefore,
it is still the same as essence.

B. DIFFERENCE

1. Absolute difference

Difference is the negativity that
reflection possesses in itself,
the nothing which is said in identity discourse,
the essential moment of identity itself
which, as the negativity of itself,
at the same time determines itself
and is differentiated from difference.

1. This difference is difference in and for itself,
absolute difference, the difference of essence.
It is difference in and for itself,
not difference through something external
but self-referring, hence simple, difference.
It is essential that we grasp absolute difference as simple.
In the absolute difference of A and not-A from each other,
it is the simple “not” which, as such,
constitutes the difference.
Difference itself is a simple concept.
“In this,” so it is said, “two things differ, in that etc.”
“In this,” that is, in one and the same respect,
relative to the same basis of determination.
It is the difference of reflection,
not the otherness of existence.
One existence and another existence are
posited as lying outside each other;
each of the two existences thus
determined over against each other
has an immediate being for itself.
The other of essence, by contrast,
is the other in and for itself,
not the other of some other
which is to be found outside it;
it is simple determinateness in itself.
Also in the sphere of existence
did otherness and determinateness
prove to be of this nature,
simple determinateness, identical opposition;
but this identity showed itself only as
the transition of a determinateness into the other.
Here, in the sphere of reflection,
difference comes in as reflected,
so posited as it is in itself.

2. Difference in itself is the difference
that refers itself to itself;
thus it is the negativity of itself,
the difference not from another
but of itself from itself;
it is not itself but its other.
What is different from difference, however, is identity.
Difference is, therefore, itself and identity.
The two together constitute difference;
difference is the whole and its moment.
One can also say that difference,
as simple difference, is no difference;
it is such only with reference to identity;
even better, that as difference it entails
itself and this reference equally.
Difference is the whole and its own moment,
just as identity equally is its whole and its moment.
This is to be regarded as
the essential nature of reflection
and as the determined primordial origin
of all activity and self-movement.
Both difference and identity make themselves
into moment or positedness
because, as reflection, they are negative self-reference.
Difference, thus as unity of itself and of identity,
is internally determined difference.
It is not the transition into another,
not reference to another outside it;
it has its other, identity, within,
and in like manner identity,
in being included in the determination of difference,
has not lost itself in it as its other
but retains itself therein is
the reflection-into-itself of difference, its moment.

3. Difference has both these moments,
identity and difference;
thus the two are both a positedness, determinateness.
But in this positedness each refers to itself.
The one, identity, is itself immediately
the moment of immanent reflection;
but no less is the other, difference,
difference in itself, reflected difference.
Difference, inasmuch as it has two such moments
which are themselves reflections into themselves,
is diversity.

2. Diversity

1. Identity internally breaks apart into diversity
because, as absolute difference in itself,
it posits itself as the negative of itself
and these, its two moments
(itself and the negative of itself),
are reflections into themselves,
are identical with themselves;
or precisely because it itself
immediately sublates its negating
and is in its determination reflected into itself.
The different subsists as diverse,
indifferent to any other,
because it is identical with itself,
because identity constitutes its base and element;
or, the diverse remains what it is
even in its opposite, identity.

Diversity constitutes the otherness
as such of reflection.
The other of existence has immediate being,
where negativity resides, for its foundation.
But in reflection it is self-identity,
the reflected immediacy, that constitutes
the subsistence of the negative and its indifference.

The moments of difference are identity and difference itself.
These moments are diverse when reflected into themselves,
referring themselves to themselves;
thus, in the determination of identity,
they are only self-referring;
identity is not referred to difference,
nor is difference referred to identity;
hence, inasmuch as each of these moments is
referred only to itself, the two
are not determined
with respect to each other.
Now because in this way the two are not differentiated within,
the difference is external to them.
The diverse moments, therefore,
conduct themselves with respect to each other,
not as identity and difference,
but only as moments different in general,
indifferent to each other and to their determinateness.

2. In diversity, as the indifference of difference,
reflection has in general become external;
difference is only a positedness or as sublated,
but is itself the whole reflection.
On closer consideration, both, identity and difference
are reflections, as we have just established;
each is the unity of it and its other,
each is the whole.
But the determinateness,
to be only identity or only difference,
is thus a sublated something.
They are not, therefore, qualities,
since their determinateness,
because of the immanent reflection,
is at the same time only as negation.
What we have is therefore this duplicity,
immanent reflection as such
and determinateness as negation or positedness.
Positedness is the reflection that is external to itself;
it is negation as negation
and consequently, indeed in itself
self-referring negation and immanent reflection,
but only in itself, implicitly;
its reference is to a something external.

Reflection in itself and external reflection are
thus the two determinations
in which the moments of difference,
identity and difference, are posited.
They are these moments themselves
as they have determined themselves at this point.
Immanent reflection is identity,
but determined to be indifferent to difference,
not to have difference at all but to conduct
itself towards difference as identical with itself;
it is diversity.
It is identity that has so reflected itself into itself
that it truly is the one reflection of
the two moments into themselves;
both are immanent reflections.
Identity is this one reflection of the two,
the identity which has difference within it
only as an indifferent difference
and is diversity in general.
External reflection, on the contrary,
is their determinate difference,
not as absolute immanent reflection,
but as a determination towards which
the implicitly present reflection is indifferent;
its two moments, identity and difference themselves,
are thus externally posited,
are not determinations that
exist in and for themselves.

Now this external identity is likeness,
and external difference is unlikeness.
Likeness is indeed identity,
but only as a positedness,
an identity which is not in and for itself.
Unlikeness is equally difference,
but an external difference which is not, in and for itself,
the difference of the unlike itself.
Whether something is like or unlike something else is
not the concern of either the like or the unlike;
each refers only to itself, each is in and for itself what it is;
identity or non-identity, in the sense of likeness or unlikeness,
depend on the point of view of a third external to them.

3. External reflection connects diversity by
referring it to likeness and unlikeness.
This reference, which is a comparing,
moves back and forth from likeness
to unlikeness and from unlikeness to likeness.
But this back and forth referring of
likeness and unlikeness is
external to these determinations themselves;
moreover, they are not referred to each other,
but each, for itself, is referred to a third.
In this alternation,
each immediately stands out on its own.
External reflection is as such external to itself;
determinate difference is negated absolute difference;
it is not simple difference, therefore,
not an immanent reflection,
but has this reflection outside it;
hence its moments come apart
and both refer,
each also outside the other,
to the immanent reflection
confronting them.

In reflection thus alienated from itself,
likeness and unlikeness present themselves,
therefore, as themselves unconnected,
and reflection keeps them apart,
for it refers them to one and the same
something by means of “in so far,”
“from this side or that,”
and “from this view or that.”
Thus diverse things that are one and the same,
when likeness and unlikeness are said of them,
are from one side like each other,
but from another side unlike,
and in so far as they are alike,
to that extent they are not unlike.
Likeness thus refers only to itself,
and unlikeness is equally only unlikeness.

Because of this separation from each other,
they sublate themselves.
Precisely that which should save them
from contradiction and dissolution,
namely that something is like another in one respect
but unlike in another precisely this keeping of
likeness and unlikeness apart, is their destruction.
For both are determinations of difference;
they are references to each other,
each intended to be what the other is not;
the like is not the unlike,
and the unlike is not the like;
both have this connecting reference essentially,
and have no meaning outside it;
as determinations of difference,
each is what it is as different from its other.
But because of their indifference to each other,
the likeness is referred to itself,
and similarly is unlikeness a point of view of
its own and a reflection unto itself;
each, therefore, is like itself;
difference has vanished, since they have no
determinateness to oppose them;
in other words, each is consequently only likeness.

Accordingly, this indifferent viewpoint
or the external difference sublates itself
and it is in itself the negativity of itself.
It is the negativity which in comparing
belongs to that which does the comparing.
This latter oscillates from likeness
to unlikeness and back again;
hence it lets the one disappear into the other
and is in fact the negative unity of both.
This negative unity transcends at first
what is compared as well as
the moments of the comparing as
a subjective operation that falls outside them.
But the result is that this unity is
in fact the nature of likeness and unlikeness themselves.
Even the independent viewpoint
that each of these is,
is rather the self-reference
that sublates their distinctness
and so, too, themselves.

From this side, as moments of external reflection
and as external to themselves,
likeness and unlikeness disappear together into their likeness.
But this, their negative unity,
is in addition also posited in them;
for their reflection implicitly exists
outside them, that is, they are the likeness and
unlikeness of a third,
of another than they themselves are.
Thus the like is not the like of itself,
and the unlike, as the unlike not of itself
but of an unlike to it,
is itself the like.
The like and the unlike is
each therefore the unlike of itself.
Each is thereby this reflection:
likeness, that it is itself and the unlikeness;
unlikeness, that it is itself and the likeness.
Likeness and unlikeness constituted
the side of positedness as against
what is being compared or the diverse
which, as contrasted with them,
had determined itself as implicitly existent reflection.
But this positedness has consequently equally
lost its determinateness as against this reflection.

Likeness and unlikeness,
the determinations of external reflection,
are precisely the merely
implicitly existent reflection
which the diverse as such was supposed to be,
its only indeterminate difference.
Implicitly existent reflection is
self-reference without negation,
abstract self-identity
and therefore positedness itself.
The merely diverse thus passes over
through the positedness
into negative reflection.
The diverse is difference
which is merely posited,
hence a difference which is no difference,
hence a negation that negates itself within.
Likeness and unlikeness themselves, the positedness,
thus return through indifference
or through implicitly existing reflection
back into negative unity with themselves,
into the reflection which is
the implicit difference of likeness and unlikeness.
Diversity, the indifferent sides of which
are just as much simply and solely
moments of a negative unity, is opposition.

3. Opposition

In opposition, the determinate reflection,
difference, is brought to completion.
Opposition is the unity of identity and diversity;
its moments are diverse in one identity,
and so they are opposites.

Identity and difference are the moments of
difference as held inside difference itself;
they are reflected moments of its unity.
Likeness and unlikeness are instead
the externalized reflection;
their self-identity is not only the indifference
of each towards the other differentiated from it,
but towards being-in-and-for-itself as such;
theirs is a self-identity that contrasts with
identity reflected into itself,
hence an immediacy which is not reflected into itself.
The positedness of the sides of
external reflection is therefore a being,
just as their non-positedness is a non-being.

On closer consideration, the moments of opposition are
positedness reflected into itself
or determination in general.
Positedness is likeness and unlikeness;
these two, reflected into themselves,
constitute the determinations of opposition.
Their immanent reflection consists in that
each is within it the unity of likeness and unlikeness.
Likeness is only in a reflection
which compares according to the unlikeness
and is therefore mediated by its
other indifferent moment; similarly,
unlikeness is only in the same
reflective reference in which likeness is.
Each of these moments, in its determinateness,
is therefore the whole.
It is the whole because it also contains its other moment;
but this, its other, is an indifferent existent;
thus each contains a reference to its non-being,
and it is reflection-into-itself, or the whole,
only as essentially referring to its non-being.

This self-likeness, reflected into itself
and containing the reference to
unlikeness within it, is the positive;
and the unlikeness that contains within itself
the reference to its non-being,
to likeness, is the negative.
Or again, both are positedness;
now in so far as the differentiated determinateness is
taken as a differentiated determinate reference of
positedness to itself, opposition is, on the one hand,
positedness reflected into its likeness with itself;
and, on the other hand, it is the same positedness
reflected into its inequality with itself:
the positive and the negative.
The positive is positedness as reflected into self-likeness;
but what is reflected is positedness, that is,
the negation as negation,
and so this immanent reflection has
the reference to the other for its determination.
The negative is positedness as reflected into unlikeness;
but positedness is the unlikeness itself,
and so this reflection is therefore
the identity of unlikeness with itself
and absolute self-reference.
Each, therefore, equally has the other in it:
positedness reflected into self-likeness has the unlikeness;
and positedness reflected into self-unlikeness, the likeness.

The positive and the negative are thus
the sides of opposition that have become self-subsisting.
They are self-subsisting because they are
the reflection of the whole into itself,
and they belong to opposition in so far
as the latter is determinateness
which, as the whole, is reflected into itself.
Because of their self-subsistence,
the opposition which they constitute is
implicitly determinate.
Each is itself and its other;
for this reason, each has its determinateness
not in an other but within.
Each refers itself to itself
only as referring itself to its other.
This has a twofold aspect.
Each is the reference to its non-being as
the sublating of this otherness in itself;
its non-being is thus only a moment in it.
But, on the other hand, here positedness
has become a being, an indifferent subsistence;
the other of itself which each contains is
therefore also the non-being of that in which
it should be contained only as a moment.
Each is, therefore, only to the
extent that its non-being is,
the two in an identical reference.

The determinations which constitute
the positive and the negative consist,
therefore, in that the positive and the negative are,
first, absolute moments of opposition;
their subsistence is indivisibly one reflection;
it is one mediation in which each is
by virtue of the non-being of its other,
hence by virtue of its other
or its own non-being.
Thus they are simply opposites;
or each is only the opposite of the other;
the one is not yet the positive
and the other not yet the negative,
but both are negative with
respect to each other.
Each, therefore, simply is,
first, to the extent that the other is;
it is what it is by virtue of the other,
by virtue of its own non-being;
it is only positedness.
Second, it is to the extent that
the other is not; it is what it is
by virtue of the non-being of the other;
it is reflection into itself.
The two, however, are both
the one mediation of opposition as such
in which they simply are only posited moments.

Moreover, this mere positedness is
reflected into itself in general
and, according to this moment of external reflection,
the positive and the negative are indifferent towards
this first identity where they are only moments;
or again, because that first reflection is
the positive's and the negative's own
reflection into itself,
each is indifferent towards its reflection
into its non-being, towards its own positedness.
The two sides are thus merely diverse,
and because their determinateness
that they are positive or negative
constitutes their positedness as against each other,
each is not specifically so determined internally
but is only determinateness in general;
to each side, therefore, there belongs indeed
one of the two determinacies,
the positive or the negative;
but the two can be interchanged,
and each side is such as
can be taken equally as positive or negative.

But, in third place, the positive and the negative are
not only a posited being,
nor are they something merely indifferent,
but their positedness,
or the reference to the other in the one unity
which they themselves are not,
is rather taken back into each.
Each is itself positive and negative within;
the positive and the negative are
the determination of reflection in and for itself;
only in this reflection of the opposite into itself is
the opposite either positive or negative.
The positive has within it the reference to
the other in which the determinateness of the positive consists.
And the same applies to the negative:
it is not negative as contrasted with another
but has the determinateness by which it is negative within.

Each is thus self-subsistent unity existing for itself.
The positive is indeed a positedness,
but in such a way that the positedness is
for it posited being as sublated.
It is the non-opposed, the sublated opposition,
but as the side of the opposition itself.
As positive, it is indeed a something
which is determined with reference to an otherness,
but in such a way that its nature
is not to be something posited;
it is the immanent reflection
that negates otherness.
But its other, the negative,
is itself no longer positedness or a moment
but itself a self-subsisting being
and so the negating reflection
of the positive is internally
determined to exclude this being,
which is its non-being, from itself.

Thus the negative, as absolute reflection,
is not the immediate negative
but is the negative as sublated positedness,
the negative in and for itself
which positively rests upon itself.
As immanent reflection,
it negates its reference to its other;
its other is the positive,
a self-subsisting being
hence its negative reference
to this positive is
the excluding of it from itself.
The negative is the independently existing opposite,
over against the positive
which is the determination of the sublated opposition,
the whole opposition resting upon itself,
opposed to the self-identical positedness.

The positive and the negative are such, therefore,
not just in themselves, but in and for themselves.
They are in themselves positive and negative
when they are abstracted from their excluding
reference to the other
and are taken only in accordance
with their determination.
Something is in itself positive or negative
when it is not supposed to be
determined as positive or negative
merely in contrast with the other.
But the positive and the negative,
taken not as a positedness
and hence not as opposed,
are each an immediate,
being and non-being.
They are, however, moments of opposition:
their in-itself constitutes only
the form of their immanent reflectedness.
Something is said to be positive in itself,
outside the reference to something negative,
and something negative in itself,
outside the reference to something negative:
in this determination, merely the abstract moment of
this reflectedness is held on to.
However, to say that the positive and the negative
exist in themselves essentially implies that
to be opposed is not a mere moment,
nor that it is just a matter of comparison,
but that it is the determination of the sides
themselves of the opposition.
The sides, as positive or negative in themselves,
are not, therefore, outside the reference to the other;
on the contrary, this reference, precisely as exclusive,
constitutes their determination or their in-itselfness;
in this, therefore, they are at the same time
in and for themselves.

IV.5
tatra dhyana-jam anasayam

C. CONTRADICTION

1. Difference in general contains both its sides as moments;
in diversity, these sides fall apart as indifferent to each other;
and in opposition as such, they are the moments of difference,
each determined by the other and hence only moments.
But in opposition these moments are equally determined within,
indifferent to each other and mutually exclusive,
self-subsisting determinations of reflection.

One is the positive and the other the negative,
but the former as a positive which is such within,
and the latter as a negative which is such within.
Each has indifferent self-subsistence for itself
by virtue of having the reference to
its other moment within it;
each moment is thus the whole self-contained opposition.
As this whole, each moment is self-mediated
through its other and contains this other.
But it is also self-mediated
through the non-being of its other
and is, therefore, a unity existing for itself
and excluding the other from itself.

Since the self-subsisting determination of reflection
excludes the other in the same respect as it contains it
and is self-subsisting for precisely this reason,
in its self-subsistence the determination excludes
its own self-subsistence from itself.
For this self-subsistence consists in
that it contains the determination
which is other than it in itself
and does not refer to anything external
for just this reason;
but no less immediately in that
it is itself and excludes from itself
the determination that negates it.
And so it is contradiction.

Difference as such is already implicitly contradiction;
for it is the unity of beings which are,
only in so far as they are not one
and it is the separation of beings which are,
only in so far as they are separated
in the same reference connecting them.
The positive and the negative, however,
are the posited contradiction,
for, as negative unities,
they are precisely their self-positing
and therein each the sublating of itself
and the positing of its opposite.
They constitute determining reflection as exclusive;
for the excluding is one act of distinguishing
and each of the distinguished beings,
as exclusive, is itself the whole act of excluding,
and so each excludes itself internally.

If we look at the two self-subsisting
determinations of reflection on their own,
the positive is positedness as reflected
into likeness with itself
positedness which is not reference to another,
hence subsistence inasmuch as
the positedness is sublated and excluded.
But with this the positive makes itself
into the reference of a non-being into a positedness.
In this way the positive is contradiction in that,
as the positing of self-identity by the
excluding of the negative,
it makes itself into a negative,
hence into the other
which it excludes from itself.
This last, as excluded, is posited
free of the one that excludes;
hence, as reflected into itself and itself as excluding.
The reflection that excludes is thus
the positing of the positive as excluding the other,
so that this positing immediately is
the positing of its other which excludes it.

This is the absolute contradiction of the positive;
but it is immediately the absolute contradiction of the negative;
the positing of both in one reflection.
Considered in itself as against the positive,
the negative is positedness as reflected into unlikeness to itself,
the negative as negative.
But the negative is itself the unlike,
the non-being of another;
consequently, reflection is in its unlikeness
its reference rather to itself.
Negation in general is the negative
as quality or immediate determinateness;
but taken as negative, it is referred to
the negative of itself, to its other.
If this second negative is taken only
as identical with the first,
then it is also only immediate,
just like the first;
they are not taken, therefore,
as each the other of the other,
hence not as negatives:
the negative is not at all an immediate.
But now, since each is moreover equally
the same as what the other is,
this reference connecting them as unequal
is just as much their identical connection.

This is therefore the same contradiction which the positive is,
namely positedness or negation as self-reference.
But the positive is only implicitly this contradiction,
is contradiction only in itself;
the negative, on the contrary, is the posited contradiction;
for in its reflection into itself,
as a negative which is in and for itself
or a negative which is identical with itself,
its determination is to be the not-identical,
the exclusion of identity.
The negative is this,
to be identical with itself over against identity,
and consequently, because of this excluding reflection,
to exclude itself from itself.

The negative is therefore the whole opposition
the opposition which, as opposition, rests upon itself;
distinction that absolutely does not refer itself to another;
distinction which, as opposition, excludes identity from itself,
but thereby also excludes itself,
for as reference to itself it determines itself
as the very identity which it excludes.

2. Contradiction resolves itself.

In the self-excluding reflection
we have just considered,
the positive and the negative,
each in its self-subsistence,
sublates itself;
each is simply the passing over,
or rather the self-translating of itself into its opposite.
This internal ceaseless vanishing of the opposites is
the first unity that arises by virtue of contradiction;
it is the null.

But contradiction does not contain merely the negative;
it also contains the positive;
or the self-excluding reflection is
at the same time positing reflection;
the result of contradiction is not only the null.
The positive and the negative constitute
the positedness of the self-subsistence;
their own self-negation sublates it.
It is this positedness which in truth
founders to the ground in contradiction.

The immanent reflection by virtue of which
the sides of opposition are turned into
self-subsistent self-references is,
first of all, their self-subsistence as distinct moments;
thus they are this self-subsistence only in themselves,
for they are still opposites,
and that they are in themselves self-subsistent
constitutes their positedness.
But their excluding reflection
sublates this positedness,
turns them into self-subsistent beings
existing in and for themselves,
such as are self-subsistent not only in themselves
but by virtue of their negative reference to their other;
in this way, their self-subsistence is also posited.
But, further, by thus being posited as self-subsistent,
they make themselves into a positedness.
They fate themselves to founder,
since they determine themselves as self-identical,
yet in their self-identity they are rather the negative,
a self-identity which is reference-to-other.

However, on closer examination,
this excluding reflection is not only
this formal determination.
It is self-subsistence existing in itself,
and the sublating of this positedness
is only through this sublating a unity that
exists for itself and is in fact self-subsistent.
Of course, through the sublating of otherness or positedness,
positedness or the negative of an other is indeed present again.
But in fact, this negation is not just a return to the first
immediate reference to the other,
is not positedness as sublated immediacy,
but positedness as sublated positedness.
The excluding reflection of self-subsistence,
since it is excluding,
makes itself a positedness but is just as much
the sublation of its positedness.
It is sublating reference to itself;
in that reference, it first sublates the negative
and it secondly posits itself as a negative,
and it is only this posited negative that it sublates;
in sublating the negative,
it both posits and sublates it at the same time.
In this way the exclusive determination is
itself that other of itself of which it is the negation;
the sublation of this positedness is not, therefore,
once more positedness as the negative of an other,
but is self-withdrawal, positive self-unity.
Self-subsistence is thus unity that turns back into itself
by virtue of its own negation,
for it turns into itself through the negation of its positedness.
It is the unity of essence to be identical with itself
through the negation not of an other, but of itself.

3. According to this positive side,
since self-subsistence in opposition,
as excluding reflection,
makes itself into a positedness
and equally sublates this positedness,
not only has opposition foundered
but in foundering it has gone back
to its foundation, to its ground.
The excluding reflection of
the self-subsisting opposition turns it
into a negative, something only posited;
it thereby reduces its formerly self-subsisting determinations,
the positive and the negative,
to determinations which are only determinations;
and the positedness, since it is now made into positedness,
has simply gone back to its unity with itself;
it is simple essence, but essence as ground.
Through the sublating of the determinations of essence,
which are in themselves self-contradictory,
essence is restored,
but restored in the determination of
an exclusive, reflective unity
a simple unity which determines itself as negation,
but in this positedness is immediately like itself
and withdrawn into itself.

In the first place, therefore, because of its contradiction,
the self-subsisting opposition goes back into a ground;
this opposition is what comes first,
the immediate from which the beginning is made,
while the sublated opposition
or the sublated positedness is itself a positedness.
Accordingly, essence is as ground a positedness,
something that has become.
But conversely, only this has been posited,
namely that the opposition or the positedness is
something sublated, only is as positedness.
As ground, therefore, essence is excluding reflection
because it makes itself into a positedness;
because the opposition from which the start
was just now made and was the immediate is
the merely posited determinate self-subsistence of essence;
because opposition only sublates itself within,
whereas essence is in its determinateness reflected into itself.
As ground, therefore, essence excludes
itself from itself, it posits itself;
its positedness which is what is excluded
is only as positedness,
as identity of the negative with itself.
This self-subsistent is the negative
posited as the negative,
something self-contradictory
which, consequently, remains in
the essence as in its ground.

The resolved contradiction is therefore ground,
essence as unity of the positive and the negative.
In opposition, determinateness has progressed to self-subsistence;
but ground is this self-subsistence as completed;
in it, the negative is self-subsistent essence, but as negative;
and, as self-identical in this negativity,
ground is thus equally the positive.
In ground, therefore, opposition and its contradiction
are just as much removed as preserved.
Ground is essence as positive self-identity
which, however, at the same time
refers itself to itself as negativity
and therefore determines itself,
making itself into an excluded positedness;
but this positedness is the whole self-subsisting essence,
and essence is ground, self-identical in its negation and positive.
The self-contradictory self-subsistent opposition
was itself, therefore, already ground;
all that was added to it was the determination of self-unity
which emerges as each of the self-subsisting opposites
sublates itself and makes itself into its other,
thereby founders and sinks to the ground
but therein also reunites itself with itself;
thus in this foundering, that is,
in its positedness or in the negation,
it rather is for the first time the essence
that is reflected into itself and self-identical.

CHAPTER 3

Ground

Essence determines itself as ground.

IV.6
karma-asukla-akrsnam yogina trividham itaresam

Just as nothing is at first
in simple immediate unity with being,
so here too the simple identity of essence is
at first in simple unity with its absolute negativity.
Essence is only this negativity which is pure reflection.
It is this pure reflection as
the turning back of being into itself;
hence it is determined, in itself or for us,
as the ground into which being resolves itself.
But this determinateness is not posited by the essence itself;
in other words, essence is not ground precisely because
it has not itself posited this determinateness that it possesses.
Its reflection, however, consists in positing itself as
what it is in itself, as a negative, and in determining itself.
The positive and the negative constitute the essential determination
in which essence is lost in its negation.
These self-subsisting determinations of reflection sublate themselves,
and the determination that has foundered to the ground is
the true determination of essence.

Consequently, ground is itself one of
the reflected determinations of essence,
but it is the last, or rather,
it is determination determined as sublated determination.
In foundering to the ground, the determination of reflection
receives its true meaning that it is the absolute
repelling of itself within itself;
or again, that the positedness that accrues to essence is
such only as sublated,
and conversely that only the self-sublating positedness is
the positedness of essence.
In determining itself as ground,
essence determines itself as the not-determined,
and only the sublating of its being determined is its determining.
Essence, in thus being determined as self-sublating,
does not proceed from an other but is,
in its negativity, identical with itself.

Since the advance to the ground is made starting
from determination as an immediate first
(is done by virtue of the nature of determination itself
that founders to the ground through itself),
the ground is at first determined by that immediate first.
But this determining is, on the one hand,
as the sublating of the determining,
the merely restored, purified or manifested identity of essence
which the determination of reflection is in itself;
on the other hand, this negating movement is, as determining,
the first positing of that reflective determinateness
that appeared as immediate determinateness,
but which is posited only by the self-excluding reflection of ground
and therein is posited as only something posited or sublated.
Thus essence, in determining itself as ground, proceeds only from itself.
As ground, therefore, it posits itself as essence,
and its determining consists in just this positing of itself as essence.
This positing is the reflection of essence
that sublates itself in its determining;
on that side is a positing, on this side is the positing of essence,
hence both in one act.

Reflection is pure mediation in general;
ground, the real mediation of essence with itself.
The former, the movement of nothing through nothing back to itself,
is the reflective shining of one in an other;
but, because in this reflection opposition does not
yet have any self-subsistence,
neither is the one, that which shines, something positive,
nor is the other in which it reflectively shines something negative.
Both are substrates, actually of the imagination;
they are still not self-referring.
Pure mediation is only pure reference,
without anything being referred to.
Determining reflection, for its part, does posit
such terms as are identical with themselves;
but these are at the same time only determined references.
Ground, on the contrary, is mediation that is real,
since it contains reflection as sublated reflection;
it is essence that turns back into itself
through its non-being and posits itself.
According to this moment of sublated reflection,
what is posited receives the determination of immediacy,
of an immediate which is self-identical
outside its reference or its reflective shining.
This immediacy is being as restored by essence,
the non-being of reflection through which essence mediates itself.
Essence returns into itself as it negates;
therefore, in its turning back into itself,
it gives itself the determinateness that precisely
for this reason is the self-identical negative,
is sublated positedness, and consequently,
as the self-identity of essence as ground,
equally an existent.

The ground is, first, absolute ground
one in which the essence is first of all
the general substrate for the ground-connection.
It then further determines itself as form and matter
and gives itself a content.

Second, it is determinate ground,
the ground of a determinate content.
Because the ground-connection, in being realized,
becomes as such external,
it passes over into conditioning mediation.

Third, ground presupposes a condition;
but the condition equally presupposes the ground;
the unconditioned is the unity of the two,
the fact itself that, by virtue of
the mediation of the conditioning reference,
passes over into concrete existence.

IV.7
tatas tad-vipaka-anugunanam eva-abhivyakti vasananam

A. ABSOLUTE GROUND

a. Form and essence

The determination of reflection,
inasmuch as this determination returns into ground,
is a first immediate existence in general
from which the beginning is made.
But existence still has only the meaning of positedness
and essentially presupposes a ground,
in the sense that it does not really posit a ground;
that the positing is a sublating of itself;
that it is rather the immediate that is posited,
and the ground the non-posited.
As we have seen, this presupposing is the positing
that rebounds on that which posits;
as sublated determinate being, the ground is not an indeterminate
but is rather essence determined through itself,
but determined as indeterminate or as sublated positedness.
It is essence that in its negativity is identical with itself.

The determinateness of essence as ground is thus twofold:
it is the determinateness of the ground and of the grounded.
It is, first, essence as ground,
essence determined to be essence
as against positedness, as non-positedness.
Second, it is that which is grounded,
the immediate that, however, is not anything in and for itself:
is positedness as positedness.
Consequently, this positedness is equally identical with itself,
but in an identity which is that of the negative with itself.
The self-identical negative and the self-identical positive
are now one and the same identity.
For the ground is the self-identity
of the positive or even also of positedness;
the grounded is positedness as positedness,
but this its reflection-into-itself is the identity of the ground.
This simple identity, therefore, is not itself ground,
for the ground is essence posited as
the non-posited as against positedness.
As the unity of this determinate identity (the ground)
and of the negative identity (the grounded),
it is essence in general distinct from its mediation.

For one thing, this mediation,
compared with the preceding reflections
from which it derives,
is not pure reflection,
which is undistinguished from essence
and still does not have the negative in it,
consequently also does not as yet contain
the self-subsistence of the determinations.
These have their subsistence, rather,
in the ground understood as sublated reflection.
And it is also not the determining reflection
whose determinations have essential self-subsistence,
for that reflection has foundered, has sunk to the ground,
and in the unity of the latter
the determinations are only posited determinations.
This mediation of the ground is thus
the unity of pure reflection and determining reflection;
their determinations or that which is posited has self-subsistence,
and conversely the self-subsistence of
the determinations is a posited subsistence.
Since this subsistence of the determinations is
itself posited or has determinateness,
the determinations are consequently distinguished
from their simple identity,
and they constitute the form as against essence.

Essence has a form and determinations of this form.
Only as ground does it have a fixed immediacy or is substrate.
Essence as such is one with its reflection,
inseparable from its movement.
It is not essence, therefore, through which
this movement runs its reflective course;
nor is essence that from which the movement begins,
as from a starting point.
It is this circumstance that above all makes
the exposition of reflection especially difficult,
for strictly speaking one cannot say
that essence returns into itself,
that essence shines in itself,
for essence is neither before its movement nor in the movement:
this movement has no substrate on which it runs its course.
A term of reference arises in the ground only following upon the
moment of sublated reflection.
But essence as the referred-to term is determinate essence,
and by virtue of this positedness it has form as essence.
The determinations of form, on the contrary,
are now determinations in the essence;
the latter lies at their foundation
as an indeterminate which in its determination
is indifferent to them;
in it, they are reflected into themselves.
The determinations of reflection should have
their subsistence in them and be self-subsistent.
But their self-subsistence is their dissolution,
which they thus have in an other;
but this dissolution is itself this self-identity
or the ground of the subsistence that they give to themselves.

Everything determinate belongs in general to form;
it is a form determination inasmuch as it is something posited
and hence distinguished from that of which it is the form.
As quality, determinateness is one with its substrate, being;
being is the immediate determinate,
not yet distinct from its determinateness
or, in this determinateness,
still unreflected into itself,
just as the determinateness is, therefore,
an existent determinateness,
not yet one that is posited.
Moreover, the form determinations of essence are,
in their more specific determinateness,
the previously considered moments of reflections:
identity and difference, the latter as both
diversity and opposition.
But also the ground-connection belongs among
these form determinations of essence,
because through it, though itself the
sublated determination of reflection,
essence is at the same time as posited.
By contrast, the identity that has the ground immanent in it
does not pertain to form, because positedness,
as sublated and as such (as ground and grounded),
is one reflection, and this reflection constitutes
essence as simple substrate which is the subsistence of form.
But in ground this subsistence is posited,
or this essence is itself essentially as determinate and, consequently,
is in turn also the moment of the ground-connection and form.
This is the absolute reciprocal connecting reference of form and essence:
essence is the simple unity of ground and grounded
but, in this unity, is itself determined, or is a negative,
and it distinguishes itself as substrate from form,
but at the same time it thereby becomes itself
ground and moment of form.

Form is therefore the completed whole of reflection;
it also contains this determination of reflection, that it is sublated;
just like reflection, therefore, it is one unity of its determining,
and it is also referred to its sublatedness,
to another that is not itself form but in which the form is.
As essential self-referring negativity,
in contrast with that simple negative,
form is positing and determining;
simple essence, on the contrary, is indeterminate and inert
substrate in which the determinations of form have their subsistence
or their reflection into themselves.
External reflection normally halts at
this distinction of essence and form;
the distinction is necessary,
but the distinguishing itself of the two is their unity,
just as this unity of ground is essence repelling itself from itself
and making itself into positedness.
Form is absolute negativity itself
or the negative absolute self-identity
by virtue of which essence is indeed not being but essence.
This identity, taken abstractly, is essence as against form,
just as negativity, taken abstractly as positedness,
is the one determination of form.
But this determination has shown itself to be in truth
the whole self-referring negativity
which within, as this identity, thus is simple essence.
Consequently, form has essence in its own identity,
just as essence has absolute form in its negative nature.
One cannot therefore ask, how form comes to essence,
for form is only the internal reflective shining of essence,
its own reflection inhabiting it.
Form equally is, within it,
the reflection turning back into itself
or the identical essence;
in its determining, form makes the determination
into positedness as positedness.

Form, therefore, does not determine essence,
as if it were truly presupposed, separate from essence,
for it would then be the unessential,
constantly foundering determination of reflection;
here it rather is itself the ground of its sublating
or the identical reference of its determinations.
That the form determines the essence means, therefore,
that in its distinguishing form sublates this very distinguishing
and is the self-identity that essence is
as the subsistence of the determinations;
form is the contradiction of being sublated in its positedness
and yet having subsistence in this sublatedness;
it is accordingly ground as essence which
is self-identical in being determined or negated.

These distinctions, of form and of essence,
are therefore only moments of the simple reference of form itself.
But they must be examined and fixed more closely.
Determining form refers itself to itself as sublated positedness;
it thereby refers itself to its identity as to another.
It posits itself as sublated;
it therefore presupposes its identity;
according to this moment, essence is the indeterminate
to which form is an other.
It is not the essence which is absolute reflection within,
but essence determined as formless identity:
it is matter.

b. Form and matter

Essence becomes matter in that its reflection is
determined as relating itself
to essence as to the formless indeterminate.
Matter, therefore, is the simple identity,
void of distinction, that essence is,
with the determination that it is the other of form.
Hence it is the proper base or substrate of form,
since it constitutes the immanent reflection
of the determinations of form,
or the self-subsistent term,
to which such determinations refer
as to their positive subsistence.

If abstraction is made from every determination,
from every form of a something, matter is what is left over.
Matter is the absolutely abstract.
(One cannot see, feel, etc. matter;
what one sees or feels is a determinate matter,
that is, a unity of matter and form.)
This abstraction from which matter derives is not, however,
an external removal and sublation of form;
it is rather the form itself which, as we have just seen,
reduces itself by virtue of itself to this simple identity.

Further, form presupposes a matter to which it refers.
But for this reason the two do not find themselves
confronting each other externally and accidentally;
neither matter nor form derives from itself, is a se,
or, in other words, is eternal.
Matter is indifferent with respect to form,
but this indifference is the determinateness
of self-identity to which
form returns as to its substrate.
Form presupposes matter for the very reason
that it posits itself as a sublated,
hence refers to this, its identity,
as to something other.
Contrariwise, form is presupposed by matter;
for matter is not simple essence,
which immediately is itself absolute reflection,
but is essence determined as something positive,
that is to say, which only is as sublated negation.
But, on the other hand, since form posits itself
as matter only in sublating itself,
hence in presupposing matter,
matter is also determined as groundless subsistence.
Equally so, matter is not determined as the ground of form;
but rather, inasmuch as matter posits itself as
the abstract identity of the sublated determination of form,
it is not that identity as ground,
and form is therefore groundless with respect to it.
Form and matter are consequently alike determined as
not to be posited each by the other,
each not to be the ground of the other.
Matter is rather the identity of
the ground and the grounded,
as the substrate that stands over
against this reference of form.
This determination of indifference that
the two have in common is
the determination of matter as such
and also constitutes their reciprocal reference.
The determination of form, that it is
the connection of the two as distinct,
equally is also the other moment of
the relating of the two to each other.
Matter, determined as indifferent,
is the passive as contrasted to form,
which is determined as the active.
This latter, as self-referring negative,
is inherently contradiction, self-dissolving,
self-repelling, and self-determining.
It refers to matter, and it is posited to refer to this matter,
which is its subsistence, as to another.
Matter is posited, on the contrary,
as referring only to itself
and as indifferent to the other;
but, implicitly, it does refer to the form,
for it contains the sublated negativity
and is matter only by virtue of this determination.
It refers to it as an other only
because form is not posited in it,
because it is form only implicitly.
It contains form locked up inside it,
and it is an absolute receptivity for form
only because it has the latter within it absolutely,
because to be form is its implicit vocation.
Hence matter must be informed,
and form must materialize itself;
it must give itself self-identity
or subsistence in matter.

2. Consequently, form determines matter,
and matter is determined by form.
Because form is itself absolute self-identity
and hence implicitly contains matter;
and equally because matter in its pure abstraction
or absolute negativity possesses form within it,
the activity of the form on the matter
and the reception by the latter of the form determination is only
the sublating of the semblance of their indifference and distinctness.
Thus the determination referring each to the other is
the self-mediation of each through its own non-being.
But the two mediations are one movement,
and the restoration of their original identity is
the inner recollection of their exteriorization.

First, form and matter presuppose each other.
As we have seen, this only means that the one essential unity is
negative self-reference, and that it therefore splits,
determined as an indifferent substrate in the essential identity,
and as determining form in essential distinction or negativity.
That unity of essence and form, the two opposed to each other as
form and matter, is the absolute self-determining ground.
Inasmuch as this unity differentiates itself,
the reference connecting the two diverse terms,
because of the unity that underlies them,
becomes a reference of reciprocal presupposition.

Second, the form already is, as self-subsisting,
self-sublating contradiction;
but it is also posited as in this way self-sublating,
for it is self-subsisting and at the same time
essentially referred to another,
and consequently it sublates itself.
Since it is itself two-sided, its sublating also has two sides.
For one, form sublates its self-subsistence
and transforms itself into something posited,
something that exists in an other,
and this other is in its case matter.
For the other, form sublates its determinateness vis-à-vis matter,
sublates its reference to it, consequently its positedness,
and it thereby gives itself subsistence.
Its reflection in thus sublating its positedness is
its own identity into which it passes over.
But since form at the same time externalizes this identity
and posits it over against itself as matter,
that reflection of the positedness into itself is
a union with a matter in which it obtains subsistence.
In this union, therefore, it is equally both:
is united with matter as with something other
(in accordance with the first side, viz. in that it makes
itself into a positedness),
and, in this other, is united with its own identity.

The activity of form by which matter is determined consists,
therefore, in a negative relating of the form to itself.
But, conversely, form thereby negatively relates itself to matter also;
the movement, however, by which matter becomes determined is
just as much the form's own movement.
Form is free of matter, but it sublates its self-subsistence;
but this, its self-subsistence, is matter itself,
for it is in this matter that it has its essential identity.
It makes itself into a positedness, but this is one and the same
as making matter into something determinate.
But, considered from the other side,
the form's own identity is at the same time externalized,
and matter is its other;
for this reason, because form sublates its own self-subsistence,
matter is also not determined.
But matter only subsists vis-à-vis form;
as the negative sublates itself, so does the positive also.
And as the form sublates itself, the determinateness of matter
that the latter has vis-à-vis form also falls away
the determinateness, namely, of being the indeterminate subsistence.

What appears here as the activity of form is, moreover,
just as much the movement that belongs to matter itself.
The determination that implicitly exists in matter,
what matter is supposed to be, is its absolute negativity.
Through it matter does not just refer to form simply as to an other,
but this external other is the form rather that
matter itself contains locked up within itself.
Matter is in itself the same contradiction that form contains,
and this contradiction, like its resolution, is only one.
But matter is thus in itself self-contradictory because,
as indeterminate self-identity,
it is at the same time absolute negativity;
it sublates itself within:
its identity disintegrates in its negativity
while the latter obtains in it its subsistence.
Since matter is therefore determined by form as by something external,
it thereby attains its determination,
and the externality of the relating, for both form and matter,
consists in that each, or rather in that the original unity of each,
in positing is at the same time presupposing:
the result is that self-reference is at the same time
a reference to the self as sublated or is reference to its other.

Third, through this movement of form and matter,
the original unity of the two is, on the one hand, restored;
on the other hand, it is henceforth a posited unity.
Matter is just as much a self-determining as
this determining is for it an activity of form external to it;
contrariwise, form determines only itself,
or has the matter that it determines within it,
just much as in its determining it relates itself to another;
and both, the activity of form and the movement of matter,
are one and the same thing, only that the former is an activity,
that is, it is the negativity as posited,
while the latter is movement or becoming,
the negativity as determination existing in itself.
The result, therefore, is the unity of the in-itself and positedness.
Matter is as such determined or necessarily has a form,
and form is simply material, subsistent form.

Inasmuch as form presupposes a matter as its other, it is finite.
It is not a ground but only the active factor.
Equally so, matter, inasmuch as it presupposes
form as its non-being, is finite matter;
it is not the ground of its unity with form
but is for the latter only the substrate.
But neither this finite matter nor the finite form have any truth;
each refers to the other, or only their unity is their truth.
The two determinations return to
this unity and there they sublate their self-subsistence;
the unity thereby proves to be their ground.
Consequently, matter is the ground
of its form determination not as matter
but only inasmuch as it is the absolute unity of essence and form;
similarly, form is the ground of the subsistence of its determinations
only to the extent that it is that same one unity.
But this one unity, as absolute negativity,
and more specifically as exclusive unity,
is, in its reflection, a presupposing;
or again, that unity is one act,
of preserving itself as positedness in positing,
and of repelling itself from itself;
of referring itself to itself as itself
and to itself as to another.
Or, the act by which matter is determined by form
is the self-mediation of essence as ground, in one unity:
through itself and through the negation of itself.

Informed matter or form that possesses subsistence is now,
not only this absolute unity of ground with itself,
but also unity as posited.
The movement just considered is the one
in which the absolute ground has exhibited
its moments at once as self-sublating
and consequently as posited.
Or the restored unity, in withdrawing into itself,
has repelled itself from itself and
has determined itself;
for its unity has been established through negation
and is, therefore, also negative unity.
It is, therefore, the unity of form and matter,
as the substrate of both, but a substrate which is determinate:
it is formed matter, but matter at the same time
indifferent to form and matter,
indifferent to them because sublated and unessential.
This is content.

c. Form and content

Form stands at first over against essence;
it is then the ground-connection in general,
and its determinations are the ground and the grounded.
It then stands over against matter,
and so it is determining reflection,
and its determinations are the determination of reflection itself
and the subsistence of the latter.
Finally, it stands over against content,
and then its determinations are again itself and matter.
What was previously the self-identical
at first the ground,
then subsistence in general,
and finally matter
now passes under the dominion of form
and is once more one of its determinations.

Content has, first, a form and a matter that
belong to it essentially; it is their unity.
But, because this unity is at the same time determinate
or posited unity, content stands over against form;
the latter constitutes the positedness
and is the unessential over against content.
The latter is therefore indifferent towards form;
form embraces both the form as such as well as the matter,
and content therefore has a form and a matter,
of which it constitutes the substrate
and which are to it mere positedness.

Content is, second, what is identical in form and matter,
so that these would be only indifferent external determinations.
They are positedness in general, but a positedness
that has returned in the content to its unity or its ground.
The identity of the content with itself is,
therefore, in one respect that identity which is indifferent to form,
but in another the identity of ground.
The ground has at first disappeared into content;
but content is at the same time the negative reflection of
the form determinations into themselves;
its unity, at first only the unity indifferent to form, is
therefore also the formal unity or the ground-connection as such.
Content, therefore, has this ground-connection as its essential form,
and, contrariwise, the ground has a content.

The content of the ground is therefore the ground
that has returned into its unity with itself;
the ground is at first the essence that in its
positedness is identical with itself;
as diverse from and indifferent to its positedness,
the ground is indeterminate matter;
but as content it is at the same time informed identity,
and this form becomes for this reason a ground-connection,
since the determinations of its oppositions are posited
in the content also as negated.
Content is further determined within,
not like matter as an indifferent in general,
but like informed matter,
so that the determinations of form have
a material, indifferent subsistence.
On the one hand, content is the essential self-identity
of the ground in its positedness;
on the other hand, it is posited identity
as against the ground-connection;
this positedness, which is in this identity as determination of form,
stands over against the free positedness, that is to say,
over against the form as the whole connection of ground and grounded.
This form is the total positedness returning into itself;
the other form, therefore, is only the positedness as immediate,
the determinateness as such.

The ground has thus made itself into a determinate ground in general,
and the determinateness is itself twofold:
of form first, and of content second.
The former is its determinateness of being external to the content as such,
the content that remains indifferent to this external reference.
The latter is the determinateness of the content that the ground has.

IV.8
jati-desa-kala vyavahitanam api-anantaryam smrti-samskarayo eka-rupatvat

B. DETERMINATE GROUND

a. Formal ground

The ground has a determinate content.
For the form, as we have seen,
the determinateness of content is the substrate,
the simple immediate as against the mediation of form.
The ground is negatively self-referring identity which,
for this reason, makes itself into a positedness;
it negatively refers to itself because in its negativity
it is identical with itself;
this identity is the substrate or the content
which thus constitutes the indifferent
or positive unity of the ground-connection
and, in this connection, is the mediating factor.

In this content, the determinateness that
the ground and the grounded have over
against one another has at first disappeared.
The mediation, however, is also negative unity.
The negative implicit in that indifferent substrate is
this substrate's immediate determinateness through which
the ground has a determinate content.
But then, the negative is the negative reference of form to itself.
What has been posited sublates itself on its side
and returns to its ground;
the ground, however, the essential self-subsistence,
refers negatively to itself and makes itself into a positedness.
This negative mediation of ground and grounded is
the mediation that belongs to form as such, formal mediation.
Now both sides of form, because each passes over into the other,
thereby mutually posit themselves into one identity as sublated;
in this, they presuppose the identity.
The latter is the determinate content
to which the formal mediation thus refers itself
through itself as to the positive mediating factor.
That content is the identical element of both,
and because the two are distinct,
yet in their distinction each is
the reference to the other,
it is their subsistence,
the subsistence of each as the whole itself.

Accordingly, the result is that
in the determinate ground we have the following.
First, a determinate content is considered from two sides,
once in so far as it is ground,
then again in so far as it is grounded.
The content itself is indifferent to these forms;
it is in each simply and solely one determination.
Second, the ground is itself just as much a moment of form
as what is posited by it; this is its identity according to form.
It is a matter of indifference which of
the two determinations is made the first,
whether the transition is
from the one as posited to the other as ground
or from the one as ground to the other as posited.
The grounded, considered for itself, is the sublating of itself;
it thereby makes itself on the one side into a posited,
and is at the same time the positing of the ground.
The same movement is the ground as such;
it makes itself into something posited,
and thereby becomes the ground of something,
that is to say, is present therein
both as a posited and also first as ground.
That there be a ground, of that the posited is the ground,
and, conversely, the ground is thereby the posited.
The mediation begins just as much from the one as from the other;
each side is just as much ground as posited,
and each is the whole mediation or the whole form.
Further, this whole form is itself, as self-identical,
the substrate of the two determinations
that constitute the two sides of the ground and the grounded;
form and content are thus themselves one and the same identity.

Because of this identity of the ground and the grounded,
according both to content and form,
the ground is sufficient
(the sufficiency being limited to this relation);
there is nothing in the grounded which is not in the ground.
Whenever one asks for a ground,
one expects to see the same determination
which is the content doubled,
once in the form of that which is posited,
and again in the form of existence
reflected into itself, of essentiality.

Now inasmuch as in the determined ground,
the ground and the grounded are each the whole form, and their content,
though determinate, is nevertheless one and the same,
the two sides of the ground do not as yet have a real determination,
do not have a different content;
the determinateness is only one simple determinateness
that has yet to pass over into the two sides;
the determinate ground is present
only in its pure form, as formal ground.
Because the content is only this simple determinateness,
one that does not have in it the form of the ground-connection,
the determinateness is a self-identical content indifferent to form,
and the form is external to it;
the content is other than the form.

b. Real ground

The determinateness of ground is, as we have seen,
on the one hand determinateness of the substrate
or content determination;
on the other hand,
it is the otherness in the ground-connection itself,
namely the distinctness of its content and the form;
the connection of ground and grounded strays
in the content as an external form,
and the content is indifferent to these determinations.
But in fact the two are not external to each other;
for this is what the content is:
to be the identity of the ground
with itself in the grounded,
and of the grounded in the ground.
The side of the ground
has shown itself to be itself a posited,
and the side of the grounded to be
itself ground;
each side is this identity of the whole within it.
But since they equally belong to form
and constitute its determinate difference,
each is in its determinateness the identity of the whole with itself.
Consequently, each has a diverse content as against the other.
Or, considering the matter from the side of the content,
since the latter is the self-identity of the ground-connection,
it essentially possesses this difference of form within,
and is as ground something other than what it is as grounded.

Now the moment ground and grounded have a diverse content,
the ground-connection has ceased to be a formal one;
the turning back to the ground and
the procession forward from ground to posited
is no longer a tautology; the ground is realized.
Henceforth, whenever we ask for a ground,
we actually demand another content determination for it
than the determination of the content whose ground we are asking for.

This connection now determines itself further.
For inasmuch as its two sides are of different content,
they are indifferent to each other;
each is an immediate, self-identical determination.
Moreover, as referred to each other as ground and grounded,
the ground reflects itself in the other,
as in something posited by it, back to itself;
the content on the side of the ground,
therefore, is equally in the grounded;
the latter, as the posited, has its
self-identity and subsistence only in the ground.
But besides this content of the ground,
the grounded also now possesses a content of its own
and is accordingly the unity of a twofold content.
Now this unity, as the unity of sides that are different,
is indeed their negative unity;
but since the two determinations of content are indifferent to each other,
that unity is only their empty reference to each other,
in itself void of content, and not their mediation;
it is a one or a something externally holding them together.

In the real grounding connection
there is present, therefore, a twofold.
For one thing, the content determination which is ground
extends continuously into the positedness,
so that it constitutes the simple identity
of the ground and the grounded;
the grounded thus contains the ground
fully within itself;
their connection is one of
undifferentiated essential compactness.
Anything else in the grounded
added to this simple essence is,
therefore, only an unessential form,
external determinations of the content
that, as such, are free from the ground
and constitute an immediate manifold.
Of this unessential more, therefore,
the essential is not the ground,
nor is it the ground of any connection
between it and the unessential in the grounded.
The unessential is a positively identical element
that resides in the grounded but does not posit itself
there in any distinctive form;
as self-referring content, it is rather
an indifferent positive substrate.
For another thing, that which in the something is
linked with this substrate is an indifferent content,
but as the unessential side.
The main thing is the connection of the substrate
and the unessential manifold.
But this connection, since the determinations
that it connects are an indifferent content,
is also not a ground;
true, one determination is determined as essential content
and the other as only unessential or as posited;
but this form is to each, as a self-referring content, an external one.
The one of the something that constitutes their connection is
for this reason not a reference of form,
but only an external tie that does not hold
the unessential manifold content as posited;
it too is therefore likewise only a substrate.

Ground, in determining itself as real,
because of the diversity of the content
that constitutes its reality,
thus breaks down into external determinations.
The two connections of the essential reality content,
as the simple immediate identity of ground and grounded;
and then the something connecting distinct contents
are two different substrates.
The self-identical form of ground,
according to which one and the same thing
is at one time the essential
and at another the posited, has vanished.
The ground-connection has thus become external to itself.

Consequently, it is an external ground that now
holds together a diversified content
and determines what is ground and what is posited by it;
this determination is not to be found in the two-sided content itself.
The real ground is therefore the reference to another,
on the one hand, of a content to another content
and, on the other, of the ground-connection itself
(the form) to another, namely to an immediate,
to something not posited by it.

c. Complete ground

1. In real ground, ground as content
and ground as connection are only substrates.
The former is only posited as essential and as ground;
the connection is what the grounded immediately is
as the indeterminate substrate of a diversified content,
a linking of this content which is not the content's own reflection
but is rather external and consequently a reflection which is only posited.
The real ground-connection is ground, therefore, rather as sublated;
consequently, it rather makes up the side of
the grounded or of the positedness.
As positedness, however, the ground itself
has now returned to its ground;
it is now something grounded: it has another ground.
This ground will therefore be so determined that,
first, it is identical with the ground by which it is grounded;
both sides have in this determination one and the same content;
the two content determinations and their linkage in
a something are equally to be found in the new ground.
But, second, the new ground into which
the previously merely posited and external link
is now sublated is the immanent reflection of this link:
the absolute reference of the two content determinations to each other.

Because real ground has itself thus returned to its ground,
the identity of ground and grounded
or the formality of ground reasserts itself in it.
The newly arisen ground-connection is
therefore the one which is complete,
which contains the formal and real ground in itself
at the same time and mediates the content determinations
which in the real ground confronted each other immediately.

2. Thus the ground-connection has more precisely
determined itself as follows.

First, something has a ground;
it contains the content determination which is the ground
and, in addition, a second determination as posited by the ground.
But, because of the indifference of content,
the one determination is not ground in itself,
nor is the other in itself one that is grounded by the first;
this connection of ground and grounded is rather
sublated in the immediacy of their content, is posited,
and as such has its ground in another such connection.

Since this second connection is
distinguished only according to form,
it has the same content as the first;
it still has the same two determinations of content
but is now their immediate linking together.
This linking, however, is of a general nature,
and the content, therefore, is diversified into determinations
that are indifferent to each other.
The linking is not, therefore, their true absolute connection
that would make one determination the element of
self-identity in the positedness,
and the other determination
the positedness of this same self-identity;
on the contrary, the two are supported by a something
and this something is what connects them,
but in a connection which is not reflected,
is rather only immediate and, therefore,
only a relative ground as against
the linking in the other something.
The two somethings are therefore the two distinct
connections of content that have transpired.
They stand in the identical ground-connection of form;
they are one and the same whole content,
namely the two content determinations and their connection;
they are distinct only by the kind of this connection,
which in the one is an immediate
and in the other a posited connection;
through this, they are distinguished
one from another as ground and grounded only according to form.
Second, this ground-connection is not only formal, but also real.
Formal ground passes over into real ground, as has been shown;
the moments of the form reflect themselves into themselves;
they are a self-subsistent content,
and the ground-connection contains
also one content with the character of ground
and another with that of grounded.
The content constitutes at first the immediate
identity of both sides of the formal ground;
so the two sides have one and the same content.
But the content also has the form in it,
and so it is a twofold content
that behaves as ground and grounded.
One of the two content determinations of
the two somethings is therefore determined,
not merely as being common to them
according to external comparison,
but as their identical substrate
and the foundation of their connection.
As against the other determination of the content,
this determination is essential
and is the ground of the other which is posited,
that is, posited in the something,
the connection of which is the grounded.
In the first something, which is the ground-connection,
this second determination of the content is
also immediately and in itself linked with the first.
But the other something only contains
the one determination in itself as that
in which it is immediately identical with the first something,
but the other as the one which is posited in it.
The former content determination is its
ground by virtue of its being originally linked
in the first something with
the other content determination.

The ground-connection of the content determinations
in the second something is thus mediated
through the connection present in the first something.
The inference is this:
since determination B is implicitly linked
with determination A in a something,
in a second something to which only
the one determination A immediately belongs,
also B is linked with it.
In the second something, not only is
this second determination mediated;
also mediated is that its immediate ground is mediated,
namely by virtue of its original connection
with B in the first something.
This connection is thus the ground of the ground A,
and the whole ground-connection is present in
the second something as posited or grounded.

3. Real ground shows itself to be the self-external reflection of ground;
its complete mediation is the restoration of its identity with itself.
But because this identity has in the process equally acquired
the externality of real ground,
the formal ground-connection in this unity
of itself and real ground is just as much
self-positing as self-sublating ground;
the ground-connection mediates itself with itself through its negation.
The ground is at first, as the original connection,
the connection of immediate content determinations.
The ground-connection, being essential form,
has for sides such that are sublated or are as moments.
Consequently, as the form of immediate determinations,
it connects itself with itself as self-identical
while at the same time connecting with their negation;
accordingly, it is ground not in and for itself
but as connected with the sublated ground-connection.
Second, the sublated connection or the immediate,
which in the original and in the posited connection
is the identical substrate, is likewise real ground
not in and for itself; that it is ground is
rather posited by virtue of that original link.

Thus the ground-connection is in its totality
essentially presupposing reflection;
formal ground presupposes the immediate content determination,
and this content presupposes form as real ground.
Ground is therefore form as an immediate linkage
but in such a manner that it repels itself from itself
and rather presupposes immediacy,
referring itself therein as to another.
This immediate is the content determination, the simple ground;
but as such, that is, as ground, it is equally repelled from itself
and refers itself to itself equally as to an other.
Thus the total ground-connection has taken on
the determination of conditioning mediation.

IV.9
tasam anaditvam ca-asisa nityatvat

C. CONDITION

a. The relatively unconditioned

1. Ground is the immediate,
and the grounded the mediated.
But ground is positing reflection;
as such, it makes itself into positedness
and is presupposing reflection;
as such it refers itself to itself
as to something sublated,
to an immediate through which
it is itself mediated.
This mediation, as an advance
from the immediate to the ground,
is not an external reflection
but, as we have seen, the ground's own doing
or, what is the same, the ground-connection,
as reflection into its self-identity,
is just as essentially self-externalizing reflection.
The immediate to which ground refers as
to its essential presupposition is condition;
real ground is accordingly essentially conditioned.
The determinateness that it contains is
the otherness of itself.
Condition is therefore,
first, an immediate, manifold existence.
Second, it is this existence referred to an other,
to something which is ground,
not of this existence but in some other respect,
for existence itself is immediate and without ground.
According to this reference, it is something posited;
as condition, the immediate existence is supposed to be
not for itself but for another.
But this, that it thus is for another, is at the same time
itself only a positedness;
that it is posited is sublated in its immediacy:
an existence is indifferent to being a condition.
Third, condition is something immediate in the sense
that it constitutes the presupposition of ground.
In this determination, it is the form-connection of ground
withdrawn into self-identity, hence the content of ground.
But content is as such only the indifferent unity of ground,
as in the form: without form, no content.
It nevertheless frees itself
from this indifferent unity
in that the ground-connection,
in the complete ground,
becomes a connection external to its identity,
whereby content acquires immediacy.
In so far, therefore, as condition is
that in which the ground-connection has
its identity with itself,
it constitutes the content of ground;
but since this content is indifferent to form,
it is only implicitly the content of form,
is something which has yet to become content
and hence constitutes the material for the ground.
Posited as condition,
and in accordance with the second moment,
existence is determined to lose its indifferent immediacy
and to become the moment of another.
By virtue of its immediacy, it is indifferent to this connection;
inasmuch as it enters into it, however,
it constitutes the in-itself of the ground
and is for it the unconditioned.
In order to be condition,
it has its presupposition in the ground
and is itself conditioned;
but this condition is external to it.

2. Something is not through its condition;
its condition is not its ground.
Condition is for the ground
the moment of unconditioned immediacy,
but is not itself the movement and the positing
that refers itself to itself negatively
and that makes itself into a positedness.
Over against condition there stands,
therefore, the ground-connection.
Something has, besides its condition, also a ground.
This ground is the empty movement of reflection,
for the latter has the immediacy
which is its presupposition outside it.
But it is the whole form
and the self-subsistent process of mediation,
for the condition is not its ground.
Since this mediating refers itself to itself as positing,
it equally is according to this side
something immediate and unconditioned;
it does indeed presuppose itself,
but as an externalized or sublated positing;
whatever it is in accordance with its determination,
that it is, on the contrary, in and for itself.
Inasmuch as the ground-connection is
thus a self-subsisting self-reference
and has within it the identity of reflection,
it has a content which is peculiarly its own
as against the content of the condition.
The one content is that of the ground
and is therefore essentially informed;
the other content, that of the condition,
is on the contrary only an immediate material
whose connecting reference to the ground,
while at the same time constituting
the in-itself of the latter,
is also equally external to it;
it is thus a mingling of a self-subsisting content
that has no reference to the content of the ground determination
and of the content that enters into the latter
and, as its material,
should become a moment of it.

3. The two sides of the whole,
condition and ground,
are thus, on the one hand,
indifferent and unconditioned
with respect to each other:
the one as the non-referred-to side,
to which the connecting reference
in which it is the condition is external;
the other as the connecting reference, or form,
for which the determinate existence of
the condition is only a material,
something passive whose form,
such as it possesses on its own account,
is unessential.
On the other hand, the two sides are also mediated.
Condition is the in-itself of the ground;
so much is it the essential moment of the ground-connection,
that it is the simple self-identity of the ground.
But this also is sublated;
this in-itself is only something posited;
immediate existence is indifferent to being a condition.
The fact, therefore, that condition is the in-itself
of the ground constitutes the side of it
by which it is a mediated condition.
Likewise, the ground-connection has
in its self-subsistence also a presupposition;
it has its in-itself outside itself.
Consequently, each of the two sides is this contradiction,
that they are indifferent immediacy and essential mediation,
both in one reference
or the contradiction of independent subsistence
and of being determined as only moments.

b. The absolutely unconditioned

At first, each of the two relatively unconditioned sides
reflectively shines in the other;
condition, as an immediate, is reflected
in the form connection of the ground,
and this form in the immediate existence as its positedness;
but each, apart from this reflective shine of its other in it,
stands out on its own and has a content of its own.

Condition is at first immediate existence;
its form has these two moments:
that of positedness, according to which it is, as condition,
material and moment of the ground;
and that of the in-itself, according to which
it constitutes the essentiality of ground
or its simple reflection into itself.
Both sides of the form are external to immediate existence,
for the latter is the sublated ground-connection.

But, first, existence is in it only this:
to sublate itself in its immediacy
and to founder, going to the ground.
Being is as such only the becoming of essence;
it is its essential nature to
make itself into a positedness
and into an identity which is
an immediacy through the negation of itself.
The form determinations of positedness
and of self-identical in-itself,
the form through which immediate existence is condition,
are not, therefore, external to that existence;
the latter is, rather, this very reflection.

Second, as condition, being is now posited as
that which it essentially is,
namely as a moment and consequently as the being of an other,
and at the same time as the in-itself of an other;
it is in itself but only through the negation of itself,
namely through the ground and through its self-sublating
and consequent presupposing reflection;
the in-itself of being is thus only something posited.
This in-itself of the condition has two sides:
one side is its essentiality as essentiality of the ground,
while the other is the immediacy of its existence.
Or rather, both sides are the same thing.
Existence is an immediate, but immediacy
is essentially something mediated,
namely through the self-sublating ground.
Existence, as this immediacy mediated by a self-sublating mediating,
is at the same time the in-itself of the ground and its unconditioned side;
but again, this in-itself is at the same time itself
equally only moment or positedness, since it is mediated.
Condition is, therefore, the whole form of the ground-connection;
it is the presupposed in-itself of the latter,
but, consequently, is itself a positedness
and its immediacy is this, to make itself into a positedness
and thereby to repel itself from itself,
in such as way that it both founders to the ground and is ground,
the ground that makes itself into a positedness
and thereby into a grounded, and both are one and the same.

Likewise in the conditioned ground, the in-itself is not
just as the reflective shining of an other in it.
This ground is the self-subsistent,
that is, self-referring reflection of the positing,
and consequently the self-identical;
or it is in it its in-itself and its content.
But it is at the same time presupposing reflection;
it negatively refers to itself
and posits its in-itself as an other opposite to it,
and condition, according to both its moment of in-itself
and of immediate existence, is
the ground-connection's own moment;
the immediate existence essentially is only through its ground
and is a moment of itself as a presupposing.
This ground, therefore, is equally the whole itself.

What we have here, therefore, is only one whole of form,
but equally so only one whole of content.
For the proper content of condition is essential content
only in so far as it is the self-identity of reflection in the form,
or the ground-connection is in it this immediate existence.
Further, this existence is condition only through
the presupposing reflection of the ground;
it is the ground's self-identity, or its content,
to which the ground posits itself as opposite.
Therefore, the existence is not a merely
formless material for the ground-connection;
on the contrary, because it has this form in it, it is informed matter,
and because in its identity with it it is at the same time
indifferent to it, it is content.
Finally, it is the same content as that possessed by the ground,
for it is precisely content as that which is self-identical
in the form connection.

The two sides of the whole,
condition and ground,
are therefore one essential unity,
as content as well as form.
They pass into one another,
or, since they are reflections,
they posit themselves as sublated,
refer themselves to this their negation,
and reciprocally presuppose each other.
But this is at the same time only one reflection of the two,
and their presupposing is, therefore, one presupposing only;
the reciprocity of this presupposing ultimately amounts to this,
that they both presuppose one identity
for their subsistence and their substrate.
This substrate, the one content and unity of form of both,
is the truly unconditioned; the fact in itself.
Condition is, as it was shown above, only the relatively unconditioned.
It is usual, therefore, to consider it as itself something conditioned
and to ask for a new condition,
whereby the customary progression ad infinitum
from condition to condition is set in motion.
But now, why is it that at one condition
a new condition is asked for, that is,
why is that condition assumed to be something conditioned?
Because it is some finite determinate existence or other.
But this is a further determination of condition
that does not enter into its concept.
Condition is as such conditioned solely because
it is the posited in-itselfness;
it is, therefore, sublated in the absolutely unconditioned.

Now this contains within itself the two sides,
condition and ground, as its moments;
it is the unity to which they have returned.
Together, the two constitute its form or its positedness.
The unconditioned fact is the condition of both,
but the condition which is absolute, that is to say,
one which is itself ground.
As ground, the fact is now the negative identity
that has repelled itself into those two moments:
first, in the shape of the sublated ground-connection,
the shape of an immediate manifold void of
unity and external to itself,
one that refers to the ground as an other to it
and at the same time constitutes its in-itself;
second, in the shape of an inner, simple form which is ground,
but which refers to the self-identical
immediate as to an other, determining it as condition,
that is, determining the in-itself of it as its own moment.
These two sides presuppose the totality,
presuppose that it is that which posits them.
Contrariwise, because they presuppose the totality,
the latter seems to be in turn also conditioned by them,
and the fact to spring forth from its condition and its ground.
But since these two sides have shown themselves to be an identity,
the relation of condition and ground has disappeared;
the two are reduced to a mere reflective shine;
the absolutely unconditioned is in its movement of positing
and presupposing only the movement in which this shine sublates itself.
It is the fact's own doing that it conditions itself
and places itself as ground over against its conditions;
but in connecting conditions and ground,
the fact is a reflection shining in itself;
its relation to them is a rejoining itself.

IV.10
hetu-phala-asraya-alambana sangrhitatvad esam abhave tad-abhava

c. Procession of the fact into concrete existence

The absolutely unconditioned is the absolute ground
that is identical with its condition,
the immediate fact as the truly essential.
As ground, it refers negatively to itself
and makes itself into a positedness;
but this positedness is a reflection
that is complete in both its sides
and is in them the self-identical form of connection,
as has transpired from its concept.
This positedness is therefore first the sublated ground,
the fact as an immediacy void of reflection,
the side of the conditions.
This is the totality of the determinations of the fact,
the fact itself, but the fact as thrown into
the externality of being, the restored circle of being.
In condition, essence lets go of the unity of its immanent reflection;
but it lets it go as an immediacy that now carries
the character of being a conditioning presupposition
and of essentially constituting only one of its sides.
For this reason the conditions are the whole content of the fact,
because they are the unconditioned in the form of formless being.
But because of this form, they also have yet another shape besides
the conditions of the content as this is in the fact as such.
They appear as a manifold without unity,
mingled with extra-essential elements
and other circumstances that do not belong
to the circle of existence as constituting
the conditions of this determinate fact.
For the absolute, unrestricted fact,
the sphere of being itself is the condition.
The ground, returning into itself, posits
that sphere as the first immediacy
to which it refers as to its unconditioned.
This immediacy, as sublated reflection,
is reflection in the element of being,
which thus forms itself as such into a whole;
form proliferates as determinateness of being
and thus appears as a manifold distinct from
the determination of reflection
and as a content indifferent to it.
The unessential, which is in the sphere of being
but which the latter sheds in so far as it is condition,
is the determinateness of the immediacy into which
the unity of form has sunk.
This unity of form, as the connection of being,
is in the latter at first as becoming the passing over
of a determinateness of being into another.
But the becoming of being is also the coming to be
of essence and a return to the ground.
The existence that constitutes the conditions, therefore,
is in truth not determined as condition by an other
and is not used by it as material;
on the contrary, it itself makes itself, through itself,
into the moment of an other.
Further, the becoming of this existence
does not start off from itself
as if it were truly the first and immediate;
on the contrary, its immediacy is
something only presupposed, and the movement of
its becoming is the doing of reflection itself.
The truth of existence is thus that it is condition;
its immediacy is solely by virtue of
the reflection of the ground-connection
that posits itself as sublated.
Consequently, like immediacy, becoming is only
the reflective shine of the unconditioned
inasmuch as this presupposes itself
and has its form in this presupposing,
and hence the immediacy of being is essentially
only a moment of the form.

The other side of this reflective shining of
the unconditioned is the ground-connection as such,
determined as form as against the immediacy
of the conditions and the content.
But this side is the form of the absolute fact
that possesses the unity of its form with itself
or its content within it,
and, in determining this content as condition,
in this very positing sublates the diversity of the content
and reduces it to a moment;
just as, contrariwise, as a form void of essence,
in this self-identity it gives itself
the immediacy of subsistence.
The reflection of the ground
sublates the immediacy of the conditions,
connecting them and making them
moments within the unity of the fact;
but the conditions are that which
the unconditioned fact itself presupposes
and the latter, therefore, sublates its own positing;
consequently, its positing converts itself
just as immediately into a becoming.
The two, therefore, are one unity;
the internal movement of the conditions is a becoming,
the return into the ground and the positing of the ground;
but the ground as posited, and this means as sublated, is the immediate.
The ground refers negatively to itself,
makes itself into a positedness and grounds the conditions;
in this, however, in that the immediate existence is
thus determined as a positedness,
the ground sublates it and only then makes itself into a ground.
This reflection is therefore the self-mediation of
the unconditioned fact through its negation.
Or rather, the reflection of the unconditioned is
at first a presupposing,
but this sublating of itself is
immediately a positing which determines;
secondly, in this positing the reflection is
immediately the sublating of the presupposed
and a determining from within itself;
this determining is thus in turn the sublating of the positing:
it is a becoming within itself.
In this, the mediation as a turning back
to itself through negation has disappeared;
mediation is simple reflection
reflectively shining within itself
and groundless, absolute becoming.
The fact's movement of being posited,
on the one hand through its conditions,
and on the other hand through its ground,
now is the disappearing of
the reflective shine of mediation.
The process by which the fact is posited
is accordingly a coming forth,
the simple self-staging of
the fact in concrete existence,
the pure movement of the fact to itself.

When all the conditions of a fact are at hand,
the fact steps into concrete existence.
The fact is, before it exists concretely;
it is, first, as essence or as unconditioned;
second, it has immediate existence or is determined,
and this in the twofold manner just considered,
on the one hand in its conditions
and on the other in its ground.
In the former case, it has given itself the form
of the external, groundless being,
for as absolute reflection the fact is
negative self-reference
and makes itself into its presupposition.
This presupposed unconditioned is,
therefore, the groundless immediate
whose being is just to be there, without grounds.
If, therefore, all the conditions of the fact are at hand,
that is, if the totality of the fact is
posited as a groundless immediate,
then this scattered manifold
internally recollects itself.
The whole fact must be there,
within its conditions,
or all the conditions belong
to its concrete existence;
for the all of them constitutes
the reflection of the fact.
Or again, immediate existence,
since it is condition,
is determined by form;
its determinations are therefore determinations of reflection
and with the positing of one the rest also are essentially posited.
The recollecting of the conditions is at first
the foundering to the ground of immediate existence
and the coming to be of the ground.
But the ground is thereby a posited ground, that is,
to the extent that it is ground,
to that extent it is sublated as ground
and is immediate being.
If, therefore, all the conditions of the fact are at hand,
they sublate themselves as immediate existence and as presupposition,
and the ground is equally sublated.
The latter proves to be only a reflective shine
that immediately disappears;
this coming forth is thus the tautological movement
of the fact to itself:
its mediation through the conditions and through the ground
is the disappearing of both of these.
The coming forth into concrete existence is therefore so immediate,
that it is mediated only by the disappearing of the mediation.

The fact proceeds from the ground.
It is not grounded or posited by it
in such a manner that the ground
would still stay underneath, as a substrate;
on the contrary, the positing is
the outward movement of ground to itself
and the simple disappearing of it.
Through its union with the conditions,
it obtains the external immediacy
and the moment of being.
But it does not obtain them
as a something external,
nor by referring to them externally;
rather, as ground it makes
itself into a positedness;
its simple essentiality rejoins
itself in the positedness
and, in this sublating of itself,
it is the disappearing of
its difference from its positedness,
and is thus simple essential immediacy.
It does not, therefore, linger on
as something distinct from the grounded;
on the contrary, the truth of the grounding is
that in grounding the ground unites with itself,
and its reflection into another is
consequently its reflection into itself.
The fact is thus the unconditioned
and, as such, equally so the groundless;
it arises from the ground only in so far as
the latter has foundered and is no longer ground:
it rises up from the groundless, that is,
from its own essential negativity or pure form.

This immediacy, mediated by ground and condition
and self-identical through the sublating of mediation,
is concrete existence.

SECTION II

Appearance

Essence must appear.

IV.11
atita-anagatam svarupato asti-adhva-bheda dharmanam

Being is the absolute abstraction;
this negativity is not something external to it,
but being is rather being,
and nothing but being,
only as this absolute negativity.
Because of this negativity,
being is only as self-sublating being
and is essence.
But, conversely,
essence as simple self-equality
is likewise being.
The doctrine of being contains
the first proposition, “being is essence.”
The second proposition, “essence is being,”
constitutes the content of the first section
of the doctrine of essence.
But this being into which
essence makes itself
is essential being,
concrete existence,
a being which has come forth
out of negativity and inwardness.

Thus essence appears.
Reflection is the internal shining of essence.
The determinations of this reflection are included
in the unity purely and simply as posited, sublated;
or reflection is essence immediately
identical with itself in its positedness.
But since this essence is ground,
through its self-sublating reflection,
or the reflection that which returns into itself,
essence determines itself as something real;
further, since this real determination, or the otherness,
of the ground-connection sublates itself
in the reflection of the ground
and becomes concrete existence,
the form determinations acquire therein
an element of independent subsistence.
Their reflective shine comes to completion in appearance.

The essentiality that has advanced to immediacy is,

first, concrete existence,
and a concrete existent or thing,
an undifferentiated unity of
essence and its immediacy.
The thing indeed contains reflection,
but its negativity is at first
dissolved in its immediacy;
but, because its ground is
essentially reflection,
its immediacy is sublated
and the thing makes itself
into a positedness.

Second, then, it is appearance.
Appearance is what the thing is in itself,
or the truth of it.
But this concrete existence,
only posited and reflected into otherness,
is equally the surpassing of itself into its infinity;
opposed to the world of appearance
there stands the world that exists
in itself reflected into itself.
But the being that appears and essential being
stand referred to each other absolutely.

Thus concrete existence is, third, essential relation;
what appears shows the essential,
and the essential is in its appearance.
Relation is the still incomplete union of
reflection into otherness and reflection into itself;
the complete interpenetrating of the two is actuality.

CHAPTER 1

Concrete existence

IV.12
te vyakta-suksma guna-atmana

Just as the principle of sufficient reason says
that whatever is has a ground,
or is something posited,
something mediated,
so there would also have to be
a principle of concrete existence saying
that whatever is, exists concretely.
The truth of being is to be,
not an immediate something,
but essence that has
come forth into immediacy.

But when it was further said
that whatever exists concretely
has a ground and is conditioned,
it also would have had to be said
that it has no ground and is unconditioned.
For concrete existence is the immediacy
that has come forth from the sublating
of the mediation that results
from the connection of ground and condition,
and which, in coming forth,
sublates this very coming forth.

Inasmuch as mention may be made here of
the proofs of the concrete existence of God,
it is first to be noted that besides
immediate being that comes first,
and concrete existence
(or the being that proceeds from essence)
that comes second, there is still a third being,
one that proceeds from the concept,
and this is objectivity.
Proof is, in general, mediated cognition.
The various kinds of being require or contain
each its own kind of mediation,
and so will the nature of the proof also vary accordingly.
The ontological proof wants to start from the concept;
it lays down as its basis the sum total of all realities,
where under reality also concrete existence is subsumed.
Its mediation, therefore, is that of the syllogism,
and syllogism is not yet under consideration here.
We have already commented above (Part 1, Section 1)
on Kant's objection to the ontological proof,
and have remarked that by concrete existence
Kant understands the determinate immediate existence
with which something enters into the context of total experience,
that is, into the determination of being an other
and of being in reference to an other.
As an existent concrete in this way,
something is thus mediated by an other,
and concrete existence is in general the side of its mediation.
But in what Kant calls the concept, namely,
something taken as only simply self-referring,
or in representation as such, this mediation is missing;
in abstract self-identity, opposition is left out.
Now the ontological proof would have
to demonstrate that the absolute concept,
namely the concept of God,
attains to a determinate existence, to mediation,
or to demonstrate how simple essence
mediates itself with mediation.
This is done by the just mentioned
subsumption of concrete existence
under its universal, namely reality,
which is assumed as the middle term
between God in his concept, on the one hand,
and concrete existence, on the other.
This mediation, inasmuch as it has the form of a syllogism,
is not at issue here, as already said.
However, how that mediation of
essence and concrete existence truly comes about,
this is contained in the preceding exposition.
The nature of the proof itself will be considered
in the doctrine of cognition.
Here we have only to indicate what pertains
to the nature of mediation in general.

The proofs of the existence of God
adduce a ground for this existence.
It is not supposed to be
an objective ground of the existence of God,
for this existence is in and for itself.
It is, therefore, solely a ground for cognition.
It thereby presents itself as a ground
that vanishes in the subject matter
that at first seems to be grounded by it.
Now the ground which is derived
from the contingency of the world
entails the regress of the latter
into the absolute essence,
for the accidental is that which is
in itself groundless and self-sublating.
In this way, therefore,
the absolute essence does indeed proceed
from that which has no ground,
for the ground sublates itself
and with this there also vanishes
the reflective shine of the relation
that was given to God,
that it is grounded in an other.
This mediation is therefore true mediation.
But the reflection involved
in that proof does not know
the nature of the mediation that it performs.
On the one hand, it takes itself
to be something merely subjective,
and it consequently distances
its mediation from God himself;
on the other hand, for that same reason
it also fails to recognize its mediating movement,
that this movement is in the essence itself
and how it is there.
The true relation of reflection consists
in being both in one:
mediation as such but, of course, at the same time
a subjective, external mediation,
that is to say, a self-external mediation
which in turn internally sublates itself.
In that other presentation, however,
concrete existence is given the false relation
of appearing only as mediated or posited.

So, on the other side, concrete existence also
cannot be regarded merely as an immediate.
Taken in the determination of an immediacy,
the comprehension of God's concrete existence
has been declared to be beyond proof
and the knowledge of it
an immediate consciousness only, a faith.
Knowledge should arrive at
the conclusion that it knows nothing,
and this means that it gives up its mediating movement
and the determinations themselves
that have come up in the course of it.
This is what has also occurred in the foregoing;
but it must be added that reflection,
by ending up with the sublation of itself,
does not thereby have nothing for result,
so that the positive knowledge of the essence
would then be an immediate reference to it,
divorced from that result and self-originating,
an act that starts only from itself;
on the contrary, the end itself,
the foundering of the mediation,
is at the same time the ground
from which the immediate proceeds.
In “zu Grunde gehen,” the German language unites,
as we remarked above,
the meaning of foundering and of ground;
the essence of God is said to be the
abyss (Abgrund in German) for finite reason.
This it is, indeed, in so far as
reason surrenders its finitude therein,
and sinks its mediating movement;
but this abyss, the negative ground,
is at the same time the positive ground
of the emergence of the existent,
of the essence immediate in itself;
mediation is an essential moment.
Mediation through ground sublates itself
but does not leave the ground standing under it,
so that what proceeds from it would be a posited
that has its essence elsewhere;
on the contrary, this ground is,
as an abyss, the vanished mediation,
and, conversely,
only the vanished mediation is
at the same time the ground
and, only through this negation,
the self-equal and immediate.

Concrete existence, then, is not to be taken here
as a predicate, or as a determination of essence,
of which it could be said in a proposition,
“essence exists concretely,” or “it has concrete existence.”
On the contrary, essence has passed over into concrete existence;
concrete existence is the absolute self-emptying of essence,
an emptying that leaves nothing of the essence behind.
The proposition should therefore run:
“Essence is concrete existence;
it is not distinct from its concrete existence.”
Essence has passed over into concrete existence
inasmuch as essence as ground
no longer distinguishes itself from itself as grounded,
or inasmuch as the ground has sublated itself.
But this negation is no less essentially its position,
or the simply positive continuity with itself;
concrete existence is
the reflection of the ground into itself,
its self-identity as attained in its negation,
therefore the mediation that has posited itself
as identical with itself and through that is immediacy.

Now because concrete existence is
essentially self-identical mediation,
it has the determinations of mediation in it,
but in such a way that the determinations are
at the same time reflected into themselves
and have essential and immediate subsistence.
As an immediacy which is posited through sublation,
concrete existence is negative unity and being-within-itself;
it therefore immediately determines itself
as a concrete existent and as thing.

IV.13
parinama-ekatvad vastu-tattvam

A. THE THING AND ITS PROPERTIES

Concrete existence as a concrete existent is posited in
the form of the negative unity which it essentially is.
But this negative unity is at first only immediate determination,
hence the oneness of the something in general.
But the concretely existent something is different
from the something that exists immediately.
The former is essentially an immediacy that has arisen
through the reflection of mediation into itself.
The concretely existent something is thus a thing.

The thing is distinct from its concrete existence
just as the something can be distinguished from its being.
The thing and the concrete existent
are immediately one and the same.
But because concrete existence is not
the first immediacy of being
but has the moment of mediation within it,
its further determination as thing
and the distinguishing of the two is
not a transition but truly an analysis.
Concrete existence as such contains
this very distinction in the moment of its mediation:
the distinction of thing-in-itself
and external concrete existence.

a. The thing in itself and concrete existence

1. The thing in itself is the concrete existent
as the essential immediate that has resulted
from the sublated mediation.
Mediation is therefore equally essential to it;
but this distinction in this first
or immediate concrete existence
falls apart into indifferent determinations.
The one side, namely the mediation of the thing,
is its non-reflected immediacy,
and hence its being in general;
and this being, since it is
at the same time determined as mediation,
is an existence which is other to itself,
manifold and external within itself.
But it is not just immediate existence;
it also refers to the sublated mediation
and the essential immediacy;
it is therefore immediate existence
as unessential, as positedness.
(When the thing is differentiated
from its concrete existence,
it is then the possible,
the thing of representation,
or the thing of thought,
which as such is at the same time
not supposed to exist.
However, the determination of possibility
and of the opposition of the thing
and its concrete existence comes later.)
But the thing-in-itself and its mediated being are
both contained in the concrete existence,
and both are themselves concrete existences;
the thing-in-itself exists concretely
and is the essential concrete existence,
but the mediated being is
the thing's unessential concrete existence.

The thing in itself, as the simple reflectedness of
the concrete existence within itself,
is not the ground of unessential existence;
it is the unmoved, indeterminate unity,
for it has precisely the determination of
being the sublated mediation,
and is therefore the substrate of that existence.
For this reason reflection, too,
as an immediate existence
which is mediated through some other,
falls outside the thing-in-itself.
The latter is not supposed to have
any determinate manifold in it;
for this reason it obtains it only
when exposed to external reflection,
though it remains indifferent to it.
(The thing-in-itself has color
only when exposed to the eye,
smell when exposed to the nose, and so on.)
Its diversity consists of aspects
which an other picks out,
specific points of reference
which this other assumes
with respect to the thing-in-itself
and which are not the thing's own determinations.

2. Now this other is reflection
which, determined as external, is,
first, external to itself and determinate manifoldness.
Second, it is external to the essential concrete existent
and refers to it as to its absolute presupposition.
These two moments of external reflection,
its own manifoldness and its reference to
the thing-in-itself as its other,
are however one and the same.
For this concrete existence is
external only in so far as it refers to
the essential identity as to an other.
The manifoldness, therefore, does not have
an independent  subsistence of its own
besides the thing-in-itself
but, over against it,
it is rather only as reflective shine;
in its necessary reference to it,
it is like a reflex refracting itself in it.
Diversity, therefore, is present as the reference
of an other to the thing-in-itself;
but this other is nothing that subsists on its own
but is only as reference to the thing-in-itself;
but at the same time it only is in being repelled from it;
thus it is the unsupported rebound of itself within itself.

Now since the thing-in-itself is
the essential identity of the concrete existence,
this essenceless reflection does not accrue to it
but collapses within itself externally to it.
It founders to the ground
and thus itself comes to be essential identity
or thing-in-itself.
This can also be looked at in this way:
the essenceless concrete existence has
in the thing-in-itself its reflection into itself;
it refers to it in the first place as to its other;
but as the other over against that which is in itself,
it is only the sublation of its self,
and its coming to be in the in-itself.
The thing-in-itself is thus identical
with external concrete existence.

This is exhibited in the thing-in-itself as follows.
The thing-in-itself is self-referring
essential concrete existence;
it is self-identity only in so far as
it holds negativity's reflection in itself;
that which appeared as concrete existence
external to it is, consequently, a moment in it.
It is for this reason also self-repelling thing-in-itself
which thus relates itself to itself as to an other.
Hence, there are now a plurality of things-in-themselves
standing in the reciprocal reference of external reflection.
This unessential concrete existence is
their reciprocal relation as others;
but it is, further, also essential to them
or, in other words, this unessential concrete existence,
in collapsing internally, is thing-in-itself,
but a thing-in-itself which is other than the first,
for that first is immediate essentiality
whereas the present proceeds from
the unessential concrete existence.
But this other thing-in-itself is only an other in general;
for, as self-identical thing, it has no
further determinateness vis-à-vis the first;
like the first, it is the reflection within itself
of the unessential concrete existence.
The determinateness of the various things-in-themselves
over against one another falls therefore into external reflection.

3. This external reflection is henceforth a relating of
the things-in-themselves to one another,
their reciprocal mediation as others.
The things-in-themselves are thus
the extreme terms of a syllogism,
the middle term of which is made up
by their external concrete existence,
the concrete existence by virtue of which
they are other to each other and distinct.
This, their difference, falls only
in their connecting reference;
they send determinations, as it were,
from their surface into the reference,
while remaining themselves indifferent to it.
This relation now constitutes
the totality of the concrete existence.
The thing-in-itself is drawn into
a reflection external to it
in which it has a manifold of determinations;
this is the repelling of itself from itself
into another thing-in-itself,
a repelling which is its rebounding back into itself,
for each thing-in-itself is an other
only as reflected back from the other;
it has its supposition not in itself but in the other,
is determined only through the determinateness of the other;
this other is equally determined only
through the determinateness of the first.
But the two things-in-themselves,
since each has its difference
not in it but in the other,
are not therefore distinct things;
the thing-in-itself, in relating as it should to
the other extreme as to another thing-in-itself,
relates to it as to something non-distinguished from it,
and the external reflection that should constitute
the mediating reference between the extremes is a
relation of the thing-in-itself only to itself,
or is essentially its reflection within itself;
the reflection is, therefore,
determinateness existing in itself,
or the determinateness of the thing-in-itself.
The latter, therefore, does not have this determinateness
in a reference, external to it,
to another thing-in-itself,
and of this other to it;
the determinateness is not just its surface
but is rather the essential mediation of
itself with itself as with an other.
The two things-in-themselves that should
constitute the extremes of the reference,
since they are supposed not
to have any contrasting determinateness,
collapse in fact into one;
it is only one thing-in-itself that
relates itself to itself in the external reflection,
and it is its own reference to itself as to another
that constitutes its determinateness.

This determinateness of the thing-in-itself is
the property of the thing.

b. Property

Quality is the immediate determinateness of something;
the negative itself by virtue of which being is something.
The property of the thing is, for its part,
the negativity of reflection,
by virtue of which concrete existence
in general is a concrete existent
and, as simple self-identity, is thing-in-itself.
But the negativity of reflection, the sublated mediation,
is itself essentially mediation and reference,
though not to an other in general like quality
which is not reflected determinateness;
it is rather reference to itself as to an other,
or mediation which immediately is no less self-identity.
The abstract thing-in-itself is itself this relation
which turns from another back to itself;
it is thereby determined in itself;
but its determinateness is constitution,
which is as such itself determination,
and in relating to the other
it does not pass over into otherness
and is excluded from alteration.

A thing has properties;

these are, first, its determinate references to something other;
the property is there only as a way of reciprocal relating;
it is, therefore, the external reflection of the thing
and the side of its positedness.

But, second, in this positedness the thing is in itself;
it maintains itself in its reference to the other
and thus is admittedly only a surface
where the concrete existence is exposed to
the becoming of being and to alteration;
the property is not lost in this.
A thing has the property to effect this or that in an other,
and in this connection to express itself in some characteristic way.
It demonstrates this property only under the condition
that another thing has a corresponding constitution,
but at the same time the property is characteristically
the thing's own and its self-identical substrate;
for this reason this reflected quality is called property.
The thing thereby passes over into an externality,
but the property maintains itself in this transition.
Through its properties the thing becomes cause,
and to be a cause is this, to preserve itself as effect.
However, the thing is here still the static thing of many properties;
it is not yet determined as actual cause;
it is so far only the reflection of
its determinations immediately existing in itself,
not yet itself the reflection that posits them.

Essentially, therefore, the thing-in-itself has
just shown itself to be thing-in-itself
not only in such a way that its properties are
the positedness of an external reflection;
on the contrary, those properties are its own determinations
by virtue of which it relates in some determinate manner;
it is not an indeterminate substrate located on
the other side of its external concrete existence
but is present in its properties rather as ground,
that is to say, it is self-identity in its positedness;
but, at the same time, it is conditioned ground,
that is to say, its positedness is
equally reflection external to itself;
it is reflected into itself and in itself only to
the extent that it is external.
Through concrete existence the thing-in-itself
enters into external references,
and the concrete existence consists
precisely in this externality;
it is the immediacy of being
and because of that the thing is
subjected to alteration;
but it is also the reflected immediacy of the ground,
hence the thing in itself in its alteration.
This mention of the ground-connection is
not however to be taken here as if
the thing in general were determined
as the ground of its properties;
thinghood itself is, as such, the ground-connection;
the property is not distinguished from its ground,
nor does it constitute just the positedness
but is rather the ground that has
passed over into its externality
and is consequently truly reflected into itself;
the property is itself, as such,
the ground, implicitly existent positedness;
it is the ground, in other words,
that constitutes the form of the property's identity,
and the property's determinateness is
the self-external reflection of the ground;
the whole is the ground which in its repelling and determining,
in its external immediacy, refers itself to itself.
The thing-in-itself thus concretely exists essentially,
and that it concretely exists essentially means,
conversely, that concrete existence, as external immediacy,
is at the same time in-itselfness.

c. The reciprocal action of things

The thing-in-itself exists in concreto by essence;
external immediacy and determinateness
belong to its being-in-itself,
or to its immanent reflection.
The thing in-itself is thus a thing that has properties,
and hence there are a number of things distinct from one another,
not because of some viewpoint alien to them but through themselves.
These many diverse things stand in essential reciprocal action
by virtue of their properties;
the property is this reciprocal connecting reference itself,
apart from which the thing is nothing;
the reciprocal determination,
the middle term of the things-in-themselves
that are taken as extreme terms
indifferent to the reference connecting them,
is itself the self-identical reflection
and the thing-in-itself
which those extremes were supposed to be.
Thinghood is thus reduced to the
form of indeterminate self-identity
having its essentiality only in its property.
Thus, if one speaks of a thing
or of things in general without a determinate property,
then their difference is merely indifferent, quantitative.
What is considered as a thing can just as well be made into
a plurality of things or be considered as a plurality of things;
their separation or their union is an external one.
A book is a thing, and each of its pages is also a thing,
and equally so every tiny piece of its pages,
and so on to infinity.
The determinateness, in virtue of which
a thing is this thing only,
lies solely in its properties.
It is through them that the thing
differentiates itself from other things,
for the property is the negative reflection
and the differentiating;
only in its property, therefore, does the thing possess
in it the difference of itself from others.
This is the difference reflected into itself,
by virtue of which the thing, in its positedness,
that is, in its reference to others,
is equally indifferent to the other
and to its reference to it.
Without its properties, therefore,
there is nothing that remains to the thing
except the unessential compass
and the external gathering of an abstract in-itselfness.
With this, thinghood has passed over into property.

The thing, as the extreme term that exists in itself,
was supposed to relate to the property,
and this property to constitute the middle term
between things that stand connected.
But this connection is where the things meet
as self-repelling reflection,
where they are distinguished and connected.
This, their distinction and their connecting reference,
is one reflection and one continuity of both.
Accordingly, the things themselves fall only
within this continuity which is the property;
they vanish as would-be self-subsisting extremes
that would have a concrete existence outside this property.

The property, which was supposed to connect
the self-subsisting extremes,
is therefore itself self-subsistent.
The things are, on the contrary, the unessential.
They are something essential only as
the self-differentiating
and self-referring reflection;
but this is the property.
The latter is in the thing,
therefore, not as something sublated,
not just a moment of it;
on the contrary, the truth of the thing is
that it is only an unessential compass
which is indeed a negative unity,
but only like the one of the something,
that is to say, a one which is immediate.
Whereas earlier the thing was determined as
an unessential compass because it was made such
by an external abstraction that omits the property,
this abstraction now happens through the transition of
the thing-in-itself into the property itself.
But there is now an inversion of values,
for the earlier abstraction still envisaged
the abstract thing without its property
as being the essential,
and the property as an external determination,
whereas it is the thing as such which is now reduced,
through itself, to the determination of
an indifferent external form of the property.
The latter is henceforth thus freed of
the indeterminate and impotent bond
which is the unity of the thing;
the property is what constitutes
the subsistence of the thing;
it is a self-subsisting matter.
Since this matter is simple continuity with itself,
it only possesses at first the form of diversity.
There is, therefore, a manifold of
these self-subsisting matters,
and the thing consists of them.

IV.14
vastu-samye citta-bhedat tayo vibhakta pantha

B. THE CONSTITUTION OF THE THING OUT OF MATTERS

The transition of property into a matter
or into a self-subsistent stuff is the familiar
transition performed on sensible matter by chemistry
when it seeks to represent the properties of color, smell, etc.,
as luminous matter, coloring matter, odorific matter,
sour, bitter matter and so on;
or when it simply assumes others,
like calorific matter, electrical, magnetic matter,
in the conviction that it has thereby gotten hold
of properties as they truly are.
Equally current is the saying that
things consist of various matters or stuffs.
One is careful about calling these matters or stuffs “things,”
even though one will readily admit that,
for example, a pigment is a thing;
but I do not know whether luminous matter,
for instance, or calorific matter,
or electrical matter, etc., are called things.
The distinction is made between things and their components
without any exact statement as to whether these components also,
and to what extent, are things or perhaps just half-things;
but they are at least concretes in general.

The necessity of making the transition
from properties to matters,
or of assuming that the properties are truly matters,
has resulted from the fact that they are
what is the essential in things
and consequently their true self-subsistence.
At the same time, however,
the reflection of the property into itself
constitutes only one side of the whole reflection,
namely the sublation of the distinction
and the continuity of the property
(which was supposed to be a concrete existence for an other)
with itself.
Thinghood, as immanent negative reflection
and as a distinguishing that repels itself from the other,
has consequently been reduced to an unessential moment;
at the same time, however, it has further determined itself.

First, this negative moment has preserved itself,
for property has become a matter continuous with itself
and self-subsisting only inasmuch as
the difference of things has sublated itself;
thus the continuity of the property in the otherness
itself contains the moment of the negative,
and, as this negative unity,
its self-subsistence is at the same time
the restored something of thinghood,
negative self-subsistence versus
the positive self-subsistence of the stuff.

Second, the thing has thereby progressed
from its indeterminacy to full determinateness.
As thing in itself, it is abstract identity,
simple negative concrete existence,
or this concrete existence
determined as the indeterminate;
it is then determined through its properties,
by virtue of which it is supposed to be
distinguished from other things;
but, since through the property the thing is
rather continuous with other things,
this imperfect distinction is sublated;
the thing has thereby returned into itself
and is now determined as determined;
it is determined in itself or is this thing.

But, third, this turning back into itself,
though a self-referring determination,
is at the same time an unessential determination;
the self-continuous subsistence makes up
the self-subsistent matter
in which the difference of things,
their determinateness existing in and for itself,
is sublated and is something external.
Therefore, although the thing as this thing
is complete determinateness,
this determinateness is such
in the element of inessentiality.

Considered from the side of the movement of the property,
this result follows in this way.
The property is not only external determination but
concrete existence immediately existing in itself.
This unity of externality and essentiality repels itself from itself,
for it contains reflection-into-itself and reflection-into-other,
and, on the one hand, it is determination as simple,
self-identical and self-referring self-subsistent in which the negative unity,
the one of the thing, is sublated;
on the other hand, it is this determination over against an other,
but likewise as a one which is reflected into itself
and is determined in itself;
it is, therefore, the matters and this thing.
These are the two moments of self-identical externality,
or of property reflected into itself.
The property was that by which things
were supposed to be distinguished.
Since the thing has freed itself of its
negative side of inhering in an other,
it has thereby also become free
from its being determined by other things
and has returned into itself
from the reference connecting it to the other.
At the same time, however, it is only the thing-in-itself
now become the other of itself,
for the manifold properties on their part
have become self-subsistent
and their negative connection
in the one of the thing is
now only a sublated connection.
Consequently, the thing is self-identical negation
only as against the positive continuity of the material.

The “this” thus constitutes the
complete determinateness of the thing,
a determinateness which is at the same time
an external determinateness.
The thing consists of self-subsistent matters
indifferent to the connection they have in the thing.
This connection is therefore only
an unessential linking of them,
the difference of one thing from another
depending on whether there is in it
a more or less of particular matters
and in what amount.
These matters overrun this thing,
continue into others,
and that they belong to this thing
is no restriction for them.
Just as little are they, moreover,
a restriction for one another,
for their negative connection is
only the impotent “this.”
Hence, in being linked together in it,
they do not sublate themselves;
they are as self-subsistent,
impenetrable to each other;
in their determinateness they refer only to themselves
and are a mutually indifferent manifold of subsistence;
the only limit of which they are capable is a quantitative one.
The thing as this is just their merely quantitative connection,
a mere collection, their “also.”
The thing consists of some quantum or other of a matter,
also of the quantum of another, and also of yet another;
this combination, of not having any combination alone
constitutes the thing.

IV.15
na ca-eka-citta-tantram vastu tad apramanakam tada kim syat

C. DISSOLUTION OF THE THING

This thing, in the manner it has determined itself
as the merely quantitative combination of free matters,
is the absolutely alterable.
Its alteration consists in one or more matters
being dropped from the collection,
or being added to this “also,”
or in the rearrangement of the matters'
respective quantitative ratio.
The coming-to-be and the passing-away of this thing is
the external dissolution of such an external bond,
or the binding of such for which it is indifferent
whether they are bound or not.
The stuffs circulate unchecked in or out of “this” thing,
and the thing itself is absolute porosity
without measure or form of its own.

So the thing, in the absolute determinateness
through which it is a “this,”
is the absolutely dissoluble thing.
This dissolution is an external process of being determined,
just like the being of the thing;
but its dissolution and the externality of its being
is the essential of this being;
the thing is only the “also”;
it consists only of this externality.
But it consists also of its matters,
and not just the abstract “this” as such
but the “this” thing whole is the dissolution of itself.
For the thing is determined as an external collection
of self-subsisting matters;
such matters are not things,
they lack negative self-subsistence;
it is the properties which are rather self-subsistent,
that is to say, are determined with a being
which, as such, is reflected into itself.
Hence the matters are indeed simple, referring only to themselves;
but it is their content which is a determinateness;
the immanent reflection is only the form of this content,
a content which is not, as such, reflected-into-itself
but refers to an other according to its determinateness.
The thing, therefore, is not only their “also,”
is not their reference to each other as indifferent
but is, on the contrary, equally so their negative reference;
and on account of their determinateness
the matters are themselves this negative reflection
which is the puncticity of the thing.
The one matter is not what the other is
according to the determinateness of its content
as contrasted to that of an other;
and the one is not to the extent that the other is,
in accordance with their self-subsistence.

The thing is, therefore, the connecting reference of
the matters of which it consists to each other,
in such a manner that the one matter,
and the other also, subsist in it,
and yet, at the same time,
the one matter does not subsist
in it in so far as the other does.
To the extent, therefore, that
the one matter is in the thing,
the other is thereby sublated;
but the thing is at the same time
the “also,” or the subsistence of the other matter.
In the subsistence of the one matter, therefore,
the other matter does not subsist,
and it also no less subsists in it;
and so with all these diverse matters
in respect to each other.
Since it is thus in the same respect
as the one matter subsists
that the other subsists also,
and this one subsistence of both is
the puncticity or the negative unity of the thing,
the two interpenetrate absolutely;
and since the thing is at the same time
only the “also” of the matters,
and these are reflected into their determinateness,
they are indifferent to one another,
and in interpenetrating they do not touch.
The matters are, therefore, essentially porous,
so that the one subsists in the pores
or in the non-subsistence of the others;
but these others are themselves porous;
in their pores or their non-subsistence
the first and also all the rest subsist;
their subsistence is at the same time
their sublatedness and the subsistence of others;
and this subsistence of the others is
just as much their sublatedness
and the subsisting of the first
and equally so of all others.
The thing is, therefore,
the self-contradictory mediation of
independent self-subsistence through its opposite,
that is to say, through its negation,
or of one self-subsisting matter
through the subsisting and non-subsisting of an other.

In “this” thing, concrete existence has attained its completion,
namely, that it is at once being that exists in itself,
or independent subsistence, and unessential concrete existence.
The truth of concrete existence is thus this:
that it has its in-itself in unessentiality,
or that it subsists in an other,
indeed in the absolute other,
or that it has its own nothingness for substrate.
It is, therefore, appearance.

CHAPTER 2

Appearance

Concrete existence is the immediacy of being
to which essence has again restored itself.
In itself this immediacy is the reflection of essence into itself.
As concrete existence, essence has stepped out of its ground
which has itself passed over into it.
Concrete existence is this reflected immediacy
in so far as, within, it is absolute negativity.
It is now also posited as such,
in that it has determined itself as appearance.

At first, therefore, appearance is
essence in its concrete existence;
essence is immediately present in it.
That it is not immediate,
but rather reflected concrete existence,
constitutes the moment of essence in it;
or concrete existence, as essential concrete existence,
is appearance.

Something is only appearance,
in the sense that concrete existence is
as such only a posited being,
not something that is in-and-for-itself.
This is what constitutes its essentiality,
to have the negativity of reflection,
the nature of essence, within it.
There is no question here of an alien,
external reflection to which essence would belong
and which, by comparing this essence with concrete existence,
would declare the latter to be appearance.
On the contrary, as we have seen,
this essentiality of concrete existence,
that it is appearance, is
concrete existence's own truth.
The reflection by virtue of which
it is this is its own.

But if it is said that something is only appearance,
meaning that as contrasted with it
immediate concrete existence is the truth,
then the fact is that appearance is the higher truth,
for it is concrete existence as essential,
whereas concrete existence is appearance
that is still void of essence
because it only contains in it
the one moment of appearance,
namely that of concrete existence
as immediate, not yet negative, reflection.
When appearance is said to be essenceless,
one thinks of the moment of its negativity as if,
by contrast with it, the immediate were
the positive and the true;
in fact, however, this immediate does not
yet contain essential truth in it.
Concrete existence rather ceases to be essenceless
by passing over into appearance.

Essence reflectively shines at first
just within, in its simple identity;
as such, it is abstract reflection,
the pure movement of nothing
through nothing back to itself.
Essence appears, and so it now is real shine,
since the moments of the shine have concrete existence.
Appearance, as we have seen, is the thing as
the negative mediation of itself with itself;
the differences which it contains
are self-subsisting matters
which are the contradiction of
being an immediate subsistence,
yet of obtaining their subsistence
only in an alien self-subsistence,
hence in the negation of their own,
but then again, just because of that,
also in the negation of that alien self-subsistence
or in the negation of their own negation.
Reflective shine is this same mediation,
but its fleeting moments obtain in appearance
the shape of immediate self-subsistence.
On the other hand, the immediate self-subsistence
which pertains to concrete existence is reduced to a moment.
Appearance is therefore the unity of
reflective shine and concrete existence.

Appearance now determines itself further.
It is concrete existence as essential;
as essential, concrete existence
differs from the concrete existence
which is unessential,
and these two sides
refer to each other.

Appearance is, therefore,
first, simple self-identity
which also contains
diverse content determinations
and, both as identity
and as the connecting reference
of these determinations,
is that which remains self-equal
in the flux of appearance;
this is the law of appearance.

But, second, the law which is
simple in its diversity
passes over into opposition;
the essential moment of appearance becomes
opposed to appearance itself
and, confronting the world of appearance,
the world that exists in itself
comes onto the scene.

Third, this opposition returns into its ground;
that which is in itself is in the appearance
and, conversely, that which appears is determined
as taken up into its being-in-itself.
Appearance becomes relation.

IV.16
tad-uparaga-apeksitvat-cittasya vastu jnata-ajnatam

A. THE LAW OF APPEARANCE

1. Appearance is the concrete existent
mediated through its negation,
which constitutes its subsistence.
This, its negation, is
indeed another self-subsistent;
but the latter is just as
essentially something sublated.
The concrete existent is consequently
the turning back of itself into itself
through its negation and through
the negation of this negation;
it has, therefore, essential self-subsistence,
just as it is equally immediately an absolute positedness
that has a ground and an other for its subsistence.
In the first place, therefore, appearance is
concrete existence along with its essentiality,
the positedness along with its ground;
but this ground is the negation,
and the other self-subsistent,
the ground of the first,
is equally only a positedness.
Or the concrete existent is,
as an appearance,
reflected into an other
and has this other for its ground,
and this ground is itself only this,
to be reflected into another.
The essential self-subsistence
that belongs to it because
it is a turning back into itself is,
for the sake of the negativity of the moments,
the return of nothing through nothing back to itself;
the self-subsistence of the concrete existent is
therefore only the reflective shine of essence.
The linkage of the reciprocally grounding
concrete existents consists, therefore,
in this reciprocal negation,
namely that the subsistence of the one is not
the subsistence of the other but is its positedness,
where this connection of positedness
alone constitutes their subsistence.
The ground is present as it is in truth,
namely as being a first which is only a presupposed.

This now constitutes the negative side of appearance.
In this negative mediation, however,
there is immediately contained the positive identity of
the concrete existent with itself.
For this concrete existent is not positedness
vis-à-vis an essential ground,
or is not the reflective shine in a self-subsistent,
but is rather positedness that refers itself to a positedness,
or a reflective shine only in a reflective shine.
In this, its negation, or in its other
which is itself something sublated,
it refers to itself and is thus
self-identical or positive essentiality.
This identity is not the immediacy
that pertains to concrete existence as such
and only is its unessential moment of subsisting in an other.
It is rather the essential content of appearance which has two sides:
first, to be in the form of positedness or external immediacy;
second, to be positedness as self-identical.
According to the first side, it is as a determinate being,
but one which in keeping with its immediacy is accidental, unessential,
and subject to transition, to coming-to-be and passing-away.
According to the other side, it is the simple content determination
exempted from that flux, the permanent element in it.

This content, besides being in general
the simple element of the transient,
is also a determined content, varied in itself.
It is the reflection of appearance,
of the negative determinate being, into itself,
and therefore contains determinateness essentially.
Appearance is however the multifarious diversity of
immediately existing beings that revels in unessential manifoldness;
its reflected content, on the other hand,
is its manifoldness reduced to simple difference.
Or, more precisely, the determinate essential content is not
just determined in general but,
as the essential element of appearance,
is complete determinateness; the one and its other.
Each of these two has in appearance
its subsistence in the other,
but in such a way that it is at the same time
only in the other's non-subsistence.
This contradiction sublates itself;
and its reflection into itself is
the identity of their two-sided subsistence,
namely that the positedness of the one is
also the positedness of the other.
The two constitute one subsistence,
each at the same time as a different content
indifferent to the other.
In the essential side of appearance,
the negativity of the unessential content,
that it sublates itself, has thus gone back into identity;
it is an indifferent subsistence which is not
the sublatedness of the other but rather its subsistence.

This unity is the law of appearance.

2. The law is thus the positive element
of the mediation of what appears.
Appearance is at first concrete existence
as negative self-mediation,
so that the concrete existent,
through its own non-subsistence,
through an other and again through
the non-subsistence of this other,
is mediated with itself.
In this there is contained,
first, the merely reflective shining
and the disappearing of both,
the unessential appearance;
second, also the persistence or the law;
for each of the two concretely exists
in the sublation of the other,
and their positedness is as
their negativity at the same time
the identical positive positedness of both.

This permanent subsistence which appearance
obtains in the law is thus,
as it has determined itself,

first, opposed to the immediacy
of the being which concrete existence has.
This immediacy is indeed one which is in itself reflected,
namely the ground that has gone back into itself;
but in appearance this simple immediacy is now distinguished
from the reflected immediacy that first began
to separate itself in the “thing.”
The concretely existing thing in its dissolution
has become this opposition;
the positive element of its dissolution is
the said self-identity of what appears,
a positedness in the positedness of its other.

Second, this reflected immediacy is itself determined
as positedness over against the immediate determinate
being of concrete existence.
This positedness is henceforth what is essential
and the true positive.
The German expression Gesetz [law] likewise contains this
note of positedness or Gesetztsein.
In this positedness there lies the essential connection of
the two sides of the difference that the law contains;
they are a diverse content,
each immediate with respect to the other,
and they are this as the reflection of
the disappearing content belonging to appearance.
As essential difference, the different sides are
simple, self-referring determinations of content.
But just as equally, neither is immediate,
just for itself, but is rather essential positedness,
or is only to the extent that the other is.

Third, appearance and law have one and the same content.
The law is the reflection of appearance into self-identity;
appearance, as an immediate which is null,
thus stands opposed to that which is immanently reflected,
and the two are distinguished according to form.
But the reflection of appearance
by virtue of which this difference is,
is also the essential identity of
appearance itself and its reflection,
and this is in general the nature of reflection;
it is what in the positedness is self-identical
and indifferent to that difference,
which is form or positedness
hence a content continuous
from appearance to law,
the content of the law
and of the appearance.

This content thus constitutes the substrate of appearance;
the law is this substrate itself,
appearance is the same content but contains still more,
namely the unessential content of its immediate being.
And so is also the form determination by which
appearance as such is distinguished from the law,
namely a content and equally a content distinguished
from the content of the law.
For concrete existence, as immediacy in general,
is likewise a self-identity of matter and form
which is indifferent to its form determinations
and is, therefore, a content;
the concrete existence is the thinghood
with its properties and matters.
But it is the content whose self-subsisting immediacy is
at the same time also only a non-subsistence.
But the self-identity of the content
in this its non-subsistence
is the other, essential content.
This identity, the substrate of appearance,
which constitutes law,
is appearances's own moment;
it is the positive side of the essentiality
by virtue of which concrete existence is appearance.

The law, therefore, is not beyond appearance
but is immediately present in it;
the kingdom of laws is the restful copy
of the concretely existing or appearing world.
But, more to the point, the two are one totality,
and the concretely existing world is itself
the kingdom of laws which, simple identity,
is at the same time self-identical in the positedness
or in the self-dissolving self-subsistence of concrete existence.
In the law, concrete existence returns to its ground;
appearance contains both of these, the simple ground
and the dissolving movement of the appearing universe,
of which the law is the essentiality.

3. The law is therefore the essential appearance;
it is the latter's reflection into itself in its positedness,
the identical content of itself and the unessential concrete existence.

In the first place, this identity of the law
with its concrete existence is now, to start with,
immediate, simple identity, and the law is indifferent
with respect to its concrete existence;
appearance still has another content as
contrasted with the content of the law.
That content is indeed the unessential one
and the return into the latter;
but for the law it is an original starting point not posited by it;
as content, therefore, it is externally bound up with the law.
Appearance is an aggregate of more detailed determinations
that belong to the “this” or the concrete,
and are not contained in the law
but are rather determined each by an other.

Secondly, that which appearance contains
distinct from the law determined itself as
something positive or as another content;
but it is essentially a negative;
it is the form and its movement is
a movement that belongs to appearance.
The kingdom of laws is
the restful content of appearance;
the latter is this same content
but displayed in restless flux
and as reflection-into-other.
It is the law as negative,
relentlessly self-mutating concrete existence,
the movement of the passing over into the opposite,
of self-sublation and return into unity.
This side of the restless form
or of the negativity
does not contain the law;
as against the law, therefore,
appearance is the totality,
for it contains the law but more yet,
namely the moment of the self-moving form.

Thirdly, this shortcoming is manifested
in the law in the mere diversity at first,
and the consequent internal indifference, of its content;
the identity of its sides with one another
is at first, therefore,
only immediate and hence inner,
not yet necessary in other words.
In a law two content determinations are
essentially bound together
(for instance, spatial and temporal magnitudes
in the law of falling bodies:
the traversed spaces vary as
the squares of the elapsed times);
they are bound together;
this connection is at first
only an immediate one.

At first, therefore, it is likewise only a posited connection,
just as the immediate has obtained in appearance
the meaning of positedness in general.
The essential unity of the two sides
of the law would be their negativity,
namely that each contains the other in it;
but in the law this essential unity has not yet come the fore.
(Thus it is not contained in the concept of
the space traversed by a falling body
that time corresponds to it as a square.
Because the falling is a sensible movement,
it is the ratio of space and time;
but first, that time refers to space and space to time
does not lie in the determination of time itself,
that is to say, in time as ordinarily represented;
it is said that time can very well be represented
without space and space without time;
the one thus comes to the other externally,
and their external reference to each other is movement.
Second, the more particular determination of
how the magnitudes further relate to
each other in movement is indifferent.
The relevant law here is drawn from experience
and is to this extent immediate;
there is still required a proof,
that is, a mediation,
in order to know that the law
not only occurs but is necessary;
the law as such does not contain
this proof and its objective necessity.)

The law is, therefore, only
the positive essentiality of appearance,
not its negative essentiality according to which
the content determinations are moments of the form,
as such pass over into their other
and are in their own selves
not themselves but their other.
In the law, therefore, although
the positedness of the one side of it is
the positedness of the other side,
the content of the two sides is
indifferent to this connection;
it does not contain this positedness in it.
Law, therefore, is indeed essential form,
but not as yet real form which is reflected
into its sides as content.

IV.17
sada jnata citta-vrttaya tat-prabho purusasya-aparinamitvat

B. THE WORLD OF APPEARANCE AND THE WORLD-IN-ITSELF

1. The concrete existing world tranquilly
raises itself to a kingdom of laws;
the null content of its manifold determinate being
has its subsistence in an other;
its subsistence is therefore its dissolution.
In this other, however, that which appears also comes to itself;
thus appearance is in its changing also an enduring,
and its positedness is law.
Law is this simple identity of appearance with itself;
it is, therefore, its substrate and not its ground,
for it is not the negative unity of appearance
but, as its simple identity, is its immediate unity,
the abstract unity, alongside which, therefore,
its other content also occurs.
The content is this content; it holds together internally,
or has its negative reflection inside itself.
It is reflected into an other;
this other is itself a concrete existence of appearance;
the appearing things have their grounds and conditions
in other appearing things.

In fact, however, law is also
the other of appearance as appearance,
and its negative reflection as in its other.
The content of appearance,
which differs from the content of law,
is the concrete existent
which has negativity for its ground
or is reflected into its non-being.
But this other, which is also a concrete existent,
is such an existent as likewise reflected into its non-being;
it is thus the same and that which appears in it
is in fact reflected not into an other but into itself;
it is this very reflection of positedness into itself
which is law.
But as something that appears
it is essentially reflected into its non-being,
or its identity is itself essentially
just as much its negativity and its other.
The immanent reflection of appearance,
law, is therefore not only
the identical substrate of appearance
but the latter has in law its opposite,
and law is its negative unity.

Now through this, the determination of law
has been altered within the law itself.
At first, law is only a diversified content
and the formal reflection of positedness into itself,
so that the positedness of one of its sides is
the positedness of the other side.
But because it is also the negative reflection into itself,
its sides behave not only as different
but as negatively referring to each other.
Or, if the law is considered just for itself,
the sides of its content are indifferent to each other;
but they are no less sublated through their identity;
the positedness of the one is the positedness of the other;
consequently, the subsistence of each is
also the non-subsistence of itself.
This positedness of the one side in the other
is their negative unity,
and each positedness is not only the positedness
of that side but also of the other,
or each side is itself this negative unity.
The positive identity which they have in the law as such is
at first only their inner unity
which stands in need of proof and mediation,
since this negative unity is not yet posited in them.
But since the different sides of law are now determined
as being different in their negative unity,
or as being such that each contains the other within
while at the same time repelling this otherness from itself,
the identity of law is now also one which is posited and real.

Consequently, law has likewise obtained
the missing moment of the negative form of its sides,
the moment that previously still belonged to appearance;
concrete existence has thereby returned into itself fully
and has reflected itself into its absolute otherness
which has determinate being-in-and-for-itself.
That which was previously law, therefore,
is no longer only one side of the whole.
It is the essential totality of appearance,
so that it now obtains also the moment of
unessentiality that belonged to the latter
but as reflected unessentiality
that has determinate being in itself,
that is, as essential negativity.
As immediate content, law is determined in general,
distinguished from other laws,
of which there is an indeterminate multitude.
But because now it explicitly is essential negativity,
it no longer contains that merely
indifferent, accidental content determination;
its content is rather every determinateness in general,
essentially connected together in a totalizing connection.
Thus appearance reflected-into-itself is
now a world that discloses itself above
the world of appearance as one
which is in and for itself.

The kingdom of laws contains only
the simple, unchanging but diversified content
of the concretely existing world.
But because it is now the total reflection of this world,
it also contains the moment of its essenceless manifoldness.
This moment of alterability and alteration,
reflected into itself and essential,
is the absolute negativity
or the form in general as such:
its moments, however, have  the reality of
self-subsisting but reflected concrete existence
in the world  that has determinate being in-and-for-itself,
just as, conversely, this reflected
self-subsistence has form in it,
and its content is therefore not a mere manifold
but a content holding itself together essentially.

This world which is in and for itself is
also called the suprasensible world,
inasmuch as the concretely existing world
is characterized as sensible,
that is, as one intended for intuition,
which is the immediate attitude of consciousness.
The suprasensible world likewise has
immediate, concrete existence,
but reflected, essential concrete existence.
Essence has no immediate existence yet;
but it is, and in a more profound sense than being;
the thing is the beginning of the reflected concrete existence;
it is an immediacy which is not yet posited,
not yet essential or reflected;
but it is in truth not an immediate which is simply there.
Things are posited only as the
things of another, suprasensible, world
first as true concrete existences,
and, second, as the truth in contrast to that which just is.
What is recognized in them is that there is
a being distinguished from immediate being,
and this being is true concrete existence.
On the one side, the sense-representation
that ascribes concrete existence
only to the immediate being of
feeling and intuition is in this determination overcome;
but, on the other side, also overcome is
the unconscious reflection which,
although it possesses the representation of things,
forces, the inner, and so on, does not know
that such determinations are not sensible
or immediately existing beings,
but reflected concrete existences.

2. The world which is in and for itself is
the totality of concrete existence;
outside it there is nothing.
But, within it, it is absolute negativity or form,
and therefore its immanent reflection is
negative self-reference.
It contains opposition,
and splits internally
as the world of the senses
and as the world of otherness
or the world of appearance.
For this reason, since it is totality,
it is also only one side of the totality
and constitutes in this determination
a self-subsistence different from the world of appearance.
The world of appearance has its negative unity
in the essential world to which it founders
and into which it returns as to its ground.
Further, the essential world is also
the positing ground of the world of appearances;
for, since it contains the absolute form essentially,
it sublates its self-identity,
makes itself into positedness
and, as this posited immediacy,
it is the world of appearance.

Further, it is not only ground in general
of the world of appearance but its determinate ground.
Already as the kingdom of laws it is a manifold of content,
indeed the essential content of the world of appearance,
and, as ground with content, it is
the determinate ground of that other world.
But it is such only according to that content,
for the world of appearance still had other
and manifold content than the kingdom of laws,
because the negative moment was still the one peculiarly its own.
But because the kingdom of laws now has this moment likewise in it,
it is the totality of the content of the world of appearance
and the ground of all its manifoldness.
But it is at the same time the negative of
this manifoldness and thus a world opposed to it.
That is to say, in the identity of the two worlds,
because the one world is determined
according to form as the essential
and the other as the same world
but posited and unessential,
the connection of ground has indeed been restored.
But it has been restored as the ground-connection of appearance,
namely as the connection,
not of the two sides of an identical content,
nor of a mere diversified content, like law,
but as total connection,
or as negative identity and essential connection
of the opposed sides of the content.
The kingdom of laws is not only this,
that the positedness of a content
is the positedness of an other,
but rather that this identity, as we have seen,
is essentially also negative unity,
and in this negative unity
each of the two sides of law is in it,
therefore, its other content;
consequently, the other is not
an other in general, indeterminedly,
but is its other, equally containing
the content determination of that other;
and thus the two sides are opposed.
Now, because the kingdom of laws now has in it
this negative moment, namely opposition,
and thus, as totality, splits into a world
which exists in and for itself and a world of appearance,
the identity of these two is
the essential connection of opposition.
The connection of ground is, as such, the opposition
which, in its contradiction, has foundered to the ground;
and concrete existence is the ground that has come to itself.
But concrete existence becomes appearance;
ground is sublated in concrete existence;
it reinstates itself as the return of appearance into itself,
but does so as sublated ground, that is to say,
as the ground-connection of opposite determinations;
the identity of such determinations, however, is
essentially a becoming and a transition,
no longer the connection of ground as such.

The world that exists in and for itself is
thus itself a world distinguished within itself,
in the total compass of a manifold content.
That is to say, it is identical with the world of appearance
or the posited world and to this extent it is its ground.
But its identity connection is at the same time
determined as opposition,
because the form of the world of appearance is
reflection into its otherness
and this world of appearance, therefore, in the
world that exists in and for itself
has truly returned into itself,
in such a manner that
that other world is its opposite.
Their connection is, therefore, specifically this,
that the world that exists in and for itself is
the inversion of the world of appearance.

IV.18
na tat sva-abhasam drsyatvat

C. THE DISSOLUTION OF APPEARANCE

The world that exists in and for itself is
the determinate ground of the world of appearance
and is this only in so far as, within it,
it is the negative moment
and hence the totality of the content determinations
and their alterations that correspond to that world of appearance,
yet constitutes at the same time its completely opposed side.
The two worlds thus relate to each other in such a way
that what in the world of appearance is positive,
in the world existing in and for itself is negative,
and, conversely, what is negative in the former
is positive in the latter.
The north pole in the world of appearance
is the south pole in and for itself, and vice-versa;
positive electricity is in itself negative, and so forth.
What is evil in the world of appearance is
in and for itself goodness and a piece of good luck.

In fact it is precisely in this opposition
of the two worlds that their difference has disappeared,
and what was supposed to be
the world existing in and for itself is
itself the world of appearance
and this last, conversely,
the world essential within.
The world of appearance is in the first instance
determined as reflection into otherness,
so that its determinations and concrete existences have
their ground and subsistence in an other;
but because this other, as other,
is likewise reflected into an other,
the other to which they both refer is
one which sublates itself as other;
the two consequently refer to themselves;
the world of appearance is within it,
therefore, law equal to itself.
Conversely, the world existing in and for itself is
in the first instance self-identical content,
exempt from otherness and change;
but this content, as complete reflection of
the world of appearance into itself,
or because its diversity is difference
reflected into itself and absolute,
consequently contains negativity as a moment
and self-reference as reference to otherness;
it thereby becomes self-opposed, self-inverting, essenceless content.
Further, this content of the world existing in and for itself has
thereby also retained the form of immediate concrete existence.
For it is at first the ground of the world of appearance;
but since it has opposition in it, it is equally
sublated ground and immediate concrete existence.

Thus the world of appearance and the essential world are
each, each within it, the totality of
self-identical reflection and of reflection-into-other,
or of being-in-and-for-itself.
They are both the self-subsisting wholes of concrete existence;
the one is supposed to be only reflected concrete existence,
the other immediate concrete existence;
but each continues into the other and, within, is
therefore the identity of these two moments.
What we have, therefore, is this totality
that splits into two totalities,
the one reflected totality and the other immediate totality.
Both, in the first instance, are self-subsistent;
but they are this only as totalities,
and this they are inasmuch as each essentially contains
the moment of the other in it.
Hence the distinct self-subsistence of each,
one determined as immediate and one as reflected,
is now so posited as to be essentially the reference to the other
and to have its self-subsistence in this unity of the two.

We started off from the law of appearance;
this law is the identity of a content
and another content different from it,
so that the positedness of the one
is the positedness of the other.
Still present in law is this difference,
that the identity of its sides is
at first only an internal identity
which the two sides do not yet have in them.
Consequently the identity is, for its part, not realized;
the content of law is not identical
but indifferent, diversified.
This content, therefore, is on its side only in itself
so determined that the positedness of the one is
the positedness of the other;
this determination is not yet present in it.
But now law is realized;
its inner identity is existent at the same time
and, conversely, the content of law is raised to ideality;
for it is sublated within, is reflected into itself,
for each side has the other in it,
and therefore is truly identical
with it and with itself.

Thus is law essential relation.
The truth of the unessential world is at first
a world in and for itself and other to it;
but this world is a totality,
for it is itself and the first world;
both are thus immediate concrete existences
and consequently reflections in their otherness,
and therefore equally truly reflected into themselves.
“World” signifies in general
the formless totality of a manifoldness;
this world has foundered both as essential world
and as world of appearance;
it is still a totality or a universe
but as essential relation.
Two totalities of content have arisen in appearance;
at first they are determined as
indifferently self-subsisting vis-à-vis each other,
each having indeed form within it
but not with respect to the other;
this form has however demonstrated
itself to be their connecting reference,
and the essential relation is
the consummation of their unity of form.

CHAPTER 3

The essential relation

IV.19
eka-samaye ca-ubhaya-anavadharanam

The truth of appearance is the essential relation.
Its content has immediate self-subsistence:
the existent immediacy and the reflected immediacy
or the self-identical reflection.
In this self-subsistence, however,
it is at the same time a relative content;
it is simply and solely as a reflection into its other,
or as unity of the reference with its other.
In this unity, the self-subsistent content is
something posited, sublated;
but precisely this unity is what constitutes
its essentiality and self-subsistence;
this reflection into an other is reflection into itself.
The relation has sides, since it is reflection into an other;
so its difference is internal to it,
and its sides are independent subsistence,
for in their mutually indifferent diversity
they are thrown back into themselves,
so that the subsistence of each equally has its meaning
only in its reference to the other
or in the negative unity of both.

The essential relation is therefore not yet
the true third to essence and to concrete existence
but already contains the determinate union of the two.
Essence is realized in it in such a way that
it has self-subsistent, concrete existents for its subsistence,
and these concrete existents have returned
from their indifference back into their essential unity
so that they have only this unity as their subsistence.
Also the reflective determinations
of positive and negative are
reflected into themselves only as
each is reflected into its opposite;
but they have no other determination
besides this their negative unity,
whereas the essential relation has sides
that are posited as self-subsistent totalities.
It is the same opposition as that of positive and negative,
but it is such as an inverted world.
The side of the essential relation is a totality
which, however, essentially has an opposite or a beyond;
it is only appearance;
its concrete existence,
rather than being its own,
is that of its other.
It is, therefore, something internally fractured;
but this, its sublated being, consists in
its being the unity of itself and its other,
therefore a whole, and precisely for this reason
it has self-subsistent concrete existence
and is essential reflection into itself.

This is the concept of relation.
At first, however, the identity it contains
is not yet perfect;
the totality which each relative is as relative,
is only an inner one;
the side of the relation is posited at first
in one of the determinations of negative unity;
what constitutes the form of the relation is
the specific self-subsistence of each of the two sides.
The identity of the form is therefore only a reference,
and the self-subsistence of the sides falls outside it,
that is to say, it falls in the sides;
we still do not have the reflected unity
of the identity of the relation
and of the self-subsistent concrete existents;
we still do not have substance.
It follows that the concept of relation has
indeed shown itself to be the unity
of reflected and immediate self-subsistence.
But it is this concept still immediately at first;
immediate are therefore its moments vis-à-vis each other,
and immediate is the unity of the reference
connecting them essentially;
a unity this, which only then is the true unity
that conforms to the concept,
when it has realized itself, that is to say,
through its movement has posited itself as this unity.

The essential relation is therefore immediately
the relation of the whole and the parts
the reference of reflected and immediate self-subsistence,
so that both are at the same time
mutually conditioning and presupposing.

In this relation, neither of the sides is
yet posited as moment of the other;
their identity is therefore itself one side,
or not their negative unity.
Hence, secondly, the relation passes over into one
in which one side is the moment of the other
and is present there as in its ground,
the true self-subsistent element of both.
This is the relation of force and its expression.

Third, the inequality still present
in this reference sublates itself,
and the final relation is that of inner and outer.
In this difference,
which has now become totally formal,
relation itself founders,
and substance or actuality come on the stage
as the absolute unity of
immediate and reflected concrete existence.

IV.20
citta-antara-drsye buddhi-buddher atiprasanga smrti-sankara ca

A. THE RELATION OF WHOLE AND PARTS

First, the essential relation contains the self-subsistence
of concrete existence reflected into itself;
it is then the simple form whose determinations are
indeed also concrete existences,
but they are posited at the same time,
moments held in the unity.
This self-subsistence reflected into itself is
at the same time reflection into its opposite,
namely the immediate self-subsistence,
and its subsistence is this identity with its opposite
no less than its own self-subsistence.
Second, the other side is
thereby also immediately posited.
This is the immediate self-subsistence
which, determined as the other,
is in itself a multifarious manifold,
but in such a way that this manifold also essentially
has within it the reference of the other side,
the unity of the reflected self-subsistence.
That one side, the whole, is the self-subsistence
that constitutes the world existing in and for itself;
the other side, the parts,
is the immediate concrete existence
which was the world of appearance.
In the relation of whole and parts,
the two sides are these self-subsistences
but in such a way that each has the other
reflectively shining in it
and, at the same time, only is as the identity of both.
Now because the essential relation is
at first only the first, immediate relation,
the negative unity and the positive self-subsistence
are bound together by the “also”;
the two sides are indeed both posited as moments,
but equally so as concretely existing self-subsistences.
Their being posited as moments is henceforth so distributed
that the whole, the reflected self-subsistence, is
as concrete self-existent first,
and the other, the immediate, is in it as a moment.
The whole constitutes here the
unity of the two sides, the substrate,
and the immediate concrete existence is as positedness.
Conversely, on the other side which is the side of the parts,
the immediate and internally manifold concrete existence is
the self-subsistent substrate;
the reflected unity, the whole,
is on the contrary only external reference.

2. This relation thus contains
the self-subsistence of the sides,
and their sublatedness no less,
and the two simply in one reference.
The whole is the self-subsistent;
the parts are only moments of this unity,
but they are also equally self-subsistent
and their reflected unity is only a moment;
and each is, in its self-subsistence,
simply the relative of an other.
This relation is within it, therefore,
immediate contradiction, and it sublates itself.

On closer inspection,
the whole is the reflected unity
that stands independently on its own;
but this subsistence that belongs to it
is equally repelled by it;
it is thus self-externalized;
it has its subsistence in its opposite,
in the manifold immediacy, the parts.
The whole thus consists of the parts,
and apart from them it is not anything.
It is therefore the whole relation
and the self-subsistent totality,
but, for precisely this reason,
it is only a relative,
for what makes it a totality is
rather its other, the parts;
it does not have its subsistence
within it but in its other.

The parts, too, are likewise the whole relation.
They are the immediate as against
the reflected self-subsistence,
and do not subsist in the whole
but are for themselves.
Further, they have this whole within them as their moment;
the whole constitutes their connecting reference;
without the whole there are no parts.
But because they are the self-subsistent,
this connection is only an external moment
with respect to which they are
in and for themselves indifferent.
But at the same time the parts,
as manifold concrete existence, collapse together,
for this concrete existence is reflectionless being;
they have their self-subsistence
only in the reflected unity
which is this unity as well as
the concrete existent manifoldness;
this means that they have
self-subsistence only in the whole,
but this whole is at the same time the
self-subsistence which is the other to the parts.

The whole and the parts thus
reciprocally condition each other;
but the relation here considered is
at the same time higher than the
reference of conditioned and condition
to each other as earlier determined.
Here this reference is realized, that is to say,
it is posited that the condition is
the essential self-subsistence of the conditioned
in such a manner that it is presupposed by the latter.
The condition as such is only the immediate,
and it is only implicitly presupposed.
But the whole, through the condition of the parts,
itself immediately entails that it, too,
is only in so far as it has
the parts for presupposition.
Thus, since both sides of the relation
are posited as conditioning each other reciprocally,
each is on its own an immediate self-subsistence,
but their self-subsistence is equally
mediated or posited through the other.
The whole relation, because of this reciprocity,
is the turning back of the conditioning into itself,
the non-relative, the unconditioned.

Now inasmuch as each side of the relation has
its self-subsistence not in it but in its other,
what we have is only one identity of the two
in which they are both only moments;
but inasmuch as each is self-subsistent on its own,
the two are two self-subsistent concrete existences
indifferent to each other.

In the first respect, that of
the essential identity of the two sides,
the whole is equal to the parts
and the parts are equal to the whole.
Nothing is in the whole which is not in the parts,
and nothing is in the parts which is not in the whole.
The whole is not an abstract unity
but the unity of a diversified manifoldness;
but this unity within which the manifold is
held together is the determinateness
by virtue of which the latter is the parts.
The relation has, therefore, an indivisible identity
and only one self-subsistence.

But further, the whole is equal to the parts
but not to them as parts;
the whole is the reflected unity
whereas the parts constitute
the determinate moment
or the otherness of the unity
and are the diversified manifold.
The whole is not equal to them
as this self-subsistent diversity
but to them together.
But this, their “together,” is
nothing else but their unity,
the whole as such.
In the parts, therefore, the whole
is only equal to itself,
and the equality of it and the parts expresses
only this tautology,
namely that the whole as whole is equal
not to the parts but to the whole.

Conversely, the parts are equal to the whole;
but because, as parts,
they are the moment of otherness,
they are not equal to it as the unity,
but in such a way that one of the whole's
manifold determinations maps over a part,
or that they are equal to the whole as manifold,
and this is to say that they are equal to it
as an apportioned whole, that is, as parts.
Here we thus have the same tautology,
that the parts as parts are equal
not to the whole as such
but, in the whole, to themselves.

The whole and the parts thus
fall indifferently apart;
each side refers only to itself.
But, as so held apart, they destroy themselves.
The whole which is indifferent towards the parts is
abstract identity, undifferentiated in itself.
Identity is a whole only inasmuch
as it is differentiated in itself,
so differentiated indeed that the manifold
determinations are reflected into themselves
and have immediate self-subsistence.
And the identity of reflection has
shown through its movement
that it has this reflection
into its other for its truth.
In just the same way are the parts,
as indifferent to the unity of the whole,
only the unconnected manifold,
the inherently other which, as such,
is the other of itself and only sublates itself.
This self-reference of each of the two sides
is their self-subsistence;
but this self-subsistence which
each side has for itself is rather
the negation of their respective selves.
Each side has its self-subsistence, therefore,
not within but in the other side;
this other, which constitutes the subsistence,
is its presupposed immediate which is supposed
to be the first and its starting point;
but this first of each side is itself
only a first which is not first
but has its beginning in its other.

The truth of the relation consists
therefore in the mediation;
its essence is the negative unity
in which both the reflected
and the existent immediacy
are equally sublated.
The relation is the contradiction
that returns to its ground,
into the unity which,
as turning back,
is reflected unity
but which, since it has
equally posited itself as sublated,
refers to itself negatively
and makes itself into existent immediacy.
But this unity's negative reference,
in so far as it is a first and an immediate,
only is as mediated by its other
and equally as posited.
This other, the existent immediacy,
is equally only as sublated;
its self-subsistence is a first,
but only in order to disappear,
and it has an existence
which is posited and mediated.

Determined in this way, the relation is
no longer one of whole and parts.
The previous immediacy of its sides has
passed over into positedness and mediation.
Each side is posited, in so far as it is immediate,
as self-sublating and as passing over into the other;
and, in so far as it is itself negative reference,
it is at the same time posited as conditioned
through the other, as through its positive.
And the same applies to the immediate transition of each;
it is equally a mediation, a sublating
which is posited through the other.
Thus the relation of whole and parts has passed over
into the relation of force and its expressions.

B. THE RELATION OF FORCE AND ITS EXPRESSION

Force is the negative unity into which
the contradiction of whole and parts has resolved itself;
it is the truth of that first relation.
That of whole and parts is the thoughtless relation
which the understanding first happens to come up with;
or, objectively speaking, it is a dead mechanical aggregate
that indeed has form determinations
and brings the manifoldness of
its self-subsisting matter together into one unity;
but this unity is external to the manifoldness.
But the relation of force is the higher immanent turning back
in which the unity of the whole that made up the connection
of the self-subsisting otherness ceases to be something external
and indifferent to this manifoldness.

In the essential relation as now determined,
the immediate and the reflected self-subsistence are
now posited in that manifoldness as sublated or as moments,
whereas in the preceding relation they were self-subsisting
sides or extremes.

In this there is contained,
first, that the reflected unity
and its immediate existence,
in so far as they are both first and immediate,
sublate themselves and pass over into their other:
the former, force, passes over into its expression,
and what is expressed is a disappearing something
that returns into force as its ground
and only exists as supported and posited by it.

Second, this transition is not only
a becoming and a disappearing
but is rather negative reference to itself;
that is, that which alters its determination is
in this altering reflected-into-itself and preserves itself;
the movement of force is not
as much a transition as a translation,
and in this alteration posited through itself
it remains what it is.

Third, this reflected, self-referring unity is
itself also sublated and a moment;
it is mediated through its other
and it has this as condition;
its negative self-reference,
which is a first
and begins the movement of
the transition out of itself,
has equally a presupposition
by which it is solicited,
and an other from which it begins.

a. The conditionedness of force

Considered in its closer determinations,
force contains, first, the moment of existing immediacy;
it itself is determined over against this immediacy
as negative unity.
But this unity,
in the determination of immediate being,
is an existing something.
This something appears as a first,
since as an immediate
it is negative unity;
force, on the contrary,
since it is a reflected something,
appears as positedness
and to this extent as pertaining to
the existing thing or to a matter.
Not that force is the form of this thing
and the thing is determined by it;
on the contrary, the thing is as
an immediate indifferent to it.
As so determined, there is no ground in
the thing for having a force;
force, on the other hand,
since it is the side of positedness,
presupposes the thing essentially.
If it is therefore asked,
how the thing or matter happens to have a force,
the latter appears as externally connected to it
and impressed upon the thing by some alien power.

As this immediate subsistence,
force is a quiescent determinateness of
the thing in general;
not anything that expresses itself
but something immediately external.
Hence force is also designated as matter,
and instead of a magnetic force,
and an electric force,
and other such forces,
a magnetic matter,
an electric matter,
and so on, are assumed;
or again, instead of
the renowned force of attraction,
a fine ether is assumed
that holds everything together.
These are the matters,
which we considered above,
into which the inert, powerless
negative unity of the thing dissolved itself.
But force contains immediate
concrete existence as a moment,
one which, though a condition,
is transient and self-sublating;
it contains it, therefore,
not as a concretely existing thing.
Further, it is not negation as determinateness,
but negative unity reflected into itself.
Consequently, the thing where the force was
supposed to be no longer has any significance here;
the force itself is rather
the positing of the externality
that appears as concrete existence.
It also no longer is, therefore,
merely a determinate matter;
such self-subsistence has long since passed over
into positedness and appearance.

Second, force is the unity of
reflected and immediate subsistence,
or of form-unity and external self-subsistence.
It is both in one;
it is the contact of sides of
which one is in so far as the other is not,
self-identical positive reflection and negated reflection.
Force is thus self-repelling contradiction;
it is active;
or it is self-referring negative unity
in which the reflected immediacy
or the essential in-itselfness is
posited as being only as sublated or as a moment,
and consequently, in so far as it distinguishes itself
from immediate concrete existence,
as passing over into it.
Force, as the determination of
the reflected unity of the whole,
is thus posited as  becoming
concretely existent external manifoldness
from out of itself.

But, third, force is activity at first only
in principle and immediately;
it is reflected unity,
and just as essentially the negation of it;
inasmuch as it differs from this unity,
but is only the identity of itself and its negation,
it essentially refers to this identity as
an immediacy external to it
and one which it has as
presupposition and condition.

Now this presupposition is not
a thing standing over against it;
in force any such indifferent
self-subsistence is sublated;
as the condition of force,
the thing is a self-subsistent other to it.
But because it is not a thing,
and the self-subsistent immediacy has
on the contrary attained here the
determination of self-referring negative unity,
the self-subsistent other is itself a force.
The activity of force is conditioned
through itself as through an other to itself,
through a force.

Accordingly, force is a relation
in which each side is the same as the other.
They are forces that stand in relation,
and refer to each other essentially.
Further, they are different at first only in general;
the unity of their relation
is at first one which is internal and exists only implicitly.
The conditionedness of a force through another force is
thus the doing of the force itself in itself;
that is, the force is at first a positing act as pre-supposing,
an act that only negatively refers to itself;
the other force still lies beyond its positing activity,
namely the reflection that in its determining
immediately returns into itself.

b. The solicitation of force

Force is conditioned because
the moment of immediate concrete existence
which it contains is something only posited,
but, because it is at the same time an immediate,
is posited as something presupposed in which
the force negates itself.
Accordingly, the externality
which is present to force is its
own activity of presupposing posited
at first as another force.
This presupposing is moreover reciprocal.
Each of the two forces contains the unity
reflected into itself as sublated
and is therefore a presupposing;
it posits itself as external;
this moment of externality is its own;
but since it is equally a unity reflected into itself,
it posits that externality at the same time
not within itself but as another force.

But the external as such is self-sublating;
further, the activity that reflects
itself into itself essentially refers to
that externality as to its other,
but equally to it as to something
which is null in itself and identical with it.
Since the presupposing activity is
equally immanent reflection,
it sublates that external negation,
and posits it as something external to it,
or as its externality.
Thus force, as conditioning,
is reciprocally a stimulus for the other force
against which it is active.
The attitude of each force is not one
of passive determination,
as if something other than it were
thereby being elicited in it;
the stimulus rather only solicits it.
The force is within it the negativity of itself,
the repelling of itself from itself is its own positing.
Its act, therefore, consists in sublating
the externality of the stimulus,
reducing it to just a stimulus
and positing it as its own repelling
of itself from itself,
as its own expression.

The force that expresses itself is thus
the same as what was at first a presupposing activity,
that is, one which makes itself external;
but, as self-expressive,
force also negates externality
and posits it as its own activity.
Now in so far as in this examination
we start from force as the negative unity of itself,
and consequently as presupposing reflection,
this is the same as when,
in the expression of force,
we start from the soliciting stimulus.
Thus force is in its concept at first
determined as self-sublating identity,
and in its reality one of the two forces
is determined as soliciting
and the other as being solicited.
But the concept of force is as such
the identity of positing and presupposing reflection,
or of reflected and immediate unity,
and each of these determinations is simply a moment, in unity,
and consequently is as mediated through the other.
But, equally so, there is nothing in the two forces
thus alternately referring to each other that determines
which would be the soliciting and which the solicited,
or rather, both of these form determinations
belong to each in equal manner.
And this identity is not just one of
external comparison but an essential unity of the two.
Thus one force is determined first as soliciting
and the other as being solicited;
these determinations of form appear in this guise as two differences
present in the forces immediately.
But they are essentially mediated.
The one force is solicited;
this stimulus is a determination posited in it from outside.
But the force is itself a presupposing;
it essentially reflects into itself
and sublates the fact that
the stimulus is something external.
That it is solicited is thus its own doing,
or, it is through its own determining
that the other force is an other force in general
and the one soliciting.
The soliciting force refers to the other negatively
and so sublates its externality and is positing;
but it is this positing only on the presupposition
that it has an other over against it;
that is to say, it is itself soliciting only to
the extent that it has an externality in it,
and hence to the extent that it is solicited.
Or it is soliciting only to the extent that
it is solicited to be soliciting.
And so, conversely, the first is solicited only
to the extent that it itself solicits
the other to solicit it, that is, the first force.
Each thus receives the stimulus from the other;
but the stimulus that each delivers as
active consists in receiving
a stimulus from the other;
the stimulus which it receives is solicited by itself.
Both, the given and the received stimulus,
or the active expression and the passive externality,
are each, therefore, nothing immediate but are mediated:
indeed, each force is itself the determinateness
which the other has over against it,
is mediated through this other,
and this mediating other is again
its own determining positing.

This then that a force happens to incur
a stimulus through another force;
that it therefore behaves passively
but then again passes over from
this passivity into activity,
this is the turning back of force into itself.
Force expresses itself.
The external expression is a reaction
in the sense that it posits
the externality as its own moment
and thus sublates its having been
solicited through an other force.
The two are therefore one:
the expression of the force
by virtue of which the latter,
through its negative activity
which is directed at itself,
imparts a determinate being-for-other to itself;
and the infinite turning in
this externality back to itself,
so that there it only refers to itself.
The presupposing reflection,
to which belong the conditionedness and the stimulus,
is therefore immediately also the reflection
that returns into itself,
and the activity is essentially reactive,
against itself.
The positing of the stimulus
or the external is itself
the sublation of it,
and, conversely, the sublation of the stimulus is
the positing of the externality.

c. The infinity of force

Force is finite inasmuch as its moments
still have the form of immediacy.
In this determination its presupposing
and its self-referring reflection are different:
the one appears as an external self-subsisting force
and the other as passively referring to it.
Force is thus still conditioned according to form,
and according to content likewise still restricted,
for a determinateness of form still entails a restriction of content.
But the activity of force consists in expressing itself;
that is, as we have seen, in sublating the externality and
determining it as that in which
it is identical with itself.
What force truly expresses, therefore,
is that its reference to an other
is its reference to itself;
that its passivity consists in its activity.
The stimulus by virtue of which
it is solicited to activity is its own soliciting;
the externality that comes to it is nothing immediate
but something mediated by it,
just as its own essential
self-identity is not immediate
but is mediated by virtue of its negation.
In brief, force expresses this,
that its externality is identical with its inwardness.

C. RELATION OF OUTER AND INNER

1. The relation of whole and parts is the immediate relation;
in it, therefore, reflected and existent immediacy have
a self-subsistence of their own.
But now, since they stand in essential relation,
their self-subsistence is their negative unity,
and this is now posited in the expression of force;
the reflected unity is essentially a becoming-other,
the unity's translation of itself into externality;
but this externality is just as immediately
taken back into that unity;
the difference of the self-subsisting forces sublates itself;
the expression of force is only a mediation
of the reflected unity with itself.
What is present is only an empty
and transparent difference, a reflective shine,
but this shine is the mediation
which is precisely the independent subsistence.
What we have is not just opposite determinations
openly sublating themselves,
and their movement is not only a transition;
rather, what we have is both that
the immediacy from which the start
and the transition into otherness
were made is itself only posited,
and that, consequently, each of the determinations is
already in its immediacy the unity with its other,
so that the transition equally is
a self-positing turning back into itself.

The inner is determined as
the form of reflected immediacy
or of essence over against
the outer as the form of being;
the two, however, are only one identity.
This identity is, first,
the sustaining unity of the two
as substrate replete of content,
or the absolute fact with respect to which
the two determinations are indifferent, external moments.
To this extent, it is content and totality,
a totality which is an inner
that has equally become an outer
but, in this outer, is not something-that-has-become
or something-that-has-been-left-behind but is self-equal.
The outer, in this determination, is not only
equal to the inner according to content
but the two are rather only one fact.
But this fact, as simple identity with itself,
is different from its form determinations,
or these determinations are external to it;
it is itself, therefore, an inner
which is different from its externality.
But this externality consists in the two determinations,
the inner and the outer, both constituting it.
But the fact is itself nothing other
than the unity of the two.
Again, therefore, the two sides are
the same according to content.
But in the fact they are as self-penetrating identity,
as substrate full of content.
But in the externality, as forms of the fact,
they are indifferent to that identity
and consequently each is indifferent to the other.

2. They are in this wise the different form determinations
that have an identical substrate, not in them but in an other.
These are determinations of reflection
which are each for itself:
the inner, as the form of immanent reflection,
the form of essentiality;
the outer, as the form instead of
immediacy reflected into an other,
or the form of unessentiality.
But the nature of relation has shown that
these determinations constitute just one identity alone.
In its expression force is a determining
which is one and the same as presupposing
and as returning into itself.
Inasmuch as the inner and the outer are
considered as determinations of form,
they are, therefore,
first, only the simple form itself,
and, second, because in this form they are
at the same time determined as opposite,
their unity is the pure abstract determination
in which the one is immediately the other,
and is this other because it is the one that it is.
Thus the inner is immediately only the outer,
and it is this determinateness of externality
for the reason that it is the inner;
conversely, the outer is only an inner
because it is only an outer.
In other words, since the unity of form
holds its two determinations as opposites,
their identity is only this transition,
and is in this transition only the other of both,
not their identity replete with content.
Or this holding fast to form is in general
the side of determinateness.
What is determined according to this side is
not the real totality of the whole
but the totality or the fact itself
only in the determinacy of form;
since this unity is simply the coincidence
of two opposed determinations,
then when one of them is taken first
(it is indifferent which),
it must be said of the substrate or the fact
that it is for this reason just as essentially
in the other determinateness,
but also only in the other,
just as it was first said
that it is only in the first.

Thus something which is at first only an inner,
is for just that reason only an outer.
Or conversely something which is only an outer,
is for that reason only an inner.
Or if the inner is determined as essence
but the outer as being,
then inasmuch as a fact is only in its essence,
it is for that very reason only an immediate being;
or a fact which only is, is for that very reason
as yet only in its essence.
Outer and inner are determinateness
so posited that each, as a determination,
not only presupposes the other
and passes over into it as its truth,
but, in being this truth of the other,
remains posited as determinateness
and points to the totality of both.
The inner is thus the completion
of essence according to form.
For in being determined as inner,
essence implies that it is deficient
and that it is only with reference to
its other, the outer;
but this other is not just being,
or even concrete existence,
but is the reference to essence or the inner.
What we have here is not just
the reference of the two to each other,
but the determining element of absolute form,
namely that each term is immediately its opposite,
and each is their common reference
to a third or rather to their unity.
Their mediation, however, still misses
this identical substrate that contains them both;
their reference is for this reason
the immediate conversion of the one into the other,
and this negative unity tying them together is
the simple point empty of content.

3. The first of the identities considered,
the identity of inner and outer,
is the substrate which is indifferent
to the difference of these determinations
as to a form external to it,
or the identity is as content.
The second is the unmediated identity of their difference,
the immediate conversion of each into its opposite,
or it is inner and outer as pure form.
But both these identities are only
the sides of one totality,
or the totality itself is only the
conversion of the one identity into the other.
The totality, as substrate and content,
is this immediacy reflected into itself
only through the presupposing reflection
of form that sublates their difference
and posits itself as indifferent identity,
as reflected unity over against it.
Or again, the content is the form itself
in so far as the latter determines itself as difference
and makes itself into one side of this difference as externality,
but into the other side as an immediacy
which is reflected into itself,
or into an inner.

It follows that, conversely,
the differences of form,
the inner and the outer,
are each posited as the totality
within it of itself and its other;
the inner, as simple identity
reflected into itself, is immediacy
and hence, no less than essence,
being and externality;
and the external,
as the manifold and determined being,
is only external,
that is, is posited as unessential
and as having returned into its ground,
therefore as inner.
This transition of each into the other is
their immediate identity, as substrate,
but also their mediated identity, that is,
each is what it is in itself,
the totality of the relation,
precisely through its other.
Or, conversely, the determinateness of
either side is mediated through
the determinateness of the other
because each is in itself the totality;
the totality thus mediates itself with itself
through the form or the determinateness,
and the determinateness mediates
itself with itself through its simple identity.

Therefore, what something is, that it is
entirely in its externality;
its externality is its totality
and equally so its unity reflected into itself.
Its appearance is not only reflection-into-other
but immanent reflection,
and its externality is therefore
the expression of what it is in itself;
and since its content and its form
are thus absolutely identical,
it is, in and for itself, nothing but this:
to express itself.
It is the revealing of its essence,
and this essence, accordingly,
consists simply in being self-revealing.

The essential relation, in this identity
of appearance with the inner or with essence,
has determined itself as actuality.

SECTION III

Actuality

Actuality is the unity of essence and concrete existence;
in it, shapeless essence and unstable appearance
(subsistence without determination
and manifoldness without permanence)
have their truth.
Although concrete existence is the immediacy
that has proceeded from ground,
it still does not have form explicitly posited in it;
inasmuch as it determines and informs itself, it is appearance;
and in developing this subsistence that otherwise only is
a reflection-into-other into an immanent reflection,
it becomes two worlds, two totalities of content,
one determined as reflected into itself
and the other as reflected into other.
But the essential relation exposes
the formality of their connection,
and the consummation of the latter is
the relation of the inner and the outer
in which the content of both is equally
only one identical substrate
and only one identity of form.
This identity has come about also in regard to form,
the form determination of their difference is sublated,
and that they are one absolute totality is posited.

This unity of the inner and outer is absolute actuality.
But this actuality is, first, the absolute as such
(in so far as it is posited as a unity
in which the form has sublated itself)
making itself into the empty or external
distinction of an outer and inner.
Reflection relates to this absolute
as external to it;
it only contemplates it
rather than being its own movement.
But it is essentially this movement
and is, therefore, as the absolute's
negative turning back into itself.

Second, it is actuality proper.
Actuality, possibility, and necessity constitute
the formal moments of the absolute,
or its reflection.

Third, the unity of the absolute
and its reflection is
the absolute relation,
or rather the absolute as
relation to itself, substance.

CHAPTER 1

The absolute

The simple solid identity of the absolute is indeterminate,
or rather, every determinateness of essence and concrete existence,
or of being in general as well as of reflection,
has dissolved itself into it.
Accordingly, the determining of what is
the absolute appears to be a negating,
and the absolute itself appears only as
the negation of all predicates, as the void.
But since it must equally be spoken of
as the position of all predicates,
it appears as the most formal of contradictions.
In so far as that negating and this positing
belong to external reflection,
what we have is a formal, unsystematic dialectic
that has an easy time picking up
a variety of determinations here and there,
and is just as at ease demonstrating, on the one hand,
their finitude and relativity, as declaring, on the other,
that the absolute, which it vaguely envisages as totality,
is the dwelling place of all determinations,
yet is incapable of raising
either the positions or the negations
to a true unity.
The task is indeed to demonstrate what the absolute is.
But this demonstration cannot be either
a determining or an external reflection
by virtue of which determinations
of the absolute would result,
but is rather the exposition of the absolute,
more precisely the absolute's own exposition,
and only a displaying of what it is.

IV.21
citer apratisankramaya tad-akara-apattau svabuddhi-samvedanam

A. THE EXPOSITION OF THE ABSOLUTE

The absolute is not just being, nor even essence.
The former is the first unreflected immediacy;
the latter, the reflected immediacy;
further, each is explicitly a totality,
but a determinate totality.
Being emerges in essence as concrete existence,
and the connection of being and essence develops
into the relation of inner and outer.
The inner is essence, but as a totality
whose essential determination is
to be referred to being and to be being immediately.
The outer is being, but with the essential determination of
being immediately connected with reflection
and, equally, in a relationless identity with essence.
The absolute itself is the absolute unity of the two;
it is that which constitutes in general
the ground of the essential relation
which, as only relation, has yet
to return into this its identity
and whose ground is not yet posited.

It follows that the determination of
the absolute is to be absolute form,
but at the same time not as an identity
whose moments only are simple determinacies,
but, on the contrary, as an identity
whose moments are each explicitly the totality
and hence, indifferent with respect to the form,
the complete content of the whole.
But, conversely, the absolute is absolute content
in such a way that this content,
which is as such indifferent plurality,
explicitly has the negative connection of form
by virtue of which its manifold is
only one substantial identity.

Thus the identity of the absolute is
for this reason absolute identity,
because each of its parts is itself the whole
or each determinateness is the totality, that is,
because determinateness has become as such
a thoroughly transparent reflective shine,
a difference that has disappeared in its positedness.
Essence, concrete existence, the world existing in itself,
whole, parts, force:
these reflected determinations appear to representation
as true being valid in and for itself;
but against them the absolute is the ground
into which they have foundered.
Because in the absolute the form is
now only simple self-identity,
the absolute does not determine itself,
for the determination is a difference of form
which is valid as such from the start.
But because the absolute at the same time contains
every difference and form determination in general,
or because it is itself absolute form and reflection,
the difference of content must also come into it.
But the absolute itself is the absolute identity;
to be this identity is its determination,
for the manifoldness of the world-in-itself
and of the phenomenal world has all been sublated in it.
In the absolute itself there is no becoming,
since the absolute is not being;
nor does the absolute determine itself reflectively,
for it is not the essence which determines itself only inwardly;
and it also does not externalize itself,
for it is the identity of inner and outer.
But in this way the movement of reflection
stands over against its absolute identity.
The movement is sublated in this identity
and is thus only its inner;
but consequently its outer.

At first, therefore, the movement consists only
in sublating its act in the absolute.
It is the beyond of the manifold differences
and determinations and of their movement,
a beyond that lies at the back of the absolute.
It is thus the negative exposition of the absolute
earlier alluded to.
In its true presentation, this exposition is
the preceding whole of the logical movement
of the spheres of being and essence,
the content of which has not been gathered in
from outside as something given and contingent;
nor has it been sunk into the abyss of the absolute
by a reflection external to it;
on the contrary, it has determined itself within it
by virtue of its inner necessity,
and, as being's own becoming
and as the reflection of essence,
has returned into the absolute
as into its ground.

But this exposition has itself also a positive side,
for in foundering to the ground the finite demonstrates
that its nature is to be referred to the absolute,
or to contain the absolute within.
However, this side is not as much
the positive exposition of the absolute
as it is rather the exposition of the determinations,
namely that these have the absolute for their abyss,
but also for their ground,
or that that which imparts subsistence to them,
to their reflective shine, is the absolute itself.
Being as shine is not nothing but reflection,
reference to the absolute;
or it is a shine inasmuch as
that which shines in it is the absolute.
This positive exposition thus halts the finite
just before its disappearing:
it considers it an expression and
a copy of the absolute.
But this transparency of the finite
that lets only the absolute transpire through it
ends up in complete disappearance,
for there is nothing in the finite
which would retain for it a difference
over against the absolute;
as a medium, it is absorbed by
that through which it shines.

This positive exposition of the absolute is
therefore itself only a reflective shine,
for the true positive, that which contains
the exposition and the expounded content,
is the absolute itself.
Whatever the further determinations that may occur,
the form in which the absolute reflectively shines is
a nullity which the exposition gathers up from outside
and in which it gains for itself
a starting point for its activity.
Any such determination has in the absolute,
not its beginning but its end.
This expository process, therefore,
though it is an absolute act
because of its reference to
the absolute into which it returns,
is not so at its starting point
which is a determination
external to the absolute.

But in actual fact
the exposition of the absolute
is the absolute's own doing,
an act that begins from itself
and arrives at itself.
The absolute, only as absolute identity,
is absolute in a determined guise,
that is, as identical absolute;
it is posited as such by reflection
over against opposition and manifoldness;
or it is only the negative of
reflection and determination in general.
It is not just the exposition of the absolute
which is therefore something incomplete,
but this absolute itself
which is only arrived at.
Or again, the absolute
which is only as absolute identity
is only the absolute of an external reflection.
It is, therefore, not the absolutely absolute
but the absolute in a determination,
or it is attribute.

But the absolute is not attribute just because
it is the subject matter of an external reflection
and is consequently something determined by it.
Or, reflection is not only external to it;
but, precisely because it is external to it,
it is immediately internal to it.
The absolute is absolute only because
it is not abstract identity
but is the identity of being and essence,
or the identity of the inner and the outer.
It is therefore itself the absolute form
that makes it reflectively shine within itself
and determines it as attribute.

IV.22
drastr-drsya-uparaktam cittam sarva-artham

B. THE ABSOLUTE ATTRIBUTE

The expression which we have used, “the absolute absolute,”
denotes the absolute which in its form
has returned back into itself
or whose form is equal to its content.
The attribute is just the relative absolute,
a combination which only signifies the absolute
in a form determination.
For at first, before its complete exposition,
the form is only internally
or, which is the same, only externally;
it is at first determinate form in general
or negation in general.
But because form is at the same time
as the form of the absolute,
the attribute is the whole content of the absolute;
it is the totality which earlier appeared as a world,
or as one of the sides of the essential relation,
each of which is itself the whole.
But both worlds, the phenomenal world
and the world that exists in and for itself,
were supposed to be opposed
to each other in their essence.
Each side of the essential relation was
indeed equal to the other:
the whole as much as the parts,
the expression of force the same content
as force itself,
and the outer everywhere the same as the inner.
But these sides were at the same time
supposed each to have still
an immediate subsistence of its own,
the one side as existent immediacy
and the other as reflected immediacy.
In the absolute, on the contrary,
these different immediacies have been
reduced to a reflective shine,
and the totality that the attribute is
is posited as its true and single subsistence,
while the determination in which it is
is posited as unessential subsistence.

The absolute is attribute because,
as simple absolute identity,
it is in the determination of identity;
now to the determination as such
other determinations can be attached,
for instance, also that there are several attributes.
But because absolute identity has only this meaning,
that not only all determinations have been sublated
but that reflection itself has also sublated itself,
all determinations are thus posited in it as sublated.
Or the totality is posited as absolute totality.
Or again, the attribute has the absolute
for its content and subsistence
and, consequently, its form determination
by which it is attribute is also posited,
posited immediately as mere reflective shine;
the negative is posited as negative.
The positive reflective shine that
the exposition gives itself through the attribute
in that it does not take the finite in its limitation as
something that exists in and for itself
but dissolves its subsistence into the absolute
and expands it into attribute;
sublates precisely this, that the attribute is attribute;
it sinks it and its differentiating act
into the simple absolute.

But since reflection thus reverts
from its differentiating act
only to the identity of the absolute,
it has not at the same time
left its externality behind
and has not arrived at the true absolute.
It has only reached the
indeterminate, abstract identity,
which is to say, the identity
in the determinateness of identity.
Or, since reflection determines the
absolute into attribute as inner form,
this determining is something
still distinct from externality;
the inner determination does not
penetrate the absolute;
the attribute's expression,
as something merely posited,
is to disappear into the absolute.

The form by virtue of which
the absolute would be attribute,
whether it is taken as outer or inner,
is therefore posited as something null in itself,
an external reflective shine,
or a mere way and manner.

IV.23
tad asamkhyeya-vasanabhi citram api para-artham samhatya-karitvat

C. THE MODE OF THE ABSOLUTE

The attribute is first
the absolute in simple self-identity.
Second, it is negation,
a negation which is as such
formal immanent reflection.
These two sides constitute at first
the two extremes of the attribute,
the middle term of which is the attribute itself,
since it is both the absolute and the determinateness.
The second of these extremes is the negative as negative,
the reflection external to the absolute.
Or inasmuch as the negative is
taken as the inner of the absolute
and its own determination is to posit itself as mode,
it is then the self-externality of the absolute,
the loss of itself in the
changeability and contingency of being,
its having passed over into its opposite
without turning back into itself,
the manifoldness of form and content
determinations that lacks totality.

But the mode, the externality of the absolute, is not just this.
It is rather externality posited as externality,
a mere way and manner,
hence the reflective shine as reflective shine,
or the reflection of form into itself;
hence, the self-identity which is the absolute.
In actual fact, therefore, the absolute is first posited
as absolute identity only in the mode;
it is what it is, namely self-identity,
only as self-referring negativity,
as reflective shining which is posited as reflective shining.

Hence, in so far as the exposition of the absolute
begins from its absolute identity
and passes over to the attribute
and from there to the mode,
it has therein exhaustively run through its moments.

But first, in this course it does not just behave
negatively towards these determinations;
its act is rather the reflective movement itself,
and it is only as such a movement that
the absolute truly is absolute identity.

Second, the exposition does not thereby deal with mere externality,
and the mode is not only the most external externality.
Rather, since the mode is reflective shine as shine,
it is an immanent turning back, the self-dissolving reflection,
and it is in being this reflection that
the absolute is absolute being.

Third, the reflective act of exposition seems to begin
from its own determinations and from something external,
to take up the modes or even the determinations of the attribute
as if they were found outside the absolute
and its contribution were only to reduce them
to undifferentiated identity.
But it has in fact found the determinateness
from which it begins in the absolute itself.
For as first undifferentiated identity,
the absolute is itself only the determinate absolute,
or attribute, because it is the unmoved,
still unreflected absolute.
This determinateness, since it is determinateness,
belongs to the reflective movement,
and it is through this movement alone
that the absolute is determined as the first identity;
through it alone that it has absolute form
and does not just exist as self-equal
but posits itself as self-equal.

Accordingly the true meaning of mode is that
it is the absolute's own reflective movement;
it is a determining by virtue of which
the absolute would become, not an other,
but what it already is;
a transparent externality
which is a pointing to itself;
a movement out of itself,
but in such a way that being outwardly is
just as much inwardness,
and consequently equally a positing
which is not mere positedness
but absolute being.

When therefore one asks for a content of the exposition,
for what the absolute manifests,
the reply is that the distinction of form and content
in the absolute has been dissolved;
or that just this is the content of the absolute,
that it manifests itself.
The absolute is the absolute form
which in its diremption of itself is
utterly identical with itself,
is the negative as negative
or the negative that rejoins itself
and in this way alone is the absolute self-identity
which equally is indifferent towards its distinctions
or is absolute content.
The content is therefore only this exposition itself.

As this self-bearing movement of exposition,
as a way and manner which is
its absolute identity with itself,
the absolute is expression,
not of an inner,
nor over against an other,
but simply as absolute manifestation
of itself for itself.
Thus it is actuality.

CHAPTER 2

Actuality

The absolute is the unity of inner and outer
as a first implicitly existent unit.
The exposition appeared as an external reflection
which, for its part, has the immediate
as something it has found,
but it equally is its movement
and the reference connecting it to the absolute
and, as such, it leads it back to the latter,
determining it as a mere “way and manner.”
But this “way and manner” is the
determination of the absolute itself,
namely its first identity
or its mere implicitly existent unity.
And through this reflection, not only is
that first in-itself posited as essenceless determination,
but, since the reflection is negative self-reference,
it is through it that the in-itself becomes
a mode in the first place.
It is this reflection that,
in sublating itself in its determinations
and as a movement which as such turns back upon itself,
is first truly absolute identity
and, at the same time, the determining of
the absolute or its modality.
The mode, therefore, is the externality of the absolute,
but equally so only its reflection into itself;
or again, it is the absolute's own manifestation,
so that this externalization is its immanent reflection
and therefore its being in-and-for-itself.

So, as the manifestation that it is nothing,
that it has no content, save to be
the manifestation of itself,
the absolute is absolute form.
Actuality is to be taken as
this reflected absoluteness.
Being is not yet actual;
it is the first immediacy;
its reflection is therefore becoming
and transition into an other;
or its immediacy is not being-in-and-for-itself.
Actuality also stands higher than concrete existence.
It is true that the latter is the immediacy
that has proceeded from ground and conditions,
or from essence and its reflection.
In itself or implicitly, it is therefore
what actuality is, real reflection;
but it is still not the posited unity of reflection and immediacy.
Hence concrete existence passes over into appearance
as it develops the reflection contained within it.
It is the ground that has foundered to the ground;
its determination, its vocation, is to restore this ground,
and therefore it becomes essential relation,
and its final reflection is that its
immediacy be posited as immanent reflection and conversely.
This unity, in which concrete existence
or immediacy and the in-itself,
the ground or the reflected, are simply moments,
is now actuality.
The actual is therefore manifestation.
It is not drawn into
the sphere of alteration by its externality,
nor is it the reflective shining of itself in an other.
It just manifests itself,
and this means that in its externality,
and only in it, it is itself, that is to say,
only as a self-differentiating and self-determining movement.

Now in actuality as this absolute form,
the moments only are as sublated or formal, not yet realized;
their differentiation thus belongs at first to external reflection
and is not determined as content.

Actuality, as itself immediate form-unity of inner and outer,
is thus in the determination of immediacy
as against the determination of immanent reflection;
or it is an actuality as against a possibility.
The connection of the two to each other is the third,
the actual determined both as being reflected into itself
and as this being immediately existing.
This third is necessity.

But first, since the actual and the possible
are formal distinctions,
their connection is likewise only formal,
and consists only in this,
that the one just like the other
is a positedness, or in contingency.

Second, because in contingency
the actual as well as the possible
are a positedness,
because they have retained their determination,
real actuality now arises,
and with it also real possibility
and relative necessity.

Third, the reflection of relative necessity
into itself yields absolute necessity,
which is absolute possibility and actuality.

IV.24
visesa-darsina atma-bhava-bhavana-vinivrtti

IV.25
tada viveka-nimnam kaivalya-prag-bharam cittam

IV.26
tad-chidresu pratyaya-antarani samskarebhya

IV.27
hanam esam klesavad uktam

A. CONTINGENCY OR FORMAL ACTUALITY, POSSIBILITY, AND NECESSITY

1. Actuality is formal inasmuch as, as a first actuality,
it is only immediate, unreflected actuality,
and hence is only in this form determination
but not as the totality of form.
And so it is nothing more than a being,
or concrete existence in general.
But because by essence it is not mere concrete existence
but is the form-unity of the in-itselfness
or inwardness and externality,
it immediately contains in-itselfness or possibility.
What is actual is possible.

2. This possibility is actuality reflected into itself.
But this reflectedness, itself a first, is equally something formal
and consequently only the determination of self-identity
or of the in-itself in general.

But because the determination is here totality of form,
this in-itself is determined as sublated
or essentially only with reference to actuality;
as the negative of actuality, it is posited as negative.
Possibility entails, therefore, two moments.
It has first the positive moment
of being a being-reflected-into-itself.
But this being-reflected-into-itself,
since in the absolute form it is reduced to a moment,
no longer has the value of essence but has rather
the negative meaning that possibility is (in a second moment)
something deficient, that it points to an other, to actuality,
and is completed in this other.

According to the first, merely positive side,
possibility is therefore the mere
form determination of self-identity,
or the form of essentiality.
As such it is the relationless, indeterminate
receptacle of everything in general.
In this formal sense of possibility,
everything is possible
that does not contradict itself;
the realm of possibility is therefore
limitless manifoldness.
But every manifold is determined in itself
and as against an other:
it possesses negation within.
Indifferent diversity passes over
as such into opposition;
but opposition is contradiction.
Therefore, all things are
just as much contradictory
and hence impossible.

When we therefore say of something
that “it is possible,”
this purely formal assertion is
just as superficial and empty
as the principle of contradiction,
and any content that we put into it,
“A is possible,” says no more than “A is A.”
Left undeveloped, this content has
the form of simplicity;
only after being resolved
into its determinations,
does difference emerge within it.
To the extent that we stop at that
simple for the content remains
something self-identical
and hence a possible.
But we do not say anything by it,
just as we do not with the principle of identity.

Yet the possible amounts to more
than just the principle of identity.
The possible is reflected immanent reflectedness;
or the identical simply as a moment of the totality,
hence also as determined not to be in itself;
it therefore has the second determination of being only a possible
and the ought-to-be of the totality of form.
Without this ought-to-be, possibility is essentiality as such;
but the absolute form entails this,
that essence itself is only a moment
and that it has no truth without being.
Possibility is this mere essentiality,
but so posited as to be only a moment,
to be disproportionate with respect to the absolute form.
It is the in-itself, determined as only a posited
or, equally, as not to be in itself.

Internally, therefore, possibility is contradiction,
or it is impossibility.

This finds expression at first in this way,
that possibility as form determination
posited as sublated possesses a content in general.
As possible, this content is an in-itself
which is at the same time something sublated
or an otherness.
But because this content is only a possible,
an other opposite to it is equally possible.
“A is A”; then, too, “not-A is not-A.”
These two statements each express
the possibility of its content determination.
But, as identical statements,
they are indifferent to each other;
that the other is also added,
is not posited in either.
Possibility is the connection comparing the two;
as a reflection of the totality,
it implies that the opposite also is possible.
It is therefore the ground for drawing the connection that,
because A equals A, not-A also equals not-A;
entailed in the possible A there is also the possible not-A,
and it is this reference itself connecting them
which determines both as possible.

But this connection, in which
the one possible also contains its other,
is as such a contradiction that sublates itself.
Now, since it is determined to be reflective
and, as we have just seen, reflectively self-sublating,
it is also therefore an immediate
and it consequently becomes actuality.

3. This actuality is not the first actuality
but reflected actuality,
posited as unity of itself and possibility.
What is actual is as such possible;
it is in immediate positive identity with possibility;
but the latter has determined itself as only possibility;
consequently the actual is also determined as only a possible.
And because possibility is immediately contained in actuality,
it is immediately in it as sublated, as only possibility.
Conversely, actuality which is in unity with possibility
is only sublated immediacy;
or again, because formal actuality is only immediate first actuality,
it is only a moment, only sublated actuality, or only possibility.

With this we also have a more precise expression of
the extent to which possibility is actuality.
Possibility is not yet all actuality;
there has been no talk yet of real and absolute actuality.
It is still only the possibility as it first presented itself,
namely the formal possibility that has determined itself
as being only possibility and hence the formless actuality
which is only being or concrete existence in general.
Everything possible has therefore in general
a being or a concrete existence.

This unity of possibility and actuality is contingency.
The contingent is an actual which is at the same time
determined as only possible,
an actual whose other or opposite equally is.
This actuality is, therefore, mere being or concrete existence,
but posited in its truth as having the value
of a positedness or a possibility.
Conversely, possibility is immanent reflection
or the in-itself posited as positedness;
what is possible is an actual in this sense of actuality,
that it has only as much value as contingent actuality;
it is itself something contingent.

The contingent thus presents these two sides.
First, in so far as it has possibility immediately in it,
or, what is the same, in so far as
this possibility is sublated in it,
it is not positedness, nor is it mediated,
but is immediate actuality; it has no ground.
Because this immediate actuality pertains also to the possible,
the latter is determined no less than the actual as contingent
and is likewise groundless.

But, second, the contingent is the actual
as what is only possible, or as a positedness;
thus the possible also, as formal in-itself, is only positedness.
Consequently, the two are both not in and for themselves
but have their immanent reflection in an other,
or they do have a ground.

The contingent thus has no ground because it is contingent;
and for that same reason it has a ground, because it is contingent.

It is the posited, immediate conversion of inner and outer,
or of immanently-reflected-being and being,
each into the other posited,
because possibility and actuality
both have this determination in them
by being moments of the absolute form.
So actuality, in its immediate unity with possibility,
is only concrete existence and is determined as groundless,
something only posited or only possible;
or, as reflected and determined over against possibility,
it is separated from possibility,
from immanent reflectedness,
and then, too, is no less
immediately only a possible.
Likewise possibility, as simple in-itself,
is something immediate,
only an existent in general;
or, opposed to actuality,
it equally is an in-itself
without actuality, only a possible,
but, for that very reason,
again only a concrete,
not immanently reflected,
existence in general.

This absolute restlessness of the becoming of
these two determinations is contingency.
But for this reason, because each determination
immediately turns into the opposite,
in this opposite each equally rejoins itself,
and this identity of the two,
of each in the other,
is necessity.

The necessary is an actual;
as such it is immediate, groundless;
but it equally has its actuality
through an other or in its ground
and is at the same time the positedness of this ground
and its reflection into itself;
the possibility of the necessary is a sublated one.
The contingent is therefore necessary
because the actual is determined as a possible;
its immediacy is consequently sublated
and is repelled into the ground or the in-itself,
and into the grounded, equally because its possibility,
this ground-grounded-connection,
is simply sublated and posited as being.
What is necessary is,
and this existent is itself the necessary.
At the same time it is in itself;
this immanent reflection is an other than that immediacy of being,
and the necessity of the existent is an other.
Thus the existent is not the necessary;
but this in-itself is itself only positedness;
it is sublated and itself immediate.
And so actuality, in that from which it is distinguished,
in possibility, is identical with itself.
As this identity, it is necessity.

IV.28
prasankhyane api-akusidasya sarvatha viveka-khyate dharma-megha samadhi

IV.29
tata klesa-karma-nivrtti

IV.30
tada sarva-avarana-mala-apetasya jnanasya-anantya jneyam alpam

IV.31
tata-krta-arthanam parinama-krama-samapti gunanam

B. RELATIVE NECESSITY OR REAL ACTUALITY, POSSIBILITY, AND NECESSITY

1. The necessity which has resulted is formal
because its moments are formal,
that is, simple determinations which are a totality
only as an immediate unity,
or as an immediate conversion of the one into the other,
and thus lack the shape of self-subsistence.
The unity in this formal necessity is therefore simple at first,
and indifferent to its differences.
As the immediate unity of the form determinations,
this necessity is actuality,
but an actuality which, since its unity is now determined as indifferent
to the difference of the form determinations, has a content.
This content as an indifferent identity contains the form
also as indifferent that is, as a mere variety of determinations,
and is a manifold content in general.
This actuality is real actuality.

Real actuality is as such at first
the thing of many properties,
the concretely existing world;
but it is not the concrete existence
that dissolves into appearance
but, as actuality, it is at the same time
an in-itself and immanent reflection;
it preserves itself in the manifoldness of mere concrete existence;
its externality is an inner relating only to itself.
What is actual can act;
something announces its actuality by what it produces.
Its relating to an other is the manifestation of itself,
and this manifestation is
neither a transition
(the immediate something refers to the other in this way)
nor an appearing
(in this way the thing only is in relation to an other);
it is a self-subsistent which has its immanent reflection,
its determinate essentiality, in another self-subsistent.

Now real actuality likewise has possibility immediately present in it.
It contains the moment of the in-itself;
but, since it is in the first instance only immediate unity,
it is in one of the determinations of form
and hence distinguished, as immediate existent,
from the in-itself or possibility.

2. This possibility, as the in-itself of real actuality,
is itself real possibility, at first the in-itself full of content.
Formal possibility is immanent reflection only as abstract identity,
the absence of contradiction in a something.
But when we delve into the determinations,
the circumstances, the conditions of a fact
in order to discover its possibility,
we do not stop at this formal possibility
but consider its real possibility.

This real possibility is itself immediate concrete existence,
but no longer because possibility as such, as a formal moment,
is immediately its opposite, a non-reflected actuality,
but because this determination pertains to it
by the very fact of being real possibility.
The real possibility of a fact is therefore
the immediately existent manifoldness of
circumstances that refer to it.

This manifoldness of existence is therefore indeed
both possibility and actuality,
but their identity is at first only the content
which is indifferent to these form determinations;
they therefore constitute the form,
determined as against their identity.
Or the immediate real actuality, because it is immediate,
is determined as against its possibility;
as this determinate and hence reflected actuality,
it is real possibility.
This real possibility is now indeed the posited whole of the form,
but of the form in the determinateness of actuality as formal
or immediate and equally of possibility as the abstract in-itself.
This actuality, therefore, which constitutes the possibility of a fact,
is not its own possibility but the in-itself of an other actual;
itself, it is the actuality that ought to be sublated,
the possibility as only possibility.
Real possibility thus constitutes the totality of conditions,
a dispersed actuality which is not reflected into itself
but is determined to be the in-itself of an other
and intended in this determination to return to itself.

What is really possible is, therefore,
something formally identical according to its in-itself,
free of contradiction because of its simple content determination;
but, as self-identical, this something must also not contradict
itself according to its developed and differentiated circumstances
and all else connected with it.
But, secondly, because it is manifold in itself
and in manifold connection with others,
and variety inherently passes over into opposition,
it is contradictory.
Whenever a possibility is in question,
and the issue is to demonstrate its contradiction,
one need only fasten on to the multiplicity that it contains as content
or as its conditioned concrete existence,
and from this the contradiction will easily be discovered.
And this contradiction is not just a function of comparing;
on the contrary, the manifold of concrete existence is in itself this,
to sublate itself and to founder to the ground:
in this it explicitly has the determination of
being only a possibility.
Whenever all the conditions of a fact are completely present,
the fact is actually there;
the completeness of the conditions is
the totality as in the content,
and the fact is itself this content determined
as being equally actual as possible.
In the sphere of the conditioned ground,
the conditions have the form
(that is, the ground or the reflection that stands on its own)
outside them,
and it is this form that makes them moments
of the fact and elicits concrete existence in them.
Here, on the contrary, the immediate actuality is
not determined to be condition by virtue of
a presupposing reflection,
but the supposition is rather that the immediate actuality is
itself the possibility.

In self-sublating real possibility,
it is a twofold that is now sublated;
for this possibility is itself
the twofold of actuality and possibility.
(1) The actuality is formal, or is a concrete existence
which appeared to subsist immediately,
and through its sublating becomes reflected being,
the moment of an other,
and thus comes in possession of the in-itself.
(2) That concrete existence was also determined
as possibility or as the in-itself, but of an other.
As it sublates itself, this in-itself of the other is
also sublated and passes over into actuality.
This movement of self-sublating real possibility
thus produces the same moments that are already present,
but each as it comes to be out of the other;
in this negation, therefore, the possibility
is also not a transition but a self-rejoining.
In formal possibility, if something was possible,
then an other than it, not itself, was also possible.
Real possibility no longer has such an other over against it,
for it is real in so far as it is itself also actuality.
Therefore, as its immediate concrete existence,
the circle of conditions, sublates itself,
it makes itself into the in-itselfness which it already is,
namely the in-itself of an other.
And conversely, since its moment of in-itselfness
thereby sublates itself at the same time,
it becomes actuality, hence the moment
which it likewise already is.
What disappears is consequently this,
that actuality was determined as the possibility
or the in-itself of an other,
and, conversely, the possibility as an actuality
which is not that of which it is the possibility.

3. The negation of real possibility is thus its self-identity;
inasmuch as in its sublating it is thus within itself
the recoiling of this sublating, it is real necessity.

What is necessary cannot be otherwise;
but what is only possible can be,
for possibility is the in-itself
which is only positedness
and hence essentially otherness.
Formal possibility is this identity
as transition into the other as such;
but real possibility, since it has
the other moment of actuality within it,
is already itself necessity.
Hence what is really possible can no longer be otherwise;
under the given conditions and circumstances,
nothing else can follow.
Real possibility and necessity are, therefore,
only apparently distinguished;
theirs is an identity that does not first come to be
but is already presupposed at their base.
Real possibility is therefore a connection full of content,
for the content is that identity, existing in itself,
which is indifferent to form.

But this necessity is at the same time relative.
For it has a presupposition from which it begins;
it takes its start from the contingent.
For the real actual is as such the determinate actual,
and first has its determinateness as immediate being
in that it is a multiplicity of concretely existing circumstances;
but this immediate being as determinateness is also the negative of
itself, is an in-itself or possibility and so real possibility.
As this unity of the two moments, it is the totality of form,
but a totality which is still external to itself;
it is the unity of possibility and actuality in such a way that
(1) the manifold concrete existence is possibility immediately or positively:
it is a possible, something self-identical as such, because it is an actual;
(2) inasmuch as this possibility of concrete existence is posited,
it is determined as only possibility,
as the immediate conversion of actuality into its opposite, or as contingency.
Hence this possibility which immediate actuality has within
in so far as it is condition, is only the in-itself
or the possibility of an other.
Because this in-itself, as shown, sublates itself and this positedness
is itself posited, real possibility becomes indeed necessity;
but this necessity thus begins from that unity of the possible and the actual
which is not yet reflected into itself;
this presupposing and the movement which turns back
unto itself are still separate;
or necessity has not yet determined itself
out of itself into contingency.

The relativity of real possibility is manifested in the content
by the fact that the latter is at first only
the identity indifferent to form,
is therefore distinct from it
and a determinate content in general.
A necessary reality is for this reason any limited actuality
which, because of its limitation, is in some other respect
also only something contingent.

In actual fact, therefore, real necessity is
in itself also contingency.
This first becomes apparent because real necessity,
although something necessary according to form,
is still something limited according to content,
and derives its contingency through the latter.
But this contingency is to be found also
in the form of real necessity because, as shown,
real possibility is the necessary only in itself,
but as posited it is the mutual otherness of
actuality and possibility.
Real necessity thus contains contingency;
it is the turning back into itself from the restless
being-the-other-of-each-other of actuality and possibility,
but not the turning back from itself to itself.
In itself, therefore, we have here the unity
of necessity and contingency;
this unity is to be called absolute actuality.

IV.32
ksana-pratiyogi parinama-aparanta-nirgrahya krama

C. ABSOLUTE NECESSITY

Real necessity is determinate necessity;
formal necessity does not yet have
any content and determinateness in it.
The determinateness of necessity consists in
its having its negation, contingency, within it.
This is how it has shown itself to be.

But in its first simplicity this determinateness is actuality;
determinate necessity is therefore immediate actual necessity.
This actuality which is itself as such necessary,
since it contains necessity as its in-itself,
is absolute actuality;
an actuality which can no longer be otherwise,
or its in-itself is not possibility but necessity itself.

But because this actuality is posited to be absolute,
that is to say, to be itself
the unity of itself and possibility,
it is consequently only an empty determination,
or it is contingency.
This emptiness of its determination makes it
into a mere possibility,
one which can just as well be an other
and is determined as possibility.
But this possibility is itself absolute possibility,
for it is precisely the possibility of being
equally determined as possibility and actuality.
For this reason, because it is this indifference towards itself,
it is posited as empty, contingent determination.

Thus real necessity not only contains contingency implicitly,
but the latter also becomes in it;
but this becoming, as externality,
is itself only the in-itself of the necessity,
because it is only an immediate determinateness.
But it is not only this but the necessity's own becoming
or the presupposition which it had is its own positing.
For as real necessity, it is the sublatedness
of actuality into possibility
and of possibility into actuality;
because it is this simple conversion of
one of these moments into the other,
it is also their positive unity,
for in the other each rejoins itself.
And so it is actuality, yet an actuality
which is nothing but this rejoining of form with itself.
Its negative positing of these moments is
thereby itself the presupposing
or the positing of itself as sublated,
or the positing of immediacy.

But it is precisely in this positing that
this actuality is determined as the negative;
it rejoins itself from the actuality which was real possibility;
this new actuality thus comes to be only out of its in-itself,
out of the negation of itself.
Consequently, it is at the same time
immediately determined as possibility,
as mediated by virtue of its negation.
But accordingly, this possibility is
immediately nothing but this mediating
in which the in-itself,
namely the possibility itself and the mediating,
both in the same manner, are positedness.
Thus it is necessity which is
equally the sublating of this positedness,
or the positing of immediacy and of the in-itself,
just as in this very sublating
it is the determining of it as positedness.
It is necessity itself, therefore,
that determines itself as contingency:
in its being it repels itself from itself,
in this very repelling has only returned to itself,
and in this turning back which is its being
has repelled itself from itself.

Thus has form pervaded in its realization all its distinctions;
it has made itself transparent
and, as absolute necessity,
is only this simple self-identity of
being in its negation, or in essence.
The distinction itself of content and form
has thus equally vanished;
for that unity of possibility in actuality
and actuality in possibility is the form
which in its determinateness or in positedness is
indifferent towards itself:
it is the fact full of content
on which the form of necessity
externally ran its course.
But necessity is thus this
reflected identity of the two determinations
as indifferent to them,
and hence the form determination of the in-itself
as against the positedness,
and this possibility constitutes the limitation of
the content which real necessity had.
The resolution of this difference is
however the absolute necessity
whose content is this difference
which in this necessity penetrates itself.

Absolute necessity is therefore the truth
in which actuality and possibility in general
as well as formal and real necessity return.
As we have just seen, it is being which in its negation,
in essence, refers itself to itself and is being.
It is equally simple immediacy or pure being
and simple immanent reflection or pure essence;
it is this, that the two are one and the same.
The absolutely necessary only is because it is;
it otherwise has neither condition nor ground.
But it equally is pure essence,
its being the simple immanent reflection;
it is because it is.
As reflection, it has a ground and a condition
but has only itself for this ground and condition.
It is in-itself, but its in-itself is its immediacy,
its possibility is its actuality.
It is, therefore, because it is;
as the rejoining of being with itself,
it is essence;
but because this simple is
equally immediate simplicity,
it is being.

Absolute necessity is thus
the reflection or form of the absolute,
the unity of being and essence,
simple immediacy which is absolute negativity.
On the one hand, therefore, its differences are
not like the determinations of reflection
but an existing manifoldness,
a differentiated actuality in the shape of others
independently subsisting over against each other.
On the other hand, since its connection is
that of absolute identity,
it is the absolute conversion of
its actuality into its possibility
and its possibility into its actuality.
Absolute necessity is therefore blind.
On the one hand, the two different terms
determined as actuality and possibility have
the shape of immanent reflection as being;
they are therefore free actualities,
neither of which reflectively shines in the other,
nor will either allow in it
a trace of its reference to the other;
grounded in itself, each is inherently necessary.
Necessity as essence is concealed in this being;
the reciprocal contact of these actualities appears
therefore, as an empty externality;
the actuality of the one in the other is
the possibility which is only possibility, contingency.
For being is posited as absolutely necessary,
as the self-mediation which is
the absolute negation of mediation-through-other,
or being which is identical only with being;
consequently, an other that has actuality in being,
is therefore determined as something merely possible,
as empty positedness.

But this contingency is rather absolute necessity;
it is the essence of those free, inherently necessary actualities.
This essence is averse to light,
because there is no reflective shining
in these actualities, no reflex:
because they are grounded purely in themselves,
are shaped for themselves,
manifest themselves only to themselves;
because they are only being.
But their essence will break forth in them
and will reveal what it is and what they are.
The simplicity of their being,
their resting just on themselves,
is absolute negativity;
it is the freedom of their reflectionless immediacy.
This negative breaks forth in them because being,
through this same negativity which is its essence,
is self-contradiction;
it will break forth against this
being in the form of being,
hence as the negation of those actualities,
a negation absolutely different from their being;
it will break forth as their nothing,
as an otherness which is just as free towards them
as their being is free.
Yet this negative was not to be missed in them.
In their self-based shape
they are indifferent to form,
are a content and consequently
different actualities
and a determinate content.
This content is the mark
that necessity impressed upon them
by letting them go free as absolutely actual;
for in its determination it is
an absolute turning back into itself.
It is the mark to which necessity appeals
as witness to its right,
and, overcome by it,
the actualities now perish.
This manifestation of what
determinateness is in its truth,
that it is negative self-reference,
is a blind collapse into otherness;
in the sphere of immediate existence,
the shining or the reflection
that breaks out in it is a becoming,
a transition of being into nothing.
But, conversely, being is equally essence,
and becoming is reflection or a shining.
Thus the externality is its inwardness;
their connection is one of absolute identity;
and the transition of the actual into the possible,
of being into nothing, is a self-rejoining;
contingency is absolute necessity;
it is itself the presupposing of that
first absolute actuality.

This identity of being with itself
in its negation is now substance.
It is this unity as in its negation
or as in contingency;
and so, as relation to itself, it is substance.
The blind transition of necessity is
rather the absolute's own exposition,
its movement in itself which, in its externalization,
reveals itself instead.

CHAPTER 3

The absolute relation

Absolute necessity is not so much the necessary,
even less a necessary, but necessity:
being simply as reflection.
It is relation because it is a distinguishing
whose moments are themselves
the whole totality of necessity,
and therefore subsist absolutely,
but do so in such a way that
their subsisting is one subsistence,
and the difference only the reflective shine of
the movement of exposition,
and this reflective shine is the absolute itself.
Essence as such is reflection or a shining;
as absolute relation, however, essence is the
reflective shine posited as reflective shine,
one which, as such self-referring, is absolute actuality.
The absolute, first expounded by external reflection,
as absolute form or as necessity now expounds itself;
this self-exposition is its self-positing,
and is only this self-positing.
Just as the light of nature is not a something,
nor is it a thing, but its being is rather only its shining,
so manifestation is self-identical absolute actuality.

The sides of the absolute relation
are not, therefore, attributes.
In the attribute the absolute reflectively shines
only in one of its moments,
as in a presupposition that
external reflection has simply assumed.
But the expositor of the absolute is the absolute necessity
which, as self-determining, is identical with itself.
Since this necessity is the reflective shining
posited as reflective shining, the sides of this relation,
because they are as shine, are totalities;
for as shine, the differences are
themselves and their opposite,
that is, they are the whole;
and, conversely, they thus are only shine
because they are totalities.
Thus this distinguishing,
this reflecting shining of the absolute,
is only the identical positing of itself.

This relation in its immediate concept is
the relation of substance and accidents,
the immediate internal disappearing and becoming
of the absolute reflective shine.
If substance determines itself as a being-for-itself over
against an other or is absolute relation as something real,
then we have the relation of causality.
Finally, when this last relation passes over into
reciprocal causality by referring itself to itself,
we then have the absolute relation also posited
in accordance with the determination it contains;
this posited unity of itself in its determinations,
which are posited as the whole itself
and consequently equally as determinations,
is then the concept.

IV.33
purusa-artha-sunyanam gunanam pratiprasava kaivalyam
svarupa-pratistha va citi-sakti iti

A. THE RELATION OF SUBSTANTIALITY

Absolute necessity is absolute relation
because it is not being as such
but being that is because it is,
being as the absolute mediation
of itself with itself.
This being is substance;
as the final unity of essence and being,
it is the being in all being.
It is neither the unreflected immediate,
nor something abstract standing behind
concrete existence and appearance,
but the immediate actuality itself,
and it is this actuality
as being absolutely reflected into itself,
as a subsisting that exists in and for itself.
Substance, as this unity of being and reflection,
is essentially the shining and the positedness of itself.
The shining is a self-referring shining, thus it is;
this being is substance as such.
Conversely, this being is only the self-identical positedness,
and as such it is shining totality, accidentality.

This shining is identity as identity of form,
the unity of possibility and actuality.
It is becoming at first,
contingency as the sphere of
coming-to-be and passing-away;
for in the determination of immediacy
the connection of possibility and actuality is
the immediate conversion of the two
into each other as existents,
of each into its other as only an other to it.
But because being is shine,
their relation is also one of identical terms
or of terms shining in one another, that is, reflection.
The movement of accidentality, therefore,
exhibits in each of its moments the mutual
reflective shine of the categories of being
and of the reflective determinations of essence.
The immediate something has a content;
its immediacy is at the same time
reflected indifference towards the form.
This content is determinate,
and because this determinateness is one of being,
the something passes over into an other.
But quality is also a determinateness of reflection;
as such, it is indifferent diversity.
But this diversity is animated into opposition,
and returns to the ground which is the nothing,
but also immanent reflection.
This reflection sublates itself;
but it is itself also reflected in-itselfness:
so it is possibility, and this in-itselfness,
in its transition which is equally immanent reflection,
is necessary actuality.

This movement of accidentality is the actuosity of substance
as the tranquil coming forth of itself.
It is not active against something,
but only against itself as a simple unresisting element.
The sublating of a presupposition is the disappearing shine;
only in the act of sublating the immediate does this
immediate itself come to be, or is that shining;
the beginning that begins from itself is first of all
the positing of this itself from which the beginning is made.

Substance, as this identity of the reflective shining,
is the totality of the whole and embraces accidentality in itself,
and accidentality is the whole substance itself.
Its differentiation into the simple identity
of being and the flux of accidents
within it is one form of its shining.
That simple being is the formless substance of the imagination
for which the shine has not determined itself as shine,
but which holds on, as on an absolute,
to this indeterminate identity that has no truth
but only is the determinateness of immediate actuality,
or equally so of in-itselfness or possibility,
form determinations that fall into accidentality.
The other determination, the flux of accidents,
is the absolute form-unity of accidentality,
substance as absolute power.
The ceasing-to-be of the accident is
its return as actuality into itself,
as into its in-itself or into its possibility;
but this, its in-itself, is itself only a positedness
and therefore also actuality,
and because these form determinations are
equally determinations of content,
this possible is an actual differently determined
also according to content.
Substance manifests itself through the actuality,
with the content of the latter into which
it translates the possible, as creative power,
and, through the possibility to which
it reduces the actual, as destructive power;
the creating is destructive and the destructing creative,
for the negative and the positive, possibility and negativity
are in substantial necessity absolutely united.

The accidents as such, and there are several of them,
because plurality is one of the determinations of being,
have no power over each other.
They are the immediately existent something,
or the something that immediately exists for itself;
concretely existing things of manifold properties;
or wholes consisting of parts, self-subsisting parts;
forces in need of reciprocal solicitation
and conditioning each other.
In so far as such an accidental
being seems to exercise a power over an other,
that power is that of substance that
encompasses them both within itself
and, as negativity, posits an inequality of value:
one it determines as ceasing-to-be
and another as having a different content
and as coming-to-be,
the one as passing over into its possibility
and the other into actuality
accordingly ever dividing itself
into this difference of form and content
and ever purifying itself of this one-sidedness,
but in this purification ever falling back
into determination and division.
One accident thus drives out another
only because its own subsisting is
this very totality of form and content
into which it, as well as its other, equally perishes.

Because of this immediate identity and presence
of substance in the accidents,
there is still no real difference present.
In this first determination, substance is not yet manifested
according to its whole concept.
When substance, as self-identical being-in-and-for-itself,
is differentiated from itself as a totality of accidents,
it is substance itself, as power, that mediates the difference.
This power is necessity, the positive persistence
of the accidents in their negativity
and their mere positedness in their subsistence;
this middle is thus the unity of
substantiality and accidentality themselves,
a middle whose extremes have no subsistence of their own.
Substantiality is, therefore, only
the relation as immediately vanishing;
it refers to itself not as a negative
and, as the immediate unity of power with itself,
is in the form only of its identity,
not of its negative essence;
only one of its moments,
that of negativity or of difference,
vanishes altogether;
the other moment of identity does not.

Another way of considering the matter is this.

The shine or the accidentality is indeed
in itself substance by virtue of the power,
but is not thus posited as this self-identical shine;
and therefore substance has only the accidentality,
not itself, for its shape or positedness;
it is not substance as substance.
The relation of substantiality is at first, therefore, only this,
that substance manifests itself as a formal power
whose differences are not substantial;
in fact, substance only is as the inner of the accidents,
and these only are in the substance.
Or this relation is only the shining of totality as becoming;
but it is equally reflection and, for this reason,
the accidentality which substance is in itself is also posited as such;
it is thus determined as self-referring negativity over against itself,
determined as self-referring simple identity with itself;
and it is substance that exists in and for itself,
substance endowed with power.
Thus the relation of substantiality passes over
into the relation of causality.

B. THE RELATION OF CAUSALITY

Substance is power, power reflected into itself,
not transitive power but power that posits determinations
and distinguishes them from itself.
As self-referring in its determining,
it is itself that which it posits as a negative
or makes into a positedness.
This positedness is, as such, sublated substantiality,
the merely posited, the effect;
the substance that exists for itself is, however, cause.

This relation of causality is in the first place
only this relation of cause and effect;
as such, it is the formal relation of causality.

a. Formal causality

1. Cause is originative as against the effect.
As power, substance is the reflective shining,
or it has accidentality.
But in this shining, as power,
it equally is an immanent reflection;
it thus expounds its transition,
and this reflective shine is
determined as reflective shine,
or the accident is posited
as being just this, something posited.
But in its determining substance
does not proceed from accidentality,
as if the latter were an other beforehand
and were determined as determinateness only then,
but the two are one actuosity.
Substance as power determines itself;
but this determining is immediately itself
the sublation of the determining and a turning back.
It determines itself: substance, that which determines, is
thus the immediate and that which is itself already determined;
in determining itself it therefore posits
the already determined as determined;
and thus it has sublated the
positedness and has returned into itself.
Conversely, because this turning back is
the negative reference of substance to itself,
it is itself a determining
or the repelling of itself from itself;
it is through this turning back that
the determinate comes to be from which
substance seems to begin
and now to posit as something
which it has found already determined.
Absolute actuosity is thus cause,
the power of substance in its truth
as the manifestation by which
that which is in itself,
the accident or the positedness,
is immediately expounded in its becoming,
is posited as positedness, as effect.
This effect is, therefore, first the same as
what the accidentality of the relation
of substance is, namely substance as positedness;
but, second, an accident
is substantially such only by vanishing, only as transient;
but as effect it is positedness as self-identical;
in the effect the cause is manifested as the whole substance,
that is to say, as reflected into itself
in the positedness itself as such.

2. Over against this positedness reflected into itself,
this determined as determined,
there stands substance as the non-posited original.
Because substance is as absolute power
a turning back into itself,
yet this turning back is itself a determining,
it is no longer the mere in-itself of its accident
but is also posited as this in-itself.
Substance has actuality, therefore, only as cause.
But this actuality in which its in-itself,
its determinateness in the relation of substantiality,
is now posited as determinateness, is effect;
therefore substance has the actuality
which it has as cause only in its effect.
This is the necessity which is cause.
It is actual substance,
because as power substance determines itself;
but it is at the same time cause,
because it expounds this determinateness
or posits it as positedness
and thus posits its actuality
as positedness or effect.
This is the other of cause,
the positedness as against the original
and as mediated through it.
But cause, as necessity,
equally sublates this mediating and,
in determining itself as the originally
self-referring term, as against the mediated,
turns back to itself;
for positedness is determined as positedness,
and consequently as self-identical;
therefore, cause is truly actual and self-identical
only in its effect.
The effect is therefore necessary,
because it is the manifestation of the cause
or is this necessity which the cause is.
Only as this necessity is cause self-moving,
self-initiating without being solicited by another,
self-subsisting source of production out of itself;
it must effect;
its originariness is this,
that it is because its immanent reflection
is a positing that determines and conversely;
the two are one unity.

Consequently, an effect contains nothing whatever
that the cause does not contain.
Conversely, a cause contains nothing
that is not in its effect.
A cause is cause only to the extent
that it produces an effect;
to be cause is nothing but
this determination of having an effect,
and to be effect is nothing but
this determination of having a cause.
Cause as such entails its effect,
and the effect entails the cause;
in so far as a cause has not acted yet
or has ceased to act,
it is not a cause;
and the effect, in so far as
its cause is no longer present,
is no longer an effect
but an indifferent actuality.

3. Now in this identity of cause and effect
the form distinguishing them respectively,
as that which exists in itself
and that which is posited,
is sublated.
The cause is extinguished in its effect
and the effect too is thereby extinguished,
for it only is the determinateness of the cause.
Hence this causality which has been extinguished
in the effect is an immediacy
which is indifferent to the relation of cause and effect
and comes to it externally.

b. The determinate relation of causality

1. The self-identity of cause in its effect is
the sublation of its power and negativity,
hence a unity which is indifferent
to differences of form,
that is to say, content.
This content, therefore, refers to form
(here causality) only implicitly.
The two are thus posited as diverse,
and with respect to content the form is itself
a causality which is only immediately efficient,
a contingent causality.

Further, the content is as thus determined
an internally diversified content;
and the cause is determined in accordance with its content,
and so is therefore also the effect.
The content, since reflectedness here
is also immediate actuality,
is to this extent actual,
but finite, substance.

This is now the relation of causality
in its reality and finitude.
As formal, it is the infinite relation of absolute power,
the content of which is pure manifestation or necessity.
As finite causality, on the contrary,
it has a given content
and, as an external difference,
it runs its course here and there over it,
this identical content which in its determination
is one and the same substance.

Because of this identity of content,
this causality is an analytic proposition.
It is the same fact that comes up once as cause
and then again as effect,
in one case as something subsisting on its own
and in the other as positedness or determination.
Since these determinations of form
are an external reflection,
it is up to the essentially tautological
consideration of a subjective understanding
to determine an appearance as effect
and to rise from it to its cause
in order to comprehend and explain it.
The same content is being repeated;
there is nothing else in the cause
which is not in the effect.
For instance, rain is the cause of
wetness which is its effect;
“the rain makes wet,”
this is an analytical proposition;
the same water which is rain is wetness;
as rain, this water is only in
the form of a subject by itself;
as wetness or moisture, it is
on the contrary in adjectival form,
something posited no longer meant to have a subsistence on its own;
and the one determination, just like the other, is external to water.
Again, the cause of this color is a coloring agent,
a pigment which is one and the same actuality,
once in the form of an agent external to it,
that is, is externally linked to an agent different from it;
but again in the determination, equally external
to it, of an effect.
The cause of an act is the inner intention of
the subject who is the agent,
and this intention is the same in content and value
as the existence which it attains through the action.
If the movement of a body is considered as effect,
the cause of this effect is then a propulsive force;
but it is the same quantum of movement
which is present before and after the propulsion,
the same concrete existence which the propulsive body
contained and which it communicated to the one propelled;
and what it communicated, it lost in equal measure.

The cause, say the painter or the propulsive body,
does have yet another content than,
in the case of the painter,
the colors and the form combining
these into a painting;
and, in the other case,
the movement of specific strength and direction.
But this further content is a contingent side-being
which has nothing to do with the cause;
whatever other qualities the painter
might possess besides being the painter of this painting,
this does not enter into the painting;
only those of his properties
which are displayed in the effect
are present in him as cause;
as for the rest, he is not a cause.
Likewise, whether the propulsive body is
of stone or wood, green, yellow, and so on,
all this does not enter into its propulsion
and, to this extent, is not a cause.

It is worth noting in regard to
this tautology of the relation of causality
that the tautology does not seem to occur
whenever it is not the proximate,
but the remote cause which is at issue.
The alteration of form
which the basic fact undergoes
as it passes through several middle terms
hides the identity which it preserves across them.
In this proliferation of causes
introduced between it and the last effect,
that fact is linked to other things
and circumstances,
so that it is not that first term,
which is declared the cause,
but all these several causes together,
that contain the complete effect.

For instance, if a man developed his talents
in circumstances due to the loss of his father
who was hit by a bullet in battle, then this shot
(or still further back,
the war or some cause of the war,
and on to infinity)
could be adduced as the cause of the man's skillfulness.
But it is clear that the shot, for one,
is not the cause by itself
but only in conjunction with the
other efficient determinations.
Or more precisely, the shot is not the cause at all,
but only a single moment that pertained to
the circumstances of the possibility.

But it is the inadmissible application of
the relation of causality to the relations of
physico-organic and spiritual life that must be noted above all.
Here that which is called the cause does indeed show itself
to be of a different content than the effect,
but this is because anything that has an effect on
a living thing is independently determined,
altered, and transmuted by the latter,
for the living thing will not let
the cause come to its effect,
that is, it sublates it as cause.
Thus it is inadmissible to say that
nourishment is the cause of blood,
or that such and such a dish, or chill and humidity,
are the causes of fever or of what have you;
it is equally inadmissible to give the Ionic climate
as the cause of Homer's works,
or Caesar's ambition as the cause of
the fall of Rome's republican constitution.
In history in general there are indeed
spiritual masses and individuals
at play and influencing each other;
but it is of the nature of spirit,
in a much higher sense than it is of
the character of living things,
that it will not admit another
originative principle within itself,
or that it will not let a cause
continue to work its causality in it undisturbed
but will rather interrupt and transmute it.
But these relations belong to the idea,
and will come up for discussion then.
This much can still be noted here,
namely that in so far as the relation of cause
and effect is admitted,
albeit in an inappropriate sense,
the effect cannot be greater than the cause.
It has become a common witticism in history
to let great effects arise from small causes
and to cite as the first cause of an event
of far-reaching and profound consequence an anecdote.
Any such so-called first cause is to be regarded
as no more than an occasion, an external stimulus,
of which the inner spirit of the event had no need,
or could have used a countless number of others,
in order to make its first appearance,
to give itself a first breath and announce itself.
The converse is rather the case.
It is by the spirit that any such
triviality and contingency is determined
in the first place to be the occasion of spirit.
Historical arabesques that draw a full-blown figure
out of a slender stalk are no doubt an ingenious,
but highly superficial, practice.
It is true that in the rise of the great out of the small
we witness everywhere the conversion
that spirit works on the external;
but precisely for this reason
the external is not the cause within spirit;
rather, that conversion itself sublates the relation of causality.

2. But this determinateness of the relation of causality,
that content and form are different
and indifferent to each other,
extends further.
The determination of form is also content determination;
cause and effect, the two sides of the relation,
are therefore also another content.
Or the content, because it is only as the content of a form,
has the difference of this form within it
and is essentially different.
But this form of the content is the
relation of causality,
which is a content identical
in cause and the effect,
and consequently the different content is externally connected,
on the one hand with the cause
and on the other with the effect;
hence the content itself does not enter into
the effective action and into the relation.
This external content is therefore relationless,
an immediate concrete existence,
or because it is as content
the implicit identity of cause and effect,
it is also immediate, existent identity.
This content is, therefore, anything at all
which has manifold determinations of its existence,
among them also this, that it is
in some respect or other cause or also effect.
In it, the form determinations of cause and effect
have their substrate, that is to say,
their essential subsistence
and each has a particular subsistence
(since their identity is their subsistence);
but it is a subsistence which is at the same time immediate,
not their subsistence as unity of form or as relation.
But this thing is not only substrate but also substance,
for it is identical subsistence only as subsistence of the relation.
Moreover, the substance is finite substance,
for it is determined as immediate over against its causality.
But it has causality at the same time,
for it is just as much an identity as this relation.
Now this substrate is, as cause,
negative reference to itself.
But this “itself” to which it refers is,
first, a positedness because it is determined as immediately actual;
this positedness, as content, is any determination whatever.
Second, causality is external to the substrate
and itself constitutes, therefore, its positedness.
Now since it is causal substance,
its causality consists in negatively referring itself to itself,
hence to its positedness and external causality.
The effective action of this substance thus
begins from something external,
frees itself from this external determination,
and its turning back into itself is
the preservation of its immediate concrete existence
and the sublation of the one which is posited,
and consequently of its causality as such.

Take a stone that moves. It is a cause.
Its movement is a determination which it has.
But, besides it, it contains yet many other determinations
(color, shape, and so on)
that do not enter into its causality.
Because its immediate concrete existence is
separated from its form-connection,
namely the form of causality,
the latter is something external;
the stone's movement and the causality attaching to it
is in it only positedness.
But the causality of the stone is also the stone's own causality,
as follows from the fact that its substantial subsistence is
the stone's identical self-reference,
but that this is now determined as positedness
and is therefore at the same time negative self-reference.
Its causality, which is directed against itself as
a positedness or as an externality,
consists therefore in sublating this
and through its removal in returning to itself,
to this extent, therefore, in not
being self-identical in its positedness
but only in restoring its originariness.
Or again, rain is the cause of wetness,
which is the same water as the rain.
This water has the determination of being rain and cause
because this determination has been posited in it by another;
another force, or what have you, has lifted it into the air
and compressed it into a mass, the weight of which makes it fall.
Its being removed from the earth is
a determination alien to its original self-identity, to its gravity;
its causality consists in removing such determination
and in restoring its original identity;
but this means also sublating its causality.

We now consider the second determinateness of causality
which concerns form;
this relation is causality external to itself,
as the originariness which is
within just as much positedness or effect.
This union of opposite determination
in an existent substrate constitutes
the infinite regress from cause to cause.
We start from an effect;
the latter has as effect a cause;
but this cause has a cause in turn, and so on.
Why does the cause have a cause in turn?
That is to say, why is the same side
which was previously determined as cause
now determined as effect
and therefore demands a new cause?

Because the cause is something finite,
a determinate in general;
determined as one moment of the form
as against the effect;
so it has its determinateness or negation outside it;
but for this very reason it is itself finite,
has its determinateness within it
and is thereby positedness or effect.
Its identity as this positedness is also posited,
but it is a third term, the immediate substrate;
causality is therefore external to itself,
because its originariness is here an immediacy.
The difference of form is therefore
a first determinateness,
not yet determinateness posited as determinateness;
it is existent otherness.
Finite reflection, on the one hand,
stops short at this immediate,
removes the unity of form from it
and makes it be cause in one respect
and effect in another;
on the other hand, it transfers
the unity of form into the infinite,
and through the endless progression
expresses its impotence in attaining
and holding fast to this unity.

Exactly the same is the case of the effect,
or rather the endless progression from effect to effect
is one and the same as  the regression from cause to cause.
Just as in the latter a cause becomes an effect
which has another cause in turn,
so too, conversely, the effect becomes a cause
which has another effect in turn.
The determinate cause under consideration
begins from an externality
and returns in its effect back to itself,
but not as cause;
on the contrary, it loses its causality in that process.
But, conversely, the effect arrives at
a substrate which is substance,
an original self-referring subsistence;
in it, therefore, that positedness becomes a positedness,
that is to say, this substance, as the effect is posited in it,
behaves as cause.
But that first effect, the positedness
that accrues to the substance externally,
is other than the second which the substance produces;
for this second effect is determined as
the immanent reflection of substance
whereas the first is in it as an externality.
But because causality is here causality external to itself,
it also equally fails to return in its effect back to itself
but becomes therein external to itself;
its effect becomes again a positedness in a substrate;
as in another substance which however
equally makes this positedness into positedness,
in other words, manifests itself as cause,
again repels its effect from itself,
and so on, into bad infinity.

3. We now have to see what has resulted from
the movement of determinate causality.
Formal causality expires in the effect
and the element of identity of
these two moments emerges as a result,
but it does so only as an implicit unity of
cause and effect to which the form connection is external.
For this reason, the element of identity is immediate also
with respect to both of the two determinations of immediacy,
first as in-itself,
as a content on which causality is deployed externally;
second, as a concrete existent substrate
in which cause and effect inhere
as different determinations of form.
In this substrate, the two determinations are implicitly one,
but, on account of this implicitness
or of the externality of form,
each is external to itself
and hence, in its unity with the other,
is also determined as other with respect to it.
Consequently, the cause has indeed an effect
and is at the same time itself effect;
and the effect not only has a cause
but is itself also cause.
But the effect which the cause has,
and the effect which it is,
are different;
as are also the cause which the effect has
and the cause which it is.

The outcome of the movement of the
determinate relation of causality is then this,
that the cause does not just expire in the effect,
and thereby the effect as well, as in formal causality,
but that by expiring in the effect
the cause comes to be again;
that the effect vanishes in the cause,
but equally comes to be again in it.
Each of these determinations
sublates itself in its positing,
and posits itself in its sublating;
what we have is not an external
transition of causality from one substrate to another,
but its becoming-other is at the same time its own positing.
Causality thus pre-supposes itself or conditions itself.
The previously only implicit identity, the substrate,
is therefore now determined as presupposition
or posited as against the efficient causality,
and the reflection hitherto only external to
the identity is now in relation to it.

c. Action and reaction

Causality is a presupposing activity.
The cause is conditioned;
it is a negative reference
to itself as a presupposed,
as an external other which in itself,
but only in itself, is causality itself.
This other is, as we have seen,
the substantial identity into which
formal causality passes over,
which now has determined itself
as against this causality as its negative.
Or it is the same as the substance
of the causal relation,
but a substance which is confronted by
the power of accidentality
as itself substantial activity.
It is the passive substance.
Passive is that which is immediate,
or which exists-in-itself
but is not also for itself;
pure being or essence in just
this determinateness of abstract self-identity.
Confronting the passive substance is
the negatively self-referring substance,
the efficient substance.
It is cause inasmuch as in determinate causality
it has restored itself out of the effect
through the negation of itself;
a reflected being which in its otherness
or as an immediate behaves essentially
as a positing activity
and through its negation mediates itself.
Here, therefore, causality no longer has
a substrate in which it inheres;
it is not a determination of form
as against this identity
but is itself substance,
or in other words, causality alone is at the origin.
The substrate is the passive substance
which causality has presupposed for itself.

This cause now acts,
for it is the negative power over itself;
at the same time it is its own presupposition;
thus it acts upon itself as upon an other,
upon the passive substance.
Hence, it first sublates
the otherness of this substance
and returns in it back to itself;
second, it determines this same substance,
posits this sublation of its otherness
or the substance's turning back into itself
as a determinateness.
This positedness, because it is at the same time
the substance's turning back into itself,
is at first its effect.
But conversely, because as presupposing
it determines itself as its other,
it then posits the effect in this other,
in the passive substance.
Or again, because the passive substance is itself this double,
namely a self-subsistent other,
and at the same time something presupposed
and already implicitly identical
with the efficient cause;
because of this, the action of the passive
substance is therefore itself double.
It is at once both the sublation of its determinateness,
namely of its condition,
or the sublation of the self-subsistence
of the passive substance;
and also, in sublating its identity as
it sublates this substance,
the pre-supposing of itself,
that is, the positing or supposing of itself as other.
Through this last moment,
the passive substance is preserved;
that first sublation of it appears in this respect
at the same time also in this way,
namely that only some determinations are sublated in it,
and its identity in the effect
with the efficient cause
occurs in it externally.

To this extent it suffers violence.
Violence is the appearance of power,
or power as external.
But power is something external only in so far as in its
action, that is, in the positing of itself,
the causal substance is at the same
time a presupposing, that is, posits itself as sublated.
Conversely, the act of
violence is therefore equally an act of power.
The violent cause acts only on
an other which it presupposes;
its effect on it is its negative self-reference,
or the manifestation of itself.
The passive is the self-subsistent
which is only a posited,
something internally fractured
an actuality which is condition,
though a condition that now is in its truth
as an actuality that is only a possible,
or, conversely, an in-itself that is only
the determinateness of the in-itself,
is only passive.
To that which suffers violence, therefore,
not only is it possible to do violence,
but violence must be done to it;
that which has dominion over an other,
only has it because its power is that of the other,
a power which in that dominion
manifests both itself and the other.
Through violence the passive substance
is only posited as what it is in truth,
namely, that because it is
the simple positive or the immediate substance,
for that very reason it is only something posited;
the “pre-” that it has as condition is
the reflective shine of immediacy
that the efficient causality strips off from it.

Passive substance, therefore, is only given its due
by the action on it of another power.
What it loses is the immediacy it had,
the substantiality alien to it.
What comes to it as an alien something,
namely that it is determined as a positedness,
is its own determination.
But now in being determined in its positedness,
or in its own determination,
the result is that it is not sublated
but rather that it only rejoins itself
and in its being determined is,
therefore, an originariness.
On the one hand, therefore,
the passive substance is preserved
or posited by the active,
namely in so far as the latter sublates itself;
but, on the other hand, it is the act of
the passive substance itself to rejoin itself
and thus to make itself into
what is originary and a cause.
The being posited by an other
and its own becoming
are one and the same.

Now, because the passive substance has been converted into a cause,
it follows, first, that the effect is sublated in it;
therein consists its reaction in general.
As passive substance, it is in itself as positedness;
also, positedness has been posited in it by the other substance,
namely in so far as it received its effect within it.
Its reaction contains, therefore, a twofold aspect.
For one, what it is in itself is posited.
And two, what it is as posited
displays itself as its in-itself;
it is positedness in itself,
hence through the other substance it
receives an effect within;
but, conversely, this positedness is
its own in-itself,
it is thus its own effect,
it itself displays itself as a cause.

Second, the reaction is directed at
the first efficient cause.
For the effect which the hitherto
passive substance sublates within itself
is precisely the effect of that other cause.
But a cause has its substantial actuality only in its effect;
inasmuch as this effect is sublated,
so is also the causal substantiality of the other cause.
This happens first in itself through itself, in that
the cause makes itself into an effect;
its negative determination disappears
in this identity and the cause becomes passive;
and, second, it happens through the hitherto passive,
but now reacting substance,
which sublates its effect.

Now in determinate causality
the substance acted upon becomes a cause,
for it acts against the positing of an effect in it.
But it did not react against the cause of that effect
but posited its effect rather in another substance,
and thus there arose the progression to infinity of effects;
for here the cause is only implicitly
identical with itself in the effect,
and hence, on the one hand,
it expires into an immediate identity
as it comes to rest,
but, on the other hand,
it revives in another substance.
In conditioned causality, on the contrary,
the cause refers back to itself in the effect,
for the latter is as a condition,
as a presupposition, its other,
and its act is therefore just as much a becoming
as a positing and sublating of the other.

Further, causality behaves in all this as passive substance;
but, as we have seen, the latter becomes causal
through the effect it incurs.
That first cause, the one which acts first
and receives its effect back into itself as a reaction,
thus comes up again as a cause,
whereby the activity which in finite causality
runs into the bad infinite progression
is bent around and becomes an action
that returns to itself,
an infinite reciprocal action.

C. RECIPROCITY OF ACTION

In finite causality it is substances
that actively relate to each other.
Mechanism consists in this externality of causality,
where the cause's reflection in its effect into itself
is at the same time a repelling being,
or where, in the self-identity which
the causal substance has in its effect,
the substance is equally immediately external to itself
and the effect is transposed into another substance.
In reciprocity of action this mechanism is now sublated,
for it contains first the disappearing of
that original persistence of immediate substantiality;
second, the coming to be of the cause,
and hence originariness mediating
itself with itself through its negation.

At first, the reciprocity of action takes on
the form of a reciprocal causality of substances
that are presupposed and that condition each other;
each is with respect to the other
both active and passive substance.
Since the two are thus passive and active at once,
their difference is thereby already sublated;
it is a totally transparent reflective shine;
they are substances only in being
the identity of the active and the passive.
The reciprocity of action is itself,
therefore, only a still empty way and manner,
and all that is still needed is merely
the external bringing together of what is already there,
both in itself and as posited.
First of all, it is no longer substrates
that are referred to each other but substances;
in the movement of conditional causality,
the still left over presupposed immediacy has been sublated,
and what conditions the causing activity is only an influence,
or its own passivity.
But this influence, moreover, does not come from
another substance originating it
but from precisely a causality
which is conditioned by influence,
or one which is mediated.
This at first external factor that accrues to the cause
and constitutes the side of its passivity is
therefore mediated through the causality itself,
is produced through its own activity
and is, consequently, a passivity
posited by its own very activity.
Causality is conditioned and conditioning.
As conditioning, it is passive;
but it is equally so as conditioned.
This conditioning or passivity is
the negation of the cause through itself
in that it makes itself essentially into an effect
and is cause precisely for that reason.
Reciprocity of action is, therefore, only causality itself;
the cause does not just have an effect
but, in the effect, refers as cause back to itself.

Causality has thereby returned to its absolute concept
and has at the same time attained the concept itself.
At first, it is real necessity, absolute self-identity
in which the difference between itand the determinations
referring to each other within it are substances,
free actualities, over against one another.
Necessity is in this way inner identity;
causality is the manifestation of it
in which its reflective shine of
substantial otherness has been sublated,
and necessity is elevated to freedom.
In the reciprocity of action,
originative causality displays itself
as arising from its negation, from passivity,
and as passing away into it, as a becoming,
but in such a way that this becoming is
at the same time equally only shining;
the transition into otherness is reflection-into-itself;
negation, which is the ground of the cause,
is its positive rejoining with itself.

In the reciprocity of action, therefore,
necessity and causality have disappeared;
they contain both the immediate identity
as combination and reference
and the absolute substantiality of the differences,
consequently their contingency,
the original unity of substantial difference
and therefore the absolute contradiction.
Necessity is being, because being is;
it is the unity of being with itself
that has itself as ground, but, conversely,
because this being has a ground, it is not being;
it is simply and solely reflective shining, reference or mediation.
Causality is this posited transition of original being,
of cause, into reflective shine or mere positedness,
and, conversely, of positedness into originariness;
but the identity itself of being and reflective shine
still is the inner necessity.
This inwardness or this in-itself sublates
the movement of causality;
the result is that the substantiality of the sides
that stand in relation is lost, and necessity unveils itself.
Necessity does not come to be freedom by vanishing
but in that its still only inner identity is manifested,
and this manifestation is the identical movement immanent to
the different sides,
the immanent reflection of shine as shine.
Conversely, contingency thereby
comes to be freedom at the same time,
for the sides of necessity,
which have the shape of independent, free actualities
that do not reflectively shine into each other,
are now posited as an identity,
so that now these totalities of immanent reflection,
in their differences, also shine as identical,
in other words, they are also posited as
only one and the same reflection.

No longer, therefore, does absolute substance
as self-differentiating absolute form
repel itself as necessity from itself,
nor does it fall apart as contingency
into indifferent, external substances,
but, on the contrary, it differentiates itself:
on the one hand, into the totality
(the heretofore passive substance)
which is at the origin,
as the reflection from internal determinateness,
as simple whole that contains its positedness within itself
and in this positedness is posited as self-identical;
this is the universal;
on the other hand, into the totality
(the hitherto causal substance)
which is the reflection,
equally from internal determinateness,
into the negative determinateness
which, just as the self-identical determinateness,
equally is the whole,
but posited as the self-identical negativity;
the singular.
But, because the universal is self-identical
only in that the determinateness that it holds within is sublated,
hence it is the negative as negative,
it immediately is the same negativity that singularity is.
And the singularity, because it equally is
the determinedly determined, the negative as negative,
immediately is the same identity that universality is.
This, their simple identity, is the particularity that,
from the singular, holds the moment of determinateness;
from the universal, that of immanent reflection,
the two in immediate unity.
These three totalities are therefore one and the same reflection
that, as negative self-reference, differentiates itself
into the other two totalities;
but as into a perfectly transparent difference,
namely into the determinate simplicity,
or into the simple determinateness,
which is their one same identity.
This is the concept,
the realm of subjectivity or of freedom.
