A. THE CHEMICAL OBJECT

The chemical object is distinguished from
the mechanical in that the latter is
a totality indifferent to determinateness,
whereas in the chemical object the determinateness,
and hence the reference to other,
and the mode and manner of this reference,
belong to its nature.
This determinateness is at the same time
essentially a particularization,
that is, it is taken up into universality;
thus it is a principle:
a determinateness which is universal,
not only the determinateness
of the one singular object
but also of the other.
In the chemical object there is now, therefore,
a distinction in its concept,
between the inner totality of the two determinacies
and the determinateness that constitutes
the nature of the singular object
in its externality and concrete existence.
Since in this way the object is implicitly the whole concept,
it has within it the necessity and the impulse
to sublate its opposed, one-sided subsistence,
and to bring itself in existence to the real whole
which it is according to its concept.

Regarding the expression “chemism” for the said
relation of the non-indifference of objectivity,
it may be further remarked that the expression is
not to be understood here as though the relation were
only to be found in that form of elemental nature
that strictly goes by that name.
Already the meteorological relation must be regarded
as a process whose parts have more the nature
of physical than chemical elements.
In animate things, the sex relation falls under this schema,
and the schema also constitutes the formal basis
for the spiritual relations of love, friendship, and the like.

On closer examination, the chemical object is
at first a self-subsistent totality in general,
one reflected into itself and therefore distinct
from its reflectedness outwards, an indifferent basis,
the individual not yet determined as non-indifferent;
the person, too, is in the first instance a basis
of this kind, one that refers only to itself.
But the immanent determinateness
that constitutes the object's non-indifference is,
first, reflected into itself in such a manner
that this retraction of the reference outwards is
only a formal abstract universality;
the outwards reference is thus a determination
of the object's immediacy and concrete existence.
From this side the object does not return,
within it, to individual totality:
the negative unity has its two moments
of opposition in two particular objects.
Accordingly, a chemical object is
not comprehensible from itself,
and the being of one object is
the being of another.
But, second, the determinateness is
absolutely reflected into itself
and is the concrete moment of
the individual concept of the whole
which is the universal essence,
the real genus of the particular objects.
The chemical object,
which is thus the contradiction
of its immediate positedness
and its immanent individual concept,
is a striving to sublate
the immediate determinateness of its existence
and to give concrete existence to
the objective totality of the concept.
Hence it does still remain a non-self-subsistent object,
but in such a way that it is by nature in tension
with this lack of self-subsistence
and initiates the process as a self-determining.

B. THE PROCESS

1. It begins with the presupposition
that the objects in tension,
as much as they are tensed against themselves,
just as much are they by that very fact
at first tensed against each other,
a relation which is called their affinity.
Each stands through its concept in contradiction
to its concrete existence's own one-sidedness
and each consequently strives to sublate it,
and in this there is immediately posited
the striving to sublate the one-sidedness of the other
and, through this reciprocal balancing and combining,
to posit a reality conformable to the concept
that contains both moments.

Since each is posited within it
as self-contradictory and self-sublating,
they are held apart from each other
and from their reciprocal complementation
only by external violent force.
The middle term whereby these extremes are
now concluded into a unity is,
first, the implicitly existent nature of both,
the whole concept containing both within.
But, second, since in concrete existence
the two stand over against each other,
their absolute unity is also a still formal element
that concretely exists distinct from them,
the element of communication wherein they enter
into external community with each other.
Since the real difference belongs to the extremes,
this middle term is only the abstract neutrality,
the real possibility of those extremes,
the theoretical element, as it were,
of the concrete existence of the chemical objects,
of their process and its result.
In the realm of bodies, water fulfills
the function of this medium;
in that of spirit, inasmuch as there is in it
an analog of such a relation,
the sign in general, and language more specifically,
can be regarded as fulfilling it.

The relation of the objects,
as mere communication in this element,
is on the one hand a tranquil coming-together,
but on the other it is equally a negative relating,
for in communication the concrete concept
which is their nature is posited in reality,
and the real differences of the object are
thereby reduced to its unity.
Their prior self-subsistent determinateness
is thus sublated in the union that conforms to the concept,
which is one and the same in both;
their opposition and tension are thereby blunted,
with the result that in this reciprocal complementation
the striving attains its tranquil neutrality.

The process is in this way dissolved;
since the contradiction between
concept and reality has been resolved,
the extremes of the syllogism have
consequently lost their opposition
and have ceased to be extremes as
against each other and the middle term.
The product is something neutral, that is,
something in which the ingredients,
which can no longer be called objects,
are no longer in tension
and therefore no longer have
the properties that accrued to them in tension,
though in the product the capacity for
their prior self-subsistence and tension is retained.
For the negative unity of the neutral product proceeds
from a presupposed non-indifference;
the determinateness of the chemical object is
identical with its objectivity; it is original.
Through the process just considered,
this non-indifference is only immediately sublated;
the determinateness, therefore, is
not as yet absolutely reflected into itself,
and consequently the product of the process is
only a formal unity.

2. In this product the tension of opposition,
and the negative unity which is the activity of the process,
are now indeed dissolved.
But since this unity is essential to the concept
and has also itself come into concrete existence,
it is still present but has stepped outside the neutral object.
The process does not spontaneously re-start itself,
for it had non-indifference only as its presupposition;
it did not posit it.
This self-subsistent negativity outside the object,
the concrete existence of the abstract singularity
whose being-for-itself has its reality in
the non-indifferent object,
is in itself now in tension with its abstraction,
an inherently restless activity outwardly bent on consuming.
It connects immediately with the object
whose tranquil neutrality is the real possibility
of an opposition to this neutrality;
the same object is now the middle term
of the prior formal neutrality,
now concrete in itself and determined.

The more precise immediate connection
of the extreme of negative unity with the object is
in that the latter is determined by it
and is thereby disrupted.
This disruption may at first be regarded
as the restoration of the opposition of the objects
in tension with which chemism began.
But this determination does not constitute
the other extreme of the syllogism
but belongs to the immediate connection
of the differentiating principle
with the middle in which this principle
gives itself its immediate reality;
it is the determinateness which the middle term,
besides at the same time being
the universal nature of the subject matter,
possesses in the disjunctive syllogism,
whereby that object is both
objective universality and determinate particularity.
The other extreme of the syllogism stands opposed to the
external self-subsistent extreme of singularity;
it is, therefore, the equally
self-subsisting extreme of universality;
hence the disruption that the real neutrality
of the middle term undergoes in it is
that it breaks up into moments
that are not non-indifferent
but, on the contrary, neutral.
Accordingly these moments are, on the one side,
the abstract and indifferent base,
and, on the other, this base's activating principle
which, separated from it, equally attains
the form of indifferent objectivity.

This disjunctive syllogism is the totality of chemism
in which the same objective whole is
exhibited as self-standing negative unity;
then, in the middle term, as real unity;
and finally as the chemical reality
resolved into its abstract moments.
In these moments the determinateness has
not reached its immanent reflection in an other
as in the neutral product,
but has in itself returned into its abstraction,
an originally determined element.

3. These elemental objects are therefore
liberated from chemical tension;
in them, the original basis of
that presupposition with which chemism began
has been posited through the real process.
Now further, their inner determinateness is
as such essentially the contradiction
of their simple indifferent subsistence
and themselves as determinateness,
and is the outward impulse that disrupts itself
and posits tension in its determined object
and in an other, in order that the object
may have something to which it can relate as non-indifferent,
with which it can neutralize itself
and give to its simple determinateness an existent reality.
Consequently, on the one hand
chemism has gone back to its beginning
in which objects in a state of reciprocal tension
seek one another and then combine in a neutral product
by means of a formal and external middle term;
and, on the other hand,
by thus going back to its concept,
chemism sublates itself
and has gone over into a higher sphere.

C. TRANSITION OF CHEMISM

Even ordinary chemistry shows examples of
chemical alterations in which a body, for example,
imparts a higher oxidation to one part of its mass
and thereby reduces another part to a lower degree of the same,
at which degree alone it can enter into a neutral combination
with another differing body brought into contact with it,
a combination to which it would not have been receptive
at that other first immediate degree.
What happens here is that the object does not connect
with another in accordance with an immediate, one-sided determinateness,
but, in accordance with the inner totality of an original relation,
posits the presupposition which it needs for a real connection
and thereby gives itself a middle term by virtue of which
it unites its concept with its reality in conclusion;
it is a singularity determined in and for itself,
the concrete concept as the principle of the disjunction
into extremes whose re-union is the activity of
that same negative principle that thereby
returns to its first determination,
but returns to it objectified.

Chemism is itself the first negation
of the indifferent objectivity
and of the externality of determinateness;
it is still burdened, therefore,
by the immediate self-subsistence
of the object and with externality.
Consequently it is not yet for itself
that totality of self-determination
that proceeds from it
and in which it is rather sublated.
The three syllogisms that have resulted
constitute its totality.
The first has formal neutrality
for its middle term and
for extremes the objects in tension.
The second has for its middle term the
product of the first, real neutrality;
and for extremes the disrupting activity
and its product, the indifferent element.
But the third is the self-realizing concept
that posits for itself the presupposition
by virtue of which the process of its
realization is conditioned:
a syllogism that has the universal for its essence.
Yet, on account of the immediacy and externality
by which the chemical objectivity is still determined,
these three syllogisms fall apart.
The first process whose product is
the neutrality of the tensed objects is
extinguished in this product
and is re-activated only by
a differentiation that comes to it from outside;
conditioned by an immediate presupposition,
the process is exhausted in it.
The excretion out of the neutral product
of the non-indifferent extremes,
as also their decomposition into their abstract elements,
must likewise proceed from conditions and
stimulations of activity brought in from the outside.
But the two essential moments of the process,
neutralization on the one hand
and dissolution and reduction on the other,
since they too are bound together
in one and the same process
and the union blunting the tension of the extremes is
also a separation into these,
constitute on account of the still underlying
externality two diverse sides;
the extremes that are separated in that same process
are other than the objects or matters uniting in it;
in so far as the former proceed from it again
as non-indifferent, they must turn outwards;
their renewed neutralization is a process other
than the one that took place in the first.

But these various processes,
which have demonstrated themselves to be necessary,
are equally so many stages by which
externality and conditionality are sublated,
and from which the concept emerges as
determined in and for itself,
a totality unconditioned by externality.
In the first process, what is sublated is
the externality of the mutually non-indifferent extremes that
constitute the whole reality,
or the distinction between the implicitly
determinate concept and its existing determinateness.
Sublated in the second process is the externality
of the real unity, union as merely neutral.
Or more precisely, the formal activity sublates itself
in bases that are equally formal, neutral determinacies
whose inner concept is now the absolute activity
that has withdrawn into itself
and now realizes itself internally,
that is, posits the determinate difference within itself
and through this mediation constitutes itself as real unity;
this is a mediation which is thus the concept's
own mediation, its self-determination
and, considering that in it
the concept reflects itself back into itself,
an immanent presupposing.
The third syllogism, which on the one hand is
the restoration of the preceding processes,
sublates on the other hand the last
remaining moment of indifferent bases:
it sublates the whole abstract external immediacy
that becomes in this way
the concept's own moment of self-mediation.
The concept that has thus sublated as external
all the moments of its objective existence,
and has posited them in its simple unity,
is thereby completely liberated from
the objective externality to which
it refers only as an unessential reality.
This objective free concept is purpose.
