The doctrine of the concept

The concept is the free [actuality],
as the substantial power that is for itself,
and it is the totality,
since each of the moments is the whole that it is,
and each is posited as an undivided unity with it.
So, in its identity with itself,
it is what is determinate in and for itself.

The way the concept proceeds is
no longer passing over or shining in an other.
It is instead development
since what are differentiated are
at the same time immediately posited as
identical with one another and with the whole,
each being the determinacy that it is
as a free being of the whole concept.

The doctrine of the concept is divided into the doctrine of

(1) the subjective or formal concept,
(2) the concept as determined to immediacy,
    or the objectivity,
(3) the idea,
    the subject-object,
    the unity of the concept and objectivity,
    the absolute truth.

Ordinary logic apprehends only matters in themselves
that surface here as a part of the third part of the whole
and, in addition, the so-called 'laws of thinking' (that surfaced earlier)
and, in applied logic, some from the sort of knowing
bound up with material that is still
psychological, metaphysical, and otherwise empirical,
since those forms of thinking in the end
no longer sufficed for it.
Nonetheless, this science thereby lost any solid orientation.
Moreover, those forms that pertain at least
to the genuine domain of logic
are taken merely as determinations of conscious thinking
and, indeed, conscious thinking at the level merely
of the understanding, not of reason.

The preceding logical determinations,
the determinations of being and essence,
are not mere determinations of thought, to be sure.
In their process of passing over (the dialectical moment),
and in their return into themselves and in their totality,
they have proven themselves to be concepts.
But they are merely determinate concepts,
concepts in themselves or, what is the same,
concepts for us since the other
(into which each determination passes over
or in which it shines and is accordingly something relative)
is determined not as something particular.
Nor is the third factor determined as
something individual or as a subject,
which is to say that the identity of the determination is
not posited in the determination opposite it,
that its freedom is not posited,
since it is not universality.
What is usually understood by 'concepts' are
determinations of understanding,
even merely universal representations,
hence, in general, finite determinations.

The logic of the concept is
usually understood as a merely formal science,
revolving around the form as such of
the concept, the judgment, and the syllogism,
but not at all around whether something is true;
this depends, to the contrary, completely on the content alone.
Were the logical forms of the concept actually dead, ineffective,
and indifferent receptacles of representations or thoughts,
then familiarity with them would be a historical record that
is quite superfluous and dispensable for the truth.
In fact, however, as forms of the concept,
they are, to the contrary,
the living spirit of the actual,
and what is true of the actual is true
only by virtue of these forms, through them, and in them.
However, the truth of these forms for themselves,
let alone their necessary connection,
has never been considered and investigated until now.

YS III.1

desa-bandha cittasya dharana

YS III.2

tatra pratyaya-eka-tanata dhyana

YS III.3

tad evartha-matra-nirbhasa svarupa-sunya iva samadhi

YS III.4

trayam ekatra samyama

YS III.5

taj-jayat prajna-aloka

YS III.6

tasya bhumisu viniyoga

YS III.7

trayam antar-angam purvebhya

YS III.8

tad api bahir-angam nirbijasya

A. THE SUBJECTIVE CONCEPT

YS III.9

vyutthana-nirodha-samskarayor abhibhava-pradur-bhavau
nirodha-kshana-citta-anvaya nirodha-parinama

YS III.10

tasya prasanta-vahita samskara

a. The concept as such

The concept as such contains the moments of

universality
(as the free sameness with itself in its determinacy),
particularity
(as the determinacy in which
the universal remains the same as itself, unalloyed),
and individuality
(as the reflection-in-itself of
the determinacies of universality and particularity,
the negative unity with itself
that is the determinate in and for itself
and at the same time identical with itself or universal).

The individual is the same as the actual,
with the difference that the former has
gone forth from the concept
and is accordingly posited as universal,
as the negative identity with itself.
Because it is first only in itself
or immediately the unity of
the essence and concrete existence,
the actual can be productive.
But the individuality of the concept is
simply what produces and, indeed,
no longer as the cause with
the semblance of producing an other,
but as what produces its very self.
The individuality, however, is
not to be taken in the sense of
only immediate individuality in terms of which
we speak of individual things, human beings.
This determinate sense of individuality
surfaces first in the case of judgment.
While each moment of the concept is
itself the entire concept,
individuality, the subject, is
the concept posited as the totality.

The concept is what is utterly concrete
since the negative unity with itself
(as being-determined-in-and-for-itself
which is the individuality)
itself makes up its relation to itself, the universality.
To this extent, the moments of the concept
cannot be detached from one another;
the determinations of reflection are supposed
to be grasped and to be valid each for itself,
detached from the opposed determination.
Since, however, their identity is posited in the concept,
each of its moments can be immediately grasped only
on the basis of and with the others.

Taken in an abstract sense,
universality, particularity, and individuality
are the same as identity, difference, and ground.
But the universal is what is identical with itself
explicitly in the sense that at the same time
the particular and the individual are contained in it.
Furthermore, the particular is what has been
differentiated or the determinacy,
but in the sense that it is
universal in itself and as an individual.
Similarly, the individual has
the meaning of being the subject,
the foundation which contains
the genus and species in itself
and is itself substantial.
This is the posited inseparability of
the moments in their difference,
the clarity of the concept in which
no difference interrupts or obscures the concept,
but in which each difference is instead equally transparent.

There is nothing said more commonly
than that the concept is something abstract.
This is correct in part insofar as
its element is thinking generally
and not the empirically concrete sphere of the senses,
in part insofar as it is not yet the idea.
In this respect, the subjective concept is still formal,
yet not at all as if it should respectively have or acquire
some other content than itself.
As the absolute form itself,
the concept is every determinacy,
but as it is in its truth.
Thus, although the concept is
at the same time abstract,
it is what is concrete
and, indeed, the absolutely concrete,
the subject as such.
The absolutely concrete is the spirit,
the concept insofar as it concretely exists as concept,
differentiating itself from its objectivity
which, despite the differentiating,
remains the concept's own objectivity.
Everything else concrete, as rich as it may be,
is not so inwardly identical with itself
and, for that reason, in itself not as concrete,
least of all what one commonly understands
by the concrete, a manifold externally held together.
What are also called concepts
and, to be sure, determinate concepts,
e.g. human being, house, animal, and so forth, are
simple determinations and abstract representations,
abstractions that, taking only the factor of universality
from the concept while omitting
the particularity and individuality,
are thus not developed in themselves
and accordingly abstract precisely from the concept.

The moment of individuality first posits
the moments of the concept as differences,
since it is the concept's negative reflection-in-itself.
Thus it is initially the free differentiating of
the concept as the first negation,
by means of which the determinacy of the concept is posited,
but posited as particularity.
That is to say, first, that the moments differentiated have
the determinacy of conceptual moments only opposite one another
and, second, that their identity
(that the one is the other)
is equally posited.
This posited particularity of
the concept is the judgment.

The usual species of clear, distinct, and adequate concepts pertain,
not to the concept, but to psychology insofar as,
by 'clear and distinct concepts', representations are meant,
where 'clear' means an abstract, simply determinate representation
and 'distinct' the sort of representation in which
a distinguishing mark, i.e. some sort of determinacy
has been singled out as a sign for subjective knowing.
Nothing is so much the distinguishing mark of
the externality and decay of logic
than the cherished category of the distinguishing mark.
The adequate concept is more of
a play on the concept, indeed even the idea,
but still expresses nothing but the formal aspect of
the agreement of a concept or even a representation
with its object, some external thing.
Underlying the so-called subordinate and coordinate concepts is
[a] the concept-less difference between the universal and the particular
as well as [b] their relatedness in an external reflection.
An enumeration of species of contrary and contradictory,
affirmative, negative concepts and so forth is, moreover,
nothing other than a process of arbitrarily reading off
determinacies of thought that for their part belong to
the sphere of being or essence,
where they have already been considered,
and that have nothing to do
with the determinacy of the concept itself as such.
The genuine differences of the concept,
the universal, particular, and individual,
constitute species of the concept,
if at all only insofar as they are held apart
from one another by external reflection.
The immanent differentiating and determining of
the concept is on hand in the judgment,
since the judging is the determining of the concept.

YS III.11

sarva-arthata-ekagrataya kshayodaya cittasya samadhi-parinama

b. The judgment

The judgment is the concept in its particularity as
the differentiating relation of its moments,
which are posited as being for themselves
and, at the same time, as identical with themselves,
not with one another.

In the case of a judgment one usually thinks first of
the self-sufficiency of the extremes, subject and predicate,
such that the subject is a thing or a determination for itself
and the predicate, too, is a universal determination
outside that subject, in my head somehow.
I then bring the predicate together with the subject
and, by this means, I judge.
However, since the copula 'is' asserts the predicate of the subject,
that external, subjective subsuming is sublated in turn
and the judgment is taken as a determination of the object itself.
The etymological meaning of 'judgment' in our language is
profounder and expresses the unity of the concept
as what comes first and its differentiation
as the original division that the judgment truly is.

The abstract judgment is the sentence:
'the individual is the universal'.
These are the determinations
that the subject and predicate
first have opposite one another,
in that the moments of the concept are taken
in their immediate determinacy or first abstraction.
(The sentences 'the particular is the universal' and
'the individual is the particular' belong to
the further determination of the judgment.)
It has to be viewed as an amazing lack of attentiveness that
in the logic books there is nowhere to be found [acknowledgment of]
the fact that in each judgment
one is articulating a sentence such as
'the individual is the universal'
or, even more determinately,
'the subject is the predicate'
(e.g. 'God is absolute spirit').
To be sure, the determinations, individuality and universality,
subject and predicate, are also distinct,
but on that account, nonetheless,
the completely universal fact remains that
each judgment asserts them as identical.

The copula 'is' comes from the concept's nature, namely,
to be identical with itself in its externalization.
The individual and the universal, as its moments,
are the sort of determinacies that cannot be isolated.
The earlier determinacies of reflection, in their relationships,
are equally related to one another,
but their connection is only that of having, not being,
the identity posited as such or the universality.
For this very reason, the judgment is
the true particularity of the concept,
since it is the determinacy or differentiation of the same,
a differentiation that, however, remains the universality.

Judgment is usually taken in the subjective sense
as an operation and form that surfaces
merely in self-conscious thinking.
This difference, however, is not yet on hand
in the logical [sphere, where]
judgment is supposed to be taken
in the completely universal sense:
all things are a judgment;
they are individuals which are a universality
or inner nature in themselves,
or a universal that is individuated.
The universality and individuality
distinguish themselves in them [the things]
but are at the same time identical.

The sense of the judgment that is supposed to be merely subjective,
as if it were I who attributes a predicate to a subject,
is contradicted by the objective expression of the judgment:
'the rose is red', 'gold is metal', and so forth;
I do not first attribute something to them.
Judgments are different from sentences;
the latter contain the determination of the subjects
that does not stand in a connection of universality with them,
a condition, an individual action, and the like;
'Caesar was born in Rome in such and such a year,
conducted the war in Gaul for ten years, crossed the Rubicon,
and so forth' are sentences, not judgments.
There is, furthermore, something quite empty
in saying that sentences of the sort, e.g.
'I slept well last night' or even 'Present arms!'
can be put into the form of judgments.
A sentence like 'a carriage is passing by' would be a judgment
and, to be sure, a subjective one only if it could be doubted
whether what is moving by is a carriage,
whether it is the object that is moving and not
the standpoint from which we are observing it;
where what then matters is finding the determination
for a representation not properly determined yet.

The standpoint of the judgment is finitude,
and from this standpoint the finitude of things consists
in the fact that they are a judgment,
that their existence and their universal nature
(their body and their soul) are certainly unified
(otherwise the things would be nothing),
but that these, their moments, are
both already diverse and generally able to be separated.

In the abstract judgment 'the individual is the universal',
the subject relates itself negatively to itself
and, as such, is the immediately concrete,
while the predicate is, by contrast,
the abstract, indeterminate, the universal.
But since they are joined by 'is',
the predicate in its universality
must also contain the determinacy of the subject
and it [that determinacy] is the particularity
and the latter is the posited identity of
the subject and predicate.
As thus indifferent to this difference of form,
it is the content.

Only in the predicate does the subject
have its explicit determinacy and content;
hence, taken by itself it is a mere representation or a bare name.
In the judgments 'God is the supremely real' and so forth
or 'the absolute is identical with itself' and so forth,
'God' and 'absolute' are mere names.
What the subject is, is first said in the predicate.
What it might otherwise also be as something concrete
does not matter to this judgment.

As far as the more precise determinacy
of subject and predicate is concerned,
the former, as the negative relation to itself,
is the underlying fixity in which
the predicate has its subsistence
and is in an ideal way
(it inheres in the subject).
Moreover, since the subject is
generally and immediately concrete,
the determinate content of the predicate is
only one of the many determinacies of the subject
and the latter is richer and broader than the predicate.

Conversely, the predicate, as the universal
subsisting for itself and indifferent to
whether this subject is or not,
goes beyond the subject, subsumes the subject under it,
and is, for its part, broader than the subject.
The determinate content of the predicate (see preceding section)
alone makes up the identity of both.

Subject, predicate, and the determinate content or the identity [of them]
are initially posited in the judgment, in their relation,
as themselves diverse, falling outside one another.
But in themselves, in terms of the concept, they are identical,
since the concrete totality of the subject is this,
not to be some sort of indeterminate manifold,
but instead individuality alone,
the particular and universal in an identity,
and precisely this unity is the predicate.
In the copula, furthermore, the identity of
the subject and predicate is of course posited
but initially only as the abstract 'is'.
In keeping with this identity, the subject is also
to be posited in the determination of the predicate,
by means of which the latter also acquires
the determination of the subject
and the copula is fulfilled.
This is the further determination of the judgment,
by means of the copula full of content, into the syllogism.
But first, in terms of the judgment,
there is the further determination of it,
the determining of the initially abstract,
sensory universality into a set of all, genus, and species
and into the developed universality of the concept.

Only knowledge of the further development of the judgment
gives a context as well as a sense to what are customarily
put forward as species of judgment.
In addition to appearing completely contingent,
the usual enumeration is superficial and
even barren and wild in the presentation of the differences.
In part, the manner in which positive, categorical, and assertoric
judgments are differentiated is generally pulled out of the air
and in part it remains undetermined.
The various judgments should be considered as
following necessarily from one another
and as a further determining of the concept,
since the judgment is nothing other
than the determinate concept.

In relation to the two previous spheres of being and essence,
the determinate concepts, qua judgments,
are reproductions of these spheres,
but posited in the simple relation of the concept.

YS III.12

tata puna-shanti-udita tulya-pratyaya cittasya-ekagrata-parinama

c. The syllogism

The syllogism is the unity of the concept and the judgment;
it is the concept as the simple identity
(into which the judgment's differences of form have gone back),
and [it is] judgment insofar as it is posited
at the same time in reality,
namely, in the difference of its determinations.
The syllogism is what is rational
and everything rational.

The syllogism tends to be put forward usually
as the form of the rational,
but as a subjective form and without pointing up
any sort of connection between it
and any other rational content, e.g.
a rational grounding principle, a rational action, idea, and so forth.
In general, there is much and frequent talk of reason
and appeal is made to it without indicating what it is,
what its determinacy is and without giving the slightest thought
to what inferring via syllogism is.
In fact, formally inferring via syllogism is
the rational in such a non-rational manner,
that it has nothing to do with a rational basic content.
Since, however, such a content can be rational
only through the determinacy through which thinking is reason,
it can be rational only through the form which the syllogism is.
This, however, is nothing else than
the posited, (at first formally) real concept,
as this section expresses.
The syllogism is, on account of this,
the essential ground of everything true;
and the definition of the absolute is from now on
that it is the syllogistic inference,
or, articulated in the form of a sentence,
it is this determinacy:
'everything is a syllogism'.
Everything is a concept,
and its existence is the difference of its moments,
so that its universal nature provides itself
with external reality through particularity
and, by this means and as negative reflection-in-itself,
makes itself something individual.
Or conversely, the actual is an individual
that by means of particularity
elevates itself into universality
and makes itself identical with itself.
The actual is one, but [it is] similarly
the segregation of the moments of the concept,
and the syllogism is the cyclical course
taken by the mediation of its moments,
a course through which it posits itself as one.

The immediate syllogism is such that
the determinations of the concept stand opposite one another
in an external connection as abstract determinations,
so that the two extremes [are] the individuality and universality,
but the concept, as the middle joining the two together,
is likewise only the abstract particularity.
The extremes are accordingly posited
as subsisting for themselves,
as indifferent to one another as
they are to the middle [term that joins them].
This syllogism is thus rational but non-conceptual;
it is the formal syllogism of the understanding.
In it the subject is joined together with another determinacy;
or through this mediation the universal
subsumes a subject external to it.
In a rational syllogism, by contrast,
the subject joins itself together with itself
by means of this mediation.
It is only a subject in this way,
or the subject is only in itself
the syllogism of reason.

In the following consideration,
the syllogism of the understanding is expressed
in terms of its ordinary, usual meaning, [namely,]
in the subjective manner attributed to it
in the sense that we make such syllogistic inferences.
In fact it is only a subjective inferring via syllogism,
though this has equally the objective meaning
that it expresses only the finitude of things,
but in the determinate manner that the form has attained here.
With respect to finite things, subjectivity as thinghood,
separable from its properties, its particularity, is
equally separable from its universality insofar as this is
the mere quality of the thing
and its external connection with other things
as its genus and concept.

B. THE OBJECT

The object is immediate being by virtue of
the indifference towards the difference
that has sublated itself in it.
It is in itself the totality
and, at the same time,
since this identity is the identity of the moments
but an identity that only is in itself,
it is just as indifferent to its immediate unity.
It breaks down into differentiated [moments],
each of which is itself the totality.
The object is thus the absolute contradiction of
the complete self-sufficiency of the manifold
and the equally complete lack of self-sufficiency
of the differentiated [moments].

The definition 'the absolute is the object'
is contained in the most determinate manner
in the Leibnizian monad which is supposed to be an object,
yet [an object] in itself representing [things]
and, indeed, is supposed to be the totality of
the representation of the world.
In its simple unity,
every difference is merely something ideal,
something not self-sufficient.
Nothing enters into the monad from the outside;
it is in itself the entire concept,
only differentiated by its own
greater or lesser development.
By the same token, this simple totality breaks down
into the absolute plurality of differences such
that they are self-sufficient monads.
In the monad of monads and the pre-established harmony
of their inner developments,
these substances are just as much reduced in turn
to the level of something ideal
and lacking in self~sufficiency.
The Leibnizian philosophy is thus
the perfectly developed contradiction.

YS III.13

etena bhutendriyesu dharma-laksana-vastha-parinama vyakhyata

a. Mechanism

The object, taken first in its immediacy, is
(1) the concept only in itself,
it has the concept at first
as something subjective outside it,
and every determinacy is posited
as an external determinacy.
As the unity of differences,
it is thus something composite, an aggregate,
and the effect on another remains
an external relation: formal mechanism.
In this relation and lack of self-sufficiency,
the objects remain equally self-sufficient,
resistant, external to one another.

Just as pressure and impulse are mechanical relationships,
so we also know in a mechanical way, by rote,
insofar as the words are devoid of any sense for us
and remain external to the senses, representing, thinking;
they [the words] are equally external to themselves,
a senseless sequence.
Acting, piety, and so forth are equally mechanical
insofar as what a person does is determined by
ceremonial laws, a counsel of conscience, and so forth,
while his own spirit and will are not in his actions,
such that these actions are external to him himself.

Only insofar as the object is self-sufficient
(see the preceding section)
does it have the lack of self-sufficiency
in terms of which it suffers violence.
Insofar as the object is the posited concept in itself,
neither of these determinations sublates itself
in its other determination;
instead the object joins itself together with itself
through the negation of itself,
through its lack of self-sufficiency,
and only then is it self-sufficient.
Thus, at the same time, in the difference from externality,
and in its self-sufficiency negating this externality,
it [the object] is the negative unity with itself, centrality,
subjectivity in which it is itself directed and related to the external.
The latter is equally centred in itself
and, in that, just as much related
to the other centre,
having its centrality just as much in the other.
[Hence, the object in the second place is]
(2) a differentiated mechanism
(fall, desire, social drive, and the like).

The development of this relationship
forms the syllogistic inference
that the immanent negativity
as the central individuality of an object (the abstract centre)
relates itself to objects lacking self-sufficiency
as the other extreme,
relating to them through a middle [term]
that unifies the objects' centrality
and lack of self-sufficiency,
the relative centre.
[Hence, the object is]
(3) absolute mechanism.

The syllogism that has been given here (I-P-U)
is a triad of syllogistic inferences.
The flawed individuality of the objects lacking self-sufficiency,
in which the formal mechanism is at home, is,
in keeping with its lack of self-sufficiency,
just as much the external universality.
These objects are thus the middle also
between the absolute and the relative centre
(the form of the syllogism: U-I-P).
For it is by means of this lack of self-sufficiency
that those two are separated and are extremes
just as they are related to one another.
So, too, the absolute centrality as the substantial universal
(the gravity that remains identical),
which as the pure negativity
also encapsulates in itself the individuality,
is the mediating factor between the relative centre
and the objects lacking self-sufficiency;
(thus amounting toJ the form of the inference P-U-I)
and, to be sure, just as essential
in terms of the immanent individuality
where it functions to separate,
as it is in terms of the universality
as the identical cohesion and
as the undisturbed being-in~itself.

Like the solar system, the state, for instance, is,
in the practical sphere, a system of three syllogisms.
(1) The individual (the person) joins itself through its particularity
(physical and spiritual needs, what becomes the civil society,
once they have been further developed for themselves)
with the universal (the society, justice, law, government).
(2) The will, the activity of individuals, is the mediating factor
which satisfies the needs in relation to society, the law, and so forth,
just as it fulfils and realizes the society, the law, and so forth.
(3) But the universal (state, government, law)
is the substantial middle [term]
in which the individuals and their satisfaction
have and acquire their fulfilled reality, mediation, and subsistence.
Since the mediation joins each of the determinations with the other extreme,
each joins itself precisely in this way together with itself;
it produces itself and this production is its self-preservation.
It is only through the nature of this joining together,
through this triad of syllogisms with the same terminis,
that a whole is truly understood in its organization.

The immediacy of concrete existence
that objects have in absolute mechanism
is in itself negated by the fact that
their self-sufficiency is mediated by
their relations to one another, hence,
through their lack of self-sufficiency.
Thus, the object must be posited as
differentiated in its concrete existence,
opposite its other.

YS III.14

shanta-udita-avyapadesya-dharma-anupati dharmi

b. Chemism

The differentiated object has an immanent determinacy
constituting its nature and in that determinacy
it has concrete existence.
But as the posited totality of the concept,
it is the contradiction of this its totality
and the determinacy of its concrete existence;
it is thus the [process of] striving to sublate this contradiction
and make its existence equal to the concept.

The chemical process thus has as its product
the neutral dimension of these strung-out extremes,
a neutral dimension which these extremes are in themselves;
by means of the differentiation of the objects (the particularization),
the concept, the concrete universal, joins itself
with the individuality, the product,
and so merely with itself equally contained
in this process are the other syllogisms;
the individuality, as activity, is likewise
the mediating factor just like the concrete universal,
the essence of the strung-out extremes,
which enters into existence in the product.

Chemism, as the reflexive relationship of objectivity,
still presupposes, together with
the differentiated nature of the objects,
the immediate self-sufficiency of those same objects.
The process is that of passing back and forth
from one form into the other,
forms determinate properties that the extremes
had opposite one another are sublated.
This is, indeed, in keeping with the concept;
but the animating principle of differentiating
does not exist concretely in it
since it has sunk back into immediacy.
For this reason, the neutral dimension
is a separable dimension.
Yet the judging principle that severs
the neutral dimension into differentiated extremes
and gives the undifferentiated objects in general
their difference and animation opposite an other
falls outside that first process,
and so does the process as
the separation that strings things out.

The externality of these two processes,
the reduction of what are differentiated to something neutral
and the differentiation of the undifferentiated or neutral,
which allows them to appear as self-sufficient opposite one another,
shows its finitude in passing over into products
in which they are sublated.
Conversely, the process presents the presupposed immediacy
of the differentiated objects as a vacuous immediacy.
By means of this negation of externality and immediacy,
into which the concept as object was immersed,
it is posited freely and for itself
opposite that externality and immediacy,
as purpose.

YS III.15

krama-anyatva parinama-anyatve hetu

c. Teleology

Purpose is the concept that is for itself
and that has entered into a free concrete existence
via the negation of immediate objectivity.
It is determined as something subjective,
in that this negation initially is abstract
and thus objectivity also only stands
over against it [i.e. the purpose] at first.
In contrast to the totality of the concept, however,
this determinacy of the subjectivity is one-sided and, indeed,
for it [the purpose] itself, since all determinacy has posited
itself as sublated in it.
Thus, too, for it [the purpose] the presupposed object is
only an ideal, in itself vacuous reality.
As this contradiction of its identity
with itself opposite the negation
and the opposition posited in it,
it is itself the sublating,
the activity of so negating the opposition
that it posits it as identical with itself.
This is the process of realizing the purpose in which,
by rendering itself something other than its subjectivity
and objectifying itself,
it has sublated the difference of both,
has joined itself together only with itself
and has preserved itself.

The concept of purpose is, on the one hand, superfluous;
on the other hand, it is rightly labelled a concept of reason
and contrasted with the understanding's abstract-universal
that relates itself to the particular
(which it does not have in itself)
only by way of subsuming it.
Furthermore, the difference of the purpose
as the final cause
from the merely efficient cause
(what is ordinarily called the cause)
is of the utmost importance.
The cause pertains to the not yet uncovered, blind necessity;
for this reason it appears to pass over into its other
and lose its originality in it in the course of being posited.
Only in itself or for us is the cause in the effect first a cause,
and does it come back into itself.
The purpose, by contrast, is posited as in itself the determinacy,
or what there [in efficient causality] still appears as being-other
contains the effect [here], so that, in its efficacy,
it does not pass over [into something else]
but instead preserves itself.
That is to say, it brings about itself alone
and is, in the end, what it was in the beginning,
in the original state.
What is truly original is so only
by means of this self-preservation.
The purpose requires a speculative construal,
as the concept that itself,
in its own unity and in the ideality of its determinations,
contains the judgment or the negation,
the opposition of the subjective and the objective,
and is equally the sublating of them.

With regard to the purpose,
one should not immediately
or should not merely think of
the form in which it is in consciousness,
as a determination on hand in the representation.
Through the concept of inner purposiveness,
Kant re-awakened the idea in general
and that of life in particular.
Aristotle's determination of life
already contains the inner purposiveness
and thus stands infinitely far beyond
the concept of modern teleology
which has only the finite,
the external purposiveness in view.

Need and drive are the examples of purpose lying closest at hand.
They are the flit contradiction that takes place within the living
subject itself and they enter into the activity of negating this
negation that is still mere subjectivity.
The satisfaction produces the peace between the subject and object,
in that the objective dimension standing over there
in the still on hand contradiction (to the need)
is equally sublated with respect to this, its one-sidedness,
through the unification with the subjective dimension.
Those who speak so much of the solidity and invincibility of
the finite have an example of the opposite in every drive.
The drive is, so to speak, the certainty that
the subjective dimension is only one-sided
and has just as little truth as the objective dimension.
The drive is, furthermore, the implementation of this, its certainty.
It manages to sublate this opposition;
that the subjective dimension
would be and remain only something subjective,
just as the objective dimension
would equally be and remain only something objective;
and [to sublate] this finitude of them.

With regard to the activity of the purpose,
attention may also be drawn to the fact that,
in the syllogism that conjoins the purpose with itself
through the means of the realization,
the negation of the termini surfaces;
the just mentioned negation of immediate subjectivity
that surfaces in the purpose as such,
like that of the immediate objectivity
(of the means and the presupposed objects).
This is the same negation that is exercised
in the elevation of the spirit to God
in contrast to the finite things of the world
as much as in contrast to one's own subjectivity.
This is the moment which is overlooked
and left aside in the form of the syllogisms
at the level of the understanding,
the form that is given to this elevation
in the so called proofs of God's existence.

The teleological relation in its immediacy is
initially the external purposiveness,
and the concept is opposite the object
which is something presupposed.
The purpose is thus finite,
partly in terms of the content,
partly in terms of the fact that
it has an external condition in an extant object
as the material of its realization.
To this extent, its self-determination is merely formal.
The immediacy entails, more precisely, that the particularity
(as a determination of form, the subjectivity of the purpose)
appears as reflected in itself,
the content as distinct from the totality of the form,
the subjectivity in itself, the concept.
This diversity constitutes the finitude of
the purpose within itself.
The content is, by this means,
as limited, contingent, and given
as the object is something particular and extant.

The teleological relation is the syllogism
in which the subjective purpose joins itself
together with the objectivity external to it
through a middle term that is the unity of the two,
both as the purposive activity and as the objectivity
immediately posited under the purpose, the means.

Thus, the realized purpose is the posited unity of
the subjective and the objective dimensions.
This unity, however, is essentially determined in such a way that
the subjective and objective dimensions are neutralized
and sublated only with respect to their one-sidedness,
while the objective dimension is subjected
and made to conform to the purpose as the free concept
and, thereby, to the power over it.
The purpose preserves itself
against and in the objective dimension
because, in addition to being
the one-sided subjective dimension (the particular),
it is also the concrete universal,
the identity of both, that is in itself.
This universal, that as simple is reflected in itself,
is the content that remains the same
through all three termini [terms] of
the syllogism and their movement.

In the finite purposiveness, however,
the purpose carried out is also
something as internally broken
as was the middle term and the initial purpose.
What has come about is thus only a form
posited externally in the material found before it,
a form that, on account of the restricted content of
the purpose, is likewise a contingent determination.
The purpose attained is thus only an object
that is also in turn a means
or material for other purposes
and so on ad infinitum.

What happens, however, in the process of
realizing the purpose in itself is
that the one-sided subjectivity
and the semblance of objective self-sufficiency
on hand opposite it are sublated.
In seizing the means, the concept posits itself
as the object's essence as it is in itself,
in the mechanical and chemical process,
the self-sufficiency of the object
has already evaporated in itself
and in the course it takes under
the dominance of the purpose,
the semblance of that self-sufficiency,
the negative dimension opposite the concept,
sublates itself.
Yet this object is immediately
already posited as vacuous in itself,
as only ideal by virtue of the fact
that the executed purpose is determined
only as means and material.
With this, the opposition of content and form
has vanished as well.
Since the purpose, by sublating the formal determinations,
joins itself together with itself,
the form is posited as identical with itself,
thus as content,
so that the concept as the activity of
the form has only itself as content.
It is thus posited through this process generally
what the concept of the purpose was:
the unity, being in itself,
of the subjective and the objective dimensions
now posited as being for itself:
the idea.

C. THE IDEA

The idea is the true in and for itself,
the absolute unity of the concept and objectivity.
Its ideal content is none other than
the concept in its determinations.
Its real content is only its exhibition,
an exhibition that it provides for itself
in the form of external existence
and, with this shape incorporated into
the concept's ideality and in its power,
the concept thus preserves itself in that exhibition.

The definition of the absolute,
that it is the idea, is itself absolute.
All previous definitions go back to this one.
The idea is the truth;
for the truth is this,
that objectivity corresponds to the concept,
not that external things correspond
to my representations;
these are only correct representations
that I, this person, have.
In the idea it is not a matter of an indexical this,
it is a matter neither of representations nor of external things.
But everything actual, insofar as it is something true,
is also the idea and possesses its truth
only through and in virtue of the idea.
The individual being is some side or other of the idea,
but for this still other actualities are needed,
actualities that likewise appear
as obtaining particularly for themselves;
the concept is realized only in them together
and in their relation.
The individual taken by itself
does not correspond to its concept;
this limitation of its existence constitutes
its finitude and its demise.

The idea itself is no more to be taken as an idea of something or other
than the concept is to be taken merely as a determinate concept.
The absolute is the universal idea and the one idea that, by judging,
particularizes itself into a system of determinate ideas:
ideas, however, that are only this,
the process of going back into the one idea, their truth.
On the basis of this judgment, the idea is at the outset only
the one, universal substance,
but its developed, true actuality is
that it is as subject and thus as spirit.

The idea is frequently taken for
something logical in a merely formal sense,
insofar as it does not have some concrete existence
as its point of departure and support.
One must leave such a view to the standpoints for which
the concretely existing thing and all further determinations
that have not yet penetrated to the idea still count as
so-called realities and true actualities.
Equally false is the representation of the idea
as though it were only something abstract.
It is this, of course, insofar as
everything untrue is consumed in it.
However, in itself it is essentially concrete
since it is the free concept,
the concept determining itself
and thereby determining itself as reality.
It would be something formally abstract
only if the concept that is its principle were taken as
the abstract unity and not as it is, namely,
as the negative return of it into itself
and as subjectivity.

The idea can be grasped as reason
(this is the genuine philosophical meaning of reason),
further as subject-object,
as the unity of the ideal and the real,
of the finite and the infinite,
of the soul and the body,
as the possibility that has
its actuality in itself,
as that the nature of which can
only be conceived as existing,
and so forth,
because in it [the idea]
all relationships of the understanding are contained,
but in their infinite return and identity in themselves.

The understanding makes easy work of pointing out
that everything said of the idea is self-contradictory.
This can be equally conceded to it or rather
it is already accomplished in the idea;
a work that is the work of reason and, of course,
not as easy as that of the understanding.
The understanding shows that the idea is
self-contradictory because, for example,
the subjective dimension is only subjective
and the objective dimension, by contrast, is opposed to it;
because being is something completely different from the concept
and thus cannot be plucked from it;
similarly, because the finite is only finite
and precisely the opposite of the infinite,
and consequently is not identical with it
and so on for all determinations.
If the understanding thus shows that
the idea is self-contradictory,
the [science of] logic points out the opposite instead, namely,
that the subjective dimension
that is supposed to be merely subjective
lacks any truth, contradicts itself,
and passes over into its opposite,
as does the finite that is supposed to be merely finite,
the infinite that is supposed to be merely infinite, and so on.
By this means, the process of passing over into its opposite
and the unity in which the extremes are as something sublated,
as a shining or as moments, reveals itself as their truth.

The understanding that tackles the idea
suffers from a twofold misunderstanding.
In the first place, it takes the extremes of the idea,
however they may be expressed, insofar as they are in their unity,
yet in the sense and determination proper to them
insofar as they are not in their concrete unity
but instead are still abstractions outside it.
The understanding mistakes no less the relation,
even if it is already posited explicitly.
In this way the understanding overlooks, for example,
the nature of the copula in a judgment,
which asserts of the individual, the subject,
that the individual is just as much
something not individual
but instead something universal.
In the second place, the understanding
holds its reflection that
the idea that is identical with itself
contains the negative of itself
(that it contains the contradiction)
to be an external reflection,
that does not fall to the idea itself.
In fact, however, this is not a wisdom
proper to the understanding.
The idea is instead itself the dialectic that
eternally separates and distinguishes what is
identical with itself from the differentiated,
the subjective from the objective,
the finite from the infinite,
the soul from the body and,
only insofar as it does,
is it eternal creation,
eternally alive,
and eternal spirit.
Because it is thus itself the passing over
or rather the transposing of itself
into abstract understanding,
it is also eternally reason.
It [the idea] is the dialectic that
takes what is understandable in this [superficial] way,
including the diversity of its finite nature
and the false semblance of self-sufficiency of its productions
and renders it understandable in a recursive way
and leads it back to unity.
Since this twofold movement is
neither temporal nor separate and distinct in any way,
otherwise it would be again only abstract understanding,
it is the process of eternally intuiting itself in the other;
the concept that has carried itself out in its objectivity,
the object that is inner purposiveness, essential subjectivity.

The diverse ways of construing the idea,
as unity of the ideal and the real,
of the finite and infinite,
of identity and difference,
and so on, are more or less formal,
since they designate some sort of
stage of the determinate concept.
Only the concept itself is free and the truly universal;
in the idea its determinacy is thus equally only itself,
an objectivity into which it, as the universal,
continuously sets itself and in which it has only
its own determinacy, the total determinacy.
The idea is the infinite judgment,
each of the sides of which is
the self-sufficient totality and,
precisely by virtue of completing itself to this end,
has just as much passed over into the other.
None of the other determinate concepts is
this totality completed in its two sides,
except the concept itself and the objectivity.

The idea is essentially a process since its identity is
that of the absolute and free concept only insofar as
it is the absolute negativity and thus dialectical.
It is the course in which the concept
as the universality that is individuality
determines itself to be objectivity
and to be the opposite of objectivity,
and in which this externality
that has the concept as its substance leads
itself back into subjectivity
through its immanent dialectic.

Because the idea is (a) a process, the expression
'the unity of the finite and infinite,
of thinking and being, and so on',
as an expression for the absolute,
is false, as often noted.
For this unity expresses
an abstract, calmly enduring identity.
The expression is likewise false
because the idea is (b) subjectivity,
since that unity expresses the in itself,
the substantial dimension of the true unity.
The infinite thus appears as only
neutralized relative to the finite,
and so too the subjective relative to the objective,
thinking relative to being.
But in the negative unity of the idea
the infinite reaches over and beyond the finite,
as does thinking over being,
subjectivity over objectivity.
The unity of the idea is
subjectivity, thinking, infinity,
and hence it is essentially distinct
from the idea as substance
just as this overreaching subjectivity (thinking, infinity)
is to be distinguished from the one-sided subjectivity
(one-sided thinking, one-sided infinity)
to which it reduces itself in
judging and making determinations.

YS III.16

parinama-traya-samyamad atita-nagata-jnana

a. Life

The immediate idea is life.
The concept is realized as the soul in a body;
the soul is the immediate, self-referring
universality of the body's externality,
just as much as it is the body's particularization,
so that the body expresses no other differences
than the determination of the concept,
and finally it is the individuality as infinite negativity;
the dialectic of the body's objectivity,
[the factors of which are] outside one another,
an objectivity that is led back into subjectivity
from the semblance of self-sufficient subsistence,
so that all members are reciprocally momentary means
as much as momentary purposes,
while life, inasmuch as it is the inceptive particularization,
results in itself as the negative unity that is for itself
and, in the dialectic of embodiment,
joins itself together only with itself.
Life is thus essentially a living entity
and, with regard to its immediacy,
this individual living entity.
In this sphere, finitude has the determination
that soul and body are separable,
on account of the immediacy of the idea;
this constitutes the mortality of the living.
But those two sides of the idea are diverse component parts
only insofar as it is dead.

The living is the syllogism, whose moments are
systems and syllogisms in themselves
which, however, are active syllogisms, processes, and
in the subjective unity of the living,
they are only one process.
The living is thus the process of
its coming to closure together with itself,
that runs its course by means of three processes.

The process of the genus brings this [genus] to
the point of being-for-itself.
Because life is still the immediate idea,
the product of the process breaks down into two sides.
On the one side, the living individual in general,
at first presupposed as immediate,
emerges now as something mediated and produced.
On the other side, however, the living individuality that,
on account of its initial immediacy,
behaves negatively towards the universality,
perishes in this [universality] as the power.

By this means, however, the idea of life
has not only freed itself from just
any (particular) immediate 'this',
but from this initial immediacy altogether.
In this way, it comes to itself, to its truth,
entering into concrete existence
as the free genus for itself.
The death of the merely immediate,
individual living thing is the Spirit emerging.

YS III.17

sabda-artha-pratyayanam itaretara-adhyasat sankaras
tat-pravibhaga-samyama sarva-bhuta-ruta-jnana

YS III.18

samskara-saksa-karana purva-jati-jnana

YS III.19

pratyayasya para-citta-jnana

YS III.20

na ca tat salambana tasya-avisayi-bhutatva

b. Knowing

The idea concretely exists freely for itself
insofar as universality is the element
in which it exists concretely
or insofar as it is objectivity itself as the concept;
[that is to say,] the idea has itself for an object.
Its subjectivity, determined as universality,
is pure differentiating within it
intuiting that keeps itself in this identical universality.
But, as a differentiating in a determinate way,
it is the further judgment of thrusting itself as
a totality away from itself
and, indeed, initially
presupposing itself as the external universe.
These are two judgments that are in themselves identical
but not yet posited as identical.

The relation of these two ideas
that are identical in themselves
or as life is thus the relative relation
that makes up the determination of finitude in this sphere.
It is the relationship of reflection,
since the differentiation of the idea
in it [the idea] itself is only the first judgment,
the presupposing is not yet a positing,
and thus, for the subjective idea,
the objective dimension is the extant immediate world
or the idea as life in the appearance of individual concrete existence.
At the same time, insofar as this judgment is a pure differentiating
within it [the idea] itself (see the preceding section),
the idea is for itself both itself and its other.
Thus it is the certainty of being in itself
the identity of this objective world with it.
Reason comes to the world with the absolute faith
in its capacity to posit the identity
and elevate its certainty to truth,
and with the drive to posit as
also vacuous for it that opposition
that is in itself vacuous.

In general, this process is knowing.
In it, in one activity, the opposition,
the one-sidedness of subjectivity together with
the one-sidedness of objectivity,
is sublated in itself.
But this process of sublating takes place
at the outset only in itself.
The process as such is thus itself
immediately beset with the finitudeof this sphere
and falls apart into the twofold, diversely posited
movement of the drive.
[In one respect,] it is the drive to sublate
the one-sidedness of the subjectivity of the idea
by taking up into itself the world that is,
taking it up into subjective representing and thinking,
and to fill out the abstract certainty of itself
with this objectivity as content,
an objectivity that thus counts as true.
Conversely, it is the drive to sublate
the one-sidedness of the objective world
that here accordingly, by contrast, counts as a semblance,
a collection of contingencies and shapes vacuous in themselves,
and to determine and mould it through
the inner dimension of the subjective,
that counts here as the objective, as what truly is.
The former is the drive of knowledge to truth, knowing as such,
the theoretical (activity);
the latter is the drive of the good to bring itself about, willing,
the practical activity of the idea.

(a) Knowing

The universal finitude of knowing that lies in the first judgment,
the presupposition of the opposition, which its very action contradicts,
specifies itself more precisely in its own idea in this direction,
that its moments receive the form of diversity from one another and,
since those moments are in fact complete, they come to stand in
the relationship of reflection, not of the concept, to one another.
The assimilation of the material as something given thus appears as
a way of taking it up into conceptual determinations that at the same time
remain external to it, determinations that likewise display themselves
opposite one another as diverse.
It is reason active as understanding.
The truth that this knowing comes to is thus likewise only finite;
the infinite truth of the concept is fixed as a goal that is only in itself,
something beyond this knowing.
But in its external action, it stands under the guidance of the concept,
and conceptual determinations make up the inner thread of the progression.

Because it presupposes what is differentiated
as a being that is found to be already on hand, standing opposite it
(the manifold facts of external nature or of consciousness),
finite knowing has
(1) the formal identity or
the abstraction of universality
as the form of its activity at the outset.
This activity thus consists in
dissolving the given concrete dimension,
individuating its differences,
and giving them the form of abstract universality;
or in leaving the concrete dimension as the ground and,
through abstraction from the particularities that seem inessential,
extracting a concrete universal,
the genus or the force and the law.
Such is the analytic method.

This universality is
(2) also a determinate one.
The activity here proceeds according to the moments of
the concept that, in finite knowing, is not in its infinity
but is the understandable, determinate concept instead.
Taking up the object into the forms of the latter concept
is the synthetic method.

(aa) Knowing initially puts the object into the form of
the determinate concept in general so that, by this means,
its genus and its universal determinacy are posited.
The respective object is the definition.

Its material and justification are procured by the analytic method.
The determinacy is, nevertheless, supposed to be only a characteristic,
that is to say, something to assist merely subjective knowing
that is external to the object.

(bb) The account of the second moment of the concept,
the determinacy of the universal as particularization,
is given by the division in terms of some sort of external aspect.

(cc) In the concrete individuality
(such that the simple determinacy in the definition
is construed as a relationship),
the object is a synthetic relation of
differentiated determinations, a theorem.
Because they are diverse, their identity is a mediated identity.
The process of supplying the material that constitutes
the middle members is the construction;
and the mediation itself,
out of which the necessity of
that relation for knowing goes forth,
is the proof.

The necessity which finite knowing produces in a proof
is initially an external necessity,
determined only for the subjective discernment.
But in the necessity as such, it has itself left behind
its presupposition and point of departure,
the finding and givenness of its content.
The necessity as such is, in itself,
the concept relating itself to itself.
The subjective idea has thus, in itself,
come to what is determined in and for itself,
what is not given, and is thus immanent to it as the subject.
As such, it passes over into the idea of willing.

(b) Willing

The subjective idea, as what is determinate in and for itself,
the simple, self-same content, is the good.
Its drive of realizing itself inverts the relationship
that holds relative to the idea of the true,
and is bent on determining, in terms of its purpose,
the world that it finds.
This willing is, on the one hand,
certain of the vacuousness of the presupposed object
but, on the other hand, as finite,
it at the same time presupposes
both the purpose of the good as a merely subjective idea
and the independence of the object.

The finitude of this activity is thus the contradiction that,
in the self-contradicting determinations of the objective world,
the purpose of the good is both carried out and not carried out,
and that it is posited as something inessential
just as much as something essential,
as something actual and at the same time
as merely possible.
This contradiction presents itself
as the endless progression in the actualization of the good,
that is therein established merely as an ought.
Formally, however, this contradiction disappears
in that the activity sublates the subjectivity of the purpose
and thereby the objectivity,
the opposition through which both are finite,
and not only the one-sidedness of this subjectivity
but subjectivity in general;
another such subjectivity, that is to say,
a new generation of the opposition,
is not distinct from what was supposed to be an earlier one.
This return into itself is at the same time
the recollection of the content into itself,
which is the good and the identity in itself of both sides,
the recollection of the presupposition of the theoretical stance,
that the object is what is substantial in itself and true.

The truth of the good is, by this means, posited as
the unity of the theoretical and practical idea,
[the notion] that the good has been attained in and for itself,
that the objective world is thus in and for itself the idea precisely as
it [the idea] at the same time eternally posits itself as purpose and
through activity produces its actuality.
This life, having come back to itself from
the differentiation and finitude of knowing,
and having become identical with the concept
through the activity of the concept,
is the speculative or absolute idea.

YS III.21

kaya-rupa-samyamat tad-grahya-sakti-stambhe
caksu-prakasa-asamprayoge 'ntardhanam

YS III.22

etena sabda-adi-antardhanam uktam

c. The absolute idea

The idea as the unity of
the subjective and the objective idea is
the concept of the idea,
for which the idea as such is the object,
for which it is the object
an object into which
all determinations have gone together.
This unity is accordingly
the absolute and entire truth,
the idea thinking itself,
and here, indeed, as thinking,
as the logical idea.

The absolute idea is for itself,
since in it there is no transition or presupposing
and no determinacy at all that is not fluid and transparent;
it is the pure form of the concept
that intuits its content as itself.
It is content for itself
insofar as it is the ideal differentiating of
itself from itself,
and one side of what has been differentiated is
the identity with itself,
in which, however, the totality of the form is
contained as the system of the determinations of the content.
This content is the system of the logical.
Nothing remains here of the idea, as form,
but the method of this content,
the determinate knowledge of the validity of its moments.

The moments of the speculative method are

(a) The beginning, which is being or the immediate;
for itself for the simple reason that it is the beginning.
From the vantage point of the speculative idea, however,
it is the speculative idea's self-determining which,
as the absolute negativity or movement of the concept,
judges and posits itself as the negative of itself.

Being, which from the vantage point of the beginning as such
appears as abstract affirmation, is thus instead the negation,
positedness, being-mediated in general and being presupposed.
But as the negation of the concept that is
simply identical with itself in its otherness
and is the certainty of itself,
it is the concept not yet posited as concept
or, in other words,
it is the concept in itself.
For that reason, as the still undetermined concept,
i.e. the concept determined only in itself or immediately,
this being is just as much the universal.

(b) The progression is the posited judgment of the idea.
The immediate universal, as the concept in itself,
is the dialectic of reducing, within itself,
its immediacy and universality to a moment.
It is accordingly the negative [aspect] of the beginning
or the first [moment] posited in its determinacy;
it is for something,
the relation of what has been differentiated,
the moment of reflection.

The abstract form of the progression
within the stage of being is
[to be] an other and a passing over into an other;
in the stage of the essence,
it is the shining in something opposite;
in the stage of the concept,
it is the differentiated status of
the individual from the universality
which continues itself as such in
what is differentiated from it
and is as an identity with the latter.

In the second sphere, the concept at first
being in itself came to shine forth
and is thus in itself already the idea.
The development of this sphere becomes
the return to the first,
just as the development of the first sphere is
a transition into the second.
Only by means of this double movement is
justice done to the difference,
since each of the two differentiated factors,
each considered in itself,
completes itself so as to form the totality
and, in that totality,
puts itself into unity with the other.
Only the fact that both sublate the one-sidedness
in themselves prevents the unity from becoming one-sided.

The second sphere develops the relation of
what has been differentiated
into what the relation is at first,
namely a contradiction in the relation itself
in the infinite progression.

(c) This contradiction resolves itself into the end,
where the differentiated is posited as what it is in the concept.
It is the negative of the first,
and, as the identity with the latter,
the negativity of itself.
Hence, it is the unity
in which these first two,
as ideal and as moments,
are sublated, i.e. preserved at the same time.
The concept, starting out from its being-in-itself,
thus comes to a close with itself
by means of its difference
and the process of sublating that difference.
This concept is the realized concept,
the concept that contains the positedness
of its determinations in its being-for-itself;
it is the idea for which, as the absolutely first (in the method),
this end is at the same time
nothing more than the process
by which the semblance that
the beginning is something immediate
and it [the idea] a result vanishes;
in other words, this end is the knowledge
that the idea is the one totality.

In this way, the method is not an external form
but the soul and concept of the content,
from which it is distinguished only insofar as
the moments of the concept, even in themselves,
in their [respective] determinacy,
come to appear as the totality of the concept.
Insofar as this determinacy or the content, with the form,
leads itself back to the idea,
this idea exhibits itself as
the systematic totality which is only one idea,
the particular moments of which are
in themselves this same idea
to the same extent that they bring forth
the simple being-for-itself of the idea
through the dialectic of the concept.
The science concludes in this way
by grasping the concept of itself
as the pure idea,
for which the idea is.

The idea, which is for itself,
considered in terms of this,
its unity with itself,
is the process of intuiting
and the idea insofar as it intuits is nature.
As intuiting, however,
the idea is posited by external reflection
in a one-sided determination of immediacy or negation.
Yet the absolute freedom of the idea is
that it does not merely pass over into life
or let life shine in itself as finite knowing,
but instead, in the absolute truth of itself,
resolves to release freely from itself
the moment of its particularity
or the first determining and otherness,
the immediate idea, as its reflection,
itself as nature.
