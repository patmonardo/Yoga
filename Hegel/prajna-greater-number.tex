
A. NUMBER

Quantity is quantum, or has a limit,
both as continuous and discrete magnitude.
The distinction between these two species has here,
in the first instance, no significance.

As the sublated being-for-itself,
quantity is already in and for itself
indifferent to its limit.
But, equally, the limit or to be a quantum is
not thereby indifferent to quantity;
for quantity contains within itself as its own moment
the absolute determinateness of the one,
and this moment, posited in the continuity
or unity of quantity, is its limit,
but a limit which remains as the one
that quantity in general has become.

This one is therefore the principle of quantum,
but as the one of quantity.
For this reason it is,
first, continuous, it is a unity;
second, it is discrete, a plurality
(implicit in continuous magnitude
or posited in discrete magnitude)
of ones that have equality with one another,
the said continuity, the same unity.
Third, this one is also the negation of
the many ones as a simple limit,
an excluding of its otherness from itself,
a determination of itself in opposition to other quanta.
The one is thus
(a) self-referring,
(b) enclosing, and
(c) other-excluding limit.

Thus completely posited in these determinations,
quantum is number.
The complete positedness lies in
the existence of the limit as a plurality
and so in its being distinguished from the unity.
Number appears for this reason as a discrete magnitude,
but in unity it has continuity as well.
It is, therefore, also quantum in complete determinateness,
for in it the limit is the determinate plurality
that has the one, the absolutely determined, for its principle.
Continuity, in which the one is only implicitly present
as a sublated moment [posited as unity] is
the form of the indeterminateness.

Quantum, only as such, is limited in general;
its limit is its abstract, simple determinateness.
But as number, this limit is posited as in itself manifold.
It contains the many ones that make up its existence,
but does not contain them in an indeterminate manner,
for the determinateness of the limit falls rather in it;
the limit excludes the existence of other ones,
that is, of other pluralities,
and those which it encloses are a determinate aggregate:
they are the amount or the how many times with respect to which,
taken as discreteness in the way it is in number,
the other is the unit, the continuity of the amount.
Amount and unit constitute the moments of number.

Regarding amount, we must examine yet more closely
how the many ones in which it consists are in the limit.
It is rightly said of amount that it consists of the many,
for the ones are not in it as sublated but are rather present in it,
only posited with the excluding limit
to which they are indifferent.
But the limit is not indifferent to them.
In the sphere of existence, the limit was
at first so placed in relation to existence
that the latter was left on this side of its limit,
standing there as the affirmative,
while the limit, the negation,
stood outside on the border of existence;
similarly, with respect to the many ones,
their being truncated and the exclusion of the remaining ones
appears in them as a determination
that falls outside the enclosed ones.
But it was found in that sphere of existence
that the limit pervades existence,
that it extends so far as existence does,
and that the something is for this reason
limited by its very determination, that is, is finite.
Now, in the quantitative sphere, a number, say a hundred,
is so represented that only the hundredth unit brings to the many
the limit that makes them a hundred.
In one respect this is correct;
but, in another respect, none of the ones in the hundred
has precedence over any other, for they are only equal;
each is just as much the hundredth;
they all thus belong to the limit that
makes the number a hundred;
this number cannot dispense with
any of them for its determinateness;
with respect to the hundredth, therefore,
the rest do not constitute a determinate existence
which is in any way different from it
whether inside or outside the limit.
Consequently, the number is not a plurality
over against the limiting one that encloses it,
but itself constitutes this delimitation
which is a determinate quantum;
the many constitute a number,
a one, a two, a ten, a hundred, etc.

Now, the limiting one is a discriminating determinateness,
the distinguishing of a number from other numbers.
But this distinguishing does not become
a qualitative determinateness;
it remains rather quantitative,
falling only within the compass
of comparing, of external reflection;
as one, number remains turned back onto itself
and indifferent to others.
This indifference of number to others
is its essential determination;
it constitutes its being-determined-in-itself,
but at the same time also its own exteriority.
Number is thus a numerical one that is absolutely determined
but which has at the same time the form of simple immediacy,
and to which, therefore, the connecting reference to
an other remains completely external.
Further, as numerical, the one possesses the determinateness
(such as consists in the reference to other) as a moment in it,
in its distinction of unit and amount;
and amount is itself the plurality of the ones,
that is, this absolute exteriority is in the one itself.
This intrinsic contradiction of number
or of quantum in general is the quality of quantum,
and the contradiction will develop in
the further determinations of this quality.

B. EXTENSIVE AND INTENSIVE QUANTUM

a. Their difference

1. We have seen that quantum has
its determinateness as limit in amount.
In itself quantum is discrete, a plurality
which does not have a being different
from its limit or its limit outside it.
Quantum, thus with its limit
which as limit is a plurality,
is extensive magnitude.

Extensive magnitude is to be distinguished from continuous magnitude;
its direct opposite is not the discrete, but the intensive magnitude.
Extensive and intensive magnitudes are determinacies of
the quantitative limit itself,
whereas quantum is identical with its limit;
continuous and discrete magnitudes are, on the contrary,
determinations of magnitude in itself,
that is, of quantity as such, in so far as
in quantum abstraction is made from the limit.
Extensive magnitude has the moment of continuity in it
in its limit, for its many is everywhere continuous;
the limit as negation appears, therefore,
in this equality of the many as a limiting of the unity.
Continuous magnitude is quantity that continues
without regard to any limit,
and in so far as it is represented with one such limit,
the latter is a limitation in general,
without discreteness being posited in it.
Determined as only continuous magnitude,
quantum is not yet determined for itself
because the magnitude lacks the one
(in which the determinateness-for-itself lies)
and number.
Similarly, a discrete magnitude is
immediately only a differentiated many in general
which, if it were to have a limit as a many,
would be only an aggregate,
that is, would be only indeterminately limited;
in order for quantum to be determinate,
the many must be concentrated
into one and thereby be posited as identical with the limit.
Continuous and discrete magnitude, taken as quantum in general,
have each posited in it only one of the two sides
by virtue of which quantum is fully determined and a number.
Taken immediately, this latter is extensive quantum
the simple determinateness which is amount essentially,
but the amount of one and the same unit;
extensive quantum is distinguished from number
only because in the latter the determinateness is
explicitly posited as plurality.

2. However, the determinateness through number
[how much there is of something]
does not require being distinguished from
how much there is of something else,
as if to the determinateness of one thing belonged
how much there is of it
and how much there is of an other,
for the determinateness of magnitude as such
is a limit determined for itself,
indifferent and simply self-referring;
and in number this limit is posited
as enclosed within the one existing for itself:
the externality that it has,
the reference to other, is inside it.
Further, like the many in general,
this many of the limit is
not internally unequal but continuous;
each many is what any other many is;
consequently, the many as a many of
existents outside one another,
or as discrete, does not constitute
the determinateness as such.
Thus this many collapses for itself
into its continuity and becomes simple unity.
Amount is but a moment of number;
but, as an aggregate of numerical ones,
it does not constitute the determinateness of number;
on the contrary, these ones as indifferent and self-external are
sublated in number whose being has turned back into itself;
the externality that constituted the ones of plurality vanishes
in the one as the self-reference of number.

The limit of quantum which, as extensive,
had its existent determinateness as self-external number,
thus passes over into simple determinateness.
In this simple determination of limit,
quantum is intensive magnitude;
and the limit or the determinateness
which is identical with quantum is now
also posited as simple: it is degree.
Degree is thus a determinate magnitude, a quantum,
but at the same time it is not an aggregate
or several within itself;
it is only a plurality;
plurality is a severality that has gathered
together into simple determination,
it is existence that has returned into being-for-itself.
It is true that its determinateness must be expressed by a number,
which is the being of the quantum as completely determined,
but the number is not an amount or a how many times
but is rather a onefold, only a degree.
When we speak of 10 or 20 degrees,
the quantum which has that many degrees
[the tenth, the twentieth degree]
is not the amount and sum of the degrees;
if that were the case, it would be an extensive quantum;
it is rather only that one degree, the tenth, the twentieth.
It does contain the determinateness found in
the number ten or twenty,
but not as several ones:
the number is there as a sublated amount,
as a simple determinateness.

3. In number, quantum is posited in its complete determinateness;
but as intensive quantum, in the being-for-itself of number,
it is posited as it is according to its concept or implicitly in itself.
For the form of self-reference which it has in degree is
at the same time the externality of degree to itself.
As extensive quantum, number is a numerical plurality
and thus has externality inside it.
This externality, as plurality in general,
collapses into a state of undifferentiatedness
and is sublated into the numerical one,
the self-reference of number.
But quantum has its determinateness as number;
it contains this determinateness, as we have said,
whether the latter is posited in it or not.
Degree, therefore, which as internally simple no
longer has this external otherness in it, has it outside it,
and refers to it as to its determinateness.
A plurality external to it constitutes the determinateness
of the simple limit which the degree is for itself.
In so far as in the extensive quantum
amount was supposed to be found within number,
it was sublated there;
now, as thus sublated, amount is posited outside number.
Number, in being posited as a one,
as self-reference reflected into itself,
excludes from itself the indifference and the externality of amount
and is self-reference as reference to an external through itself.

In this, quantum has the reality which is adequate to its concept.
The indifference of the determinateness constitutes its quality,
that is, a determinateness which is in it
as a determinateness external to itself.
Accordingly, degree is a unitary quantitative determinateness
among a plurality of such intensities
which, though diverse, each being only a simple reference to itself,
are at the same time in essential connection with each other,
so that each has its determinateness in this continuity with the others.
This reference connecting a degree through itself to its other
makes ascent and descent on the scale of degrees
a continuous progress, a flow,
which is an uninterrupted and indivisible alteration;
none of the “more or less” differentiated within it
is separate from the others
but each has its determinateness only in these others.
As a self-referring quantitative determination,
each degree is indifferent towards the others;
but, in itself, it equally refers to this externality;
it is what it is only through
the intermediary of this externality;
in short, its reference to itself is not
an indifferent reference to externality
but in this externality it possesses its quality.

b. Identity of extensive and intensive magnitude

Degree is not inherently external to itself.
It is not, however, the indeterminate one,
is not the principle of number as such
which is not amount except negatively,
that is, only in the sense of not being an amount.
Intensive magnitude is at first a simple one
of many “more or less”;
there are several degrees;
but they are not determined either as a unitary one or as a more or
less but only as referring to each other as outside each other,
or in the identity of the one and the “more or less than.”
Thus, although the several “more
or less” are as such indeed outside the unitary degree,
the determinateness of the latter lies nonetheless
in its connection with them;
the degree thus contains amount.
Just as twenty contains as extensive magnitude
twenty ones as discrete,
the specific degree contains them as continuity,
a continuity which simply is this determinate plurality;
it is the twentieth degree,
and it is this twentieth degree
only through the intermediate of this amount
which, as such, is outside it.

The determinateness of the intensive magnitude is
to be considered, therefore, from two sides.
It is determined through other intensive quanta
and is continuous with its otherness,
so that its determinateness consists
in this connection with it.
Now, in so far as this determinateness is,
first, a simple determinateness,
it is determined as against the other degrees;
it excludes them from itself
and has its determinateness in this exclusion.
But, second, it is determined within;
this it is in the amount as its amount,
not in the amount as excluded,
or not in the amount of the other degrees.
The twentieth degree contains the twenty within itself;
it is not only determined as distinguished from the nineteenth,
the twenty-first, etc.,
but its determinateness is rather its amount.
But, inasmuch as the amount is its own,
and the determinateness is at the same time
essentially as amount, the degree is extensive quantum.

Extensive and intensive magnitude are, therefore,
one and the same determinateness of quantum;
they are distinguished only inasmuch as
the one has the amount within
and the other has the same without.
Extensive magnitude passes over into intensive magnitude
because its many collapses in and for itself
into oneness and steps outside it.
But, conversely, this simple one
has its determinateness only in the amount, its amount;
indifferent to the otherwise determined intensities,
it has the externality of amount in it;
thus intensive magnitude is just as essentially extensive magnitude.

With this identity, the qualitative something comes on the scene;
for the identity is the unity that refers back to itself
through the negation of its distinct terms;
these terms, however, make up the determinateness of
the existent magnitude.
The something is a quantum, but its qualitative existence is
now posited as indifferent to it as it is in itself.
One can speak of quantum, number as such, etc.,
without any mention of a something as their substrate.
But the something, self-mediated by virtue of the negation
of its determinations, now confronts these as existing for itself,
and, since it has a quantum, it confronts them as something
which has an extensive and intensive quantum.
Its one determinateness which it has as quantum
is posited in the distinct moments of unity and amount;
it is in itself not only one and the same determinateness,
but the positing of it in these differences as
extensive and intensive quantum is the return into this unity
which, as negative, is the something posited as indifferent to them.

c. Alteration of quantum

The difference between extensive and intensive quantum is
indifferent to the determinateness of quantum as quantum.
But quantum is in general
determinateness posited as sublated:
the indifferent limit, the determinateness
which is just as much the negation of itself.
In extensive quantum, this difference is developed;
but intensive magnitude is the existence of
this externality which quantum is in itself.
The difference is posited as the contradiction which it is in itself,
of being a simple self-referring determinateness
which is the negation of itself,
of having its determinateness not in it
but in another quantum.

A quantum, according to its quality, is therefore
in absolute continuity with its externality, with its otherness.
Consequently, not only can every determinateness of magnitude
be transcended, not only can it be altered:
that it must alter is now posited.
The determination of magnitude continues
into its otherness in such a way that
it has its being only in this continuity with an other;
it is not just a limit that exists but one that becomes.

The one is infinite or self-referring negation,
and hence the repulsion of itself from itself.
Quantum is equally infinite,
posited as the self-referring negation;
it repels itself from itself.
But it is a determinate “one,”
the one which has passed over into existence and limit,
thus the repulsion of determinateness from itself,
not the generation of something that is like itself
(as the repulsion of the one) but of its otherness;
quantum is now posited in it as sending itself beyond itself.
It consists in this, that it increases or decreases;
it is within it the externality of determinateness.

Thus quantum sends itself beyond itself;
this other which it becomes is at first itself a quantum,
but a quantum which is not a static limit
but one that impels itself beyond itself.
The limit which arises in this beyond is
therefore only one that again sublates itself
and sends itself to a further limit,
and so on to infinity.

C. QUANTITATIVE INFINITY

a. Its concept

Quantum alters and becomes another quantum;
the further determination of this alteration,
that it goes on to infinity,
lies in that it is positioned as
inherently self-contradictory.
Quantum becomes an other;
but it continues in its otherness;
the other is therefore also a quantum.
This latter, however, is the other,
not of a quantum, but of the quantum as such,
the negative of itself as a limited something,
and hence its own unlimitedness, infinity.
Quantum is an ought; it implies that it be determined-for-itself,
and this being-determined-for-itself is rather
the being determined in an other;
and, conversely, it is the being-determined in
an other as sublated, is indifferent subsisting-for-itself.

In this way, finitude and infinity each at once acquires
within it a double though opposite meaning.
Quantum is finite, first, as limited in general;
second, as sending itself beyond itself,
as being-determined in an other.
On the other hand, its infinity, is,
first, the unlimitedness of quantum;
second, its being-turned-back-into-itself,
the indifferent being-for-itself.
If we now compare these moments with each other,
we find that the determination of quantum's finitude,
its sending itself beyond itself into an other
that constitutes its determination,
is equally the determination of the infinite;
the negation of limit is this
same transcendence of determinateness,
so that in this negation, in the infinite,
quantum has its final determinateness.
The other moment of infinity is the for-itself
which is indifferent to the limit;
but the quantum itself is so limited,
as to be indifferent with respect to its limit,
and hence with respect to other quanta and its “beyond.”
In quantum, finitude and infinity
(the latter supposedly separate from finitude, as bad infinity)
each already possesses within it the moment of the other.

The qualitative and quantitative infinite are distinguished
inasmuch as in the former the opposition of
the finite and infinite is qualitative,
and the transition of the finite into the infinite,
or the reference of each to the other,
is present only in the in-itself, in their concept.
Qualitative determinateness is immediate;
it refers to otherness essentially as to
a being which is other than it;
it is not posited as having its negation, its other, in it.
By contrast, magnitude is as such sublated determinateness;
it is posited as being unlike and indifferent to itself,
and hence as something alterable.
The qualitative finite and infinite, therefore,
stand opposed to each other absolutely,
that is, abstractly;
their unity is the inner connection underlying them;
hence the finite continues in its other
only in itself, not in it.
By contrast, in the infinite in which
the quantitative finite has its absolute determinateness,
this finite refers to itself in it.
This, their mutual reference, is first displayed
in the quantitatively infinite process.

b. The quantitative infinite process

The process to infinity is in general the expression of contradiction,
here, of the contradiction contained in the quantitative finite
or in quantum in general.
It is the reciprocal determination of the finite and the infinite
that came up for consideration in the sphere of the qualitative,
with the difference that, as just indicated,
in the sphere of quantity the limit inherently
sends itself beyond itself and continues there,
and hence, conversely, the quantitative infinite is
also posited as having the quantum in it,
for in its externality quantum is itself;
its externality belongs to its determination.

The infinite progress is now
the expression of this contradiction,
not the resolution of it;
however, because of the continuity of
one determinateness in the other,
the progress gives rise to the semblance of
a resolution in a union of the two.
As at first posited, such a progress is
the task of attaining the infinite
but not the attainment of it;
it is a perpetual generation of the infinite,
without the progress of ever
getting  beyond the quantum itself,
and without the infinite ever becoming
something which is positively present.
It belongs to the concept of quantum
to have a beyond of itself.
This beyond is, first, the abstract moment
of the non-being of quantum;
this resolves itself in it;
it thus refers to its beyond as to its infinity in accordance
with the qualitative moment of the opposition.
But, second, quantum is continuous with this beyond;
it consists precisely in being the other of itself,
external to itself;
this externality equally is, therefore,
no more an other than the quantum;
the beyond or the infinite is thus itself a quantum.
The beyond is thus recalled from its flight
and the infinite is attained.
But because the infinite, now become a “this-side,”
is again a quantum,
what is posited is again only a new limit;
this limit, as quantum, has also fled again from itself,
is as such beyond itself,
and has repelled itself from itself
into its non-being, into its beyond,
and as the quantum repels itself into the beyond,
so does the beyond perpetually become a quantum.

The continuity of quantum with its other brings about
the conjunction of the two in the expression
of an infinitely great or infinitely small.
Since they both still have in them the determination of quantum,
they remain alterable and the absolute determinateness
which would be a being-for-itself is thus not attained.
This being-outside-itself of the determination is in the double infinity
(posited in the relative opposition of the “more” and the “less”)
of the infinitely great and the infinitely small.
In each, the quantum is maintained in perpetual opposition to its beyond.
No matter how much the “great” is enlarged, it shrinks to insignificance;
since it refers to the infinite as to its non-being,
the opposition is qualitative;
the enlarged quantum has gained nothing, therefore, from the infinite;
the latter is its nothing now just as before.
Or again, the increase in the quantum is not
an approximation to the infinite,
for the distinction between the quantum and its infinity
essentially has also the moment of being non-quantitative.
This moment is only the sharpened expression of the contradiction
that the quantum ought to be something great,
that is, a quantum, and non-finite, that is, not a quantum.
Equally, the infinitely small is, as something small,
a quantum and therefore remains absolutely, that is, qualitatively,
too great for the infinite and opposed to it.
In both, there remains the contradiction of the infinite progress
which in them should have reached its goal.

This infinity, which persists in the determination
of the beyond of the finite,
is to be characterized as the bad quantitative infinity.
Like the qualitatively bad infinity,
it is the perpetual movement back and forth
from one side of the persistent contradiction to the other,
from the limit to its non-being,
and from the latter back again to the other, the limit.
To be sure, the term to which the advance is made
in the quantitative progress is not an abstract “other” in general
but a quantum which is explicitly posited as different;
but this quantum remains opposed to its negation in the same way.
Also the progress, therefore, is neither an advance nor a gain
but rather a repetition of one and the same move, a positing, a sublating,
and then again a positing and a sublating:
an impotence of the negative to which what it sublates
continuously comes back by its very sublation of it.
The two, the positing and the sublation,
are so bonded to each other that
they absolutely flee from each other
and yet, in thus fleeing, they are unable to part
but rather become bonded in their very flight from each other.

c. The infinity of quantum

The infinite quantum as infinitely great or infinitely small is
itself, in itself, the infinite progress;
as great or small it is a quantum
and at the same time the non-being of quantum.
The infinitely great and the infinitely small are,
therefore, figurative representations
which on closer inspection prove to be
but unsubstantial nebulous shadows.
In the infinite progress, however,
this contradiction is explicitly present
and with it that which constitutes the nature of quantum
which, as intensive magnitude, has attained its reality
and is now posited in its existence as it is in its concept.
We must now consider this identity.

Quantum is as degree simple, self-referred, and determined within it.
Because the otherness and the determinateness are sublated
in it through this simplicity, the determinateness is external to it;
it has its determinateness outside it.
This, its being-outside-itself,
is at first the abstract non-being
of quantum in general, the bad infinity.
But further, this non-being is also a magnitude;
quantum continues in its non-being,
for it is precisely in its externality
that it has its determinateness,
and this, its externality, is
itself therefore equally a quantum;
the non-being of quantum, the infinity,
is thus limited, that is, this beyond is sublated,
is itself determined as a quantum
which, consequently, in its negation is with itself.

But this is what quantum as such is in itself.
For through its externality it is precisely itself;
the externality constitutes that
in virtue of which quantum is quantum,
where it is with itself.
In the infinite progress, therefore,
the concept of quantum is posited.

If we now first look at this progress
in its abstract determinations as
they are displayed before us,
what we find in it is the sublating of quantum,
but no less also of its beyond;
what we find, therefore, is the negation of
quantum as well as the negation of this negation.
Its truth is the unity of these two negations
in which the negations are, but as moments.
This unity is the resolution of the contradiction
of which the infinite progress is the expression;
its most immediate meaning, therefore, is that
of the restoration of the concept of magnitude,
of being an indifferent or external limit.
On the subject of the infinite progress as such,
the only reflection which is usually made is
that each quantum, however great or small, can disappear,
that it must be possible to transcend it;
not, however, that this sublating of the quantum,
the beyond, the bad infinite itself, also disappears.

Even the first sublating, the negation of quality as such
whereby the quantum is posited,
is in itself the sublation of negation;
quantum is sublated qualitative limit,
consequently sublated negation
but it is at the same time only in itself;
the sublating is posited as an existence,
and its negation is then fixed as the infinite,
as the side beyond quantum,
while the latter remains on its side as an immediate;
thus the infinite is determined only as first negation
and it is in this way that it appears in the infinite progress.
But there is more to it, as has just been shown:
there is the negation of negation
or what the infinite is in truth.
And this we have just seen with
the restoration of the concept of quantum.
Such a restoration means, in the first place,
that to the existence of the quantum
there has accrued a more precise determination.
What we now have is quantum determined
according to its concept,
and this quantum is different from the immediate quantum:
externality is now the opposite of itself,
is posited as a moment of magnitude;
quantum, for its part, is posited as
having its determinateness in another quantum,
through the intermediary of its non-being,
of infinity, that is, that it is qualitatively what it is.
Yet this comparison of the concept of quantum
with its existence belongs more to our reflection,
to a relation which is not yet present here.
The next determination, rather, which is present here is
that quantum has returned to quality, is from now on
qualitatively determined.
For its defining property, its quality, is externality,
the indifference of the determinateness;
and quantum is now posited rather
as being itself in its externality,
of referring to itself therein,
of being in simple unity with itself,
that is, of being qualitatively determined.
This qualitative being is still more closely determined,
namely as being-for-itself;
for the very self-reference
which quantum has attained
has proceeded from mediation,
from the negation of the negation.
Quantum no longer has infinity,
the being-determined-for-itself,
outside it, but in it.

The infinite, which in the infinite progress only has
the empty meaning of a non-being,
of an unattained but sought beyond,
is in fact nothing other than quality.
Quantum, as indifferent limit, surpasses itself into the infinite;
it thereby seeks nothing else than its being-determined-for-itself,
the qualitative moment which, however, is only an ought.
Its indifference towards the limit,
and hence its lack of a determinateness
which is an existent-for-itself, its surpassing itself,
is that which makes the quantum what it is.
This, its surpassing, is to be negated
and quantum is to find in infinity its absolute determinateness.

Quite generally: quantum is sublated quality;
but quantum is infinite, it surpasses itself,
is the negation of itself;
this, its surpassing, is therefore in itself
the negation of the negated quality,
the restoration of it;
and what is posited is that the externality,
which seemed to be a beyond,
is determined as quantum's own moment.

Quantum is thus posited as repelled from itself,
and with that there are two quanta which are however sublated,
only moments of one unity,
and this unity is the determinateness of quantum.
Quantum, self-referred as
indifferent limit and hence qualitatively posited,
is the quantitative relation or ratio.
In ratio quantum is external to itself, different from itself;
this, its externality, is the reference
connecting a quantum to another quantum,
each quantum acquiring value only
in this connection with its other;
and this reference constitutes the determinateness
of the quantum which is this unity.
In this unity quantum possesses not an indifferent
but a qualitative determination;
in this its externality has turned back into itself;
it is in it what it is.
