BOOK TWO

The Doctrine of Essence

ESSENCE

The truth of being is essence.

Being is the immediate.
Since the goal of knowledge is the truth,
what being is in and for itself,
knowledge does not stop at
the immediate and its determinations,
but penetrates beyond it on
the presupposition that
behind this being there still is
something other than being itself,
and that this background
constitutes the truth of being.
This cognition is a mediated knowledge,
for it is not to be found
with and in essence immediately,
but starts off from an other, from being,
and has a prior way to make,
the way that leads over and beyond being
or that rather penetrates into it.
Only inasmuch as knowledge recollects
itself into itself out of immediate being,
does it find essence through this mediation.
The German language has kept “essence” (Wesen)
in the past participle (gewesen) of the verb “to be” (sein),
for essence is past [but timelessly past] being.

When this movement is represented as a pathway of knowledge,
this beginning with being and the subsequent advance
which sublates being and arrives at essence as a mediated term
appears to be an activity of cognition external to being
and indifferent to its nature.

But this course is the movement of being itself.
That it is being's nature to recollect itself,
and that it becomes essence by virtue of this interiorizing,
this has been displayed in being itself.

If, therefore, the absolute was
at first determined as being,
now it is determined as essence.
Cognition cannot in general stop
at the manifold of existence;
but neither can it stop at being, pure being;
immediately one is forced to the reflection that
this pure being, this negation of everything finite,
presupposes a recollection and a movement
which has distilled immediate existence into pure being.
Being thus comes to be determined as essence,
as a being in which everything determined and finite is negated.
So it is simple unity, void of determination,
from which the determinate has been
removed in an external manner;
to this unity the determinate was
itself something external
and, after this removal,
it still remains opposite to it;
for it has not been sublated in itself but relatively,
only with reference to this unity.
We already noted above that if pure essence is defined
as the sum total of all realities,
these realities are equally subject to
the nature of determinateness and abstractive reflection
and their sum total is reduced to empty simplicity.
Thus defined, essence is only a product, an artifact.
External reflection, which is abstraction, only lifts
the determinacies of being out of what is left over as essence
and only deposits them, as it were, somewhere else,
letting them exist as before.
In this way, however, essence is neither in itself nor for itself;
it is by virtue of another, through external abstractive reflection;
and it is for another, namely for abstraction
and in general for the existent
which still remains opposite to it.
In its determination, therefore,
it is a dead and empty absence of determinateness.

As it has come to be here, however,
essence is what it is,
not through a negativity foreign to it,
but through one which is its own:
the infinite movement of being.
It is being-in-and-for-itself,
absolute in-itselfness;
since it is indifferent to
every determinateness of being,
otherness and reference to other have been sublated.
But neither is it only this in-itselfness;
as merely being-in-itself, it would be only
the abstraction of pure essence;
but it is being-for-itself just as essentially;
it is itself this negativity,
the self-sublation of otherness
and of determinateness.

Essence, as the complete turning back of being into itself,
is thus at first the indeterminate essence;
the determinacies of being are sublated in it;
it holds them in itself but without their being posited in it.
Absolute essence in this simple unity with itself has no existence.
But it must pass over into existence,
for it is being-in-and-for-itself;
that is to say, it differentiates
the determinations which it holds in itself,
and, since it is the repelling of itself from itself
or indifference towards itself, negative self-reference,
it thereby posits itself over against itself
and is infinite being-for-itself
only in so far as in thus
differentiating itself from itself
it is in unity with itself.
This determining is thus of another nature than
the determining in the sphere of being,
and the determinations of essence have another character
than the determinations of being.
Essence is absolute unity of being-in-itself and being-for-itself;
consequently, its determining remains inside this unity;
it is neither a becoming nor a passing over,
just as the determinations themselves are
neither an other as other nor references to some other;
they are self-subsisting but, as such,
at the same time conjoined in the unity of essence.
Since essence is at first simple negativity,
in order to give itself existence and then being-for-itself,
it must now posit in its sphere the determinateness
which it contains in principle only in itself.

Essence is in the whole what quality was in the sphere of being:
absolute indifference with respect to limit.
Quantity is instead this indifference in immediate determination,
limit being in it an immediate external determinateness;
quantity passes over into quantum;
the external limit is necessary to it and exists in it.
In essence, by contrast, the determinateness does not exist;
it is posited only by the essence itself,
not free but only with reference to
the unity of the essence.
The negativity of essence is reflection,
and the determinations are reflected,
posited by the essence itself
in which they remain as sublated.

Essence stands between being and concept;
it makes up their middle,
its movement constituting the transition
of being into the concept.
Essence is being-in-and-for-itself,
but it is this in the determination of being-in-itself;
for its general determination is that it emerges from being
or that it is the first negation of being.
Its movement consists in positing negation
or determination in being,
thereby giving itself existence
and becoming as infinite being-for-itself
what it is in itself.
It thus gives itself its existence
which is equal to its being-in-itself
and becomes concept.
For the concept is the absolute
as it is absolutely,
or in and for itself,
in its existence.
But the existence which essence gives to itself is
not yet existence as it is in and for itself
but as essence gives it to itself or as posited,
and hence still distinct from the existence of the concept.

First, essence shines within itself
or is reflection;
second, it appears;
third, it reveals itself.

In the course of its movement,
it posits itself in the following determinations:

I. As simple essence existing in itself,
remaining in itself in its determinations;

II. As emerging into existence,
or according to its concrete existence and appearance;

III. As essence which is one with its appearance,
as actuality.

SECTION I

Essence as Reflection Within

IV.1
jati-antara-parinama prakrti-apurat

Essence issues from being;
hence it is not immediately in and for itself
but is a result of that movement.
Or, since essence is taken at first as something immediate,
it is a determinate existence to which another stands opposed;
it is only essential existence, as against the unessential.
But essence is being which has been sublated in and for itself;
what stands over against it is only shine.
The shine, however, is essence's own positing.

First, essence is reflection.
Reflection determines itself;
its determinations are a positedness
which is immanent reflection at the same time.
Second, these reflective determinations
or essentialities are to be considered.
Third, as the reflection of its immanent determining,
essence turns into foundation and passes over
into concrete existence and appearance.

CHAPTER 1

Shine

IV.2
nimittam aprayojakam prakrtinam varana-bhedas tu tata ksetrikavat

As it issues from being, essence seems to stand over against it;
this immediate being is, first, the unessential.

IV.3
nirmana-cittani-asmita-matra

But, second, it is more than just the unessential;
it is being void of essence; it is shine.

IV.4
pravrtti-bhede prayojakam cittam ekam anekesam

Third, this shine is not something external,
something other than essence, but is essence's own shining.
This shining of essence within it is reflection.

CHAPTER 2

Foundation

IV.5
tatra dhyana-jam anasayam

The essentialities or the determinations of reflection

Reflection is determined reflection;
accordingly, essence is determined essence, or it is essentiality.

Reflection is the shining of essence within itself.

Essence, as infinite immanent turning back is
not immediate simplicity, but negative simplicity;
it is a movement across moments that are distinct,
is absolute mediation with itself.
But in these moments it shines;
the moments are, therefore, themselves
determinations reflected into themselves.

First, essence is simple self-reference, pure identity.
This is its determination, one by which it is rather
the absence of determination.

Second, the specifying determination is difference,
difference which is either external or indefinite,
diversity in general, or opposed diversity or opposition.

Third, as contradiction this opposition is reflected into itself
and returns to its foundation.

CHAPTER 3

Ground

Essence determines itself as ground.

Just as nothing is at first
in simple immediate unity with being,
so here too the simple identity of essence is
at first in simple unity with its absolute negativity.
Essence is only this negativity which is pure reflection.
It is this pure reflection as
the turning back of being into itself;
hence it is determined, in itself or for us,
as the ground into which being resolves itself.
But this determinateness is not posited by the essence itself;
in other words, essence is not ground precisely because
it has not itself posited this determinateness that it possesses.
Its reflection, however, consists in positing itself as
what it is in itself, as a negative, and in determining itself.
The positive and the negative constitute the essential determination
in which essence is lost in its negation.
These self-subsisting determinations of reflection sublate themselves,
and the determination that has foundered to the ground is
the true determination of essence.

Consequently, ground is itself one of
the reflected determinations of essence,
but it is the last, or rather,
it is determination determined as sublated determination.
In foundering to the ground, the determination of reflection
receives its true meaning that it is the
absolute repelling of itself within itself;
or again, that the positedness that accrues to essence is
such only as sublated,
and conversely that only the self-sublating positedness is
the positedness of essence.
In determining itself as ground,
essence determines itself as the not-determined,
and only the sublating of its being determined is its determining.
Essence, in thus being determined as self-sublating,
does not proceed from an other but is,
in its negativity, identical with itself.

Since the advance to the ground is made starting
from determination as an immediate first
(is done by virtue of the nature of determination itself
that founders to the ground through itself),
the ground is at first determined by that immediate first.
But this determining is, on the one hand,
as the sublating of the determining,
the merely restored, purified or manifested identity of essence
which the determination of reflection is in itself;
on the other hand, this negating movement is, as determining,
the first positing of that reflective determinateness
that appeared as immediate determinateness,
but which is posited only by the self-excluding reflection of ground
and therein is posited as only something posited or sublated.
Thus essence, in determining itself as ground, proceeds only from itself.
As ground, therefore, it posits itself as essence,
and its determining consists in just this positing of itself as essence.
This positing is the reflection of essence
that sublates itself in its determining;
on that side is a positing, on this side is the positing of essence,
hence both in one act.

Reflection is pure mediation in general;
ground, the real mediation of essence with itself.
The former, the movement of nothing through nothing back to itself,
is the reflective shining of one in an other;
but, because in this reflection opposition does not
yet have any self-subsistence,
neither is the one, that which shines, something positive,
nor is the other in which it reflectively shines something negative.
Both are substrates, actually of the imagination;
they are still not self-referring.
Pure mediation is only pure reference,
without anything being referred to.
Determining reflection, for its part, does posit
such terms as are identical with themselves;
but these are at the same time only determined references.
Ground, on the contrary, is mediation that is real,
since it contains reflection as sublated reflection;
it is essence that turns back into itself
through its non-being and posits itself.
According to this moment of sublated reflection,
what is posited receives the determination of immediacy,
of an immediate which is self-identical
outside its reference or its reflective shining.
This immediacy is being as restored by essence,
the non-being of reflection through which essence mediates itself.
Essence returns into itself as it negates;
therefore, in its turning back into itself,
it gives itself the determinateness that precisely
for this reason is the self-identical negative,
is sublated positedness, and consequently,
as the self-identity of essence as ground,
equally an existent.

IV.6
karma-asukla-akrsnam yogina trividham itaresam

IV.7
tatas tad-vipaka-anugunanam eva-abhivyakti vasananam

The ground is, first, absolute ground
one in which the essence is first of all
the general substrate for the ground-connection.
It then further determines itself as form and matter
and gives itself a content.

IV.8
jati-desa-kala vyavahitanam api-anantaryam smrti-samskarayo eka-rupatvat

IV.9
tasam anaditvam ca-asisa nityatvat

Second, it is determinate ground,
the ground of a determinate content.
Because the ground-connection, in being realized,
becomes as such external,
it passes over into conditioning mediation.

IV.10
hetu-phala-asraya-alambana sangrhitatvad esam abhave tad-abhava

Third, ground presupposes a condition;
but the condition equally presupposes the ground;
the unconditioned is the unity of the two,
the fact itself that, by virtue of
the mediation of the conditioning reference,
passes over into concrete existence.

SECTION II

Appearance

Essence must appear.

Being is the absolute abstraction;
this negativity is not something external to it,
but being is rather being,
and nothing but being,
only as this absolute negativity.
Because of this negativity,
being is only as self-sublating being
and is essence.
But, conversely, essence as simple self-equality
is likewise being.
The doctrine of being contains the first proposition,
“being is essence.”
The second proposition, “essence is being,”
constitutes the content of the first section
of the doctrine of essence.
But this being into which
essence makes itself
is essential being,
concrete existence,
a being which has come forth
out of negativity and inwardness.

Thus essence appears.
Reflection is the internal shining of essence.
The determinations of this reflection are included
in the unity purely and simply as posited, sublated;
or reflection is essence immediately identical
with itself in its positedness.
But since this essence is ground,
through its self-sublating reflection,
or the reflection that which returns into itself,
essence determines itself as something real;
further, since this real determination, or the otherness,
of the ground-connection sublates itself
in the reflection of the ground
and becomes concrete existence,
the form determinations acquire therein
an element of independent subsistence.
Their reflective shine comes to completion in appearance.

The essentiality that has advanced to immediacy is,

first, concrete existence,
and a concrete existent or thing,
an undifferentiated unity of essence and its immediacy.
The thing indeed contains reflection,
but its negativity is at first dissolved in its immediacy;
but, because its ground is essentially reflection,
its immediacy is sublated
and the thing makes itself into a positedness.

Second, then, it is appearance.
Appearance is what the thing is in itself,
or the truth of it.
But this concrete existence,
only posited and reflected into otherness,
is equally the surpassing of itself into its infinity;
opposed to the world of appearance there stands
the world that exists in itself reflected into itself.
But the being that appears and essential being
stand referred to each other absolutely.

Thus concrete existence is, third, essential relation;
what appears shows the essential,
and the essential is in its appearance.
Relation is the still incomplete union of
reflection into otherness and reflection into itself;
the complete interpenetrating of the two is actuality.

CHAPTER 1

Concrete existence

IV.11
atita-anagatam svarupato 'styadhva-bhedad dharmanam

IV.12
te vyakta-suksma guna-atmana

IV.13
parinama-ekatvad vastu-tattvam

Just as the principle of sufficient reason says
that whatever is has a ground,
or is something posited, something mediated,
so there would also have to be
a principle of concrete existence saying
that whatever is, exists concretely.
The truth of being is to be,
not an immediate something,
but essence that has
come forth into immediacy.

But when it was further said
that whatever exists concretely
has a ground and is conditioned,
it also would have had to be said
that it has no ground and is unconditioned.
For concrete existence is the immediacy
that has come forth from the sublating
of the mediation that results
from the connection of ground and condition,
and which, in coming forth,
sublates this very coming forth.

Inasmuch as mention may be made here of
the proofs of the concrete existence of God,
it is first to be noted that besides
immediate being that comes first,
and concrete existence
(or the being that proceeds from essence)
that comes second, there is still a third being,
one that proceeds from the concept,
and this is objectivity.
Proof is, in general, mediated cognition.
The various kinds of being require or contain
each its own kind of mediation,
and so will the nature of the proof also vary accordingly.
The ontological proof wants to start from the concept;
it lays down as its basis the sum total of all realities,
where under reality also concrete existence is subsumed.
Its mediation, therefore, is that of the syllogism,
and syllogism is not yet under consideration here.
We have already commented above (Part 1, Section 1)
on Kant's objection to the ontological proof,
and have remarked that by concrete existence
Kant understands the determinate immediate existence
with which something enters into the context of total experience,
that is, into the determination of being an other
and of being in reference to an other.
As an existent concrete in this way,
something is thus mediated by an other,
and concrete existence is in general the side of its mediation.
But in what Kant calls the concept, namely,
something taken as only simply self-referring,
or in representation as such, this mediation is missing;
in abstract self-identity, opposition is left out.
Now the ontological proof would have
to demonstrate that the absolute concept,
namely the concept of God,
attains to a determinate existence, to mediation,
or to demonstrate how simple essence
mediates itself with mediation.
This is done by the just mentioned
subsumption of concrete existence
under its universal, namely reality,
which is assumed as the middle term
between God in his concept, on the one hand,
and concrete existence, on the other.
This mediation, inasmuch as it has the form of a syllogism,
is not at issue here, as already said.
However, how that mediation of
essence and concrete existence truly comes about,
this is contained in the preceding exposition.
The nature of the proof itself will be considered
in the doctrine of cognition.
Here we have only to indicate what pertains
to the nature of mediation in general.

The proofs of the existence of God
adduce a ground for this existence.
It is not supposed to be an objective
ground of the existence of God,
for this existence is in and for itself.
It is, therefore, solely a ground for cognition.
It thereby presents itself as a ground
that vanishes in the subject matter
that at first seems to be grounded by it.
Now the ground which is derived
from the contingency of the world
entails the regress of the latter
into the absolute essence,
for the accidental is that which is
in itself groundless and self-sublating.
In this way, therefore, the absolute essence
does indeed proceed from that which has no ground,
for the ground sublates itself
and with this there also vanishes
the reflective shine of the relation
that was given to God, that it is grounded in an other.
This mediation is therefore true mediation.
But the reflection involved in that proof does not know
the nature of the mediation that it performs.
On the one hand, it takes itself to be something merely subjective,
and it consequently distances its mediation from God himself;
on the other hand, for that same reason it
also fails to recognize its mediating movement,
that this movement is in the essence itself and how it is there.
The true relation of reflection consists
in being both in one:
mediation as such but, of course, at the same time
a subjective, external mediation,
that is to say, a self-external mediation
which in turn internally sublates itself.
In that other presentation, however,
concrete existence is given the false relation
of appearing only as mediated or posited.

So, on the other side, concrete existence also
cannot be regarded merely as an immediate.
Taken in the determination of an immediacy,
the comprehension of God's concrete existence
has been declared to be beyond proof
and the knowledge of it
an immediate consciousness only, a faith.
Knowledge should arrive at
the conclusion that it knows nothing,
and this means that it gives up its mediating movement
and the determinations themselves
that have come up in the course of it.
This is what has also occurred in the foregoing;
but it must be added that reflection,
by ending up with the sublation of itself,
does not thereby have nothing for result,
so that the positive knowledge of the essence
would then be an immediate reference to it,
divorced from that result and self-originating,
an act that starts only from itself;
on the contrary, the end itself,
the foundering of the mediation,
is at the same time the ground
from which the immediate proceeds.
In “zu Grunde gehen,” the German language unites,
as we remarked above,
the meaning of foundering and of ground;
the essence of God is said to be the
abyss (Abgrund in German) for finite reason.
This it is, indeed, in so far as
reason surrenders its finitude therein,
and sinks its mediating movement;
but this abyss, the negative ground, is at the same time
the positive ground of the emergence of the existent,
of the essence immediate in itself;
mediation is an essential moment.
Mediation through ground sublates itself but
does not leave the ground standing under it,
so that what proceeds from it would be a posited
that has its essence elsewhere;
on the contrary, this ground is, as an abyss,
the vanished mediation, and, conversely, only the
vanished mediation is at the same time the ground
and, only through this negation,
the self-equal and immediate.

Concrete existence, then, is not to be taken here
as a predicate, or as a determination of essence,
of which it could be said in a proposition,
“essence exists concretely,” or “it has concrete existence.”
On the contrary, essence has passed over into concrete existence;
concrete existence is the absolute self-emptying of essence,
an emptying that leaves nothing of the essence behind.
The proposition should therefore run:
“Essence is concrete existence;
it is not distinct from its concrete existence.”
Essence has passed over into concrete existence
inasmuch as essence as ground
no longer distinguishes itself from itself as grounded,
or inasmuch as the ground has sublated itself.
But this negation is no less essentially its position,
or the simply positive continuity with itself;
concrete existence is the reflection of the ground into itself,
its self-identity as attained in its negation,
therefore the mediation that has posited itself
as identical with itself and through that is immediacy.

Now because concrete existence is
essentially self-identical mediation,
it has the determinations of mediation in it,
but in such a way that the determinations are
at the same time reflected into themselves
and have essential and immediate subsistence.
As an immediacy which is posited through sublation,
concrete existence is negative unity and being-within-itself;
it therefore immediately determines itself
as a concrete existent and as thing.

CHAPTER 2

Appearance

Concrete existence is the immediacy of being
to which essence has again restored itself.
In itself this immediacy is the reflection of essence into itself.
As concrete existence, essence has stepped out of its ground
which has itself passed over into it.
Concrete existence is this reflected immediacy
in so far as, within, it is absolute negativity.
It is now also posited as such,
in that it has determined itself as appearance.

At first, therefore, appearance is
essence in its concrete existence;
essence is immediately present in it.
That it is not immediate,
but rather reflected concrete existence,
constitutes the moment of essence in it;
or concrete existence, as essential concrete existence,
is appearance.

Something is only appearance,
in the sense that concrete existence is
as such only a posited being,
not something that is in-and-for-itself.
This is what constitutes its essentiality,
to have the negativity of reflection,
the nature of essence, within it.
There is no question here of an alien,
external reflection to which essence would belong
and which, by comparing this essence with concrete existence,
would declare the latter to be appearance.
On the contrary, as we have seen,
this essentiality of concrete existence,
that it is appearance, is
concrete existence's own truth.
The reflection by virtue of which
it is this is its own.

But if it is said that something is only appearance,
meaning that as contrasted with it
immediate concrete existence is the truth,
then the fact is that appearance is the higher truth,
for it is concrete existence as essential,
whereas concrete existence is appearance
that is still void of essence
because it only contains in it
the one moment of appearance,
namely that of concrete existence
as immediate, not yet negative, reflection.
When appearance is said to be essenceless,
one thinks of the moment of its negativity as if,
by contrast with it, the immediate were
the positive and the true;
in fact, however, this immediate does not
yet contain essential truth in it.
Concrete existence rather ceases to be essenceless
by passing over into appearance.

Essence reflectively shines at first
just within, in its simple identity;
as such, it is abstract reflection,
the pure movement of nothing
through nothing back to itself.
Essence appears, and so it now is real shine,
since the moments of the shine have concrete existence.
Appearance, as we have seen, is the thing as
the negative mediation of itself with itself;
the differences which it contains are self-subsisting matters
which are the contradiction of being an immediate subsistence,
yet of obtaining their subsistence only in an alien self-subsistence,
hence in the negation of their own, but then again,
just because of that, also in the negation
of that alien self-subsistence
or in the negation of their own negation.
Reflective shine is this same mediation,
but its fleeting moments obtain in appearance
the shape of immediate self-subsistence.
On the other hand, the immediate self-subsistence which
pertains to concrete existence is reduced to a moment.
Appearance is therefore the unity
of reflective shine and concrete existence.

Appearance now determines itself further.
It is concrete existence as essential;
as essential, concrete existence differs
from the concrete existence which is unessential,
and these two sides refer to each other.

IV.14
vastu-samye citta-bhedat tayo vibhakta pantha

IV.15
na ca-eka-citta-tantram vastu tad apramanakam tada kim syat

IV.16
tad-uparaga-apeksitvat-cittasya vastu jnata-ajnatam

Appearance is, therefore, first, simple self-identity
which also contains diverse content determinations
and, both as identity and as the connecting reference
of these determinations,
is that which remains self-equal
in the flux of appearance;
this is the law of appearance.

IV.17
sada jnata citta-vrttaya tat-prabho purusasya-aparinamitvat

IV.18
na tat sva-abhasam drsyatvat

But, second, the law which is
simple in its diversity
passes over into opposition;
the essential moment of appearance becomes
opposed to appearance itself
and, confronting the world of appearance,
the world that exists in itself
comes onto the scene.

IV.19
eka-samaye ca-ubhaya-anavadharanam

Third, this opposition returns into its ground;
that which is in itself is in the appearance
and, conversely, that which appears is determined
as taken up into its being-in-itself.
Appearance becomes relation.

CHAPTER 3

The essential relation

IV.20
citta-antara-drsye buddhi-buddher atiprasanga smrti-sankara ca

The truth of appearance is the essential relation.
Its content has immediate self-subsistence:
the existent immediacy and the reflected immediacy
or the self-identical reflection.
In this self-subsistence, however,
it is at the same time a relative content;
it is simply and solely as a reflection into its other,
or as unity of the reference with its other.
In this unity, the self-subsistent content is
something posited, sublated;
but precisely this unity is what constitutes
its essentiality and self-subsistence;
this reflection into an other is reflection into itself.
The relation has sides, since it is reflection into an other;
so its difference is internal to it,
and its sides are independent subsistence,
for in their mutually indifferent diversity
they are thrown back into themselves,
so that the subsistence of each equally has its meaning
only in its reference to the other
or in the negative unity of both.

The essential relation is therefore not yet
the true third to essence and to concrete existence
but already contains the determinate union of the two.
Essence is realized in it in such a way that
it has self-subsistent, concrete existents for its subsistence,
and these concrete existents have returned
from their indifference back into their essential unity
so that they have only this unity as their subsistence.
Also the reflective determinations of positive and negative are
reflected into themselves only as each is reflected into its opposite;
but they have no other determination besides this their negative unity,
whereas the essential relation has sides
that are posited as self-subsistent totalities.
It is the same opposition as that of positive and negative,
but it is such as an inverted world.
The side of the essential relation is a totality
which, however, essentially has an opposite or a beyond;
it is only appearance;
its concrete existence,
rather than being its own,
is that of its other.
It is, therefore, something internally fractured;
but this, its sublated being, consists in
its being the unity of itself and its other,
therefore a whole, and precisely for this reason
it has self-subsistent concrete existence
and is essential reflection into itself.

This is the concept of relation.
At first, however, the identity it contains
is not yet perfect;
the totality which each relative is as relative,
is only an inner one;
the side of the relation is posited at first
in one of the determinations of negative unity;
what constitutes the form of the relation is
the specific self-subsistence of each of the two sides.
The identity of the form is therefore only a reference,
and the self-subsistence of the sides falls outside it,
that is to say, it falls in the sides;
we still do not have the reflected unity
of the identity of the relation
and of the self-subsistent concrete existents;
we still do not have substance.
It follows that the concept of relation has
indeed shown itself to be the unity
of reflected and immediate self-subsistence.
But it is this concept still immediately at first;
immediate are therefore its moments vis-à-vis each other,
and immediate is the unity of the reference
connecting them essentially;
a unity this, which only then is the true unity
that conforms to the concept,
when it has realized itself, that is to say,
through its movement has posited itself as this unity.

The essential relation is therefore immediately
the relation of the whole and the parts
the reference of reflected and immediate self-subsistence,
so that both are at the same time
mutually conditioning and presupposing.

In this relation, neither of the sides is
yet posited as moment of the other;
their identity is therefore itself one side,
or not their negative unity.
Hence, secondly, the relation passes over into one
in which one side is the moment of the other
and is present there as in its ground,
the true self-subsistent element of both.
This is the relation of force and its expression.

Third, the inequality still present
in this reference sublates itself,
and the final relation is that of inner and outer.
In this difference,
which has now become totally formal,
relation itself founders,
and substance or actuality come on the stage
as the absolute unity of
immediate and reflected concrete existence.

SECTION III

Actuality

Actuality is the unity of essence and concrete existence;
in it, shapeless essence and unstable appearance
(subsistence without determination
and manifoldness without permanence)
have their truth.
Although concrete existence is the immediacy
that has proceeded from ground,
it still does not have form explicitly posited in it;
inasmuch as it determines and informs itself, it is appearance;
and in developing this subsistence that otherwise only is
a reflection-into-other into an immanent reflection,
it becomes two worlds, two totalities of content,
one determined as reflected into itself
and the other as reflected into other.
But the essential relation exposes
the formality of their connection,
and the consummation of the latter is
the relation of the inner and the outer
in which the content of both is equally
only one identical substrate
and only one identity of form.
This identity has come about also in regard to form,
the form determination of their difference is sublated,
and that they are one absolute totality is posited.

This unity of the inner and outer is absolute actuality.
But this actuality is, first, the absolute as such
(in so far as it is posited as a unity
in which the form has sublated itself)
making itself into the empty or external
distinction of an outer and inner.
Reflection relates to this absolute
as external to it;
it only contemplates it
rather than being its own movement.
But it is essentially this movement
and is, therefore, as the absolute's
negative turning back into itself.

Second, it is actuality proper.
Actuality, possibility, and necessity constitute
the formal moments of the absolute,
or its reflection.

Third, the unity of the absolute
and its reflection is
the absolute relation,
or rather the absolute as
relation to itself, substance.

CHAPTER 1

The absolute

IV.21
citer apratisamkramayas tad-akara-apattau svabuddhi-samvedanam

IV.22
drastr-drsya-uparaktam cittam sarva-artham

IV.23
tad asamkhyeya-vasanabhi citram api para-artham samhatya-karitvat

The simple solid identity of the absolute is indeterminate, or rather,
every determinateness of essence and concrete existence,
or of being in general as well as of reflection,
has dissolved itself into it.
Accordingly, the determining of what is
the absolute appears to be a negating,
and the absolute itself appears only as
the negation of all predicates, as the void.
But since it must equally be spoken of
as the position of all predicates,
it appears as the most formal of contradictions.
In so far as that negating and this positing
belong to external reflection,
what we have is a formal, unsystematic dialectic
that has an easy time picking up
a variety of determinations here and there,
and is just as at ease demonstrating, on the one hand,
their finitude and relativity, as declaring, on the other,
that the absolute, which it vaguely envisages as totality,
is the dwelling place of all determinations,
yet is incapable of raising either
the positions or the negations to a true unity.
The task is indeed to demonstrate what the absolute is.
But this demonstration cannot be either
a determining or an external reflection
by virtue of which determinations
of the absolute would result,
but is rather the exposition of the absolute,
more precisely the absolute's own exposition,
and only a displaying of what it is.

CHAPTER 2

Actuality

The absolute is the unity of inner and outer
as a first implicitly existent unit.
The exposition appeared as an external reflection
which, for its part, has the immediate
as something it has found,
but it equally is its movement
and the reference connecting it to the absolute
and, as such, it leads it back to the latter,
determining it as a mere “way and manner.”
But this “way and manner” is the
determination of the absolute itself,
namely its first identity
or its mere implicitly existent unity.
And through this reflection, not only is
that first in-itself posited as essenceless determination,
but, since the reflection is negative self-reference,
it is through it that the in-itself becomes
a mode in the first place.
It is this reflection that,
in sublating itself in its determinations
and as a movement which as such turns back upon itself,
is first truly absolute identity
and, at the same time, the determining of
the absolute or its modality.
The mode, therefore, is the externality of the absolute,
but equally so only its reflection into itself;
or again, it is the absolute's own manifestation,
so that this externalization is its immanent reflection
and therefore its being in-and-for-itself.

So, as the manifestation that it is nothing,
that it has no content, save to be
the manifestation of itself,
the absolute is absolute form.
Actuality is to be taken as
this reflected absoluteness.
Being is not yet actual;
it is the first immediacy;
its reflection is therefore becoming
and transition into an other;
or its immediacy is not being-in-and-for-itself.
Actuality also stands higher than concrete existence.
It is true that the latter is the immediacy
that has proceeded from ground and conditions,
or from essence and its reflection.
In itself or implicitly, it is therefore
what actuality is, real reflection;
but it is still not the posited unity of reflection and immediacy.
Hence concrete existence passes over into appearance
as it develops the reflection contained within it.
It is the ground that has foundered to the ground;
its determination, its vocation, is to restore this ground,
and therefore it becomes essential relation,
and its final reflection is that its
immediacy be posited as immanent reflection and conversely.
This unity, in which concrete existence
or immediacy and the in-itself,
the ground or the reflected, are simply moments,
is now actuality.
The actual is therefore manifestation.
It is not drawn into
the sphere of alteration by its externality,
nor is it the reflective shining of itself in an other.
It just manifests itself,
and this means that in its externality,
and only in it, it is itself, that is to say,
only as a self-differentiating and self-determining movement.

Now in actuality as this absolute form,
the moments only are as sublated or formal, not yet realized;
their differentiation thus belongs at first to external reflection
and is not determined as content.

Actuality, as itself immediate form-unity of inner and outer,
is thus in the determination of immediacy
as against the determination of immanent reflection;
or it is an actuality as against a possibility.
The connection of the two to each other is the third,
the actual determined both as being reflected into itself
and as this being immediately existing.
This third is necessity.

IV.24
visesa-darsina atma-bhava-bhavana-vinivrtti

IV.25
tada viveka-nimnam kaivalya-prag-bharam cittam

IV.26
tad-chidresu pratyaya-antarani samskarebhya

IV.27
hanam esam klesavad uktam

But first, since the actual and the possible
are formal distinctions,
their connection is likewise only formal,
and consists only in this,
that the one just like the other
is a positedness, or in contingency.

IV.28
prasankhyane api-akusidasya sarvatha viveka-khyater dharma-megha samadhi

IV.29
tata klesa-karma-nivrtti

IV.30
tada sarvavarana-malapetasya jnaanasyanantyaj jnaeyam alpam

IV.31
tata-krta-arthanam parinama-krama-samaptir gunanam

Second, because in contingency
the actual as well as the possible
are a positedness,
because they have retained their determination,
real actuality now arises,
and with it also real possibility
and relative necessity.

IV.32
ksana-pratiyogi parinama-aparanta-nirgrahya krama

Third, the reflection of relative necessity
into itself yields absolute necessity,
which is absolute possibility and actuality.

CHAPTER 3

The absolute relation

IV.33
purusa-artha-sunyanam gunanam pratiprasava kaivalyam
svarupa-pratistha va citi-sakti iti

Absolute necessity is not so much the necessary,
even less a necessary, but necessity:
being simply as reflection.
It is relation
because it is a distinguishing
whose moments are themselves
the whole totality of necessity,
and therefore subsist absolutely,
but do so in such a way that
their subsisting is one subsistence,
and the difference only the reflective shine of
the movement of exposition,
and this reflective shine is the absolute itself.
Essence as such is reflection or a shining;
as absolute relation, however, essence is the
reflective shine posited as reflective shine,
one which, as such self-referring, is absolute actuality.
The absolute, first expounded by external reflection,
as absolute form or as necessity now expounds itself;
this self-exposition is its self-positing,
and is only this self-positing.
Just as the light of nature is not a something,
nor is it a thing, but its being is rather only its shining,
so manifestation is self-identical absolute actuality.

The sides of the absolute relation are not, therefore, attributes.
In the attribute the absolute reflectively shines only in one of its moments,
as in a presupposition that external reflection has simply assumed.
But the expositor of the absolute is the absolute necessity
which, as self-determining, is identical with itself.
Since this necessity is the reflective shining
posited as reflective shining, the sides of this relation,
because they are as shine, are totalities;
for as shine, the differences are themselves and their opposite,
that is, they are the whole;
and, conversely, they thus are only shine because they are totalities.
Thus this distinguishing, this reflecting shining of the absolute,
is only the identical positing of itself.

This relation in its immediate concept is
the relation of substance and accidents,
the immediate internal disappearing and becoming
of the absolute reflective shine.
If substance determines itself as a being-for-itself over
against an other or is absolute relation as something real,
then we have the relation of causality.
Finally, when this last relation passes over into
reciprocal causality by referring itself to itself,
we then have the absolute relation also posited
in accordance with the determination it contains;
this posited unity of itself in its determinations,
which are posited as the whole itself
and consequently equally as determinations,
is then the concept.
