The Doctrine of the Concept

OF THE CONCEPT IN GENERAL

What the nature of the concept is
cannot be given right away,
not any more than can the concept
of any other subject matter.
It might perhaps seem that,
in order to state the concept of a subject matter,
the logical element can be presupposed,
and that this element would not
therefore be preceded by anything else,
or be something deduced,
just as in geometry logical propositions,
when they occur applied to magnitudes
and employed in that science,
are premised in the form of axioms,
underived and underivable determinations of cognition.
Now the concept is to be regarded indeed,
not just as a subjective presupposition
but as absolute foundation;
but it cannot be the latter
except to the extent that
it has made itself into one.
Anything abstractly immediate is
indeed a first;
but, as an abstraction,
it is rather something mediated,
the foundation of which,
if it is to be grasped in its truth,
must therefore first be sought.
And this foundation will indeed
be something immediate,
but an immediate
which has made itself such
by the sublation of mediation.

From this aspect the concept is
at first to be regarded simply
as the third to being and essence,
to the immediate and to reflection.
Being and essence are therefore
the moments of its becoming;
but the concept is their foundation
and truth as the identity
into which they have sunk
and in which they are contained.
They are contained in it
because the concept is their result,
but no longer as being and essence;
these are determinations which they have
only in so far as they have not yet returned
into the identity which is their unity.
Hence the objective logic,
which treats of being and essence,
constitutes in truth
the genetic exposition of the concept.

More precisely, substance already is real essence,
or essence in so far as it is united with being
and has stepped into actuality.
Consequently, the concept has substance
for its immediate presupposition;
substance is implicitly what the concept is explicitly.
The dialectical movement of substance
through causality and reciprocal affection
is thus the immediate genesis of the concept
by virtue of which its becoming is displayed.
But the meaning of its becoming,
like that of all becoming,
is that it is the reflection of something
which passes over into its ground,
and that the at first apparent other
into which this something has passed over
constitutes the truth of the latter.

Thus the concept is the truth of substance,
and since necessity is the
determining relational mode of substance,
freedom reveals itself to be the truth of necessity
and the relational mode of the concept.
The necessary forward course of determination
characteristic of substance is the positing
of that which is in and for itself.
The concept is now this absolute
unity of being and reflection
whereby being-in-and-for-itself
only is by being equally reflection or positedness,
and positedness only is by being equally in-and-for-itself.
This abstract result is elucidated by the
exposition of its concrete genesis
which contains the nature of the concept
but had to precede its treatment.
We must briefly sum up here, therefore,
the main moments of this exposition
(which has been treated in detail in
Book Two of the Objective Logic).

Substance is the absolute,
the actual in-and-for-itself:

in itself, because it is the simple
identity of possibility and actuality;
absolute, because it is the essence containing
all actuality and possibility within itself;
for itself, because it is
this identity as absolute power
or absolutely self-referring negativity.

The movement of substantiality posited by these moments
consists in the following stages:

1. Substance, as absolute power or self-referring negativity,
differentiates itself into a relation in which
what are at first only simple moments
are substances and original presuppositions.
Their specific relation is
that of a passive substance,
of the originariness of the simple in-itself
which, powerless to posit itself, is
only originary positedness,
and of an active substance,
the self-referring negativity
which has as such posited itself as
an other and refers to it.
This other is precisely the passive substance
which the active substance, as originative power,
has presupposed for itself as its condition.
This presupposing is to be understood
in the sense that the movement of substance is
at first in the form of one moment of its concept,
that of the in-itself that the determinateness
of one of the substances standing in relation is
itself also the determinateness of this relation.

2. The other moment is the being-for-itself
or the power positing itself as self-referring negativity
and thereby again sublating what it presupposes.
The active substance is cause; it acts;
this means that it is now a positing,
just as before it was a presupposing, that
(a) power is also given the reflective shine of power,
positedness also the reflective shine of positedness.
What in the presupposition was the originary becomes in causality,
by virtue of the reference to an other, what it is in itself.
The cause brings about an effect.
But it does so in another substance
and it is now power with reference to an other;
it thus appears as cause but is cause
only by virtue of this appearing.
(b) The effect enters the passive substance
and by virtue of it the latter now also appears as positedness,
but is passive substance only in this positedness.

3. But there is more still present here
than just this appearance, namely:
(a) the cause acts upon the passive substance,
alters its determination;
but this determination is its positedness,
for otherwise there is nothing else to alter;
the other determination which it obtains
is however that of causality;
the passive substance thus comes to be
cause, power, and activity;
(b) the effect is posited in it by the cause;
but that which is posited by the cause is the cause itself
which, in acting, is identical with itself;
it is this cause that posits itself
in the place of the passive substance.

Similarly, with respect to the active substance:

(a) the action is the translation of the cause into the effect,
into its other, the positedness;
(b) the cause reveals itself in the effect as what it is;
the effect is identical with the cause, is not an other;
in acting the cause thus reveals the positedness to be
that which it (the cause) essentially is.
Each side, therefore, in accordance with
how it refers to the other
both as identical with it and as the negative of it,
becomes the opposite of itself, but,
in becoming this opposite, the other,
and therefore also each,
remains identical with itself.
But both, the identical and the negative reference,
are one and the same;
substance is self-identical only in its opposite
and this constitutes the absolute identity of
the two substances posited as two.
It is by its act that active substance is
manifested as cause or originary substantiality, that is,
by positing itself as the opposite of itself,
a positing which is at the same time
the sublating of its presupposed otherness,
of passive substance.
Contrariwise, it is by being acted upon
that the positedness is manifested as positedness,
the negative as negative,
and consequently the passive substance as self-referring negativity,
and in this other the cause simply rejoins itself.

Through this positing, therefore, what is presupposed,
that is, the implicit originariness, becomes explicit;
but this being, which is now in and for itself,
is only by virtue of a positing
which is equally the sublation
of what is presupposed,
or because the absolute substance
has returned to itself only
out of, and in, its positedness
and for that reason is absolute.
Hence this reciprocal action is
appearance that again sublates itself;
the revelation that the reflective shine of causality,
in which the cause is as cause,
is just that, that it is reflective shine.
This infinite immanent reflection
(that the being-in-and-for-itself is
only such by being a positedness)
is the consummation of substance.
But this consummation is
no longer the substance itself
but is something higher,
the concept, the subject.
The transition of the relation of substantiality
occurs through its own immanent necessity
and is nothing more than the manifestation of itself,
that the concept is its truth,
and that freedom is the truth of necessity.

Earlier, in Book Two of the Objective Logic,
I have already called attention to the fact that
the philosophy that assumes its position
at the standpoint of substance and stops there
is the system of Spinoza.
I have also indicated there the defect of this system,
both with respect to form and matter.
Something else, however, is the refutation of it.
Elsewhere, in connection with the
refutation of a philosophical system,
I have also remarked quite in general
that we must get over the distorted idea
that that system has to be represented
as if thoroughly false,
and as if the true system stood
to the false as only opposed to it.
It is on the basis of the context within which
the system of Spinoza is presented here that
we can see its true standpoint
and ask whether the system is true or false.
The relation of substantiality was
generated by the nature of essence;
this relation and also its exposition
as an expanded totality in the form of system
is, therefore, a necessary standpoint
at which the absolute positions itself.
Such a standpoint, therefore, is not
to be regarded as just an opinion,
an individual's subjective, arbitrary
way of representing and thinking,
as an aberration of speculation;
on the contrary, speculation necessarily runs into it
and, to this extent, the system is perfectly true.
But it is not the highest standpoint.
By itself alone, therefore, the system
cannot be regarded as false,
as either requiring or being capable of refutation.
This alone is rather to be considered false in it:
that it would be the highest standpoint.
It also follows that the true system cannot be
related to it as just its opposite,
for as so opposed it would itself be one-sided.
Rather, as the higher system, it must contain
it within as its subordinate.

Further, any refutation would have to come not from outside,
that is, not proceed from assumptions lying outside
the system and irrelevant to it.
The system need only refuse to recognize those assumptions;
the defect is such only for one who starts from such
needs and requirements as are based on them.
For this reason it has been said that there cannot be
any refutation of Spinozism for anyone
who does not presuppose a commitment to freedom
and the independence of a self-conscious subject.
Besides, a standpoint so lofty and inherently so rich as
that of the relation of substance does not ignore
those assumptions but even contains them:
one of the attributes of the Spinozistic substance is thought.
The system knows how to resolve and assimilate the
determinations in which these assumptions conflict with it,
so that they re-emerge in it, but duly modified accordingly.
The nerve, therefore, of any external refutation
consists solely in obstinately clinging
to the opposite categories of these assumptions,
for example, to the absolute self-subsistence of
the thinking individual as against the form of thought
which in the absolute substance is
posited as identical with extension.
Effective refutation must infiltrate the opponent's stronghold
and meet him on his own ground;
there is no point in attacking him outside his territory
and claiming jurisdiction where he is not.
The only possible refutation of Spinozism
can only consist, therefore, in first acknowledging
its standpoint as essential and necessary
and then raising it to a higher standpoint
on the strength of its own resources.
The relation of substantiality,
considered simply on its own,
leads to its opposite:
it passes over into the concept.
The exposition in the preceding Book
of substance as leading to the concept is, therefore,
the one and only true refutation of Spinozism.
It is the unveiling of substance,
and this is the genesis of the concept
the principal moments of which we have documented above.
The unity of substance is its relation of necessity.
But this unity is thus only inner necessity.
By positing itself through the moment of absolute negativity,
it becomes manifested or posited identity,
and also, therefore, the freedom which is the identity
of the concept.
This concept, the totality resulting from
the relation of reciprocity,
is the unity of the two substances
that stand in that relation,
but in such a way now that the two belong to freedom:
they no longer possess their identity blindly,
that is to say, internally;
on the contrary, the substances now explicitly have
the determination that they are essentially
reflective shine or moments of reflection,
and for that reason that each has immediately
rejoined its other or its positedness,
that each contains this
positedness in itself and in its other,
therefore, is posited simply and solely
as identical with itself.

In the concept, therefore,
the kingdom of freedom is disclosed.
The concept is free because the identity
that exists in and for itself
and constitutes the necessity of substance
exists at the same time as sublated or as positedness,
and this positedness, as self-referring, is that very identity.
Vanished is the obscurity which the causally related
substances have for each other,
for the originariness of their self-subsistence
that makes them causes has passed over into positedness
and has thereby become self-transparently clear;
the “originary fact” is “originary” because
it is a “self-causing fact,”
and this is the substance that has been
let go freely into the concept.

The direct result for the concept is
the following more detailed determination.
Because being which is in and for itself is
immediately a positedness,
the concept is in its simple self-reference
an absolute determinateness
which, by referring only to itself, is
however no less immediately simple identity.
But this self-reference of the determinateness
in which the latter rejoins itself
is just as much the negation of determinateness,
and thus the concept, as this equality with itself, is the universal.
But this identity equally has the determination of negativity;
it is a negation or determinateness that refers to itself
and as such the concept is the singular.
Each, the universal and the singular, is a totality;
each contains the determination of the other within it
and therefore the two are just as absolutely one totality
as their oneness is the diremption of itself into
the free reflective shine of this duality.
And this is a duality which in the differentiation
of singular and universal appears to be perfect opposition,
but an opposition which is so much of a reflective shine that,
in that the one is conceptualized and said,
immediately the other is therein conceptualized and said.
The determinateness does not go past itself,
even excludes the possibility of going past itself.
In this sense, because it precludes reference
to anything else besides itself,
it ceases to be a determinateness
and becomes a universal.
It negates its own determinateness,
i.e. itself as negation.
In precluding reference to anything besides itself,
the universal regains negativity.
It is just itself and nothing else.
In this sense, it is a universe by itself, a singular.

The foregoing is to be regarded as
the concept of the concept.
To some it may seem to depart
from the common understanding of “concept,”
and they might require that we indicate
how our result fits with other ways
of representing or defining it.
But, for one thing, this cannot be
an issue of proof based on the
authority of ordinary understanding.
In the science of the concept,
the content and determination of the latter
can be proven solely on the basis
of an immanent deduction which contains its genesis,
and such a deduction lies behind us.
And also, whereas the concept of the concept
as deduced here should in principle
be recognized in whatever else
is otherwise adduced as such a concept,
it is not as easy to ascertain what
others have said about its nature.
For in general they do not bother at all
enquiring about it but presuppose
that everyone already understands what
the concept means when speaking of it.
Of late especially one may indeed
believe that it is not worth pursuing
any such enquiry because, just as it was
for a while the fashion to say all things bad
about the imagination, then about memory,
it became in philosophy the habit some time ago,
and is still the habit now,
to heap every kind of defamation on the concept,
to hold it in contempt,
the concept which is the highest form of thought,
while the incomprehensible and the non-comprehended are
regarded as the pinnacle of both science and morality.

I confine myself to one remark which may contribute
to the comprehension of the concept here developed
and facilitate one's way into it.
The concept, when it has progressed to
a concrete existence which is itself free,
is none other than the “I” or pure self-consciousness.
True, I have concepts, that is, determinate concepts;
but the “I” is the pure concept itself,
the concept that has come into determinate existence.
It is fair to suppose, therefore,
when we think of the fundamental determinations
which constitute the nature of the “I,”
that we are referring to something familiar,
that is, a commonplace of ordinary thinking.
But the “I” is in the first place purely self-referring unity,
and is this not immediately but by abstracting
from all determinateness and content
and withdrawing into the freedom
of unrestricted equality with itself.
As such it is universality, a unity that
is unity with itself only by virtue of its negative relating,
which appears as abstraction,
and because of it contains all determinateness
within itself as dissolved.
In second place, the “I” is just as immediately
self-referring negativity, singularity,
absolute determinateness that stands opposed
to anything other and excludes it:
individual personality.
This absolute universality which is just as
immediately absolute singularization
(a being-in-and-for-itself which is absolute positedness
and being-in-and-for-itself only by virtue of
its unity with the positedness)
this universality constitutes the
nature of the “I” and of the concept;
neither the one nor the other can be comprehended
unless these two just given moments are grasped at the same time,
both in their abstraction and in their perfect unity.

When I say of the understanding that I have it,
according to ordinary ways of speaking,
what is being understood by it is
a faculty or a property that
stands in relation to my I
in the same way as the property of a thing
stands related to that thing
as to an indeterminate substrate
which is not the true ground
or the determining factor of the property.
In this view, I have concepts, and I have the concept,
just as I also have a coat, complexion,
and other external properties.
Kant went beyond this external relation
of the understanding,
as the faculty of concepts and of the concept,
to the “I.”
It is one of the profoundest and truest insights
to be found in the Critique of Reason
that the unity which constitutes
the essence of the concept is recognized
as the original synthetic unity of apperception,
the unity of the “I think,” or of self-consciousness.
This proposition is all that there is to the so-called
transcendental deduction of the categories
which, from the beginning, has however been regarded
as the most difficult piece of Kantian philosophy;
no doubt only because it demands that we should transcend
the mere representation of the relation
of the “I” and the understanding, or of the concepts,
to a thing and its properties or accidents,
and advance to the thought of it.
The object, says Kant in the Critique of
Pure Reason (2nd edn, p. 137), is that,
in the concept of which the manifold
of a given intuition is unified.
But every unification of representations
requires a unity of consciousness in the synthesis of them.
Consequently, this unity of consciousness is
alone that which constitutes the reference
of the representations to an object,
hence their objective validity,
and that on which even the possibility of the understanding rests.
Kant distinguishes this objective unity from
the subjective unity of consciousness
by which the “I” becomes conscious of a manifold,
whether simultaneously or successively
depending on empirical conditions.
In contrast to this subjective unity,
the principles of the objective determination
of representations are only to be derived
from the principle of the transcendental unity of apperception.
It is by virtue of the categories,
which are these objective determinations,
that the manifold of given representations is
so determined as to be brought
to the unity of consciousness.
On this explanation,
the unity of the concept is that by virtue of which
something is not the determination of mere feeling,
is not intuition or even mere representation, but an object,
and this objective unity is the unity of the “I” with itself.
In point of fact, the conceptual comprehension of a subject matter
consists in nothing else than in the “I” making it its own,
in pervading it and bringing it into its own form,
that is, into a universality which is immediately determinateness,
or into a determinateness which is immediately universality.
As intuited or also as represented, the subject matter is
still something external, alien.
When it is conceptualized, the being-in-and-for-itself
that it has in intuition and representation is
transformed into a positedness;
in thinking it, the “I” pervades it.
But it is only in thought that it is in and for itself;
as it is in intuition or representation, it is appearance.
Thought sublates the immediacy with which it first comes before us
and in this way transforms it into a positedness;
but this, its positedness, is its being-in-and-for-itself
or its objectivity.
This is an objectivity which the subject matter
consequently attains in the concept,
and this concept is the unity of self-consciousness
into which that subject matter has been assumed;
consequently its objectivity or the concept is
itself none other than the nature of self-consciousness,
has no other moments or determinations than the “I” itself.

Accordingly, we find in a fundamental principle of Kantian philosophy
the justification for turning to the nature of the “I”
in order to learn what the concept is.
But conversely, it is necessary to this end
that we have grasped the concept of the “I” as stated.
If we cling to the mere representation of the “I”
as we commonly entertain it,
then the “I” is only the simple thing
also known as the soul,
a thing in which the concept inheres
as a possession or a property.
This representation, which does not bother
to comprehend either the “I” or the concept,
is of little use in facilitating or advancing
the conceptual comprehension of the concept.

The position of Kant just cited contains
two other points which concern the concept
and necessitate some further comments.
First of all, preceding the stage of understanding are
the stages of feeling and of intuition.
It is an essential proposition of Kant's Transcendental Philosophy
that concepts without intuition are empty,
and that they have validity only as references
connecting the manifold given by intuition.
Second, the concept is given as
the objective element of cognition,
consequently as the truth.
Yet it is taken to be something merely subjective,
and we are not allowed to extract reality from it,
for by reality objectivity is to be understood,
since reality is contrasted with subjectivity.
Moreover, the concept and anything logical are
declared to be something merely formal
which, since it abstracts from content,
does not contain truth.

Now, in the first place, as regards the relation
of the understanding or concept to the stages presupposed by it,
the determination of the form of these stages
depends on which science is being considered.
In our science, which is pure Logic,
they are being and essence.
In Psychology, the stages preceding the understanding are
feeling and intuition, and then representation generally.
In the Phenomenology of Spirit,
which is the doctrine of consciousness,
the ascent to the understanding is
made through the stages of sensuous consciousness
and then of perception.
Kant places ahead of it only feeling and intuition.
But, for a start, he himself betrays the incompleteness of
this progression of stages by appending
to the Transcendental Logic
or the Doctrine of the Understanding
a treatise on the concepts of reflection,
a sphere lying between intuition and understanding,
or being and concept.
And if we consider the substance itself of these stages,
it must first be said that such shapes as
intuition, representation, and the like,
belong to the self-conscious spirit
which, as such, does not fall within
the scope of logical science.
Of course, the pure determinations of
being, essence, and the concept,
also constitute the substrate
and the inner sustaining structure
of the forms of spirit;
spirit, as intuiting as well as sensuous consciousness,
is in the form of immediate being,
just as spirit as representational
and also perceptual consciousness
has risen from being to the stage of essence or reflection.
But these concrete shapes are of as little interest
to the science of logic as are the concrete forms
that logical determinations assume in nature.
These last would be space and time,
then space and time as assuming a content,
as inorganic and then organic nature.
Similarly, the concept is also
not to be considered here as
the act of the self-conscious understanding,
not as subjective understanding,
but as the concept in and for itself
which constitutes a stage of nature
as well as of spirit.
Life, or organic nature, is the stage of nature
where the concept comes on the scene,
but as a blind concept that does not comprehend itself,
that is, is not thought;
only as self-aware and as thought does it belongs to spirit.
Its logical form, however, is independent of such shapes,
whether unspiritual or spiritual.
This is a point which was already
duly adumbrated in the Introduction,
and one that one must be clear about
before undertaking Logic,
not when one is already in it.

But, in second place, how the forms that
precede the concept might ever be shaped
depends on how the concept is thought in relation to them.
This relation, as assumed in ordinary psychology
as well as in Kant's Transcendental Philosophy,
is that the empirical material,
the manifold of intuition and representation,
is at first just there by itself,
and that the understanding then comes into it,
brings unity to it, and raises it
through abstraction to the form of universality.
The understanding is in this way
an inherently empty form
which, on the one hand, obtains reality
only by virtue of that given content,
and, on the other, abstracts from it,
that is to say,
discards it as something useless,
but useless only for the concept.
In both operations,
the concept is not the one which is independent,
is not what is essential and true
about that presupposed material;
rather, this material is the
reality in and for itself,
a reality that cannot be
extracted from the concept.

Now it must certainly be conceded that
the concept is as such not yet complete,
that it must rather be raised to the idea
which alone is the unity of the concept and reality;
and this is a result which will have to emerge
in what follows from the nature of the concept itself.
For the reality that the concept gives itself
cannot be picked up as it were from the outside
but must be derived from the concept itself
in accordance with scientific requirements.
But the truth is that it is not the material
given by intuition and representation
which must be validated as the real in contrast to the concept.
“It is only a concept,” people are wont to say,
contrasting the concept, as superior to it,
not only with the idea, but with sensuous,
spatial and temporal, palpable existence.
For this reason the abstract is then held
to be of less significance than the concrete,
because so much of this palpable
material has been removed from it.
In this view, to abstract means to
select from a concrete material this or that mark,
but only for our subjective purposes,
without in any way detracting from the value
and the status of the many other
properties and features that are left out;
on the contrary, by retaining them as reality,
but yonder on the other side,
still as fully valid as ever.
It is only because of its incapacity
that the understanding thus does not
draw from this wealth
and is forced rather to make do
with the impoverished abstraction.
But now, to regard the given material of intuition
and the manifold of representation as the real,
in contrast to what is thought and the concept,
is precisely the view that must be given up as
condition of philosophizing,
and that religion, moreover, presupposes as
having already been given up.
How could there be any need of religion,
how could religion have any meaning,
if the fleeting and superficial appearance
of the sensuous and the singular were
still regarded as the truth?
But it is philosophy that yields the
conceptually comprehended insight into the
status of the reality of sensuous being.
Philosophy assumes indeed that the
stages of feeling, intuition, sense consciousness,
and so forth, are prior to the understanding,
for they are the conditions of the genesis of the latter,
but they are conditions only in the sense that
the concept results from their dialectic and their nothingness
and not because it is conditioned by their reality.
Abstractive thought, therefore, is not to be regarded as the
mere discarding of a sensuous material
which does not suffer in the process
any impairment of reality;
it is rather the sublation and reduction of that
material as mere appearance to the essential,
which is manifested only in the concept.
Of course, if what is to be taken up
into the concept from the concrete appearance is
intended to serve only as a mark or sign,
then it may well be anything at all,
any mere sensuous singular determination
of the subject matter will do,
selected from the others
because of some external interest
but of like kind and nature as the rest.

In this conjunction, the prevailing fundamental
misunderstanding is that the natural principle,
or the starting point in the natural development
or the history of an individual in the process of self-formation,
is regarded as the truth and conceptually the first.
Intuition or being are no doubt first in the order of nature,
or are the condition for the concept,
but they are not for all that the unconditioned in and for itself;
on the contrary, in the concept their reality is sublated
and, consequently, so is also the reflective shine
that they had of being the conditioning reality.
If it is not the truth which is at issue but only narration,
as it is the case in pictorial and phenomenal thinking,
then we might as well stay with the story
that we begin with feelings and intuitions,
and that the understanding then extracts a universal
or an abstraction from their manifold,
for which purpose it quite understandably needs
a substrate for these feelings and intuitions
which, in the process of abstraction, retains
for representation the same complete reality
with which it first presented itself.
But philosophy ought not to be a narrative of what happens,
but a cognition of what is true in what happens,
in order further to comprehend on the basis of this truth
what in the narrative appears as a mere happening.

If on the superficial view of what the concept is
all manifoldness falls outside it,
and only the form of abstract universality
or of empty reflective identity stays with it,
we can at once call attention to the fact that
any statement or definition of a concept expressly requires,
besides the genus which in fact is already itself
more than just abstract universality,
also a specific determinateness.
And it does not take much thoughtful reflection
on the implication of this requirement to see
that differentiation is an equally
essential moment of the concept.
Kant introduced this line of reflection
with the very important thought
that there are synthetic judgments a priori.
His original synthesis of apperception is
one of the most profound principles
for speculative development;
it contains the beginning of a
true apprehension of the nature of the concept
and is fully opposed to any
empty identity or abstract universality
which is not internally a synthesis.
The further development, however,
did not live up to this beginning.
The term itself, “synthesis,” easily conjures up
again the picture of an external unity,
of a mere combination of terms that are intrinsically separate.
Then, again, the Kantian philosophy has never got over
the psychological reflex of the concept
and has once more reverted to the claim
that the concept is permanently conditioned
by the manifold of intuition.
It has declared the content of the cognitions
of the understanding, and of experience, to be phenomenal,
not because of the finitude of the categories as such
but, on the ground of a psychological idealism,
because they are only determinations derived from self-consciousness.
Here accordingly we have again the supposition
that apart from the manifoldness of intuition
the concept is without content, empty,
despite the fact that the concept is
said to be a synthesis a priori;
as such, it surely contains determinateness
and differentiation within itself.
And because this determinateness is
the determinateness of the concept,
and hence the absolute determinateness, singularity,
the concept is the ground and the source
of all finite determinateness and manifoldness.

The formal position that the concept
never abandons as understanding
is completed in Kant's exposition of what reason is.
One should expect that in reason,
which is the highest stage of thought,
the concept would lose the conditionality
with which it still appears at the stage of understanding
and would attain perfect truth.
But this expectation is disappointed.
For Kant defines the relation of reason
to the categories as merely dialectical.
Indeed, he even takes the result of this dialectic
to be simply and solely an infinite nothingness,
the result being that the synthesis is again lost,
lost also to the infinite unity of reason,
and lost with it is whatever beginning
there was of a speculative, truly infinite, concept;
reason becomes the well-known, totally formal,
merely regulative unity of the
systematic employment of the understanding.
It is declared an abuse when Logic,
which is supposed to be a mere canon of judgment,
is considered instead as an organon for
the production of objective insights.
The concepts of reason, in which we would have expected
a higher power and a deeper content,
no longer possess anything constitutive
as still do the categories;
they are mere ideas
which we are of course are quite at liberty to use,
provided that by these intelligible entities
in which all truth was to be revealed
we mean nothing more than hypotheses
to which it would be the height
of arbitrariness and recklessness
to ascribe absolute truth,
for they cannot be found in any experience.
Would anyone have ever thought that
philosophy would deny truth to intelligible entities
on the ground that they lack the
spatial and temporal material of the senses?

Directly connected with this is
the issue of how to view the concept
and the character of logic generally,
the issue namely of the relation of
the concept and its science to truth itself.
This is an issue on which the Kantian philosophy
holds the same position as is commonly taken.
We cited earlier from Kant's deduction of the categories
to the effect that, according to it,
the object in which the manifold of intuition is unified is
this unity only by virtue of the unity of self-consciousness.
The objectivity of thought is here, therefore, specifically defined:
it is an identity of concept and thing which is the truth.
In the same way it is also commonly accepted that,
as thought appropriates a given subject matter,
this subject matter thereby undergoes an alteration
and is made from something sensuous into something thought.
But nothing is changed in this alteration
in so far as the essentiality of the object goes;
on the contrary, it is accepted that the object is
in its truth only in its concept,
whereas in the immediacy in
which it is given it is only appearance and accidentality;
that the cognition conceptualizing the subject matter is
a cognition of it as it is in and for itself,
and the concept is its very objectivity.
But, on the other hand, it is also equally claimed
that we cannot know things as they are in and for themselves
and that truth is inaccessible to rational cognition;
that the aforesaid truth that would consist
in the unity of the object
and the concept is in fact only appearance,
again on the ground now that the content is
only the manifold of intuition.
But we have just remarked, regarding this point,
that it is precisely in the concept
that the manifold is sublated inasmuch
as it pertains to intuition as opposed to the concept,
and that through the concept the subject matter is
reduced to its non-contingent essentiality;
the latter does enter into appearance,
and this is why appearance is not
something merely essenceless
but is the manifestation of essence.
When this manifestation of essence is set free,
then we have the concept.
These propositions that we are now recalling
are not dogmatic assertions,
for they are results that obtained on their own
out of the whole development of essence.
The present position to which this development has led
is that the form of the absolute
which is higher than being and essence is the concept.
Viewed from this side, the concept has subjugated
the spheres of being and essence to which,
from other starting points,
feeling, intuition, and representation,
which appeared to be its antecedent conditions,
also belong;
it has demonstrated itself to be
their unconditional foundation.
But this is one side alone.
There is a second side left
to which this third book of the Logic is devoted,
namely the demonstration of how
the concept forms within and from itself
the reality that has vanished in it.
It is conceded, in other words,
that the cognition that does not go past the concept,
purely as concept, is still incomplete,
that it has only arrived at abstract truth.
But its incompleteness does not lie in
its lack of that alleged reality as would be
given in feeling and intuition,
but in the fact that the concept has yet to give
to itself its own reality,
one that it generates out of itself.
The demonstrated absoluteness of the concept
as against the material of experience and, more exactly,
the categorial and the reflective determinations of it,
consists in this,
that as this material appears
outside and before the concept,
it has no truth but that it has it only in its ideality
or in its identity with the concept.
The derivation of the real from the concept,
if “derivation” is what we want to call it,
consists at first essentially in this,
that the concept in its formal abstraction
reveals itself to be incomplete
and through a dialectic immanently grounded in it
passes over into reality:
it passes over into it, however,
as into something which it generates out of itself,
not as if it were falling back again
onto a ready-made reality
which it finds opposite it,
or as if it were taking refuge,
because it sought for something better
but found none, into something
that has already been proven to be
the unessential element of appearance.
It will always be a source of wonder
how the Kantian philosophy did acknowledge
that the relation of thought to sensuous existence
(the relation at which it stopped)
is only a relation of mere appearance,
and also well recognized in the idea in general
a higher unity of those two terms,
even gave expression to it,
as for example in the idea
of an intuitive understanding,
and yet stopped short at that relative relation
and at the claim that the concept
remains utterly separate from reality,
thus asserting as truth
what it declared to be finite cognition,
and explaining away as extravagant
and illegitimate figments of thought
what it recognized as truth
and had specifically defined as such.

Since it is logic above all
and not science generally
whose relation to truth is the issue here,
it must be further conceded that
logic as the formal science cannot also contain,
nor should contain, the kind of reality which
is the content of the other parts of philosophy,
of the sciences of nature and of spirit.
These concrete sciences do attain to a more real form
of the idea than logic does,
but not because they have turned back to the reality
which consciousness abandoned as it rose above
the appearance of it to science,
or because they have again resorted to
the use of such forms as are the categories
and the determinations of reflection,
the finitude and untruth of which
were demonstrated in the logic.
The logic rather exhibits the rise of the idea
up to the level from which it becomes the creator of nature
and passes over into the form of a concrete immediacy
whose concept, however, again shatters this shape also
in order to realize itself as concrete spirit.
These sciences, just as they had the logic as their prototype,
hold on to its logical principle or the concept
as in them their formative factor.
As contrasted with them, the logic is of course the formal science,
yet the science of the absolute form which is implicit totality
and contains the pure idea of truth itself.
This absolute form has in it a content or reality of its own;
the concept, since it is not a trivial, empty identity,
obtains its differentiated determinations in
the moment of negativity or of absolute determining;
and the content is only these determinations of
the absolute form and nothing else,
a content posited by the form itself
and therefore adequate to it.
This form is for this reason of quite another nature than
logical form is ordinarily taken to be.
It is truth already on its own account,
because this content is adequate to its form
or this reality to its concept,
and it is pure truth,
because the determinations of the content
do not yet have the form of an absolute otherness
or of absolute immediacy.
When Kant in the Critique of Pure Reason (p. 83),
in connection with logic comes to discuss
the old and famous question:
What is truth?,
he starts by passing off as a triviality
the nominal definition that it is the agreement
of cognition with its subject matter,
a definition which is of great, indeed of supreme value.
If we recall this definition together with
the fundamental thesis of transcendental idealism,
namely that rational cognition is incapable
of comprehending things in themselves,
that reality lies absolutely outside the concept,
it is then at once evident that such a reason,
one which is incapable of setting itself
in agreement with its subject matter,
and the things in themselves,
such as are not in agreement with the rational concept
a concept that does not agree with reality
and a reality that does not agree
with the concept;
that these are untrue conceptions.
If Kant had measured the idea of
an intuitive understanding against
that first definition of truth,
he would have treated that idea
which expresses the required agreement,
not as a figment of thought but rather as truth.

“What we would want to know,” Kant proceeds to say,
“is a universal and certain criterion of
truth of any cognition whatever,
one that would be valid for all cognitions
without distinction of their subject matters;
but since any such criterion would abstract from
all content of cognition (the reference to its object),
and truth has to do precisely with this content,
it would be quite impossible, even absurd,
to ask for a mark of the truth
of this content of cognitions.”
Here we have, clearly expressed,
the ordinary conception of
the formal function of logic
which gives to the adduced argument
the air of convincing.
But first of all it is to be noted
what usually happens to this
kind of formal argumentation:
it forgets as it speaks that on which it is based
and of which it speaks.
It would be absurd, it says, to ask for a criterion
of the truth of the content of cognition.
But according to the definition on
which it is based it is not
the content that constitutes the truth,
but the agreement of it with the concept.
Such a content as is here spoken of,
one without the concept,
is something void of concept
and therefore void of essence;
of course, we cannot ask of
such a content for a criterion of truth,
but for the opposite reason, namely,
not because it cannot be the required
agreement on account of its being void of concept,
but because it cannot be anything more than
just another truthless opinion.
Let us leave aside any talk about content,
which is the cause of the confusion here
the confusion in which formalism invariably falls,
and which is responsible for making it say,
every time it tries to explain itself,
the opposite of what it wants to say
and let us just stay with the abstract view
that the logic is only formal,
that it abstracts from all content.
What we then have is a one-sided cognition
which is not supposed to contain any subject matter,
an empty form void of determination
which is therefore just as little an agreement
(for it necessarily takes two for an agreement)
as it is truth.
In the a priori synthesis of the concept,
Kant did have a higher principle
in which it was possible to recognize a duality
and therefore what is required for truth;
but the material of the senses,
the manifoldness of intuition,
was too strong for him to be able
to wrest himself away from it
and turn to a consideration of the concept
and the categories in and for themselves,
and to a speculative form of philosophizing.

Since logic is the science of the absolute form,
this formal discipline, in order to be true,
must have a content in it which is adequate to its form;
all the more so, because logical form is pure form
and logical truth, accordingly, the pure truth itself.

This formal discipline must therefore be thought of
as inherently much richer in determinations and content,
and also of infinitely greater efficacy over the concrete,
than it is normally taken to be.

The laws of logic by themselves
(extraneous elements aside, such as applied logic
and the rest of the psychological and anthropological material)
are commonly restricted, apart from the law of contradiction,
to a few meager propositions concerning
the conversion of judgments and the forms of inference.

And the forms, too, that come up in this context,
as well as their further specifications,
are only taken up historically as it were,
not subjected to criticism to see whether
they are in and for themselves true.
For example, the form of the positive judgment
is accepted as something perfectly correct in itself,
and whether the judgment is true is made
to depend solely on the content.
No thought is given to investigating
whether this form of judgment is
a form of truth in and for itself;
whether the proposition it enunciates,
“the individual is a universal,”
is not inherently dialectical.
It is at once assumed that
the judgment is capable of possessing
truth on its own account,
 and that every proposition expressed in a positive
judgment is true,
even though it is patently evident that the judgment lacks
what is required by the definition of truth,
 namely the agreement of the concept with its subject matter;
for if the predicate, which here is the universal,
is taken as the concept,
and the subject, which is the singular,
as the subject matter,
then the concept does not agree with it.
But if the abstract universal which is the predicate
does not yet amount to a concept
(for surely there is more that belongs to it);
or if the subject, for its part,
still is not much more than a grammatical one,
how should the judgment
possibly contain truth seeing that its concept and the intended object do
not agree, as also that the concept is missing and indeed the object as well?
This rather is then where the impossible and the absurd lie,
in the attempt to grasp the truth in such forms as are
the positive judgment or a judgment in general.
Just as the Kantian philosophy did not consider
the categories in and for themselves,
but declared them to be finite determinations unfit
to hold the truth, on the only inappropriate ground
that they are subjective forms of self-consciousness,
still less did it subject to criticism
the forms of the concepts that make up
the content of ordinary logic.
What it did, rather, is to pick a portion of them,
namely the functions of judgments,
for the determination of categories,
and simply accepted them as valid presuppositions.
Even if there were nothing more to the forms of logic
than these formal functions of judgment,
for that reason alone they would already be
worthwhile investigating to see how far,
by themselves, they correspond to the truth.
A logic that does not perform this task
can at most claim the value of
a natural description of the phenomena of thought
as they simply occur.
It is an infinite merit of Aristotle,
one that must fill us with the highest admiration
for the power of his genius,
that he was the first to undertake this description.
But it is necessary to go further
and determine both the systematic connection
of these forms and their value.

DIVISION

The concept, as considered so far,
has demonstrated itself to be
the unity of being and essence.
Essence is the first negation of being,
which has thereby become reflective shine;
the concept is the second negation,
or the negation of this negation,
and is therefore being
which has been restored once more,
but as in itself the infinite mediation
and negation of being.
In the concept, therefore,
being and essence no longer have
determination as being and essence,
nor are they only in such a unity
in which each would reflectively shine in the other.
Consequently, the concept does not differentiate
itself into these determinations.
The concept is the truth of the substantial relation
in which being and essence attain their perfect
self-subsistence and determination each through the other.
The truth of substantiality proved
to be the substantial identity,
an identity that equally is,
and only is, positedness.
Positedness is determinate existence and differentiation;
in the concept, therefore, being-in-and-for-itself
has attained a true existence adequate to it,
for that positedness is itself being-in-and-for-itself.
This positedness constitutes the difference
of the concept in the concept itself;
and because the concept is
immediately being-in-and-for-itself,
its differences are themselves the whole concept,
universal in their determinateness
and identical in their negation.

This is now the concept itself of the concept,
but at first only the concept of the concept
or also itself only concept.
Since the concept is being-in-and-for-itself
by being a positedness, or is absolute substance,
and substance manifests the necessity of
distinct substances as an identity,
this identity must itself posit what it is.
The moments of the movement of the substantial relation
through which the concept came to be
and the reality thereby exhibited are
only in the transition to the concept;
that reality is not yet the
concept's own determination,
one that has emerged out of it;
it fell in the sphere of necessity
whereas the reality of the concept
can only be its free determination,
a determinate existence in which
the concept is identical with itself
and whose moments are themselves concepts
posited through the concept itself.

At first, therefore, the concept is
only implicitly the truth;
because it is only something inner,
it is equally only something outer.
It is at first simply an immediate
and in this shape its moments have
the form of immediate, fixed determinations.
It appears as the determinate concept,
as the sphere of mere understanding.
Because this form of immediacy is an existence
still inadequate to the nature of the concept,
for the concept is free and only refers to itself,
it is an external form in which the concept
does not exist in-and-for-itself,
but can only count as something posited or subjective.
The shape of the immediate concept
constitutes the standpoint that makes
of the concept a subjective thinking,
a reflection external to the subject matter.
This stage constitutes, therefore, subjectivity,
or the formal concept.
Its externality is manifested in
the fixed being of its determinations
that makes them come up each by itself,
isolated and qualitative,
and each only externally referred to the other.
But the identity of the concept,
which is precisely their inner or subjective essence,
sets them in dialectical movement,
and through this movement their singleness is sublated
and with it also the separation of
the concept from the subject matter,
and what emerges as their truth is
the totality which is the objective concept.

Second, in its objectivity the concept is
the fact itself as it exists in-and-for-itself.
The formal concept makes itself into the fact
by virtue of the necessary determination of its form,
and it thereby sheds the relation
of subjectivity and externality
that it had to that matter.
Or, conversely, objectivity is the real concept
that has emerged from its inwardness
and has passed over into existence.
In this identity with the fact,
the concept thus has an existence
which is its own and free.
But this existence is still a freedom
which is immediate and not yet negative.
Being at one with the subject matter,
the concept is submerged into it;
its differences are objective
determinations of existence
in which it is itself again the inner.
As the soul of objective existence,
the concept must give itself the form of subjectivity
that it immediately had as formal concept;
and so, in the form of the free concept
which in objectivity it still lacked,
it steps forth over against that objectivity
and, over against it, it makes therein the identity with it,
which as objective concept it has in and for itself,
into an identity that is also posited.

In this consummation in which
the concept has the form of freedom
even in its objectivity,
the adequate concept is the idea.
Reason, which is the sphere of the idea,
is the self-unveiled truth
in which the concept attains
the realization absolutely adequate to it,
and is free inasmuch as in this real world,
in its objectivity, it recognizes its subjectivity,
and in this subjectivity recognizes that objective world.

SECTION I

Subjectivity

The concept is, to start with, formal,
the concept in its beginning
or as the immediate concept.
In this immediate unity,
its difference or its positedness
is, first, itself initially simple
and only a reflective shine,
so that the moments of the difference
are immediately the totality of the concept
and only the concept as such.

But, second, because it is absolute negativity,
the concept divides and posits itself
as the negative or the other of itself;
yet, because it is still immediate concept,
this positing or this differentiation is
characterized by the reciprocal
indifference of its moments,
each of which comes to be on its own;
in this division the unity of the concept is
still only an external connection.
Thus, as the connection of its moments
posited as self-subsisting and indifferent,
the concept is judgment.

Third, although the judgment contains
the unity of the concept that has been lost
in its self-subsisting moments,
this unity is not posited.
It will become posited by virtue of
the dialectical movement of the judgment
which, through this movement,
becomes syllogistic inference,
and this is the fully posited concept,
for in the inference the moments of
the concept as self-subsisting extremes
and their mediating unity are both equally posited.

But since this unity itself, as unifying middle,
and the moments, as self-subsisting extremes,
stand at first immediately opposite one another,
this contradictory relation that occurs
in the formal inference sublates itself,
and the completeness of the concept passes over
into the unity of totality;
the subjectivity of the concept
into its objectivity.

CHAPTER 1

The concept

The faculty of concepts is normally
associated with the understanding,
and the latter is accordingly distinguished
from the faculty of judgment
and from the faculty of syllogistic inferences
which is formal reason.
But it is particularly with reason
that the understanding is contrasted,
and it signifies then, not the faculty of concepts in general,
but the faculty of determinate concepts,
as if, as the prevailing opinion has it,
the concept were only a determinate.
When distinguished in this meaning
from the formal faculty of judgment and from formal reason,
the understanding is accordingly to be taken
as the faculty of the single determinate concept.
For the judgment and the syllogism or reason, as formal,
are themselves only a thing of the understanding,
since they are subsumed under the form
of the abstract determinateness of the concept.
Here, however, we are definitely not taking
the concept as just abstractly determined;
the understanding is therefore
to be distinguished from reason only
in that it is the faculty of the concept as such.

This universal concept that we now have to consider
contains the three moments of
universality, particularity, and singularity.
The difference and the determinations which the concept
gives itself in its process of distinguishing constitute
the sides formerly called positedness.
Since this positedness is in the concept
identical with being-in-and-for-itself,
each of the moments is just as much
the whole concept as it is determinate concept
and a determination of the concept.

It is at first pure concept,
or the determination of universality.
But the pure or universal concept is also
only a determinate or particular concept
that takes its place alongside the other concepts.
Because the concept is a totality,
and therefore in its universality
or pure identical self-reference
is essentially a determining and a distinguishing,
it possesses in itself the norm
by which this form of its self-identity,
in pervading all the moments
and comprehending them within,
equally determines itself immediately
as being only the universal
as against the distinctness of the moments.

Second, the concept is thereby posited
as this particular or determinate concept,
distinct from others.

Third, singularity is the concept reflecting itself
out of difference into absolute negativity.
This is at the same time the moment at which
it has stepped out of its identity
into its otherness and becomes judgment.

CHAPTER 2

Judgment

Judgment is the determinateness of the concept
posited in the concept itself.
The determinations of the concept,
or, what amounts to the same thing as shown,
the determinate concepts,
have already been considered on their own;
but this consideration was rather
a subjective reflection
or a subjective abstraction.
But the concept is itself this act of abstracting;
the positioning of its determinations over
against each other is its own determining.
Judgment is this positing of the determinate concepts
through the concept itself.

Judging is therefore another function than conceiving;
or rather, it is the other function of the concept,
for it is the determining of the concept through itself.
The further progress of judgment
into a diversity of judgments is
this progressive determination of the concept.
What kind of determinate concepts there are,
and how they prove to be necessary determinations of it,
this has to be exhibited in judgment.

Judgment can therefore be called
the first realization of the concept,
for reality denotes in general
the entry into existence as determinate being.
More precisely, the nature of this realization
has presented itself in such a way
that the moments of the concept are totalities
which, on the one hand, subsist on their own
through the concept's immanent reflection
or through its singularity;
on the other hand, however, the unity of
the concept is their connection.
The immanently reflected determinations are
determinate totalities that exist just as
essentially disconnected, indifferent to each other,
as mediated through each other.
The determining itself is a totality only as
containing these totalities and their connections.
This totality is the judgment.
The latter contains, therefore, the two self-subsistents
which go under the name of subject and predicate.
What each is cannot yet be said;
they are still indeterminate, for they are
to be determined only through the judgment.
Inasmuch as judgment is the concept as determinate,
the only determination at hand is the difference
that it contains between determinate
and still indeterminate concept.
As contrasted to the predicate,
the subject can at first be taken, therefore,
as the singular over against the universal,
or also as the particular over against the universal,
or the singular over against the particular;
so far, they stand to each other only as the more determinate
and the more universal in general.

It is therefore fitting and unavoidable to have these names,
“subject” and “predicate,” for the determinations of the judgment;
as names, they are something indeterminate, still in need of determination,
and therefore nothing but names.
It is partly for this reason that the determinations themselves
of the concept could not be used for the two sides of judgment;
but a still stronger reason is because of the nature of a concept determination
which is nothing abstract, nothing fixed,
but contains its opposite in it,
explicitly posited there;
since the sides of the judgment are themselves concepts
and therefore the totality of the determinations of the concept,
each side must run through all these determinations,
exhibiting them within whether in abstract or concrete form.
But now, if in this altering of determination
we want to fix the two sides in some general way,
names will be the most useful means,
for they can be kept the same throughout the process.
But a name remains distinct from the fact or the concept,
and this is a distinction that transpires within the judgment as such;
since the subject is in general the determinate term and more,
therefore, of an immediate existent,
whereas the predicate expresses the universal,
the essence or the concept, the subject as such is
at first only a kind of name;
what it is, is first enunciated only by the predicate
which contains being in the sense of the concept.
When we ask, “What is this?,” or “What kind of plant is this?,”
the being we are enquiring about is often just a name,
and once we learn this name, we are satisfied
that we now know what the fact is.
This is being in the sense of the subject.
The concept, however, or at least the essence and
the universal in general, is only given by the predicate,
and when we ask for it, we do it in the sense of the judgment.
God, therefore, or spirit, nature, or what have you, is
as the subject of a judgment only a name at first;
what any such subject is in accordance with the concept,
is first found only in the predicate.
When we ask for the predicate that belongs to such subjects,
the required judgment must be based on
a concept that is presupposed;
yet it is the predicate that first gives this concept.
It is, therefore, the mere
representation that in fact makes up the presupposed meaning,
and this yields only a nominal definition whereby it is a mere accident,
a historical fact, what is understood by a name.
So many disputes about whether a
predicate does or does not belong to a subject are, therefore, nothing more
than verbal disputes,
for they proceed from this form;
what lies at the base (subjectum) is
still nothing more than a name.

Secondly, we now have to examine more closely
how the connection of subject and predicate in judgment is determined,
and how the two are themselves thereby determined.
Judgment has in general totalities for its sides,
totalities that are at first essentially self-subsistent.
The unity of the concept is at first, therefore,
only a connection of self-subsistent terms;
it is not yet the concrete,
the fulfilled unity that has returned into itself
from this reality
but is a unity rather outside which the two terms persist as extremes
yet unsublated in it.
Now any consideration of the judgment can start
either from the originative unity of the concept
or from the self-subsistence of the extremes.
Judgment is the self-diremption of the concept;
therefore, it is by starting from the unity
of the concept as ground that the judgment is
considered in accordance with its true objectivity.
In this respect, judgment is the originative division
(or Teilung, in German) of an originative unity;
the German word for judgment, Urteil (or “primordial division”),
thus refers to what judgment is in and for itself.
But the concept is present in the judgment as appearance,
since its moments have attained self-subsistence there,
and it is to this side of externality
that ordinary representation is more likely to fasten.

From this subjective standpoint,
the subject and the predicate are
therefore treated as ready-made,
each for itself outside the other
the subject as a subject matter
that would exist even if it did not have that predicate,
and the predicate as a universal determination
that would exist even without accruing to this subject.
The act of judgment accordingly brings with it
the further reflection whether this or that predicate
which is in someone's head can
and should be attached to the subject matter
that exists outside it on its own;
the judgment itself is simply the act
that combines the predicate with the subject,
so that, if this combination did not occur,
the subject and predicate would still each remain what it is,
the one concretely existing as thing in itself,
the other as a representation in someone's head.
But the predicate which is combined with the subject
should also pertain to it,
which is to say, should be in and for itself identical with it.
The significance of their being combined is
that the subjective sense of judgment,
and the indifferent external persistence of the subject and predicate,
are again sublated.
Thus in “this action is good,” the copula indicates
that the predicate belongs to the being of the subject
and is not merely externally combined with it.
Of course, grammatically speaking this kind of subjective relation
that proceeds from the indifferent externality of subject and predicate is
perfectly valid, for it is words that are here externally combined.
It can also be mentioned in this context that
a proposition can indeed have a subject and predicate
in a grammatical sense without however being a judgment for that.
The latter requires that the predicate behave
with respect to the subject in a relation of conceptual determination,
hence as a universal with respect to a particular or singular.
And if what is said of a singular subject is
itself only something singular, as for instance,
“Aristotle died at the age of 73
in the fourth year of the 115th Olympiad,”
then this is a mere proposition, not a judgment.
There would be in it an element of judgment
only if one of the circumstances, say,
the date of death or the age of the philosopher,
came into doubt even though the stated figures
were asserted on the strength of some ground or other.
In that case, the figures would be taken as something universal,
as a time that, even without the determinate content of Aristotle's death,
would still stand on its own filled with some other content or simply empty.
Likewise would the news that my friend N. has died be a proposition,
and a judgment only if there were a question as to whether
he is actually dead and not just apparently dead.

In the usual definition of judgment,
that it is the combination of two concepts,
we may indeed accept the vague expression
of “combination” for the external copula,
and also accept that the terms combined are
at least meant to be concepts.
But the definition is otherwise a highly superficial one.
It is not just that in the disjunctive judgment, for instance,
there are more than two so-called concepts that are combined;
more to the point is rather that the definition is
much better than the matter defined,
for it is not determinations of concepts,
but determinations of representation that are in fact meant;
it was remarked in connection with the concept in general,
and with the concept as determinate,
that what usually goes under this name of concept
does not deserve the name at all;
where should concepts then come from in the case of judgment?
Above all this definition of judgment
ignores what is essential to it,
namely the difference of its determinations;
still less does it take into account
its relation to the concept.

As regards the further determination of the subject and predicate,
we have remarked above that it is in judgment
that they must first receive their determination.
But since judgment is the posited determinateness of the concept,
this determinateness possesses the given differences
immediately and abstractly as singularity and universality.
But inasmuch as judgment is in general the immediate existence
or the otherness of the concept that has not yet
restored itself to the unity through which it exists as concept,
there also emerges the determinateness that is void of concept,
the opposition of being and reflection or the in-itself.
But since the concept constitutes the essential ground of judgment,
these determinacies are at least indifferent in the sense that,
when one accrues to the subject and the other to the predicate,
the converse relation equally holds.
The subject, being the singular, appears at first as the existent
or as the one that exists for itself with
the determinate determinateness of a singular
on which judgment is passed,
as an actual object even when it is such in representation only
as for instance in the case of bravery, right, agreement, etc.
The predicate, which is the universal,
appears on the contrary as the reflection
of this judgment on that object,
or rather as the object's immanent reflection
that transcends the immediacy of the judgment
and sublates its determinacies as mere existents,
appears, that is, as the object's in-itselfness.
In this way, the start is made
from the singular as the first, the immediate,
and through the judgment
this singular is raised to universality,
just as, conversely,
the universal that exists only in itself
descends in the singular into existence
or becomes a being that exists for itself.

This significance of the judgment is
to be taken as its objective meaning
and at the same time as the true significance
of the previous forms of transition.
The existent comes to be and becomes another,
the finite passes over into the infinite
and in it passes away;
the existent comes forth into appearance
out of its ground and to this ground it founders;
the accidents manifest the wealth of substance
as well as its might;
in being, there is transition into an other;
in essence, there is the reflective shining
in an other that manifests the necessity of a connection.
This transition and this reflective shining have now
passed over into the originative division
of the concept in judgment,
and this division,
in bringing the singular back to
the in-itselfness of its universality,
equally determines the universal as something actual.
These two are one and the same,
the positing of singularity in its immanent reflection
and of the universal as determinate.

But equally pertaining to this objective meaning is
that the said differences,
as they re-occur in the determinateness of the concept,
are at the same time posited as only appearing,
that is to say, that they are nothing fixed
but accrue rather just as much to one determination
of the concept as to the other.
The subject is therefore equally to be taken as the in-itself,
and the predicate as determinate existence in contrast to it.
The subject without the predicate is what the thing without properties,
the thing-in-itself, is in the sphere of appearance,
an empty indeterminate ground;
it is then the implicit concept that receives
a difference and a determinateness only in the predicate;
the predicate thus constitutes the side of
the determinate existence of the subject.
Through this determinate universality
the subject refers to the outside,
is open to the influence of other things
and thereby confronts them actively.
What is there comes forth from its in-itselfness
into the universal element of combination and relations,
into negative references
and into the interplay of actuality
which is a continuation of the singular
into other singulars
and is, therefore, universality.

Yet the identity just indicated, the fact that
the determination of the subject accrues equally
to the predicate and vice versa,
is not just a matter for our consideration;
it is not only in itself but is also posited in the judgment;
for the judgment is the reference connecting the two;
the copula expresses that the subject is the predicate.
The subject is the determinate determinateness,
and the predicate is this determinateness
of the subject as posited;
the subject is determined only in its predicate,
or is subject only in it;
in the predicate, it is turned back into itself
and is therein the universal.
Now in so far as the subject is the self-subsistent term,
this identity has the relation that the predicate
does not possess a self-subsistence of its own
but has its subsistence only in the subject;
it inheres in the subject.
Accordingly, since the predicate is distinguished from the subject,
it is only a singularized determinateness of the subject,
only one of its properties;
the subject itself is however the concrete,
the totality of manifold determinacies,
just as the predicate contains one of them;
the subject is the universal.
But, on the other hand, the predicate also is self-subsistent universality,
and the subject conversely only one determination of it.
The predicate thus subsumes the subject;
the singularity and the particularity are not for themselves
but have their essence and their substance in the universal.
The predicate expresses the subject in its concept;
the singular and the particular are to the
subject accidental determinations;
the subject is their absolute possibility.
When by “subsumption” an external connection
of subject and predicate is thought,
and the subject is represented as something self-subsistent,
then subsumption refers to the subjective
act of judging mentioned above,
namely the judging that starts off
from the self-subsistence of both subject and predicate.
Subsumption is then only the application of
the universal to a particular or singular
posited under it in accordance
with an indeterminate representation,
one of lesser quantity.

When we treat the identity of subject and predicate
as meaning that at one time one determination of the concept
belongs to the subject and the other to the predicate,
and at another time the converse equally applies,
then the identity is as yet still implicit;
on account of the self-subsistent
diversity of the two sides of judgment,
their posited connection also has
the two at first as diverse.
But it is the identity void of difference
that in fact constitutes the true connection
of the subject and predicate.
The determination of the concept is itself
essentially a connection, for it is a universal;
the same determinations, therefore,
which the subject and the predicate each have,
are also had by their connection.
The connection is universal,
for it is the positive identity of both,
of the subject and predicate;
but it is also determinate,
for the determinateness of the predicate is
the determinateness of the subject;
it is singular as well, for in it the self-subsisting extremes are
sublated as in their negative unity.
In judgment, however, this identity is not posited yet;
the copula is as the still indeterminate connection of being in general,
“A is B,” for the self-subsistence of the concept's determinacies,
or the extremes, is in judgment the reality
that the concept has within.
If the “is” of the copula were already posited as the determinate
and fulfilled unity of subject and predicate earlier mentioned,
were posited as their concept,
it would then already be the conclusion of syllogistic inference.

To restore again this identity of the concept,
or rather to posit it,
this is the goal of the movement of the judgment.
What is already present in the judgment is,
on the one hand, the self-subsistence
but also reciprocal determinateness of
the subject and predicate,
and, on the other hand,
their still abstract connection.
“The subject is the predicate”;
this is what the judgment says at first.
But since the predicate is not
supposed to be what the subject is,
a contradiction is at hand that must resolve itself,
must pass over into a result.
Or rather, since the subject and predicate are
in and for themselves the totality of the concept,
and judgment is the reality of the concept,
the judgment's forward movement is only development;
what comes forth from it is already present in it,
and to this extent the demonstration is a display,
a reflection as the positing of that which is already
at hand in the extreme terms of the judgment;
but even this positing is already present;
it is the connection of the extremes.

First, as immediate, judgment is the judgment of existence;
its subject is immediately an abstract, existent singular,
and the predicate is an immediate determinateness or property of it,
an abstract universal.

Second, as this qualitative character of
the subject and predicate is sublated,
the determination of the one begins
to shine reflectively in the other;
the judgment is now the judgement of reflection.

But this external combination passes over
into the essential identity of a substantial, necessary combination;
and so we have, third, the judgment of necessity.

Fourth, since in this essential identity
the difference of subject and predicate has become a form,
the judgment becomes subjective;
it entails the opposition of the concept and its reality
and the comparison of the two;
it is the judgment of the concept.

This emergence of the concept grounds
the transition of judgment into syllogistic inference.

CHAPTER 3

The syllogism

The syllogism is the result of
the restoration of the concept in the judgment,
and consequently the unity and the truth of the two.
The concept as such holds its moments
sublated in this unity;
in judgment, the unity is an internal
or, what amounts to the same, an external one,
and although the moments are connected,
they are posited as self-subsisting extremes.
In the syllogism, the determinations of the concept
are like the extremes of the judgment,
and at the same time their determinate unity is posited.

Thus the syllogism is the completely posited concept;
it is, therefore, the rational.
The understanding is taken to be
the faculty of the determinate concept
which is held fixed for itself by virtue of
abstraction and the form of universality.
But in reason the determinate concepts are
posited in their totality and unity.
Therefore, it is not just that the syllogism is rational
but that everything rational is a syllogism.
Syllogistic inference has long since been ascribed to reason;
but, on the other hand, reason in and for itself,
and rational principles and laws,
are so spoken of that no light is thrown
on why the one reason that syllogizes,
and the other which is the source
of laws and otherwise eternal truths
and absolute thoughts, hang together.
If the former is supposed to be
only a formal reason
while the latter is supposed to be
the one that generates content,
then one would expect on this distinction
that precisely the form of reason,
the inference, would not be missing in the latter.
And yet, the two are commonly held so far apart,
the one without mention of the other,
that it seems as though the reason
of absolute thoughts were ashamed,
so to speak, of inferential reason,
and the syllogism were listed as also
an activity of reason merely as matter of tradition.
But surely, as we have just remarked,
logical reason must be essentially recognizable,
when regarded as formal, also in the reason
that is concerned with a content;
indeed, no content can be rational
except by virtue of the rational form.
In this matter we cannot rely on
what is commonly said about reason,
for common views fail to tell us
what we are to understand by reason;
this would-be rational wisdom is so busy with its objects
that it forgets to pay attention to reason itself
but only identifies it by characterizing it
through the objects that it is said to have.
If reason is supposed to be a cognition
that would know about God, freedom, right and duty,
the infinite, the unconditional, the suprasensible,
or even gives only representations and feelings of such objects,
then for one thing these objects are only negative,
and for another the original question still stands,
what is there in all these objects that makes them rational?
The answer is that the infinitude in them
is not the empty abstraction from the finite,
is not a universality which is void of content and determination,
but is the fulfilled universality, the concept which is determined
and is truly in possession of its determinateness,
namely, in that it differentiates itself internally
and is the unity of its thus intelligible and determined differences.
Only in this way does reason rise above
the finite, the conditioned, the sensuous,
or however one might define it,
and is in this negativity essentially replete with content,
for as unity it is the unity of determinate extremes.
And so the rational is nothing but the syllogism.

Now the syllogism, like judgment, is at first immediate;
as such, its determinations (termini) are
simple, abstract determinacies;
it is then the syllogism of the understanding.
If one stays at this configuration of the syllogism,
then its rationality, though present there and posited,
is not apparent.
The essential element of the syllogism is
the unity of the extremes,
the middle term that unites them
and the ground that supports them.
Abstraction, by holding fast to
the self-subsistence of the extremes,
posits this unity opposite them,
as a determinateness with just as fixed
an existence of its own,
thus grasping it more as a non-unity than as a unity.
The expression, “middle term” (medius terminus),
is derived from spatial representation,
and has its share of responsibility for
why we stop short at the externality of the terms.
Now if the syllogism consists in
the positing in it of the unity of the extremes,
but if this unity is simply taken
on the one hand as a particular by itself,
and on the other hand as only an external connection,
and non-unity is made the essential relation of syllogism,
then the reason of the syllogism is of no help to rationality.

First, the syllogism of existence,
in which the terms are thus
immediately and abstractly determined,
demonstrates internally that,
since like judgment it is
the connection of those terms,
these are not in fact abstract
but each contains in it the reference
connecting it to the others,
and the determination of the middle term is
not just a determinateness opposed to the
determinations of the extremes
but contains these extremes posited in it.

Through this dialectic,
the syllogism of existence becomes
the syllogism of reflection, the second syllogism.
Its terms are such that in each the other
shines essentially reflected in it,
or are posited as mediated,
as they are indeed supposed to be
in accordance with the nature of
syllogistic inference in general.

Third, inasmuch as this reflective shining
or this mediatedness is reflected into itself,
syllogism is determined as the syllogism of necessity,
one in which the mediating factor is
the objective nature of the fact.
As this syllogism determines the extremes
of the concept also as totalities,
it has attained the correspondence
of its concept (or the middle term)
and its existence (or the difference of the extremes).
It has attained its truth;
and with that it has stepped forth
out of subjectivity into objectivity.
