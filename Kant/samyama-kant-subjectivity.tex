a. Of concepts

The concept in general and its distinction from intuition

    All cognitions, that is,
    all representations related with consciousness to an object,
    are either intuitions or concepts.
    An intuition is a singular representation (repraesentatio singularis),
    a concept a universal (repraesentatio per notas communes)
    or reflected representation (repraesentatio discursiva).
    Cognition through concepts is called thought (cognitio discursiva).

Matter and form of concepts

    With every concept we are to distinguish matter and form.
    The matter of concepts is the object,
    their form universality.

Empirical and pure concept

    A concept is either an empirical or a pure concept
    (vel empiricus vel intellectualis).
    A pure concept is one that is not abstracted from experience
    but arises rather from the understanding even as to content.
    An idea is a concept of reason whose object
    simply cannot be met with in experience.

Concepts that are given (a priori or a posteriori)
and concepts that are made

    All concepts, as to matter, are either
    given (conceptus dati) or made (conceptus factitii).
    The former are given either a priori or a posteriori.
    All concepts that are given empirically or a posteriori
    are called concepts of experience,
    all that are given a priori are called notions.

Logical origin of concepts

    The origin of concepts as to mere form rests on reflection
    and on abstraction from the difference among things
    that are signified by a certain representation.
    And thus arises here the question:
    Which acts of the understanding constitute a concept
    or what is the same,
    Which are involved in the generation of a concept
    out of given representations?

Logical Actus of comparison, reflection, and abstraction

    The logical actus of the understanding,
    through which concepts are generated as to their form, are:
    1.  comparison of representations among one another
        in relation to the unity of consciousness;
    2.  reflection as to how various representations can
        be conceived in one consciousness; and finally
    3.  abstraction of everything else
        in which the given representations differ.

b. Of judgments

Definition of a judgment in general

    A judgment is the representation of the unity of
    the consciousness of various representations,
    or the representation of their relation
    insofar as they constitute a concept.

Matter and form of judgments

    Matter and form belong to every judgment
    as essential constituents of it.
    The matter of the judgment consists in
    the given representations that are combined
    in the unity of consciousness in the judgment,
    the form in the determination of the way that
    the various representations belong, as such,
    to one consciousness.

Object of logical reflection the mere form of judgments

    Since logic abstracts from all real or objective
    difference of cognition,
    it can occupy itself as little with the matter of judgments
    as with the content of concepts.
    Thus it has only the difference among judgments
    in regard to their mere form to take into consideration.

c. Of inferences

Inference in general

    By inferring is to be understood that function of thought
    whereby one judgment is derived from another.
    An inference is thus in general
    the derivation of one judgment from the other.

Immediate and mediate inferences

    All inferences are either immediate or mediate.
    An immediate inference (consequentia immediata) is
    the derivation (deductio) of one judgment from the other
    without a mediating judgment (judicium intermedium).
    An inference is mediate if, besides the concept
    that a judgment contains in itself,
    one needs still others in order to
    derive a cognition therefrom.

Inferences of the understanding, inferences of reason,
and inferences of the power of judgment

    Immediate inferences are also called
    inferences of the understanding;
    all mediate inferences, on the other hand,
    are either inferences of reason
    or inferences of the power of judgment.
