A) Logical perfection of cognition as to quantity

Quantity

    Extensive and intensive quantity
    Extensiveness and thoroughness
    or importance and fruitfulness of cognition
    Determination of the horizon of our cognition

The quantity of cognition can be understood in two senses,
either as extensive or as intensive quantity.
The former relates to the extension of cognition and
thus consists in its multitude and manifoldness;
the latter relates to its content, which concerns the richness
or the logical importance and fruitfulness of a cognition,
insofar as it is considered as ground of
many and great consequences (multa sed multum.)

In expanding our cognitions or in perfecting them
as to their extensive quantity
it is good to make an estimate as to how far
a cognition agrees with our ends and capabilities.
This reflection concerns the determination of
the horizon of our cognitions,
by which is to be understood
the congruence of the quantity of all cognitions
with the capabilities and ends of the subject.

The horizon can be determined

1. logically, in accordance with the faculty or the powers of
cognition in relation to the interest of the understanding.

Here we have to pass judgment on how far we can go in our cognitions,
how far we must go, and to what extent certain cognitions serve,
in a logical respect, as means to various principal cognitions as our ends;

2. aesthetically, in accordance with taste
in relation to the interest of feeling.
He who determines his horizon aesthetically seeks
to arrange science according to the taste of the public,
i.e., to make it popular,
or in general to attain only such cognitions
as may be universally communicated,
and in which the class of the unlearned, too,
find pleasure and interest;

3. practically, in accordance with use
in relation to the interest of the will.
The practical horizon, insofar as it is determined
according to the influence which a cognition has
on our morality is pragmatic
and is of the greatest importance.

Thus the horizon concerns passing judgment on,
and determining,
what man can know,
what he is permitted to know,
and what he ought to know.

Now as for what concerns the theoretically
or logically determined horizon in particular
(and it is of this alone that we can speak here)
we can consider it either from the objective
or from the subjective viewpoint.

In regard to objects, the horizon is either historical or rational.
The former is much broader than the other,
indeed, it is immeasurably great,
for our historical cognition has no limits.
The rational horizon, on the other hand, may be fixed,
e.g., it may be determined to what kind of objects
mathematical cognition cannot be extended.
So too in respect of philosophical cognition of reason,
as to how far reason can go here a priori without any experience.

In relation to the subject the horizon is
either the universal and absolute,
or a particular and conditioned one (a private horizon).

By the absolute and universal horizon is to be understood
the congruence of the limits of human cognitions with
the limits of the whole of human perfection in general.
And here, then, the question is:
In general, what can man, as man, know?

The determination of the private horizon depends upon
various empirical conditions and special considerations,
e.g., age, sex, station, mode of life, etc.
Every particular class of men has its particular horizon
in relation to its special powers of cognition, ends, and standpoints,
every mind its own horizon according to the standard of
the individuality of its powers and its standpoint.
Finally, we can also think a horizon of healthy reason
and a horizon of science,
which latter still requires principles,
in accordance with which to determine
what we can and cannot know.

What we cannot know is beyond our horizon,
what we do not need to know is outside our horizon.
 This latter can hold only relatively, however,
in relation to various particular private ends, to whose accomplishment
certain cognitions not only do not contribute anything but could even be
an obstacle. For no cognition is, absolutely and for every purpose, useless
and unusable, although we may not always be able to have insight into its
use. Hence it is an objection as unwise as it is unjust that is made to great
men who labor in the sciences with painstaking industry when shallow
minds ask, What is the use ofthat?' We must simply never raise this ques-
tion if we want to occupy ourselves with the sciences. Even granted that a
science could give results only concerning some possible object, it would
still for that reason alone be useful enough. Every logically perfect cogni-
tion always has some possible use, which, although we are as yet unac-
quainted with it, will perhaps be found by posterity. If in the cultivation of
the sciences one had always looked only to material gain, their use, then
we would have no arithmetic or geometry. Besides, our understanding is
so arranged that it finds satisfaction in mere insight, even more than in the
use that arises therefrom. Plato noted this. Man feels in this his own
excellence, he senses what it means to have understanding. Men who do
not sense this must envy the animals. The inner worth that cognitions have
through logical perfection is not to be compared with the outer, their worth
in application.

Like that which lies outside our horizon, insofar as we, in accordance
with our purposes, do not need to know it, as dispensable for us, that which
lies beneath' our horizon, insofar as we ought not to know it as harmful to
us, is to be understood in a relative sense but never in an absolute one.

With respect to the extension and the demarcation of our cognition,
the following rules are to be recommended:

1. One must determine his horizon early,
but of course only when one can determine it oneself,
which usually does not occur before the 2oth year;

2. not alter it lightly or often
(not turn from one thing to another);

3. not measure the horizon of others by one's own,
and not consider as useless what is of no use to us;
it would be presumptuous to want to determine others' horizons,
because one is not sufficiently acquainted,
in part with their capabilities,
in part with their purposes;

4. neither extend it too far nor restrict it too much.
For he who wants to know too much ends by knowing nothing,
and conversely, he who believes of some things
that they do not concern him, often deceives himself;
as when, e.g., the philosopher believes of history
that it is dispensable for him.

One should also seek

5. to determine in advance the absolute horizon of
the whole human race (as to past and to future time),
as well as also

6. to determine, in particular, the position that
our science occupies in the whole of cognition.
The Universal Encyclopedia serves for this as
a universal map (mappe-monde) of the sciences.

7. In determining his own particular horizon
one should carefully consider for which part of cognition
one has the greatest capability and pleasure,
what is more or less necessary in regard to certain duties,
what cannot coexist with the necessary duties;
and finally

8. one should of course always seek
to expand his horizon rather than to narrow it.

Opposed to the logical perfection of cognition
in regard to its extension stands ignorance.
A negative imperfection, or imperfection of lack,
which, on account of the restrictions of the understanding,
is inseparable from our cognition.

We can consider ignorance
from an objective or from a subjective viewpoint.

1. Taken objectively, ignorance is either material or formal.
The former consists in a lack of historical cognitions,
the other in a lack of rational cognitions.
One does not have to be completely ignorant in any field,
but one can well restrict historical knowledge in order to
devote oneself more to rational knowledge, or conversely.

2. In the subjective sense, ignorance is either
learned, scientific, or is common.
He who has distinct insight into the restrictions of cognition,
hence into the field of ignorance from where it begins,
e.g., the philosopher who sees and proves
how little one can know of gold
in regard to its structure
due to a lack of the requisite data,
is ignorant artfully or in a learned way.
He who is ignorant, on the other hand,
without having insight into
the grounds of the limits of knowledge,
and without concerning himself with this,
is so in a common, not a scientific way.
Such a one does not even know that he knows nothing.
For one can never represent his ignorance
except through science,
as a blind man cannot represent darkness
until he has become sighted.

Cognition of one's ignorance presupposes science, then,
and makes one at the same time modest,
while imagined knowledge puffs one up.
Hence Socrates' non-knowledge was a laudable ignorance,
a knowledge of non-knowledge, according to his own admission.
It is precisely those who possess very many cognitions then,
and who for all that are astounded at
the multitude of what they do not know,
who cannot be reproached with their ignorance.

Ignorance in things whose cognition lies beyond
our horizon is in general irreproachable (inculpabilis),
and in regard to the speculative use of our faculty of cognition
it can be allowed (although only in the relative sense),
insofar as the objects here lie
not beyond our horizon but yet outside it.
It is shameful, however, in things that
it is quite necessary and also easy to know.

There is a distinction between
not knowing something
and ignoring something,
i.e., taking no notice of it.
It is good to ignore much that
it is not good for us to know.
Abstracting is distinct from both of these.
One abstracts from a cognition
when one ignores its application,
whereby one gets it in abstracto
and can better consider it in
the universal as a principle.
Such abstraction from what does not belong
to our purpose in the cognition of a thing
is useful and praiseworthy.

This perfection of cognition,
whereby it qualifies for
easy and universal communication,
could also be called external extension
or the extensive quantity of a cognition,
insofar as it is widespread externally among men.

Since cognitions are so many and manifold,
one will do well to make himself a plan,
in accordance with which he orders the sciences
in the way that best agrees with,
and contributes to the furtherance of, his ends.
All cognitions stand in a certain
natural connection with one another.
Now if, in striving to expand his cognitions,
one does not look to their connection,
then extensive knowledge amounts to
nothing more than a mere rhapsody.
If one makes one principal science his end, however,
and considers all other cognitions only
as means for achieving it,
then he brings a certain systematic character into his knowledge.
And in order to go to work on extending his cognitions
according to such a well ordered and purposive plan,
one must seek, therefore, to become acquainted with
this connection of cognitions among themselves.
For this, the sciences get guidance from architectonic,
which is a system in accordance with ideas,
in which the sciences are considered
in regard to their kinship
and systematic connection in
a whole of cognition that interests humanity.

Now as for what concerns the intensive quantity of cognition;
its content, or its richness and importance,
which is essentially distinct from its extensive quantity,
its mere extensiveness, as we noted above,
we want here to add only the following few remarks:

1. A cognition that is concerned with what is great,
with the whole in the use of the understanding,
is to be distinguished from subtlety in what is small (micrology).

2. Every cognition that furthers logical perfection
as to form is to be called logically important;
e.g., every mathematical proposition,
every law of nature into which we have distinct insight,
every correct philosophical explanation.
Practical importance cannot be foreseen,
one must simply wait and watch for it.

3. Importance must not be confused with difficulty.
A cognition can be difficult without being important, and conversely.
Difficulty, then, does not decide either for or against
the worth or the importance of a cognition.
This rests on the quantity or multiplicity of its consequences.
A cognition is the more important accordingly as
it has more or greater consequences,
as the use that may be made of it is more.
Cognition without important consequences is called cavilling;
scholastic philosophy, e.g., was of this sort.

B) Logical perfection of cognition as to relation

Truth

    Material and formal, or logical, truth;
    Criteria of logical truth
    Falsehood and error
    Illusion, as source of error;
    Means for avoiding errors

A principal perfection of cognition, indeed,
the essential and inseparable condition
of all its perfection, is truth.
Truth, it is said, consists in
the agreement of cognition with its object.

In consequence of this mere nominal explanation, my cognition,
to count as true, is supposed to agree with its object.
Now I can compare the object with my cognition, however,
only by cognizing it.
Hence my cognition is supposed to confirm itself,
which is far short of being sufficient for truth.
For since the object is outside me, the cognition in me,
all I can ever pass judgment on is
whether my cognition of the object
agrees with my cognition of the object.
The ancients called such a circle in explanation a diallelon.
And actually the logicians were always reproached
with this mistake by the skeptics,
who observed that with this explanation of truth
it is just as when someone makes a statement before a court
and in doing so appeals to a witness with whom no one is acquainted,
but who wants to establish his credibility by maintaining
that the one who called him as witness is an honest man.
The accusation was grounded, too.
Only the solution of the indicated problem is
impossible without qualification and for every man.

The question here is, namely, whether and to what extent
there is a criterion of truth that is
certain, universal, and useful in application.
For this is what the question, What is truth?, ought to mean.

To be able to decide this important question
we must distinguish that which belongs to
the matter in our cognition
and is related to the object
from that which concerns its mere form,
as that condition without which a cognition
would in general never be a cognition.
With respect to this distinction
between the objective, material relation in our cognition
and the subjective, formal relation,
the question above thus breaks down into
these two particular ones:

Is there a universal material, and
Is there a universal formal criterion of truth?

A universal material criterion of truth is not possible;
it is even self-contradictory.
For as a universal criterion,
valid for all objects in general,
it would have to abstract fully
from all difference among objects,
and yet at the same time,
as a material criterion,
it would have to deal with just this difference,
in order to be able to determine whether a cognition agrees
with just that object to which it is related
and not just with any object in general,
in which case nothing would really be said.
Material truth must consist in
this agreement of a cognition
with just that determinate object
to which it is related, however.
For a cognition that is true in regard to one object
can be false in relation to other objects.
Hence it is absurd to demand
a universal material criterion of truth,
which should abstract and at the same time not abstract
from all difference among objects.

If the question is about
universal formal criteria of truth, however,
then here it is easy to decide that
of course there can be such a thing.
For formal truth consists merely in
the agreement of cognition with itself,
in complete abstraction from all objects whatsoever
and from all difference among them.
And the universal formal criteria of truth are
accordingly nothing other than universal logical marks of
the agreement of cognition with itself
or, what is one and the same,
with the universal laws of the understanding and of reason.
These formal, universal criteria are of course
not sufficient for objective truth,
but they are nonetheless to be regarded
as its conditio sine qua non.
For the question of whether
cognition agrees with its objects
must be preceded by the question of whether
it agrees with itself (as to form).
And this is a matter for logic.

The formal criteria of truth in logic are

1. the principle of contradiction,
2. the principle of sufficient reason.

Through the former the logical possibility of a cognition is determined,
through the latter its logical actuality.
To the logical actuality of a cognition it pertains, namely:

First: that it be logically possible, not contradict itself.

    This characteristic of internal logical truth is only negative, however;
    for a cognition that contradicts itself is of course false,
    but if it does not contradict itself it is not always true.

Second: that it be logically grounded,

that it (a) have grounds and (b) not have false consequences.

    This second criterion of external logical truth or
    of accessibility to reason, which concerns the logical connection of
    a cognition with grounds and consequences, is positive.

And here the following rules are valid:

1. From the truth of the consequence we may infer
the truth of the cognition as ground, but only negatively:

    if one false consequence flows from a cognition,
    then the cognition itself is false.
    For if the ground were true,
    then the consequence would also have to be true,
    because the consequence is determined by the ground.
    But one cannot infer conversely that
    if no false consequence flows from a cognition, then it is true;
    for one can infer true consequences from a false ground.

2. If all the consequences of a cognition are true,
then the cognition is true too.

    For if there were something false in the cognition,
    then there would have to be a false consequence too.
    From the consequence, then, we may infer to a ground,
    but without being able to determine this ground.
    Only from the complex of all consequences can one
    infer to a determinate ground,
    infer that it is the true ground.

The former mode of inference, according to which the consequence
can only be a negatively and indirectly sufficient criterion of the truth of a
cognition, is called in logic the apagogic mode (modus tollens).

    This procedure, of which frequent use is made in geometry,
    has the advantage that I may derive just one false consequence
    from a cognition in order to prove its falsehood.
    To show, e.g., that the earth is not flat, I may
    just infer apagogically and indirectly,
    without bringing forth positive and direct grounds:
    If the earth were flat, then the pole star would
    always have to be at the same height;
    but this is not the case, consequently it is not flat.

With the other, the positive and direct mode of inference (modus ponens)
the difficulty enters that the totality of the consequences cannot be
cognized apodeictically, and that one is therefore led by the indicated
mode of inference only to a probable and hypothetically true cognition
(hypotheses), in accordance witii the presupposition that where many
consequences are true, all the remaining ones' may be true too.

Thus we will be able to advance three principles here as universal,
merely formal or logical criteria of truth; these are

the principle of contradiction and of identity

(principium contradictionis and identitatis),
through which the internal possibility of a cognition
is determined for problematic judgments;

the principle of sufficient reason

(principium rationis suffidentis),
on which rests the (logical) actuality of a cognition,
the fact that it is grounded, as material for assertoric judgments;

the principle of the excluded middle

(principium exclusi medii inter duo contradictoria),
on which the (logical) necessity of a cognition is grounded,
that we must necessarily judge thus and not otherwise,
that the opposite is false, for apodeictic judgments.

The opposite of truth is falsehood, which,
insofar as it is taken for truth, is called error.
An erroneous judgment, for there is error as well as truth only in judgment,
is thus one that confuses the illusion of truth with truth itself.
It is easy to have insight into how truth is possible,
since here the understanding acts in accordance with its essential laws.
But it is hard to comprehend how error in the formal sense of the word,
how the form of thought contrary to the understanding is possible,
just as we cannot in general comprehend how any power
should deviate from its own essential laws.
We cannot seek the ground of errors in
the understanding itself and its essential laws, then,
just as little as we can in the restrictions of the understanding,
in which lies the cause of ignorance, to be sure,
but not in any way the cause of error.
Now if we had no other power of cognition
but the understanding, we would never err.
But besides the understanding, there lies in us
another indispensable source of cognition.
That is sensibility, which gives us the material for thought, and
in doing this works according to other laws than those the understanding does.
Error cannot arise from sensibility in and by itself, however,
because the senses simply do not judge.

The ground for the origin of all error will therefore have to be sought
simply and solely in the unnoticed influence of sensibility upon
the understanding, or to speak more exactly, upon judgment.
This influence, namely, brings it about that in judgment
we take merely subjective grounds to be objective, and
consequently confuse the mere illusion of truth with truth itself.
For it is just in this that the essence of illusion consists,
which on this account is to be regarded as a ground
for holding a false cognition to be true.

What makes error possible, then, is illusion,
in accordance with which the merely subjective is
confused in judgment with the objective.
In a certain sense, however, one can make the understanding the author
of errors, namely, insofar as it allows itself,
due to a lack of requisite attention to that influence of sensibility,
to be misled by the illusion arising therefrom into holding
merely subjective determining grounds of judgment to be objective ones,
or into letting that which is true only according to
the laws of sensibility hold as true in accordance with its own laws.
In the restrictions of the understanding, then,
lies only the responsibility for ignorance;
the responsibility for error we have to assign to ourselves.
Nature has denied us many cognitions, to be sure,
it leaves us in unavoidable ignorance concerning so much,
but still it does not cause error.
We are misled into this by our own inclination to judge and
to decide even where, on account of our limitedness,
we are not able to judge and to decide.

Every error into which
the human understanding can fall is only partial, however,
and in every erroneous judgment
there must always lie something true.
For a total error would be a complete opposition to
the laws of the understanding and of reason.
But how could that, as such, in any way come from the understanding
and, insofar as it is still a judgment,
be held to be a product of the understanding.

In respect to the true and the erroneous in our cognition,
we distinguish an exact cognition from a rough one.

    Cognition is exact when it is adequate to its object, or
    when there is not the slightest error in regard to its object,
    and it is rough when there can be errors in it yet
    without being a hindrance to its purpose.
    This distinction concerns the broader or narrower determinateness
    of our cognition (cognitio late vel stride determinata).
    Initially it is sometimes necessary to determine a cognition
    in a broader extension (late determinare),
    particularly in historical things.
    In cognitions of reason everything must be
    determined exactly (striae), however.
    In the case of broad determination
    one says that a cognition is determined praeter propter.
    Whether a cognition ought to be determined roughly or exactly
    always depends on its purpose.
    Broad determination leaves a certain play for error,
    which still can have its determinate limits, however.
    Error occurs particularly where a broad determination
    is  taken for a strict one,
    e.g., in matters of morality,
    where everything must be determined striae.
    Those who do not do this are called by the English latitudinarians.
    One can distinguish subtlety, as a subjective perfection of cognition,
    from exactness, as an objective perfection,
    since here cognition is fully congruent with its object.
    A cognition is subtle when one discovers in it
    that which usually escape the attention of others.
    It requires a higher degree of attention, then,
    and a greater application of power of the understanding.
    Many reprove all subtlety because they cannot attain it.
    But in itself it always brings honor to the understanding
    and is even laudable and necessary,
    insofar as it is applied to an object worthy of observation.
    When one could have attained the same end
    with less attention and effort of the understanding, however,
    and yet one uses more, then one makes a useless expense
    and falls into subtleties, which are difficult, to be sure,
    but do not have any use (nugae difficiles).
    As the rough is opposed to the exact,
     so is the crude to the subtle.

From the nature of error whose concept, as we noted,
contains as an essential mark, besides falsehood,
also the illusion of truth
we get the following important rule
for the truth of our cognition:

    To avoid errors and no error is unavoidable,
    at least not absolutely or without qualification,
    although it can be unavoidable relatively,
    for the cases where it is unavoidable for us to judge,
    even with the danger of error,
    to avoid errors, then, one must seek to
    disclose and to explain their source, illusion.
    Very few philosophers have done that, however.
    They have only sought to refute the errors themselves,
    without indicating the illusion from which they arise.
    This disclosure and breaking up of illusion is
    a far greater service to truth, however, than the direct refutation
    of errors, whereby one does not block their source and cannot guard
    against the same illusion misleading one into errors again in other cases
    because one is not acquainted with it.
    For even if we are convinced that we have erred,
    then in case the illusion that grounds our error
    has not been removed we still have scruples,
    however little we can bring forth in justification of them.

Error in principles is greater than in their application.

An external mark or an external touchstone of truth is
the comparison of our own judgments with those of others,
because the subjective will not be present in all others in the same way,
so that illusion can thereby be cleared up.
The incompatibility of the judgments of others with our own is
thus an external mark of error and is to be regarded as a cue to investigate
our procedure in judgment, but not for that reason to reject it at once;
one can perhaps be right about the thing but not right in manner,
in the exposition.

The common human understanding (sensus communis) is also in itself
a touchstone for discovering the mistakes of the artificial use
of the understanding.
This is what it means to orient oneself in thought or
speculative use of reason by means of the common understanding,
one uses the common understanding as a test for passing judgment
on the correctness of the speculative use.

Universal rules and conditions for avoiding error in general are:

1) to think for oneself,
2) to think oneself in the position of someone else, and
3) always to think in agreement with oneself.

The maxim of thinking for oneself can be called
the enlightened mode of thought;
the maxim of putting oneself in the viewpoint of others in thought,
the extended mode of thought;
and the maxim of always thinking in agreement with one self,
the consequent or coherent mode of thought.

C) Logical perfection of cognition as to quality

Clarity

    Concept of a mark in general
    Various kinds of marks
    Determination of the logical essence of a thing
    Its distinction from the real essence
    Distinctness, a higher degree of clarity
    Aesthetic and logical distintness
    Distinction between analytic and synthetic distinctness

From the side of the understanding,
human cognition is discursive;
it takes place through representations
which take as the ground of cognition
that which is common to many things,
hence through marks' as such.
Thus we cognize things through marks
and that is called cognizing,
[the German word for which] comes from
[the German word for] being acquainted.

A mark is that in a thing which constitutes
a part of the cognition of it, or
what is the same - a partial representation,
insofar as it is considered as
ground of cognition of the whole representation.
All our concepts are marks, accordingly, and
all thought is nothing other than a representing through marks.
Every mark may be considered from two sides:
First, as a representation in itself;
Second, as belonging, as a partial concept, to
the whole representation of a thing, and thereby
as ground of cognition of this thing itself.
All marks, considered as grounds of cognition, have two uses,
internal or an external use.
The internal use consists in derivation,
in order to cognize the thing itself
through marks as its grounds of cognition.
The external use consists in comparison,
insofar as we can compare one thing with others
through marks in accordance with the rules of
identity or diversity.

There are many specific differences among marks,
on which the following classification of them is grounded.

1. Analytic or synthetic marks.

The former are partial concepts of my actual concept
(marks that I already think therein),
while the latter are partial concepts of
the merely possible complete concept
(which is supposed to come to be
through a synthesis of several parts).

The former are all concepts of reason,
the latter can be concepts of experience.

2. Coordinate or subordinate.

This division of marks concerns
their connection after or under one another.

Marks are coordinate insofar as each of them is
represented as an immediate mark of the thing
and are subordinate insofar as one mark is
represented in the thing only by means of the other.
The combination of coordinate marks to form
the whole of a concept is called an aggregate,
the combination of subordinate concepts a series.
The former, the aggregation of coordinate marks,
constitutes the totality of the concept,
regard to synthetic empirical concepts, can never be completed,
but rather resembles a straight line without limits.
The series of subordinate marks terminates a pane ante,
or on the side of the grounds, in concepts which cannot be broken up,
which cannot be further analyzed on account of their simplicity;
a pane post, or in regard to the consequences, it is infinite,
because we have a highest genus but no lowest species.
With the synthesis of every new concept in the aggregation of coordinate marks,
the extensive or extended distinctness grows,
as intensive or deep distinctness grows with the further
analysis of the concept in the series of subordinate marks.
This latter kind of distinctness, since it necessarily
contributes to thoroughness and coherence of the cognition,
is thus principally a matter of philosophy and is pursued
to the highest degree in metaphysical investigations in particular.

3. Affirmative or negative marks.

Through the former we cognize what the thing is,
through the latter what it is not.

Negative marks serve to keep us from errors.
Hence they are unnecessary where it is impossible to err,
and are necessary and of importance only in
those cases where they keep us from an important error
into which we can easily fall.
Thus in regard to the concept, e.g., of a being like God,
negative marks are quite necessary and important.
Through affirmative marks we seek to understand something,
through negative marks, into which all marks can be transformed,
we only seek not to misunderstand or not to err,
even if we should not thereby become acquainted with anything.

4. Important and fruitful, or empty and unimportant, marks.

A mark is important and fruitful
if it is a ground of cognition for great
and numerous consequences,
partly in regard to its internal use,
its use in derivation, insofar as
it is sufficient for cognizing thereby
a great deal in the thing itself,
partly in respect to its external use,
its use in comparison,
insofar as it thereby contributes to cognizing
both the similarity of a thing to many others
and its difference from many others.
We have to distinguish logical importance and fruitfulness
from practical, from usefulness and utility, by the way.

5. Sufficient and necessary or insufficient and accidental marks.

A mark is sufficient insofar as it suffices always
to distinguish the thing from all others;
otherwise it is insufficient, as the mark of barking is,
for example, for dogs.
The sufficiency of marks, as well as their importance,
is to be determined only in a relative sense,
in relation to ends that are intended through a cognition.
Necessary marks, finally, are those that must always be there to be found
in the thing represented.
Marks of this sort are also called essential and are
opposed to extra-essential and accidental marks,
which can be separated from the concept of the thing.
Among necessary marks there is another distinction, however.
Some of them belong to the thing as grounds of other marks of one and
the same thing, while others belong only as consequences of other marks.
The former are primitive and constitutive marks
(constitutiva, essentialia in sensu strictissimo),
the others are called attributes (consectaria, rationata)
and belong admittedly to the essence of the thing,
but only insofar as they must first be derived from its essential points,
as the three angles follow from the three sides
in the concept of the triangle, for example.
Extra-essential marks are again of two kinds;
they concern either internal determinations of a thing (moat) or
its external relations (relationes).
Thus the mark of learnedness signifies an inner determination of a man,
but being a master or a servant only an external relation.

The complex of all the essential parts of a thing, or the sufficiency of its
marks as to coordination or subordination, is the essence
(complexus notarum primitivarum, interne conceptui dato sufficientium;
s. complexus notarum, conceptum aliquem primitive constituentium).

In this explanation, however, we must not think at all of
the real or natural essence of things,
into which we are never able to have insight.
For since logic abstracts from all content of cognition, and
consequently also from the thing itself, in this science
the talk can only be of the logical essence of things.
And into this we can easily have insight.
For it includes nothing further than the cognition of
all the predicates in regard to which
an object is determined through its concept;
whereas for the real essence of the thing (esse ret)
we require cognition of those predicates on which,
as grounds of cognition, everything that belongs
to the existence of the thing depends.
If we wish to determine, e.g., the logical essence of body, then
we do not necessarily have to seek for the data for this in nature;
we may direct our reflection to the marks which, as essential points
(constitutiva, rationes) originally constitute the basic concept of the thing.
For the logical essence is nothing but the first basic concept of
all the necessary marks of a thing (esse conceptus).

The first stage of the perfection of our cognition
as to quality is thus its clarity.
A second stage, or a higher degree of clarity, is distinctness.
This consists in clarity of marks.

First of all we must here distinguish
logical distinctness in general from aesthetic distinctness.
Logical distinctness rests on objective clarity of marks,
aesthetic distinctness on subjective clarity.
The former is a clarity through concepts,
the latter a clarity through intuition.
The latter kind of distinctness consists, then,
in a mere liveliness and understandability,
in a mere clarity through examples in concreto
(for much that is not distinct
can still be understandable,
and conversely, much that is hard to understand
can still be distinct,
because it goes back to remote marks,
whose connection with intuition
is possible only through a long series).
Objective distinctness frequently causes
subjective obscurity, and conversely.
Hence logical distinctness is often possible
only to the detriment of aesthetic distinctness,
and conversely aesthetic distinctness
through examples and similarities
which do not fit exactly but are only
taken according to an analogy often becomes harmful to logical distinctness.
Besides, examples are simply not marks and do not belong to the
concept as parts but, as intuitions, to the use of the concept.
Distinctness through examples, mere understandability, is hence
of a completely different kind than distinctness through concepts as marks.
Lucidity consists in the combination of both,
of aesthetic or popular distinctness and
of scholastic or logical distinctness.
For one thinks of a lucid mind as the talent for a
luminous presentation of abstract and thorough cognitions
that is congruent with the common understanding's power of comprehension.
Next, as for what concerns logical distinctness in particular,
it is to be called complete distinctness insofar as all the marks
which, taken together, make up the whole concept have come to clarity.
A completely distinct concept can be so, again,
either in regard to the totality of its coordinate marks or
in respect to the totality of its subordinate marks.
Extensively complete or sufficient distinctness of a concept
consists in the total clarity of its coordinate marks,
which is also called exhaustiveness.
Total clarity of subordinate marks constitutes
intensively complete distinctness, profundity.

The former kind of logical distinctness can also be called the external
completeness (completudo externd) of the clarity of marks, the other the inter-
nal completeness (completudo internet). The latter can be attained only with
pure concepts of reason and with arbitrary concepts, but not with empiri-
cal concepts.

The extensive quantity of distinctness,
insofar as it is not superfluous is called precision.
Exhaustiveness (completudo) and precision (praecisio)
together constitute adequacy (cognitio, quae rem adaequat);
and the completed perfection of a cognition
(consummata cognitionis perfectio) consists (as to quality)
in intensively adequate cognition,
profundity, combined with extensively adequate cognition,
exhaustiveness and precision.

Since, as we have noted, it is the business of logic to make clear concepts
distinct, the question now is in what way it makes them distinct.
Logicians of the Wolffian school place the act of making cognitions
distinct' entirely in mere analysis of them. But not all distinctness rests on
analysis of a given concept. It arises thereby only in regard to those marks
that we already thought in the concept, but not in respect to those marks
that are first added to the concept as parts of the whole possible concept.
The kind of distinctness that arises not through analysis but through
synthesis of marks is synthetic distinctness. And thus there is an essential
difference between the two propositions: to make a distinct concept and to
make a concept distinct.
For when I make a distinct concept, I begin with the parts and proceed
from these toward the whole. Here there are no marks as yet at hand; I
acquire them only through synthesis. From this synthetic procedure
emerges synthetic distinctness, then, which actually extends my concept
as to content through what is added as a mark beyond' the concept in (pure
or empirical) intuition. The mathematician and the natural philosopher
make use of this synthetic procedure in making distinctness in concepts.*
For all distinctness of properly mathematical cognition, as of all cognition
based on experience, rests on such an expansion of it through the synthe-
sis of marks.
When I make a concept distinct, however, my cognition does not grow
at all as to content through this mere analysis. The content remains the
same, only the form is altered, in that I learn to distinguish better, or to
cognize with clearer consciousness, what lay in the given concept already.
As nothing is added to a map through the mere illumination' of it, so a
given concept is not in the least increased through its mere illumination
by means of the analysis of its marks.
To synthesis pertains the making distinct of objects, to analysis the
making distinct of concepts. In the latter case the whole precedes the parts,
in the former the parts precede the whole. The philosopher only makes given
concepts distinct.
Sometimes one proceeds synthetically even when the
concept that one wants to make distinct in this way is already given.
This is often the case with propositions based on experience,
in case one is not yet satisfied with
the marks already thought in a given concept.
The analytic procedure for creating distinctness, with which alone logic
can occupy itself, is the first and principal requirement in making our
cognition distinct.
For the more distinct our cognition of a thing is, the
stronger and more effective it can be too.
But analysis must not go so far
that in the end the object itself disappears.
If we were conscious of all that we know, we would have to be aston-
ished at the great multitude of our cognitions.

In regard to the objective content of our cognition in general,
we may think the following degrees, in accordance with which
cognition can, in this respect, be graded:

The first degree of cognition is: to represent something;

The second: to represent something with consciousness,
or to perceive (percipere);

The third: to be acquainted with something (noscere), or
to represent something in comparison with other things,
both as to sameness and as to difference;

The fourth: to be acquainted with something with consciousness,
to cognize it (cognoscere).

Animals are acquainted with objects too,
but they do not cognize them.

The fifth: to understand something (intelligere),
to cognize something through the understanding
by means of concepts,
or to conceive.

One can conceive much, although one cannot comprehend it;
e.g., a perpetuum mobile, whose impossibility is shown in mechanics.

The sixth: to cognize something through reason, or
to have insight into it (perspicere).

With few things do we get this far,
and our cognitions become fewer and fewer in number
the more that we seek to perfect them as to content.

The seventh, finally: to comprehend something (comprehendere),
to cognize something through reason or a priori
to the degree that is sufficient for our purpose.

For all our comprehension is only relative,
sufficient for a certain purpose;
we do not comprehend anything without qualification.
Nothing can be comprehended more than
what the mathematician demonstrates,
that all lines in the circle are proportional.
And yet he does not comprehend how it happens
that such a simple figure has these properties.
The field of understanding or of the understanding is
thus in general much greater than
the field of comprehension or of reason.

D) Logical perfection of cognition as to modality

Certainty

    Concept of holding-to-be-true in general
    Modi of holding-to-be-true: opining, believing and knowing
    Conviction and persuasion

Truth is an objective property of cognition;
the judgment through which something is represented as true,
the relation to an understanding, and thus to a particular subject, is,
subjectively, holding-to-be-true.

Holding-to-be-true is in general of two kinds, certain or uncertain.
Certain holding-to-be-true, or certainty, is
combined with consciousness of necessity,
while uncertain holding-to-be-true, or uncertainty, is
combined with consciousness of the contingency
or the possibility of the opposite.
The latter is again either subjectively
as well as objectively insufficient,
or objectively insufficient but subjectively sufficient.
The former is called opinion, the latter must be called belief.

Accordingly, there are three kinds or modi of holding-to-be-true:
opining, believing, and knowing.
Opining is problematic judging,
believing is assertoric judging,
and knowing is apodeictic judging.

For what I merely opine I hold in judging,
with consciousness, only to be problematic;
what I believe I hold to be assertoric,
but not as objectively necessary,
only as subjectively so (holding only for me);
what I know, finally, I hold to be apodeictically certain,
i.e., to be universally and objectively necessary
(holding for all),
even granted that the object
to which this certain holding-to-be-true relates
should be a merely empirical truth.
For this distinction in holding-to-be-true
according to the three modi just named concerns only
the power of judgment in regard to
the subjective criteria for subsumption of
a judgment under objective rules.

Thus, for example, our holding-to-be-true of immortality would be
merely problematic in case we only act as if we were immortal,
but it would be assertoric in case we believe that we are immortal,
and it would be apodeictic, finally, in case we all knew that
there is another life after this one.

There is an essential difference, then,
between opining, believing, and knowing,
which we wish to expound more exactly
and in more detail here.

1. Opining.

Opining, or holding-to-be-true based on a ground of cognition
that is neither subjectively nor objectively sufficient,
can be regarded as provisional judging
(sub conditione suspensiva ad interim)
that one cannot easily dispense with.
One must first opine before one accepts and maintains,
but in doing so must guard oneself against holding an opinion
to be something more than mere opinion.
For the most part, we begin with
opining in all our cognizing.
Sometimes we have an obscure premonition of truth,
a thing seems to us to contain marks of truth;
we suspect its truth even before we cognize it
with determinate certainty.

But now where does mere opining really occur?
Not in any sciences that contain cognitions a priori,
hence neither in mathematics nor in metaphysics nor in morals,
but merely in empirical cognitions: in physics, psychology, etc.
For it is absurd to opine a priori.
In fact, too, nothing could be more ridiculous than, e.g.,
only to opine in mathematics.
Here, as in metaphysics and in morals,
the rule is either to know or not to know.
Thus matters of opinion can only be
objects of a cognition by experience,
a cognition which is possible in itself
but impossible for us in accordance
with the restrictions and conditions of
our faculty of experience
and the attendant degree of this faculty that we possess.
Thus, for example, the ether of modern physicists is
a mere matter of opinion.
For with this as with every opinion in general, whatever it may be,
I see that the opposite could perhaps yet be proved.
Thus my holding-to-be-true is here
both objectively and subjectively insufficient,
although it can become complete, considered in itself.

2. Believing.

Believing, or holding-to-be-true based on a ground that is
objectively insufficient but subjectively sufficient, relates to objects in
regard to which we not only cannot know anything but also cannot opine
anything, indeed, cannot even pretend there is probability, but can only be
certain that it is not contradictory to think of such objects as one does
think of them. What remains here is a free holding-to-be-true, which is
necessary only in a practical respect given a priori, hence a holding-to-be-
true of what I accept on moral grounds, and in such a way that I am certain
that the opposite can never be proved.

Matters of belief are thus
I) not objects of empirical cognition. Hence so-
called historical belief cannot really be called belief, either, and cannot be
opposed as such to knowledge, since it can itself be knowledge. Holding-
to-be-true based on testimony is not distinguished from holding-to-be-
true through one's own experience either as to degree or as to kind.

II) [N]or [are they] objects of cognition by reason (cognition a priori),
whether theoretical, e.g., in mathematics and metaphysics, or practical, in
morals.

One can believe mathematical truths of reason on testimony, to be sure,
partly because error here is not easily possible, partly, too, because it can
easily be discovered, but one cannot know them in this way, of course. But
philosophical truths of reason may not even be believed, they must simply
be known; for philosophy does not allow mere persuasion. And as for
what concerns in particular the objects of practical cognition by reason in
morals, rights and duties, there can just as little be mere belief in regard to
them. One must be fully certain whether something is right or wrong, in
accordance with duty or contrary to duty, allowed or not allowed. In moral
things one cannot risk anything on the uncertain, one cannot decide any-
thing on the danger of trespass against the law. Thus it is not enough for the
judge, for example, that he merely believe that someone accused of a crime
actually committed this crime. He must know it (juridically), or he acts
unconscientiously.

III) The only objects that are matters of belief are those in which
holding-to-be-true is necessarily free, i.e., is not determined through
objective grounds of truth that are independent of the nature and the
interest of the subject.

Thus also on account of its merely subjective grounds, believing yields
no conviction that can be communicated and that commands universal
agreement, like the conviction that comes from knowledge. Only / myself
can be certain of the validity and unalterability of my practical belief, and
my belief in the truth of a proposition or the actuality of a thing is what
takes the place of a cognition only in relation to me without itself being a
cognition.

He who does not accept what it is impossible to know but morally neces-
sary to presuppose is morally unbelieving. At the basis of this kind of
unbelief lies always a lack of moral interest. The greater a man's moral
sentiment," the firmer and more lively will be his belief in all that he feels
himself necessitated to accept and to presuppose out of moral interest, for
practically necessary purposes.

3. Knowing.

Holding-to-be-true based on a ground of cognition that is
objectively as well as subjectively sufficient, or certainty, is
either empirical or rational, accordingly as it is grounded either
on experience, one's own as well as that communicated by others, or on reason.
This distinction relates, then, to the two sources from which
the whole of our cognition is drawn: experience and reason.

Rational certainty, again, is either
mathematical or philosophical certainty.
The former is intuitive, the latter discursive.

Mathematical certainty is also called evidence, because an intuitive cog-
nition is clearer than a discursive one. Although the two, mathematical
and philosophical cognition of reason, are in themselves equally certain,
the certainty is different in kind in them.

Empirical certainty is original (originarie empirica) insofar as I become
certain of something from my own experience, and derived (derivative em-
pirica) insofar as I become certain through someone else's experience. The
latter is also usually called historical certainty.

Rational certainty is distinguished from empirical certainty
by the consciousness of necessity that is combined with it;
hence it is apodeictic certainty,
while empirical certainty is only assertoric.
We are rationally certain of that
into which we would have had insight a priori
even without any experience.
Hence our cognitions can concern objects of experience
and the certainty concerning them can still be
both empirical and rational at the same time, namely,
insofar as we cognize an empirically certain proposition
from principles a priori.

We cannot have rational certainty of everything,
but where we can have it,
we must put it before empirical certainty.

All certainty is either unmediated or mediated, i.e., it either requires a
proof, or it is not capable of and does not require any proof. Even if so
much in our cognition is certain only mediately, i.e., through a proof,
there must still be something indemonstrable or immediately certain, and the
whole of our cognition must proceed from immediately certain proposi-
tions.

The proofs on which any mediated or mediate certainty of a cognition
rests are either direct proofs or indirect, i.e., apagogical ones.
When I prove a truth from its grounds I provide a direct proof for it,
and when I infer the truth of a proposition from the falsehood of its
opposite I provide an indirect one.
If this latter is to have validity, however,
the propositions must be opposed contradictorily or diametraliter.
For two propositions opposed only as contraries
(contrarie opposita) can both be false.
A proof that is the ground of mathematical certainty
is called a demonstration,
and that which is the ground of philosophical certainty
is called an acroamatic proof.
The essential parts of any proof in general are
its matter and its form,
or the ground of proof and the consequentia.

From [the German word for] knowing comes [the German word for] science,
by which is to be understood the complex of a cognition as a system.
It is opposed to common cognition;
to the complex of a cognition as mere aggregate.
A system rests on an idea of the whole,
which precedes the parts,
while with common cognition on the other hand,
or a mere aggregate of cognitions,
the parts precede the whole.
There are historical sciences and sciences of reason.

In a science we often know only the cognitions
but not the things represented through them;
hence there can be a science of that
of which our cognition is not knowledge.

From the foregoing observations concerning
the nature and the kinds of holding-to-be-true
we can now draw the universal result that
all our conviction is thus either logical or practical.
When we know, namely, that we are free of all subjective grounds
and yet the holding-to-be-true is sufficient,
then we are convinced, and in fact logically convinced, or
convinced on objective grounds (the object is certain).
Complete holding-to-be-true on subjective grounds,
which in a practical relation hold just as much as objective grounds,
is also conviction, though not logical
but rather pratical conviction (I am certain).
And this practical conviction, or this moral belief of reason,
is often firmer than all knowledge.
With knowledge one still listens to opposed grounds,
but not with belief, because here it does not depend on
objective grounds but on the moral interest of the subject.
