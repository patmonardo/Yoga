I. INFERENCES OF THE UNDERSTANDING

Peculiar nature of inferences of the understanding

The essential character of all immediate inferences
and the principle of their possibility consists simply in
an alteration of the mere form of judgments,
while the matter of the judgments,
the subject and predicate,
remains unaltered, the same.

Modi of inferences of the understanding

Inferences of the understanding run through
all the classes of the logical functions of judgment
and consequently are determined in their principal kinds
through the moments of quantity, quality, relation, and modality.
On this rests the following division of these inferences.

1.  Inferences of the understanding
    (in relation to the quantity of judgments)
    per judicia subalternata

In inferences of the understanding per judicia subalternata
the two judgments are distinct as to quantity,
and here the particular judgment is derived from the universal
in consequence of the principle:
The inference from the universal to the particular is valid
(ab universali ad particulare valet consequentia).

2.  Inferences of the understanding
    (in relation to the quality of judgments)
    per judicia opposita

In inferences of the understanding of this kind,
the alteration concerns the quality of the judgments,
and this considered in relation to opposition.
Now since this opposition can be of three kinds,
the following particular division of immediate inference results:
through contradictorily opposed,
through contrary,
and subcontrary judgments.

a.  Inferences of the understanding
    per judicia contradictorie opposita

In inferences of the understanding through judgments
which are opposed to one another contradictorily
and which, as such, constitute genuine, pure opposition,
the truth of one of the contradictorily opposed judgments
is deduced from the falsehood of the other, and conversely.
For genuine opposition, which occurs here,
contains no more and no less than what belongs to opposition.
In consequence of the principle of the excluded middle,
the two contradicting judgments cannot both be true,
and just as little can they both be false.
If the one is true, then the other is false, and conversely.

b. Inferences of the understanding per judicia contrarie opposita

Contrary or conflicting judgments (judicia contrarie opposita)
are judgments of which the one is universally affirmative,
the other universally negative.
Now since one of them says more than the other,
and since in the excess,
that it says more than the mere negation of the other,
there can lie falsehood, they cannot both be true, of course,
but they can both be false.
In regard to these judgments, therefore,
only the inference from the truth of the one
to the falsehood of the other holds, hut not conversely.

c. Inferences of the understanding per judicia subcontrarie opposita

Subcontrary judgments are ones of which
the one affirms or denies in particular (particulariter)
what the other denies or affirms in particular.
Since they can both be true but cannot both be false,
only the following inference holds in regard to them:
If one of these propositions is false,
the other is true, but not conversely.

3. Inferences of the understanding
(with respect to the relation of judgments)
per judicia conversa sive per conversionem

Immediate inferences through conversion
concern the relation of judgments
and consist in the transposition of
subject and predicate in the two judgments,
so that the subject of the one judgment is made
the predicate of the other judgment, and conversely.

Pure and altered conversion

In conversion the quantity of the judgments is
either altered or it remains unaltered.
In the first case the converted (conversum) is
distinct from what converts (convertens) as to quantity,
and the conversion is said to be altered
(conversio per accidens);
in the latter case the conversion is called pure
(conversio simpliciter talis).

Universal rules of conversion

With respect to inferences of the understanding through conversion,
the following rules hold:

1.  Universal affirmative judgments may be converted only per accidens,
    for the predicate in these judgments is a broader concept
    and thus only some of it is contained in the concept of the subject.

2.  But all universal negative judgments may be converted simpliciter,
    for here the subject is removed from the sphere of the predicate.

3.  Thus too, finally,
    All particular affirmative propositions may be converted simpliciter;
    for in these judgments a part of the sphere of the subject
    has been subsumed under the predicate,
    hence a part of the sphere of the predicate
    may be subsumed under the subject.

4. Inferences of the understanding
(in relation to the modality of judgments)
per judicia contraposita

The immediate mode of inference through contraposition
consists in that transposition (metathesis) of judgments
in which merely the quantity remains the same
while the quality is altered.
They concern only the modality of judgments,
since they transform an assertoric into an apodeictic judgment.

Universal rule of contraposition

With respect to contraposition, the universal rule holds:

All universal affirmative judgments may be contraposed simpliciter.
For if the predicate, as that which contains the subject under itself,
is denied, and hence the whole sphere,
then a part of it must also be denied,
i.e., the subject.

II. INFERENCES OF REASON

Inference of reason in general

An inference of reason is the cognition of the necessity of a proposition
through the subsumption of its condition under a given universal rule.

Universal principle of all inferences of reason

The universal principle on which the validity of
all inference through reason rests may be
determinately expressed in the following formula:

What stands under the condition of a rule
also stands under the rule itself.

Essential components of the inference of reason

To every inference of reason belong the following essential three parts:

1.  a universal rule, which is called the major proposition
    (propositio major),

2.  the proposition which subsumes a cognition
    under the condition of the universal rule,
    and which is called the minor proposition
    (propositio minor), and finally

3.  the proposition which affirms or denies
    the rule's predicate of the subsumed cognition:
    the conclusion (conclusio).

The first two propositions, in their combination with one another,
are called premises.

Matter and form of inferences of reason

The matter of inferences of reason consists in
the antecedent propositions or premises,
the form in the conclusion insofar as it contains the consequentia.

Division of inferences of reason (as to relation)
into categorical, hypothetical, and disjunctive

All rules (judgments) contain objective unity of
consciousness of the manifold of cognition,
hence a condition under which one cognition
belongs with another to one consciousness.
Now only three conditions of this unity may be thought, however, namely:
as subject of the inherence of marks, or
as ground of the dependence of one cognition on another, or, finally,
as combination of parts in a whole (logical division).
Consequently there can only be just as many
kinds of universal rules (propositiones majores),
through which the consequentia of one judgment
from another is mediated.
And on this is grounded the division of
all inferences of reason into
categorical, hypothetical, and disjunctive.

Peculiar difference between categorical, hypothetical and disjunctive
inferences of reason

The distinguishing feature among the three mentioned
kinds of inferences of reason lies in the major premise.
In categorical inferences of reason the major is a categorical proposition,
in hypothetical ones it is a hypothetical or problematic proposition,
and in disjunctive ones it is a disjunctive proposition.

1. Categorical inferences of reason

In every categorical inference of reason there are
three principal concepts (termini), namely:

1.  the predicate in the conclusion,
    which concept is called the major concept (terminus major),
    because it has a larger sphere than the subject,

2.  the subject (in the conclusion),
    whose concept is called the minor concept (terminus minor), and

3.  a mediating mark (nota intermedia),
    which is called the middle concept (terminus medius),
    because through it a cognition is subsumed under
    the condition of the rule.

Principle of categorical inferences of reason

The principle on which the possibility and validity of
all categorical inferences of reason rests is this:
What belongs to the mark of a thing
belongs also to the thing itself;
and what contradicts the mark of a thing
contradicts also the thing itself
(nota notae est nota rei ipsius;
repugnans notae, repugnat rei ipsi).

Rules for categorical inferences of reason

From the nature and the principle of
categorical inferences of reason
flow the following rules for them:

1.  In no categorical inference of reason can
    either more or fewer than three
    principal concepts (termini) be contained;
    for here I am supposed to combine
    two concepts (subject and predicate)
    through a mediating mark.

2.  The antecedent propositions or premises
    may not be wholly negative
    (expuris negatives nihil sequitur);
    or the subsumption in the minor premise must be affirmative,
    as that which states that a cognition
    stands under the condition of the rule.

3.  The premises may also not be wholly particular propositions
    (ex puris particularibus nihil sequitur);
    for then there would be no rule, i.e., no universal proposition,
    from which a particular cognition could be deduced.

4.  The conclusion always follows the weaker part of the inference, i.e.,
    the negative and the particular proposition in the premises,
    as that which is called the weaker part of
    the categorical inference of reason
    (conclusio sequitur partem debiliorem).
    If, therefore,

5.  one of the premises is a negative proposition,
    then the conclusion must be negative too, and

6.  if one premise is a particular proposition,
    then the conclusion must be particular too.

7.  In all categorical inferences of reason
    the major must be a universal proposition (universalis),
    the minor an affirmative one (affirmans),
    and from this it follows, finally,

8.  that the conclusion must follow
    the major premise in regard to quality,
    but the minor premise as to quantity.

Pure and mixed categorical inferences of reason

A categorical inference of reason is pure (punts)
if no immediate inference is mixed in with it,
nor is the legitimate order of the premises altered;
in the contrary case it is called impure or mixed
(ratiocinium impurum or hybridum).

Mixed inferences of reason through conversion of
propositions - Figures

Those inferences which arise through the conversion of propositions
and in which the position of these propositions is thus
not the legitimate one
are to be counted as mixed inferences.
This case occurs in the three latter so-called
figures of the categorical inference of reason.

Four figures of inferences

By figures are to be understood those four modes of inferring whose
difference is determined through the particular position of the premises
and their concepts.

Ground of the determination of their distinction
through the position of the middle concept

The middle concept, namely,
on whose position things really depend here,
can occupy either
(1.)    the place of the subject in the major premise
        and the place of the predicate in the minor premise, or
(2.)    the place of the predicate in both premises, or
(3.)    the place of the subject in both, or finally
(4.)    the place of the predicate in the major premise
        and the place of the subject in the minor premise.

Through these four cases,
the distinction among the four figures is determined.
If we let S signify the subject of the conclusion,
P its predicate,
and M the terminus medius,
then the schema for the four mentioned figures
may be exhibited in the following table:

Rules for the first figure, as the only legitimate one

The rule of first figure is that
the major be a universal proposition,
the minor an affirmative.
And since this must be the universal rule of
all categorical inferences of reason in general,
it results from this that
the first figure is the only legitimate one,
which lies at the basis of all the others,
and to which all the others must be traced back
through conversion of premises (metathesis praemissorum),
insofar as they are to have validity.

Condition of the reduction of the three latter figures to the first

The condition of the validity of the three latter figures,
under which a correct modus of inference is possible in each of them,
amounts to this:

that the medius terminus occupies a place such that from it,
through immediate inferences (consequentias immediatas),
their position in accordance with the rules of the first figure can arise.
From this the following rules for the three latter figures emerge.

Rules of the second figure

In the second figure the minor stands rightly,
hence the major must be converted,
and in such a way that it remains universal (universalis).
This is possible only if it is universally negative;
but if it is affirmative, then it must be contraposed.
In both cases the conclusion becomes negative
(sequitur partem debiliorem).

Rule of the third figure

In the third figure the major stands rightly,
hence the minor must be converted,
yet in such a way that an affirmative proposition arises therefrom.
This is only possible, however,
when the affirmative proposition is particular,
consequently the conclusion is particular.

Rule of the fourth figure

If in the fourth figure the major is universal negative,
then it may be converted purely (simpliter),
just as the minor may be as particular;
hence the conclusion is negative.
If, on the other hand, the major is universal affirmative,
then it may either be converted only per accidens
or it may be contraposed;
hence the conclusion is either particular or negative.
If the conclusion is not to be converted
(P S transformed into S P),
a transposition of the premises (metathesis praemissorum)
or a conversion (conversio) must occur.

Universal results concerning the three latter figures

From the rules stated for the three latter figures it is clear

1.  that in none of them is there a universal affirmative conclusion,
    but rather the conclusion is always either negative or particular;

2.  that in every one an immediate inference
    (consequentia immediata) is mixed in,
    which is not expressly signified, to be sure,
    but still must be silently included;

3.  that on this account, too, these three latter modi of inference
    must all be called impure inferences (ratiocinia hybrida, impura),
    not pure ones, since no pure inference can have more than
    three principal propositions (termini).

2. Hypothetical inferences of reason

A hypothetical inference is one
that has a hypothetical proposition as major.
Thus it consists of two propositions,
(1.) an antecedent proposition (antecedent) and
(2.) a consequent proposition (consequens), and
here the deduction is either according to modus ponens or to modus tollens.

Principle of hypothetical inferences

The principle of hypothetical inferences is
the principle of the ground,

a ratione ad rationatum;
a negatione rationati ad negationem rationis valet consequential

3. Disjunctive inferences of reason

In disjunctive inferences, the major is a disjunctive proposition
and as such must therefore have members of division or disjunction.
Here we infer either

(1.)    from the truth of one member of the disjunction
        to the falsehood of the others, or

(2.)    from the falsehood of all members but one
        to the truth of this one.

The former occurs through the modus ponens (or ponendo tollens),
the latter through the modus tollens (tollendo ponens).
The consequentia from the ground to the grounded,
and from the negation of the grounded to negation of the ground,
is valid.

Principle of disjunctive inferences of reason

The principle of disjunctive inferences is
the principle of the excluded middle:

A contradictorie oppositorum negatione unius ad affirmationem
positione unius ad negationem alterius valet consequential alterius

Dilemma

A dilemma is a hypothetical-disjunctive inference of reason,
or a hypothetical inference,
whose consequens is a disjunctive judgment.
The hypothetical proposition whose consequens is disjunctive
is the major proposition;
the minor proposition affirms that the consequens
(per omnia membra) is false,
and the conclusion affirms that the antecedent is false.
(A remotione consequents ad negationem antecedents valet consequential)

Formal and covert inferences of reason
(ratiocinia formalia and cryptica)

A formal inference of reason is one that
not only contains everything required as to matter
but also is expressed correctly and completely as to form.
Opposed to formal inferences of reason are covert ones (cryptica),
which all those can be reckoned,
in which either the premises are transposed,
or one of the premises is left out, or, finally,
the middle concept alone is combined with the conclusion.
A covert inference of reason of the second kind,
in which one premise is not expressed but only thought,
is called a truncated one or an enthymeme.
Those of the third kind are called contracted inferences.

III. INFERENCES OF THE POWER OF JUDGMENT

Determinative and reflective power of judgment

The power of judgment is of two kinds:
the determinative or the reflective power of judgment.
The former goes from the universal to the particular,
the second from the particular to the universal.
The latter has only subjective validity,
for the universal to which it proceeds from the particular
is only empirical universality, a mere analogue of the logical.

Inferences of the (reflective) power of judgment

Inferences of the power of judgment are
certain modes of inference for coming
from particular concepts to universal ones.
They are not functions of the determinative power of judgment, then,
but rather of the reflective;
hence they also do not determine the object,
but only the mode of reflection concerning it,
in order to attain its cognition.

Principle of these inferences

The principle that lies at the basis of
these inferences of the power of judgment is this:
that the many will not agree in one without a common ground,
but rather that which belongs to the many in this way will be
necessary due to a common ground.

Induction and analogy — The two modes of inference of the power of judgment

The power of judgment, by proceeding
from the particular to the universal
in order to draw from experience (empirically)
universal (hence not a priori) judgments,
infers either from many to all things of a kind,
or from many determinations and properties,
in which things of one kind agree,
to the remaining ones,
insofar as they belong to the same principle.
The former mode of inference is called inference through induction,
the other inference according to analogy.

Simple and composite inferences of reason

An inference of reason is called simple
if it consists of one inference of reason,
composite if of several.

Ratiocinatio polysyllogistica

A composite inference, in which
the several inferences of reason
are combined with one another
not through mere coordination
but through subordination,
i.e., as grounds and consequences,
is called a chain of inferences of reason
(ratiocinatio polysyllogistica).

Prosyllogisms and episyllogisms

In the series of composite inferences one can infer in two ways,
either from the grounds down to the consequences, or
from the consequences up to the grounds.
The first occurs through episyllogisms,
the other through prosyllogisms.
An episyllogism is that inference, namely,
in the series of inferences,
whose premise becomes the conclusion of a prosyllogism,
inference that has the premises of the former as conclusion.

Sorites or chain inference

An inference consisting of several inferences
that are shortened and combined with one another for one conclusion
is called a sorites or a chain inference,
which can be either progressive or regressive,
accordingly as one climbs from the nearer grounds
up to the more distant ones,
or from the more distant grounds
down to the nearer ones.

Categorical and hypothetical sorites

Both progressive and regressive chain inferences
can in turn be either categorical or hypothetical.
The former consists of categorical propositions,
as a series of predicates,
the latter of hypothetical ones,
as a series of consequences.

Fallacy — Paralogism - Sophism

An inference of reason that is wrong as to form,
although it has for itself the illusion of a correct inference,
is called a fallacy (fallacia).
Such an inference is a paralogism insofar as
one deceives oneself through it,
a sophism insofar as one intentionally seeks
to deceive others through it.

Leap in inference

A leap (saltus) in inference or proof is
the combination of a premise with the conclusion
so that the other premise is left out.
Such a leap is legitimat (legitimus)
if everyone can easily add the missing premise in thought,
but illegitimate (illegitimus) if the subsumption is not clear.
Here a distant mark is connected with a thing
without an intermediate mark (nota intermedia).

Petitio principii - Circulus in probando

By a petitio principii is understood
the acceptance of a proposition as ground of proof
as an immediately certain proposition,
although it still requires a proof.
And one commits a circle in proof
if one lays at the basis of its own proof
the very proposition that one wanted to prove.

Probatio plus and minus probans

A proof can prove too much, but also too little.
In the latter case it proves only a part of what is to be proved,
in the former it also goes on to that which is false.
