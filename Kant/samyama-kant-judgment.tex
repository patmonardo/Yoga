Logical forms of judgments: Quantity, quality, relation, and modality

The distinctions among judgments in respect of their form
may be traced back to the four principal moments of
quantify, quality, relation, and modality,
in regard to which just as many different kinds of judgments are determined.

Quantity of judgments: Universal, particular, singular

As to quantity, judgments are either
universal or particular or singular,
accordingly as the subject is either
wholly included in or excluded from
the notion of the predicate
or is only in part included in or excluded from it.
In the universal judgment, the sphere of one concept is
wholly enclosed within the sphere of another;
in the particular, a part of the former is
enclosed under the sphere of the other;
and in the singular judgment, finally,
a concept that has no sphere at all is enclosed,
merely as part then, under the sphere of another.

Quality of judgments: Affirmative, negative, infinite

As to quality, judgments are either
affirmative or negative or infinite.
In the affirmative judgment the subject
is thought under the sphere of a predicate,
in the negative it is posited outside the sphere of the latter,
and in the infinite it is posited in the sphere of a concept
that lies outside the sphere of another.

Relation of judgments: Categorical, hypothetical, disjunctive

As to relation, judgments are either
categorical or hypothetical or disjunctive.
The given representations in judgment are
subordinated one to another
for the unity of consciousness, namely, either
as predicate to subject, or
as consequence to ground, or
as member of the division to the divided concept.
Through the first relation categorical judgments are determined,
through the second hypothetical,
and through the third disjunctive.

Categorical judgments

In categorical judgments, subject and predicate constitute their matter;
the form, through which the relation (of agreement or of opposition)
between subject and predicate is determined and expressed,
is called the copula.

Hypothetical judgments

The matter of hypothetical judgments consists of
two judgments that are connected with one another
as ground and consequence.
One of these judgments, which contains the ground,
is the antecedent (antecedens, prius),
the other, which is related to it as consequence,
is the consequent (consequens, posterius),
and the representation of this kind of connection of
two judgments to one another for the unity of consciousness
is called the consequentia,
which constitutes the form of hypothetical judgments.

Modes of connection in hypothetical judgments: modus ponens and modus tollens

The form of the connection in hypothetical judgments is of two kinds:
the positing (modus ponens) and the denying (modus tollens).
1. If the ground (antecedens) is true,
then the consequence (consequens) determined by it is true too;
called modus ponens.
2. If the consequence (consequens) is false,
then the ground (antecedens) is false too;
modus tollens.

Disjunctive judgments

A judgment is disjunctive if the parts of
the sphere of a given concept
determine one another in the whole
or toward a whole as complements (complementa).

Matter and form of disjunctive judgments

The several given judgments of which the disjunctive judgment
is composed constitute its matter
and are called the members of the disjunction or opposition.
The form of these judgments consists in the disjunction itself,
in the determination of the relation of the various judgments
as members of the whole sphere of the divided cognition
which mutually exclude one another and complement one another.

Peculiar character of disjunctive judgments

The peculiar character of all disjunctive judgments,
whereby their specific difference from others,
in particular from categorical judgments,
is determined as to the moment of relation,
consists in this:
that the members of the disjunction are all problematic judgments,
of which nothing else is thought except that,
taken together as parts of the sphere of a cognition,
each the complement of the other toward the whole
(complementum ad totum),
they are equal to the sphere of the first.
And from this it follows that
in one of these problematic judgments
the truth must be contained
or (what is the same)
that one of them must hold assertorically,
because outside of them the sphere of the cognition
includes nothing more under the given conditions,
and one is opposed to the other,
consequently neither something outside them
nor more than one among them can be true.

Modality of judgments: Problematic, assertoric, apodeictic

As to modality, through which moment the relation of
the whole judgment to the faculty of cognition is determined,
judgments are either problematic or assertoric or apodeictic.
The problematic ones are accompanied with
the consciousness of the mere possibility of the judging,
the assertoric ones with the consciousness of its actuality,
the apodeictic ones, finally, with the consciousness of its necessity.

Exponible judgments

Judgments in which an affirmation and a negation
are contained simultaneously, but in a covert way,
so that the affirmation occurs distinctly
but the negation covertly, are exponible propositions.

Theoretical and practical propositions

Those propositions that relate to the object
and determine what belongs or does not belong to it
are called theoretical;
practical propositions, on the other hand,
are those that state the action whereby,
as its necessary condition,
an object becomes possible.

Indemonstrable and demonstrable propositions

Demonstrable propositions are those that are capable of a proof;
those not capable of a proof are called indemonstrable.
Immediately certain judgments are indemonstrable
and thus are to be regarded as elementary propositions.

Principles

Immediately certain judgments a priori can be called principles,
insofar as other judgments are proved from them,
but they themselves cannot be subordinated to any other.
On this account they are also called principles (beginnings).

Intuitive and discursive principles: Axioms and acroamata

Principles are either intuitive or discursive.
The former can be exhibited in intuition
and are called axioms (axiomata),
the latter may be expressed only through concepts
and can be called acroamata.

Analytic and synthetic propositions

Propositions whose certainty rests on identity of concepts
(of the predicate with the notion of the subject)
are called analytic propositions.
Propositions whose truth is not grounded on identity of concepts
must be called synthetic.

Tautological propositions

The identity of the concepts in analytic judgments can be
either explicit (explicita) or non-explicit (implicita).
In the first case the analytic propositions are tautological.

Postulate and problem

A postulate is a practical, immediately certain proposition,
or a principle that determines a possible action,
in the case of which it is presupposed that
the way of executing it is immediately certain.

Problems (problemata) are demonstrable propositions
that require a directive,
or ones that express an action,
the manner of whose execution is not immediately certain.

Theorems, corollaries, lemmas, and scholia

Theorems are theoretical propositions
that are capable of and require proof.
Corollaries are immediate consequences
from a preceding proposition.
Propositions that are not indigenous to the science
in which they are presupposed as proved,
but rather are borrowed from other sciences,
are called lemmas (lemmata).
Scholia, finally, are mere elucidative propositions,
which thus do not belong to the whole of the system as members.

Judgments of perception and of experience

A judgment of perception is merely subjective,
an objective judgment from perceptions is a judgment of experience.
