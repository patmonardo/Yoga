
I. FURTHERING LOGICAL PERFECTION OF COGNITION THROUGH
DEFINITION, EXPOSITION, AND DESCRIPTION OF CONCEPTS

Definition

A definition is a sufficiently distinct and precise concept
(conceptus rei adaequatus in minimis terminis, complete determinatus).

Analytic and synthetic definition

All definitions are either analytic or synthetic.
The former are definitions of a concept that is given,
the latter of one that is made.
A concept adequate to the thing, in minimal terms, completely determined.

Concepts that are given and made a priori and a posteriori

The given concepts of an analytic definition
are given either a priori or a posteriori,
just as the concepts of a synthetic definition, which are made,
are made either a priori or a posteriori.

Synthetic definitions through exposition or construction

The synthesis of concepts that are made, out of which synthetic defini-
tions arise, is either that of e3Cposition s (of appearances) or that of construc-
tion. The latter is the synthesis of concepts that are made arbitrarily, the
former the synthesis of concepts that are made empirically, i.e., from given
appearances as their matter (conceptus factitii vel a priori vel per synthesin
empiricam 1 '). Concepts that are made arbitrarily are the mathematical ones.

Impossibility of empirically synthetic definitions

Since the synthesis of empirical concepts is not arbitrary but
rather is empirical and as such can never be complete
(because one can always discover in experience more marks of the concept),
empirical concepts cannot be defined, either.

Analytic definitions through analysis of
concepts given a priori or a posteriori

All given concepts, be they given a priori or a posteriori,
can be defined only through analysis.
For one can make given concepts distinct only insofar as
one successively makes their marks clear.
If all the marks of a given concept are made clear,
then the concept becomes completely distinct;
if it does not contain too many marks,
then it is at the same time precise,
and from this there arises a definition of the concept.

Expositions and descriptions

Not all concepts can be defined, and not all need to be.
There are approximations to the definition of certain concepts;
these are partly expositions (expositiones), partly descriptions (descriptions).
The expounding of a concept consists in the connected (successive)
representation of its marks, insofar as these are found through analysis.
Description is the exposition of a concept, insofar as it is not precise.
Description can occur only with empirically given concepts. It has no
determinate rules and contains only the materials for definition.

Nominal and real definitions

By mere definitions of names, or nominal definitions, are to be understood
those that contain the meaning that one wanted arbitrarily to give to a
certain name, and which therefore signify only the logical essence of their
object, or which serve merely for distinguishing it from other objects.
Definitions of things," 1 or real definitions, on the other hand, are ones that
suffice for cognition of the object according to its inner determinations,
since they present the possibility of the object from inner marks.

Principal requirements of definition

The essential and universal requirements that pertain to the completeness
of a definition in general may be considered under the four principal
moments of quantity, quality, relation, and modality:
as to quantity - what concerns the sphere of the definition - the definition
and the definitum must be convertible concepts" (conceptus redprocf), and hence
the definition must be neither broader nor narrower than its definitum;
as to quality, the definition must be
a detailed and at the same time precise concept,
as to relation, it must not be tautological,
i.e., the marks of the definitum must, as grounds of its cognition,
be different from it itself, and finally
as to modality, the marks must be necessary, and
hence not such as are added through experience.

Rules for testing definitions

In the testing of definitions four acts are to be performed;
it is to be investigated, namely,
1.  whether the definition considered as a proposition is true,
2.  whether as a concept it is distinct,
3.  whether as a distinct concept it is also detailed, and finally
4.  whether as a detailed concept it is at the same time determinate,
    i.e., adequate to the thing itself.

Rules for preparation of definitions

The very same acts that belong to the testing of definition are
also to be performed in the preparation of them.
Toward this end, then, seek
(1.)    true propositions,
(2.)    whose predicate does not presuppose the concept of the thing;
(3.)    collect several of them and compare them with
        the concept of the thing itself
        to see if they are adequate; and finally
(4.)    see whether one mark does not lie in another
        or is not subordinated to it.

II. FURTHERING THE PERFECTION OF COGNITION
THROUGH LOGICAL DIVISION OF CONCEPTS

Concept of logical division

Every concept contains a manifold under itself
insofar as the manifold agrees,
but also insofar as it is different.
The determination of a concept in regard to
everything possible that is contained under it,
insofar as things are opposed to one another,
are distinct from one another,
is called the logical division of the concept.
The higher concept is called the divided concept (divisus),
the lower concepts the members of the division (membra dividentia).

Universal rules of logical division

In every division of a concept we must see to it:
1.  that the members of the division exclude
    or are opposed to one another,
    that furthermore
2.  they belong under one higher concept (conceptus communis),
    and finally that
3.  taken together they constitute the sphere of the divided concept
    or are equal to it.

Codivision and subdivision

Various divisions of a concept,
which are made in various respects,
are called codivisions,
and division of the members of division
is called a subdivision (subdivisio).

Dichotomy and polytomy

A division into two members is called dichotomy;
but if it has more than two members, it is called polytomy.

Various divisions of method

Now as for what concerns in particular method itself
in working up and treating scientific cognitions,
there are various principal kinds of it,
which we can present in accordance with the following division.

1. Scientific or popular method

Scientific or scholastic method differs from popular method
through the fact that the former proceeds from
basic and elementary propositions,
but the latter from the customary and the interesting.
The former aims for thoroughness and thus removes everything foreign,
the latter aims at entertainment.

2. Systematic or fragmentary method

Systematic method is opposed to fragmentary or rhapsodic method.
If one has thought in accordance with a method
and then also expressed this method in the exposition,
and if the transition from one proposition to
another is distinctly presented,
then one has treated a cognition systematically.
If, on the other hand, one has thought according to a method
but has not arranged the exposition methodically,
such a method is to be called rhapsodic.

3. Analytic or synthetic method

Analytic is opposed to synthetic method.
The former begins with the conditioned and grounded
and proceeds to principles (a prindpiatis ad prindpia),
while the latter goes from principles to consequences
or from the simple to the composite.
The former could also be called regressive,
as the latter could progressive.

4. Syllogistic [or] tabular method

Syllogistic method is that according to which
a science is expounded in a chain of inferences.
That method in accordance with which a finished system
is exhibited in its complete connection is called tabular.

5. Acromatic or eromatic method

Method is acroamatic insofar as someone only teaches,
erotematic insofar as he asks as well.
The latter method can be divided in turn
into dialogic or Socratic method and catechistic method,
accordingly as the questions are directed
either to the understanding or merely to memory.

Meditation

By meditation is to be understood reflection, or methodical thought.
Meditation must accompany all reading and learning,
and for this it is requisite that
one first undertake provisional investigations
and then put his thoughts in order,
or connect them in accordance with a method.
