C. THE IDEA

First of all, then, of the means for furthering
the distinctness of concepts in regard to their content.

a. Life

Manner and method

All cognition, and a whole of cognition,
must be in conformity with a rule.
(Absence of rules is at the same time unreason.)
But this rule is either that of manner (free)
or that of method (compulsion).

Form of science — Method

Cognition, as science, must be arranged
in accordance with a method.
For science is a whole of cognition as a system,
and not merely as an aggregate.
It therefore requires a systematic cognition,
hence one composed in accordance with rules
on which we have reflected.

Doctrine of method — Its object and end

As the doctrine of elements in logic has for its content
the elements and conditions of the perfection of a cognition,
so the universal doctrine of method, as the other part of logic,
has to deal with the form of a science in general,
or with the ways of acting so as to connect
the manifold of cognition in a science.

Means for furthering the logical perfection of cognition

The doctrine of method is supposed to expound the way for us
to attain the perfection of cognition.
Now one of the most essential logical perfections of cognition
consists in its distinctness, thoroughness, and systematic ordering
into the whole of a science.
Accordingly, the doctrine of method will have principally to provide
the means through which these perfections of cognition are furthered.

Conditions of the distinctness of cognition

The distinctness of cognitions and their combination
in a systematic whole depends on the distinctness of concepts
both in regard to what is contained in them
and in respect of what is contained under them.
The distinct consciousness of the content of concepts is furthered
by exposition and definition of them,
while the distinct consciousness of their extension, on the other hand,
is furthered through logical division of them.

b. Knowing

Cognition in general
    Intuitive and discursive cognition;
        intuition and concept
        and in particular their difference
    Logical and aesthetic perfection of cognition

All our cognition has a twofold relation,
first, a relation to the object,
second a relation to the subject.
In the former respect it is related to representation,
in the latter to consciousness,
the universal condition of all cognition in general.
(Consciousness is really a representation
that another representation is in me.)

In every cognition we must distinguish matter as the object,
and form as the way in which we cognize the object.

    If a savage sees a house from a distance, for example,
    with whose use he is not acquainted,
    he admittedly has before him in his representation
    the very same object as someone else
    who is acquainted with it determinately
    as a dwelling established for men.
    But as to form, this cognition of one and the same object
    is different in the two.
    With the one it is mere intuition,
    with the other it is intuition and concept at the same time.

The difference in the form of the cognition rests on
a condition that accompanies all cognition, on consciousness.
If I am conscious of the representation, it is clear,
if I am not conscious of it, obscure.

    Since consciousness is the essential condition of
    all logical form of cognitions,
    logic can and may occupy itself only
    with clear but not with obscure representations.
    In logic we do not see how representations arise,
    but merely how they agree with logical form.
    In general logic cannot deal at all with
    mere representations and their possibility either.
    This it leaves to metaphysics.
    Logic itself is occupied merely with the rules of thought in
    concepts, judgments, and inferences,
    as that through which all thought takes place.
    Something precedes, of course, before a representation becomes a concept.
    We will indicate that in its place, too.
    But we will not investigate how representations arise.
    Logic deals with cognition too, to be sure,
    because in cognition there is already thought.
    But representation is not yet cognition, rather,
    cognition always presupposes representation.
    And this latter cannot be explained at all.
    For we would always have to explain what representation is
    by means of yet another representation.

All clear representations, to which alone logical rules can be applied,
can now be distinguished in regard to distinctness and indistinctness.
If we are conscious of the whole representation,
but not of the manifold that is contained in it,
then the representation is indistinct.

    First, to elucidate this, an example in intuition.

    We glimpse a country house in the distance.
    If we are conscious that the intuited object is a house,
    then we must necessarily have a representation of
    the various parts of this house, the windows, doors, etc.
    For if we did not see the parts,
    we would not see the house itself either.
    But we are not conscious of this
    representation of the manifold of its parts,
    and our representation of the object indicated is
    thus itself an indistinct representation.

    If we want an example of indistinctness in concepts, furthermore, then
    the concept of beauty may serve. Everyone has a clear concept of beauty.
    But in this concept many different marks occur, among others that the
    beautiful must be something that (i.) strikes the senses and (2.) pleases
    universally. Now if we cannot explicate the manifold of these and other
    marks of the beautiful, then our concept of it is still indistinct.

    This is the situation with all simple representations,
    which never become distinct,
    not because there is confusion in them,
    but rather because there is no manifold to be found in them.
    One must call them indistinct, therefore, but not confused.

    And even with compound representations, too, in which
    a manifold of marks can be distinguished,
    indistinctness often derives not from confusion
    but from weakness of consciousness.
    Thus something can be distinct as to form,
    I can be conscious of the manifold in the representation,
    but the distinctness can diminish as to matter if
    the degree of consciousness becomes smaller,
    although all the order is there.
    This is the case with abstract representations.

Distinctness itself can be of two sorts:

First, sensible.

This consists in the consciousness of the manifold in intuition.

    I see the Milky Way as a whitish streak, for example;
    the light rays from the individual stars located in it
    must necessarily have entered my eye.
    But the representation of this was merely clear,
    and it becomes distinct only through the telescope,
    because then I glimpse the individual stars contained in the Milky Way.

Secondly, intellectual;

distinctness in concepts or distinctness of the understanding.
This rests on the analysis of the concept in regard to
the manifold that lies contained within it.

Thus in the concept of virtue, for example, are contained as marks
(1.) the concept of freedom,
(2.) the concept of adherence to rules (to duty),
(3.) the concept of overpowering the force of the inclinations,
in case they oppose those rules.

    Now if we break up the concept of virtue
    into its individual constituent parts,
    we make it distinct for ourselves through this analysis.
    By thus making it distinct, however, we add nothing to a concept;
    we only explain it.
    With distinctness, therefore, concepts are improved
    not as to matter but only as to form.

If we reflect on our cognitions in regard to the two essentially different
basic faculties, sensibility and the understanding, from which they arise,
then here we come upon the distinction between intuitions and concepts.
Considered in this respect, all our cognitions are, namely,
either intuitions or concepts.

The former have their source in sensibility, the faculty of intuitions,
the latter in the understanding, the faculty of concepts.
This is the logical distinction between understanding and sensibility,
according to which the latter provides nothing but intuitions,
the former on the other hand nothing but concepts.

    The two basic faculties may of course be considered
    from another side and defined in another way:
    sensibility, namely, as a faculty of receptivity,
    the understanding as a faculty of spontaneity.
    But this mode of explanation is not logical but rather metaphysical.
    It is also customary to call sensibility the lower faculty,
    the understanding on the other hand the higher faculty,
    on the ground that sensibility gives the mere material for thought,
    but the understanding rules over this material
    and brings it under rules or concepts.

Logical and aesthetic perfection of cognition

The difference between aesthetic and logical perfection of cognition
is grounded on the distinction stated here
between intuitive and discursive cognitions, or
between intuitions and concepts.

A cognition can be perfect either
according to laws of sensibility or
according to laws of the understanding;
in the former case it is aesthetically perfect,
in the other logically perfect.

The two, aesthetic and logical perfection, are thus of different kinds;
the former relates to sensibility, the latter to the understanding.
The logical perfection of cognition rests on its agreement with the object,
hence on universally valid laws, and hence we
can pass judgment on it according to norms a priori.

Aesthetic perfection consists in the agreement of
cognition with the subject
and is grounded on the particular sensibility of man.
In the case of aesthetic perfection, therefore,
there are no objectively and universally valid laws,
in relation to which we can pass judgment on it a priori
in a way that is universally valid
for all thinking beings in general.
Insofar as there are nonetheless universal laws of sensibility,
which have validity subjectively for the whole of humanity
although not objectively and for all thinking beings in general,
we can think of an aesthetic perfection that contains
the ground of a subjectively universal pleasure.

This is beauty, that which pleases the senses in intuition and
can be the object of a universal pleasure just because
the laws of intuition are universal laws of sensibility.
Through this agreement with the universal laws of sensibility
the really, independently beautiful,
whose essence consists in mere form,
is distinguished in kind from the pleasant,
which pleases merely in sensation
through stimulation or excitement,
and which on this account can only be
the ground of a merely private pleasure.
It is this essential aesthetic perfection, too,
which, among all [perfections], is compatible with
logical perfection and may best be combined with it.

    Considered from this side, aesthetic perfection in regard to the essentially
    beautiful can thus be advantageous to logical perfection. In another
    respect it is also disadvantageous, however, insofar as we look, in the case
    of aesthetic perfection, only to the non-essentially beautiful, the stimulating
    or the exciting, which pleases the senses in mere sensation and does not
    relate to mere form but rather to the matter of sensibility. For stimulation
    and excitement, most of all, can spoil the logical perfection in our
    cognitions and judgments.

    In general, however, there always remains a kind of conflict between the
    aesthetic and the logical perfection of our cognition, which cannot be fully
    removed. The understanding wants to be instructed, sensibility enlivened;
    the first desires insight, the second comprehensibility. If cognitions are to
    instruct then they must to that extent be thorough; if they are to entertain
    at the same time, then they have to be beautiful as well. If an exposition is
    beautiful but shallow, then it can only please sensibility but not the understanding,
    but if it is thorough yet dry, only the understanding but not
    sensibility as well.

    Since the needs of human nature
    and the end of popularity in cognition
    demand, however, that we seek to unite
    the two perfections with one another,
    we must make it our task to provide
    aesthetic perfection for those cognitions
    that are in general capable of it,
    and to make a scholastically correct,
    logically perfect cognition
    popular through its aesthetic form.

But in this effort to combine aesthetic with logical perfection
in our cognitions we must not fail to attend to the following rules, namely:
(1.) that logical perfection is the basis of all other perfections and
hence cannot be wholly subordinated or sacrificed to any other;
(2.) that one should look principally to formal aesthetic perfection,
the agreement of a cognition with the laws of intuition,
because it is just in this that the essentially beautiful,
which may best be combined with logical perfection, consists;
(3.) that one must be very cautious with stimulation and excitement,
whereby a cognition affects sensation and acquires an interest for it,
because attention can thereby so easily be drawn
from the object to the subject,
whence a very disadvantageous influence on
the logical perfection of cognition must evidently arise.

To acquaint us better with the essential differences that exist between
the logical and the aesthetic perfection of cognition,
not merely in the universal but from several particular sides,
we want to compare the two with one another in respect to
the four chief moments of quantity, quality, relation, and modality,
on which the passing of judgment as to the perfection of cognition depends.

A cognition is perfect
(1.) as to quantity if it is universal;
(2.) as to quality if it is distinct;
(3.) as to relation if it is true; and finally
(4.) as to modality if it is certain.

Considered from the viewpoints indicated,
a cognition will thus be logically perfect

as to quantity if it has objective universality

    (universality of the concept or of the rule),

as to quality if it has objective distinctness

    (distinctness in the concept),

as to relation if it has objective truth, and finally

as to modality if it has objective certainty.

To these logical perfections correspond now the following aesthetic
perfections in relation to those four principal moments, namely

aesthetic universality.

    This consists in the applicability of a cognition to a
    multitude of objects that serve as examples,
    to which application of it can be made,
    and whereby it becomes useful at the same time for the end of popularity;

aesthetic distinctness.

    This is distinctness in intuition, in which a concept
    thought abstractly is exhibited or elucidated in concrete through examples;

aesthetic truth.

    A merely subjective truth, which consists only in the agreement
    of cognition with the subject and the laws of sensory illusion,
    and which is consequently nothing more than a universal semblance.

aesthetic certainty.

    This rests on what is necessary in consequence of
    the testimony of the senses;
    what is confirmed through sensation and experience.

With the perfections just mentioned two things are always to be found,
which in their harmonious union make up perfection in general,
namely, manifoldness and unity.
Unity in the concept lies with the understanding,
unity of intuition with the senses.
Mere manifoldness without unity cannot satisfy us.
And thus truth is the principal perfection among them all,
because it is the ground of unity through
the relation of our cognition to the object.
Even in the case of aesthetic perfection,
truth always remains the conditio sine qua non,
the foremost negative condition,
apart from which something cannot please taste universally.
Hence no one may hope to make progress in the belles lettres
if he has not made logical perfection the ground of his cognition.
It is in the greatest possible unification of
logical with aesthetic perfection in general,
in respect to those cognitions that are both
to instruct and to entertain,
that the character and the art of
the genius actually shows itself.

c. The absolute idea
