Content and extension of concepts

Every concept, as partial concept, is contained in
the representation of things;
as ground of cognition, i.e., as mark,
these things are contained under it.
In the former respect every concept has a content,
in the other an extension.

The content and extension of a concept
stand in inverse relation to one another.
The more a concept contains under itself, namely,
the less it contains in itself, and conversely.

Quantity of the extension of concepts

The more the things that stand under a concept
and can be thought through it,
the greater is its extension or sphere.

Higher and lower concepts

Concepts are called higher (conceptus superiores) insofar as
they have other concepts under themselves,
which, in relation to them, are called lower concepts.
A mark of a mark, a remote mark, is a higher concept,
the concept in relation to a remote mark is a lower one.

Genus and species

The higher concept, in respect to its lower one, is called genus,
the lower concept in regard to its higher one species.
Like higher and lower concepts, genus and species concepts are
distinguished not as to their nature, then,
but only in regard to their relation to one another
(termini a quo or ad quod) in logical subordination.

Highest genus and lowest species

The highest genus is that which is not a species
(genus summum non est species),
just as the lowest species is that which is not a genus
(species, quae non est genus, est infima).
In consequence of the law of continuity, however,
there cannot be either a lowest or a next species.

Broader and narrower concept - Convertible concepts

The higher concept is also called a broader concept,
the lower concept a narrower one.

Concepts that have one and the same sphere are called convertible concepts
(conceptus reciproci).

Relation of the lower concept to the higher,
of the broader to the narrower

The lower concept is not contained in the higher,
for it contains more in itself than does the higher one;
it is contained under it, however,
because the higher contains the ground of cognition of the lower.

Furthermore, one concept is not broader than another
because it contains more under itself
[for one cannot know that]
but rather insofar as it contains under itself
the other concept and besides this still more.

Universal rules in respect of the subordination of concepts

In regard to the logical extension of concepts,
the following universal rules hold:
1. What belongs to or contradicts higher concepts
also belongs to or contradicts all lower concepts
that are contained under those higher ones; and
2. conversely: What belongs to or contradicts all lower concepts
also belongs to or contradicts their higher concept.

Conditions for higher and lower concepts to arise:
Logical abstraction and logical determination

Through continued logical abstraction
higher and higher concepts arise,
just as through continued logical determination,
on the other hand,
lower and lower concepts arise.
The greatest possible abstraction yields
the highest or most abstract concept,
that from which no determination can be further thought away.
The highest, completed determination would yield
a thoroughly determinate concept
(conceptus omnimode determinate),
i.e., one to which no further determination might be added in thought.

Use of concepts in abstracto and in concreto

Every concept can be used universally or particularly
(in abstracto or in concreto).
The lower concept is used in abstracto in regard to its higher one,
the higher concept in concreto in regard to its lower one.
