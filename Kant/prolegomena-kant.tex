Prolegomena to any future metaphysics
that will be able to come forward as science

Contents

Preface

Preamble on the Distinguishing Feature of All Metaphysical Cognition

    On the sources of metaphysics
    On the type of cognition,
    that alone can be called metaphysical
        a. On the distinction between
        synthetic and analytic judgments in general
        b. The common principle of all analytic judgments is
        the principle of contradiction
        c. Synthetic judgments require a principle other than
        the principle of contradiction
    Note on the general division of judgments
    into analytic and synthetic

General Question of the Prolegomena:

    Is metaphysics possible at all?

General Question:

    How is cognition from pure reason possible?

Main Transcendental Question, First Part:

    How is pure mathematics possible?
    Note I
    Note II
    Note III

Main Transcendental Question, Second Part:

    How is pure natural science possible?

    Tables: Logical table of judgments
        Transcendental table of
        concepts of the understanding
        Pure physiological table of
        universal principles of natural science
    How is nature itself possible?
    Appendix to pure natural science: On the system of categories

Main Transcendental Question, Third Part:

    How is metaphysics in general possible?

    Preliminary remark on the Dialectic of Pure Reason
    I. Psychological ideas
    II. Cosmological ideas
    III. Theological ideas
    General note to the transcendental ideas
    Conclusion: On determining the boundary of pure reason

Solution to the General Question of the Prolegomena:

    How is metaphysics possible as science?

Appendix: On what Can Be Done in Order to
Make Metaphysics as Science Actual
    Specimen of a judgment about the Critique
    which precedes the investigation
    Proposal for an investigation of the Critique,
    after which the judgment can follow

Preface

These prolegomena are not for the use of apprentices, but of future
teachers, and indeed are not to help them to organize the presentation
of an already existing science, but to discover this science itself for the
first time.

There are scholars for whom the history of philosophy
(ancient as well as modern) is itself their philosophy;
the present prolegomena have not been written for them.
They must wait until those who endeavor to
draw from the wellsprings of reason itself have finished their business,
and then it will be their turn to bring news of these events to the world.
Otherwise, in their opinion nothing can be said that has not already
been said before; and in fact this opinion can stand for all time as an
infallible prediction, for since the human understanding has wandered
over countless subjects in various ways through many centuries, it can
hardly fail that for anything new something old should be found that has
some similarity to it.

My intention is to convince all of those who find it worthwhile to
occupy themselves with metaphysics that it is unavoidably necessary to
suspend their work for the present, to consider all that has happened until
now as if it had not happened, and before all else to pose the question:
“whether such a thing as metaphysics is even possible at all.”

If metaphysics is a science, why is it that it cannot, like other sciences,
attain universal and lasting acclaim? If it is not, how does it happen
that, under the pretense of a science it incessantly shows off, and strings
along the human understanding with hopes that never dim but are never
fulfilled? Whether, therefore, we demonstrate our knowledge or our
ignorance, for once we must arrive at something certain concerning the
nature of this self-proclaimed science; for things cannot possibly remain
on their present footing. It seems almost laughable that, while every other
science makes continuous progress, metaphysics, which desires to be
wisdom itself, and which everyone consults as an oracle, perpetually turns
round on the same spot without coming a step further. Furthermore,
it has lost a great many of its adherents, and one does not find that
those who feel strong enough to shine in other sciences wish to risk
their reputations in this one, where anyone, usually ignorant in all other
things, lays claim to a decisive opinion, since in this region there are
in fact still no reliable weights and measures with which to distinguish
profundity from shallow babble.

It is, after all, not completely unheard of, after long cultivation of a
science, that in considering with wonder how much progress has been
made someone should finally allow the question to arise: whether and
how such a science is possible at all. For human reason is so keen on
building that more than once it has previously erected a tower, but has
afterwards torn it down again in order to see how well constituted its
foundation may have been. It is never too late to grow reasonable and
wise; but if the insight comes late, it is always harder to bring it into
play.

To ask whether a science might in fact be possible assumes a doubt
about its actuality. a Such a doubt, though, offends everyone whose entire
belongings may perhaps consist in this supposed jewel; hence he who
allows this doubt to develop had better prepare for opposition from all
sides. Some, with their metaphysical compendia in hand, will look down
on him with scorn, in proud consciousness of their ancient, and hence
ostensibly legitimate, possession; others, who nowhere see anything that
is not similar to something they have seen somewhere else before, will
not understand him; and for a time everything will remain as if nothing at
all had happened that might yield fear or hope of an impending change.

Nevertheless I venture to predict that the reader of these prolegom-
ena who thinks for himself will not only come to doubt his previous
science, but subsequently will be fully convinced that there can be no
such science unless the requirements expressed here, on which its pos-
sibility rests, are met, and, as this has never yet been done, that there is
as yet no metaphysics at all. Since, however, the demand for it can never
be exhausted, ∗ because the interest of human reason in general is much
too intimately interwoven with it, the reader will admit that a complete
reform or rather a rebirth of metaphysics, according to a plan completely
unknown before now, is inevitably approaching, however much it may
be resisted in the meantime.

Since the Essays of Locke and Leibniz, or rather since the rise of
metaphysics as far as the history of it reaches, no event has occurred that
could have been more decisive with respect to the fate of this science
than the attack made upon it by David Hume. He brought no light to
this kind of knowledge, c but he certainly struck a spark from which a
light could well have been kindled, if it had hit some welcoming tinder
whose glow had then been carefully kept going and made to grow.

Hume started mainly from a single but important concept in meta-
physics, namely, that of the connection a of cause and effect (and of course
also its derivative concepts, of force and action, etc.), and called upon
reason, which pretends to have generated this concept in her womb, to
give him an account of by what right she thinks: that something could
be so constituted that, if it is posited, something else necessarily must
thereby be posited as well; for that is what the concept of cause says.
He undisputably proved that it is wholly impossible for reason to think
such a connection b a priori and from concepts, because this connec-
tion contains necessity; and it is simply not to be seen how it could be,
that because something is, something else necessarily must also be, and
therefore how the concept of such a connection c could be introduced a
priori. From this he concluded that reason completely and fully deceives
herself with this concept, falsely taking it for her own child, when it is
really nothing but a bastard of the imagination, which, impregnated by
experience, and having brought certain representations under the law of
association, passes off the resulting subjective necessity (i.e., habit) for
an objective necessity (from insight). 5 From which he concluded that
reason has no power at all to think such connections, not even merely
in general, because its concepts would then be bare fictions, and all of
its cognitions allegedly established a priori would be nothing but falsely
marked ordinary experiences; which is as much as to say that there is no
metaphysics at all, and cannot be any.

As premature and erroneous as his conclusion was, nevertheless it was
at least founded on inquiry, and this inquiry was of sufficient value, that
the best minds of his time might have come together to solve (more
happily if possible) the problem in the sense in which he presented it,
from which a complete reform of the science must soon have arisen.

But fate, ever ill-disposed toward metaphysics, would have it that
Hume was a understood by no one. One cannot, without feeling a
certain pain, behold how utterly and completely his opponents, Reid,
Oswald, Beattie, and finally Priestley, missed the point of his problem, and
misjudged his hints for improvement – constantly taking for granted just
what he doubted, and, conversely, proving with vehemence and, more
often than not, with great insolence exactly what it had never entered
his mind to doubt – so that everything remained in its old condition, as
if nothing had happened. The question was not, whether the concept of
cause is right, useful, and, with respect to all cognition of nature, indis-
pensable, for this Hume had never put in doubt; it was rather whether
it is thought through reason a priori, and in this way has an inner truth
independent of all experience, and therefore also a much more widely
extended use which is not limited merely to objects of experience: re-
garding this Hume awaited enlightenment. The discussion was only
about the origin of this concept, not about its indispensability in use; if
the former were only discovered, the conditions of its use and the sphere
in which it can be valid would already be given.

In order to do justice to the problem, however, the opponents of this
celebrated man would have had to penetrate very deeply into the nature
of reason so far as it is occupied solely with pure thought, something
that did not suit them. They therefore found a more expedient means to
be obstinate without any insight, namely, the appeal to ordinary common
sense. It is in fact a great gift from heaven to possess right (or, as it
has recently been called, plain) c common sense. But it must be proven
through deeds, by the considered and reasonable things one thinks and
says, and not by appealing to it as an oracle when one knows of nothing
clever to advance in one’s defense. To appeal to ordinary common sense
when insight and science run short, and not before, is one of the subtle
discoveries of recent times, whereby the dullest windbag can confidently
take on the most profound thinker and hold his own with him. So long
as a small residue of insight remains, however, one would do well to
avoid resorting to this emergency help. And seen in the light of day,
this appeal is nothing other than a call to the judgment of the multitude;
applause at which the philosopher blushes, but at which the popular wag
becomes triumphant and defiant. I should think, however, that Hume
could lay just as much claim to sound common sense d as Beattie, and on
top of this to something that the latter certainly did not possess, namely,
a critical reason, which keeps ordinary common sense in check, so that
it doesn’t lose itself in speculations, or, if these are the sole topic of
discussion, doesn’t want to decide anything, since it doesn’t understand
the justification for its own principles; for only so will it remain sound
common sense. Hammer and chisel are perfectly fine for working raw
lumber, but for copperplate one must use an etching needle. 9 Likewise,
sound common sense and speculative understanding are both useful,
but each in its own way; the one, when it is a matter of judgments that
find their immediate application in experience, the other, however, when
judgments are to be made in a universal mode, out of mere concepts, as
in metaphysics, where what calls itself (but often per antiphrasin) a sound
common sense has no judgment whatsoever.

I freely admit that the remembrance of David Hume was the very
thing that many years ago first interrupted my dogmatic slumber 10 and
gave a completely different direction to my researches in the field of
speculative philosophy. I was very far from listening to him with respect
to his conclusions, which arose solely because he did not completely set
out his problem but only touched on a part of it, which, without the
whole being taken into account, can provide no enlightenment. b If we
begin from a well-grounded though undeveloped thought that another
bequeaths us, then we can well hope, by continued reflection, to take it
further than could the sagacious man whom one has to thank for the first
spark of this light.

So I tried first whether Hume’s objection might not be presented in a
general manner, and I soon found that the concept of the connection of
cause and effect is far from being the only concept through which the
understanding thinks connections of things a priori; rather, metaphysics
consists wholly of such concepts. I sought to ascertain their number, and
once I had successfully attained this in the way I wished, namely from
a single principle, I proceeded to the deduction of these concepts, from
which I henceforth became assured that they were not, as Hume had
feared, derived from experience, but had arisen from the pure under-
standing. This deduction, which appeared impossible to my sagacious
predecessor, and which had never even occurred to anyone but him, even
though everyone confidently made use of these concepts without asking
what their objective validity is based on – this deduction, I say, was the
most difficult thing that could ever be undertaken on behalf of meta-
physics; and the worst thing about it is that metaphysics, as much of it
as might be present anywhere at all, could not give me the slightest help
with this, because this very deduction must first settle the possibility of
a metaphysics. As I had now succeeded in the solution of the Humean
problem not only in a single case but with respect to the entire faculty of
pure reason, I could therefore take sure, if still always slow, steps toward
finally determining, completely and according to universal principles,
the entire extent of pure reason with regard to its boundaries as well as
its content, which was indeed the very thing that metaphysics requires
in order to build its system according to a sure plan.

But I fear that the elaboration of the Humean problem in its greatest
possible amplification (namely, the Critique of Pure Reason) may well fare
just as the problem itself fared when it was first posed. It will be judged in-
correctly, because it is not understood; it will not be understood, because
people will be inclined just to skim through the book, but not to think
through it; and they will not want to expend this effort on it, because
the work is dry, because it is obscure, because it opposes all familiar con-
cepts and is long-winded as well. Now I admit that I do not expect to hear
complaints from a philosopher regarding lack of popularity, entertain-
ment, and ease, when the matter concerns the existence of highly prized
knowledge that is indispensable to humanity, knowledge that cannot be
constituted except according to the strictest rules of scholarly exactitude,
and to which popularity may indeed come with time but can never be
there at the start. But with regard to a certain obscurity – arising in part
from the expansiveness of the plan, which makes it difficult to survey the
main points upon which the investigation depends – in this respect the
complaint is just; and I will redress it through the present Prolegomena. a

The previous work, which presents the faculty of pure reason in its
entire extent and boundaries, thereby always remains the foundation to
which the Prolegomena refer only as preparatory exercises; for this Critique
must stand forth as science, systematic and complete to its smallest parts,
before one can think of permitting metaphysics to come forward, or even
of forming only a distant hope for metaphysics.

We have long been accustomed to seeing old, threadbare cognitions
newly trimmed by being taken from their previous connections and
fitted out by someone in a systematic garb of his own preferred cut,
but under new titles; and most readers will beforehand expect nothing
else from the Critique. Yet these Prolegomena will bring them to un-
derstand that there exists a completely new science, of which no one
had previously formed so much as the thought, of which even the bare
idea was unknown, and for which nothing from all that has been pro-
vided before now could be used except the hint that Hume’s doubts
had been able to give; Hume also foresaw nothing of any such possi-
ble formal science, but deposited his ship on the beach (of skepticism)
for safekeeping, 11 where it could then lie and rot, whereas it is important
to me to give it a pilot, who, provided with complete sea-charts and a
compass, might safely navigate the ship wherever seems good to him,

To approach a new science – one that is entirely isolated and is the
only one of its kind – with the prejudice that it can be judged by means
of one’s putative cognitions already otherwise obtained, even though
it is precisely the reality of those that must first be completely called
into question, results only in believing that one sees everywhere some-
thing that was already otherwise known, because the expressions per-
haps sound similar; except that everything must seem to be extremely
deformed, contradictory, and nonsensical, because one does not thereby
make the author’s thoughts fundamental, but always simply one’s own,
made natural through long habit. Yet the copiousness of the work, in-
sofar as it is rooted in the science itself and not in the presentation, and
the inevitable dryness and scholastic exactitude that result, are qualities
that indeed may be extremely advantageous to the subject matter itself,
but must of course be detrimental to the book itself.

It is not given to everyone to write so subtly and yet also so alluringly as
David Hume, or so profoundly and at the same time so elegantly as Moses
Mendelssohn; but I could well have given my presentation popularity
(as I flatter myself) if all I had wanted to do was to sketch a plan and to
commend its execution to others, and had I not taken to heart the well-
being of the science that kept me occupied for so long; for after all it
requires great perseverance and also indeed not a little self-denial to set
aside the enticement of an earlier, favorable reception for the expectation
of an admittedly later, but lasting approval.

To make plans is most often a presumptuous, boastful mental preoccu-
pation, through which one presents the appearance of creative genius,
in that one requires what one cannot provide oneself, censures what one
cannot do better, and proposes what one does not know how to attain
oneself – though merely for a sound plan for a general critique of rea-
son, somewhat more than might be expected would already have been
required if it were not, as is usual, to be merely a recitation of idle wishes.
But pure reason is such an isolated domain, within itself so thoroughly
connected, that no part of it can be encroached upon without disturbing
all the rest, nor adjusted without having previously determined for each
part its place and its influence on the others; for, since there is nothing
outside of it that could correct our judgment within it, the validity and
use of each part depends on the relation in which it a stands to the others
within reason itself, and, as with the structure of an organized body, the
purpose of any member can be derived only from the complete concept
of the whole. That is why it can be said of such a critique, that it is never
trustworthy unless it is entirely complete down to the least elements of
pure reason, and that in the domain of this faculty one must determine
and settle either all or nothing.

But although a mere plan that might precede the Critique of Pure
Reason would be unintelligible, undependable, and useless, it is by con-
trast all the more useful if it comes after. For one will thereby be put in
the position to survey the whole, to test one by one the main points at
issue in this science, and to arrange many things in the exposition better
than could be done in the first execution of the work.

Here then is such a plan subsequent to the completed work, which
now can be laid out according to the analytic method, a whereas the work
itself absolutely had to be composed according to the synthetic method, b
so that the science might present all of its articulations, as the structural
organization of a quite peculiar faculty of cognition, in their natural
connection. 13 Whosoever finds this plan itself, which I send ahead as
prolegomena for any future metaphysics, still obscure, may consider
that it simply is not necessary for everyone to study metaphysics, that
there are some talents that proceed perfectly well in fundamental and
even deep sciences that are closer to intuition, but that will not succeed
in the investigation of purely abstract concepts, and that in such a case
one should apply one’s mental gifts to another object; that, however,
whosoever undertakes to judge or indeed to construct a metaphysics,
must thoroughly satisfy the challenge made here, whether it happens that
they accept my solution, or fundamentally reject it and replace it with
another – for they cannot dismiss it; and finally, that the much decried
obscurity (a familiar cloaking for one’s own indolence or dimwittedness)
has its use as well, since everybody, who with respect to all other sciences
observes a wary silence, speaks masterfully, and boldly passes judgment
in questions of metaphysics, because here to be sure their ignorance does
not stand out clearly in relation to the science of others, but in relation
to genuine critical principles, which therefore can be praised:

Preamble on the
Distinguishing Feature of
All Metaphysical Cognition

§ 1

On the sources of metaphysics

If one wishes to present a body of cognition as science, a then one must first
be able to determine precisely the differentia it has in common with no
other science, and which is therefore its distinguishing feature; otherwise
the boundaries of all the sciences run together, and none of them can be
dealt with thoroughly according to its own nature.
Whether this distinguishing feature consists in a difference of the
object or the source of cognition, or even of the type of cognition, or several
if not all of these things together, the idea of the possible science and its
territory depends first of all upon it.
First, concerning the sources of metaphysical cognition, it already
lies in the concept of metaphysics that they cannot be empirical. The
principles b of such cognition (which include not only its fundamental
propositions, c but also its fundamental concepts) must therefore never
be taken from experience; for the cognition is supposed to be not phys-
ical but metaphysical, i.e., lying beyond experience. Therefore it will
be based upon neither outer experience, which constitutes the source
of physics proper, nor inner, which provides the foundation of empirical
psychology. It is therefore cognition a priori, or from pure understanding
and pure reason.
In this, however, there would be nothing
to differentiate it from pure mathematics;
it must therefore be denominated pure philosophical cognition;
but concerning the meaning of this expression
I refer to the Critique of Pure Reason, pp. 712 f.,
where the distinction between these
two types of use of reason has been presented clearly and sufficiently.
So much on the sources of metaphysical cognition.

§ 2

On the type of cognition that alone can be called metaphysical

(a) On the distinction between synthetic and analytic judgments in general

Metaphysical cognition must contain nothing
but judgments a priori, as required by
the distinguishing feature of its sources.
But judgments may have any origin whatsoever,
or be constituted in whatever manner
according to their logical form,
and yet there is nonetheless a distinction
between them according to their content,
by dint of which they are
either merely explicative and add nothing
to the content of the cognition,
or ampliative and augment the given cognition;
the first may be called analytic judgments,
the second synthetic.
Analytic judgments say nothing in the predicate except what was ac-
tually thought already in the concept of the subject, though not so clearly
nor with the same consciousness. If I say: All bodies are extended, then
I have not in the least amplified my concept of body, but have merely
resolved it, since extension, although not explicitly said of the former
concept prior to the judgment, nevertheless was actually thought of it;
the judgment is therefore analytic. By contrast, the proposition: Some
bodies are heavy, contains something in the predicate that is not actu-
ally thought in the general concept of body; it therefore augments my
cognition, since it adds something to my concept, and must therefore be
called a synthetic judgment.

(b) The common principle of all analytic judgments is
the principle of contradiction

All analytic judgments rest entirely on the principle of contradiction
and are by their nature a priori cognitions, whether the concepts that
serve for their material be empirical or not. For since the predicate
of an affirmative analytic judgment is already thought beforehand in
the concept of the subject, it cannot be denied of that subject without
contradiction; exactly so is its opposite necessarily denied of the subject
in an analytic, but negative, judgment, and indeed also according to the
principle of contradiction. So it stands with the propositions: Every body
is extended, and: No body is unextended (simple).
For that reason all analytic propositions are still a priori judgments
even if their concepts are empirical, as in: Gold is a yellow metal; for
in order to know this, I need no further experience outside my concept
of gold, which includes that this body is yellow and a metal; for this
constitutes my very concept, and I did not have to do anything except
analyze it, without looking beyond it to something else.

(c) Synthetic judgments require a principle
other than the principle of contradiction

There are synthetic judgments a posteriori whose origin is empirical; but
there are also synthetic judgments that are a priori certain and that arise
from pure understanding and reason. Both however agree in this, that
they can by no means arise solely from the principle a of analysis, namely
the principle of contradiction; they demand yet a completely different
principle, b though they always must be derived from some fundamen-
tal proposition, c whichever it may be, in accordance with the principle of
contradiction; for nothing can run counter to this principle, even though
everything cannot be derived from it. I shall first classify the synthetic
judgments.

1. Judgments of experience are always synthetic. For it would be ab-
surd to base an analytic judgment on experience, since I do not at all
need to go beyond my concept in order to formulate the judgment and
therefore have no need for any testimony from experience. That a body is
extended, is a proposition that stands certain a priori, and not a judgment
of experience. For before I go to experience, I have all the conditions
for my judgment already in the concept, from which I merely extract
the predicate in accordance with the principle of contradiction, and by
this means can simultaneously become conscious of the necessity of the
judgment, which experience could never teach me.

2. Mathematical judgments are one and all synthetic. This proposition
appears to have completely escaped the observations of analysts of human
reason up to the present, and indeed to be directly opposed to all of their
conjectures, although it is incontrovertibly certain and very important
in its consequences. Because they found that the inferences of the math-
ematicians all proceed in accordance with the principle of contradiction
(which, by nature, is required of any apodictic certainty), they were per-
suaded that even the fundamental propositions were known through the
principle of contradiction, in which they were very mistaken; for a syn-
thetic proposition can of course be discerned in accordance with the
principle of contradiction, but only insofar as another synthetic proposi-
tions is presupposed from which the first can be deduced, never however
in itself.

First of all it must be observed: that properly mathematical proposi-
tions are always a priori and not empirical judgments, because they carry
necessity with them, which cannot be taken from experience. But if this
will not be granted me, very well, I will restrict my proposition to pure
mathematics, the concept of which already conveys that it contains not
empirical but only pure cognition a priori.
One might well at first think: that the proposition 7 + 5 = 12 is a
purely analytical proposition that follows from the concept of a sum
of seven and five according to the principle of contradiction. However,
upon closer inspection, one finds that the concept of the sum of 7 and
5 contains nothing further than the unification of the two numbers into
one, through which by no means is thought what this single number may
be that combines the two. The concept of twelve is in no way already
thought because I merely think to myself this unification of seven and
five, and I may analyze my concept of such a possible sum for as long as
may be, still I will not meet with twelve therein. One must go beyond
these concepts, in making use of the intuition that corresponds to one of
the two, such as one’s five fingers, or (like Segner in his arithmetic) 18 five
points, and in that manner adding the units of the five given in intuition
step by step to the concept of seven. One therefore truly amplifies one’s
concept through this proposition 7 + 5 = 12 and adds to the first concept
a new one that was not thought in it; that is, an arithmetical proposition
is always synthetic, which can be seen all the more plainly in the case
of somewhat larger numbers, for it is then clearly evident that, though
we may turn and twist our concept as we like, we could never find the
sum through the mere analysis of our concepts, without making use of
intuition.

Nor is any fundamental proposition of pure geometry analytic. That
the straight line between two points is the shortest is a synthetic propo-
sition. For my concept of the straight contains nothing of magnitude, a
but only a quality. The concept of the shortest is therefore wholly an
addition and cannot be extracted by any analysis from the concept of the
straight line. Intuition must therefore be made use of here, by means of
which alone the synthesis is possible.
Some other fundamental propositions that geometers presuppose
are indeed actually analytic and rest on the principle of contradic-
tion; however, they serve only, like identical propositions, as links in
the chain of method and not as b principles: e.g., a = a, the whole is
equal to itself, or (a + b) > a, i.e., the whole is greater than its part.
And indeed even these, although they are valid from concepts alone,
are admitted into mathematics only because they can be exhibited in
intuition.

It is merely ambiguity of expression which makes us commonly believe
here that the predicate of such apodictic judgments already lies in our
concept and that the judgment is therefore analytic. Namely, we are re-
quired b to add in thought a particular predicate to a given concept, and
this necessity is already attached to the concepts. But the question is not,
what we are required to add in thought to a given concept, but what we
actually think in it, c even if only obscurely, and then it becomes evident
that the predicate attaches to such concepts indeed necessarily, though
not immediately, but rather through an intuition that has to be added.
The essential feature of pure mathematical cognition, differentiating
it from all other a priori cognition, is that it must throughout proceed not
from concepts, but always and only through the construction of concepts
(Critique, p. 713). 19 Because pure mathematical cognition, in its propo-
sitions, must therefore go beyond the concept to that which is contained
in the intuition corresponding to it, its propositions can and must never
arise through the analysis of concepts, i.e., analytically, and so are one
and all synthetic.

I cannot, however, refrain from noting the damage that neglect of
this otherwise seemingly insignificant and unimportant observation has
brought upon philosophy. Hume, when he felt the call, worthy of a
philosopher, to cast his gaze over the entire field of pure a priori cognition,
in which the human understanding claims such vast holdings, inadver-
tently lopped off a whole (and indeed the most considerable) province
of the same, namely pure mathematics, by imagining that the nature and
so to speak the legal constitution of this province rested on completely
different principles, namely solely on the principle of contradiction; and
although he had by no means made a classification of propositions so
formally and generally, or with such nomenclature, as I have here, it
was nonetheless just as if he had said: Pure mathematics contains only
analytic propositions, but metaphysics contains synthetic propositions a
priori. Now he erred severely in this, and this error had decisively dam-
aging consequences for his entire conception. For had he not done this,
he would have expanded his question about the origin of our synthetic
judgments far beyond his metaphysical concept of causality and extended
it also to the possibility of a priori mathematics; for he would have had
to accept mathematics as synthetic as well. But then he would by no
means have been able to found his metaphysical propositions on mere
experience, for otherwise he would have had to subject the axioms of
pure mathematics to experience as well, which he was much too reason-
able to do. 21 The good company in which metaphysics would then have
come to be situated would have secured it against the danger of scornful
mistreatment; for the blows that were intended for the latter would have
had to strike the former as well, which was not his intention, and could
not have been; and so the acute man would have been drawn into re-
flections which must have been similar to those with which we are now
occupied, but which would have gained infinitely from his inimitably
fine presentation.

3. a Properly metaphysical judgments are one and all synthetic. Judg-
ments belonging to metaphysics must be distinguished from properly meta-
physical judgments. Very many among the former are analytic, but they
merely provide the means to metaphysical judgments, toward which the
aim of the science is completely directed, and which are always synthetic.
For if concepts belong to metaphysics, e.g., that of substance, then nec-
essarily the judgments arising from their mere analysis belong to meta-
physics as well, e.g., substance is that which exists only as subject, etc.,
and through several such analytic judgments we try to approach the def-
inition of those concepts. Since, however, the analysis of a pure concept
of the understanding (such as metaphysics contains) does not proceed in
a different manner from the analysis of any other, even empirical, con-
cept which does not belong to metaphysics (e.g., air is an elastic fluid, the
elasticity of which is not lost with any known degree of cold), therefore
the concept may indeed be properly metaphysical, but not the analytic
judgment; for this science possesses something special and proper to it in
the generation of its a priori cognitions, which generation must therefore
be distinguished from what this science has in common with all other
cognitions of the understanding; thus, e.g., the proposition: All that is
substance in things persists, is a synthetic and properly metaphysical
proposition.
If one has previously assembled, according to fixed principles, the a
priori concepts that constitute the material of metaphysics and its tools,
then the analysis of these concepts is of great value; it can even be pre-
sented apart from all the synthetic propositions that constitute meta-
physics itself, as a special part (as it were a philosophia definitiva) 23 contain-
ing nothing but analytic propositions belonging to metaphysics. For in
fact such analyses do not have much use anywhere except in metaphysics,
i.e., with a view toward the synthetic propositions that are to be generated
from such previously analyzed concepts.
The conclusion of this section is therefore: that metaphysics properly
has to do with synthetic propositions a priori, and these alone constitute
its aim, for which it indeed requires many analyses of its concepts (there-
fore many analytic judgments), in which analyses, though, the procedure
is no different from that in any other type of cognition when one seeks
simply to make its concepts clear through analysis. But the generation
of cognition a priori in accordance with both intuition and concepts,
ultimately of synthetic propositions a priori as well, and specifically in
philosophical cognition, forms a the essential content of metaphysics.

§ 3

Note on the general division of judgments
into analytic and synthetic

This division is indispensable with regard to the critique of human un-
derstanding, and therefore deserves to be classical in it; other than that I
don’t know that it has much utility anywhere else. And in this I find the
reason why dogmatic philosophers (who always sought the sources of
metaphysical judgments only in metaphysics itself, and not outside it in
the pure laws of reason in general) neglected this division, which appears
to come forward of itself, and, like the famous Wolf, or the acute Baum-
garten following in his footsteps, could try to find the proof of the prin-
ciple of sufficient reason, which obviously is synthetic, in the principle of
contradiction. By contrast I find a hint of this division already in Locke’s
essays on human understanding. For in Book IV, Chapter III, §9 f., after
he had already discussed the various connections of representations in
judgments and the sources of the connections, of which he located the
one in identity or contradiction (analytic judgments) but the other in the
existence of representations in a subject (synthetic judgments), he then
acknowledges in §10 that our cognition (a priori) of these last is very con-
stricted and almost nothing at all. But there is so little that is definite
and reduced to rules in what he says about this type of cognition, that it
is no wonder if no one, and in particular not even Hume, was prompted
by it to contemplate propositions of this type. 26 For such general yet
nonetheless definite principles are not easily learned from others who
have only had them floating obscurely before them. One must first have
come to them oneself through one’s own reflection, after which one also
finds them elsewhere, where one certainly would not have found them
before, because the authors did not even know themselves that their
own remarks were grounded on such an idea. Those who never think for
themselves in this way nevertheless possess the quick-sightedness to spy
everything, after it has been shown to them, in what has already been
said elsewhere, where no one at all could see it before.

General Question of the Prolegomena

Is metaphysics possible at all?

§ 4

If a metaphysics that could assert itself as science were actual, if one could
say: here is metaphysics, you need only to learn it, and it will convince
you of its truth irresistibly and immutably, then this question would be
unnecessary, and there would remain only that question which would
pertain more to a test of our acuteness than to a proof of the existence of
the subject matter itself, namely: how it is possible, and how reason should
set about attaining it. Now it has not gone so well for human reason in
this case. One can point to no single book, as for instance one presents
a Euclid, and say: this is metaphysics, here you will find the highest aim
of this science, knowledge a of a supreme being and a future life, proven
from principles of pure reason. For one can indeed show us many propo-
sitions that are apodictically certain and have never been disputed; but
they are one and all analytic and pertain more to the materials and imple-
ments of metaphysics than to the expansion of knowledge, which after
all ought to be our real aim for it. (§2c) But although you present syn-
thetic propositions as well (e.g., the principle of sufficient reason), which
you have never proven from bare reason and consequently a priori, as
was indeed your obligation, and which are gladly ceded to you all the
same: then if you want to use them toward your main goal, you still fall
into assertions so illicit and precarious that one metaphysics has always
contradicted the other, either in regard to the assertions themselves or
their proofs, and thereby metaphysics has itself destroyed its claim to
lasting approbation. The very attempts to bring such a science into exis-
tence were without doubt the original cause of the skepticism that arose
so early, 27 a mode of thinking in which reason moves against itself with
such violence that it never could have arisen except in complete despair
as regards satisfaction of reason’s most important aims. For long before
we began to question nature methodically, we questioned just our iso-
lated reason, which already was practiced to a certain extent through
common experience: for reason surely is present to us always, but laws
of nature must normally be sought out painstakingly; and so metaphysics
was floating at the top like foam, though in such a way that as soon as
what had been drawn off had dissolved, more showed itself on the sur-
face, which some always gathered up eagerly, while others, instead of
seeking the cause of this phenomenon in the depths, thought themselves
wise in mocking the fruitless toil of the former.

Weary therefore of dogmatism, which teaches us nothing, and also
of skepticism, which promises us absolutely nothing at all, not even
the tranquility of a permitted ignorance; summoned by the importance
of the knowledge b that we need, and made mistrustful, through long
experience, with respect to any knowledge that we believe we possess or
that offers itself to us under the title of pure reason, there remains left
for us but one critical question, the answer to which can regulate our
future conduct: Is metaphysics possible at all ? But this question must not
be answered by skeptical objections to particular assertions of an actual
metaphysics (for at present we still allow none to be valid), but out of
the still problematic concept of such a science.

In the Critique of Pure Reason I worked on this question synthetically,
namely by inquiring within pure reason itself, and seeking to determine
within this source both the elements and the laws of its pure use, accord-
ing to principles. This work is difficult and requires a resolute reader to
think himself little by little into a system that takes no foundation as given
except reason itself, and that therefore tries to develop cognition out of its
original seeds without relying on any fact whatever. Prolegomena e should
by contrast be preparatory exercises; they ought more to indicate what
needs to be done in order to bring a science into existence if possible,
than to present the science itself. They must therefore rely on something
already known to be dependable, from which we can go forward with
confidence and ascend to the sources, which are not yet known, and
whose discovery not only will explain what is known already, but will
also exhibit an area with many cognitions that all arise from these same
sources. The methodological procedure of prolegomena, and especially
of those that are to prepare for a future metaphysics, will therefore be
analytic.

Fortunately, it happens that, even though we cannot assume that meta-
physics as science is actual, we can confidently say that some pure syn-
thetic cognition a priori is actual and given, namely, pure mathematics
and pure natural science; for both contain propositions that are fully ac-
knowledged, some as apodictically certain through bare reason, some
from universal agreement with experience (though these are still rec-
ognized as independent of experience). We have therefore some at least
uncontested synthetic cognition a priori, and we do not need to ask whether
it is possible (for it is actual), but only: how it is possible, in order to be able
to derive, from the principle of the possibility of the given cognition, the
possibility of all other synthetic cognition a priori.

Prolegomena

General Question

How is cognition from pure reason possible?

§ 5

We have seen above the vast difference between analytic and synthetic
judgments. The possibility of analytic propositions could be compre-
hended very easily; for it is founded solely upon the principle of con-
tradiction. The possibility of synthetic propositions a posteriori, i.e., of
such as are drawn from experience, also requires no special explana-
tion; for experience itself is nothing other than a continual conjoin-
ing (synthesis) of perceptions. There remain for us therefore only syn-
thetic propositions a priori, whose possibility must be sought or in-
vestigated, since it must rest on principles other than the principle of
contradiction.

Here, however, we do not need first to seek the possibility of such
propositions, i.e., to ask whether they are possible. For there are plenty
of them actually given, and indeed with indisputable certainty, and since
the method we are now following is to be analytic, we will consequently
start from the position: that such synthetic but pure rational cognition is
actual; but we must nonetheless next investigate the ground of this possi-
bility, and ask: how this cognition is possible, so that we put ourselves in
a position to determine, from the principles of its possibility, the condi-
tions of its use and the extent and boundaries of the same. 28 Expressed
with scholastic precision, the exact problem on which everything hinges
is therefore:

How are synthetic propositions a priori possible?

For the sake of popularity I have expressed this problem somewhat dif-
ferently above, namely as a question about cognition from pure reason
which I could well have done on this occasion without disadvantage for
the desired insight; for, since we assuredly have to do here only with
metaphysics and its sources, it will, I hope, always be kept in mind, fol-
lowing the earlier reminders, that when we here speak of cognition from
pure reason, the discussion is never about analytic cognition, but only
synthetic.

Whether metaphysics is to stand or fall, and hence its existence, now
depends entirely on the solving of this problem. Anyone may present his
contentions on the matter with ever so great a likelihood, piling conclu-
sion on conclusion to the point of suffocation; if he has not been able
beforehand to answer this question satisfactorily then I have the right
to say: it is all empty, baseless philosophy and false wisdom. You speak
through pure reason and pretend as it were to create a priori cognitions,
not only by analyzing given concepts, but by alleging new connections
that are not based on the principle of contradiction and that you nonethe-
less presume to understand completely independently of all experience;
now how do you come to this, and how will you justify such pretenses?
You cannot be allowed to call on the concurrence of general common
sense; b for that is a witness whose standing is based solely on public
rumor.

As indispensable as it is, however, to answer this question, at the same
time it is just as difficult; and although the principal reason why the an-
swer has not long since been sought rests in the fact that it had occurred
to no one that such a thing could be asked, nonetheless a second reason
is that a satisfactory answer to this one question requires more assidu-
ous, deeper, and more painstaking reflection than the most prolix work
of metaphysics ever did, which promised its author immortality on its
first appearance. Also, every perceptive reader, if he carefully ponders
what this problem demands, being frightened at first by its difficulty, is
bound to consider it insoluble and, if such pure synthetic cognitions a
priori were not actual, altogether impossible; which is what actually befell
David Hume, although he was far from conceiving the question in such
universality as it is here, and as it must be if the reply is to be decisive for
all metaphysics. For how is it possible, asked the acute man, that when
I am given one concept I can go beyond it and connect another one to
it that is not contained in it, and can indeed do so, as though the latter
necessarily belonged to the former? Only experience can provide us with
such connections (so he concluded from this difficulty, which he took
for an impossibility), and all of this supposed necessity – or, what is the
same – this cognition taken for a priori, is nothing but a long-standing
habit of finding something to be true and consequently of taking sub-
jective necessity to be objective.

If the reader complains about the toil and trouble that I shall give
him with the solution to this problem, he need only make the attempt
to solve it more easily himself. Perhaps he will then feel himself obliged
to the one who has taken on a task of such profound inquiry for him,
and will rather allow himself to express some amazement over the ease
with which the solution could still be given, considering the nature of
the matter; for indeed it cost years of toil to solve this problem in its full
universality a (as this word is understood by the mathematicians, namely,
as sufficient for all cases), and also ultimately to be able to present it in
analytic form, as the reader will find it here.

All metaphysicians are therefore solemnly and lawfully suspended
from their occupations until such a time as they shall have satisfactorily
answered the question: How are synthetic cognitions a priori possible? For in
this answer alone consists the credential which they must present if they
have something to advance to us in the name of pure reason; in default
of which, however, they can expect only that reasonable persons, who
have been deceived so often already, will reject their offerings without
any further investigation.

If, on the contrary, they want to put forth their occupation not as
science, but as an art of beneficial persuasions accommodated to general
common sense, then they cannot justly be barred from this trade. They
will then use the modest language of reasonable belief, they will acknowl-
edge that it is not allowed them even once to guess, let alone to know, b
something about that which lies beyond the boundaries of all possible
experience, but only to assume something about it (not for speculative
use, for they must renounce that, but solely for practical use), as is pos-
sible and even indispensable for the guidance of the understanding and
will in life. Only thus will they be able to call themselves useful and wise
men, the more so, the more they renounce the name of metaphysicians;
for metaphysicians want to be speculative philosophers, and since one
cannot aim for vapid probabilities when judgments a priori are at stake
(for what is alleged to be cognized a priori is thereby announced as neces-
sary), it cannot be permitted them to play with guesses, but rather their
assertions must be science or they are nothing at all.

t can be said that the whole of transcendental philosophy, which nec-
essarily precedes all of metaphysics, is itself nothing other than simply
the complete solution of the question presented here, but in systematic
order and detail, and that until now there has therefore been no tran-
scendental philosophy; for what goes under this name is really a part of
metaphysics, but this science is to settle the possibility of metaphysics
in the first place, and therefore must precede all metaphysics. Hence
there need be no surprise because a science is needed that is utterly de-
prived of assistance from other sciences and hence is in itself completely
new, in order just to answer a single question adequately, when the so-
lution to it is conjoined with trouble and difficulty and even with some
obscurity.

In now setting to work on this solution – and indeed following the
analytic method, in which we presuppose that such cognitions from pure
reason are actual – we can appeal to only two sciences of theoretical knowl-
edge (which alone is being discussed here), namely, pure mathematics and
pure natural science; for only these can present objects to us in intuition,
and consequently, if they happen to contain an a priori cognition, can
show its truth or correspondence with the object in concreto, i.e., its ac-
tuality, from which one could then proceed along the analytic path to
the ground of its possibility. This greatly facilitates the work, in which
general considerations are not only applied to facts, but even start from
them, instead of, as in the synthetic procedure, having to be derived
wholly in abstracto from concepts.

But in order to ascend from these pure a priori cognitions (which are
not only actual but also well-founded) to a possible cognition that we
seek – namely, a metaphysics as science – we need to comprehend un-
der our main question that which gives rise to metaphysics and which
underlies its purely naturally given (though not above suspicion as re-
gards truth) cognition a priori (which cognition, when pursued without
any critical investigation of its possibility, is normally called metaphysics
already) – in a word, the natural disposition to such a science; and so the
main transcendental question, divided into four other questions, will be
answered step by step:

1. How is pure mathematics possible?
2. How is pure natural science possible?
3. How is metaphysics in general possible?
4. How is metaphysics as science possible?

It can be seen that even if the solution to these problems is intended
principally to present the essential content of the Critique, still it also
possesses something distinctive that is worthy of attention in its own
right, namely, the search for the sources of given sciences in reason itself,
in order to investigate and to measure out for reason, by way of the deed
itself, its power to cognize something a priori; whereby these sciences
themselves then benefit, if not with respect to their content, nonetheless
as regards their proper practice, and, while bringing light to a higher
question regarding their common origin, they simultaneously provide
occasion for a better explanation of their own nature.

Main Transcendental Question,

First Part

How is pure mathematics possible?

§ 6

Here now is a great and proven body of cognition, a which is already of
admirable extent and promises unlimited expansion in the future, which
carries with it thoroughly apodictic certainty (i.e., absolute necessity),
hence rests on no grounds of experience, and so is a pure product of rea-
son, but beyond this is thoroughly synthetic. “How is it possible then for
human reason to achieve such cognition wholly a priori?” Does not this
capacity, since it is not, and cannot be, based on experience, presuppose
some a priori basis for cognition, which lies deeply hidden, but which
might reveal itself through these its effects, if their first beginnings were
but diligently tracked down?

§ 7

We find, however, that all mathematical cognition has this distinguishing
feature, that it must present its concept beforehand in intuition and in-
deed a priori, consequently in an intuition that is not empirical but pure,
without which means it cannot take a single step; therefore its judg-
ments are always intuitive, b in the place of which philosophy can content
itself with discursive judgments from mere concepts, and can indeed exem-
plify its apodictic teachings through intuition c but can never derive them
from it. This observation with respect to the nature of mathematics al-
ready guides us toward the first and highest condition of its possibility;
namely, it must be grounded in some pure intuition or other, in which it
can present, or, as one calls it, construct all of its concepts in concreto yet
a priori. ∗ If we could discover this pure intuition and its possibility, then
from there it could easily be explained how synthetic a priori propositions
are possible in pure mathematics, and consequently also how this science
itself is possible; for just as empirical intuition makes it possible for us,
without difficulty, to amplify (synthetically in experience) the concept we
form of an object of intuition through new predicates that are presented
by intuition itself, so too will pure intuition do the same only with this
difference: that in the latter case the synthetic judgment will be a priori
certain and apodictic, but in the former only a posteriori and empirically
certain, because the former contains only what is met with in contingent
empirical intuition, while the latter contains what necessarily must be
met with in pure intuition, since it is, as intuition a priori, inseparably
bound with the concept before all experience or individual perception.

§ 8

But with this step the difficulty seems to grow rather than to diminish.
For now the question runs: How is it possible to intuit something a priori? An
intuition is a representation of the sort which would depend immediately
on the presence of an object. It therefore seems impossible originally to
intuit a priori, since then the intuition would have to occur without an
object being present, either previously or now, to which it could refer,
and so it could not be an intuition. Concepts are indeed of the kind that
we can quite well form some of them for ourselves a priori (namely, those
that contain only the thinking of an object in general) without our being
in an immediate relation to an object, e.g., the concept of quantity, of
cause, etc.; but even these still require, in order to provide them with
signification and sense, a certain use in concreto, i.e., application to some
intuition or other, by which an object for them is given to us. But how
can the intuition of an object precede the object itself?

§ 9

If our intuition had to be of the kind that represented things as they are
in themselves, then absolutely no intuition a priori would take place, but it
would always be empirical. For I can only know what may be contained in
the object in itself if the object is present and given to me. Of course, even
then it is incomprehensible how the intuition of a thing that is present
should allow me to cognize it the way it is in itself, since its properties
cannot migrate over into my power of representation; but even granting
such a possibility, the intuition still would not take place a priori, i.e.,
before the object were presented to me, for without that no basis for
the relation of my representation to the object can be conceived; so it
would have to be based on inspiration. There is therefore only one way
possible for my intuition to precede the actuality of the object and occur
as an a priori cognition, namely if it contains nothing else except the form of
sensibility, which in me as subject precedes all actual impressions through which
I am affected by objects. For I can know a priori that the objects of the
senses can be intuited only in accordance with this form of sensibility.
From this it follows: that propositions which relate merely to this form of
sensory intuition will be possible and valid for objects of the senses; also,
conversely, that intuitions which are possible a priori can never relate to
things other than objects of our senses.

§ 10

Therefore it is only by means of the form of sensory intuition that we
can intuit things a priori, though by this means we can cognize objects
only as they appear to us (to our senses), not as they may be in themselves;
and this supposition is utterly necessary, if synthetic propositions a priori
are to be granted as possible, or, in case they are actually encountered,
if their possibility is to be conceived and determined in advance.
Now space and time are the intuitions upon which pure mathemat-
ics bases all its cognitions and judgments, which come forward as at
once apodictic and necessary; for mathematics must first exhibit all of
its concepts in intuition – and pure mathematics in pure intuition – i.e.,
it must first construct them, failing which (since mathematics cannot
proceed analytically, namely, through the analysis of concepts, but only a
synthetically) it is impossible for it to advance a step, that is, as long as
it lacks pure intuition, in which alone the material b for synthetic judg-
ments a priori can be given. Geometry bases itself on the pure intuition
of space. Even arithmetic forms its concepts of numbers through succes-
sive addition of units in time, but above all pure mechanics can form its
concepts of motion only by means of the representation of time. 30 Both
representations are, however, merely intuitions; for, if one eliminates
from the empirical intuitions of bodies and their alterations (motion)
everything empirical, that is, that which belongs to sensation, then space
and time still remain, which are therefore pure intuitions that underlie
a priori the empirical intuitions, and for that reason can never themselves
be eliminated; but, by the very fact that they are pure intuitions a priori,
they prove that they are mere forms of our sensibility that must precede
all empirical intuition (i.e., the perception of actual objects), and in ac-
cordance with which objects can be cognized a priori, though of course
only as they appear to us.

§ 11

The problem of the present section is therefore solved. Pure mathemat-
ics, as synthetic cognition a priori, is possible only because it refers to
no other objects than mere objects of the senses, the empirical intuition
of which is based on a pure and indeed a priori intuition (of space and
time), and can be so based because this pure intuition is nothing but the
mere form of sensibility, which precedes the actual appearance of ob-
jects, since it in fact first makes this appearance possible. This faculty of
intuiting a priori does not, however, concern the matter of appearance –
i.e., that which is sensation in the appearance, for that constitutes the
empirical – but only the form of appearance, space and time. If anyone
wishes to doubt in the slightest that the two are a not determinations in-
hering in things in themselves but only mere determinations inhering
in the relation of those things to sensibility, I would very much like to
know how he can find it possible to know, a priori and therefore before
all acquaintance with things, how their intuition must be constituted –
which certainly is the case here with space and time. But this is com-
pletely comprehensible as soon as the two are taken for nothing more
than formal conditions of our sensibility, and objects are taken merely
for appearances; for then the form of appearance, i.e., the pure intuition,
certainly can be represented from ourselves, i.e., a priori.

§ 12

In order to add something by way of illustration and confirmation, we
need only to consider the usual and unavoidably necessary procedure of
the geometers. All proofs of the thoroughgoing equality of two given
figures (that one can in all parts be put in the place of the other) ulti-
mately come down to this: that they are congruent with one another;
which plainly is nothing other than a synthetic proposition based upon
immediate intuition; and this intuition must be given pure and a priori,
for otherwise that proposition could not be granted as apodictically cer-
tain but would have only empirical certainty. It would only mean: we
observe it always to be so and the proposition holds only as far as our
perception has reached until now. That full-standing space (a space that
is itself not the boundary of another space) 31 has three dimensions, and
that space in general cannot have more, is built upon the proposition
that not more than three lines can cut each other at right angles in one
point; this proposition can, however, by no means be proven from con-
cepts, but rests immediately upon intuition, and indeed on pure a priori
intuition, because it is apodictically certain; indeed, that we can require
that a line should be drawn to infinity (in indefinitum), or that a series of
changes (e.g., spaces traversed through motion) should be continued to
infinity, presupposes a representation of space and of time that can only
inhere in intuition, that is, insofar as the latter is not in itself bounded
by anything; for this could never be concluded from concepts. There-
fore pure intuitions a priori indeed actually do underlie mathematics, and
make possible its synthetic and apodictically valid propositions; and con-
sequently our transcendental deduction of the concepts of a space and
time 33 at the same time explains the possibility of a pure mathematics, a
possibility which, without such a deduction, and without our assuming
that “everything which our senses may be given (the outer in space, the
inner in time) is only intuited by us as it appears to us, not as it is in itself,”
could indeed be granted, but into which we could have no insight at all.

§ 13

All those who cannot yet get free of the conception, as if space and time
were actual qualities attaching to things in themselves, can exercise their
acuity on the following paradox, and, if they have sought its solution in
vain, can then, free of prejudice at least for a few moments, suppose that
perhaps the demotion of space and of time to mere forms of our sensory
intuition may indeed have foundation.
If two things are fully the same (in all determinations belonging to
magnitude and quality) in all the parts of each that can always be cog-
nized by itself alone, it should indeed then follow that one, in all cases
and respects, can be put in the place of the other, without this exchange
causing the least recognizable difference. In fact this is how things stand
with plane figures in geometry; yet various spherical figures, 34 notwith-
standing this sort of complete inner agreement, nonetheless reveal such
a difference b in outer relation that one cannot in any case be put in the
place of the other; e.g., two spherical triangles from each of the hemi-
spheres, which have an arc of the equator for a common base, can be fully
equal with respect to their sides as well as their angles, so that nothing
will be found in either, when it is fully described by itself, that is not also
in the description of the other, and still one cannot be put in the place
of the other (that is, in the opposite hemisphere); and here is then after
all an inner difference between the triangles that no understanding can
specify as inner, and that reveals itself only through the outer relation
in space. But I will cite more familiar instances that can be taken from
ordinary life.
What indeed can be more similar to, and in all parts more equal to, my
hand or my ear than its image in the mirror? And yet I cannot put such a
hand as is seen in the mirror in the place of its original; for if the one was a
right hand, then the other in the mirror is a left, and the image of the right
ear is a left one, which can never take the place of the former. Now there
are no inner differences here that any understanding could merely think;
and yet the differences are inner as far as the senses teach, for the left hand
cannot, after all, be enclosed within the same boundaries as the right (they
cannot be made congruent), despite all reciprocal equality and similarity;
one hand’s glove cannot be used on the other. What then is the solution?
These objects are surely not representations of things as they are in
themselves, and as the pure understanding would cognize them, rather,
they are sensory intuitions, i.e., appearances, whose possibility rests on
the relation of certain things, unknown in themselves, to something
else, namely our sensibility. Now, space is the form of outer intuition
of this sensibility, and the inner determination of any space is possible
only through the determination of the outer relation to the whole space
of which the space is a part (the relation to outer sense); that is, the
part is possible only through the whole, which never occurs with things
in themselves as objects of the understanding alone, but well occurs
with mere appearances. We can therefore make the difference between
similar and equal but nonetheless incongruent things (e.g., oppositely
spiralled snails) intelligible through no concept alone, but only through
the relation to right-hand and left-hand, which refers immediately to
intuition.

Note I

Pure mathematics, and especially pure geometry, can have objective re-
ality only under the single condition that it refers merely to objects of
the senses, with regard to which objects, however, the principle remains
fixed, that our sensory representation is by no means a representation
of things in themselves, but only of the way in which they appear to us.
From this it follows, not at all that the propositions of geometry are a
determinations of a mere figment of our poetic phantasy, 35 and there-
fore could not with certainty be referred to actual objects, but rather,
that they are valid necessarily for space and consequently for everything
that may be found in space, because space is nothing other than the form
of all outer appearances, under which alone objects of the senses can be
given to us. Sensibility, whose form lies at the foundation of geometry, is
that upon which the possibility of outer appearances rests; these, there-
fore, can never contain anything other than what geometry prescribes
to them. It would be completely different if the senses had to represent
objects as they are in themselves. For then it absolutely would not follow
from the representation of space, a representation that serves a priori,
with all the various properties of space, as foundation for the geometer,
that all of this, together with what is deduced from it, must be exactly so
in nature. The space of the geometer would be taken for mere fabrication
and would be credited with no objective validity, because it is simply not
to be seen how things would have to agree necessarily with the image that
we form of them by ourselves and in advance. If, however, this image –
or, better, this formal intuition – is the essential property of our sensibil-
ity by means of which alone objects are given to us, and if this sensibility
represents not things in themselves but only their appearances, then it
is very easy to comprehend, and at the same time to prove incontrovert-
ibly: that all outer objects of our sensible world must necessarily agree,
in complete exactitude, with the propositions of geometry, because sen-
sibility itself, through its form of outer intuition (space), with which the
geometer deals, first makes those objects possible, as mere appearances.
It will forever remain a remarkable phenomenon in the history of phi-
losophy that there was a time when even mathematicians who were at
the same time philosophers began to doubt, not, indeed, the correctness
of their geometrical propositions insofar as they related merely to space,
but the objective validity and application to nature of this concept itself
and all its geometrical determinations, since they were concerned that
a line in nature might indeed be composed of physical points, conse-
quently that true space in objects might be composed of simple parts,
notwithstanding that the space which the geometer holds in thought can
by no means be composed of such things. 36 They did not realize that
this space in thought itself makes possible physical space, i.e., the ex-
tension of matter; that this space is by no means a property of things in
themselves, but only a form of our power of sensory representation; that
all objects in space are mere appearances, i.e., not things in themselves
but representations of our sensory intuition; and that, since space as the
geometer thinks it is precisely the form of sensory intuition which we
find in ourselves a priori and which contains the ground of the possibility
of all outer appearances (with respect to their form), these appearances
must of necessity and with the greatest precision harmonize with the
propositions of the geometer, which he extracts not from any fabricated
concept, but from the subjective foundation of all outer appearances,
namely sensibility itself. In this and no other way can the geometer be
secured, regarding the indubitable objective reality of his propositions,
against all the chicaneries of a shallow metaphysics, however strange this
way must seem to such a metaphysics because it does not go back to the
sources of its concepts.

Note II

Everything that is to be given to us as object must be given to us in intu-
ition. But all our intuition happens only by means of the senses; the un-
derstanding intuits nothing, but only reflects. Now since, in accordance
with what has just been proven, the senses never and in no single instance
enable us to cognize things in themselves, but only their appearances,
and as these are mere representations of sensibility, “consequently all
bodies together with the space in which they are found must be taken
for nothing but mere representations in us, and exist nowhere else than
merely in our thoughts.” Now is this not manifest idealism? 37
Idealism consists in the claim that there are none other than thinking
beings; the other things that we believe we perceive in intuition are only
representations in thinking beings, to which in fact no object existing
outside these beings corresponds. I say in opposition: There are things
given to us as objects of our senses existing outside us, yet we know a
nothing of them as they may be in themselves, but are acquainted b only
with their appearances, i.e., with the representations that they produce
in us because they affect our senses. Accordingly, I by all means avow
that there are bodies outside us, i.e., things which, though completely
unknown c to us as to what they may be in themselves, we know d through
the representations which their influence on our sensibility provides for
us, and to which we give the name of a body – which word therefore
merely signifies the appearance of this object that is unknown to us but
is nonetheless real. Can this be called idealism? It is the very opposite of it.
That one could, without detracting from the actual existence of outer
things, say of a great many of their predicates: they belong not to these
things in themselves, but only to their appearances and have no existence
of their own outside our representation, is something that was generally
accepted and acknowledged long before Locke’s time, though more com-
monly thereafter. To these predicates belong warmth, color, taste, etc.
That I, however, even beyond these, include (for weighty reasons) also
among mere appearances the remaining qualities of bodies, which are
called primarias: extension, place, and more generally space along with
everything that depends on it (impenetrability or materiality, shape, etc.),
is something against which not the least ground for uncertainty can be
raised; and as little as someone can be called an idealist because he wants
to admit colors as properties that attach not to the object in itself, but
only to the sense of vision as modifications, just as little can my system
be called idealist simply because I find that even more of, nay, all of the
properties that make up the intuition of a body belong merely to its appear-
ance: for the existence of the thing that appears is not thereby nullified,
as with real idealism, but it is only shown that through the senses we
cannot cognize it at all as it is in itself.
I would very much like to know how then my claims must be framed
so as not to contain any idealism. Without doubt I would have to say: that
the representation of space not only is perfectly in accordance with the
relation that our sensibility has to objects, for I have said that, but that it
is even fully similar to the object; an assertion to which I can attach no
sense, any more than to the assertion that the sensation of red is similar
to the property of cinnabar that excites this sensation in me.

Note III

From this an easily foreseen but empty objection can now be quite eas-
ily rejected: “namely that through the ideality of space and time the
whole sensible world would be transformed into sheer illusion.” 38 After
all philosophical insight into the nature of sensory cognition had previ-
ously been perverted by making sensibility into merely a confused kind
of representation, through which we might still cognize things as they
are but without having the ability to bring everything in this represen-
tation of ours to clear consciousness, we showed on the contrary that
sensibility consists not in this logical difference of clarity or obscurity,
but in the genetic difference of the origin of the cognition itself, since
sensory cognition does not at all represent things as they are but only in
the way in which they affect our senses, and therefore that through the
senses mere appearances, not the things themselves, are given to the un-
derstanding for reflection; 39 from this necessary correction an objection
arises, springing from an inexcusable and almost deliberate misinterpre-
tation, as if my system transformed all the things of the sensible world
into sheer illusion.
If an appearance is given to us, we are still completely free as to how
we want to judge things from it. The former, namely the appearance,
was based on the senses, but the judgment on the understanding, and
the only question is whether there is truth in the determination of the
object or not. The difference between truth and dream, however, is not
decided through the quality of the representations that are referred to
objects, for they are the same in both, but through their connection ac-
cording to the rules that determine the combination of representations
in the concept of an object, and how far they can or cannot stand to-
gether in one experience. And then it is not the fault of the appearances
at all, if our cognition takes illusion for truth, i.e., if intuition, through
which an object is given to us, is taken for the concept of the object,
or even for its existence, which only the understanding can think. The
course of the planets is represented to us by the senses as now pro-
gressive, now retrogressive, and herein is neither falsehood nor truth,
because as long as one grants that this is as yet only appearance, one
still does not judge at all the objective quality of their motion. Since,
however, if the understanding has not taken good care to prevent this
subjective mode of representation from being taken for objective, a false
judgment can easily arise, one therefore says: they appear to go back-
wards; but the illusion is not ascribed to the senses, but to the under-
standing, whose lot alone it is to render an objective judgment from the
appearance.

In this manner, if we do not reflect at all on the origin of our represen-
tations, and we connect our intuitions of the senses, whatever they may
contain, in space and time according to rules for the combination of all
cognition in one experience, then either deceptive illusion or truth can
arise, according to whether we are heedless or careful; that concerns only
the use of sensory representations in the understanding, and not their
origin. In the same way, if I take all the representations of the senses
together with their form, namely space and time, for nothing but ap-
pearances, and these last two for a mere form of sensibility that is by no
means to be found outside it in the objects, and I make use of these same
representations only in relation to possible experience: then in the fact
that I take a them for mere appearances is contained not the least illusion
or temptation toward error; for they nonetheless can be connected to-
gether correctly in experience according to rules of truth. In this manner
all the propositions of geometry hold good for space as well as for all
objects of the senses, and hence for all possible experience, whether I
regard space as a mere form of sensibility or as something inhering in
things themselves; though only in the first case can I comprehend how
it may be possible to know those propositions a priori for all objects of
outer intuition; otherwise, with respect to all merely possible experience,
everything remains just as if I had never undertaken this departure from
the common opinion.

But if I venture to go beyond all possible experience with my concepts
of space and time – which is inevitable if I pass them off for qualities that
attach to things in themselves (for what should then prevent me from
still permitting them to hold good for the very same things, even if my
senses might now be differently framed and either suited to them or
not?) – then an important error can spring up which rests on an illusion,
since I passed off as universally valid that which was a condition for the
intuition of things (attaching merely to my subject, and surely valid for all
objects of the senses, hence for all merely possible experience), because I
referred it to the things in themselves and did not restrict it to conditions
of experience.

Therefore, it is so greatly mistaken that my doctrine of the ideality of
space and time makes the whole sensible world a mere illusion, that, on
the contrary, my doctrine is the only means for securing the application
to actual objects of one of the most important bodies of cognition –
namely, that which mathematics expounds a priori – and for prevent-
ing it from being taken for nothing but mere illusion, since without
this observation it would be quite impossible to make out whether the
intuitions of space and time, which we do not derive from experience
but which nevertheless lie a priori in our representations, were not mere
self-produced fantasies, to which no object at all corresponds, at least
not adequately, and therefore geometry itself a mere illusion, whereas
we have been able to demonstrate the incontestable validity of geometry
with respect to all objects of the sensible world for the very reason that
the latter are mere appearances.

Second, it is so greatly mistaken that these principles of mine, be-
cause they make sensory representations into appearances, are supposed,
in place of the truth of experience, to transform sensory representations
into mere illusion, that, on the contrary, my principles are the only means
of avoiding the transcendental illusion by which metaphysics has always
been deceived and thereby tempted into the childish endeavor of chasing
after soap bubbles, because appearances, which after all are mere rep-
resentations, were taken for things in themselves; from which followed
all those remarkable enactments of the antinomy of reason, which I will
mention later on, and which is removed through this single observation:
that appearance, as long as it is used in experience, brings forth truth, but
as soon as it passes beyond the boundaries of experience and becomes
transcendent, brings forth nothing but sheer illusion.
Since I therefore grant their reality to the things that we represent
to ourselves through the senses, and limit our sensory intuition of these
things only to the extent that in no instance whatsoever, not even in the
pure intuitions of space and time, does it represent a anything more than
mere appearances of these things, and never their quality in themselves,
this is therefore no thoroughgoing illusion ascribed by me to nature,
and my protestation against all imputation of idealism is so conclusive
and clear that it would even seem superfluous if there were not unautho-
rized judges who, being glad to have an ancient name for every deviation
from their false though common opinion, and never judging the spirit
of philosophical nomenclatures but merely clinging to the letter, were
ready to put their own folly in the place of well-determined concepts,
and thereby to twist and deform them. For the fact that I have myself
given to this theory of mine the name of transcendental idealism cannot
justify anyone in confusing it with the empirical idealism of Descartes
(although this idealism was only a problem, whose insolubility left ev-
eryone free, in Descartes’ opinion, to deny the existence of the corporeal
world, since the problem could never be answered satisfactorily) or with
the mysticaland visionary a, idealism of Berkeley (against which, along
with other similar fantasies, our Critique, on the contrary, contains the
proper antidote). For what I called idealism did not concern the ex-
istence of things (the doubting of which, however, properly constitutes
idealism according to the received meaning), for it never came into my
mind to doubt that, but only the sensory representation of things, to
which space and time above all belong; and about these last, hence in
general about all appearances, I have only shown: that they are not things
(but mere modes of representation), nor are they determinations that be-
long to things in themselves. The word transcendental, however, which
with me never signifies a relation of our cognition to things, but only
to the faculty of cognition, was intended to prevent this misinterpretation.
But before it prompts still more of the same, b I gladly withdraw this
name, and I will have it called critical idealism. But if it is an in fact
reprehensible idealism to transform actual things (not appearances) into
mere representations, 42 with what name shall we christen that idealism
which, conversely, makes mere representations into things? I think it
could be named dreaming idealism, to distinguish it from the preced-
ing, which may be called visionary idealism, both of which were to have
been held off by my formerly so-called transcendental, or better, critical
idealism.

Second Part

How is pure natural science possible?

§ 14

Nature is the existence of things, insofar as that existence is determined
according to universal laws. If nature meant the existence of things in
themselves, we would never be able to cognize it, either a priori or a pos-
teriori. Not a priori, for how are we to know what pertains to things in
themselves, inasmuch as this can never come about through the analysis
of our concepts (analytical propositions), since I do not want to know
what may be contained in my concept of a thing (for that belongs to
its logical essence), but what would be added to this concept in the ac-
tuality of a thing, and what the thing itself would be determined by in
its existence apart from my concept. My understanding, and the con-
ditions under which alone it can connect the determinations of things
in their existence, prescribes no rule to the things themselves; these do
not conform to my understanding, but my understanding would have to
conform to them; they would therefore have to be given to me in advance
so that these determinations could be drawn from them, but then they
would not be cognized a priori.
Such cognition of the nature of things in themselves would also be
impossible a posteriori. For if experience were supposed to teach me
laws to which the existence of things is subject, then these laws, inso-
far as they relate to things in themselves, would have to apply to them
necessarily even apart from my experience. Now experience teaches me
what there is and how it is, but never that it necessarily must be so and
not otherwise. Therefore it can never teach me the nature of things in
themselves.

§ 15

Now we are nevertheless actually in possession of a pure natural sci-
ence, which, a priori and with all of the necessity required for apodictic
propositions, propounds laws to which nature is subject. Here I need call
to witness only that propaedeutic to the theory of nature which, under
the title of universal natural science, precedes all of physics (which is
founded on empirical principles). a Therein we find mathematics applied
to appearances, and also merely discursive principles b (from concepts),
which make up the philosophical part of pure cognition of nature.
But indeed there is also much in it that is not completely pure and in-
dependent of sources in experience, such as the concept of motion, of
impenetrability (on which the empirical concept of matter is based), of
inertia, among others, so that it cannot be called completely pure natu-
ral science; furthermore it refers only to the objects of the outer senses,
and therefore does not provide an example of a universal natural sci-
ence in the strict sense; for that would have to bring nature in general –
whether pertaining to an object of the outer senses or of the inner sense
(the object of physics as well as psychology) – under universal laws. But
among the principles of this universal physics 44 a few are found that
actually have the universality we require, such as the proposition: that
substance remains and persists, that everything that happens always pre-
viously is determined by a cause according to constant laws, and so on.
These are truly universal laws of nature, that exist fully a priori. There
is then in fact a pure natural science, and now the question is: How is it
possible?

§ 16

The word nature assumes yet another meaning, namely one that de-
termines the object, whereas in the above meaning it only signified the
conformity to law of the determinations of the existence of things in gen-
eral. Nature considered materialiter 45 is the sum total of all objects of expe-
rience. We are concerned here only with this, since otherwise things
that could never become objects of an experience if they had to be
cognized according to their nature would force us to concepts whose
significance could never be given in concreto (in any example of a pos-
sible experience), and we would therefore have to make for ourselves
mere concepts of the nature of those things, c the reality of which con-
cepts, i.e., whether they actually relate to objects or are mere beings
of thought, could not be decided at all. Cognition of that which can-
not be an object of experience would be hyperphysical, and here we
are not concerned with such things at all, but rather with that cogni-
tion of nature, the reality of which can be confirmed through experi-
ence, even though such cognition is possible a priori and precedes all
experience.

§ 17

The formal in nature in this narrower meaning is therefore the confor-
mity to law of all objects of experience, and, insofar as this conformity
is cognized a priori, the necessary conformity to law of those objects. But
it has just been shown: that the laws of nature can never be cognized a
priori in objects insofar as these objects are considered, not in relation to
possible experience, but as things in themselves. We are here, however,
concerned not with things in themselves (the properties of which we
leave undetermined), but only with things as objects of a possible expe-
rience, and the sum total of such objects is properly what we here call
nature. And now I ask whether, if the discussion is of the possibility of a
cognition of nature a priori, it would be better to frame the problem in
this way: How is it possible to cognize a priori the necessary conformity
to law of things as objects of experience, or: How is it possible in general
to cognize a priori the necessary conformity to law of experience itself with
regard to all of its objects?

On closer examination, whether the question is posed one way or the
other, its solution will come out absolutely the same with regard to the
pure cognition of nature (which is actually the point of the question).
For the subjective laws under which alone a cognition of things through
experience a is possible also hold good for those things as objects of a
possible experience (but obviously not for them as things in themselves,
which, however, are not at all being considered here). It is completely the
same, whether I say: A judgment of perception can never be considered as
valid for experience without the law, that if an event is perceived then it is
always referred to something preceding from which it follows according
to a universal rule; or if I express myself in this way: Everything of which
experience shows that it happens must have a cause.
It is nonetheless more appropriate to choose the first formulation.
For since we can indeed, a priori and previous to any objects being given,
have a cognition of those conditions under which alone an experience
regarding objects is possible, but never of the laws to which objects may
be subject in themselves without relation to possible experience, we will
therefore be able to study a priori the nature of things in no other way
than by investigating the conditions, and the universal (though subjec-
tive) laws, under which alone such a cognition is possible as experience (as
regards mere form), and determining the possibility of things as objects
of experience accordingly; for were I to choose the second mode of ex-
pression and to seek the a priori conditions under which nature is possible
as an object of experience, I might then easily fall into misunderstanding
and fancy that I had to speak about nature as a thing in itself, and in that
case I would be wandering about fruitlessly in endless endeavors to find
laws for things about which nothing is given to me.
We will therefore be concerned here only with experience and with
the universal conditions of its possibility which are given a priori, and
from there we will determine nature as the whole object of all possible
experience. I think I will be understood: that here I do not mean the rules
for the observation of a nature that is already given, which presuppose
experience already; and so do not mean, how we can learn the laws from
nature (through experience), for these would then not be laws a priori
and would provide no pure natural science; but rather, how the a priori
conditions of the possibility of experience are at the same time the sources
out of which all universal laws of nature must be derived.

§ 18

We must therefore first of all note: that, although all judgments of ex-
perience are empirical, i.e., have their basis in the immediate perception
of the senses, nonetheless the reverse is not the case, that all empirical
judgments are therefore judgments of experience; rather, beyond the
empirical and in general beyond what is given in sensory intuition, spe-
cial concepts must yet be added, which have their origin completely a
priori in the pure understanding, and under which every perception can
first be subsumed and then, by means of the same concepts, transformed
into experience.
Empirical judgments, insofar as they have objective validity, are judgments
of experience; those, however, that are only subjectively valid I call mere
judgments of perception. The latter do not require a pure concept
of the understanding, but only the logical connection of perceptions in
a thinking subject. But the former always demand, in addition to the
representations of sensory intuition, special concepts originally generated
in the understanding, which are precisely what make the judgment of
experience objectively valid.
All of our judgments are at first mere judgments of perception; they
hold only for us, i.e., for our subject, and only afterwards do we give
them a new relation, namely to an object, and intend that the judgment
should also be valid at all times for us and for everyone else; for if a
judgment agrees with an object, then all judgments of the same object
must also agree with one another, and hence the objective validity of a
judgment of experience signifies nothing other than its necessary uni-
versal validity. But also conversely, if we find cause to deem a judgment
necessarily, universally valid (which is never based on the perception, but
on the pure concept of the understanding under which the perception
is subsumed), we must then also deem it objective, i.e., as expressing
92Prolegomena to any future metaphysics
not merely a relation of a perception to a subject, but a property of an
object; for there would be no reason why other judgments necessarily
would have to agree with mine, if there were not the unity of the object –
an object to which they all refer, with which they all agree, and, for that
reason, also must all harmonize among themselves.

§ 19

Objective validity and necessary universal validity (for everyone) are
therefore interchangeable concepts, and although we do not know the
object in itself, nonetheless, if we regard a judgment as universally valid
and hence necessary, objective validity is understood to be included.
Through this judgment we cognize the object (even if it otherwise re-
mains unknown as it may be in itself) by means of the universally valid
and necessary connection of the given perceptions; and since this is the
case for all objects of the senses, judgments of experience will not de-
rive their objective validity from the immediate cognition of the object
(for this is impossible), but merely from the condition for the universal
validity of empirical judgments, which, as has been said, never rests on
empirical, or indeed sensory conditions at all, but on a pure concept
of the understanding. The object always remains unknown in itself; if,
however, through the concept of the understanding the connection of
the representations which it provides to our sensibility is determined as
universally valid, then the object is determined through this relation, and
the judgment is objective.
Let us provide examples: that the room is warm, the sugar sweet,
the wormwood 46 repugnant, ∗ are merely subjectively valid judgments.
I do not at all require that I should find it so at every time, or that
everyone else should find it just as I do; they express only a relation of
two sensations to the same subject, namely myself, and this only in my
present state of perception, and are therefore not expected to be valid for
the object: these I call judgments of perception. The case is completely
different with judgments of experience. What experience teaches me
under certain circumstances, it must teach me at every time and teach
everyone else as well, and its validity is not limited to the subject or its
state at that time. Therefore I express all such judgments as objectively
valid; as, e.g., if I say: the air is elastic, then this judgment is to begin
with only a judgment of perception; I relate two sensations in my senses
only to one another. If I want it to be called a judgment of experience, I
then require that this connection be subject to a condition that makes it
universally valid. I want therefore that I, at every time, and also everyone
else, would necessarily have to connect the same perceptions a under the
same circumstances.

§ 20

We will therefore have to analyze experience in general, in order to see
what is contained in this product of the senses and the understanding,
and how the judgment of experience is itself possible. At bottom lies the
intuition of which I am conscious, i.e., perception ( perceptio), which be-
longs solely to the senses. But, secondly, judging (which pertains solely
to the understanding) also belongs here. Now this judging can be of two
types: first, when I merely compare the perceptions and connect them
in a consciousness of my state, or, second, when I connect them in a
consciousness in general. The first judgment is merely a judgment of
perception and has thus far only subjective validity; it is merely a con-
nection of perceptions within my mental state, without reference to the
object. Hence it is not, as is commonly imagined, sufficient for experi-
ence to compare perceptions and to connect them in one consciousness
by means of judging; from that there arises no universal validity and ne-
cessity of the judgment, on account of which alone it can be objectively
valid and so can be experience.
A completely different judgment therefore occurs before experience
can arise from perception. The given intuition must be subsumed under
a concept that determines the form of judging in general with respect
to the intuition, connects the empirical consciousness of the latter in
a consciousness in general, and thereby furnishes empirical judgments
with universal validity; a concept of this kind is a pure a priori concept
of the understanding, which does nothing but simply determine for an
intuition the mode in general in which it can serve for judging. The
concept of cause being such a concept, it therefore determines the intu-
ition which is subsumed under it, e.g., that of air, with respect to judging
in general – namely, so that the concept of air serves, with respect to
expansion, in the relation of the antecedent to the consequent in a hy-
pothetical judgment. The concept of cause is therefore a pure concept
of the understanding, which is completely distinct from all possible per-
ception, and serves only, with respect to judging in general, to determine
that representation which is contained under it and so to make possible
a universally valid judgment.
Now before a judgment of experience can arise from a judgment of
perception, it is first required: that the perception be subsumed under
a concept of the understanding of this kind; e.g., the air belongs under
the concept of cause, a which determines the judgment about the air as
hypothetical with respect to expansion. ∗ This expansion is thereby rep-
resented not as belonging merely to my perception of the air in my state
of perception or in several of my states or in the state of others, but as
necessarily belonging to it, and the judgment: the air is elastic, becomes
universally valid and thereby for the first time a judgment of experi-
ence, because certain judgments occur beforehand, which subsume the
intuition of the air under the concept of cause and effect, and thereby
determine the perceptions not merely with respect to each other in my
subject, but with respect to the form of judging in general (here, the hypo-
thetical), and in this way make the empirical judgment universally valid.
If one analyzes all of one’s synthetic judgments insofar as they are ob-
jectively valid, one finds that they never consist in mere intuitions that
have, as is commonly thought, merely been connected in a b judgment
through comparison, 47 but rather that they would not be possible if,
over and above the concepts drawn from intuition, a pure concept of
the understanding had not been added under which these concepts had
been subsumed and in this way first combined into an objectively valid
judgment. Even the judgments of pure mathematics in its simplest ax-
ioms are not exempt from this condition. The principle: a straight line is
the shortest line between two points, presupposes that the line has been
subsumed under the concept of magnitude, which certainly is no mere
intuition, but has its seat solely in the understanding and serves to deter-
mine the intuition (of the line) with respect to such judgments as may be
passed on it as regards the quantity of these judgments, namely plurality
(as judicia plurativa), since through such judgments it is understood that
in a given intuition a homogeneous plurality is contained.

§ 21

In order therefore to explain the possibility of experience insofar as it
rests on pure a priori concepts of the understanding, we must first present
that which belongs to a judgments in general, and the various moments of
the understanding therein, in a complete table; for the pure concepts of
the understanding – which are nothing more than concepts of intuitions
in general insofar as these intuitions are, with respect to one or another
of these moments, in themselves determined to judgments and therefore
determined necessarily and with universal validity – will come out exactly
parallel to them. By this means the a priori principles of the possibility
of all experience as objectively valid empirical cognition will also be
determined quite exactly. For they are nothing other than propositions
that subsume all perception (according to certain universal conditions of
intuition) under those pure concepts of the understanding.

Logical table of Judgments

1. According to Quantity

Universal
Particular
Singular

2. According to Quality

Affirmative
Negative
Infinite

3. According to Relation

Categorical
Hypothetical
Disjunctive

4. According to Modality

Problematic
Assertoric
Apodictic

∗ So I would prefer those judgments to be called, which are called particularia in
logic. For the latter expression already contains the thought that they are not
universal. If, however, I commence from unity (in singular judgments) and then
continue on to the totality, I still cannot mix in any reference to the totality; I
think only a plurality without totality, not the exception to the latter. 48 This is
necessary, if the logical moments are to be placed under the pure concepts of
the understanding; in logical usage things can remain as they were.
a Reading zu for zum, with Vorländer.
96Prolegomena to any future metaphysics
transcendental table
of Concepts of the Understanding
1.
According to Quantity
Unity (measure)
Plurality (magnitude)
Totality (the whole)
2.
According to Quality 3.
According to Relation
Reality
Negation
Limitation Substance
Cause
Community
4.
According to Modality
Possibility
Existence
Necessity
pure physiological table
of Universal Principles of Natural Science
1.
Axioms
of intuition
2.
Anticipations
of perception
3.
Analogies
of experience
4.
Postulates
of empirical thinking in general
§ 21[a] a
In order to comprise all the preceding in one notion, it is first of all
necessary to remind the reader that the discussion here is not about
the genesis of experience, but about that which lies in experience. The
former belongs to empirical psychology and could never be properly
developed even there without the latter, which belongs to the critique of
cognition and especially of the understanding.
a
Adding the letter “a,” with Ak and Vorländer, to distinguish this section from the pre-
ceding one, both of which are shown as “§21” in the original edition.
97
4: 304Prolegomena to any future metaphysics
Experience consists of intuitions, which belong to sensibility, and of
judgments, which are solely the understanding’s business. Those judg-
ments that the understanding forms solely from sensory intuitions are,
however, still not judgments of experience by a long way. For in the one a
case the judgment would only connect perceptions as they are given in
sensory intuition; but in the latter case the judgments are supposed to
say what experience in general contains, therefore not what mere per-
ception – whose validity is merely subjective – contains. The judgment
of experience must still therefore, beyond the sensory intuition and its
logical connection (in accordance with which the intuition has been ren-
dered universal through comparison in a judgment), add something that
determines the synthetic judgment as necessary, and thereby as univer-
sally valid; and this can be nothing but that concept which represents the
intuition as in itself determined with respect to one form of judgment
rather than the others, b i.e., c a concept of that synthetic unity of intu-
itions which can be represented only through a given logical function of
judgments.

§ 22

To sum this up: the business of the senses is to intuit; that of the
understanding, to think. To think, however, is to unite representations
in a consciousness. This unification either arises merely relative to the
subject and is contingent and subjective, or it occurs without condition
and is necessary or objective. The unification of representations in a
consciousness is judgment. Therefore, thinking is the same as judging
or as relating representations to judgments in general. Judgments are
therefore either merely subjective, if representations are related to
one consciousness in one subject alone and are united in it, or they
are objective, if they are united in a consciousness in general, i.e., are
united necessarily therein. The logical moments of all judgments are
so many possible ways of uniting representations in a consciousness. If,
however, the very same moments serve as concepts, they are concepts
of the necessary unification of these representations in a consciousness,
and so are principles d of objectively valid judgments. This unification
in a consciousness is either analytic, through identity, or synthetic,
through combination and addition of various representations with one
another. Experience consists in the synthetic connection of appearances
(perceptions) in a consciousness, insofar as this connection is necessary.
Therefore pure concepts of the understanding are those under which all
perceptions must first be subsumed before they can serve in judgments
of experience, in which the synthetic unity of perceptions is represented
as necessary and universally valid. ∗

§ 23

Judgments, insofar as they are regarded merely as the condition for the
unification of given representations in a consciousness, are rules. These
rules, insofar as they represent the unification as necessary, are a priori
rules, and provided that there are none above them from which they can
be derived, are principles. Now since, with respect to the possibility of all
experience, if merely the form of thinking is considered in the experience,
no conditions on judgments of experience are above those that bring
the appearance (according to the varying form of their intuition) under
pure concepts of the understanding (which make the empirical judgment
objectively valid), these conditions are therefore the a priori principles
of possible experience.
Now the principles of possible experience are, at the same time, uni-
versal laws of nature that can be cognized a priori. And so the problem
that lies in our second question, presently before us: how is pure natural
science a possible? is solved. For the systematization that is required for the
form of a science is here found to perfection, since beyond the aforemen-
tioned formal conditions of all judgments in general, hence of all rules
whatsoever furnished by logic, no others are possible, and these form
a logical system; but the concepts based thereon, which contain b the a
priori conditions for all synthetic and necessary judgments, for that very
reason form a transcendental system; finally, the principles by means of
∗ But how does this proposition: that judgments of experience are supposed to
contain necessity in the synthesis of perceptions, square with my proposition,
urged many times above: that experience, as a posteriori cognition, can provide
merely contingent judgments? If I say: Experience teaches me something, I
always mean only the perception that is in it – e.g., that upon illumination of
the stone by the sun, warmth always follows – and hence the proposition from
experience is, so far, always contingent. That this warming follows necessarily
from illumination by the sun is indeed contained in the judgment of experience
(in virtue of the concept of cause), but I do not learn it from experience; rather,
conversely, experience is first generated through this addition of a concept of
the understanding (of cause) to the perception. Concerning how the perception
may come by this addition, the Critique must be consulted, in the section on
transcendental judgment, pp. 137ff. 49
a Reading Naturwissenschaft for Vernunftwissenschaft, with Ak.
Reading enthalten for erhalten, with Ak.
b
99
4: 306Prolegomena to any future metaphysics
which all appearances are subsumed under these concepts form a phys-
iological system, i.e., a system of nature, a which precedes all empirical
cognition of nature and first makes it possible, and can therefore be called
the true universal and pure natural science.
§ 24
4: 307
The b first ∗ of the physiological principles subsumes all appearances, as
intuitions in space and time, under the concept of magnitude and is to
that extent a principle for the application of mathematics to experience.
The c second does not subsume the properly empirical – namely sensa-
tion, which signifies the real d in intuitions – directly under the concept
of magnitude, since sensation is no intuition containing space or time, al-
though it does place the object corresponding to it in both; but there
nonetheless is, between reality (sensory representation) and nothing,
i.e., the complete emptiness of intuition in time, a difference that has
a magnitude, for indeed between every given degree of light and dark-
ness, every degree of warmth and the completely cold, every degree of
heaviness and absolute lightness, every degree of the filling of space and
completely empty space, ever smaller degrees can be thought, just as be-
tween consciousness and total unconsciousness (psychological darkness)
ever smaller degrees occur; therefore no perception is possible that would
show a complete absence, e.g., no psychological darkness is possible that
could not be regarded as a consciousness that is merely outweighed by
another, stronger one, and thus it is in all cases of sensation; as a result
of which the understanding can anticipate even sensations, which form
the proper quality of empirical representations (appearances), by means
of the principle that they all without exception, hence the real in all ap-
pearance, have degrees – which is the second application of mathematics
(mathesis intensorum) to natural science.

§ 25

With respect to the relation of appearances, and indeed exclusively
with regard to their existence, the determination of this relation is
not mathematical but dynamical, and it can a never be objectively valid,
hence fit for experience, if it is not subject to a priori principles, which
first make cognition through experience possible with respect to that
determination. 52 Therefore appearances must be subsumed under the
concept of substance, which, as a concept of the thing itself, underlies all
determination of existence; or second, insofar as a temporal sequence,
i.e., an event, is met with among the appearances, they must be sub-
sumed under the concept of an effect in relation to a b cause; or, insofar
as coexistence is to be cognized objectively, i.e., through a judgment of
experience, they must be subsumed under the concept of community
(interaction): and so a priori principles underlie objectively valid, though
empirical, judgments, i.e., they underlie the possibility of experience
insofar as it is supposed to connect objects in nature according to exis-
tence. These principles are the actual laws of nature, which can be called
dynamical.
Finally, there also belongs to judgments of experience the cognition
of agreement and connection: not so much of the appearances among
themselves in experience, but of their relation to experience in general, a
relation that contains either their agreement with the formal conditions
that the understanding cognizes, or their connection with the material
of the senses and perception, or both c united in one concept, and thus
possibility, existence, and necessity according to universal laws of nature;
all of which would constitute the physiological theory of method (the
distinction of truth and hypotheses, and the boundaries of the reliability
of the latter).

§ 26

Although the third table of principles, which is drawn from the nature of
the understanding itself according to the critical method, in itself exhibits
a perfection through which it raises itself far above every other that
has (albeit vainly) ever been attempted or may yet be attempted in the
future from the things themselves through the dogmatic method: namely,
that in it d all of the synthetic principles a priori are exhibited completely
and according to a principle, e namely that of the faculty for judging
in general (which constitutes the essence of experience with respect to
the understanding), so that one can be certain there are no more such
principles (a satisfaction that the dogmatic method can never provide) –
nevertheless this is still far from being its greatest merit.
Notice must be taken of the ground of proof that reveals the pos-
sibility of this a priori cognition and at the same time limits all such
principles to a condition that must never be neglected if they are not to
be a misunderstood and extended in use further than the original sense
which the understanding places in them will allow: namely, that they
contain only the conditions of possible experience in general, insofar as
it is subject to a priori laws. Hence I do not say: that things in themselves
contain b a magnitude, their reality a degree, their existence a connection
of accidents in a substance, and so on; for that no one can prove, because
such a synthetic connection out of mere concepts, in which all relation
to sensory intuition on the one hand and all connection of such intuition
in a possible experience on the other is lacking, is utterly impossible.
Therefore the essential limitation on the concepts in these principles is:
that only as objects of experience are all things necessarily subject a priori
to the aforementioned conditions.
From this there follows then secondly a specifically characteristic
mode of proving the same thing: that the above-mentioned principles
are not referred directly to appearances and their relation, but to the pos-
sibility of experience, for which appearances constitute only the matter
but not the form, i.e., they are referred to the objectively and univer-
sally valid synthetic propositions through which judgments of experience
are distinguished from mere judgments of perception. This happens be-
cause the appearances, as mere intuitions that fill a part of space and time,
are subject to the concept of magnitude, which synthetically unifies the
multiplicity of intuitions a priori according to rules; and because the real
in the appearances must have a degree, insofar as perception contains,
beyond intuition, sensation as well, between which and nothing, i.e.,
the complete disappearance of sensation, a transition always occurs by
diminution, insofar, that is, as sensation itself fills no part of space and time, ∗
∗ Warmth, light, etc. are just as great (according to degree) in a small space as in a
large one; just as the inner representations (pain, consciousness in general) are
not smaller according to degree whether they last a short or a long time. Hence
the magnitude here is just as great in a point and in an instant as in every space
and time however large. Degrees are therefore magnitudes, c not, however, in
intuition, but in accordance with mere sensation, or indeed with the magnitude
of the ground of an intuition, and can be assessed as magnitudes only through
the relation of 1 to 0, i.e., in that every sensation can proceed in a certain
time to vanish through infinite intermediate degrees, or to grow from nothing
to a determinate sensation through infinite moments of accretion. (Quantitas
qualitatis est gradus.)
but yet the transition to sensation from empty time or space is possi-
ble only in time, with the consequence that although sensation, as the
quality of empirical intuition with respect to that by which a sensation
differs specifically from other sensations, can never be cognized a priori,
it nonetheless can, in a possible experience in general, as the magnitude
of perception, be distinguished intensively from every other sensation
of the same kind; from which, then, the application of mathematics to
nature, with respect to the sensory intuition whereby nature is given to
us, is first made possible and determined.
Mostly, however, the reader must attend to the mode of proving the
principles that appear under the name of the Analogies of Experience.
For since these do not concern the generation of intuitions, as do the
principles for applying mathematics to natural science in general, but the
connection of their existence in one experience, and since this connec-
tion can be nothing a other than the determination of existence in time
according to necessary laws, under which alone the connection is objec-
tively valid and therefore is experience: it follows that the proof does not
refer to synthetic unity in the connection of things in themselves, but of
perceptions, and of these indeed not with respect to their content, but to
the determination of time and to the relation of existence in time in ac-
cordance with universal laws. These universal laws contain therefore the
necessity of the determination of existence in time in general (hence a
priori according to a rule of the understanding), if the empirical determi-
nation in relative time is to be objectively valid, and therefore to be expe-
rience. For the reader who is stuck in the long habit of taking experience
to be a mere empirical combining of perceptions – and who therefore has
never even considered that it extends much further than these reach, that
is, that it gives to empirical judgments universal validity and to do so re-
quires a pure unity of the understanding that precedes a priori – I cannot
adduce more here, these being prolegomena, except only to recommend:
to heed well this distinction of experience from a mere aggregate of per-
ceptions, and to judge the mode of proof from this standpoint.

§ 27

Here is now the place to dispose thoroughly of the Humean doubt. He
rightly affirmed: that we in no way have insight through reason into
the possibility of causality, i.e., the possibility of relating the existence
of one thing to the existence of some other thing that would necessar-
ily be posited through the first one. I add to this that we have just as
little insight into the concept of subsistence, i.e., of the necessity that
a subject, which itself cannot be a predicate of any other thing, should
underlie the existence of things – nay, that we cannot frame any concept
of the possibility of any such thing (although we can point out examples
of its use in experience); and I also add that this very incomprehensi-
bility affects the community of things as well, since we have no insight
whatsoever into how, from the state of one thing, a consequence could
be drawn about the state of completely different things outside it (and
vice versa), and into how substances, each of which has its own sepa-
rate existence, should depend on one another and should indeed do so
necessarily. Nonetheless, I am very far from taking these concepts to be
merely borrowed from experience, and from taking the necessity repre-
sented in them to be falsely imputed and a mere illusion through which
long habit deludes us; rather, I have sufficiently shown that they and the
principles taken from them stand firm a priori prior to all experience,
and have their undoubted objective correctness, though of course only
with respect to experience.

§ 28

Although I therefore do not have the least concept of such a connection
of things in themselves, how they can exist as substances or act as causes
or stand in community with others (as parts of a real whole), and though
I can still less think such properties of appearances as appearances (for
these concepts do not contain what lies in appearances, but what the
understanding alone must think), we nonetheless do have a concept of
such a connection of representations in our understanding, and indeed
in judging in general, namely: that representations belong in one kind
of judgments as subject in relation to predicate, in another as ground in
relation to consequence, and in a third as parts that together make up
a whole possible experience. Further, we cognize a priori: that, without
regarding the representation of an object as determined with respect to
one or another a of these moments, we could not have any cognition at
all that was valid for the object; and if we were to concern ourselves with
the object in itself, then no unique characteristic would be possible by
which I could cognize that it b had been determined with respect to one
or another of the above-mentioned moments, i.e., that it belonged under
the concept of substance, or of cause, or (in relation to other substances)
under the concept of community; for I have no concept of the possibility
of such a connection of existence. The question is not, however, how
things in themselves, but how the cognition of things in experience is
determined with respect to said moments of judgments in general, i.e.,
how things as objects of experience can and should be subsumed under
those concepts of the understanding. And then it is clear that I have
complete insight into not only the possibility but also the necessity of
subsuming all appearances under these concepts, i.e., of using them as
principles of the possibility of experience.

§ 29

For having a try at Hume’s problematic concept (this, his crux
metaphysicorum), a namely the concept of cause, there is first given to
me a priori, by means of logic: the form of a conditioned judgment in
general, that is, the use of a given cognition as ground and another as
consequent. It is, however, possible that in perception a rule of relation
will be found, which says this: that a certain appearance is constantly
followed by another (though not the reverse); and this is a case for me to
use hypothetical judgment and, e.g., to say: If a body is illuminated by
the sun for long enough, then it becomes warm. Here there is of course
not yet a necessity of connection, hence not yet the concept of cause.
But I continue on, and say: if the above proposition, which is merely a
subjective connection of perceptions, is to be a proposition of experi-
ence, then it must be regarded as necessarily and universally valid. But a
proposition of this sort would be: The sun through its light is the cause
of the warmth. The foregoing empirical rule is now regarded as a law,
and indeed as valid not merely of appearances, but of them on behalf of a
possible experience, which requires universally and therefore necessarily
valid rules. I therefore have quite good insight into the concept of cause,
as a concept that necessarily belongs to the mere form of experience,
and into its possibility as a synthetic unification of perceptions in a con-
sciousness in general; but I have no insight at all into the possibility of
a thing in general as a cause, and that indeed because the concept of
cause indicates a condition that in no way attaches to things, but only to
experience, namely, that experience can be an objectively valid cognition
of appearances and their sequence in time only insofar as the antecedent
appearance can be connected with the subsequent one according to the
rule of hypothetical judgments.

§ 30

Consequently, even the pure concepts of the understanding have no
significance at all if they depart from objects of experience and want to
be referred to things in themselves (noumena). They serve as it were
only to spell out appearances, so that they can be read as experience;
the principles that arise from their relation to the sensible world serve
our understanding for use in experience only; beyond this there are
arbitrary connections without objective reality whose possibility cannot
be cognized a priori and whose relation to objects cannot, through any
example, be confirmed or even made intelligible, since all examples can
be taken only from some possible experience or other and hence the
objects of these concepts can be met with nowhere else but in a possible
experience.
This complete solution of the Humean problem, though coming out
contrary to the surmise of the originator, thus restores to the pure con-
cepts of the understanding their a priori origin, and to the universal laws
of nature their validity as laws of the understanding, but in such a way
that it restricts their use to experience only, because their possibility is
founded solely in the relation of the understanding to experience: not,
however, in such a way that they are derived from experience, but that ex-
perience is derived from them, a completely reversed type of connection
that never occurred to Hume.
From this now flows the following result of all the foregoing investiga-
tions: “All synthetic a priori principles a are nothing more than principles b
of possible experience,” and can never be related to things in themselves,
but only to appearances as objects of experience. Therefore both pure
mathematics and pure natural science can never refer to anything more
than mere appearances, and they can only represent either that which
makes experience in general possible, or that which, being derived from
these principles, c must always be able to be represented in some possible
experience or other.

§ 31

And so for once one has something determinate, and to which one can
adhere in all metaphysical undertakings, which have up to now boldly
enough, but always blindly, run over everything without distinction. It
never occurred to dogmatic thinkers that the goal of their efforts might
have been set up so close, nor even to those who, obstinate in their so-
called sound common sense, d sallied forth to insights with concepts and
principles of the pure understanding that were indeed legitimate and
natural, but were intended for use merely in experience, and for which
they neither recognized nor could recognize any determinate bound-
aries, because they neither had reflected on nor were able to reflect on
the nature or even the possibility of such a pure understanding.
Many a naturalist of pure reason (by which I mean he who trusts
himself, without any science, to decide in matters of metaphysics) would
like to pretend that already long ago, through the prophetic spirit of
his sound common sense, he had not merely suspected, but had known
and understood, that which is here presented with so much prepara-
tion, or, if he prefers, with such long-winded pedantic pomp: “namely
that with all our reason we can never get beyond the field of experi-
ences.” But since, if someone gradually questions him on his rational
principles, a he must indeed admit that among them there are many that
he has not drawn from experience, which are therefore independent of
it and valid a priori – how and on what grounds will he then hold within
limits the dogmatist (and himself), who makes use of b these concepts
and principles beyond all possible experience for the very reason that
they are cognized independently of experience. And even he, this adept
of sound common sense, is not so steadfast that, despite all of his pre-
sumed and cheaply gained wisdom, he will not stumble unawares out
beyond the objects of experience into the field of chimeras. Ordinarily,
he is indeed deeply enough entangled therein, although he cloaks his
ill-founded claims in a popular style, since he gives everything out as
mere probability, reasonable conjecture, or analogy.

§ 32

Already from the earliest days of philosophy, apart from the sensible
beings or appearances (phaenomena) that constitute the sensible world,
investigators of pure reason have thought of special intelligible beings d
(noumena), which were supposed to form an intelligible world; e and they
have granted reality to the intelligible beings alone, because they took
appearance and illusion to be one and the same thing (which may well
be excused in an as yet uncultivated age).
In fact, if we view the objects of the senses as mere appearances, as is
fitting, then we thereby admit at the very same time that a thing in itself
underlies them, although we are not acquainted with this thing as it may
be constituted in itself, but only with its appearance, i.e., with the way in
which our senses are affected by this unknown something. Therefore the
understanding, just by the fact that it accepts appearances, also admits to
the existence of things in themselves, and to that extent we can say that
the representation of such beings as underlie the appearances, hence of
mere intelligible beings, is not merely permitted but also unavoidable.
Our critical deduction in no way excludes things of such kind
(noumena), but rather restricts the principles of aesthetic 54 in such a way
that they are not supposed to extend to all things, whereby everything
would be transformed into mere appearance, but are to be valid only for
objects of a possible experience. Hence intelligible beings are thereby al-
lowed only with the enforcement of this rule, which brooks no exception
whatsoever: that we do not know and cannot know anything determinate
about these intelligible beings at all, because our pure concepts of the
understanding as well as our pure intuitions refer to nothing but objects
of possible experience, hence to mere beings of sense, and that as soon as
one departs from the latter, not the least significance remains for those
concepts.

§ 33

There is in fact something insidious in our pure concepts of the under-
standing, as regards enticement toward a transcendent use; for so I call
that use which goes out beyond all possible experience. It is not only
that our concepts of substance, of force, of action, of reality, etc., are
wholly independent of experience, likewise contain no sensory appear-
ance whatsoever, and so in fact seem to refer to things in themselves
(noumena); but also, which strengthens this supposition yet further, that
they contain in themselves a necessity of determination which experience
never equals. The concept of cause contains a rule, according to which
from one state of affairs another follows with necessity; but experience
can only show us that from one state of things another state often, or,
at best, commonly, follows, and it can therefore furnish neither strict
universality nor necessity (and so forth).
Consequently, the concepts of the understanding appear to have much
more significance and content than they would if their entire vocation
were exhausted by mere use in experience, and so the understanding
unheededly builds onto the house of experience a much roomier wing,
which it crowds with mere beings of thought, without once noticing that
it has taken its otherwise legitimate concepts far beyond the boundaries
of their use.

§ 34

Two important, nay completely indispensable, though utterly dry inves-
tigations were therefore needed, which were carried out in the Critique,
108Prolegomena to any future metaphysics
pp.137ff. and 235ff. 55 Through the first of these it was shown that the
senses do not supply pure concepts of the understanding in concreto, but
only the schema for their use, and that the object appropriate to this
schema is found only in experience (as the product of the understand-
ing from materials of sensibility). In the second investigation (Critique,
p. 235) it is shown: that notwithstanding the independence from expe-
rience of our pure concepts of the understanding and principles, and
even their apparently larger sphere of use, nonetheless, outside the field
of experience nothing at all can be thought by means of them, because
they can do nothing but merely determine the logical form of judgment
with respect to given intuitions; but since beyond the field of sensibility
there is no intuition at all, these pure concepts lack completely all signif-
icance, in that there are no means through which they can be exhibited
in concreto, and so all such noumena, together with their aggregate – an
intelligible ∗ world – are nothing but representations of a problem, whose
object is in itself perfectly possible, but whose solution, given the nature
of our understanding, is completely impossible, since our understanding
is no faculty of intuition but only of the connection of given intuitions
in an experience; and experience therefore has to contain all the ob-
jects for our concepts, whereas apart from it all concepts will be without
significance, since no intuition can be put under them.

§ 35

The imagination may perhaps be excused if it daydreams c every now
and then, i.e., if it does not cautiously hold itself inside the limits of
experience; for it will at least be enlivened and strengthened through
such free flight, and it will always be easier to moderate its boldness
than to remedy its languor. That the understanding, however, which is
supposed to think, should, instead of that, daydream – for this it can never
be forgiven; for all assistance in setting bounds, where needed, to the
revelry a of the imagination depends on it alone.
The understanding begins all this very innocently and chastely. First,
it puts in order the elementary cognitions that dwell in it prior to all expe-
rience but must nonetheless always have their application in experience.
Gradually, it removes these constraints, and what is to hinder it from
doing so, since the understanding has quite freely taken its principles
from within itself? And now reference is made first to newly invented
forces in nature, soon thereafter to beings outside nature, in a word, to a
world for the furnishing of which building materials cannot fail us, since
they are abundantly supplied through fertile invention, and though not
indeed confirmed by experience, are also never refuted by it. That is
also the reason why young thinkers so love metaphysics of the truly dog-
matic sort, and often sacrifice their time and their otherwise useful talent
to it.
It can, however, help nothing at all to want to curb these fruitless
endeavors of pure reason by all sorts of admonitions about the difficulty
of resolving such deeply obscure questions, by complaints over the limits
of our reason, and by reducing assertions to mere conjectures. For if the
impossibility of these endeavors has not been clearly demonstrated, and if
reason’s knowledge of itself b does not become true science, in which the
sphere of its legitimate use is distinguished with geometrical certainty
(so to speak) from that of its empty and fruitless use, then these futile
efforts will never be fully abandoned.

§ 36

How is nature itself possible?

This question, which is the highest point that transcendental philosophy
can ever reach, and up to which, as its boundary and completion, it must
be taken, actually contains two questions.
first: How is nature possible in general in the material sense, namely,
according to intuition, as the sum total of appearances; how are space,
time, and that which fills them both, the object of sensation, possible in
general? The answer is: by means of the constitution of our sensibility,
in accordance with which our sensibility is affected in its characteristic
way by objects that are in themselves unknown to it and that are wholly
distinct from said appearances. This answer is, in the book itself, given
in the Transcendental Aesthetic, but here in the Prolegomena through the
solution of the first main question.
second: How is nature possible in the formal a sense, as the sum total
of the rules to which all appearances must be subject if they are to be
thought as connected in one experience? The answer cannot come out
otherwise than: it is possible only by means of the constitution of our
understanding, in accordance with which all these representations of
sensibility are necessarily referred to one consciousness, and through
which, first, the characteristic mode of our thinking, namely by means
of rules, is possible, and then, by means of these rules, experience is
possible – which is to be wholly distinguished from insight into objects in
themselves. This answer is, in the book itself, given in the Transcendental
Logic,  but here in the Prolegomena, in the course of solving the second
main question.
But how this characteristic property of our sensibility itself may be
possible, or that of our understanding and of the necessary apperception
that underlies it and all thinking, cannot be further solved and answered,
because we always have need of them in turn for all answering and for
all thinking of objects.
There are many laws of nature that we can know only through ex-
perience, but lawfulness in the connection of appearances, i.e., nature
in general, we cannot come to know through any experience, because
experience itself has need of such laws, which lie a priori at the basis of
its possibility.
The possibility of experience in general is thus at the same time the
universal law of nature, and the principles of the former are themselves
the laws of the latter. For we are not acquainted with nature except as
the sum total of appearances, i.e., of the representations in us, and so we
cannot get the laws of their connection from anywhere else except the
principles of their connection in us, i.e., from the conditions of neces-
sary unification in one consciousness, which unification constitutes the
possibility of experience.
Even the main proposition that has been elaborated throughout this
entire part, that universal laws of nature can be cognized a priori, already
leads by itself to the proposition: that the highest legislation for nature
must lie in ourselves, i.e., in our understanding, and that we must not seek
the universal laws of nature from nature by means of experience, but, con-
versely, must seek nature, as regards its universal conformity to law, solely
in the conditions of the possibility of experience that lie in our sensibility
and understanding; for how would it otherwise be possible to become ac-
quainted with these laws a priori, since they are surely not rules of analytic
cognition, but are genuine synthetic amplifications of cognition? Such
agreement, and indeed necessary agreement, between the principles of
possible experience and the laws of the possibility of nature, can come
about only from one of two causes: either these laws are taken from na-
ture by means of experience, or, conversely, nature is derived from the
laws of the possibility of experience in general and is fully identical with
the mere universal lawfulness of experience. The first one contradicts
itself, for the universal laws of nature can and must be cognized a priori
(i.e., independently of all experience) and set at the foundation of all
empirical use of the understanding; so only the second remains. ∗
We must, however, distinguish empirical laws of nature, which al-
ways presuppose particular perceptions, from the pure or universal laws
of nature, which, without having particular perceptions underlying them,
contain merely the conditions for the necessary unification of such per-
ceptions in one experience; with respect to the latter laws, nature and
possible experience are one and the same, and since in possible experience
the lawfulness rests on the necessary connection of appearances in one
experience (without which we would not be able to cognize any object of
the sensible world at all), and so on the original laws of the understand-
ing, then, even though it sounds strange at first, it is nonetheless certain,
if I say with respect to the universal laws of nature: the understanding does
not draw its (a priori) laws from nature, but prescribes them to it.

§ 37

We will elucidate this seemingly daring proposition through an exam-
ple, which is supposed to show: that laws which we discover in objects
of sensory intuition, especially if these laws have been cognized as nec-
essary, are already held by us to be such as have been put there by the
understanding, although they are otherwise in all respects like the laws
of nature that we attribute to experience.

§ 38

If one considers the properties of the circle by which this figure unifies
in a universal rule at once so many arbitrary determinations of the space
within it, one cannot refrain from ascribing a nature to this geometrical
thing. Thus, in particular, two lines that intersect each other and also the
circle, however they happen to be drawn, nonetheless always partition
each other in a regular manner such that the rectangle from the parts of
one line is equal to that from the other. Now I ask: “Does this law lie
in the circle, or does it lie in the understanding?” i.e., does this figure,
independent of the understanding, contain the basis for this law in itself,
or does the understanding, since it has itself constructed the figure in ac-
cordance with its concepts (namely, the equality of the radii), at the same
time insert into it the law that chords cut one another in geometrical
proportion? If one traces the proofs of this law, one soon sees that it can
be derived only from the condition on which the understanding based
the construction of this figure, namely, the equality of the radii. If we
now expand upon this concept so as to follow up still further the unity
of the manifold properties of geometrical figures under common laws,
and we consider the circle as a conic section, which is therefore sub-
ject to the very same fundamental conditions of construction as other
conic sections, we then find that all chords that intersect within these
latter (within the ellipse, the parabola, and the hyperbola) always do so
in such a way that the rectangles from their parts are a not indeed equal,
but always stand to one another in equal proportions. If from there we
go still further, namely to the fundamental doctrines of physical astron-
omy, there appears a physical law of reciprocal attraction, extending to
all material nature, the rule of which is that these attractions b decrease
inversely with the square of the distance from each point of attraction,
exactly as the spherical surfaces into which this force spreads itself in-
crease, something that seems to reside as necessary in the nature of the
things themselves and which therefore is customarily presented as cog-
nizable a priori. As simple as are the sources of this law – in that they
rest merely on the relation of spherical surfaces c with different radii –
the consequence therefrom is nonetheless so excellent with respect to
the variety and regularity of its agreement that not only does it follow
that all possible orbits of the celestial bodies are conic sections, but
also that their mutual relations are such that no other law of attrac-
tion save that of the inverse square of the distances can be conceived as
suitable for a system of the world.

Here then is nature that rests on laws that the understanding cognizes
a priori, and indeed chiefly from universal principles d of the determina-
tion of space. Now I ask: do these laws of nature lie in space, and does
the understanding learn them in that it merely seeks to investigate the
wealth of meaning that lies in space, or do they lie in the understanding
and in the way in which it determines space in accordance with the con-
ditions of the synthetic unity toward which its concepts are one and all
directed? Space is something so uniform, and so indeterminate with re-
spect to all specific properties, that certainly no one will look for a stock
of natural laws within it. By contrast, that which determines space into
the figure of a circle, a cone, or a sphere is the understanding, insofar
as it contains the basis for the unity of the construction of these figures.
The mere universal form of intuition called space is therefore certainly
the substratum of all intuitions determinable upon particular objects,
and, admittedly, the condition for the possibility and variety of those
intuitions lies in this space; but the unity of the objects is determined
solely through the understanding, and indeed according to conditions
that reside in its own nature; and so the understanding is the origin of
the universal order of nature; in that it comprehends all appearances un-
der its own laws and thereby first brings about experience a priori (with
respect to its form), in virtue of which everything that is to be cognized
only through experience is necessarily subject to its laws. For we are
not concerned with the nature of the things in themselves, which is in-
dependent of the conditions of both our senses and understanding, but
with nature as an object of possible experience, and here the understand-
ing, since it makes experience possible, at the same time makes it that
the sensible world is either not an object of experience at all, or else is
nature.

§ 39

Appendix to pure natural science

On the system of categories

Nothing can be more desirable to a philosopher than to be able to de-
rive, a priori from one principle, a the multiplicity of concepts or basic
principles b that previously had exhibited themselves to him piecemeal
in the use he had made of them in concreto, and in this way to be able to
unite them all in one cognition. Previously, he believed simply that what
was left to him after a certain abstraction, and that appeared, through
mutual comparison, to form a distinct kind of cognitions, had been com-
pletely assembled: but this was only an aggregate; now he knows that
only precisely so many, not more, not fewer, can constitute this c kind of
cognition, and he has understood the necessity of his division: this is a
comprehending, d and only now does he have a system.

To pick out from ordinary cognition the concepts that are not based
on any particular experience and yet are present in all cognition from
experience (for which they constitute as it were the mere form of con-
nection) required no greater reflection or more insight than to cull from
a language rules for the actual use of words in general, and so to com-
pile the elements for a grammar (and in fact both investigations are very
closely related to one another) without, for all that, even being able to
give a reason why any given language should have precisely this and no
other formal constitution, and still less why precisely so many, neither
more nor fewer, of such formal determinations of the language can be
found at all.

Aristotle had compiled ten such pure elementary concepts under the
name of categories. ∗ To these, which were also called predicaments, he
later felt compelled to append five post-predicaments, ∗∗ some of which
(like prius, simul, motus) are indeed already found in the former; but this
rhapsody 59 could better pass for, and be deserving of praise as, a hint for
future inquirers than as an idea worked out according to rules, and so
with the greater enlightenment of philosophy it too has been rejected as
completely useless.
During an investigation of the pure elements of human cognition
(containing nothing empirical), I was first of all able after long reflection
to distinguish and separate with reliability the pure elementary concepts
of sensibility (space and time) from those of the understanding. By this
means the seventh, eighth, and ninth categories were now excluded from
the above list. The others could be of no use to me, because no principle c
was available whereby the understanding could be fully surveyed and all
of its functions, from which its pure concepts arise, determined exhaus-
tively and with precision.
In order, however, to discover such a principle, d I cast about for an
act of the understanding that contains all the rest and that differentiates
itself only through various modifications or moments in order to bring
the multiplicity of representation under the unity of thinking in general;
and there I found that this act of the understanding consists in judg-
ing. Here lay before me now, already finished though not yet wholly
free of defects, the work of the logicians, through which I was put in
∗
1. Substantia. 2. Qualitas. 3. Quantitas. 4. Relatio. 5. Actio. 6. Passio. 7. Quando.
8. Ubi. 9. Situs. 10. Habitus. a
∗ ∗ Oppositum, Prius, Simul, Motus, Habere. b
a
b
c
d
Substance, quality, quantity, relation, action, affection, time, place, position, state.
Opposition, priority, simultaneity, motion, possession.
Princip
Princip
115
4: 323Prolegomena to any future metaphysics
4: 324
4: 325
the position to present a complete table of pure functions of the un-
derstanding, which were however undetermined with respect to every
object. Finally, I related these functions of judging to objects in gen-
eral, or rather to the condition for determining judgments as objectively
valid, and there arose pure concepts of the understanding, about which I
could have no doubt that precisely these only, and of them only so many,
neither more nor fewer, can make up our entire cognition of things out
of the bare understanding. As was proper, I called them categories, af-
ter their ancient name, whereby I reserved for myself to append in full,
under the name of predicables, all the concepts derivable from them –
whether by connecting them with one another, or with the pure form of
appearance (space and time) or its matter, provided the latter is not yet
determined empirically (the object of sensation in general) – just as soon
as a system of transcendental philosophy should be achieved, on behalf of
which I had, at the time, been concerned only with the critique of reason
itself.
The essential thing, however, in this system of categories, by which
it is distinguished from that ancient rhapsody (which proceeded without
any principle), a and in virtue of which it alone deserves to be counted
as philosophy, consists in this: that through it b the true signification of
the pure concepts of the understanding and the condition of their use
could be exactly determined. For here it became apparent that the pure
concepts of the understanding are, of themselves, nothing but logical
functions, but that as such they do not constitute the least concept of an
object in itself but rather need sensory intuition as a basis, and even then
they serve only to determine empirical judgments, which are otherwise
undetermined and indifferent with respect to all the functions of judging,
with respect to those functions, so as to procure universal validity for
them, and thereby to make judgments of experience possible in general.
This sort of insight into the nature of the categories, which would
at the same time restrict their use merely to experience, never occurred
to their first originator, or to anyone after him; but without this insight
(which depends precisely on their derivation or deduction), they are
completely useless and are a paltry list of names, without explanation
or rule for their use. Had anything like it ever occurred to the ancients,
then without doubt the entire study of cognition through pure reason,
which under the name of metaphysics has ruined so many good minds
over the centuries, would have come down to us in a completely different
form and would have enlightened the human understanding, instead of,
as has actually happened, exhausting it in murky and vain ruminations
and making it unserviceable for true science.

This system of categories now makes all treatment of any object of
pure reason itself systematic in turn, and it yields an undoubted instruc-
tion or guiding thread as to how and through what points of inquiry any
metaphysical contemplation must be directed if it is to be complete; for
it exhausts all moments of the understanding, under which every other
concept must be brought. Thus too has arisen the table of principles,
of whose completeness we can be assured only through the system of
categories; and even in the division of concepts that are supposed to go
beyond the physiological use of the understanding (Critique, p. 344, also
p. 415), 60 there is always the same guiding thread, which, since it always
must be taken through the same fixed points determined a priori in the
human understanding, forms a closed circle every time, leaving no room
for doubt that the object of a pure concept of the understanding or rea-
son, insofar as it is to be examined philosophically and according to a
priori principles, can be cognized completely in this way. I have not even
been able to refrain from making use of this guide with respect to one
of the most abstract of ontological classifications, namely the manifold
differentiation of the concepts of something and nothing, and accordingly
from achieving a rule-governed and necessary table (Critique, p. 292). ∗61
This very system, like every true system founded on a universal
principle, c also exhibits its inestimable usefulness in that it expels all the
∗ All sorts of nice notes can be made on a laid-out table of categories, such as:
1. that the third arises from the first and second, conjoined into one concept,
2. that in those for quantity and quality there is merely a progression a from
Unity to Totality, or from something to nothing (for this purpose the categories
of quality must stand thus: Reality, Limitation, full Negation), without correlata
or opposita, while those of relation and modality carry the latter with them,
3. that, just as in the logical table, categorical judgments underlie all the others,
so the category of substance underlines all concepts of real things, 4. that,
just as modality in a judgment is not a separate predicate, so too the modal
concepts b do not add a determination to things, and so on. Considerations
such as these all have their great utility. If beyond this all the predicables are
enumerated – they can be extracted fairly completely from any good ontology
(e.g., Baumgarten’s) 62 – and if they are ordered in classes under the categories
(in which one must not neglect to add as complete an analysis as possible of all
these concepts), then a solely analytical part of metaphysics will arise, which
as yet contains no synthetic proposition whatsoever and could precede the
second (synthetic) part, and, through its determinateness and completeness,
might have not only utility, but beyond that, in virtue of its systematicity, a
certain beauty. 63
a Reading fortgehe for forgehen, with Vorländer.
Reading Modalbegriffe for Modelbegriffe, with Ak.
Princip
b
c
117
4: 326Prolegomena to any future metaphysics
extraneous concepts that might otherwise creep in among these pure
concepts of the understanding, and it assigns each cognition its place.
Those concepts that, under the name of concepts of reflection, I had also
put into a table under the guidance of the categories mingle in ontology
with the pure concepts of the understanding without privilege and legit-
imate claims, although the latter are concepts of connection and thereby
of the object itself, whereas the former are only concepts of the mere
comparison of already given concepts, and therefore have an entirely
different nature and use; through my law-governed division (Critique,
p. 260) 64 they are extricated from this amalgam. But the usefulness of
this separated table of categories shines forth yet more brightly if, as
will soon be done, we separate from the categories the table of transcen-
dental concepts of reason, which have a completely different nature and
origin than the concepts of the understanding (so that the table must
also have a different form), a separation that, necessary as it is, has never
occurred in any system of metaphysics, as a result of which a these ideas
of reason and concepts of the understanding run confusedly together as
if they belonged to one family, like siblings, an intermingling that also
could never have been avoided in the absence of a separate system of
categories.

Third Part

How is metaphysics in general possible?

§ 40

Pure mathematics and pure natural science would not have needed, for
the purpose of their own security and certainty, a deduction of the sort
that we have hitherto accomplished for them both; for the first is sup-
ported by its own evidence, whereas the second, though arising from
pure sources of the understanding, is nonetheless supported from ex-
perience and thoroughgoing confirmation by it – experience being a
witness that natural science cannot fully renounce and dispense with,
because, as philosophy, 65 despite all its certainty it can never rival math-
ematics. Neither science had need of the aforementioned investigation
for itself, but for another science, namely metaphysics.
Apart from concepts of nature, which always find their application
in experience, metaphysics is further concerned with pure concepts
of reason that are never given in any possible experience whatsoever,
hence with concepts whose objective reality (that they are not mere
fantasies) and with assertions whose truth or falsity cannot be confirmed
or exposed by any experience; and this part of metaphysics is moreover
precisely that which forms its essential end, toward which all the rest
is only a means – and so this science needs such a deduction for its own
sake. The third question, now put before us, therefore concerns as it
were the core and the characteristic feature of metaphysics, namely,
the preoccupation of reason simply with itself, and that acquaintance a
with objects which is presumed to arise immediately from reason’s
brooding over its own concepts without its either needing mediation
from experience for such an acquaintance, or being able to achieve such
an acquaintance through experience at all.

Without a solution to this question, reason will never be satisfied
with itself. The use in experience to which reason limits the pure under-
standing does not entirely fulfill reason’s own vocation. Each individual
experience is only a part of the whole a sphere of the domain of expe-
rience, but the absolute totality b of all possible experience is not itself an
experience, and yet is still a necessary problem for reason, for the mere
representation of which reason needs concepts entirely different from
the pure concepts of the understanding, whose use is only immanent, i.e.,
refers to experience insofar as such experience can be given, whereas the
concepts of reason extend to the completeness, i.e., the collective unity
of the whole of possible experience, and in that way exceed any given
experience and become transcendent.

Hence, just as the understanding needed the categories for experience,
reason contains in itself the basis for ideas, by which I mean necessary
concepts whose object nevertheless cannot be given in any experience.
The latter are just as intrinsic to the nature of reason as are the former
to that of the understanding; and if the ideas carry with them an illusion
that can easily mislead, this illusion is unavoidable, although it can very
well be prevented “from leading us astray.”
Since all illusion consists in taking the subjective basis for a judgment
to be objective, pure reason’s knowledge of itself in its transcendent
(overreaching) c use will be the only prevention against the errors into
which reason falls if it misconstrues its vocation and, in transcendent
fashion, refers to the object in itself that which concerns only its own
subject and the guidance of that subject in every use that is immanent.

§ 41

The distinction of ideas, i.e., of pure concepts of reason, from categories,
or pure concepts of the understanding, as cognitions of completely dif-
ferent type, origin, and use, is so important a piece of the foundation of
a science which is to contain a system of all these cognitions a priori that,
without such a division, metaphysics is utterly impossible, or at best is
a disorderly and bungling endeavor to patch together a house of cards,
without knowledge of the materials with which one is preoccupied and
of their suitability for one or another end. If the Critique of Pure Reason
had done nothing but first point out this distinction, it would thereby
have already contributed more to elucidating d our conception of, and to
guiding inquiry in, the field of metaphysics, than have all the fruitless
efforts undertaken previously to satisfy the transcendent problems of
pure reason, without it ever being imagined that one may have been sit-
uated in a completely different field from that of the understanding, and
as a result was listing the concepts of the understanding together with
those of reason as if they were of the same kind.

§ 42

All the pure cognitions of the understanding are such that their concepts
can be given in experience and their principles confirmed through expe-
rience; by contrast, the transcendent cognitions of reason neither allow
what relates to their ideas to be given in experience, nor their theses a ever
to be confirmed or refuted through experience; hence, only pure reason
itself can detect the error that perhaps creeps into them, though this is
very hard to do, because this selfsame reason by nature becomes dialec-
tical through its ideas, and this unavoidable illusion cannot be kept in
check through any objective and dogmatic investigation b of things, but
only through a subjective investigation of reason itself, as a source of
ideas.

§ 43

In the Critique I always gave my greatest attention not only to how I
could distinguish carefully the types of cognition, but also to how
I could derive all d the concepts belonging to each type from their
common source, so that I might not only, by learning their origin, be
able to determine their use with certainty, but also have the inestimable
advantage (never yet imagined) of cognizing a priori, hence according
to principles, e the completeness of the enumeration, classification, and
specification of the concepts. Failing this, everything in metaphysics
is nothing but rhapsody, in which one never knows whether what one
has is enough, or whether and where something may still be lacking.
Such an advantage is, of course, available only in pure philosophy, but
it constitutes the essence of that philosophy.
Since I had found the origin of the categories in the four logical
functions of all judgments of the understanding, it was completely
natural to look for the origin of the ideas in the three functions of
syllogisms; f for once such pure concepts of reason (transc. ideas) have
been granted; then, if they are not to be taken for innate, they could
indeed be found nowhere else except in this very act of reason which,
insofar as it relates merely to form, constitutes the logical in syllogisms,
but, insofar as it represents the judgments of the understanding as
determined with respect to one or another a priori form, constitutes the
transcendental concepts of pure reason.
The formal distinction of syllogisms necessitates their division into
categorical, hypothetical, and disjunctive. Therefore the concepts of rea-
son based thereupon contain first, the idea of the complete subject (the
substantial), second, the idea of the complete series of conditions, and
third, the determination of all concepts in the idea of a complete sum
total of the possible. ∗ The first idea was psychological, c the second cos-
mological, the third theological; and since all three give rise to a dialectic,
but each in its own way, all this provided the basis for dividing the entire
dialectic of pure reason into the paralogism, the antinomy, and finally
the ideal of pure reason – through which derivation it is rendered com-
pletely certain that all claims of pure reason are represented here in full,
and not one can be missing, since the faculty of reason itself, whence
they all originate, is thereby fully surveyed.

§ 44

In this examination it is in general further noteworthy: that the ideas
of reason d are not, like the categories, helpful to us in some way in us-
ing the understanding with respect to experience, but are completely
dispensable with respect to such use, nay, are contrary to and obstruc-
tive of the maxims for the cognition of nature through reason, although
they are still quite necessary in another respect, yet to be determined. 66
∗ In disjunctive judgments we consider all possibility as divided with respect to
a certain concept. The ontological principle a of the thoroughgoing determi-
nation of a thing in general (out of all possible opposing predicates, one is
attributed to each thing), which is at the same time the principle of all disjunc-
tive judgments, founds itself upon the sum total of all possibility, in which the
possibility of each thing in general is taken to be determinable. b The follow-
ing helps provide a small elucidation of the above proposition: That the act of
reason in disjunctive syllogisms is the same in form with that by which reason
achieves the idea of a sum total of all reality, which contains in itself the positive
members of all opposing predicates.

In explaining the appearances of the soul, we can be completely indif-
ferent as to whether it is a simple substance or not; for we are unable
through any possible experience to make the concept of a simple being
sensorily intelligible, hence intelligible in concreto; and this concept is
therefore completely empty with respect to all hoped-for insight into
the cause of the appearances, and cannot serve as a principle a of expla-
nation of that which supplies inner or outer experience. Just as little can
the cosmological ideas of the beginning of the world or the eternity of
the world (a parte ante) b help us to explain any event in the world itself.
Finally, in accordance with a correct maxim of natural philosophy, we
must refrain from all explanations of the organization of nature drawn
from the will of a supreme being, c because this is no longer natural phi-
losophy but an admission that we have come to the end of it. These
ideas therefore have a completely different determination of their use
from that of the categories, through which (and through the principles
built upon them) experience itself first became possible. Nevertheless
our laborious analytic of the understanding 67 would have been entirely
superfluous, if our aim had been directed toward nothing other than
mere cognition of nature insofar as such cognition can be given in ex-
perience; for reason conducts its affairs in both mathematics and natural
science quite safely and quite well, even without any such subtle de-
duction; hence our critique of the understanding joins with the ideas of
pure reason for a purpose that lies beyond the use of the understanding
in experience, though we have said above that the use of the under-
standing in this regard is wholly impossible and without object or sig-
nificance. There must nonetheless be agreement between what belongs
to the nature of reason and of the understanding, and the former must
contribute to the perfection of the latter and cannot possibly confuse
it.

The solution to this question is as follows: Pure reason does not,
among its ideas, have in view particular objects that might lie beyond
the field of experience, but it merely demands completeness in the use of
the understanding in the connection of experience. This completeness
can, however, only be a completeness of principles, d but not of intuitions
and objects. Nonetheless, in order to represent these principles deter-
minately, reason conceives of them as the cognition of an object, cog-
nition of which is completely determined with respect to these rules –
though the object is only an idea – so as to bring cognition through
the understanding as close as possible to the completeness that this idea
signifies.

§ 45

Preliminary Remark On the Dialectic of Pure Reason

We have shown above (§§33, 34): that the purity of the categories from
all admixture with sensory determinations can mislead reason into ex-
tending their use entirely beyond all experience to things in themselves;
and yet, because the categories are themselves unable to find any in-
tuition that could provide them with significance and sense in concreto,
they cannot in and of themselves provide any determinate concept of
anything at all, though they can indeed, as mere logical functions, rep-
resent a thing in general. Now hyperbolical objects of this kind are what
are called noumena or pure beings of the understanding (better: beings
of thought) a – such as, e.g., substance, but which is thought without persis-
tence in time, or a cause, which would however not act in time, and so on –
because such predicates are attributed to these objects as serve only to
make the lawfulness of experience possible, and yet they are nonetheless
deprived of all the conditions of intuition under which alone experi-
ence is possible, as a result of which the above concepts again lose all
significance.

There is, however, no danger that the understanding will of itself
wantonly stray beyond its boundaries into the field of mere beings of
thought, without being urged by alien laws. But if reason, which can
never be fully satisfied with any use of the rules of the understanding in
experience because such use is always still conditioned, requires com-
pletion of this chain of conditions, then the understanding is driven out
of its circle, in order partly to represent the objects of experience in a
series stretching so far that no experience can comprise the likes of it,
partly (in order to complete the series) even to look for noumena entirely
outside said experience to which reason can attach the chain and in that
way, independent at last of the conditions of experience, nonetheless can
make its hold complete. These then are the transcendental ideas, which,
although in accordance with the true but hidden end of the natural deter-
mination of our reason they may be aimed not at overreaching concepts
but merely at the unbounded expansion of the use of concepts in experi-
ence, may nonetheless, through an unavoidable illusion, elicit from the
understanding a transcendent use, which, though deceitful, nonetheless
cannot be curbed by any resolve to stay within the bounds of experience,
but only through scientific instruction and hard work.

§ 46

I. Psychological ideas (Critique, pp. 341ff.)

It has long been observed that in all substances the true subject – namely
that which remains after all accidents (as predicates) have been removed –
and hence the substantial itself, is unknown to us; and various complaints
have been made about these limits to our insight. But it needs to be said
that human understanding is not to be blamed because it does not know
the substantial in things, i.e., cannot determine it by itself, but rather
because it wants to cognize determinately, like an object that is given,
what is only an idea. Pure reason demands that for each predicate of a
thing we should seek its appropriate subject, but that for this subject,
which is in turn necessarily only a predicate, we should seek its subject
again, and so forth to infinity (or as far as we get). But from this it
follows that we should take nothing that we can attain for a final subject,
and that the substantial itself could never be thought by our ever-so-
deeply penetrating understanding, even if the whole of nature were laid
bare before it; for the specific nature of our understanding consists in
thinking everything discursively, i.e., through concepts, hence through
mere predicates, among which the absolute subject must therefore always
be absent. Consequently, all real properties by which we cognize bodies
are mere accidents for which we lack a subject – even impenetrability,
which must always be conceived only as the effect of a force.
Now it does appear as if we have something substantial in the con-
sciousness of ourselves (i.e., in the thinking subject), and indeed have it
in immediate intuition; for all the predicates of inner sense are referred
to the I as subject, and this I cannot again be thought as the predicate of
some other subject. It therefore appears that in this case completeness in
referring the given concepts to a subject as predicates is not a mere idea,
but that the object, namely the absolute subject itself, is given in experi-
ence. But this expectation is disappointed. For the I is not a concept ∗ at
all, but only a designation of the object of inner sense insofar as we do
not further cognize it through any predicate; hence although it cannot
itself be the predicate of any other thing, just as little can it be a deter-
minate concept of an absolute subject, but as in all the other cases it can
only be the referring of inner appearances to their unknown subject.
∗ If the representation of apperception, the I, were a concept through which
anything might be thought, it could then be used as a predicate for other
things, or contain such predicates in itself. But it is nothing more than a feeling
of an existence without the least concept, and is only a representation of that
to which all thinking stands in relation (relatione accidentis). a
a “relation of accident”

Nevertheless, through a wholly natural misunderstanding, this idea
(which, as a regulative principle, serves perfectly well to destroy com-
pletely all materialistic explanations of the inner appearances of our
soul) a gives rise to a seemingly plausible argument for inferring the
nature of our thinking being from this presumed cognition of the sub-
stantial in it, inasmuch as knowledge of its nature falls completely outside
the sum total of experience.

§ 47

This thinking self (the soul), as the ultimate subject of thinking, which
cannot itself be represented as the predicate of another thing, may now
indeed be called substance: but this concept nonetheless remains com-
pletely empty and without any consequences, if persistence (as that which
renders the concept of substances fertile within experience) cannot be
proven of it.

Persistence, however, can never be proven from the concept of a
substance as a thing in itself, but only for the purposes of experience.
This has been sufficiently established in the first Analogy of Experience
(Critique, p. 182); 70 and anyone who will not grant this proof can test for
themselves whether they succeed in proving, from the concept of a sub-
ject that does not exist as the predicate of another thing, that the existence
of that subject is persistent throughout, and that it can neither come into
being nor pass away, either in itself or through any natural cause. Syn-
thetic a priori propositions of this type can never be proven in themselves,
but only in relation to things as objects of a possible experience.

§ 48

If, therefore, we want to infer the persistence of the soul from the concept
of the soul as substance, this can be valid of the soul only for the purpose of
possible experience, and not of the soul as a thing in itself and beyond all
possible experience. But life is the subjective condition of all our possible
experience: consequently, only the persistence of the soul during life can
be inferred, for the death of a human being is the end of all experience
as far as the soul as an object of experience is concerned (provided that
the opposite has not been proven, which is the very matter in question).
Therefore the persistence of the soul can be proven only during the life
of a human being (which proof will doubtless be granted us), but not
after death (which is actually our concern) – and indeed then only from
the universal ground that the concept of substance, insofar as it is to
be considered as connected necessarily with the concept of persistence,
can be so connected only in accordance with a principle of possible
experience, and hence only for the purpose of the latter.

§ 49

That our outer perceptions not only do correspond to something real
outside us, but must so correspond, also can never be proven as a connec-
tion of things in themselves, but can well be proven for the purpose of
experience. This is as much as to say: it can very well be proven that there
is something outside us of an empirical kind, and hence as appearance in
space; for we are not concerned with objects other than those which be-
long to a possible experience, just because such objects cannot be given to
us in any experience and therefore are nothing for us. Outside me empir-
ically is that which is intuited in space; and because this space, together
with all the appearances it contains, belongs to those representations
whose connection according to laws of experience proves their objective
truth, just as the connection of the appearances of the inner sense proves
the reality c of my soul (as an object of inner sense), it follows that I am,
by means of outer appearances, just as conscious of the reality of bodies
as outer appearances in space, as I am, by means of inner experience,
conscious of the existence of my soul in time – which soul I cognize a
only as an object of inner sense through the appearances constituting an
inner state, and whose being as it is in itself, b which underlies these ap-
pearances, is unknown to me. Cartesian idealism therefore distinguishes
only outer experience from dream, and lawfulness as a criterion of the
truth of the former from the disorder and false illusion of the latter. c In
both cases it presupposes space and time as conditions for the existence
of objects and merely asks whether the objects of the outer senses are
actually d to be found in the space in which we put them while awake,
in the way that the object of inner sense, the soul, actually is in time,
i.e., whether experience carries with itself sure criteria to distinguish it
from imagination. Here the doubt can easily be removed, and we always
remove it in ordinary life by investigating the connection of appearances
in both space and time according to universal laws of experience, and
if the representation of outer things consistently agrees therewith, we
cannot doubt that those things should not constitute truthful experience.
Because appearances are considered as appearances only in accordance
with their connection within experience, material idealism can therefore
very easily be removed; and it is just as secure an experience that bodies
exist outside us (in space) as that I myself exist in accordance with the
representation of inner sense (in time) – for the concept: outside us, sig-
nifies only existence in space. Since, however, the I in the proposition
I am does not signify merely the object of inner intuition (in time) but
also the subject of consciousness, just as body does not signify merely
outer intuition (in space) but also the thing in itself that underlies this ap-
pearance, accordingly the question of whether bodies (as appearances of
outer sense) exist outside my thought as bodies in nature e can without hes-
itation be answered negatively; but here matters do not stand otherwise
for the question of whether I myself as an appearance of inner sense (the
soul according to empirical psychology) exist in time outside my power
of representation, for this question must also be answered negatively. In
this way everything is, when reduced to its true signification, conclusive
and certain. Formal idealism (elsewhere called transcendental idealism
by me) actually destroys f material or Cartesian idealism. For if space is
nothing but a form of my sensibility, then it is, as a representation in me,
just as real a as I am myself, and the only question remaining concerns
the empirical truth of the appearances in this space. If this is not the case,
but rather space and the appearances in it are something existing outside
us, then all the criteria of experience can never, outside our perception,
prove the reality of these objects outside us.

§ 50

II. Cosmological ideas (Critique, pp. 405 ff.)

This product of pure reason in its transcendent use is its most remarkable
phenomenon, and it works the most strongly of all to awaken philosophy
from its dogmatic slumber, and to prompt it toward the difficult business
of the critique of reason itself.
I call this idea cosmological because it always finds its object only in
the sensible world and needs no other world than that whose object b is
an object c for the senses, and so, thus far, is immanent d and not tran-
scendent, and therefore, up to this point, is not yet an idea; by contrast,
to think of the soul as a simple substance already amounts to thinking of
it as an object (the simple) the likes of which cannot be represented at all
to the senses. Notwithstanding all that, the cosmological idea expands
the connection of the conditioned with its condition (be it mathematical
or dynamic) so greatly that experience can never match it, and therefore
it is, with respect to this point, always an idea whose object can never be
adequately given in any experience whatever.

§ 51

In the first place, the usefulness of a system of categories is here revealed
so clearly and unmistakably that even if there were no further grounds
of proof of that system, this alone would sufficiently establish their in-
dispensability in the system of pure reason. There are no more than four
such transcendental e ideas, as many as there are classes of categories; in
each of them, however, they refer only to the absolute completeness of
the series of conditions for a given conditioned. In accordance with these
cosmological ideas there are also only four kinds of dialectical assertions
of pure reason, which show themselves to be dialectical because for each
such assertion a contradictory one stands in opposition in accordance
with equally plausible principles of pure reason, a conflict that cannot
be avoided by any metaphysical art of the most subtle distinctions, but
that requires the philosopher to return to the first sources of pure reason
itself. This antinomy, by no means arbitrarily contrived, but grounded
in the nature of human reason and so unavoidable and neverending,
contains the following four theses together with their antitheses.
1. b
Thesis
The world has, as to time and space,
a beginning (a boundary).
Antithesis
The world is, as to time and space,
infinite.
2.
Thesis
Everything in the world
is constituted out of the
simple. 3.
Thesis
There exist in the world
causes through
freedom.
Antithesis
There is nothing simple,
but everything is
composite. Antithesis
There is no freedom,
but everything is
nature.
4.
Thesis
In the series of causes in the world there is a
necessary being.
Antithesis
There is nothing necessary in this series, but in it
everything is contingent.

§ 52a
4: 340
Here is now the strangest phenomenon of human reason, no other ex-
ample of which can be pointed to in any of its other uses. If (as normally
happens) we think of the appearances of the sensible world as things in
themselves, if we take the principles of their connection to be princi-
ples that are universally valid for things in themselves and not merely
for experience (as is just as common, nay, is unavoidable without our
Critique): then an unexpected conflict comes to light, which can never
be settled in the usual dogmatic manner, since both thesis and antithe-
sis can be established through equally evident, clear, and incontestable
proofs – for I will vouch for the correctness of all these proofs – and
therefore reason is seen to be divided against itself, a situation that makes
the skeptic rejoice, but must make the critical philosopher pensive and
uneasy.

§ 52b

One can tinker around with metaphysics in sundry ways without even
suspecting that one might be venturing into untruth. For if only we do
not contradict ourselves – something that is indeed entirely possible with
synthetic, though completely fanciful, propositions – then we can never
be refuted by experience in all such cases where the concepts we connect
are mere ideas, which can by no means be given (in their entire content) in
experience. For how would we decide through experience: Whether the
world has existed from eternity, or has a beginning? Whether matter is
infinitely divisible, or is constituted out of simple parts? Concepts such as
these cannot be given in any experience (even the greatest possible), and
so the falsity of the affirmative or negative thesis cannot be discovered
through that touchstone.
The single possible case in which reason would reveal (against its
will) its secret dialectic (which it falsely passes off as dogmatics) would
be that in which it based an assertion on a universally a acknowledged
principle, and, with the greatest propriety in the mode of inference,
derived the direct opposite from another equally accredited principle.
Now this case is here actual, and indeed is so with respect to four natural
ideas of reason, from which there arise – each with proper consistency
and from universally acknowledged principles – four assertions on one
side and just as many counterassertions on the other, thereby revealing
the dialectical illusion of pure reason in the use of these principles, which
otherwise would have had to remain forever hidden.
Here is, therefore, a decisive test, which must necessarily disclose to us
a fault that lies hidden in the presuppositions of reason. ∗ Of two mutually
∗
I therefore desire that the critical reader concern himself mainly with this
antinomy, because nature itself seems to have set it up to make reason suspicious
in its bold claims and to force a self-examination. I promise to answer for each
proof I have given of both thesis b and antithesis, c and thereby to establish the
certainty of the inevitable antinomy of reason. If the reader is induced, through
this strange phenomenon, to reexamine the presupposition that underlies it,
he will then feel constrained to investigate more deeply with me the primary
foundation of all cognition through pure reason.
131
4: 341Prolegomena to any future metaphysics
contradictory propositions, both cannot be false save when the concept
underlying them both is itself contradictory; e.g., the two propositions:
a square circle is round, and: a square circle is not round, are both false.
For, as regards the first, it is false that the aforementioned circle is round,
since it is square; but it is also false that it is not round, i.e., has corners,
since it is a circle. The logical mark of the impossibility of a concept
consists, then, in this: that under the presupposition of this concept,
two contradictory propositions would be false simultaneously; and since
between these two no third proposition can be thought, through this
concept nothing at all is thought.
§ 52c
4: 342
Now underlying the first two antinomies, which I call mathematical
because they concern adding together or dividing up the homogeneous,
is a contradictory concept of this type; and by this means I explain how
it comes about that thesis and antithesis are false in both.
If I speak of objects in time and space, I am not speaking of things
in themselves (since I know nothing of them), but only of things in
appearance, i.e., of experience as a distinct mode of cognition of objects
that is granted to human beings alone. I must not say of that which I
think in space or time: that it is in itself in space and time, independent
of this thought of mine; for then I would contradict myself, since space
and time, together with the appearances in them, are nothing existing
in themselves and outside my representations, but are themselves only
modes of representation, and it is patently contradictory to say of a mere
mode of representation that it also exists outside our representation. The
objects of the senses therefore exist only in experience; by contrast, to
grant them a self-subsistent existence of their own, without experience
or prior to it, is as much as to imagine that experience is also real without
experience or prior to it.
Now if I ask about the magnitude of the world with respect to space
and time, for all of my concepts it is just as impossible to assert that it is
infinite as that it is finite. For neither of these can be contained in expe-
rience, because it is not possible to have experience either of an infinite
space or infinitely flowing time, or d of a bounding of the world by an
empty space or by an earlier, empty time; these are only ideas. There-
fore the magnitude of the world, determined one way or the other, must
a
b
c
d
Reading allgemein for allgemeinen, with Ak.
Thesis
Antithesis
Reading noch for nach, with Ak.
132Prolegomena to any future metaphysics
lie in itself, apart from all experience. But this contradicts the concept
of a sensible world, which is merely a sum total of appearance, whose
existence and connection takes place only in representation, namely in
experience, since it is not a thing in itself, a but is itself nothing but a
kind of representation. From this it follows that, since the concept of a
sensible world existing for itself is self-contradictory, any solution to this
problem as to its magnitude will always be false, whether the attempted
solution be affirmative or negative.
The same holds for the second antinomy, which concerns dividing
up the appearances. For these appearances are mere representations,
and the parts exist only in the representation of them, hence in the
dividing, i.e., in a possible experience in which they are given, and the
dividing therefore proceeds only as far as possible experience reaches. To
assume that an appearance, e.g., of a body, contains within itself, before
all experience, all of the parts to which possible experience can ever attain,
means: to give to a mere appearance, which can exist only in experience,
at the same time an existence of its own previous to experience, which is to
say: that mere representations are present before they are encountered
in the representational power, which contradicts itself and hence also
contradicts every solution to this misunderstood problem, whether that
solution asserts that bodies in themselves consist of infinitely many parts
or of a finite number of simple parts.

§ 53

In the first (mathematical) class of antinomy, the falsity of the presup-
position consisted in the following: that something self-contradictory
(namely, appearance as a thing in itself) b would be represented as being
unifiable in a concept. But regarding the second, namely the dynamical,
class of antinomy, the falsity of the presupposition consists in this: that
something that is unifiable is represented as contradictory; consequently,
while in the first case both of the mutually opposing assertions were false,
here on the contrary the assertions, which are set in opposition to one
another through mere misunderstanding, can both be true.
Specifically, mathematical combination necessarily presupposes the
homogeneity of the things combined (in the concept of magnitude), but
dynamical connection does not require this at all. If it is a question of
the magnitude of something extended, all parts must be homogeneous
among themselves and with the whole; by contrast, in the connection of
cause and effect homogeneity can indeed be found, but is not necessary;
for the concept of causality (whereby through one thing, something
completely different from it is posited) at least does not require it.
If the objects of the sensible world were taken for things in themselves,
and the previously stated natural laws for laws of things in themselves,
contradiction would be unavoidable. In the same way, if the subject of
freedom were represented, like the other objects, as a mere appearance,
contradiction could again not be avoided, for the same thing would be
simultaneously affirmed and denied of the same object in the same sense.
But if natural necessity is referred only to appearances and freedom only
to things in themselves, then no contradiction arises if both kinds of
causality are assumed or conceded equally, however difficult or impos-
sible it may be to make causality of the latter kind conceivable.
Within appearance, every effect is an event, or something that hap-
pens in time; the effect must, in accordance with the universal law of
nature, be preceded by a determination of the causality of its cause (a
state of the cause), from which the effect follows in accordance with a
constant law. But this determination of the cause to causality must also
be something that occurs or takes place; the cause must have begun to act,
for otherwise no sequence in time could be thought between it and the
effect. Both the effect and the causality of the cause would have always ex-
isted. Therefore the determination of the cause to act must also have arisen
among the appearances, and so must, like its effect, be an event, which
again must have its cause, and so on, and hence natural necessity must be
the condition in accordance with which efficient causes are determined.
Should, by contrast, freedom be a property of certain causes of appear-
ances, then that freedom must, in relation to the appearances as events,
be a faculty of starting those events from itself (sponte), a i.e., without the
causality of the cause itself having to begin, and hence without need for
any other ground to determine its beginning. But then the cause, as to its
causality, would not have to be subject to temporal determinations of its
state, i.e., would not have to be appearance at all, i.e., would have to be
taken for a thing in itself, and only the effects would have to be taken for
appearances. ∗ If this sort of influence of intelligible beings on appearances
∗
The idea of freedom has its place solely in the relation of the intellectual, b as
cause, to the appearance, as effect. Therefore we cannot bestow freedom upon
matter in consideration of the unceasing activity by which it fills its space,
even though this activity occurs through an inner principle. We can just as
little find any concept of freedom to fit a purely intelligible being, e.g., God,
insofar as his action is immanent. For his action, although independent of
causes determining it from outside, nevertheless is determined in his eternal
reason, hence in the divine nature. Only if something should begin through an
action, hence the effect be found in the time series, and so in the sensible
world (e.g., the beginning of the world), does the question arise of whether the
causality of the cause must itself also have a beginning, or whether the cause
134Prolegomena to any future metaphysics
can be thought without contradiction, then natural necessity will indeed
attach to every connection of cause and effect in the sensible world,
and yet that cause which is itself not an appearance (though it under-
lies appearance) will still be entitled to freedom, and therefore nature
and freedom will be attributable without contradiction to the very same
thing, but in different respects, in the one case as appearance, in the
other as a thing in itself.
We have in us a faculty that not only stands in connection with its
subjectively determining grounds, which are the natural causes of its
actions – and thus far is the faculty of a being which itself belongs to
appearances – but that also is related to objective grounds that are mere
ideas, insofar as these ideas can determine this faculty, a connection that
is expressed by ought. c This faculty is called reason, and insofar as we are
considering a being (the human being) solely as regards this objectively
determinable reason, this being cannot be considered as a being of the
senses; rather, the aforesaid property is the property of a thing in itself,
and the possibility of that property – namely, how the ought, which has
never yet happened, can determine the activity of this being and can be
the cause of actions whose effect is an appearance in the sensible world –
we cannot comprehend at all. Yet the casuality of reason with respect to
effects in the sensible world would nonetheless be freedom, insofar as
objective grounds, which are themselves ideas, are taken to be determining
with respect to that causality. For the action of that causality would in
that case not depend on any subjective, hence also not on any temporal
conditions, and would therefore also not depend on the natural law that
serves to determine those conditions, because grounds of reason provide
the rule for actions universally, from principles, without influence from
the circumstances of time or place.
What I adduce here counts only as an example, for intelligibility, and
does not belong necessarily to our question, which must be decided from
mere concepts independently of properties that we find in the actual
world.
I can now say without contradiction: all actions of rational beings,
insofar as they are appearances (are encountered in some experience or
can originate an effect without its causality itself having a beginning. In the first
case the concept of this causality is a concept of natural necessity, in the second
of freedom. From this the reader will see that, since I have explained freedom
as the faculty to begin an event by oneself, I have exactly hit that concept which
is the problem of metaphysics.
a
b
c
“spontaneously”
des Intellektuellen
Sollen
135Prolegomena to any future metaphysics
4: 346
other), are subject to natural necessity; but the very same actions, with
respect only to the rational subject and its faculty of acting in accordance
with bare reason, are free. What, then, is required for natural necessity?
Nothing more than the determinability of every event in the sensible
world according to constant laws, and therefore a relation to a cause
within appearance; whereby the underlying thing in itself and its causality
remain unknown. But I say: the law of nature remains, whether the rational
being be a cause of effects in the sensible world through reason and hence
through freedom, or whether that being does not determine such effects
through rational grounds. For if the first is the case, the action takes place
according to maxims whose effect within appearance will always conform
to constant laws; if the second is the case, and the action does not take
place according to principles of reason, then it is subject to the empirical
laws of sensibility, and in both cases the effects are connected according
to constant laws; but we require nothing more for natural necessity, and
indeed know nothing more of it. In the first case, however, reason is the
cause of these natural laws and is therefore free, in the second case the
effects flow according to mere natural laws of sensibility, because reason
exercises no influence on them; but, because of this, reason is not itself
determined by sensibility (which is impossible), and it is therefore also
free in this case. Therefore freedom does not impede the natural law of
appearances, any more than this law interferes with the freedom of the
practical use of reason, a use that stands in connection with things in
themselves as determining grounds.
In this way practical freedom – namely, that freedom in which rea-
son has causality in accordance with objective determining grounds – is
rescued, without natural necessity suffering the least harm with respect
to the very same effects, as appearances. This can also help elucidate
what we have had to say about transcendental freedom and its unifica-
tion with natural necessity (in the same subject, but not taken in one and
the same respect). For, as regards transcendental freedom, any beginning
of an action of a being out of objective causes is always, with respect to
these determining grounds, a first beginning, although the same action
is, in the series of appearances, only a subalternate beginning, which has
to be preceded by a state of the cause which determines that cause, and
which is itself determined in the same way by an immediately preceding
cause: so that in rational beings (or in general in any beings, provided
that their causality is determined in them as things in themselves) one
can conceive of a faculty for beginning a series of states spontaneously,
without falling into contradiction with the laws of nature. For the re-
lation of an action to the objective grounds of reason is not a tempo-
ral relation; here, that which determines the causality does not precede
the action as regards time, because such determining grounds do not
represent the relation of objects to the senses (and so to causes within
136Prolegomena to any future metaphysics
appearance), but rather they represent determining causes as things in
themselves, which are not subject to temporal conditions. Hence the
action can be regarded as a first beginning with respect to the causal-
ity of reason, but can nonetheless at the same time be seen as a mere
subordinated beginning with respect to the series of appearances, and
can without contradiction be considered in the former respect as free,
in the latter (since the action is mere appearance) as subject to natural
necessity.
As regards the fourth antinomy, it is removed in a similar a manner
as was the conflict of reason with itself in the third. For if only the
cause in the appearances is distinguished from the cause of the appearances
insofar as the latter cause can be thought as a thing in itself, then these
two propositions can very well exist side by side, as follows: that there
occurs no cause of the sensible world (in accordance with similar laws
of causality) whose existence is absolutely necessary, as also on the other
side: that this world is nonetheless connected with a necessary being
as its cause (but of another kind and according to another law) – the
inconsistency of these two propositions resting solely on the mistake of
extending what holds merely for appearances to things in themselves,
and in general of mixing the two of these up into one concept.

§ 54

This then is the statement and solution of the whole antinomy in which
reason finds itself entangled in the application of its principles b to the
sensible world, and of which the former (the mere statement) even by
itself would already be of considerable benefit toward a knowledge c of
human reason, even if the solution of this conflict should not yet fully
satisfy d the reader, who has here to combat a natural illusion that has
only recently been presented to him as such, after he had hitherto always
taken that illusion for the truth. One consequence of all this is, indeed,
unavoidable; namely, that since it is completely impossible to escape from
this conflict of reason with itself as long as the objects of the sensible
world are taken for things in themselves e – and not for what they in fact
are, that is, for mere appearances – the reader is obliged, for that reason,
to take up once more the deduction of all our cognition a priori (and
the examination of that deduction which I have provided), in order to
come to a decision about it. For the present I do not require more; for
if, through this pursuit, he has first thought himself deeply enough into
the nature of pure reason, then the concepts by means of which alone
the solution to this conflict of reason is possible will already be familiar
to him, a circumstance without which I cannot expect full approbation
from even the most attentive reader.

§ 55

III. Theological idea (Critique, pp. 571ff.)

The third transcendental idea, which provides material for the most im-
portant among all the uses of reason – but one that, if pursued merely
speculatively, is overreaching (transcendent) and thereby dialectical – is
the ideal of pure reason. Here reason does not, as with the psychological
and the cosmological idea, start from experience and become seduced
by the ascending sequence of grounds into aspiring, if possible, to ab-
solute completeness in their series, but instead breaks off entirely from
experience and descends from bare concepts of what would constitute
the absolute completeness of a thing in general – and so by means of
the idea of a supremely perfect first being a – to determination of the
possibility, hence the reality, of all other things; in consequence, here
the bare presupposition of a being that, although not in the series of
experiences, is nonetheless thought on behalf of experience, for the sake
of comprehensibility in the connection, ordering, and unity of that ex-
perience – i.e., the idea – is easier to distinguish from the concept of the
understanding than in the previous cases. Here therefore the dialectical
illusion, which arises from our taking the subjective conditions of our
thinking for objective conditions of things themselves b and our taking a
hypothesis that is necessary for the satisfaction of our reason for a dogma,
is easily exposed, and I therefore need mention nothing more about the
presumptions of transcendental theology, since what the Critique says
about them is clear, evident, and decisive.

§ 56

General Note to the Transcendental Ideas

The objects that are given to us through experience are incomprehen-
sible to us in many respects, and there are many questions to which
natural law carries us, which, if pursued to a certain height (yet al-
ways in conformity with those laws) cannot be solved at all; e.g., how
pieces of matter attract one another. But if we completely abandon na-
ture, or transcend a all possible experience in advancing the connection
of nature and so lose ourselves in mere ideas, then we are unable to say
that the object b is incomprehensible to us and that the nature of things
presents us with unsolvable problems; for then we are not concerned with
nature or in general with objects c that are given, but merely with con-
cepts that have their origin solely in our reason, and with mere beings of
thought, with respect to which all problems, which must originate from
the concepts of those very beings, can be solved, since reason certainly
can and must be held fully accountable for its own proceedings. ∗ Be-
cause the psychological, d cosmological, and theological ideas are nothing
but pure concepts of reason, which cannot be given in any experience,
the questions that reason puts before us with respect to them are not
set for us through objects, but rather through mere maxims of reason
for the sake of its self-satisfaction, and these questions must one and
all be capable of sufficient answer – which occurs by its being shown
that they are principles for bringing the use of our understanding into
thoroughgoing harmony, completeness, and synthetic unity, and to that
extent are valid only for experience, though in the totality of that expe-
rience. But although an absolute totality of experience is not possible,
nonetheless the idea of a totality of cognition according to principles
in general is what alone can provide it with a special kind of unity,
namely that of a system, without which unity our cognition is noth-
ing but piecework and cannot be used for the highest end (which is
nothing other than the system of all ends); and here I mean not only
the practical use of reason, but also the highest end of its speculative
use.

Therefore the transcendental ideas express the peculiar vocation of
reason, namely to be a principle a of the systematic unity of the use of
the understanding. But if one looks upon this unity of mode of cognition
as if it were inhering in the object of cognition, if one takes that which
really is only regulative to be constitutive, and becomes convinced that by
means of these ideas one’s knowledge b can be expanded far beyond all
possible experience, hence can be expanded transcendently, even though
this unity serves only to bring experience in itself as near as possible to
completeness (i.e., to have its advance constrained by nothing that cannot
belong to experience), then this is a mere misunderstanding in judging
the true vocation of our reason and its principles, and it is a dialectic,
which partly confounds the use of reason in experience, and partly divides
reason against itself.

Conclusion

On determining the Boundary of Pure Reason

§ 57

After the extremely clear proofs we have given above, it would be an
absurdity for us, with respect to any object, to hope to cognize more
than belongs to a possible experience of it, or for us, with respect to any
thing that we assume not to be an object of possible experience, to claim
even the least cognition for determining it according to its nature as it is
in itself; for by what means will we reach this determination, since time,
space, and all the concepts of the understanding, and especially the con-
cepts drawn from empirical intuition or perception in the sensible world,
do not and cannot have any other use than merely to make experience
possible, and if we relax this condition even for the pure concepts of the
understanding, they then determine no object whatsoever, and have no
significance anywhere.
But, on the other hand, it would be an even greater absurdity for us
not to allow any things in themselves at all, or for us to want c to pass
off our experience for the only possible mode of cognition of things –
hence our intuition in space and time for the only possible intuition
and our discursive understanding for the archetype of every possible
understanding – and so to want to take principles d of the possibility of
experience for universal conditions on things in themselves.

Our principles, a which limit the use of reason to possible experi-
ence alone, could accordingly themselves become transcendent and could
pass off the limits of our reason for limits on the possibility of things
themselves (for which Hume’s Dialogues 74 can serve as an example), if
a painstaking critique did not both guard the boundaries of our reason
even with respect to its empirical use, and set a limit to its pretensions.
Skepticism originally arose from metaphysics and its unpoliced dialectic.
At first this skepticism wanted, solely for the benefit of the use of reason
in experience, to portray everything that surpasses this use as empty and
deceitful; but gradually, as it came to be noticed that it was the very same a
priori principles b which are employed in experience that, unnoticed, had
led still further than experience reaches – and had done so, as it seemed,
with the very same right – then even the principles of experience began
to be doubted. There was no real trouble with this, for sound common
sense c will always assert its rights in this domain; but there did arise a
particular confusion in science, which cannot determine how far (and
why only that far and not further) reason is to be trusted, and this con-
fusion can be remedied and all future relapses prevented only through
a formal determination, derived from principles, of the boundaries for
the use of our reason.

It is true: we cannot provide, beyond all possible experience, any
determinate concept of what things in themselves may be. But we are
nevertheless not free to hold back entirely in the face of inquiries about
those things; for experience never fully satisfies reason; it directs us ever
further back in answering questions and leaves us unsatisfied as regards
their full elucidation, as everyone can sufficiently observe in the dialectic
of pure reason, which for this very reason has its good subjective ground.
Who can bear being brought, as regards the nature of our soul, both to
the point of a clear consciousness of the subject and to the conviction
that the appearances of that subject cannot be explained materialistically,
without asking what then the soul really is, and, if no concept of expe-
rience suffices thereto, without perchance adopting a concept of reason
(that of a simple immaterial d being) just for this purpose, although we can
by no means prove the objective reality of that concept? Who can satisfy
himself with mere cognition through experience in all the cosmological
questions, of the duration and size of the world, of freedom or natural
necessity, since, wherever we may begin, any answer given according
to principles of experience e always begets a new question which also
requires an answer, and for that reason clearly proves the insufficiency
of all physical modes of explanation for the satisfaction of reason? Finally,
who cannot see, from the throughgoing contingency and dependency of
everything that he might think or assume according to principles of ex-
perience, the impossibility of stopping with these, and who does not feel
compelled, regardless of all prohibition against losing himself in tran-
scendent ideas, nevertheless to look for peace and satisfaction beyond
all concepts that he can justify through experience, in the concept of a
being the idea of which indeed cannot in itself be understood as regards
possibility – though it cannot be refuted either, because it pertains to
a mere being of the understanding – an idea without which, however,
reason would always have to remain unsatisfied?
Boundaries (in extended things) always presuppose a space that is
found outside a certain fixed location, and that encloses that location;
limits c require nothing of the kind, but are mere negations that affect
a magnitude insofar as it does not possess absolute completeness. Our
reason, however, sees around itself as it were a space for the cognition of
things in themselves, although it can never have determinate concepts
of those things and is limited to appearances alone.
As long as reason’s cognition is homogeneous, no determinate bound-
aries can be thought for it. In mathematics and natural science, human
reason recognizes limits but not boundaries, i.e., it indeed recognizes
that something lies beyond it to which it can never reach, but not that
it would itself at any point ever complete its inner progression. The
expansion of insight in mathematics, and the possibility of ever-new in-
ventions, goes to infinity; so too does the discovery of new properties
in nature (new forces and laws) through continued experience and the
unification of that experience by reason. But limits here are nonethe-
less unmistakable, for mathematics refers only to appearances, and that
which cannot be an object of sensory intuition, like the concepts of meta-
physics and morals, lies entirely outside its sphere, and it can never lead
there; but it also has no need whatsoever for such concepts. There is
therefore no continuous progress and advancement toward those sci-
ences, or any point or line of contact, as it were. Natural science will
never reveal to us the inside of things, i.e., that which is not appear-
ance but can nonetheless serve as the highest ground of explanation
for the appearances; but it does not need this for its physical expla-
nations; nay, if such were offered to it from elsewhere (e.g., the influ-
ence of immaterial beings), natural science should indeed reject it and
ought by no means bring it into the progression of its explanations,
but should always base its explanations only on that which can be-
long to experience as an object of the senses and which can be brought
into connection with our actual perceptions in accordance with laws of
experience.
But metaphysics, in the dialectical endeavors of pure reason (which
are not initiated arbitrarily or wantonly, but toward which the nature
of reason itself drives), does lead us to the boundaries; and the tran-
scendental ideas, just because they cannot be avoided and yet will never
be realized, serve not only actually to show us the boundaries of rea-
son’s pure use, but also to show us the way to determine such bound-
aries; and that too is the end and use of this natural predisposition of
our reason, which bore metaphysics as its favorite child, whose procre-
ation (as with any other in the world) is to be ascribed not to chance
accident but to an original seed that is wisely organized toward great
ends. For metaphysics, perhaps more than any other science, is, as re-
gards its fundamentals, placed in us by nature itself, and cannot at all be
seen as the product of an arbitrary choice, or as an accidental extension
from the progression of experiences (it wholly separates itself from those
experiences).
Reason, through all of its concepts and laws of the understanding,
which it finds to be adequate for empirical use, and so adequate within the
sensible world, nonetheless does not thereby find satisfaction for itself;
for, as a result of questions that keep recurring to infinity, it is denied
all hope of completely answering those questions. The transcendental
ideas, which have such completion as their aim, are such problems for
reason. Now reason clearly sees: that the sensible world could not contain
this completion, any more than could therefore all of the concepts that
serve solely for understanding that world: space and time, and everything
that we have put forward under the name of the pure concepts of the
understanding. The sensible world is nothing but a chain of appearances
connected in accordance with universal laws, which therefore has no
existence for itself; it truly is not the thing in itself, and therefore it
necessarily refers to that which contains the ground of those appearances,
to beings that can be cognized not merely as appearances, a but as things
in themselves. Only in the cognition of the latter can reason hope to see
its desire for completeness in the progression from the conditioned to
its conditions satisfied for once.
Above (§§33, 34) we noted limits of reason with respect to all cog-
nition of mere beings of thought; now, since the transcendental ideas
nevertheless make the progression up to these limits necessary for us,
and have therefore led us, as it were, up to the contiguity of the filled
space (of experience) with empty space (of which we can know nothing –
the noumena), we can also determine the boundaries of pure reason; for
in all boundaries there is something positive (e.g., a surface is the bound-
ary of corporeal space, yet is nonetheless itself a space; a line is a space,
which is the boundary of a surface; a point is the boundary of a line,
yet is nonetheless a locus in space), whereas limits contain mere nega-
tions. The limits announced in the cited sections are still not enough
after we have found that something lies beyond them (although we will
never cognize what that something may be in itself). For the question
now arises: How does our reason cope with this connection of that with
which we are acquainted to that with which we are not acquainted, and
never will be? Here is a real connection of the known to a wholly un-
known (which will always remain so), and even if the unknown should
not become the least bit better known – as is not in fact to be hoped –
the concept of this connection must still be capable of being determined
and brought to clarity.
We should, then, think for ourselves an immaterial being, an intelli-
gible world, and a highest of all beings (all noumena), because only in
these things, as things in themselves, does reason find completion and
satisfaction, which it can never hope to find in the derivation of the ap-
pearances from the homogeneous grounds of those appearances; and we
should think such things for ourselves because the appearances actually
do relate to something distinct from them (and so entirely heteroge-
neous), in that appearances always presuppose a thing in itself, a and so
provide notice of such a thing, whether or not it can be cognized more
closely.
Now since we can, however, never cognize these intelligible beings
according to what they may be in themselves, i.e., determinately – though
we must nonetheless assume such beings in relation to the sensible world,
and connect them with it through reason – we can still at least think this
connection by means of such concepts as express the relation of those
beings to the sensible world. For, if we think an intelligible being through
nothing but pure concepts of the understanding, we really think nothing
determinate thereby, and so our concept is without significance; if we
think it through properties borrowed from the sensible world, it is no
longer an intelligible being: it is thought as one of the phenomena and
belongs to the sensible world. We will take an example from the concept
of the supreme being.
The deistic concept is a wholly pure concept of reason, which however
represents merely a thing that contains every reality, without being able
to determine a single one of them, since for that an example would have
to be borrowed from the sensible world, in which case I would always
have to do only with an object of the senses, and not with something
completely heterogeneous which cannot be an object of the senses at
all. For I would, for instance, attribute understanding to it; but I have
no concept whatsoever of any understanding save one like my own, that
is, one such that intuitions must be given to it through the senses, and
that busies itself with bringing them under rules for the unity of con-
sciousness. But then the elements of my concept would still lie within
appearance; I was, however, forced by the inadequacy of the appearances
to go beyond them, to the concept of a being that is in no way dependent
on appearances nor bound up with them as conditions for its determi-
nation. If, however, I separate understanding from sensibility, in order
to have a pure understanding, then nothing but the mere form of think-
ing, without intuition, is left; through which, by itself, I cannot cognize
anything determinate, hence cannot cognize any object. To that end I
would have to think to myself a different understanding, which intu-
its objects, 75 of which, however, I do not have the least concept, since
the human understanding is discursive and can cognize only by means
of universal concepts. The same thing happens to me if I attribute a
will to the supreme being: For I possess this concept only by drawing
it from my inner experience, where, however, my a dependence on sat-
isfaction through objects whose existence we need, and so sensibility, is
the basis – which completely contradicts the pure concept of a supreme
being.
Hume’s objections to deism are weak and always concern the grounds
of proof but never the thesis of the deistic assertion itself. But with
respect to theism, which is supposed to arise through a closer deter-
mination of our (in deism, merely transcendent) concept of a supreme
being, they are very strong, and, depending on how this concept has
been framed, they are in certain cases (in fact, in all the usual ones) ir-
refutable. Hume always holds to this: that through the mere concept of
a first being to which we attribute none but ontological predicates (eter-
nity, omnipresence, omnipotence), we actually do not think anything
determinate at all; rather, properties would have to be added that can
yield a concept in concreto: it is not enough to say: this being is a cause,
rather we need to say how its causality is constituted, e.g., by under-
standing and willing – and here begin Hume’s attacks on the matter in
question, namely on theism, whereas he had previously assaulted only
the grounds of proof for deism, an assault that carries no special danger
with it. His dangerous arguments relate wholly to anthropomorphism,
of which he holds that it is inseparable from theism and makes theism
self-contradictory, but that if it is eliminated, theism falls with it and
nothing but deism remains – from which nothing can be made, which
can be of no use to us, and can in no way serve as a foundation for
religion and morals. If this unavoidability of anthropomorphism were
certain, then the proofs for the existence of a supreme being might be
what they will, and might all be granted, and still the concept of this be-
ing could never be determined by us without our becoming entangled in
contradictions.
If we combine the injunction to avoid all transcendent judgments
of pure reason with the apparently conflicting command to proceed
to concepts that lie beyond the field of immanent (empirical) use, we
become aware that both can subsist together, but only directly on the
boundary of all permitted use of reason – for this boundary belongs just as
much to the field of experience as to that of beings of thought – and we
are thereby at the same time taught how those remarkable ideas serve
solely for determining the boundary of human reason: that is, we are
taught, on the one hand, not to extend cognition from experience without
bound, so that nothing at all remains for us to cognize except merely the
world, and, on the other, nevertheless not to go beyond the boundary
of experience and to want to judge of things outside that boundary as
things in themselves.
But we hold ourselves to this boundary if we limit our judgment
merely to the relation that the world may have to a being whose concept
itself lies outside all cognition that we can attain within the world. For
we then do not attribute to the supreme being any of the properties in
themselves by which we think the objects of experience, and we thereby
avoid dogmatic anthropomorphism; but we attribute those properties,
nonetheless, to the relation of this being to the world, and allow ourselves
a symbolic anthropomorphism, which in fact concerns only language and
not the object itself.
If I say that we are compelled to look upon the world as if it were the
work of a supreme understanding and will, I actually say nothing more
than: in the way that a watch, a ship, and a regiment are related to an
artisan, a builder, and a commander, the sensible world (or everything
that makes up the basis of this sum total of appearances) is related to the
unknown – which I do not thereby cognize according to what it is in
itself, but only according to what it is for me, that is, with respect to the
world of which I am a part.

§ 58

This type of cognition is cognition according to analogy, which surely does
not signify, as the word is usually taken, an imperfect similarity between
146Prolegomena to any future metaphysics
two things, but rather a perfect similarity between two relations in wholly
dissimilar things. ∗ By means of this analogy there still remains a concept
of the supreme being sufficiently determinate for us, though we have
omitted everything that could have determined this concept uncondition-
ally and in itself; for we determine the concept only with respect to the
world and hence with respect to us, and we have no need of more. The
attacks that Hume makes against those who want to determine this con-
cept absolutely – since they borrow the materials for this determination
from themselves and from the world – do not touch us; he also cannot
reproach us that nothing whatsoever would remain for us if objective
anthropomorphism were subtracted from the concept of the supreme
being.

For if one only grants us, at the outset, the deistic concept of a first
being as a necessary hypothesis (as does Hume in his Dialogues in the
person of Philo as opposed to Cleanthes), which is a concept in which
one thinks the first being by means of ontological predicates alone, of
substance, cause, etc. (something that one must do, since reason, being
driven in the sensible world solely by conditions that are always again
conditioned, cannot have any satisfaction at all without this being done,
and something that one very well can do without falling into that anthro-
pomorphism which transfers predicates from the sensible world onto a
being wholly distinct from the world, since the predicates listed here are
mere categories, which cannot indeed provide any determinate concept
of that being, but which, for that very reason, do not provide a concept
of it that is limited to the conditions of sensibility) – then nothing can
keep us from predicating of this being a causality through reason with re-
spect to the world, and thus from crossing over to theism, but without
our being compelled to attribute this reason to that being in itself, as a
∗
Such is an analogy between the legal relation of human actions and the me-
chanical relation of moving forces: I can never do anything to another without
giving him a right to do the same to me under the same conditions; just as a
body cannot act on another body with its motive force without thereby caus-
ing the other body to react just as much on it. Right and motive force are here
completely dissimilar things, but in their relation there is nonetheless complete
similarity. By means of such an analogy I can therefore provide a concept of a
relation to things that are absolutely unknown to me. E.g., the promotion of
the happiness of the children = a is to the love of the parents = b as the welfare
of humankind = c is to the unknown in God = x, which we call love: not as if
this unknown had the least similarity with any human inclination, but because
we can posit the relation between God’s love and the world to be similar to
that which things in the world have to one another. But here the concept of the
relation is a mere category, namely the concept of cause, which has nothing to
do with sensibility.
147
4: 358Prolegomena to any future metaphysics
4: 359
4: 360
property inhering in it. For, concerning the first point, a the only possible
way to compel the use of reason in the sensible world (with respect to all
possible experience) into the most thoroughgoing harmony with itself is
to assume, in turn, a supreme reason as a cause of all connections in the
world; such a principle must be thoroughly advantageous to reason and
can nowhere harm it in its use in nature. Regarding the second point, b how-
ever, reason is not thereby transposed as a property onto the first being
in itself, but only onto the relation of that being to the sensible world, and
therefore anthropomorphism is completely avoided. For here only the
cause of the rational form found everywhere in the world is considered,
and the supreme being, insofar as it contains the basis of this rational
form of the world, is indeed ascribed reason, but only by analogy, i.e.,
insofar as this expression signifies only the relation that the highest cause
(which is unknown to us) has to the world, in order to determine every-
thing in it with the highest degree of conformity to reason. We thereby
avoid using the property of reason in order to think God, but instead
think the world through it in the manner necessary to have the greatest
possible use of reason with respect to the world in accordance with a
principle. We thereby admit that the supreme being, as to what it may
be in itself, is for us wholly inscrutable and is even unthinkable by us in a
determinate manner; and we are thereby prevented from making any tran-
scendent use of c the concepts that we have of reason as an efficient cause
(through willing) in order to determine the divine nature through prop-
erties that are in any case always borrowed only from human nature, and
so from losing ourselves in crude or fanatical concepts, and, on the other
hand, we are also prevented from swamping the contemplation of the
world with hyperphysical modes of explanation according to concepts
of human reason we have transposed onto God, and so from diverting
this contemplation from its true vocation, according to which it is sup-
posed to be a study of mere nature through reason, and not an audacious
derivation of the appearances of nature from a supreme reason. The ex-
pression suitable to our weak concepts will be: that we think the world
as if it derived from a supreme reason as regards its existence and inner
determination; whereby we in part cognize the constitution belonging
to it (the world) itself, without presuming to want to determine that of
its cause in itself, and, on the other hand, we in part posit the basis of this
constitution (the rational form of the world) in the relation of the highest
cause to the world, not finding the world by itself sufficient thereto. ∗
∗
I will say: the causality of the highest cause is that, with respect to the world,
which human reason is with respect to its works of art. Thereby the nature of
the highest cause itself remains unknown to me: I compare only its effect (the
order of the world), which is known to me, and the conformity with reason
of this effect, with the effects of human reason that are known to me, and in
148Prolegomena to any future metaphysics
In this way the difficulties that appear to oppose theism disappear,
in that to Hume’s principle, not to drive the use of reason dogmatically
beyond the field of all possible experience, we conjoin another principle
that Hume completely overlooked, namely: not to look upon the field
of possible experience as something that bounds itself in the eyes of our
reason. A critique of reason indicates the true middle way between the
dogmatism that Hume fought and the skepticism he wanted to introduce
instead – a middle way that, unlike other middle ways, which we are
advised to determine for ourselves as it were mechanically (something
from one side, and something from the other), and by which no one
is taught any better, is d one, rather, that can be determined precisely,
according to principles. 76

§ 59

At the beginning of this note I made use of the metaphor of a boundary
in order to fix the limits of reason with respect to its own appropriate
use. The sensible world contains only appearances, which are still not
things in themselves, which latter things (noumena) the understanding
must therefore assume for the very reason that it cognizes the objects of
experience as mere appearances. Both are considered together in our
reason, and the question arises: how does reason proceed in setting
boundaries for the understanding with respect to both fields? Expe-
rience, which contains everything that belongs to the sensible world,
does not set a boundary for itself: From every conditioned it always
arrives merely at another conditioned. That which is to set its bound-
ary must lie completely outside it, and this is the field of pure intel-
ligible beings. For us, however, as far as concerns the determination of
the nature of these intelligible beings, this is an empty space, and to
that extent, if dogmatically determined concepts are intended, we can-
not go beyond the field of possible experience. But since a boundary
is itself something positive, which belongs as much to what is within it
as to the space lying outside a given totality, reason therefore, merely
by expanding up to this boundary, partakes of a real, positive cogni-
tion, provided that it does not try to go out beyond the boundary, since
there it finds an empty space before it, in which it can indeed think the
consequence I call the highest cause a reason, without thereby ascribing to it
as its property the same thing I understand by this expression in humans, or in
anything else known to me.
a
b
c
d
“something that one must do . . .”
“something that one very well can do . . .”
Reading von for nach, with Vorländer.
Adding ist after Mittelwege, as suggested by Vorländer.
149
4: 361Prolegomena to any future metaphysics
4: 362
forms to things, but no things themselves. But setting the boundary to
the field of experience through something that is otherwise unknown to
it is indeed a cognition that is still left to reason from this standpoint,
whereby reason is neither locked inside the sensible world nor adrift
outside it, but, as befits knowledge of a boundary, restricts itself solely
to the relation of what lies outside the boundary to what is contained
within.

Natural theology is a concept of this kind, on the boundary of human
reason, since reason finds itself compelled to look out toward the idea of
a supreme being (and also, in relation to the practical, to the idea of an
intelligible world), not in order to determine something with respect to
this mere intelligible being (and hence outside the sensible world), but
only in order to guide its own use within the sensible world in accor-
dance with principles of the greatest possible unity (theoretical as well
as practical), and to make use (for this purpose) of the relation of that
world to a free-standing reason as the cause of all of these connections –
not, however, in order thereby merely to fabricate a being, but, since
beyond the sensible world there must necessarily be found something
that is thought only by the pure understanding, in order, in this way, to
determine this being, a though of course merely through analogy.
In this manner our previous proposition, which is the result of the
entire Critique, remains: “that reason, through all its a priori principles,
never teaches us about anything more than objects of possible experi-
ence alone, and of these, nothing more than what can be cognized in
experience”; but this limitation does not prevent reason from carrying
us up to the objective boundary of experience – namely, to the relation to
something that cannot itself be an object of experience, but which must
nonetheless be the highest ground of all experience – without, however,
teaching us anything about this ground in itself, but only in relation to
reason’s own complete use in the field of possible experience, as directed
to the highest ends. This is, however, all of the benefit that can reasonably
even be wished for here, and there is cause to be satisfied with it.

§ 60

We have thus fully exhibited metaphysics in accordance with its subjec-
tive possibility, as metaphysics is actually given in the natural predisposition
of human reason, and with respect to that which forms the essential goal
of its cultivation. But because we found that, if reason is not reined in and
given limits by a discipline of reason, which is only possible through a sci-
entific critique, this wholly natural use of this sort of predisposition of our
reason entangles it in transcendent dialectical inferences, which are partly
specious, partly even in conflict among themselves; and, moreover, be-
cause we found that this sophistical metaphysics is superfluous, nay, even
detrimental to the advancement of the cognition of nature, it therefore
still remains a problem worthy of investigation, to discover the natural
purposes toward which this predisposition of our a reason b to transcendent
concepts may be aimed, since everything found in nature must originally
be aimed at some beneficial purpose or other.

Such an investigation is in fact uncertain; I also admit that it is merely
conjectural (as is everything I know to say concerning the original pur-
poses of nature), something I may be permitted in this case only, since
the question does not concern the objective validity of metaphysical
judgments, but rather the natural predisposition to such judgments, and
therefore lies outside the system of metaphysics, in anthropology. 77
If I consider c all the transcendental ideas, which together constitute
the real problem for natural pure reason – a problem that compels reason
to forsake the mere contemplation of nature and go beyond all possible
experience, and, in this endeavor, to bring into existence the thing called
metaphysics (be it knowledge or sophistry) – then I believe I perceive
that this natural predisposition is aimed at making our concept suffi-
ciently free from the fetters of experience and the limits of the mere
contemplation of nature that it at the least sees a field opening before it
that contains only objects for the pure understanding which no sensibil-
ity can reach: not with the aim that we concern ourselves speculatively
with these objects (for we find no ground on which we can gain footing),
but rather with practical principles, d which, without finding such a space
before them for their necessary expectations and hopes, could not ex-
tend themselves to the universality that reason ineluctably requires with
respect to morals.

Here I now find that the psychological idea, however little insight I may
gain through it into the pure nature of the human soul elevated beyond
all concepts of experience, at least reveals clearly enough the inadequacy
of those concepts of experience, and thereby leads me away from mate-
rialism, as a psychological concept unsuited to any explanation of nature
and one that, moreover, constricts reason with respect to the practical.
Similarly, the cosmological ideas, through the manifest inadequacy of all
possible cognition of nature to satisfy reason in its rightful demands,
serve to deter us from naturalism, which would have it that nature is
sufficient unto itself. Finally, since all natural necessity in the sensible
world is always conditioned, in that it always presupposes the depen-
dence of one thing on another, and since unconditioned necessity must
be sought only in the unity of a cause distinct from the sensible world,
although the causality of that cause, in turn, if it were merely nature,
could never make comprehensible the existence of the contingent as its
consequence, reason, therefore, by means of the theological idea, frees
itself from fatalism – from blind natural necessity both in the connec-
tion of nature itself, without a first principle, and in the causality of this
principle itself – and leads the way to the concept of a cause through
freedom, and so to that of a highest intelligence. The transcendental
ideas therefore serve, if not to instruct us positively, at least to negate
the impudent assertions of materialism, naturalism, and fatalism which
constrict the field of reason, and in this way they serve to provide moral
ideas with space outside the field of speculation; and this would, I should
think, to some extent explain the aforementioned natural predisposition.
The practical benefit that a purely speculative science may have lies
outside the boundaries of this science; such benefit can therefore be seen
simply as a scholium, and like all scholia does not form part of the sci-
ence itself. Nonetheless, this relation at least lies within the boundaries
of philosophy, and especially of that philosophy which draws from the
wellsprings of pure reason, where the speculative use of reason in meta-
physics must necessarily have unity with its practical use in morals. Hence
the unavoidable dialectic of pure reason deserves, in a metaphysics con-
sidered as natural predisposition, not only to be explained as an illusion
that needs to be resolved, but also (if one can) as a natural institution in
accordance with its purpose – although this endeavor, as supererogatory,
cannot rightly be required of metaphysics proper.

The solution to the questions that proceed in the Critique from pages
647 to 668 would have to be taken for a second scholium, more closely
related to the content of metaphysics. 78 For there certain principles of
reason are put forward that determine the order of nature a priori, or
rather determine the understanding a priori, which is supposed to search
for the laws of this order by means of experience. These principles seem
to be constitutive and law-giving with respect to experience, though
they spring from mere reason, which cannot, like the understanding,
be regarded as a principle of possible experience. Now whether this
agreement rests on the fact that, just as nature does not in itself inhere
in the appearances or in their source, sensibility, but is found only in
the relation of sensibility to the understanding, so too, a thoroughgoing
unity in the use of this understanding, for the sake of a unified possible
experience (in a system), can belong to the understanding only in relation
to reason, hence experience, too, be indirectly subject to the legislation
152Prolegomena to any future metaphysics
of reason – this may be further pondered by those who want to track the
nature of reason even beyond its use in metaphysics, into the universal
principles for making natural history generally systematic; for in the
book itself I have indeed presented this problem as important, but have
not attempted its solution.
And thus I conclude the analytic solution of the main question I
myself have posed: How is metaphysics in general possible?, since I have
ascended from the place where its use is actually given, at least in the
consequences, to the grounds of its possibility.

Solution to the General Question of the Prolegomena

How is metaphysics possible as science?

Metaphysics, as a natural predisposition of reason, is actual, but it is also
of itself (as the analytical solution to the third main question proved)
dialectical and deceitful. The desire to derive principles from it, and to
follow the natural but nonetheless false illusion in their use, can therefore
never bring forth science, but only vain dialectical art, in which one
school can outdo another but none can ever gain legitimate and lasting
approbation.

In order that metaphysics might, as science, be able to lay claim, not
merely to deceitful persuasion, but to insight and conviction, a critique
of reason itself must set forth the entire stock of a priori concepts, their
division according to the different sources (sensibility, understanding,
and reason), further, a complete table of those concepts, and the analysis
of all of them along with everything that can be derived from that analysis;
and then, especially, such a critique must set forth the possibility of
synthetic cognition a priori through a deduction of these concepts, it
must set forth the principles of their use, and finally also the boundaries
of that use; and all of this in a complete system. Therefore a critique, and
that alone, contains within itself the whole well-tested and verified plan
by which metaphysics as science can be achieved, and even all the means
for carrying it out; by any other ways or means it is impossible. Therefore
the question that arises here is not so much how this enterprise is possible,
but only how it is to be set in motion, and good minds stirred from the
hitherto ill-directed and fruitless endeavor to one that will not deceive,
and how such an alliance might best be turned toward the common end.

This much is certain: whosoever has once tasted of critique forever
loathes all the dogmatic chatter which he previously had to put up with
out of necessity, since his reason was in need of something and could not
find anything better for its sustenance. Critique stands to the ordinary
school metaphysics precisely as chemistry stands to alchemy, or astronomy
to the fortune-teller’s astrology. I’ll guarantee that no one who has thought
through and comprehended the principles of critique, even if only in
these prolegomena, will ever again return to that old and sophistical
pseudoscience; he will on the contrary look out with a certain delight
upon a metaphysics that is now fully in his power, that needs no more
preliminary discoveries, and that can for the first time provide reason
with lasting satisfaction. For this is an advantage upon which metaphysics
alone, among all the possible sciences, can rely with confidence, namely,
that it can be completed and brought into a permanent state, since it
cannot be further changed and is not susceptible to any augmentation
through new discoveries – because here reason has the sources of its
cognition not in objects and their intuition (through which reason cannot
be taught one thing more), but in itself, and, if reason has presented
the fundamental laws of its faculty fully and determinately (against all
misinterpretation), nothing else remains that pure reason could cognize
a priori, or even about which it could have cause to ask. The sure prospect
of a knowledge so determinate and final has a certain attraction to it, even
if all usefulness (of which I will say more hereafter) is set aside.

All false art, all empty wisdom lasts for its time; for it finally destroys
itself, and the height of its cultivation is simultaneously the moment of its
decline. That this time has now come as regards metaphysics is proven by
the condition into which it has fallen among all learned peoples, amidst
all the zeal with which sciences of all kinds are otherwise being developed.
The old organization of university studies still preserves the shadow of
metaphysics, a lone academy of sciences now and then, by offering prizes,
moves someone or other to make an effort in it, but metaphysics is no
longer reckoned among serious sciences, and each may judge for himself
how a clever man, whom one wished to call a great metaphysician, would
perhaps receive this encomium, which might be well meant but would
hardly be envied by anyone.

But although the time for the collapse of all dogmatic metaphysics
is undoubtedly here, much is still lacking in order to be able to say
that, on the contrary, the time for its rebirth, through a thorough and
completed critique of reason, has already appeared. All transitions from
one inclination to its opposite pass through a state of indifference, and
this moment is the most dangerous for an author, but nonetheless, it
seems to me, the most favorable for the science. For if the partisan spirit
has been extinguished through the complete severance of former ties,
then minds are best disposed to a hear out, bit by bit, proposals for an
alliance according to another plan.

If I say that I hope these Prolegomena will perhaps excite investigation
in the field of critique, and provide the universal spirit of philosophy,
which seems to want nourishment in its speculative part, with a new and
quite promising object of sustenance, I can already imagine beforehand
that everyone who a has been made weary and unwilling by the thorny
paths on which I have led him in the Critique will ask me: On what do I
base this hope? I answer: On the irresistible law of necessity.

That the human mind would someday entirely give up metaphysical
investigations is just as little to be expected, as that we would someday
gladly stop all breathing so as never to take in impure air. There will
therefore be metaphysics in the world at every time, and what is more,
in every human being, and especially the reflective ones; metaphysics
that each, in the absence of a public standard of measure, will carve
out for himself in his own manner. Now what has hitherto been called
metaphysics can satisfy no inquiring mind, and yet it is also impossible to
give up metaphysics completely; therefore, a critique of pure reason itself
must finally be attempted, or, if one exists, it must be examined and put
to a general test, since there are no other means to relieve this pressing
need, which is something more than a mere thirst for knowledge.

Ever since I have known critique, I have been unable to keep myself
from asking, upon finishing reading through a book with metaphysical
content, which has entertained as well as cultivated me by the determi-
nation of its concepts and by variety and organization and by an easy
presentation: has this author advanced metaphysics even one step? I ask for-
giveness of the learned men whose writings have in other respects been
useful to me and have always contributed to a cultivation of mental pow-
ers, because I confess that I have not been able to find, either in their
attempts or in my own inferior ones (with self-love speaking in their
favor), that the science has thereby been advanced in the least, and this
for the wholly natural reason that the science did not yet exist, and also
that it cannot be assembled bit by bit but rather its seed must be fully
preformed beforehand in the critique. However, in order to avoid all
misunderstanding, it must be recalled from the preceding that although
the understanding certainly benefits very much from the analytical treat-
ment of our concepts, the science (of metaphysics) is not advanced the
least bit thereby, since these analyses of concepts are only materials, out
of which the science must first be constructed. The concept of substance
and accident may be analyzed and determined ever so nicely; that is
quite good as preparation for some future use. But if I simply cannot
prove that in all that exists the substance persists and only the accidents
change, then through all this analysis the science has not been advanced
in the least. Now metaphysics has not as yet been able to prove, as a pri-
ori valid, either this proposition or the principle of sufficient reason, still
less any more composite proposition, such as, for instance, one belonging
to psychology or cosmology, nor, in general, any synthetic proposition
whatsoever; hence, through all this analysis nothing has been achieved,
nothing created and advanced, and, after so much bustle and clatter,
the science is still right where it was in Aristotle’s time, although the
preparations for it incontestably have been much better laid than be-
fore, if only the guiding thread to synthetic cognition had first been
found.

If anyone believes himself wronged in this, he can easily remove the
above indictment if he will cite only a single synthetic proposition be-
longing to metaphysics that he offers to prove a priori in the dogmatic
manner; for only when he accomplishes this will I grant to him that
he has actually advanced the science (even if the proposition may oth-
erwise have been sufficiently established through common experience).
No challenge can be more moderate and more equitable, and in the (in-
fallibly certain) event of nonfulfillment, no verdict more just, than this:
that up to now metaphysics as science has never existed at all.

In case the challenge is accepted, I must forbid only two things: first,
the plaything of probability and conjecture, which suits metaphysics just
as poorly as it does geometry; second, decision by means of the divining
rod of so-called sound common sense, a which does not bend for everyone,
but is guided by personal qualities.

For, as regards the first, there can be nothing more absurd than to want
to base one’s judgments in metaphysics, a philosophy from pure reason,
on probability and conjecture. Everything that is to be cognized a priori is
for that very reason given out as apodictically certain and must therefore
also be proven as such. One might just as well want to base a geometry
or an arithmetic on conjectures; for as concerns the calculus probabilium b
of arithmetic, it contains not probable but completely certain judgments
about the degree of possibility of certain cases under given homogeneous
conditions, judgments which, in the sum total of all possible cases, must
be found to conform to the rule with complete infallibility, even though
this rule is not sufficiently determinate with respect to any single case.
Only in empirical natural science can conjectures (by means of induction
and analogy) be tolerated, and even then, the possibility at least of what
I am assuming must be fully certain.

Matters are, if possible, even worse with the appeal to sound common
sense, if the discussion concerns c concepts and principles, not insofar as
they are supposed to be valid with respect to experience, but rather inso-
far as they are to be taken as valid beyond the conditions of experience.
For what is sound common sense? a It is the ordinary understanding, b insofar
as it judges correctly. And what now is the ordinary understanding? It is
the faculty of cognition and of the use of rules in concreto, as distinguished
from the speculative understanding, which is a faculty of the cognition of
rules in abstracto. The ordinary understanding will, then, hardly be able
to understand the rule: that everything which happens is determined by
its cause, and it will never be able to have insight into it in such a gen-
eral way. It therefore demands an example from experience, and when it
hears that this rule means nothing other than what it had always thought
when a windowpane was broken or a household article had disappeared,
it then understands the principle and grants it. Ordinary understanding,
therefore, has a use no further than the extent to which it can see its
rules confirmed in experience (although these rules are actually present
in it a priori); consequently, to have insight into these rules a priori and
independently of experience falls to the speculative understanding, and
lies completely beyond the horizon of the ordinary understanding. But
metaphysics is concerned indeed solely with this latter type of cogni-
tion, and it is certainly a poor sign of sound common sense to appeal to
this guarantor, who has no judgment here, and whom we otherwise look
down upon, except if we find ourselves in trouble, and without either
advice or help in our speculation.

It is a common excuse, habitually employed by these false friends of or-
dinary common sense (which they extol on occasion, but usually despise),
to say: There must in the end be some propositions that are c immedi-
ately certain, and for which not only no proof, but indeed no account at
all need be given, since otherwise there would never come an end to the
grounds for one’s judgments; but in proof of this right they can never cite
anything else (other than the principle of contradiction, which is how-
ever inadequate for establishing the truth of synthetic judgments) that is
undoubted and can be ascribed directly to ordinary common sense, ex-
cept for mathematical propositions: e.g., that two times two makes four,
that between two points there is only one straight line, and still others.
These judgments are, however, worlds apart from those of metaphysics.
For in mathematics, everything that I conceive through a concept as
possible I can make for myself (construct) by means of my thought; to
one two I successively add the other two, and myself make the number
four, or I draw in thought all kinds of lines from one point to the other,
and can draw only one that is self-similar in all its parts (equal as well as
unequal). 80 But from the concept of a thing I cannot, with all my powers
of thought, draw forth the concept of something else whose existence is
necessarily connected with the first thing, but must consult experience;
and, although my understanding provides me a priori (though always only
in relation to possible experience) with the concept of a connection of
this sort (causality), I nevertheless cannot exhibit this concept in intuition
a priori, like the concepts of mathematics, and thus exhibit its possibility a
priori; rather, this concept (together with principles of its application), if
it is to be valid a priori – as is indeed required in metaphysics – always has
need of a justification and deduction of its possibility, for otherwise one
does not know the extent of its validity and whether it can be used only in
experience or also outside it. Therefore in metaphysics, as a speculative
science of pure reason, one can never appeal to ordinary common sense,
but one can very well do so if one is forced to abandon metaphysics
and to renounce all pure speculative cognition, which must always be
knowledge, a hence to renounce metaphysics itself and its teaching (on
certain matters), and if a reasonable belief b is alone deemed possible for
us, as well as sufficient for our needs (perhaps more wholesome indeed
than knowledge itself ). For then the shape of things is completely al-
tered. Metaphysics must be science, not only as a whole but also in all its
parts; otherwise it is nothing at all, since, as speculation of pure reason,
it has a hold on nothing else save universal insights. But outside meta-
physics, probability and sound common sense can very well have their
beneficial and legitimate use, though following principles entirely their
own, whose importance always depends on a relation to the practical.

That is what I consider myself entitled to require for the possibility
of a metaphysics as science.

Appendix

On what can be done in order to
make metaphysics as science actual

Since all paths hitherto taken have not attained this end, and it may never
be reached without a preceding critique of pure reason, the demand that
the attempt at such a critique which is now before the public be subjected
to an exact and careful examination does not seem unreasonable – unless
it is considered more advisable still to give up all claims to metaphysics
entirely, in which case, if one only remains true to one’s intention, there
is nothing to be said against it. If the course of events is taken as it actually
runs and not as it should run, then there are two kinds of judgments: a
judgment that precedes the investigation, and in our case this is one in which
the reader, from his own metaphysics, passes judgment on the Critique of
Pure Reason (which is supposed first of all to investigate the possibility of
that metaphysics); and then a different judgment that comes after the investi-
gation, in which the reader is able to set aside for a while the consequences
of the critical investigation, which might tell pretty strongly against the
metaphysics he otherwise accepts, and first tests the grounds from which
these consequences may have been derived. If what ordinary metaphysics
presents were undeniably certain (like geometry, for instance), the first
way of judging would be valid; for if the consequences of certain princi-
ples conflict with undeniable truths, then those principles are false and
are to be rejected without any further investigation. But if it is not the
case that metaphysics has a supply of incontestably certain (synthetic)
propositions, and perhaps is the case that a good number of them, which
are as plausible as the best among them, nevertheless are, a in their con-
sequences, in conflict even among themselves, while there is not to be
found overall in metaphysics any secure criterion whatsoever of the truth
of properly metaphysical (synthetic) propositions: then the first way of
judging cannot be allowed, but rather the investigation of the principles
of the Critique must precede all judgment of its worth or unworth.

Specimen of a judgment about the Critique
which precedes the investigation

This sort of judgment is to be found in the Göttingische gelehrte Anzeigen,
the third part of the supplement, from January 19, 1782, pages 40 ff. 81

If an author who is well acquainted with the object of his work, who
has been assiduous throughout in putting reflection into its composition
that is completely his own, falls into the hands of a reviewer who for
his part is sufficiently clear-sighted to espy the moments upon which b
the worth or unworth of the piece actually rests, who does not hang on
words but follows the subject matter, and who examines and tests only c
the principles from which the author has proceeded, then although the
severity of the judgment may certainly displease the author, the pub-
lic is, by contrast, indifferent to it, for it profits thereby; and the au-
thor himself can be content that he gets the opportunity to correct or
to elucidate his essays, which have been examined early on by an ex-
pert, and, if he believes he is basically right, in this way to remove in
good time a stumbling block that could eventually be detrimental to his
work.

I find myself in a completely different situation with my reviewer.
He appears not at all to see what really mattered in the investigation
with which I have (fortunately or unfortunately) occupied myself, and,
whether it was impatience with thinking through a lengthy work, or
ill-temper over the threatened reform of a science in which he believed
he had long since put everything in order, or whether, as I reluctantly
surmise, it was the fault of a truly limited conception, through which he
could never think himself beyond his school metaphysics – in short, he
impetuously runs through a long series of propositions, with which one
can think nothing at all without knowing their premises, he disperses his
rebukes to and fro, for which the reader no more sees any basis than he
understands the propositions toward which they are supposedly directed,
and therefore the reviewer can neither help to inform the public nor do
me the least bit of harm in the judgment of experts; consequently, I would
have passed over this review completely, if it did not provide me occasion
for a few elucidations that in some cases might save the reader of these
Prolegomena from misconception.

In order, however, that the reviewer might adopt a viewpoint from
which he could, without having to trouble himself with any special inves-
tigation, most easily present the entire work in a manner disadvantageous
to the author, he begins and also ends by saying: “this work is a system
of transcendental a (or, as he construes it, higher) ∗ idealism.” b

At the sight of this line I quickly perceived what sort of review would
issue thence – just about as if someone who had never seen or heard
anything of geometry were to find a Euclid, and, being asked to pass
judgment on it, were perhaps to say, after stumbling onto a good many
figures by turning the pages: “the book is a systematic guide to drawing;
the author makes use of a special language in order to provide obscure,
unintelligible instructions, which in the end can achieve nothing more
than what anyone can accomplish with a good natural eye, and so on.”

Let us, however, look at what sort of idealism it is that runs through
my entire work, although it does not by far constitute the soul of the
system.

The thesis of all genuine idealists,
from the Eleatic School up to Bishop Berkeley,
is contained in this formula:
“All cognition through the senses and experience is
nothing but sheer illusion,
and there is truth only in the ideas of
pure understanding and reason.”

The principle that governs and determines
my idealism throughout is, on the contrary:
“All cognition of things
out of mere pure understanding or pure reason is
nothing but sheer illusion,
and there is truth only in experience.”

But this is, of course, the direct opposite of the previous, genuine
idealism; how then did I come to use this expression with a completely
opposite intention, and how did the reviewer come to see genuine ide-
alism everywhere?

The solution to this difficulty rests upon something that could have
been seen very easily from the context of the work, if one had wanted
to. Space and time, together with everything contained in them, are
not things (or properties of things) in themselves, but belong instead
merely to the appearances of such things; thus far I am of one creed
with the previous idealists. But these idealists, and among them espe-
cially Berkeley, viewed space as a merely empirical representation, a rep-
resentation which, just like the appearances in space together with all
of the determinations of space, would be known to us only by means
of experience or perception; I show, on the contrary, first: that space
(and time as well, to which Berkeley gave no attention), together with
all its determinations, can be cognized by us a priori, since space (as
well as time) inheres in us before all perception or experience as a pure
form of our sensibility and makes possible all intuition from sensibility,
and hence all appearances. From this it follows: that, since truth rests
upon universal and necessary laws as its criteria, for Berkeley experience
could have no criteria of truth, because its appearances (according to
him) had nothing underlying them a priori; from which it then followed
that experience is nothing but sheer illusion, whereas for us space and
time (in combination with the pure concepts of the understanding) pre-
scribe a priori their law to all possible experience, which law at the same
time provides the sure criterion for distinguishing truth from illusion in
experience.

XXX

The reviewer, however, talks like a man who must be aware of im-
portant and exquisite insights, which, however, he still keeps secret; for
nothing has become known to me of late regarding metaphysics that
could justify such a tone. But he is doing a great wrong in withholding
his discoveries from the world; for there are doubtless many others like
me who, with all the fine things that have been written in this field for
some time now, have still been unable to find that the science has thereby
been advanced a finger’s breadth. In other respects, we do indeed find
definitions being sharpened, lame proofs provided with new crutches,
the patchwork garment of metaphysics given new pieces, or an altered
cut – but that is not what the world demands. The world is tired of meta-
physical assertions; what’s wanted b are the possibility of this science, the
sources from which certainty could be derived in it, and sure criteria for
distinguishing truth from the dialectical illusion of pure reason. The re-
viewer must possess the key to all this, otherwise he surely would never
have spoken in so high a tone.

But I come to suspect that this sort of need of the science perhaps may
never have come into his head; for otherwise he would have directed his
review toward this point, and in such an important matter even a failed
attempt would have gained his respect. If that is so, then we are good
friends again. He may think himself as deeply into his metaphysics as
seems good to him, no one will stop him; only he is not permitted to
judge of something that lies outside metaphysics, i.e., its source located
in reason. But that my suspicion is not unfounded, I prove by the fact
that he did not say a word about the possibility of synthetic cognition
a priori, which was the real problem, on the solution of which the fate
of metaphysics wholly rests, and to which my Critique (just as here my
Prolegomena) was entirely directed. The idealism upon which he chanced,
and to which he then held fast, was taken up into the system only as the
sole means for solving this problem (although it then also received its a
confirmation on yet other grounds); and so he would have had to show
either that this problem does not have the importance that I attribute
to it (as also now in the Prolegomena), or that it could not be solved at
all by my concept of appearances, or could better be solved in another
way; but I find not a word of this in the review. The reviewer therefore
understood nothing of my work and perhaps also nothing of the spirit
and nature of metaphysics itself, unless on the contrary, which I prefer to
assume, a reviewer’s haste, indignant at the difficulty of plowing his way
through so many obstacles, cast an unfavorable shadow over the work
lying before him and made it unrecognizable to him in its fundamentals.

There is still a great deal needed for a learned gazette, however well-
chosen and carefully selected its contributors may be, to be able to uphold
its otherwise well-deserved reputation in the field of metaphysics (just as
elsewhere). Other sciences and areas of learning b have their standards.
Mathematics has its standard within itself, history and theology in sec-
ular or sacred books, natural science and medicine in mathematics and
experience, jurisprudence in law books, and even matters of taste in an-
cient paradigms. But in order to assess the thing called metaphysics, the
standard must first be found (I have made an attempt to determine this
standard as well as its use). Until it is ascertained, what is to be done
when works of this kind must be judged? If they are of the dogmatic
kind, one may do as one likes; no one will for long play the master over
others in this without finding someone who repays him in kind. But if
they are of the critical kind, and not indeed with regard to other writings
but to reason itself, so that the standard of appraisal cannot be already
assumed but must first be sought: then objection and censure are not
to be forbidden, but they must be rooted in tolerance, since the need is
common to us all, and the lack of the required insight makes an air of
judicial decisiveness unsuitable.

But in order at the same time to tie this my defense to the interest
of the philosophizing community, I propose a test, which is decisive as
to the way in which all metaphysical investigations must be directed
toward their common end. This is nothing else than what mathemati-
cians have done before, in order to decide the merits of their methods
in a contest – that is, a challenge to my reviewer to prove in his own
way any single truly metaphysical (i.e., synthetic, and cognized a priori
from concepts) proposition a he holds, and at best one of the most indis-
pensable, such as the principle of the persistence of substance or of the
necessary determination of the events in the world through their cause
– but, as is fitting, to prove it on a priori grounds. If he can’t do this (and
silence is confession), then he must admit: that, since metaphysics is ab-
solutely nothing without the apodictic certainty of propositions of this
sort, their possibility or impossibility would first, before all else, have to
be settled in a critique of pure reason, and hence he is obliged either to
acknowledge that my principles of critique are correct or to prove their
invalidity. Since, however, I already foresee that, as heedlessly as he has
hitherto been relying on the certainty of his principles, still, now that it
comes down to a rigorous test, he will not find a single principle in the
whole compass of metaphysics with which he can dare come forward, I
will therefore grant him the most favorable terms that can ever be ex-
pected in a competition; namely, I will take the onus probandi b from him
and will have it put on me.

In particular, in these Prolegomena and in my Critique, pp. 426–61, he
will find eight propositions which are, pair by pair, always in conflict with
one another, but each of which belongs necessarily to metaphysics, which
must either accept it or refute it (although there is not a single one of
them that has not in its day been accepted by some philosopher or other).
He now has the freedom to pick any one of these eight propositions he
likes, and to assume it without proof (which I concede to him), but only
one (for wasting time will be no more useful to him than to me), and
then he is to attack my proof of the antithesis. But if I can rescue it,
and in this way show that the opposite of the proposition he adopted
can be proven exactly as clearly, in accordance with principles that every
dogmatic metaphysics must of necessity acknowledge, then by this means
it is settled that there is an hereditary defect in metaphysics that cannot
be explained, much less removed, without ascending to its birthplace,
pure reason itself, and so my Critique must either be accepted or a better
one put in its place, and therefore it must at least be studied; which is
the only thing I ask for now. If, on the contrary, I cannot rescue my
proof, then a synthetic a priori proposition is established from dogmatic
principles on my opponent’s side, my indictment of ordinary metaphysics
was therefore unjust, and I offer to recognize his censure of my Critique as
legitimate (although this is far from being the likely outcome). But hereto
it would be necessary, I should think, to emerge from being incognito, since
I do not otherwise see how to prevent my being honored or assailed with
many problems from unknown and indeed unbidden opponents, instead
of just one.

Proposal for an investigation of the Critique,
after which the judgment can follow

I am obliged to the learned public for the silence with which it has
honored my Critique for so long a time; for this after all demonstrates
a suspension of judgment, and thus some suspicion that, in a work that
abandons all the usual paths and pursues a new one in which one cannot
immediately find one’s way, something might nonetheless perhaps be
found through which an important but now moribund branch of human
knowledge could receive new life and fertility, and so demonstrates a
cautiousness, not to break off and destroy the still fresh graft through an
overly hasty judgment. A specimen of a judgment that was delayed for
such reasons has only just now come before me in the Gothaische gelehrte
Zeitung, 88 a judgment whose well-foundedness every reader will perceive
for himself (without taking into account my own suspect praise) from
the clear and candid presentation of a portion of the first principles of
my work.

And now I propose, since a large edifice cannot possibly be instantly
judged as a whole through a quick once-over, that it be examined piece by
piece from its foundation, and that in this the present Prolegomena be used
as a general synopsis, with which the work itself could then be compared
on occasion. This suggestion, if it were based on nothing more than the
imagined importance that vanity customarily imparts to all one’s own
products, would be immodest and would deserve to be dismissed with
indignation. But the endeavors of all speculative philosophy now stand
at the point of total dissolution, although human reason clings to them
with undying affection, an affection that now seeks, though vainly, to turn
itself into indifference, only because it has been constantly betrayed.

In our thinking age it is not to be expected but that many merito-
rious men would use every good opportunity to work together toward
the common interest of an ever more enlightened reason, if only there
appears some hope of thereby attaining the goal. Mathematics, natural
science, law, the arts, even morals (and so on) do not completely fill up
the soul; there still remains a space in it that is marked off for mere
pure and speculative reason, and its emptiness drives us to seek out, in
grotesques and trivialities, or else in delusions, what seems to be occu-
pation and amusement, but is at bottom only distraction to drown out
the troublesome call of reason, which, as befits its vocation, demands
something that satisfies it for itself and does not merely stir it to activity
on behalf of other purposes or in the service of inclinations. Therefore,
for everyone who has even tried to enlarge his conception in this way,
contemplation that occupies itself only with this sphere of reason exist-
ing for itself has a great attraction, because exactly in this sphere all other
areas of learning and even ends must, as I have reason to suppose, join
together and unite in a whole – and, I dare say, it has a greater attraction
than any other theoretical knowledge, for which one would not readily
exchange it.

But I propose these Prolegomena as the plan and guide for the inves-
tigation, and not the work a itself, because, with respect to the latter,
though I am even now quite satisfied as regards the content, order, and
method, and the care that was taken to weigh and test each proposi-
tion accurately before setting it down (for it took years for me to be
fully satisfied not only with the whole, but sometimes also with only a
single proposition, as regards its sources), I am not fully satisfied with
my presentation in some chapters of the Doctrine of Elements, e.g., the
Deduction of the Concepts of the Understanding or the chapter on the
Paralogisms of Pure Reason, 89 since in them a certain prolixity obstructs
the clarity, and in their stead the examination can be based on what the
Prolegomena here say with respect to these chapters.

The Germans are praised for being able to advance things further than
other peoples in matters where persistence and unremitting industry are
called for. If this opinion is well-founded, then an opportunity presents
itself here to bring to completion an endeavor whose happy outcome
is hardly to be doubted and in which all thinking persons share equal
interest, but which has not succeeded before now – and to confirm that
favorable opinion; especially since the science concerned is of such a
peculiar kind that it can be brought all at once to its full completion, and
into a permanent state such that it cannot be advanced the least bit further
and can be neither augmented nor altered by later discovery (herein I
do not include embellishment through enhanced clarity here and there,
or through added utility in all sorts of respects): an advantage that no
other science has or can have, since none is concerned with a cognitive
faculty that is so fully isolated from, independent of, and unmingled
with other faculties. The present moment does not seem unfavorable to
this expectation of mine, since in Germany nowadays one hardly knows
how he could keep himself otherwise still occupied outside the so-called
useful sciences and have it be, not mere sport, but at the same time an
endeavor through which an enduring goal is reached.

I must leave it to others to contrive the means by which the efforts
of the learned could be united toward such an end. In the meantime
it is not my intention to expect of anyone a simple adherence to my
theses, nor even to flatter myself with hope of that; rather, whether it
should, as it happens, be attacks, revisions, and qualifications that bring
it about, or confirmation, completion, and extension, if only the matter
is investigated from the ground up, then it now can no longer fail that
a system would thereby come into being (even if it were not mine) that
could become a legacy to posterity for which it would have reason to be
thankful.

It would be too much to show here what sort of metaphysics could be
expected to follow if one were first right about the principles of a critique,
and how it would by no means have to appear paltry and cut down to just a
small figure because its false feathers had been plucked, but could in other
respects appear richly and respectably outfitted; but other large benefits
that such a reform would bring with it are apparent at once. The ordinary
metaphysics has indeed already produced benefits, because it searched
for the elementary concepts of the pure understanding in order to render
them clear through analysis and determinate through explication. It was
thereby a cultivation of reason, wherever reason might subsequently
think fit to direct itself. But that was all the good that it did. For it
undid this merit again by promoting self-conceit through rash assertions,
sophistry through subtle evasions and glosses, and shallowness through
the facility with which it overcame the most difficult problems with a little
school wisdom – a shallowness that is all the more enticing the more it has
the option of, on the one hand, taking on something from the language
of science, and, on the other, from popularity, and thereby is everything
to everyone, but in fact is nothing at all. By contrast, through critique our
judgment is afforded a standard by which knowledge can be distinguished
with certainty from pseudo-knowledge; and, as a result of being brought
fully into play in metaphysics, critique establishes a mode of thinking
that subsequently extends its wholesome influence to every other use
of reason, and for the first time excites the true philosophical spirit.
Moreover, the service it renders to theology, by making it independent of
the judgment of dogmatic speculation and in that way securing it against
all attacks from such opponents, is certainly not to be underrated. For the
ordinary metaphysics, although promising to assist theology greatly, was
subsequently unable to fulfill this promise, and beyond this, in calling
speculative dogmatism to its aid, had done nothing other than to arm
enemies against itself. Fanaticism, which cannot make headway in an
enlightened age except by hiding behind a school metaphysics, under
the protection of which it can venture, as it were, to rave rationally,
will be driven by critical philosophy from this its final hiding place; and
beyond all this it cannot fail to be important to a teacher of metaphysics
to be able, for once with universal assent, to say that what he propounds
is now at last science, and that through it genuine benefit is rendered to
the commonweal.
