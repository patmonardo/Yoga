SK 22.

Prakriti mahat tata ahamkara tasmat ca sodasaka gana
sodasakat tasmad api pancabhyah panca bhutani

From nature evolves the Great Principle;
from this evolves the I Principle;
from this evolves the set of sixteen;
from five of this set of sixteen evolves the five elements.

SK 23.

buddhi adhyavasaya dharma jnana viraga aisvarya
etad rupam sattvika asmat viparyastam tamasa

Buddhi is self-certainty.
Essence, knowledge, dispassion, and power
are its manifestations
when the sattva quality abounds.
And the reverse of these,
when the tamasa quality abounds.

YS II.44

    svadhyaya ista-devata-samprayoga

YS II.45

    samadhi-siddhi îsvara-pranidhana

    SK 24.

ahamkara abhimana tasmat pravartate dvividha sarga eva
ekadasaka gana ca tanmatra pancaka

Ahamkara is self-assertion;
from that proceeds a two-fold evolution
the set of eleven and the five-fold primary elements.

YS II.46

    sthira-sukham asana

YS II.47

    prayatna-saithilya-ananta-samapattibhya

YS II.48

    tato dvandva-anabhighata

SK 25.

vaikrita ahamkara ekadasaka sattvika pravartate
tanmatra bhutade sa tamasa taijasa ubhayam

The set of eleven sattvic indriyas proceeds from
the Vaikriti form of I-Principle;
the set of five tamasic tanmatras proceed from
the Bhutadi form of I-Principle.
From the Taijasa form of I-Principle proceed both of them.

SK 26.

buddhi indriyani akhyani caksu srotram ghrana rasana tvak
karma indriyani ahu vak pani pada payu upastha

Organs of knowledge are called
    the eye, the ear, the nose, the tongue, the skin.
Organs of action are called
    the speech, the hand, the feet, the anus, the penis.

YS II.49

    tasmin sati svasa-prasvasayo gati-viccheda pranayama

YS II.50

    bahya-abhyantara-stambha-vritti
    desa-kala-sankhyabhi-paridrsta dirgha-suksma

YS II.51

    bahya-abhyantara-visaya-aksepi caturtha

SK 27.

atra mana ubhayatmaka sankalpaka ca sadharmaya indriya
nanatva bahya bheda ca guna-parinama-visesa

Of these (sense organs), the Mind possesses the nature of both.
It is the deliberating principle, and is also called a sense organ
since it posseses properties common to the sense organs.
Its multifariousness and also its external diversities are owing
to special modifications of the Qualities.

SK 28.

pancanam rupadisu alocanamatram isyate
pancanam vritti vacana adana viharana utsarga ca ananda

The function of the five in respect to form and the rest,
is considered to be mere observation.
Speech, manipulation, locomotion, excretion and gratification
are the functions of the other five.

YS II.52

    tata ksiyate prakasa-avaranam

YS II.53

    dharanasu ca yogyata manasa

YS II.54

    sva-visaya-asamprayoge cittasya svarupa-anukara ive indriyanam pratyahara

YS II.55

    tata parama vasyata indriyanam

YS III.1

    desa-bandha cittasya dharana

YS III.2

    tatra pratyaya-eka-tanata dhyanam

YS III.3

    tad evartha-matra-nirbhasa svarupa-sunyam iva samadhi

YS III.4

    trayam ekatra samyama

YS III.5

    taj-jayat prajna-aloka

YS III.6

    tasya bhumisu viniyoga

YS III.7

    trayam antar-angam purvebhya

YS III.8

    tad api bahir-angam nirbijasya

SK 29.

trayasya svalaksanyam vritti sa esa asamanya bhavati
samanya karana-vritti pranadhyah-vayavah-panca

Of the three internal instruments,
their own characteristics are their functions;
this is peculiar to each.
The common modification of the instruments is the five airs
such as prana and the rest.

YS III.9

    vyutthana-nirodha-samskarayor abhibhava-pradur-bhavau
    nirodha-kshana-cittanvayo nirodha-parinama

YS III.10

    tasya prasanta-vahita samskarat

    Manas, the understanding, first the faculty
    for the cognition of the general (of rules)

    a. Of concepts

    The concept in general and its distinction from intuition

        All cognitions, that is,
        all representations related with consciousness to an object,
        are either intuitions or concepts.
        An intuition is a singular representation (repraesentatio singularis),
        a concept a universal (repraesentatio per notas communes)
        or reflected representation (repraesentatio discursiva).
        Cognition through concepts is called thought (cognitio discursiva).

    Matter and form of concepts

        With every concept we are to distinguish matter and form.
        The matter of concepts is the object,
        their form universality.

    Empirical and pure concept

        A concept is either an empirical or a pure concept
        (vel empiricus vel intellectualis).
        A pure concept is one that is not abstracted from experience
        but arises rather from the understanding even as to content.
        An idea is a concept of reason whose object
        simply cannot be met with in experience.

    Concepts that are given (a priori or a posteriori)
    and concepts that are made

        All concepts, as to matter, are either
        given (conceptus dati) or made (conceptus factitii).
        The former are given either a priori or a posteriori.
        All concepts that are given empirically or a posteriori
        are called concepts of experience,
        all that are given a priori are called notions.

    Logical origin of concepts

        The origin of concepts as to mere form rests on reflection
        and on abstraction from the difference among things
        that are signified by a certain representation.
        And thus arises here the question:
        Which acts of the understanding constitute a concept
        or what is the same,
        Which are involved in the generation of a concept
        out of given representations?

    Logical Actus of comparison, reflection, and abstraction

        The logical actus of the understanding,
        through which concepts are generated as to their form, are:
        1.  comparison of representations among one another
            in relation to the unity of consciousness;
        2.  reflection as to how various representations can
            be conceived in one consciousness; and finally
        3.  abstraction of everything else
            in which the given representations differ.

SK 30.

drste catustayasya tu yugapat vrtti tasya kramasasca nirdista
tatha api adrste trayasya vrtti tat purvika

Of all the four, the functions are said to be
simultaneous and also successive
with regard to the seen objects;
with regard to the unseen objects
the functions of the three are preceeded by that.

YS III.11

    sarva-arthata-ekagratayo kshayodayau cittasya samadhi-parinama

    The logical form of all judgments consists of the objective unity of
    the apperception of the concepts contained therein

    Ahamkara, the determinative power of judgment,
    second the faculty for the subsumption of
    the particular under the general

    b. Of judgments

    Definition of a judgment in general

        A judgment is the representation of the unity of
        the consciousness of various representations,
        or the representation of their relation
        insofar as they constitute a concept.

    Matter and form of judgments

        Matter and form belong to every judgment
        as essential constituents of it.
        The matter of the judgment consists in
        the given representations that are combined
        in the unity of consciousness in the judgment,
        the form in the determination of the way that
        the various representations belong, as such,
        to one consciousness.

    Object of logical reflection the mere form of judgments

        Since logic abstracts from all real
        or objective difference of cognition,
        it can occupy itself as little with the matter of judgments
        as with the content of concepts.
        Thus it has only the difference among judgments
        in regard to their mere form to take into consideration.

SK 31.

svam svam vrittim pratipadyante paraspara akuta hetukam
purushartha eva hetu na kenacit karanam karyate

Instruments enter into their respective functions
being incited by mutual impulse.
The purpose of the Spirit is the sole motive
(for the activities of the instruments).
By none whatsoever is an instrument made to act.

YS III.12

    tata punashantoditau tulya-pratyayau cittasya-ekagrata-parinama

    Buddhi, reason, third the faculty for the determination of
    the particular from the general (for the derivation from principles)

    c. Of inferences

    Inference in general

        By inferring is to be understood that function of thought
        whereby one judgment is derived from another.
        An inference is thus in general
        the derivation of one judgment from the other.

    Immediate and mediate inferences

        All inferences are either immediate or mediate.
        An immediate inference (consequentia immediata) is
        the derivation (deductio) of one judgment from the other
        without a mediating judgment (judicium intermedium).
        An inference is mediate if, besides the concept
        that a judgment contains in itself,
        one needs still others in order to
        derive a cognition therefrom.

    Inferences of the understanding, inferences of reason,
    and inferences of the power of judgment

        Immediate inferences are also called
        inferences of the understanding;
        all mediate inferences, on the other hand,
        are either inferences of reason
        or inferences of the power of judgment.

SK 32.

karana trayodasavidha tad aharana dharana prakasakara
tasya karyam ca dasadha aharya dharya prakasya ca

Instruments are of thirteen kinds performing the functions of
seizing, sustaining and illuminating.
Its objects are of ten kinds, vis
the seized, the sustained, and the illumined.

YS III.13

    etena bhutendriyesu dharma-laksana-vastha-parinama vyakhyata

    a. Mechanism

SK 33.

antahkaranam trividham bahyam dasadha trayasya visayakhyam
bahyam sampratakalam abhyantaram karanam trikalam

The internal instruments are three-fold.
The external are ten-fold;
they are called the objects of the three (internal instruments).
The external instruments function at the present time and
the internal instruments function at all the three times.

YS III.14

    shantoditavyapadesya-dharmanupati dharmi

    b. Chemism

SK 34.

Tesam panca buddhi visesa-avisesa-visaya
Vak sabda-visaya-bhavati sesani tu pancavisayani

Of these, the five organs of knowledge have as their objects
both the specific and the general.
Speech has speech as its object;
the rest have all the five as its object.

YS III.15

    kramanyatva parinamanyatve hetu

    c. Teleology

SK 35.

yasmat buddhi santahkarana sarva visaya avagahate
tasmat trividham karanam dvari sesani dvarani

Since buddhi along with the other internal instruments
comprehends all of the objects these three instruments are
like the warders while the rest are like the gates.

YS III.16

    parinama-traya-samyama atitanagata-jnana

    UNIVERSAL LOGICAL PERFECTIONS OF COGNITION

    Manner and method

    All cognition, and a whole of cognition,
    must be in conformity with a rule.
    (Absence of rules is at the same time unreason.)
    But this rule is either that of manner (free)
    or that of method (compulsion).

    Form of science — Method

    Cognition, as science, must be arranged
    in accordance with a method.
    For science is a whole of cognition as a system,
    and not merely as an aggregate.
    It therefore requires a systematic cognition,
    hence one composed in accordance with rules
    on which we have reflected.

    Doctrine of method — Its object and end

    As the doctrine of elements in logic has for its content
    the elements and conditions of the perfection of a cognition,
    so the universal doctrine of method, as the other part of logic,
    has to deal with the form of a science in general,
    or with the ways of acting so as to connect
    the manifold of cognition in a science.

    Means for furthering the logical perfection of cognition

    The doctrine of method is supposed to expound the way for us
    to attain the perfection of cognition.
    Now one of the most essential logical perfections of cognition
    consists in its distinctness, thoroughness, and systematic ordering
    into the whole of a science.
    Accordingly, the doctrine of method will have principally to provide
    the means through which these perfections of cognition are furthered.

    Conditions of the distinctness of cognition

    The distinctness of cognitions and their combination
    in a systematic whole depends on the distinctness of concepts
    both in regard to what is contained in them
    and in respect of what is contained under them.
    The distinct consciousness of the content of concepts is furthered
    by exposition and definition of them,
    while the distinct consciousness of their extension, on the other hand,
    is furthered through logical division of them.

    First of all, then, of the means for furthering
    the distinctness of concepts in regard to their content.

    I. FURTHERING LOGICAL PERFECTION OF COGNITION THROUGH
    DEFINITION, EXPOSITION, AND DESCRIPTION OF CONCEPTS

    Definition

    A definition is a sufficiently distinct and precise concept
    (conceptus rei adaequatus in minimis terminis, complete determinatus).

    Analytic and synthetic definition

    All definitions are either analytic or synthetic.
    The former are definitions of a concept that is given,
    the latter of one that is made.
    A concept adequate to the thing, in minimal terms, completely determined.

    Concepts that are given and made a priori and a posteriori

    The given concepts of an analytic definition
    are given either a priori or a posteriori,
    just as the concepts of a synthetic definition, which are made,
    are made either a priori or a posteriori.

    Synthetic definitions through exposition or construction

    The synthesis of concepts that are made, out of which synthetic defini-
    tions arise, is either that of e3Cposition s (of appearances) or that of construc-
    tion. The latter is the synthesis of concepts that are made arbitrarily, the
    former the synthesis of concepts that are made empirically, i.e., from given
    appearances as their matter (conceptus factitii vel a priori vel per synthesin
    empiricam 1 '). Concepts that are made arbitrarily are the mathematical ones.

    Impossibility of empirically synthetic definitions

    Since the synthesis of empirical concepts is not arbitrary but
    rather is empirical and as such can never be complete
    (because one can always discover in experience more marks of the concept),
    empirical concepts cannot be defined, either.

    Analytic definitions through analysis of
    concepts given a priori or a posteriori

    All given concepts, be they given a priori or a posteriori,
    can be defined only through analysis.
    For one can make given concepts distinct only insofar as
    one successively makes their marks clear.
    If all the marks of a given concept are made clear,
    then the concept becomes completely distinct;
    if it does not contain too many marks,
    then it is at the same time precise,
    and from this there arises a definition of the concept.

    Expositions and descriptions

    Not all concepts can be defined, and not all need to be.
    There are approximations to the definition of certain concepts;
    these are partly expositions (expositiones), partly descriptions (descriptions).
    The expounding of a concept consists in the connected (successive)
    representation of its marks, insofar as these are found through analysis.
    Description is the exposition of a concept, insofar as it is not precise.
    Description can occur only with empirically given concepts. It has no
    determinate rules and contains only the materials for definition.

    Nominal and real definitions

    By mere definitions of names, or nominal definitions, are to be understood
    those that contain the meaning that one wanted arbitrarily to give to a
    certain name, and which therefore signify only the logical essence of their
    object, or which serve merely for distinguishing it from other objects.
    Definitions of things," 1 or real definitions, on the other hand, are ones that
    suffice for cognition of the object according to its inner determinations,
    since they present the possibility of the object from inner marks.

    Principal requirements of definition

    The essential and universal requirements that pertain to the completeness
    of a definition in general may be considered under the four principal
    moments of quantity, quality, relation, and modality:
    as to quantity - what concerns the sphere of the definition - the definition
    and the definitum must be convertible concepts" (conceptus redprocf), and hence
    the definition must be neither broader nor narrower than its definitum;
    as to quality, the definition must be
    a detailed and at the same time precise concept,
    as to relation, it must not be tautological,
    i.e., the marks of the definitum must, as grounds of its cognition,
    be different from it itself, and finally
    as to modality, the marks must be necessary, and
    hence not such as are added through experience.

    Rules for testing definitions

    In the testing of definitions four acts are to be performed;
    it is to be investigated, namely,
    1.  whether the definition considered as a proposition is true,
    2.  whether as a concept it is distinct,
    3.  whether as a distinct concept it is also detailed, and finally
    4.  whether as a detailed concept it is at the same time determinate,
        i.e., adequate to the thing itself.

    Rules for preparation of definitions

    The very same acts that belong to the testing of definition are
    also to be performed in the preparation of them.
    Toward this end, then, seek
    (1.)    true propositions,
    (2.)    whose predicate does not presuppose the concept of the thing;
    (3.)    collect several of them and compare them with
            the concept of the thing itself
            to see if they are adequate; and finally
    (4.)    see whether one mark does not lie in another
            or is not subordinated to it.

    II. FURTHERING THE PERFECTION OF COGNITION
    THROUGH LOGICAL DIVISION OF CONCEPTS

    Concept of logical division

    Every concept contains a manifold under itself
    insofar as the manifold agrees,
    but also insofar as it is different.
    The determination of a concept in regard to
    everything possible that is contained under it,
    insofar as things are opposed to one another,
    are distinct from one another,
    is called the logical division of the concept.
    The higher concept is called the divided concept (divisus),
    the lower concepts the members of the division (membra dividentia).

    Universal rules of logical division

    In every division of a concept we must see to it:
    1.  that the members of the division exclude
        or are opposed to one another,
        that furthermore
    2.  they belong under one higher concept (conceptus communis),
        and finally that
    3.  taken together they constitute the sphere of the divided concept
        or are equal to it.

    Codivision and subdivision

    Various divisions of a concept,
    which are made in various respects,
    are called codivisions,
    and division of the members of division
    is called a subdivision (subdivisio).

    Dichotomy and polytomy

    A division into two members is called dichotomy;
    but if it has more than two members, it is called polytomy.

    Various divisions of method

    Now as for what concerns in particular method itself
    in working up and treating scientific cognitions,
    there are various principal kinds of it,
    which we can present in accordance with the following division.

    1. Scientific or popular method

    Scientific or scholastic method differs from popular method
    through the fact that the former proceeds from
    basic and elementary propositions,
    but the latter from the customary and the interesting.
    The former aims for thoroughness and thus removes everything foreign,
    the latter aims at entertainment.

    2. Systematic or fragmentary method

    Systematic method is opposed to fragmentary or rhapsodic method.
    If one has thought in accordance with a method and then also expressed this
    method in the exposition, and if the transition from one proposition to
    another is distinctly presented, then one has treated a cognition systematically.
    If, on the other hand, one has thought according to a method but has
    not arranged the exposition methodically, such a method is to be called rhapsodic.

    3. Analytic or synthetic method

    Analytic is opposed to synthetic method.
    The former begins with the conditioned and grounded and
    proceeds to principles (a prindpiatis ad prindpia),
    while the latter goes from principles to consequences or
    from the simple to the composite.
    The former could also be called regressive,
    as the latter could progressive.

    4. Syllogistic [or] tabular method

    Syllogistic method is that according to which
    a science is expounded in a chain of inferences.
    That method in accordance with which a finished system is
    exhibited in its complete connection is called tabular.

    5. Acromatic or eromatic method

    Method is acroamatic insofar as someone only teaches,
    erotematic insofar as he asks as well.
    The latter method can be divided in turn
    into dialogic or Socratic method and catechistic method,
    accordingly as the questions are directed
    either to the understanding or merely to memory.

    Meditation

    By meditation is to be understood reflection, or methodical thought.
    Meditation must accompany all reading and learning,
    and for this it is requisite that
    one first undertake provisional investigations
    and then put his thoughts in order,
    or connect them in accordance with a method.

SK 36.

ete pradipa-kalpa paraspara vilaksana guna-visesa
krstnam prakasya purusasya-artham buddhau prayacchati

These (external instruments along with manas and ahamkara)
which are characteristic-wise different from one another
and are different modifications of the qualities
and which resemble a lamp (in action)
illuminating all (their respective objects)
present them to the Buddhi for the purpose of the Spirit

Cognition in general

All our cognition has a twofold relation,
first, a relation to the object,
second a relation to the subject.
In the former respect it is related to representation,
in the latter to consciousness,
the universal condition of all cognition in general.
(Consciousness is really a representation
that another representation is in me.)

In every cognition we must distinguish matter as the object,
and form as the way in which we cognize the object.

    If a savage sees a house from a distance, for example,
    with whose use he is not acquainted, he admittedly
    has before him in his representation the very same object as
    someone else who is acquainted with it determinately
    as a dwelling established for men.
    But as to form, this cognition of one and the same object
    is different in the two.
    With the one it is mere intuition,
    with the other it is intuition and concept at the same time.

The difference in the form of the cognition rests on
a condition that accompanies all cognition, on consciousness.
If I am conscious of the representation, it is clear,
if I am not conscious of it, obscure.

    Since consciousness is the essential condition of
    all logical form of cognitions,
    logic can and may occupy itself only
    with clear but not with obscure representations.
    In logic we do not see how representations arise,
    but merely how they agree with logical form.
    In general logic cannot deal at all with
    mere representations and their possibility either.
    This it leaves to metaphysics.
    Logic itself is occupied merely with the rules of thought in
    concepts, judgments, and inferences,
    as that through which all thought takes place.
    Something precedes, of course, before a representation becomes a concept.
    We will indicate that in its place, too.
    But we will not investigate how representations arise.
    Logic deals with cognition too, to be sure,
    because in cognition there is already thought.
    But representation is not yet cognition, rather,
    cognition always presupposes representation.
    And this latter cannot be explained at all.
    For we would always have to explain what representation is
    by means of yet another representation.

All clear representations, to which alone logical rules can be applied,
can now be distinguished in regard to distinctness and indistinctness.
If we are conscious of the whole representation,
but not of the manifold that is contained in it,
then the representation is indistinct.

    First, to elucidate this, an example in intuition.
    We glimpse a country house in the distance.
    If we are conscious that the intuited object is a house,
    then we must necessarily have a representation of
    the various parts of this house, the windows, doors, etc.
    For if we did not see the parts, we would not see the house itself either.
    But we are not conscious of this representation of the manifold of its parts,
    and our representation of the object indicated is thus itself an indistinct representation.

    If we want an example of indistinctness in concepts, furthermore, then
    the concept of beauty may serve. Everyone has a clear concept of beauty.
    But in this concept many different marks occur, among others that the
    beautiful must be something that (i.) strikes the senses and (2.) pleases
    universally. Now if we cannot explicate the manifold of these and other
    marks of the beautiful, then our concept of it is still indistinct.

    This is the situation with all simple representations,
    which never become distinct,
    not because there is confusion in them,
    but rather because there is no manifold to be found in them.
    One must call them indistinct, therefore, but not confused.

    And even with compound representations, too, in which
    a manifold of marks can be distinguished,
    indistinctness often derives not from confusion
    but from weakness of consciousness.
    Thus something can be distinct as to form,
    I can be conscious of the manifold in the representation,
    but the distinctness can diminish as to matter if
    the degree of consciousness becomes smaller,
    although all the order is there.
    This is the case with abstract representations.

Distinctness itself can be of two sorts:

First, sensible.

This consists in the consciousness of the manifold in intuition.

    I see the Milky Way as a whitish streak, for example;
    the light rays from the individual stars located in it
    must necessarily have entered my eye.
    But the representation of this was merely clear,
    and it becomes distinct only through the telescope,
    because then I glimpse the individual stars contained in the Milky Way.

Secondly, intellectual;

distinctness in concepts or distinctness of the understanding.
This rests on the analysis of the concept in regard to
the manifold that lies contained within it.

Thus in the concept of virtue, for example, are contained as marks
(1.) the concept of freedom,
(2.) the concept of adherence to rules (to duty),
(3.) the concept of overpowering the force of the inclinations,
in case they oppose those rules.

    Now if we break up the concept of virtue into its individual constituent parts,
    we make it distinct for ourselves through this analysis.
    By thus making it distinct, however, we add nothing to a concept; we only explain it.
    With distinctness, therefore, concepts are improved not as to matter but only as to form.

Intuitive and discursive cognition;

intuition and concept and in particular their difference

If we reflect on our cognitions in regard to the two essentially different
basic faculties, sensibility and the understanding, from which they arise,
then here we come upon the distinction between intuitions and concepts.
Considered in this respect, all our cognitions are, namely,
either intuitions or concepts.

    The former have their source in sensibility, the faculty of intuitions,
    the latter in the understanding, the faculty of concepts.
    This is the logical distinction between understanding and sensibility,
    according to which the latter provides nothing but intuitions,
    the former on the other hand nothing but concepts.

    The two basic faculties may of course be considered
    from another side and defined in another way:
    sensibility, namely, as a faculty of receptivity,
    the understanding as a faculty of spontaneity.
    But this mode of explanation is not logical but rather metaphysical.
    It is also customary to call sensibility the lower faculty,
    the understanding on the other hand the higher faculty,
    on the ground that sensibility gives the mere material for thought,
    but the understanding rules over this material
    and brings it under rules or concepts.

Logical and aesthetic perfection of cognition

The difference between aesthetic and logical perfection of cognition is
grounded on the distinction stated here
between intuitive and discursive cognitions, or
between intuitions and concepts.

A cognition can be perfect either
according to laws of sensibility or
according to laws of the understanding;
in the former case it is aesthetically perfect,
in the other logically perfect.

The two, aesthetic and logical perfection, are thus of different kinds;
the former relates to sensibility, the latter to the understanding.
The logical perfection of cognition rests on its agreement with the object,
hence on universally valid laws, and hence we
can pass judgment on it according to norms a priori.

Aesthetic perfection consists in the agreement of cognition with the subject and is grounded on
the particular sensibility of man.
In the case of aesthetic perfection, therefore,
there are no objectively and universally valid laws, in relation to
which we can pass judgment on it a priori in a way that is universally valid
for all thinking beings in general.
Insofar as there are nonetheless universal laws of sensibility,
which have validity subjectively for the whole of humanity
although not objectively and for all thinking beings in general,
we can think of an aesthetic perfection that contains the ground of a
subjectively universal pleasure.

This is beauty, that which pleases the senses in intuition and
can be the object of a universal pleasure just because
the laws of intuition are universal laws of sensibility.
Through this agreement with the universal laws of sensibility
the really, independently beautiful,
whose essence consists in mere form,
is distinguished in kind from the pleasant,
which pleases merely in sensation
through stimulation or excitement,
and which on this account can only be
the ground of a merely private pleasure.
It is this essential aesthetic perfection, too,
which, among all [perfections], is compatible with
logical perfection and may best be combined with it.

    Considered from this side, aesthetic perfection in regard to the essentially
    beautiful can thus be advantageous to logical perfection. In another
    respect it is also disadvantageous, however, insofar as we look, in the case
    of aesthetic perfection, only to the non-essentially beautiful, the stimulating
    or the exciting, which pleases the senses in mere sensation and does not
    relate to mere form but rather to the matter of sensibility. For stimulation
    and excitement, most of all, can spoil the logical perfection in our
    cognitions and judgments.

    In general, however, there always remains a kind of conflict between the
    aesthetic and the logical perfection of our cognition, which cannot be fully
    removed. The understanding wants to be instructed, sensibility enlivened;
    the first desires insight, the second comprehensibility. If cognitions are to
    instruct then they must to that extent be thorough; if they are to entertain
    at the same time, then they have to be beautiful as well. If an exposition is
    beautiful but shallow, then it can only please sensibility but not the understanding,
    but if it is thorough yet dry, only the understanding but not
    sensibility as well.

    Since the needs of human nature and the end of popularity in cognition
    demand, however, that we seek to unite the two perfections with one
    another, we must make it our task to provide aesthetic perfection for those
    cognitions that are in general capable of it, and to make a scholastically
    correct, logically perfect cognition popular through its aesthetic form.

But in this effort to combine aesthetic with logical perfection
in our cognitions we must not fail to attend to the following rules, namely:
(1.) that logical perfection is the basis of all other perfections and
hence cannot be wholly subordinated or sacrificed to any other;
(2.) that one should look principally to formal aesthetic perfection,
the agreement of a cognition with the laws of intuition,
because it is just in this that the essentially beautiful,
which may best be combined with logical perfection, consists;
(3.) that one must be very cautious with stimulation and excitement,
whereby a cognition affects sensation and acquires an interest for it,
because attention can thereby so easily be drawn from the object to the subject,
whence a very disadvantageous influence on
the logical perfection of cognition must evidently arise.

To acquaint us better with the essential differences that exist between
the logical and the aesthetic perfection of cognition,
not merely in the universal but from several particular sides,
we want to compare the two with one another in respect to
the four chief moments of quantity, quality, relation, and modality,
on which the passing of judgment as to the perfection of cognition depends.

A cognition is perfect
(1.) as to quantity if it is universal;
(2.) as to quality if it is distinct;
(3.) as to relation if it is true; and finally
(4.) as to modality if it is certain.

Considered from the viewpoints indicated,
a cognition will thus be logically perfect

as to quantity if it has objective universality

    (universality of the concept or of the rule),

as to quality if it has objective distinctness

    (distinctness in the concept),

as to relation if it has objective truth, and finally

as to modality if it has objective certainty.

To these logical perfections correspond now the following aesthetic
perfections in relation to those four principal moments, namely

aesthetic universality.

    This consists in the applicability of a cognition to a
    multitude of objects that serve as examples,
    to which application of it can be made,
    and whereby it becomes useful at the same time for the end of popularity;

aesthetic distinctness.

    This is distinctness in intuition, in which a concept
    thought abstractly is exhibited or elucidated in concrete through examples;

aesthetic truth.

    A merely subjective truth, which consists only in the agreement
    of cognition with the subject and the laws of sensory illusion,
    and which is consequently nothing more than a universal semblance.

aesthetic certainty.

    This rests on what is necessary in consequence of
    the testimony of the senses;
    what is confirmed through sensation and experience.

    With the perfections just mentioned two things are always to be found,
    which in their harmonious union make up perfection in general,
    namely, manifoldness and unity.
    Unity in the concept lies with the understanding,
    unity of intuition with the senses.
    Mere manifoldness without unity cannot satisfy us.
    And thus truth is the principal perfection among them all,
    because it is the ground of unity through
    the relation of our cognition to the object.
    Even in the case of aesthetic perfection,
    truth always remains the conditio sine qua non,
    the foremost negative condition,
    apart from which something cannot please taste universally.
    Hence no one may hope to make progress in the belles lettres
    if he has not made logical perfection the ground of his cognition.
    It is in the greatest possible unification of
    logical with aesthetic perfection in general,
    in respect to those cognitions that are both
    to instruct and to entertain,
    that the character and the art of
    the genius actually shows itself.

YS III.17

    sabda-artha-pratyayanam itaretara-adhyasat sankaras
    tat-pravibhaga-samyamat sarva-bhuta-ruta-jnanam

YS III.18

    samskara-saksat-karanat purva-jati-jnanam

YS III.19

    pratyayasya para-citta-jnanam

YS III.20

    na ca tat salambanam tasya-avisayi-bhutatvat

    PARTICULAR LOGICAL PERFECTIONS OF COGNITION

    A) Logical perfection of cognition as to quantity

    Quantity

        Extensive and intensive quantity
        Extensiveness and thoroughness
        or importance and fruitfulness of cognition
        Determination of the horizon of our cognition

    The quantity of cognition can be understood in two senses,
    either as extensive or as intensive quantity.
    The former relates to the extension of cognition and
    thus consists in its multitude and manifoldness;
    the latter relates to its content, which concerns the richness
    or the logical importance and fruitfulness of a cognition,
    insofar as it is considered as ground of many and great consequences (multa sed multum.)
    In expanding our cognitions or in perfecting them as to their extensive quantity
    it is good to make an estimate as to how far a cognition agrees with our ends and capabilities.
    This reflection concerns the determination of the horizon of our cognitions,
    by which is to be understood the congruence of the quantity of all cognitions
    with the capabilities and ends of the subject.

    The horizon can be determined

    1. logically, in accordance with the faculty or the powers of cognition
    in relation to the interest of the understanding.

    Here we have to pass judgment on how far we can go in our cognitions,
    how far we must go, and to what extent certain cognitions serve,
    in a logical respect, as means to various principal cognitions as our ends;

    2. aesthetically, in accordance with taste in relation to the interest of feeling.
    He who determines his horizon aesthetically seeks to arrange science according
    to the taste of the public, i.e., to make it popular, or in general to attain only
    such cognitions as may be universally communicated, and in which the class
    of the unlearned, too, find pleasure and interest;

    3. practically, in accordance with use in relation to the interest of the will.
    The practical horizon, insofar as it is determined according to the influence
    which a cognition has on our morality is pragmatic and is of the greatest importance.

    Thus the horizon concerns passing judgment on, and determining,
    what man can know, what he is permitted to know, and what he ought to know.
    Now as for what concerns the theoretically or logically determined horizon
    in particular - and it is of this alone that we can speak here - we can
    consider it either from the objective or from the subjective viewpoint.

    In regard to objects, the horizon is either historical or rational.
    The former is much broader than the other, indeed, it is immeasurably great,
    for our historical cognition has no limits.
    The rational horizon, on the other hand, may be fixed,
    e.g., it may be determined to what kind of objects
    mathematical cognition cannot be extended.
    So too in respect of philosophical cognition of reason,
    as to how far reason can go here a priori without any experience.

    In relation to the subject the horizon is either the universal and absolute,
    or a particular and conditioned one (a private horizon).

    By the absolute and universal horizon is to be understood
    the congruence of the limits of human cognitions with
    the limits of the whole of human perfection in general.

    And here, then, the question is:

    In general, what can man, as man, know?

    The determination of the private horizon depends upon various
    empirical conditions and special considerations,
    e.g., age, sex, station, mode of life, etc.
    Every particular class of men has its particular horizon in relation
    to its special powers of cognition, ends, and standpoints, every mind its
    own horizon according to the standard of the individuality of its powers
    and its standpoint. Finally, we can also think a horizon of healthy reason
    and a horizon of science, which latter still requires principles, in accordance
    with which to determine what we can and cannot know.
    What we cannot know is beyond f our horizon, what we do not need to
    know is outside our horizon. This latter can hold only relatively, however,
    in relation to various particular private ends, to whose accomplishment
    certain cognitions not only do not contribute anything but could even be
    an obstacle. For no cognition is, absolutely and for every purpose, useless
    and unusable, although we may not always be able to have insight into its
    use. Hence it is an objection as unwise as it is unjust that is made to great
    men who labor in the sciences with painstaking industry when shallow
    minds ask, What is the use ofthat?' We must simply never raise this ques-
    tion if we want to occupy ourselves with the sciences. Even granted that a
    science could give results only concerning some possible object, it would
    still for that reason alone be useful enough. Every logically perfect cogni-
    tion always has some possible use, which, although we are as yet unac-
    quainted with it, will perhaps be found by posterity. If in the cultivation of
    the sciences one had always looked only to material gain, their use, then
    we would have no arithmetic or geometry. Besides, our understanding is
    so arranged that it finds satisfaction in mere insight, even more than in the
    use that arises therefrom. Plato noted this. Man feels in this his own
    excellence, he senses what it means to have understanding. Men who do
    not sense this must envy the animals. The inner worth that cognitions have
    through logical perfection is not to be compared with the outer, their worth
    in application.

    Like that which lies outside our horizon, insofar as we, in accordance
    with our purposes, do not need to know it, as dispensable for us, that which
    lies beneath' our horizon, insofar as we ought not to know it as harmful to
    us, is to be understood in a relative sense but never in an absolute one.

    With respect to the extension and the demarcation of our cognition,
    the following rules are to be recommended:

    1. One must determine his horizon early,
    but of course only when one can determine it oneself,
    which usually does not occur before the 2oth year;

    2. not alter it lightly or often
    (not turn from one thing to another);

    3. not measure the horizon of others by one's own,
    and not consider as useless what is of no use to us;
    it would be presumptuous to want to determine others' horizons,
    because one is not sufficiently acquainted,
    in part with their capabilities,
    in part with their purposes;

    4. neither extend it too far nor restrict it too much.
    For he who wants to know too much ends by knowing nothing,
    and conversely, he who believes of some things
    that they do not concern him, often deceives himself;
    as when, e.g., the philosopher believes of history
    that it is dispensable for him.

    One should also seek

    5. to determine in advance the absolute horizon of
    the whole human race (as to past and to future time),
    as well as also

    6. to determine, in particular, the position that
    our science occupies in the whole of cognition.
    The Universal Encyclopedia serves for this as
    a universal map (mappe-monde) of the sciences.

    7. In determining his own particular horizon one should carefully consider for
    which part of cognition one has the greatest capability and pleasure, what is
    more or less necessary in regard to certain duties, what cannot coexist with
    the necessary duties; and finally

    8. one should of course always seek
    to expand his horizon rather than to narrow it.

    Opposed to the logical perfection of cognition
    in regard to its extension stands ignorance.

    A negative imperfection, or imperfection of lack, which,
    on account of the restrictions of the understanding,
    is inseparable from our cognition.

    We can consider ignorance from an objective or from a subjective viewpoint.

    1. Taken objectively, ignorance is either material or formal.
    The former consists in a lack of historical cognitions,
    the other in a lack of rational cognitions.
    One does not have to be completely ignorant in any field,
    but one can well restrict historical knowledge in order to
    devote oneself more to rational knowledge, or conversely.

    2. In the subjective sense, ignorance is either learned, scientific, or is common.
    He who has distinct insight into the restrictions of cognition,
    hence into the field of ignorance from where it begins,
    e.g., the philosopher who sees and proves how little one can know of
    gold in regard to its structure due to a lack of the requisite data,
    is ignorant artfully or in a learned way.
    He who is ignorant, on the other hand, without
    having insight into the grounds of the limits of knowledge,
    and without concerning himself with this, is so in a common, not a scientific way.
    Such a one does not even know that he knows nothing.
    For one can never represent his ignorance except through science,
    as a blind man cannot represent darkness until he has become sighted.

    Cognition of one's ignorance presupposes science, then, and
    makes one at the same time modest, while imagined knowledge puffs one up.
    Hence Socrates' non-knowledge was a laudable ignorance,
    a knowledge of non-knowledge, according to his own admission.
    It is precisely those who possess very many cognitions then, and
    who for all that are astounded at the multitude of what they do not know,
    who cannot be reproached with their ignorance.
    Ignorance in things whose cognition lies beyond our horizon is in
    general irreproachable (inculpabilis), and in regard to the speculative use of
    our faculty of cognition it can be allowed (although only in the relative
    sense), insofar as the objects here lie not beyond our horizon but yet outside
    it. It is shameful, however, in things that it is quite necessary and also easy
    to know.

    There is a distinction between not knowing something and ignoring
    something, i.e., taking no notice of it. It is good to ignore much that it is not
    good for us to know. Abstracting is distinct from both of these. One ab-
    stracts from a cognition when one ignores its application, whereby one
    gets it in abstracto and can better consider it in the universal as a principle.
    Such abstraction from what does not belong to our purpose in the cogni-
    tion of a thing is useful and praiseworthy.

    This perfection of cognition, whereby it qualifies for easy and universal
    communication, could also be called external extension or the extensive
    quantity of a cognition, insofar as it is widespread externally among men.
    Since cognitions are so many and manifold, one will do well to make
    himself a plan, in accordance with which he orders the sciences in the way
    that best agrees with, and contributes to the furtherance of, his ends. All
    cognitions stand in a certain natural connection with one another. Now if,
    in striving to expand his cognitions, one does not look to their connection,
    then extensive knowledge amounts to nothing more than a mere rhapsody.
    If one makes one principal science his end, however, and
    considers all other cognitions only as means for achieving it,
    then he brings a certain systematic character into his knowledge.
    And in order to go to work on extending his cognitions
    according to such a well ordered and purposive plan,
    one must seek, therefore, to become acquainted with
    this connection of cognitions among themselves.
    For this, the sciences get guidance from architectonic,
    which is a system in accordance with ideas,
    in which the sciences are considered in regard to their kinship and
    systematic connection in a whole of cognition that interests humanity.

    Now as for what concerns the intensive quantity of cognition;
    its content, or its richness and importance,
    which is essentially distinct from its extensive quantity,
    its mere extensiveness, as we noted above,
    we want here to add only the following few remarks:

    1. A cognition that is concerned with what is great,
    with the whole in the use of the understanding,
    is to be distinguished from subtlety in what is small (micrology).

    2. Every cognition that furthers logical perfection
    as to form is to be called logically important;
    e.g., every mathematical proposition, every law of nature
    into which we have distinct insight, every correct philosophical explanation.
    Practical importance cannot be foreseen, one must simply wait and watch for it.

    3. Importance must not be confused with difficulty.
    A cognition can be difficult without being important, and conversely.
    Difficulty, then, does not decide either for or against
    the worth or the importance of a cognition.
    This rests on the quantity or multiplicity of its consequences.
    A cognition is the more important accordingly as
    it has more or greater consequences, as the use that
    may be made of it is more.
    Cognition without important consequences is called cavilling;
    scholastic philosophy, e.g., was of this sort.

    B) Logical perfection of cognition as to relation

    Truth

        Material and formal, or logical, truth;
        Criteria of logical truth
        Falsehood and error
        Illusion, as source of error;
        Means for avoiding errors

    A principal perfection of cognition, indeed,
    the essential and inseparable condition of all its perfection, is truth.
    Truth, it is said, consists in the agreement of cognition with its object.

    In consequence of this mere nominal explanation, my cognition,
    to count as true, is supposed to agree with its object.
    Now I can compare the object with my cognition, however,
    only by cognizing it.
    Hence my cognition is supposed to confirm itself,
    which is far short of being sufficient for truth.
    For since the object is outside me, the cognition in me,
    all I can ever pass judgment on is whether
    my cognition of the object agrees with my cognition of the object.
    The ancients called such a circle in explanation a diallelon.
    And actually the logicians were always reproached with this mistake by the skeptics, who
    observed that with this explanation of truth it is just as when someone
    makes a statement before a court and in doing so appeals to a witness with
    whom no one is acquainted, but who wants to establish his credibility by
    maintaining that the one who called him as witness is an honest man.
    The accusation was grounded, too.
    Only the solution of the indicated problem
    is impossible without qualification and for every man.

    The question here is, namely, whether and to what extent there is a
    criterion of truth that is certain, universal, and useful in application.
    For this is what the question, What is truth?, ought to mean.

    To be able to decide this important question we must distinguish that
    which belongs to the matter in our cognition and is related to the object
    from that which concerns its mere form, as that condition without which a
    cognition would in general never be a cognition. With respect to this
    distinction between the objective, material relation in our cognition and the
    subjective, formal relation, the question above thus breaks down into these
    two particular ones:

    Is there a universal material, and
    Is there a universal formal criterion of truth?

    A universal material criterion of truth is not possible;
    it is even self-contradictory.
    For as a universal criterion, valid for all objects in general, it
    would have to abstract fully from all difference among objects, and yet at
    the same time, as a material criterion, it would have to deal with just this
    difference, in order to be able to determine whether a cognition agrees
    with just that object to which it is related and not just with any object in
    general, in which case nothing would really be said.
    Material truth must consist in this agreement of a cognition with just that determinate object
    to which it is related, however.
    For a cognition that is true in regard to one
    object can be false in relation to other objects.
    Hence it is absurd to demand a universal material criterion of truth,
    which should abstract and at the same time not abstract from all difference among objects.

    If the question is about universal formal criteria of truth, however, then
    here it is easy to decide that of course there can be such a thing.
    For formal truth consists merely in the agreement of cognition with itself,
    in complete abstraction from all objects whatsoever and from all difference among them.
    And the universal formal criteria of truth are accordingly nothing
    other than universal logical marks of the agreement of cognition with itself
    or - what is one and the same - with the universal laws of the understanding and of reason.
    These formal, universal criteria are of course not sufficient for objective truth,
    but they are nonetheless to be regarded as its conditio sine qua non.
    For the question of whether cognition agrees with its objects
    must be preceded by the question of whether it agrees with itself (as to form).
    And this is a matter for logic.

    The formal criteria of truth in logic are

    1. the principle of contradiction,
    2. the principle of sufficient reason.

    Through the former the logical possibility of a cognition is determined,
    through the latter its logical actuality.
    To the logical actuality of a cognition it pertains, namely:

    First: that it be logically possible, not contradict itself.

        This characteristic of internal logical truth is only negative, however;
        for a cognition that contradicts itself is of course false,
        but if it does not contradict itself it is not always true.

    Second: that it be logically grounded,

    that it (a) have grounds and (b) not have false consequences.

        This second criterion of external logical truth or
        of accessibility to reason, which concerns the logical connection of
        a cognition with grounds and consequences, is positive.

    And here the following rules are valid:

    1. From the truth of the consequence we may infer
    the truth of the cognition as ground, but only negatively:

        if one false consequence flows from a cognition,
        then the cognition itself is false.
        For if the ground were true,
        then the consequence would also have to be true,
        because the consequence is determined by the ground.
        But one cannot infer conversely that
        if no false consequence flows from a cognition, then it is true;
        for one can infer true consequences from a false ground.

    2. If all the consequences of a cognition are true,
    then the cognition is true too.

        For if there were something false in the cognition,
        then there would have to be a false consequence too.
        From the consequence, then, we may infer to a ground,
        but without being able to determine this ground.
        Only from the complex of all consequences can one
        infer to a determinate ground,
        infer that it is the true ground.

    The former mode of inference, according to which the consequence
    can only be a negatively and indirectly sufficient criterion of the truth of a
    cognition, is called in logic the apagogic mode (modus tollens).

        This procedure, of which frequent use is made in geometry,
        has the advantage that I may derive just one false consequence
        from a cognition in order to prove its falsehood.
        To show, e.g., that the earth is not flat, I may
        just infer apagogically and indirectly,
        without bringing forth positive and direct grounds:
        If the earth were flat, then the pole star would
        always have to be at the same height;
        but this is not the case, consequently it is not flat.

    With the other, the positive and direct mode of inference (modus ponens)
    the difficulty enters that the totality of the consequences cannot be
    cognized apodeictically, and that one is therefore led by the indicated
    mode of inference only to a probable and hypothetically true cognition
    (hypotheses), in accordance witii the presupposition that where many
    consequences are true, all the remaining ones' may be true too.

    Thus we will be able to advance three principles here as universal,
    merely formal or logical criteria of truth; these are

    the principle of contradiction and of identity

    (principium contradictionis and identitatis),
    through which the internal possibility of a cognition is determined for problematic judgments;

    the principle of sufficient reason

    (principium rationis suffidentis),
    on which rests the (logical) actuality of a cognition,
    the fact that it is grounded, as material for assertoric judgments;

    the principle of the excluded middle

    (principium exclusi medii inter duo contradictoria),
    on which the (logical) necessity of a cognition is grounded,
    that we must necessarily judge thus and not otherwise,
    that the opposite is false, for apodeictic judgments.

    The opposite of truth is falsehood, which,
    insofar as it is taken for truth, is called error.
    An erroneous judgment, for there is error as well as truth only in judgment,
    is thus one that confuses the illusion of truth with truth itself.
    It is easy to have insight into how truth is possible,
    since here the understanding acts in accordance with its essential laws.
    But it is hard to comprehend how error in the formal sense of the word,
    how the form of thought contrary to the understanding is possible,
    just as we cannot in general comprehend how any power
    should deviate from its own essential laws.
    We cannot seek the ground of errors in
    the understanding itself and its essential laws, then,
    just as little as we can in the restrictions of the understanding,
    in which lies the cause of ignorance, to be sure,
    but not in any way the cause of error.
    Now if we had no other power of cognition
    but the understanding, we would never err.
    But besides the understanding, there lies in us
    another indispensable source of cognition.
    That is sensibility, which gives us the material for thought, and
    in doing this works according to other laws than those the understanding does.
    Error cannot arise from sensibility in and by itself, however,
    because the senses simply do not judge.

    The ground for the origin of all error will therefore have to be sought
    simply and solely in the unnoticed influence of sensibility upon
    the understanding, or to speak more exactly, upon judgment.
    This influence, namely, brings it about that in judgment
    we take merely subjective grounds to be objective, and
    consequently confuse the mere illusion of truth with truth itself.
    For it is just in this that the essence of illusion consists,
    which on this account is to be regarded as a ground
    for holding a false cognition to be true.

    What makes error possible, then, is illusion,
    in accordance with which the merely subjective is
    confused in judgment with the objective.
    In a certain sense, however, one can make the understanding the author
    of errors, namely, insofar as it allows itself,
    due to a lack of requisite attention to that influence of sensibility,
    to be misled by the illusion arising therefrom into holding
    merely subjective determining grounds of judgment to be objective ones,
    or into letting that which is true only according to
    the laws of sensibility hold as true in accordance with its own laws.
    In the restrictions of the understanding, then,
    lies only the responsibility for ignorance;
    the responsibility for error we have to assign to ourselves.
    Nature has denied us many cognitions, to be sure,
    it leaves us in unavoidable ignorance concerning so much,
    but still it does not cause error.
    We are misled into this by our own inclination to judge and
    to decide even where, on account of our limitedness,
    we are not able to judge and to decide.

    Every error into which the human understanding can fall is only partial,
    however, and in every erroneous judgment there must always lie something true.
    For a total error would be a complete opposition to
    the laws of the understanding and of reason.
    But how could that, as such, in any way come from the understanding and,
    insofar as it is still a judgment, be held to be a product of the understanding.

    In respect to the true and the erroneous in our cognition,
    we distinguish an exact cognition from a rough one.

        Cognition is exact when it is adequate to its object, or
        when there is not the slightest error in regard to its object,
        and it is rough when there can be errors in it yet
        without being a hindrance to its purpose.
        This distinction concerns the broader or narrower determinateness
        of our cognition (cognitio late vel stride determinata).
        Initially it is sometimes necessary to determine a cognition
        in a broader extension (late determinare),
        particularly in historical things.
        In cognitions of reason everything must be
        determined exactly (striae), however.
        In the case of broad determination
        one says that a cognition is determined praeter propter.
        Whether a cognition ought to be determined roughly or exactly
        always depends on its purpose.
        Broad determination leaves a certain play for error,
        which still can have its determinate limits, however.
        Error occurs particularly where a broad determination is taken for a strict one,
        e.g., in matters of morality, where everything must be determined striae.
        Those who do not do this are called by the English latitudinarians.
        One can distinguish subtlety, as a subjective perfection of cognition,
        from exactness, as an objective perfection, since here cognition is
        fully congruent with its object.
        A cognition is subtle when one discovers in it that which usually escapes
        the attention of others. It requires a higher degree of attention, then, and
        a greater application of power of the understanding.
        Many reprove all subtlety because they cannot attain it.
        But in itself it always brings honor to the understanding and
        is even laudable and necessary, insofar as it is applied to an object worthy of observation.
        When one could have attained the same end with less attention and
        effort of the understanding, however, and yet one uses more,
        then one makes a useless expense and falls into subtleties,
        which are difficult, to be sure, but do not have any use (nugae difficiles).
        As the rough is opposed to the exact, so is the crude to the subtle.

    From the nature of error whose concept, as we noted,
    contains as an essential mark, besides falsehood,
    also the illusion of truth
    we get the following important rule
    for the truth of our cognition:

        To avoid errors and no error is unavoidable,
        at least not absolutely or without qualification,
        although it can be unavoidable relatively,
        for the cases where it is unavoidable for us to judge,
        even with the danger of error,
        to avoid errors, then, one must seek to
        disclose and to explain their source, illusion.
        Very few philosophers have done that, however.
        They have only sought to refute the errors themselves,
        without indicating the illusion from which they arise.
        This disclosure and breaking up of illusion is
        a far greater service to truth, however, than the direct refutation
        of errors, whereby one does not block their source and cannot guard
        against the same illusion misleading one into errors again in other cases
        because one is not acquainted with it.
        For even if we are convinced that we have erred,
        then in case the illusion that grounds our error
        has not been removed we still have scruples,
        however little we can bring forth in justification of them.

    Error in principles is greater than in their application.

    An external mark or an external touchstone of truth is
    the comparison of our own judgments with those of others,
    because the subjective will not be present in all others in the same way,
    so that illusion can thereby be cleared up.
    The incompatibility of the judgments of others with our own is
    thus an external mark of error and is to be regarded as a cue to investigate
    our procedure in judgment, but not for that reason to reject it at once;
    one can perhaps be right about the thing but not right in manner,
    in the exposition.

    The common human understanding (sensus communis) is also in itself
    a touchstone for discovering the mistakes of the artificial use
    of the understanding.
    This is what it means to orient oneself in thought or
    speculative use of reason by means of the common understanding,
    one uses the common understanding as a test for passing judgment
    on the correctness of the speculative use.

    Universal rules and conditions for avoiding error in general are:

    1) to think for oneself,
    2) to think oneself in the position of someone else, and
    3) always to think in agreement with oneself.

    The maxim of thinking for oneself can be called
    the enlightened mode of thought;
    the maxim of putting oneself in the viewpoint of others in thought,
    the extended mode of thought;
    and the maxim of always thinking in agreement with one self,
    the consequent or coherent mode of thought.

    C) Logical perfection of cognition as to quality

    Clarity

        Concept of a mark in general
        Various kinds of marks
        Determination of the logical essence of a thing
        Its distinction from the real essence
        Distinctness, a higher degree of clarity
        Aesthetic and logical distintness
        Distinction between analytic and synthetic distinctness

    From the side of the understanding, human cognition is discursive;
    it takes place through representations which take as the ground of cognition
    that which is common to many things, hence through marks' as such.
    Thus we cognize things through marks and that is called cognizing,
    [the German word for which] comes from [the German word for] being acquainted.

    A mark is that in a thing which constitutes
    a part of the cognition of it, or
    what is the same - a partial representation,
    insofar as it is considered as
    ground of cognition of the whole representation.
    All our concepts are marks, accordingly, and
    all thought is nothing other than a representing through marks.
    Every mark may be considered from two sides:
    First, as a representation in itself;
    Second, as belonging, as a partial concept, to
    the whole representation of a thing, and thereby
    as ground of cognition of this thing itself.
    All marks, considered as grounds of cognition, have two uses,
    internal or an external use.
    The internal use consists in derivation,
    in order to cognize the thing itself
    through marks as its grounds of cognition.
    The external use consists in comparison,
    insofar as we can compare one thing with others
    through marks in accordance with the rules of
    identity or diversity.

    There are many specific differences among marks,
    on which the following classification of them is grounded.

    1. Analytic or synthetic marks.

    The former are partial concepts of my actual concept
    (marks that I already think therein),
    while the latter are partial concepts of the merely possible complete concept
    (which is supposed to come to be through a synthesis of several parts).

    The former are all concepts of reason,
    the latter can be concepts of experience.

    2. Coordinate or subordinate.

    This division of marks concerns their connection after or under one another.

    Marks are coordinate insofar as each of them is
    represented as an immediate mark of the thing
    and are subordinate insofar as one mark is
    represented in the thing only by means of the other.
    The combination of coordinate marks to form
    the whole of a concept is called an aggregate,
    the combination of subordinate concepts a series.
    The former, the aggregation of coordinate marks,
    constitutes the totality of the concept,
    regard to synthetic empirical concepts, can never be completed,
    but rather resembles a straight line without limits.
    The series of subordinate marks terminates a pane ante,
    or on the side of the grounds, in concepts which cannot be broken up,
    which cannot be further analyzed on account of their simplicity;
    a pane post, or in regard to the consequences, it is infinite,
    because we have a highest genus but no lowest species.
    With the synthesis of every new concept in the aggregation of coordinate marks,
    the extensive or extended distinctness grows,
    as intensive or deep distinctness grows with the further
    analysis of the concept in the series of subordinate marks.
    This latter kind of distinctness, since it necessarily
    contributes to thoroughness and coherence of the cognition,
    is thus principally a matter of philosophy and is pursued
    to the highest degree in metaphysical investigations in particular.

    3. Affirmative or negative marks.

    Through the former we cognize what the thing is,
    through the latter what it is not.

    Negative marks serve to keep us from errors.
    Hence they are unnecessary where it is impossible to err,
    and are necessary and of importance only in
    those cases where they keep us from an important error
    into which we can easily fall.
    Thus in regard to the concept, e.g., of a being like God,
    negative marks are quite necessary and important.
    Through affirmative marks we seek to understand something,
    through negative marks, into which all marks can be transformed,
    we only seek not to misunderstand or not to err,
    even if we should not thereby become acquainted with anything.

    4. Important and fruitful, or empty and unimportant, marks.

    A mark is important and fruitful if it is a ground of cognition for great
    and numerous consequences, partly in regard to its internal use, its use in
    derivation, insofar as it is sufficient for cognizing thereby a great deal in
    the thing itself, partly in respect to its external use, its use in comparison,
    insofar as it thereby contributes to cognizing both the similarity of a thing
    to many others and its difference from many others.
    We have to distinguish logical importance and fruitfulness from practical,
    from usefulness and utility, by the way.

    5. Sufficient and necessary or insufficient and accidental marks.

    A mark is sufficient insofar as it suffices always
    to distinguish the thing from all others;
    otherwise it is insufficient, as the mark of barking is,
    for example, for dogs.
    The sufficiency of marks, as well as their importance,
    is to be determined only in a relative sense,
    in relation to ends that are intended through a cognition.
    Necessary marks, finally, are those that must always be there to be found
    in the thing represented.
    Marks of this sort are also called essential and are
    opposed to extra-essential and accidental marks,
    which can be separated from the concept of the thing.
    Among necessary marks there is another distinction, however.
    Some of them belong to the thing as grounds of other marks of one and
    the same thing, while others belong only as consequences of other marks.
    The former are primitive and constitutive marks
    (constitutiva, essentialia in sensu strictissimo),
    the others are called attributes (consectaria, rationata)
    and belong admittedly to the essence of the thing,
    but only insofar as they must first be derived from its essential points,
    as the three angles follow from the three sides
    in the concept of the triangle, for example.
    Extra-essential marks are again of two kinds;
    they concern either internal determinations of a thing (moat) or
    its external relations (relationes).
    Thus the mark of learnedness signifies an inner determination of a man,
    but being a master or a servant only an external relation.

    The complex of all the essential parts of a thing, or the sufficiency of its
    marks as to coordination or subordination, is the essence
    (complexus notarum primitivarum, interne conceptui dato sufficientium;
    s. complexus notarum, conceptum aliquem primitive constituentium).

    In this explanation, however, we must not think at all of
    the real or natural essence of things,
    into which we are never able to have insight.
    For since logic abstracts from all content of cognition, and
    consequently also from the thing itself, in this science
    the talk can only be of the logical essence of things.
    And into this we can easily have insight.
    For it includes nothing further than the cognition of
    all the predicates in regard to which
    an object is determined through its concept;
    whereas for the real essence of the thing (esse ret)
    we require cognition of those predicates on which,
    as grounds of cognition, everything that belongs
    to the existence of the thing depends.
    If we wish to determine, e.g., the logical essence of body, then
    we do not necessarily have to seek for the data for this in nature;
    we may direct our reflection to the marks which, as essential points
    (constitutiva, rationes) originally constitute the basic concept of the thing.
    For the logical essence is nothing but the first basic concept of
    all the necessary marks of a thing (esse conceptus).

    The first stage of the perfection of our cognition
    as to quality is thus its clarity.
    A second stage, or a higher degree of clarity, is distinctness.
    This consists in clarity of marks.

    First of all we must here distinguish
    logical distinctness in general from
    aesthetic distinctness.
    Logical distinctness rests on objective clarity of marks,
    aesthetic distinctness on subjective clarity.
    The former is a clarity through concepts,
    the latter a clarity through intuition.
    The latter kind of distinctness consists, then,
    in a mere liveliness and understandability,
    in a mere clarity through examples in concreto
    (for much that is not distinct can still be understandable, and
    conversely, much that is hard to understand can still be distinct,
    because it goes back to remote marks, whose connection with intuition
    is possible only through a long series).
    Objective distinctness frequently causes
    subjective obscurity, and conversely.
    Hence logical distinctness is often possible only to the detriment of
    aesthetic distinctness, and conversely aesthetic distinctness
    through examples and similarities which do not fit exactly but are only
    taken according to an analogy often becomes harmful to logical distinctness.
    Besides, examples are simply not marks and do not belong to the
    concept as parts but, as intuitions, to the use of the concept.
    Distinctness through examples, mere understandability, is hence
    of a completely different kind than distinctness through concepts as marks.
    Lucidity consists in the combination of both,
    of aesthetic or popular distinctness and
    of scholastic or logical distinctness.
    For one thinks of a lucid mind as the talent for a
    luminous presentation of abstract and thorough cognitions
    that is congruent with the common understanding's power of comprehension.
    Next, as for what concerns logical distinctness in particular,
    it is to be called complete distinctness insofar as all the marks
    which, taken together, make up the whole concept have come to clarity.
    A completely distinct concept can be so, again,
    either in regard to the totality of its coordinate marks or
    in respect to the totality of its subordinate marks.
    Extensively complete or sufficient distinctness of a concept
    consists in the total clarity of its coordinate marks,
    which is also called exhaustiveness.
    Total clarity of subordinate marks constitutes
    intensively complete distinctness, profundity.

    The former kind of logical distinctness can also be called the external
    completeness (completudo externd) of the clarity of marks, the other the inter-
    nal completeness (completudo internet). The latter can be attained only with
    pure concepts of reason and with arbitrary concepts, but not with empiri-
    cal concepts.

    The extensive quantity of distinctness, insofar as it is not superfluous is
    called precision. Exhaustiveness (completudo) and precision (praecisio) to-
    gether constitute adequacy (cognitio, quae rem adaequat); and the completed
    perfection of a cognition (consummata cognitionis perfectio) consists (as to qual-
    ity) in intensively adequate cognition, profundity, combined with extensively
    adequate cognition, exhaustiveness and precision.

    Since, as we have noted, it is the business of logic to make clear concepts
    distinct, the question now is in what way it makes them distinct.
    Logicians of the Wolffian school place the act of making cognitions
    distinct' entirely in mere analysis of them. But not all distinctness rests on
    analysis of a given concept. It arises thereby only in regard to those marks
    that we already thought in the concept, but not in respect to those marks
    that are first added to the concept as parts of the whole possible concept.
    The kind of distinctness that arises not through analysis but through
    synthesis of marks is synthetic distinctness. And thus there is an essential
    difference between the two propositions: to make a distinct concept and to
    make a concept distinct.
    For when I make a distinct concept, I begin with the parts and proceed
    from these toward the whole. Here there are no marks as yet at hand; I
    acquire them only through synthesis. From this synthetic procedure
    emerges synthetic distinctness, then, which actually extends my concept
    as to content through what is added as a mark beyond' the concept in (pure
    or empirical) intuition. The mathematician and the natural philosopher
    make use of this synthetic procedure in making distinctness in concepts.*
    For all distinctness of properly mathematical cognition, as of all cognition
    based on experience, rests on such an expansion of it through the synthe-
    sis of marks.
    When I make a concept distinct, however, my cognition does not grow
    at all as to content through this mere analysis. The content remains the
    same, only the form is altered, in that I learn to distinguish better, or to
    cognize with clearer consciousness, what lay in the given concept already.
    As nothing is added to a map through the mere illumination' of it, so a
    given concept is not in the least increased through its mere illumination
    by means of the analysis of its marks.
    To synthesis pertains the making distinct of objects, to analysis the
    making distinct of concepts. In the latter case the whole precedes the parts, in
    the former the parts precede the whole. The philosopher only makes given
    concepts distinct.
    Sometimes one proceeds synthetically even when the
    concept that one wants to make distinct in this way is already given.
    This is often the case with propositions based on experience, in case one is not
    yet satisfied with the marks already thought in a given concept.
    The analytic procedure for creating distinctness, with which alone logic
    can occupy itself, is the first and principal requirement in making our
    cognition distinct.
    For the more distinct our cognition of a thing is, the
    stronger and more effective it can be too.
    But analysis must not go so far
    that in the end the object itself disappears.
    If we were conscious of all that we know, we would have to be aston-
    ished at the great multitude of our cognitions.

    In regard to the objective content of our cognition in general,
    we may think the following degrees, in accordance with which
    cognition can, in this respect, be graded:

    The first degree of cognition is: to represent something;

    The second: to represent something with consciousness,
    or to perceive (percipere);

    The third: to be acquainted with something (noscere), or
    to represent something in comparison with other things,
    both as to sameness and as to difference;

    The fourth: to be acquainted with something with consciousness,
    to cognize it (cognoscere).

    Animals are acquainted with objects too, but they do not cognize them.

    The fifth: to understand something (intelligere),
    to cognize something through the understanding by means of concepts,
    or to conceive.

    One can conceive much, although one cannot comprehend it;
    e.g., a perpetuum mobile, whose impossibility is shown in mechanics.

    The sixth: to cognize something through reason, or
    to have insight into it (perspicere).

    With few things do we get this far,
    and our cognitions become fewer and fewer in number
    the more that we seek to perfect them as to content.

    The seventh, finally: to comprehend something (comprehendere),
    to cognize something through reason or a priori
    to the degree that is sufficient for our purpose.

    For all our comprehension is only relative,
    sufficient for a certain purpose;
    we do not comprehend anything without qualification.
    Nothing can be comprehended more than what the mathematician demonstrates,
    that all lines in the circle are proportional.
    And yet he does not comprehend how it happens that such a simple
    figure has these properties.
    The field of understanding or of the understanding is thus in general
    much greater than the field of comprehension or of reason.

    D) Logical perfection of cognition as to modality

    Certainty

        Concept of holding-to-be-true in general
        Modi of holding-to-be-true: opining, believing and knowing
        Conviction and persuasion

    Truth is an objective property of cognition;
    the judgment through which something is represented as true,
    the relation to an understanding, and thus to a particular subject, is,
    subjectively, holding-to-be-true.

    Holding-to-be-true is in general of two kinds, certain or uncertain.
    Certain holding-to-be-true, or certainty, is
    combined with consciousness of necessity,
    while uncertain holding-to-be-true, or uncertainty, is
    combined with consciousness of the contingency
    or the possibility of the opposite.
    The latter is again either subjectively
    as well as objectively insufficient,
    or objectively insufficient but subjectively sufficient.
    The former is called opinion, the latter must be called belief.

    Accordingly, there are three kinds or modi of holding-to-be-true:
    opining, believing, and knowing.
    Opining is problematic judging,
    believing is assertoric judging,
    and knowing is apodeictic judging.

    For what I merely opine I hold in judging, with consciousness, only to be problematic;
    what I believe I hold to be assertoric, but not as objectively necessary, only as
    subjectively so (holding only for me); what I know, finally, I hold to be
    apodeiäically certain, i.e., to be universally and objectively necessary (hold-
    ing for all), even granted that the object to which this certain holding-to-
    be-true relates should be a merely empirical truth. For this distinction in
    holding-to-be-true according to the three modi just named concerns only
    the power of judgment in regard to the subjective criteria for subsumption of
    a judgment under objective rules.

    Thus, for example, our holding-to-be-true of immortality would be
    merely problematic in case we only act as if we were immortal,
    but it would be assertoric in case we believe that we are immortal,
    and it would be apodeictic, finally, in case we all knew that
    there is another life after this one.

    There is an essential difference, then, between opining, believing, and knowing,
    which we wish to expound more exactly and in more detail here.

    1. Opining.

    Opining, or holding-to-be-true based on a ground of cognition
    that is neither subjectively nor objectively sufficient, can be regarded
    as provisional judging (sub conditione suspensiva ad interim) that
    one cannot easily dispense with.

    One must first opine before one accepts and maintains,
    but in doing so must guard oneself against holding an opinion to be
    something more than mere opinion.
    For the most part, we begin with
    opining in all our cognizing. Sometimes we have an obscure premonition
    of truth, a thing seems to us to contain marks of truth; we suspect its truth
    even before we cognize it with determinate certainty.
    But now where does mere opining really occur?
    Not in any sciences that contain cognitions a priori,
    hence neither in mathematics nor in metaphysics nor in morals,
    but merely in empirical cognitions: in physics, psychology, etc.
    For it is absurd to opine a priori.
    In fact, too, nothing could be more ridiculous than, e.g.,
    only to opine in mathematics.
    Here, as in metaphysics and in morals,
    the rule is either to know or not to know.
    Thus matters of opinion can only be objects of a cognition by experience,
    a cognition which is possible in itself but impossible for us in accordance
    with the restrictions and conditions of our faculty of experience and the
    attendant degree of this faculty that we possess.
    Thus, for example, the ether of modern physicists is a mere matter of opinion.
    For with this as with every opinion in general, whatever it may be,
    I see that the opposite could perhaps yet be proved.
    Thus my holding-to-be-true is here both objectively and subjectively insufficient,
    although it can become complete, considered in itself.

    2. Believing.

    Believing, or holding-to-be-true based on a ground that is
    objectively insufficient but subjectively sufficient, relates to objects in
    regard to which we not only cannot know anything but also cannot opine
    anything, indeed, cannot even pretend there is probability, but can only be
    certain that it is not contradictory to think of such objects as one does
    think of them. What remains here is a free holding-to-be-true, which is
    necessary only in a practical respect given a priori, hence a holding-to-be-
    true of what I accept on moral grounds, and in such a way that I am certain
    that the opposite can never be proved.

    Believing is not a special source of cognition. It is a kind of incomplete holding-to-be-true
    with consciousness, and if considered as restricted to a particular kind of object (which
    pertains only to believing), it is distinguished from opining not by its degree but rather by the
    relation that it has as cognition to action. Thus the businessman, for example, to strike a
    deal, needs not just to opine that there will be something to be gained thereby, but to believe
    it, i.e., to have his opinion be sufficient for an undertaking into the uncertain. Now we have
    theoretical cognitions (of the sensible) in which we can come to certainty, and in regard to
    everything that we can call human cognition this latter must be possible. We have just such
    certain cognitions, and in fact completely a priori, in practical laws, but these are grounded
    on a supersensible principle (of freedom) and in fact in us ourselves, as a principle of practical
    reason. But this practical reason is a causality in regard to a likewise supersensible object, the
    highest good, which is not possible through our faculty in the sensible world. Nature as object
    of our theoretical reason must nonetheless agree with this, for the consequence or effect of this
    idea is supposed to be met with in the world of the senses. Thus we ought to act so as to
    make this end actual.
    Now in the world of the senses we also find traces of an artistic misdom, and we believe
    that the cause of the world also works with moral wisdom toward the highest good. This is a
    holding-to-be-true that is enough for action, i.e., a belief. Now we do not need this for action
    in accordance with moral laws, for these are given through practical reason alone, but we
    need to accept a highest wisdom as the object of our moral will, an object beyond the mere
    legitimacy of our actions, toward which we cannot avoid directing our ends. Although
    objectively this would not be a necessary relation of our faculty of choice, subjectively the
    highest good is still necessarily the object of a good (even of a human) will, and hence belief
    in its attainability is necessarily presupposed.

    There is no mean between the acquisition of a cognition through experience (a posteriori)
    and through reason (apriori). But there is a mean between the cognition of an object and the
    mere presupposition of its possibility, namely, an empirical ground or a ground of reason for
    Matters of belief are thus I) not objects of empirical cognition. Hence so-
    called historical belief cannot really be called belief, either, and cannot be
    opposed as such to knowledge, since it can itself be knowledge. Holding-
    to-be-true based on testimony is not distinguished from holding-to-be-
    true through one's own experience either as to degree or as to kind.
    II) [N]or [are they] objects of cognition by reason (cognition a priori),
    whether theoretical, e.g., in mathematics and metaphysics, or practical, in
    morals.
    One can believe mathematical truths of reason on testimony, to be sure,
    partly because error here is not easily possible, partly, too, because it can
    easily be discovered, but one cannot know them in this way, of course. But
    accepting this possibility in relation to a necessary extension of the field of possible objects
    beyond those whose cognition is possible for us. This necessity occurs only in regard to that
    in which the object is cognized as practical and, through reason, as practically necessary, for
    to accept something on behalf of the mere extension of theoretical cognition is always
    contingent. This practically necessary presupposition of an object is the presupposition of the
    possibility of the highest good as object of choice, hence also of the condition of this
    possibility (God, freedom, and immortality). This is a subjective necessity to accept the
    reality of the object for the sake of the necessary determination of the will. This is the casus
    extravrdinarius, without which practical reason cannot maintain itself in regard to its neces-
    sary end, and here a favor necessitate proves useful to it in its own judgment. It cannot acquire
    an object logically, but can only oppose what hinders it in the use of this idea, which belongs
    to it practically.
    This belief is the necessity to accept the objective reality of a concept (of the highest
    good), i.e., the possibility of its object, asapriori necessary object of choice. If we look merely
    to actions, we do not need this belief. But if we wish to extend ourselves through actions to
    possession of the end that is thereby possible, then we must accept that this end is com-
    pletely possible. Hence I can only say that / see myself necessitated through my end, in
    accordance with laws of freedom, to accept as possible a highest good in the world, but I
    cannot necessitate anyone else through grounds (the belief is free).
    A belief of reason can never aim at theoretical cognition, then, for there objectively
    insufficient holding-to-be-true is merely opinion. It is merely a presupposition of reason for
    a subjective though absolutely necessary practical purpose. The sentiment toward moral
    laws leads to an object of choice, which [choice] is determinable through pure reason. The
    acceptance of the feasibility of this object, and hence of the reality of its cause, is a moral
    belief, or a free holding-to-be-true that is necessary for moral purposes for completion of
    one's ends.
    Fides is really good faith 1 in thepactum, or subjective trust-' in one another, that one will keep
    his promise to the other, with full faith and credit.* The first when the pactum is made, the
    second when it is to be concluded.

    In accordance with the analogy, practical reason is, as it were, the promisor,' man the
    promissee," the good expected from the deed the promised."
    philosophical truths of reason may not even be believed, they must simply
    be known; for philosophy does not allow mere persuasion. And as for
    what concerns in particular the objects of practical cognition by reason in
    morals, rights and duties, there can just as little be mere belief in regard to
    them. One must be fully certain whether something is right or wrong, in
    accordance with duty or contrary to duty, allowed or not allowed. In moral
    things one cannot risk anything on the uncertain, one cannot decide any-
    thing on the danger of trespass against the law. Thus it is not enough for the
    judge, for example, that he merely believe that someone accused of a crime
    actually committed this crime. He must know it (juridically), or he acts
    unconscientiously.
    The only objects that are matters of belief are those in which
    holding-to-be-true is necessarily free, i.e., is not determined through
    objective grounds of truth that are independent of the nature and the
    interest of the subject.
    Thus also on account of its merely subjective grounds, believing yields
    no conviction that can be communicated and that commands universal
    agreement, like the conviction that comes from knowledge. Only / myself
    can be certain of the validity and unalterability of my practical belief, and
    my belief in the truth of a proposition or the actuality of a thing is what
    takes the place of a cognition only in relation to me without itself being a
    cognition.
    He who does not accept what it is impossible to know but morally neces-
    sary to presuppose is morally unbelieving. At the basis of this kind of
    unbelief lies always a lack of moral interest. The greater a man's moral
    sentiment," the firmer and more lively will be his belief in all that he feels
    himself necessitated to accept and to presuppose out of moral interest, for
    practically necessary purposes.

    3. Knowing.

    Holding-to-be-true based on a ground of cognition that is
    objectively as well as subjectively sufficient, or certainty, is
    either empirical or rational, accordingly as it is grounded either
    on experience, one's own as well as that communicated by others, or on reason.
    This distinction relates, then, to the two sources from which
    the whole of our cognition is drawn: experience and reason.

    Rational certainty, again, is either mathematical or philosophical certainty.
    The former is intuitive, the latter discursive.
    Mathematical certainty is also called evidence, because an intuitive cog-
    nition is clearer than a discursive one. Although the two, mathematical
    and philosophical cognition of reason, are in themselves equally certain,
    the certainty is different in kind in them.
    Empirical certainty is original (originarie empirica) insofar as I become
    certain of something from my own experience, and derived (derivative em-
    pirica) insofar as I become certain through someone else's experience. The
    latter is also usually called historical certainty.
    Rational certainty is distinguished from empirical certainty by the con-
    sciousness of necessity that is combined with it; hence it is apodeictic cer-
    tainty, while empirical certainty is only assertoric. We are rationally certain
    of that into which we would have had insight a priori even without any
    experience. Hence our cognitions can concern objects of experience and
    the certainty concerning them can still be both empirical and rational at
    the same time, namely, insofar as we cognize an empirically certain propo-
    sition from principles a priori.
    We cannot have rational certainty of everything, but where we can have
    it, we must put it before empirical certainty.
    All certainty is either unmediated or mediated, i.e., it either requires a
    proof, or it is not capable of and does not require any proof. Even if so
    much in our cognition is certain only mediately, i.e., through a proof,
    there must still be something indemonstrable or immediately certain, and the
    whole of our cognition must proceed from immediately certain proposi-
    tions.

    The proofs on which any mediated or mediate certainty of a cognition
    rests are either direct proofs or indirect, i.e., apagogical ones.
    When I prove a truth from its grounds I provide a direct proof for it,
    and when I infer the truth of a proposition from the falsehood of its
    opposite I provide an indirect one.
    If this latter is to have validity, however,
    the propositions must be opposed contradictorily or diametraliter.
    For two propositions opposed only as contraries (contrarie opposita) can both be false.
    A proof that is the ground of mathematical certainty is called a demonstration,
    and that which is the ground of philosophical certainty is called an acroamatic proof.
    The essential parts of any proof in general are its matter and its form,
    or the ground of proof and the consequentia.

    From [the German word for] knowing comes [the German word for] science,
    by which is to be understood the complex of a cognition as a system.
    It is opposed to common cognition;
    to the complex of a cognition as mere aggregate.
    A system rests on an idea of the whole, which precedes the parts,
    while with common cognition on the other hand,
    or a mere aggregate of cognitions, the parts precede the whole.
    There are historical sciences and sciences of reason.

    In a science we often know only the cognitions
    but not the things represented through them;
    hence there can be a science of that of which
    our cognition is not knowledge.

    From the foregoing observations concerning
    the nature and the kinds of holding-to-be-true
    we can now draw the universal result that
    all our conviction is thus either logical or practical.
    When we know, namely, that we are free of all subjective grounds and
    yet the holding-to-be-true is sufficient,
    then we are convinced, and in fact logically convinced, or
    convinced on objective grounds (the object is certain).
    Complete holding-to-be-true on subjective grounds,
    which in a practical relation hold just as much as objective grounds,
    is also conviction, though not logical but rather pratical conviction (I am certain).
    And this practical conviction, or this moral belief of reason,
    is often firmer than all knowledge.
    With knowledge one still listens to opposed grounds,
    but not with belief, because here it does not depend
    on objective grounds but on the moral interest of the subject.

SK 37.

yasmat buddhi purusasya upabho-gam sarvam pratisadhyati
sa eva ca suksma pradhana-purusantaram visinasti

Because it is the Buddhi that accomplishes the experiences
with regard to all objects to the Purusha.
It is that again that discriminates the subtle difference
between the Pradhana and the Purusha.

YS III.21

    kaya-rupa-samyama tad-grahya-sakti-stambhe
    caksu-prakasa-asamprayoge antardhana

YS III.22

    etena sabda-adi-antardhana ukta
