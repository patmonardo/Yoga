YS III.1

    desa-bandha cittasya dharana

YS III.2

    tatra pratyaya-eka-tanata dhyana

YS III.3

    tad evartha-matra-nirbhasa svarupa-sunya iva samadhi

YS III.4

    traya ekatra samyama

YS III.5

    taj-jayat prajna-aloka

YS III.6

    tasya bhumisu viniyoga

YS III.7

    traya antar-angam purvebhya

YS III.8

    tad api bahir-angam nirbijasya

SK 29.

trayasya svalaksanya vritti sa esa asamanya bhavati
samanya karana-vritti pranadhyah-vayavah-panca

Of the three internal instruments,
their own characteristics are their functions;
this is peculiar to each.
The common modification of the instruments is the five airs
such as prana and the rest.

YS III.9

    vyutthana-nirodha-samskaraya abhibhava-pradur-bhava
    nirodha-kshana-citta-anvaya nirodha-parinama

YS III.10

    tasya prasanta-vahita samskara

    Manas, the understanding, first the faculty
    for the cognition of the general (of rules)

    a. Of concepts

    The concept in general and its distinction from intuition

        All cognitions, that is,
        all representations related with consciousness to an object,
        are either intuitions or concepts.
        An intuition is a singular representation (repraesentatio singularis),
        a concept a universal (repraesentatio per notas communes)
        or reflected representation (repraesentatio discursiva).
        Cognition through concepts is called thought (cognitio discursiva).

    Matter and form of concepts

        With every concept we are to distinguish matter and form.
        The matter of concepts is the object,
        their form universality.

    Empirical and pure concept

        A concept is either an empirical or a pure concept
        (vel empiricus vel intellectualis).
        A pure concept is one that is not abstracted from experience
        but arises rather from the understanding even as to content.
        An idea is a concept of reason whose object
        simply cannot be met with in experience.

    Concepts that are given (a priori or a posteriori)
    and concepts that are made

        All concepts, as to matter, are either
        given (conceptus dati) or made (conceptus factitii).
        The former are given either a priori or a posteriori.
        All concepts that are given empirically or a posteriori
        are called concepts of experience,
        all that are given a priori are called notions.

    Logical origin of concepts

        The origin of concepts as to mere form rests on reflection
        and on abstraction from the difference among things
        that are signified by a certain representation.
        And thus arises here the question:
        Which acts of the understanding constitute a concept
        or what is the same,
        Which are involved in the generation of a concept
        out of given representations?

    Logical Actus of comparison, reflection, and abstraction

        The logical actus of the understanding,
        through which concepts are generated as to their form, are:
        1.  comparison of representations among one another
            in relation to the unity of consciousness;
        2.  reflection as to how various representations can
            be conceived in one consciousness; and finally
        3.  abstraction of everything else
            in which the given representations differ.

SK 30.

drste catustayasya tu yugapat vrtti tasya kramasasca nirdista
tatha api adrste trayasya vrtti tat purvika

Of all the four, the functions are said to be
simultaneous and also successive
with regard to the seen objects;
with regard to the unseen objects
the functions of the three are preceeded by that.

YS III.11

    sarva-arthata-ekagrataya kshaya-udaya cittasya samadhi-parinama

    The logical form of all judgments consists of the objective unity of
    the apperception of the concepts contained therein

    Ahamkara, the determinative power of judgment,
    second the faculty for the subsumption of
    the particular under the general

    b. Of judgments

    Definition of a judgment in general

        A judgment is the representation of the unity of
        the consciousness of various representations,
        or the representation of their relation
        insofar as they constitute a concept.

    Matter and form of judgments

        Matter and form belong to every judgment
        as essential constituents of it.
        The matter of the judgment consists in
        the given representations that are combined
        in the unity of consciousness in the judgment,
        the form in the determination of the way that
        the various representations belong, as such,
        to one consciousness.

    Object of logical reflection the mere form of judgments

        Since logic abstracts from all real or objective
        difference of cognition,
        it can occupy itself as little with the matter of judgments
        as with the content of concepts.
        Thus it has only the difference among judgments
        in regard to their mere form to take into consideration.

SK 31.

sva sva vritti pratipadyante paraspara akuta hetuka
purushartha eva hetu na kenacit karana karyate

Instruments enter into their respective functions
being incited by mutual impulse.
The purpose of the Spirit is the sole motive
(for the activities of the instruments).
By none whatsoever is an instrument made to act.

YS III.12

    tata puna-shanta-udita tulya-pratyaya cittasya-ekagrata-parinama

    Buddhi, reason, third the faculty for the determination of
    the particular from the general (for the derivation from principles)

    c. Of inferences

    Inference in general

        By inferring is to be understood that function of thought
        whereby one judgment is derived from another.
        An inference is thus in general
        the derivation of one judgment from the other.

    Immediate and mediate inferences

        All inferences are either immediate or mediate.
        An immediate inference (consequentia immediata) is
        the derivation (deductio) of one judgment from the other
        without a mediating judgment (judicium intermedium).
        An inference is mediate if, besides the concept
        that a judgment contains in itself,
        one needs still others in order to
        derive a cognition therefrom.

    Inferences of the understanding, inferences of reason,
    and inferences of the power of judgment

        Immediate inferences are also called
        inferences of the understanding;
        all mediate inferences, on the other hand,
        are either inferences of reason
        or inferences of the power of judgment.

SK 32.

karana trayodasavidha tad aharana dharana prakasakara
tasya karyam ca dasadha aharya dharya prakasya ca

Instruments are of thirteen kinds performing the functions of
seizing, sustaining and illuminating.
Its objects are of ten kinds, vis
the seized, the sustained, and the illumined.

YS III.13

    etena bhuta-indriyesu dharma-laksana-vastha-parinama vyakhyata

    a. Mechanism

SK 33.

antakarana trividha bahya dasadha trayasya visayakhya
bahya sampratakala abhyantara karana trikala

The internal instruments are three-fold.
The external are ten-fold;
they are called the objects of the three (internal instruments).
The external instruments function at the present time and
the internal instruments function at all the three times.

YS III.14

    shanta-udita-avyapadesya-dharmanupati dharmi

    b. Chemism

SK 34.

Tesa panca buddhi visesa-avisesa-visaya
Vak sabda-visaya-bhavati sesani tu pancavisayani

Of these, the five organs of knowledge have as their objects
both the specific and the general.
Speech has speech as its object;
the rest have all the five as its object.

YS III.15

    krama-anyatva parinama-anyatva hetu

    c. Teleology

C. THE IDEA

First of all, then, of the means for furthering
the distinctness of concepts in regard to their content.

SK 35.

yasmat buddhi sa-antakarana sarva visaya avagahate
tasmat trividha karana dvari sesani dvarani

Since buddhi along with the other internal instruments
comprehends all of the objects these three instruments are
like the warders while the rest are like the gates.

YS III.16

    parinama-traya-samyama atita-nagata-jnana

    a. Life

    Manner and method

    All cognition, and a whole of cognition,
    must be in conformity with a rule.
    (Absence of rules is at the same time unreason.)
    But this rule is either that of manner (free)
    or that of method (compulsion).

    Form of science — Method

    Cognition, as science, must be arranged
    in accordance with a method.
    For science is a whole of cognition as a system,
    and not merely as an aggregate.
    It therefore requires a systematic cognition,
    hence one composed in accordance with rules
    on which we have reflected.

    Doctrine of method — Its object and end

    As the doctrine of elements in logic has for its content
    the elements and conditions of the perfection of a cognition,
    so the universal doctrine of method, as the other part of logic,
    has to deal with the form of a science in general,
    or with the ways of acting so as to connect
    the manifold of cognition in a science.

    Means for furthering the logical perfection of cognition

    The doctrine of method is supposed to expound the way
    for us to attain the perfection of cognition.
    Now one of the most essential logical perfections of cognition
    consists in its distinctness, thoroughness, and systematic ordering
    into the whole of a science.
    Accordingly, the doctrine of method will have principally to provide
    the means through which these perfections of cognition are furthered.

    Conditions of the distinctness of cognition

    The distinctness of cognitions and their combination
    in a systematic whole depends on the distinctness of concepts
    both in regard to what is contained in them
    and in respect of what is contained under them.
    The distinct consciousness of
    the content of concepts is furthered
    by exposition and definition of them,
    while the distinct consciousness of
    their extension, on the other hand,
    is furthered through logical division of them.

SK 36.

ete pradipa-kalpa paraspara vilaksana guna-visesa
krstna prakasya purusasya-artha buddha prayacchati

These (external instruments along with manas and ahamkara)
which are characteristic-wise different from one another
and are different modifications of the qualities
and which resemble a lamp (in action)
illuminating all (their respective objects)
present them to the Buddhi for the purpose of the Spirit

YS III.17

    sabda-artha-pratyayana itaretara-adhyasa sankaras
    tat-pravibhaga-samyama sarva-bhuta-ruta-jnana

YS III.18

    samskara-saksa-karana purva-jati-jnana

YS III.19

    pratyayasya para-citta-jnana

YS III.20

    na ca tat salambana tasya-avisayi-bhutatva

    b. Knowing

    Cognition in general
        Intuitive and discursive cognition;
            intuition and concept
            and in particular their difference
        Logical and aesthetic perfection of cognition

    All our cognition has a twofold relation,
    first, a relation to the object,
    second a relation to the subject.
    In the former respect it is related to representation,
    in the latter to consciousness,
    the universal condition of all cognition in general.
    (Consciousness is really a representation
    that another representation is in me.)

    In every cognition we must distinguish matter as the object,
    and form as the way in which we cognize the object.

        If a savage sees a house from a distance, for example,
        with whose use he is not acquainted,
        he admittedly has before him in his representation
        the very same object as someone else
        who is acquainted with it determinately
        as a dwelling established for men.
        But as to form, this cognition of one and the same object
        is different in the two.
        With the one it is mere intuition,
        with the other it is intuition and concept at the same time.

    The difference in the form of the cognition rests on
    a condition that accompanies all cognition, on consciousness.
    If I am conscious of the representation, it is clear,
    if I am not conscious of it, obscure.

        Since consciousness is the essential condition
        of all logical form of cognitions,
        logic can and may occupy itself only with clear
        but not with obscure representations.
        In logic we do not see how representations arise,
        but merely how they agree with logical form.
        In general logic cannot deal at all
        with mere representations
        and their possibility either.
        This it leaves to metaphysics.
        Logic itself is occupied merely
        with the rules of thought in
        concepts, judgments, and inferences,
        as that through which all thought takes place.
        Something precedes, of course,
        before a representation becomes a concept.
        We will indicate that in its place, too.
        But we will not investigate how representations arise.
        Logic deals with cognition too, to be sure,
        because in cognition there is already thought.
        But representation is not yet cognition, rather,
        cognition always presupposes representation.
        And this latter cannot be explained at all.
        For we would always have to
        explain what representation is
        by means of yet another representation.

    All clear representations,
    to which alone logical rules can be applied,
    can now be distinguished in regard to
    distinctness and indistinctness.
    If we are conscious of the whole representation,
    but not of the manifold that is contained in it,
    then the representation is indistinct.

        First, to elucidate this, an example in intuition.

        We glimpse a country house in the distance.
        If we are conscious that the intuited object is a house,
        then we must necessarily have a representation of
        the various parts of this house, the windows, doors, etc.
        For if we did not see the parts,
        we would not see the house itself either.
        But we are not conscious of this
        representation of the manifold of its parts,
        and our representation of the object indicated is
        thus itself an indistinct representation.

        If we want an example of indistinctness in concepts, furthermore,
        then the concept of beauty may serve.
        Everyone has a clear concept of beauty.
        But in this concept many different marks occur,
        among others that the beautiful must be something that
        (i.) strikes the senses and
        (2.) pleases universally.
        Now if we cannot explicate the manifold of these
        and other marks of the beautiful,
        then our concept of it is still indistinct.

        This is the situation with all simple representations,
        which never become distinct,
        not because there is confusion in them,
        but rather because there is no manifold to be found in them.
        One must call them indistinct, therefore, but not confused.

        And even with compound representations, too, in which
        a manifold of marks can be distinguished,
        indistinctness often derives not from confusion
        but from weakness of consciousness.
        Thus something can be distinct as to form,
        I can be conscious of the manifold in the representation,
        but the distinctness can diminish as to matter
        if the degree of consciousness becomes smaller,
        although all the order is there.
        This is the case with abstract representations.

    Distinctness itself can be of two sorts:

    First, sensible.

    This consists in the consciousness of the manifold in intuition.

        I see the Milky Way as a whitish streak, for example;
        the light rays from the individual stars located in it
        must necessarily have entered my eye.
        But the representation of this was merely clear,
        and it becomes distinct only through the telescope,
        because then I glimpse the individual stars
        contained in the Milky Way.

    Secondly, intellectual;

    distinctness in concepts or distinctness of the understanding.
    This rests on the analysis of the concept in regard to
    the manifold that lies contained within it.

    Thus in the concept of virtue, for example,
    are contained as marks
    (1.) the concept of freedom,
    (2.) the concept of adherence to rules (to duty),
    (3.) the concept of overpowering the force of the inclinations,
    in case they oppose those rules.

        Now if we break up the concept of virtue
        into its individual constituent parts,
        we make it distinct for ourselves through this analysis.
        By thus making it distinct, however, we add nothing to a concept;
        we only explain it.
        With distinctness, therefore, concepts are improved
        not as to matter but only as to form.

    If we reflect on our cognitions in regard to
    the two essentially different basic faculties,
    sensibility and the understanding, from which they arise,
    then here we come upon the distinction
    between intuitions and concepts.
    Considered in this respect,
    all our cognitions are, namely,
    either intuitions or concepts.

    The former have their source in sensibility, the faculty of intuitions,
    the latter in the understanding, the faculty of concepts.
    This is the logical distinction between understanding and sensibility,
    according to which the latter provides nothing but intuitions,
    the former on the other hand nothing but concepts.

        The two basic faculties may of course be considered
        from another side and defined in another way:
        sensibility, namely, as a faculty of receptivity,
        the understanding as a faculty of spontaneity.
        But this mode of explanation is not logical but rather metaphysical.
        It is also customary to call sensibility the lower faculty,
        the understanding on the other hand the higher faculty,
        on the ground that sensibility gives the mere material for thought,
        but the understanding rules over this material
        and brings it under rules or concepts.

    Logical and aesthetic perfection of cognition

    The difference between aesthetic and logical
    perfection of cognition
    is grounded on the distinction stated here
    between intuitive and discursive cognitions, or
    between intuitions and concepts.

    A cognition can be perfect either
    according to laws of sensibility or
    according to laws of the understanding;
    in the former case it is aesthetically perfect,
    in the other logically perfect.

    The two, aesthetic and logical perfection,
    are thus of different kinds;
    the former relates to sensibility,
    the latter to the understanding.
    The logical perfection of cognition rests on
    its agreement with the object,
    hence on universally valid laws,
    and hence we can pass judgment on it
    according to norms a priori.

    Aesthetic perfection consists in
    the agreement of cognition with the subject
    and is grounded on the particular sensibility of man.
    In the case of aesthetic perfection, therefore,
    there are no objectively and universally valid laws,
    in relation to which we can pass judgment on it a priori
    in a way that is universally valid for all thinking beings in general.
    Insofar as there are nonetheless universal laws of sensibility,
    which have validity subjectively for the whole of humanity
    although not objectively and for all thinking beings in general,
    we can think of an aesthetic perfection that contains
    the ground of a subjectively universal pleasure.

    This is beauty, that which pleases the senses in intuition
    and can be the object of a universal pleasure
    just because the laws of intuition are
    universal laws of sensibility.
    Through this agreement with
    the universal laws of sensibility
    the really, independently beautiful,
    whose essence consists in mere form,
    is distinguished in kind from the pleasant,
    which pleases merely in sensation
    through stimulation or excitement,
    and which on this account can only be
    the ground of a merely private pleasure.
    It is this essential aesthetic perfection, too,
    which, among all [perfections], is compatible with
    logical perfection and may best be combined with it.

        Considered from this side,
        aesthetic perfection in regard to the essentially beautiful
        can thus be advantageous to logical perfection.
        In another respect it is also disadvantageous, however,
        insofar as we look, in the case of aesthetic perfection,
        only to the non-essentially beautiful,
        the stimulating or the exciting,
        which pleases the senses in mere sensation
        and does not relate to mere form
        but rather to the matter of sensibility.
        For stimulation and excitement, most of all,
        can spoil the logical perfection
        in our cognitions and judgments.

        In general, however, there always remains a kind of conflict
        between the aesthetic and the logical perfection of our cognition,
        which cannot be fully removed.
        The understanding wants to be instructed, sensibility enlivened;
        the first desires insight, the second comprehensibility.
        If cognitions are to instruct
        then they must to that extent be thorough;
        if they are to entertain at the same time,
        then they have to be beautiful as well.
        If an exposition is beautiful but shallow,
        then it can only please sensibility
        but not the understanding,
        but if it is thorough yet dry,
        only the understanding
        but not sensibility as well.

        Since the needs of human nature
        and the end of popularity in cognition
        demand, however, that we seek to unite
        the two perfections with one another,
        we must make it our task to provide
        aesthetic perfection for those cognitions
        that are in general capable of it,
        and to make a scholastically correct,
        logically perfect cognition
        popular through its aesthetic form.

    But in this effort to combine aesthetic with logical perfection
    in our cognitions we must not fail to attend to the following rules, namely:
    (1.) that logical perfection is the basis of all other perfections
    and hence cannot be wholly subordinated or sacrificed to any other;
    (2.) that one should look principally to formal aesthetic perfection,
    the agreement of a cognition with the laws of intuition,
    because it is just in this that the essentially beautiful,
    which may best be combined with logical perfection, consists;
    (3.) that one must be very cautious with stimulation and excitement,
    whereby a cognition affects sensation and acquires an interest for it,
    because attention can thereby so easily be drawn
    from the object to the subject,
    whence a very disadvantageous influence
    on the logical perfection of cognition
    must evidently arise.

    To acquaint us better with the essential differences that exist between
    the logical and the aesthetic perfection of cognition,
    not merely in the universal but from several particular sides,
    we want to compare the two with one another in respect to
    the four chief moments of quantity, quality, relation, and modality,
    on which the passing of judgment as to the perfection of cognition depends.

    A cognition is perfect
    (1.) as to quantity if it is universal;
    (2.) as to quality if it is distinct;
    (3.) as to relation if it is true; and finally
    (4.) as to modality if it is certain.

    Considered from the viewpoints indicated,
    a cognition will thus be logically perfect

    as to quantity if it has objective universality
        (universality of the concept or of the rule),

    as to quality if it has objective distinctness
        (distinctness in the concept),

    as to relation if it has objective truth, and finally

    as to modality if it has objective certainty.

    To these logical perfections correspond now the following aesthetic
    perfections in relation to those four principal moments, namely

    aesthetic universality.

        This consists in the applicability of a cognition to
        a multitude of objects that serve as examples,
        to which application of it can be made,
        and whereby it becomes useful at the same time
        for the end of popularity;

    aesthetic distinctness.

        This is distinctness in intuition,
        in which a concept thought abstractly
        is exhibited or elucidated
        in concrete through examples;

    aesthetic truth.

        A merely subjective truth,
        which consists only in
        the agreement of cognition with the subject
        and the laws of sensory illusion,
        and which is consequently nothing more
        than a universal semblance.

    aesthetic certainty.

        This rests on what is necessary
        in consequence of the testimony of the senses;
        what is confirmed through sensation and experience.

    With the perfections just mentioned
    two things are always to be found,
    which in their harmonious union
    make up perfection in general,
    namely, manifoldness and unity.
    Unity in the concept lies with the understanding,
    unity of intuition with the senses.

    Mere manifoldness without unity cannot satisfy us.
    And thus truth is the principal perfection among them all,
    because it is the ground of unity through
    the relation of our cognition to the object.
    Even in the case of aesthetic perfection,
    truth always remains the conditio sine qua non,
    the foremost negative condition,
    apart from which something cannot please taste universally.
    Hence no one may hope to make progress in the belles lettres
    if he has not made logical perfection the ground of his cognition.
    It is in the greatest possible unification of
    logical with aesthetic perfection in general,
    in respect to those cognitions that are both
    to instruct and to entertain,
    that the character and the art of
    the genius actually shows itself.

SK 37.

yasmat buddhi purusasya upabho-gam sarva pratisadhyati
sa eva ca suksma pradhana-purusantara visinasti

Because it is the Buddhi that accomplishes the experiences
with regard to all objects to the Purusha.
It is that again that discriminates the subtle difference
between the Pradhana and the Purusha.

YS III.21

    kaya-rupa-samyama tad-grahya-sakti-stambhe
    caksu-prakasa-asamprayoga antardhana

YS III.22

    etena sabda-adi-antardhana ukta

    c. The absolute idea
