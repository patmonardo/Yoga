Twenty-second Lecture
Monday, May 28, 1804
Honored Guests:

Whether one names the absolute “being” or “light,”
it has already been completely familiar for several weeks.
Since attaining this familiarity, we are working on
deriving not, as is obvious, the thing itself,
but rather its appearance.

The request for this derivation can mean nothing else than
that something still undiscovered remains in the absolute itself,
through which it coheres with its appearance.

We know from the foregoing
that the principle of appearance is
a principle of disjunction
within the aforementioned undivided oneness
and at the same time, obviously, within appearance.

If disjunction were to be found
directly within absolute oneness,
as is unavoidably required by
the final form of the science of knowing,
and is what we intend here,
it must not be grasped as a simple disjunction,
but rather as the disjunction of
two different disjunctive foundations.
[It must be] not just a division,
but rather the self-intersecting division
of a presupposed division
that again presupposes itself.
Or, using the expression with which we have designated it
in our most recent mention of it,
[it is] no simple “from,”
but rather a “from” in a “from,”
or a “from” of a “from.”

This much as a general introduction for the week:
Now back to the point at which we stood at the end of the last hour.
Absolute self-genesis posited and given a principle,
obviously within knowing,
which is thus a “principle-providing” [occurrence].
Thus within this knowing there follows
the absolute, positive negation of genesis =
completed and enduring being;
and indeed, because this entire investigation
concerns light's pure immanence,
our investigation has long ago brought in
a presumed being external to knowing,
knowing's [own] completed and enduring being.

Now We saw into this connection during the last hour
and see into it again here;
as is evident, we see one of the two terms
in and through the insight into this connection,
determined as such.
Thus, this latter is itself mediated,
and just we (= the insight we have now achieved) are
therefore the unconditionally immediate [term].

Two remarks about this.
1. I have just recalled again,
that here the inner being and persevering is knowing's being,
the very thing already recognized as absolute genesis
and [the thing] we have also already validated in the last hour
as a priori rational knowledge of an absolute principle.
At this point, we must hold on to
the fact that it is knowing's being,
even in case we should let go of
the addition as a shortcut in speaking;
because otherwise we will fall back into
where we were before, far removed from further progress.
Therefore, our entire chain of reasoning
must always be present to us, now more than ever.

2.
Knowing's derived being will now yield
ordinary, non-transcendental, knowing.
Through our present insight into
the genesis of the former's principle
(that is, the principle of the recently derived knowing),
and through reflecting on this insight
we elevate ourselves to genuine transcendental knowing
or the science of knowing;
and [we have done so] not merely factically,
with our factical selves,
so that we are the factical root.
We have already been this since
the time when we dissolved into pure light;
instead [we have done so] objectively and intelligibly,
so that we, achieving insight factically,
at the same time penetrate the law of this insight.
Henceforth we have to work in the higher region
that has now been opened.
Only here will the principle of
appearance and disjunction
that we seek show itself to us;
which then should only be applied to
existing (= ordinary) actual knowing.
Now also this addition:
since the beginning of what we
have provisionally called the second half,
a hypothetical “should” has been evident
as simply creating a connection,
and as linking a conditioning and conditioned term
(which are both produced absolutely from it)
to a knowing, which must be originally present
independently of the “should”
and its entire operation,
if one just understands it correctly.
This could be called the first
sub-section of the second part.
Since we concerned ourselves with
an absolute presupposition about
the essence of knowing as an absolute “from,”
we wished to know nothing more about
this entire hypothetical “should”
and its power of uniting and joining,
as a merely apparent knowing.
We said at this point that so far we
(the we that still has not been grasped so far)
have concurred about the premise's arbitrary positing,
pointed out by energetic reflection,
and only the connection has made itself manifest
without our assistance.
Now the premise too presents itself without our assistance;
therefore in the premise too we coincide
with the absolutely self-active light.
Let's hold on to this.
We have been doing this for a long time
in our discussions about the “from,”
until I thought you were sufficiently prepared for
the higher flight that we began in the last hour.
You may take this as the second sub-section of the second part.
In the last hour, the bare connection presented itself again,
and (as we might suspect, but will more exactly show and demonstrate)
with it the hypothetical “should,”
from which we had already hoped to be free.
This should surprise us.
If this “should” has reappeared with the same significance
in which it was previously struck down,
then we have not advanced
and are just sailing on
in the seas of speculation without a compass.
Through the hint given recently as to
the difference between ordinary knowing
(based on the principle of knowing's being)
and transcendental [knowing]
(based in the genetic insight into this same principle),
it is probable that [the “should”] does
not arise in the same way.
Instead, the “should” that we have dropped
operates in ordinary knowing
with tacitly assumed premises;
in contrast, the “should” arising now
operates in transcendental knowing,
which grounds its premises genetically
as emanating from a “should.”
Therefore, in the preceding lecture,
we have begun a third sub-section of the second part,
and the two extreme sections come together
in the middle (with the premise) once again
distinguished by a duality in the premise.
By this means, the two outermost parts
(= transcendental and actually existent knowing)
would be the two different distinguishing grounds,
proceeding from the middle ground of the premise,
which both unites and separates them.
This is the compass that I would share with you
for the journey we have already begun.

2. I said that a hypothetical “should” appears
again in our completed insight.
To begin with, this is obvious.
“Given [that there is] a principle
for self-genesis, then it follows that ____ .”
Previously, we have found the two terms factically,
and to that extent separately;
but in the last hour, we have united them genetically,
according to our basic rules and maxims.
Because we comprehend them mediately
in the insight into their connection,
then, given this insight,
we no longer need to assume them factically.
They reside a priori in the insight,
and we can drop the empirical construction,
until perhaps it arises again in some deduction.

3. Now let us grasp this hypothetical character at its core.
According to the maxims and rules that we arbitrarily adopted
at the beginning of our entire science, and hence arbitrarily,
we appear to ourselves
(so it has been, and so it now is openly admitted)
as genetically uniting both terms.
Here is the inner root of hypotheticalness
(now fully abstracted from the hypothetical
character of the subordinate terms),
precisely the “should's” admitted
inner production, containment, and support of itself
as identical with the free We, the science of knowing.
This very inward hypotheticalness could be
what first shows itself and breaks through
in the subordinate terms' hypothetical character.
Hence, it is only a matter of
negating this inner hypotheticalness,
so that thereby the quality of being categorical
will be manifest in it,
and thereby we will justify our insight
in its truth, necessity, and absolute priority.
In that case, the inference,
made here only provisionally,
first achieves categorical validity;
namely, the inference that both terms
(knowing's self-genesis and its being)
occur only mediately in a genetic insight
into the oneness of both,
and in no wise immediately.

(A remark belonging to method:
One must not allow oneself to be distracted
[as to] how I legitimately assume this.
This is more necessary than ever,
since here the method itself becomes absolutely creative;
further, nothing besides a remark like this
can be adduced in explanation of what happens here.
Our insight has come about through
the application of the basic maxims of our science,
to apply the principle of genesis
thoroughly and without exception.
This has been required,
and it should prove itself.
Likewise, the maxims of the science of knowing itself,
and with it all of science, have been required,
and they should prove themselves.
Science itself should justify
and prove itself before it truly begins.
Thereby, the science of knowing would be liberated
from freedom, arbitrariness, and accident;
as it must be, or else one could never come to it.)

4. Without digression, we conduct the required proof,
according to a law that has already been applied, thus.
We could produce this insight,
and we actually did so; we are knowing;
thus this insight is possible in knowing
and is actual in our current knowing.
Just a few remarks about this proof.

a. The genesis first accomplished by us is
an absolute, self-enclosed, genesis;
by no means is it a genesis of a genesis,
because it negates itself inwardly within knowing,
as we have shown in the last hour.
To us, however, who are contemplating further
and constructing the process in its laws,
it manifested as genesis.
In immediate knowing, however,
it was merely a persisting intuition,
as external in its result:
which explicitly was non-genesis = being.

b. The proof of absolute genesis was conducted purely
through its possibility and facticity,
and thus [is] itself only immediately factical.
In this case, therefore, facticity and genesis entirely coincide.
Knowing's immediate facticity is absolute genesis;
and the absolute genesis is (exists as a mere fact)
without any possible further ground.
To be sure, it must happen so,
if we are ever actually to arrive at the ground.

c. This is a cogent example of how much
in the science of knowing depends on one
always having the whole context present,
since the distinctions can be drawn
provisionally only through this context.
Knowing, as genesis, is proved factically in this way.
What then happened several sessions ago when
we proved knowing as a “from” factically in the same way?
Is the “from” something other than genesis,
and have we not conducted the identical proof?
Yet the present, factically demonstrated, genesis is
something entirely different from the one proved earlier.
You could grasp this distinction only by
noticing that in this case it is a question of
the genesis of absolute knowing in its fundamental construction,
whereas previously [it was a question of] the
genesis of its absolute self-genesis.
In that lecture, to be sure, I had to let this
criterion go and abstract from it,
looking simply at the core element of the new proof
(since otherwise we would never arrive at this new proof)
and relying on the solid insight already engendered in you.
However, you can add this criterion now,
and can use it to rebuild and reinforce the insight,
in case it begins to waver.
Later of course I will add inner distinguishing criteria,
e.g., for both these points of origin,
by which they can be distinguished in themselves,
independent of their relation.
However, they are not even possible, or comprehensible,
before the distinction in the thread of relatedness
has been completed factically,
because these inner distinctions are nothing but
the genetic law of the factical difference,
which arises only in the fact.
For just this reason, the science of knowing is
not a lesson to be learned by heart,
but rather an art.
Presenting it, too, is not without art.

d. I also want to make you aware of the following.
Even in the recently conducted proof,
within whose content facticity and genesis should merge
purely into each other,
there still remains in the minor premise of the syllogism
the same term that arose above factically
and still has not been genetically mastered.
“We know, or are knowing.”
[This is] of course immediately clear and intelligible,
but its principle is by no means clear.
Investigations remain to be made here,
and herein lies perhaps the most important
part of our remaining solution.

Let me announce the process that will follow.
For good reasons lying in my art,
I will not proceed at once with this point,
but instead I will wait until it arises of itself.
On the contrary, I will add this also:
in the insight we have completed,
an insight has now indeed arisen for us
that is objective, immediately compelling for us,
as well as intrinsically determinate and clear.
It is this, if we abstract from the subordinate terms
as hypothetical and as themselves just the externalization of
the inner hypotheticalness of our performance
(of course, we need to abstract from this hypotheticalness as well),
then we intend to turn from the form of our insight
to its content, in order to explain the form from it,
as we have frequently done and as is actually a realistic move.

To repeat briefly:
the content of the insight we have recently achieved
must be clearly present to you.
Providing absolute self-genesis with a principle
yields absolute non-genesis = being.
As we just recently undertook to do,
today we have abstracted completely
from the two subordinate terms.
And, looking only at our insight itself
and the manner of its production, we have justified,
as was possible only in a factical manner,
the absolute application of both
the maxim of self-genesis and this procedure.
This is the essential content of the little, easily remembered,
bit that we have achieved for our topic.

Twenty-third Lecture
Wednesday, May 30, 1804
Honored Guests:

Providing a principle for absolute self-genesis,
as we have described the light, creates non-genesis,
or being (of knowing, it goes without saying).
We have seen this, have reflected about
the method for producing the insight,
and have justified it factically.
It is possible and actual within knowing,
because it is possible and actual within us,
and we are knowing.
However, no genetic derivation of
this last point has occurred just yet.

We can justify this procedure even more deeply
and from another point of view.
An objective, and absolutely compelling insight
has actually arisen for us;
this process hereby has also shown itself
to be coherent with the absolute, self-producing light.
Consequently, the task which we reported
at the end of the last hour as coming next
(to investigate this objective insight
in respect to its true content)
will simultaneously be our first attempt at justifying more deeply
what has up to now been presented only factically,
and perhaps even to make the point genetic.
And so to the content of the self-presenting objective insight!

1. Evidently, the absolute relation of both subordinate terms.
However, these are still hypothetical;
but without them, there is no connection;
it itself is the result of hypothetical terms,
and so itself is hypothetical;
a fact from which we should abstract.
What is still left?
Plainly nothing other than the insight's inner certainty;
and because even the insight as such depends on the terms,
[it is] nothing more than a purely inner certainty.

The first claim on you is to grasp
this certainty sharply and altogether purely.
It is not certainty about anything in particular,
as the relation of the subordinate terms was in our case,
because we have abstracted from that.
Rather, it is certainty pure and as such,
with complete abstraction from everything.

At first, accordingly, it is immediately clear
that certainty would have to be thought completely purely.
Moreover, the ground of its material whatness lies in the “what”
(that it is, that it is what it is, that it is certain);
but the ground for certainty cannot in any way be located there,
because this does not belong to the "what."
Therefore, certainty lies simply in itself,
unconditionally and purely as such;
and it is unconditionally of and through itself.
It must be thought in this way,
or else certainty is not conceived as certainty.

In passing, being is not a reality that is
derived from the sum of all possible realities
(from the possible determinations of a “what”);
rather, it is completely closed into itself,
and outwardly, in its properties, is
first the condition and support of every “what.”

2. How would you proceed, if I asked you
to describe this certainty more closely, to make it clear?
Not otherwise, I believe, than by conceiving it
as an unshakable continuance and resting
in the same unchangeable oneness;
in the same, I said, and therefore
in the very same “what,” or quality.
Accordingly, you could not describe pure certainty otherwise
than as pure unchangeability;
and [you could not describe] unchangeability otherwise
than as the persisting oneness of the “what,” or of quality.

3. I inquire further, is the main thing for you
in describing pure, bare certainty
that the “what” be something particular,
or does your description not rather explicitly contain
absolute indifference toward all more exact determination of the “what”?
Only the latter, you will say, and without doubt
you will admit that the “what” remains one;
but by no means what else it may be.
Therefore, in this case merely the pure form of the “what,”
or quality in general, is employed in the description,
and it is the required description of pure certainty
only on condition of this formal purity.

4. With this, the concept of the “what,” or quality,
is completely explained and derived for the first time.
This is a concept that has so far always remained in the dark,
as much in its content as in its factical genesis.
Quality is the absolute negation of
changeability and multiplicity, purely as such;
the concept is thereby closed without
any possibility of further supplements.
I add here as a supplement that through this negation,
the negated term (changeability) is posited at once,
purely as such, and without further determination.
(Quantifiability through quality, and vice versa.)
“Genetically derived,” I have said.
As such, certainty cannot be described otherwise
than through absolute quality.
“If it should be described, then ____ must” and so on.
Therefore, we have taken a very important step
for our primordial derivation of the “what” = appearance.
Now everything depends on how
the description (reconstruction) of certainty arises.

5. We argue thus:
In this way we have seen into and described certainty.
However, is our description of certainty itself then
certain, true, and legitimate?
As we have always done with similar questions,
let us pay attention to our manner of proceeding.
We have constructed a general “what”
and posited it as unchangeable.
The essence of certainty appeared therein.
I ask, if we repeat this procedure infinitely many times,
as we seem able to do, could we ever do it any differently?
The “what's” construction is entirely unchangeable,
and in all these infinitely many repetitions,
it is possible in only the one way we have described:
through negating changeability.
Therefore, we view ourselves in the very same way
we have described certainty, as persisting unchangeably
in the construction's single same “what”;
we are what we say, and we say what we are.

6. Certainty is grounded completely and absolutely in itself.
However, according to its description,
certainty is persistence in the same “what.”
Thus, in the description, the ground of the “what's” oneness is
to be posited completely inwardly within certainty itself.
The oneness of the “what” lies in this,
that certainty is,
and in no other external ground.

7. “Certainty is grounded in itself” also means
that it is absolute, immanent, self-enclosed,
and can never go outside itself;
in itself it is I;
in just the very same way that
the proof of being's form was previously conducted.
Therefore, it is clear that the externalized and objectivized
certainty we presented previously is not
the absolute one, according to its form,
although in its content and essence it may very well be.
Hence, it is clear that in searching for the absolute
we must abstract from the latter,
and search simply in what manifests
itself as immanence, as I (or We).

In this We, we have now surely found certainty,
as the necessity of resting
in the procedure's qualitative oneness;
and there the matter now rests
(this enumeration demands all our attention).
First and foremost, absolute certainty is,
in and from itself,
the same as “I” or “We,”
us or its own self
(all of which mean the same thing),
completely inaccessible,
purely self-enclosed,
and hidden.
For if it
(or “We,” which means the same thing)
were accessible,
then it would have to be outside itself,
which is contradictory.
As a result of the preceding insight,
we must abstract from just this:
that we are actually speaking about it now,
and thus are manifesting it;
and this is the appearance
(which is mere appearance,
because it contradicts the truth)
whose possibility we must deduce
from the system of appearance.

In a way that arises from a ground
that will show itself very soon but
that is not yet clear,
certainty now expresses itself immanently in itself
(that is, in us)
and thus in every expression,
as perception of a particular,
completely unchanging process.
This manifestation shows up here,
but only as an absolute fact,
and so an obscurity still remains.
(Process is living as living;
the process's unchanging qualitative oneness is
life's immanence and self-groundedness,
expressed immediately in life itself.)
Let us press ahead toward clarity,
to the extent that we can do so here.
Immediately living and immanent being-a-principle is light,
and is intuition with an inner necessity.
I say, “being-a-principle,”
therefore just projecting and intuiting.
I say the projected term is “immediately living and immanent”
simply in intuiting, from intuiting, and out of intuiting;
and the projecting is just light's life as “principle-providing”.
I say “with an inner necessity,”
and thus that this necessity must
completely express itself,
because it is just “principle-providing”
as being principle-providing.

It is absolute, immanent projection,
and so it is a projection of nothing else than itself,
wholly and completely as it inwardly is.
Note, as it is, it is projecting,
first of itself inwardly and qualitatively,
but not at all understood objectively.
Thus, way of inner, living “being-a-principle,”
as thinking and intuiting unconditionally at a single stroke;
but, in fact and truthfully, it is the latter as a result of the former.
Thus in this inner qualitative self-projecting,
it necessarily projects itself, objectively
(et in virtute eius, minime per actum specialem);
not yet to be sure as an objectively present I,
but rather as it inwardly is:
first of all as living, one, and grounded in itself formally;
but this is process in pure qualitative oneness.
This is the intuition of inner certainty
and oneness of process that concerned us previously.
This oneness expresses itself with necessity,
because it is the result of
the absolute, living principle-providing.
We called this process “describing certainty”
and sought a ground for it.
It has been found.
I ask to wit, does such a description of certainty
happen in itself and actually?
But how could it; it is nothing else than
the necessary expression and result of certainty's life
as pure “providing-itself-a-principle”
completely derived and explained by us.
Yet this life is unconditionally necessary in being or certainty.
Further, it projects itself as it inwardly is;
but it is not merely living, instead it is its own life,
and, as such, it is self-projection.
The life derived in this way as a constructive process is
therefore the construction of itself in the projection
(and therefore likewise in the certainty taken objectively),
which we found as a first term at the beginning of our investigation,
when we were unfamiliar with the higher terms.
In the living description, certainty is
more primordial in us than it is
objectively in itself without any description.
It is the latter only as
a result of the construction,
which is also projective.

Now let us more clearly and definitely grasp
the three primary modifications of the primordial light,
which we have disclosed today.

Certainty, or light, is an immediately living principle,
and thus the pure absolute oneness of just the light,
which cannot be described further in any way,
but rather can only be carried out.
If we wished to describe it, then we would have to
describe it as a qualitative oneness,
which in this case does not serve us.
It is always and eternally immediate I.
Therefore, since we said the foregoing,
we already contradicted ourselves in what we say.
When we said, “it is a living principle”
we began to describe it, but [to do so] primordially.
“Principle-providing” is already its effect,
but its primordial effect in us, since we are it.
It (or we, which is the same thing) describes itself in this way.
Principle-providing, if only you think it precisely, is
projecting, immanent self-projecting.
To be sure, since it lies immediately and directly in living itself,
it [consists in] making oneself into projection and intuition,
not through a gap and objectively,
but inwardly and essentially through transubstantiation.
Notice, because this is situated in light's very life,
all light whatsoever is immediately self-creating,
and so it is like this:
thus, it is absolutely intuiting.
Even the science of knowing, in all its living activity,
cannot avoid this fate.
We have not avoided it either, despite the fact that,
according to a law that we have not yet explained,
it invades the principle and becomes a self-making in it,
because otherwise [the principle] remains just a being.
This entire insight into
the primordially real principle-providing is
a matter for the science of knowing;
which is the first factor.

Living knowing intuits itself
unconditionally as it is inwardly
just because it really projects itself.
However, above all it is unconditionally from itself;
it must therefore intuit itself as being thus,
and here specifically as not existing outside intuition.
Here, therefore, the absolute gap arises,
and the projection through a gap,
as a pure, rational expression
of the true relationship of things:
the notion of intuition,
or the concept,
in its separation from essence,
not as the essence itself,
but rather merely as its image,
and the negation of the latter beforehand.

Hence, it is a “principle-providing,”
and it must intuit itself objectively and through a gap.
At this point it is clear that this “principle-providing,”
its process, must appear within the intrinsically immanent view as
from itself, out of itself, and through itself;
as by no means grounded in it, but rather as happening
because it projects through a gap.
However, the scientist of knowing,
who understands the view itself in its arising,
knows very well all the same that this entire independence is
not intrinsically true as self-producing,
but instead is only the appearance of a higher,
absolutely unintuitable, principle-providing.
Therefore [he understands] that all the flexibility
lying in the procedure's appearance is not grounded in truth,
so that it can only be grasped with difficulty,
just as the qualitative oneness of intuition
by no means comes to an end with it.

So then, it is an absolutely immanent
providing-itself-a-principle and indeed,
as will now be explained more exactly,
in absolute fact,
without any other intervening light,
or seeing: as intuition.
This very intuition must itself be intuited,
or projected through a gap:
through which arises the intuition of
a primordially complete and persisting knowing,
what we previously called the being of knowing.
The first described principle-providing in intuition
relates itself to this “being of knowing” as reconstruction.
So, from certainty we have derived both subordinate terms
(which previously just stood there hypothetically)
out of a deeper insight into the essence of their relation.

We have not sought to conceal
where the difficulty now still remains.
That is, [we need] to ground the possibility,
and justify the truth and validity
of the science of knowing's presupposition that
living certainty is genuinely “principle-providing."
I say carefully “principle-providing,”
not “providing-itself-a-principle."
If we prove the first, then the second follows
from absolute immanence and self-enclosure,
as is completely obvious from itself.
More about this tomorrow.

Twenty-fourth Lecture
Thursday, May 31, 1804
Honored Guests:

We have posited the primordial light
as “principle-providing”
living immediately in itself,
and from this, we have derived
three necessarily arising fundamental
determinations of the light.
At first, it was enough for us
to understand this proposition
and the inferences following from it,
which demanded energetic thinking and
inwardly living imagination above all.
However, I believe that I have succeeded in being comprehensible.
Finished with this business, we raised the question,
what authorized us, as scientists of knowing,
to make this assumption?
We held ourselves back from answering this, until today.
To begin with, consider the sense
in which we then asked the question.
We know that the science of knowing is I,
that light is completely I.
Further, someone could attempt to conduct a proof here
in the same way we did it previously;
the science of knowing as I, and therefore light, can and does;
hence the light can and does.
Nevertheless, this mode of proof must fall away
and receive its higher premises.
When this occurs, the place is revealed where,
at the very same time,
the I that can act and the light that can act fall away,
along with all arbitrariness,
the appearance of which our presupposition
certainly still carries.

So much generally.
Now I request that you undertake
the following reflection with me:

1. If arbitrariness is to vanish,
an immediate, factical necessity of
immediate self-projection must show up
in the science of knowing.
“Immediate necessity,” I say;
the science of knowing must really do this,
or better the necessity must take place itself,
without the science's help.
I say “necessity” and “factical”:
it must be immediately intuited as necessary,
but without additional higher grounds.
The remark, which we make from time to time and made yesterday,
that we cannot escape from knowing's
projecting and objectifying,
does not suffice for our purpose.
It has not been shown under what conditions and
in which context we cannot escape.
Thus, if we could not escape only in attentive thinking,
we nevertheless seem to escape in weak thinking.
We should explain just how it stands
with these two contradictory appearances.

However, as I will show you briefly
and without further derivation,
the following suffices and leads to our goal.
I cannot predicate anything of the light,
unless I just project and objectify it
[the light] as the subject of the predicate.
“Non nisi formaliter obiecti sunt predicata.”
If one merely ponders this proposition attentively,
it is immediately clear, without any further possible grounds;
and if it is to belong here, then it must be just so,
according to our previous remark.
It is more important for us to understand its contents properly.
“I predicate of the light”
(or, what is synonymous, the light predicates of itself)
means that it projects itself through a gap,
by means of the enduring intuition
sufficiently characterized yesterday.
“I cannot do so without projecting it” generally means:
without projecting it in
the primordial, real, inner, and essential projection,
which first makes it intuition,
as again was sufficiently described yesterday.

Let us look back at our task.
We put forward the very claim about which
we admitted that it was not made in any factical knowing whatsoever
(which is always limited to already completed intuitions)
but rather is made only by the science of knowing:
light's absolutely primordial act of
making itself an intuition.
This claim should be demonstrated
in an immediately self-producing insight and manifestness.
This is the case with the insight produced by us.
Therefore, absolute necessity, etc., is comprehended.

But how is it comprehended?
Not unconditionally, but under conditions.
Our insight brings it into connection with something else.
If [something] should be predicated
(that is, be intuition), then ____ must.
However, should the antecedent be?

In that case neither is the consequent:
the first can be seen into only under a condition
—thus it is conditioned intellectually;
if the consequent happens to occur—
that [would be] real conditionedness.
So everything is just about what we wished;
but not the unconditionedness and oneness for which we strove.

(
a. A logical clarification:
Predicate = minor premise;
absolute objectification of the logical subject = major premise.
Both posit themselves unconditionally reciprocally;
and so something else underlies the inference
much more deeply than the major premise,
with discovery of which we began.
Ignoring this, all philosophical systems without exception
fail to arrive at an absolute major premise.
Therefore, if they do not wish their thinking
to stand still somewhere arbitrarily,
they must sink into a rootless skepticism.

b. With the repeated appearance of the “should,”
of hypotheticalness, and of relatedness,
no one now fears that we would be driven back again
to the old point that we have already abandoned.
Because obviously the currently related terms are
higher than were the previous ones.
Previously, [we understood] self-construction to be
the procedure that arose yesterday
in the already established intuition,
and, [we] took the being of knowing
to be the highest, the established intuition.
Now, though, the established intuition itself is not
displaced by a higher one in relation to
pure and real self-projection,
rather it is simply discovered in it.
)

2. Since I, a scientist of knowing,
see into this connection
as absolutely necessary and unchangeable,
I myself project and objectify knowing
as just this relation,
as a “oneness through itself,”
determined without any possible assistance
from some external factor;
[a oneness] which I simultaneously permeate
and construct in its inner essence and content.
What then is the content?
Above all, something arbitrary
and dependent simply on freedom and the fact,
therefore something unconditionally necessary,
which, facticity, if it happens to be called into life,
grasps and determines without further ado.
Both in relation to one another,
so that the one is indeed the completely
proper principle of its being,
but cannot be this without
in the same undivided stroke
having its principle in the other.
Likewise, the other is not actually a principle
unless the one posits itself.

We can best name the former “law,”,
a principle that, in order to
provide a principle factically,
presupposes yet another
absolutely self-producing principle.
[We can name] the latter a pure, primordial “fact,”
which is only possible according to a law.

3. This is what knowing is, absolutely and unalterably,
without any exception, and it is understood as such.
Now, in the insight just produced and completed,
I myself, the scientist of knowing, am a knowing.
Indeed, as it seems to me,
I am a free and factical knowing,
since I could well have omitted the preceding reflection,
and moreover [I am a knowing] predicated of knowing,
describing its entire essence.
According to my own statement,
facticity always and everywhere takes up
the law of primordial projection,
therefore in actual fact it must have
taken me up, albeit invisibly.
Formal projection in general rests in the law,
and this is evident enough factically;
I add here only that it exists as a result of the law.
The fact also rests in the law
that it [facticity] is projected just as it inwardly is,
or, (as we could say more accurately
in order to be safe from all misunderstanding)
nota bene, as it must be projected according to the law.
However, at the heights of our speculation,
we know no other law at all except
the law of lawfulness itself,
that it is projected according to law.
This is just how it expressed itself to us earlier in factical terms;
therefore, we have nothing else to do regarding this material point
than to add that this projection occurs according to the absolute law.
In that way, by applying its own material assertion to its own form,
we have genetically derived the very insight from which we began today,
and which previously was produced factically.
This is the first important result.
a. No doubt, we will get to say more about such an application of
what a proposition asserts to the proposition itself.
It is remarkable.
It seems to be just the determination of
the major premise by the minor premise
that we mentioned previously;
and therefore truth seems to begin running back into itself,
which in all other systems would have neither beginning nor end,
if they were consistent.
b. On every occasion, we have posited as
the absolute of its kind that which posits itself;
and so, in the investigation just completed,
we have [posited] a law of law, or lawfulness itself.
That this is the absolute law cannot be doubted;
nor that the act we carry out in accord with it is
the highest, immediately law-conformable
act of the science of knowing.
(What the stated limitations mean
will become evident at the appropriate time.)

4. Now we proceed to another investigation
that is of the highest importance, interesting
(if I can succeed in making myself understandable to you),
and even agreeable.

Without doubt the absolute law,
according to which we projected knowing
in its essence in the insight
that we completed today and then analyzed,
has absolute real causality on the inwardness of the act
(I do not speak of its outward form, which appears free).
So that the law and the act permeate one another inwardly
(and indeed the act with the oneness of
all its distinguishable determinations)
without any gap between them.
The projection is in part formal (objectifying),
and in part material (expressing the essence of knowing).
It is by no means the latter without the former,
but instead is both at a single stroke,
because it is both through an absolutely effective law.
Hence, the material expression must
simultaneously express the projection's form.
Or, to say it exactly:
disregarding the prior proof of the opposite,
knowing in projection, at least formally,
cannot simply be just what it is in itself or
according to the law, without any projection.
What could it be that projection alters in it?
Of course, you do not forget that we are speaking here only about
projection's inner material form,
abstracting from its outer form,
on the basis of which we merely argued and which we now drop.
The inner essence of projection is living principle-providing;
this [latter] must remain in [the former]
completely and absolutely as such, and can never be destroyed.
What is this?
Answer: only absolute description, as description,
which stepped between the two terms in a wonderful way.
Among other things, [it is] quite evident in the relation as such.
For what is the relation except
describing one on the basis of the other?
The latter must remain immanently in the former
and can never be destroyed,
precisely as intrinsically living “principle-providing.”
Thus, it must allow itself to be
perpetually renewed as such, although the content,
determined by the absolute law, remains the same.
On this basis, one can now explain the appearance of
energetic reflection, and the infinite repetition of
the content that qualitatively remains absolutely one,
content which, to our great astonishment,
has not yet wished to leave us.
I said that the law should determine the inner content.
While we examine it, if we now think of description as
an absolutely living “principle-providing” in itself,
then it is clear at once that it must appear just through the law
as the re-construction of an original pre-construction,
and so, in a word, as an image.
Or, [it must appear] as the oft-mentioned mere statement,
only the enunciation and expression,
of that which to be sure should in itself be just thus.
In a word, [description is] the whole simply ideal element,
as which we must consider our seeing to be,
if we transfer ourselves into the standpoint of reflection,
of “principle-providing.”

Therefore, to apply this at once,
this is how it stands with knowing.
The entire form of objectivity,
or the form of existence,
has in itself no relation to truth.
Knowing itself, however,
and everything which should arise in it,
splits itself absolutely into a duality,
whose one term is to be the primordial,
and whose other term is to be
the reconstruction of the primordial,
completely without any diversity of content,
and so again absolutely one;
differing only inthe given form,
which obviously indicates
a reciprocal relation to one another.
(It is really like this in every possible consciousness,
if you wish to test the proposition there.
Object, representation)

However, let us carry this reasoning further.
At the beginning of our investigation,
because the looked-for absolute oneness created us,
we stood (as we later discovered) under the law
without knowing either it or our act as such.
Only when we reflected on this act were
we able to apply the content of our discovered insight
to it as the middle term of our inference,
and arrive at an insight into the previously concealed law.
Without doubt, we constructed or described
the law itself in this insight,
and could catch ourselves in the act.

Therefore, we truly stand under the law
just in case no law arises in knowing;
and we are beyond the law,
constructing it itself,
if it does arise in consciousness.
Thus, our entire inference grounds itself on a mere fact,
without law, which therefore cannot be justified;
and the inference itself only speaks of a law
without being or having one.
Hence, this reasoning too, however much luster it shows,
unravels into nothing.

Applying this to the preceding:
the ostensible primordial construction,
which is supposed to justify the reconstruction
(that admittedly presents itself as such),
is itself really just a reconstruction,
but one that does not present itself as such.
However, the entire appearance disappears on closer inspection.

We must be glad that it now drops away as well,
and that this standpoint was not the highest.
Because yet another disjunction lies in it,
whose genetic principle is not yet fundamentally clear,
and at which we cannot remain.
Of course, it is not the outward
disjunction into subject and object,
which fell away by means of the full annulment
of the persistent form of projection and objectivity;
rather [it is] the inner living difference between both:
two forms of life.

Just how this difficulty is resolved
and where our path will go on from there
can be gathered very easily.
We described and constructed the absolute law;
that is the difficulty.
It must be evident that we cannot construct it;
rather it constructs itself on us and in us.
In short, it is the law itself,
which posits us, and itself in us.
On this subject, tomorrow.

Twenty-fifth Lecture
Friday, June 1, 1804
Honored Guests:

We understood that, as a condition,
if knowing is to predicate something of itself,
then it must in general simply project itself.
Looking simply at the form of this insight,
it contains a free, arbitrary fact and an absolute law,
of which this fact, becoming actual,
should immediately avail itself.
This is the first point.

Now in this “understanding”, we, as scientists of knowing,
are also knowing, indeed a free factical knowing,
and so in the same circumstance of which we have spoken.
Therefore, we ourselves, along with this fact of ours,
come under the law to project knowing in general,
and to project it as it is inwardly, or according to law.
Thus, knowing has disclosed itself in our insight
as objective and unchangeable oneness,
and with the absolute manifestness that
it behaves in just the way we have said.
We add only the new point that this too
exists in this way because of the invisible law.

This point was argued further as follows.
This projection occurs
(at least according to the material,
or to the content of knowing stated in it)
according to an absolute law,
which cannot not be a law
and cannot not have causality.
Hence, it is an absolutely immanent projection
and can never escape being so;
and, as we very illuminatingly add,
the light in the projection can not be just
the same as it is inwardly or
(according to the law)
apart from all projection.
What does this mean?
It must permanently bear the mark of
the intrinsic living principle-providing;
and [it must] appear in its form as
the product of that sort of “principle-providing.”
Therefore, [it must appear] repeatedly to infinity
as a reconstruction related to
the primordial projection through the law.
This establishes the fundamental disjunction in knowing.
At this point, we raised this objection:
is not the law itself reconstructed by us?
Obviously; therefore, we never really arrive
at a primordial construction and law,
instead, viewing the matter aright,
[we] merely have two reconstructions,
one of which presents itself as what it is,
and one of which lies.
However, [we] can uncover this illusion.
Thus, we still find ourselves in
the realm of arbitrariness,
not yet having entered the necessary.
Now to solve this.

1. In the first place, why is it, really,
that we will not trust this reconstruction of the law?
Because it seems arbitrary.
If it showed itself as necessary,
then it would show itself also as
something conforming to law
and as itself the immediate
inner expression and causality of the law.
We received an immediately factical law,
pervading the facts: posit the law.
Now, merely because it is projected,
the projected law can always be
the result of a reconstruction;
but at least the inner construction of
this objective law is not a reconstruction,
but rather the primordial construction itself.

2. Can this proof of a law's reconstruction be carried out?
I say: as it seems to me, it can be done easily in the following way.
The first, primordially permanent projection
intrinsically bears the character of an image,
of a reconstruction, etc.
However, the image as such refers to a content;
and reconstruction as such refers to an original.
Hence, the task of understanding this concept of intuition
completely and lawfully contains another [task],
to posit this prior one.
Now I ask, how then is the image an image,
the reconstruction a reconstruction?
Because they presuppose a higher law,
and exist because of this law, we have said;
and therefore [they] point in the image, as an image,
to the law, which is already contained there,
at least virtually and in its results.
We, scientists of knowing, stand presently
just in the image as an image;
thus, it is the law implicitly and virtually
present in us ourselves that constructs, or posits, itself ideally.
So, what we undertook to prove yesterday is completely proven,
“The law itself posits itself in us ourselves.”
Image as image is the crucial premise {nervus probandi}.

Notice also:

1. In order to conduct the proof we have just completed,
we first had really and intrinsically to presuppose
the law as the image's primordial ground,
without giving any account of how we arrive
at the concept or its projection.
Now we have completely explained how we arrived at it;
but the variation of this concept's form
has not yet been explained.
I content myself here with merely pointing out
this as yet obscure section historically,
and adding that its clarification lies
in the reply to the question of
the science of knowing's possibility
as a science of knowing,
which will now be addressed continually,
but achieved only at the end.

2. We have now changed the following
in yesterday's sketch of knowing.
Knowing stands neither in
the reconstruction as such (representation),
nor in the original (the thing-in-itself),
but rather wholly in a standpoint between the two.
It stands in the image of the reconstruction, as an image,
in which image the positing of a law arises
immediately through an inner law.
This permeation of the image's essence is
the primordial, absolute, unchangeable oneness.
As wholly internal, in projection,
it just divides itself projectively
into a permanent objective image
and a permanent objective law.

I wish to be completely understood on this.
The light lives what it is
in its very self,
it inhabits its living.
Now it is image:
as image, I have added,
that is living, self-enclosed imaging.
You have to attend very closely to the latter;
because otherwise, it is not yet clear
how you could have come to the former,
to the image as a closed oneness
that you have undoubtedly objectified.
It is an imaging, formally immanent;
it images, or projects itself as
just what it inwardly is, as an image.
It is intelligibly immanent and self-enclosed.
Yet, the image posits a law,
therefore it projects a law;
and it projects both as
standing completely in
the one-sided, determinate relation
in which we have thought them.

Note further here: according to yesterday's disjunction,
both primordial image and copy should be qualitatively one,
because otherwise they could not cohere.
Now, both cohere inwardly and essentially
as image, positing a law, etc.,
and qualitative oneness can not come
into it in any way.
Qualitative oneness is the absolute negation of variation;
therefore, it can come into play only
where variability can be posited.
However, image, as image, is intrinsically invariant.
It is essential oneness,
the law of the image is likewise oneness,
and they posit one another entirely
only through their inner essence,
without any further supplement.
This total removal of the material, qualitative oneness,
which so far has not left us,
is a new guarantee that we have climbed higher.

Since we end the week today, I will not begin any new investigation;
instead, I will estimate what we have left to do in the coming week,
during which I intend to complete this entire presentation,
should that prove possible.

Initially, it is readily clear that,
if we only establish ourselves correctly in
what has just been presented,
no possibility can be foreseen as to
how we should escape it and go further.
Here we are immediately absolute knowing.
This is an “image-making process”
positing itself as an image,
and positing a law of the image-making process
as an explanation of the image.
With this everything has been unfolded, and
is completely explained and comprehensible in itself.
The terms come together to form a synthetic cycle,
into which nothing else can enter.

The particular point at which we have closed off
further progress is very clearly evident above.
[We did so] through the total negation of
the concept of qualitative oneness.
By negating this concept,
quantifiability was established
at the same time that we opened
a more comfortable way to descend
into life and its multiplicity,
as we know from the preceding.
Now, this qualitative oneness has been
negated through positing a oneness of image
and its law that repels all variability,
a oneness essential in itself
but still merely hypothetical.
Since this oneness has become manifest absolutely,
it must be applicable here,
so that we cannot possibly prevent ourselves
blending this quality in, without further ado.
As to this, I wished particularly
that you had noticed for yourselves,
that we had to descend again to this quality
only through a term of the necessary relation,
one which is produced lawfully.
(The occult quality is entirely cut off.)

Noteworthy too is the fact that
the concept of the science of knowing
as a particular knowing has now disappeared completely.
What we have derived is the one pure knowing
in its absolute disjunction,
which is explained from its essence as oneness.
We now are this one pure knowing;
we are now the science of knowing as well
and [we] hope to become it again,
so the science of knowing is absolute knowing itself,
and we are the latter too only to the extent that
the science of knowing is it.

The last consideration leads us to
the path on which we can go further;
the science of knowing must emerge again
as a particular knowing.
Of course, we know very well that
we have not always been the one insight
that we now are and live,
but that we have ascended to it
by means of all our previous considerations.
We must retrace this,
our transformation into absolute knowing
not empirically and artificially, as we have before;
instead, we must explain it.
In short, we must once more see in its genesis
what we are at the conclusion of today's lecture.
This insight into the genesis,
not of absolute knowing in itself,
because this knows no origin;
but rather of absolute knowing's
actual existence and appearance in us;
this would be the science of knowing in specie,
in so far as it is a particular knowing whose
nonexistence is as possible as its existence.

Now, it may happen that
ordinary knowing is
the primordial condition for
the genetic possibility of
absolute knowing's existence,
or of the science of knowing.
Hence, [it may happen] that
its determinations can be explained
simply from the presupposition that
the science of knowing ought to arise,
and the sum of our entire system resolves itself
into the following rational inference.
“If absolute knowing is to appear, then ____ must, etc.”
Now, knowing is determined like this,
so consequently this should clearly happen.

I have said that all determinations must be
explicable and understandable simply on
the basis of the presupposition:
“It should unconditionally____ , etc.”;
and I ask you to take this in its full strength.
By way of explanation, I add the following:
we have seen that knowing in itself is
unconditionally one, without any material quality or quantity.
How then does this knowing descend in itself
to qualitative multiplicity and difference,
and to the entire infinity of quantity and its forms
(time, space, etc.) in which we encounter it?
We have to prove the following:
simply because absolute knowing's being
can be produced only genetically,
and because it can be so only under
just the types of conditions
that we find originally in living,
therefore life coheres indivisibly
with the science of knowing,
and with that which it produces.
Everyone must confess that,
apart from the extent to which
he elevates himself to absolute knowing,
his entire life would be nothing,
would lack worth and meaning,
and would truly not even exist.

In brief:
absolute affirmation of the genesis of
the existence of absolute knowing
(according to the preceding,
no term in this description is superfluous)
unites both ends of knowing,
the ordinary and also the absolute and transcendental,
and clarifies them reciprocally.
We must put ourselves into this point,
as the genuine standpoint of the science of knowing in specie.
With this, I believe that I have shared with you
a very important conception of our entire procedure
up to now and into the future.

Hence, the entire outcome of our doctrine is this.
Simple existence, whatever name it may have,
from the very lowest up to the highest
(the existence of absolute knowing),
does not have its ground in itself,
but instead in an absolute purpose.
And this purpose is that absolute knowing should be.
Everything is posited and determined through this purpose;
and it achieves and exhibits its true destination
only in the attainment of this purpose.
Value exists only in knowing, indeed in absolute knowing;
all else is without value.
I have deliberately said “in absolute knowing,”
and by no means “in the science of knowing in specie,”
because the latter is only a means,
and has only instrumental value,
by no means intrinsic value.
Whoever has arrived no longer worries about the ladder.

This result, that only true knowing or wisdom has value,
is very shocking in our age, which counts only on external workings,
and undoubtedly it appears to us as a great innovation.
It is remarkable that this doctrine,
which is an innovation to our age,
is really the primordial one, as is almost always the
case with all our characteristic works and ways.
I will prove this, not to support by age and authority
what can prove itself, but rather to give you
a parenthetical opportunity for comparison.
In Christianity, (which may in its essence
be much older than we assume,
and concerning which I have frequently said that,
in its roots and especially in its charter [the Gospels],
which I hold to be its purest expression,
it [Christianity] completely agrees with realized philosophy)
the final purpose, especially in the record of it,
which I hold as the purest,
is that people come to eternal life,
to having this life and its joy and blessedness
in themselves and out of themselves.

In what then does eternal life consist?
“This is eternal life,” it says,
“that they know [recognize] you.
and [him] whom you have sent”
(for us, this means the primordial law
and its eternal image);
merely know [recognize].
Yet, indeed, this recognition not
only leads to life, instead it is life.
Thus, from that time on,
through all the centuries and
in all forms of Christianity,
and consistently with this principle,
faith in the teaching that
true knowledge of the super-sensible is
the main and essential thing,
has been insisted upon.
Only in the last half a century,
after the almost total decline of
true scholarship and deep thinking,
have people changed Christianity into
a doctrine and ethics of prudence.

This doctrine has not forgotten to teach
that in genuine, truly living knowledge
right conduct arises on its own,
and that, even if he wishes to do so,
the one in whom the light has inwardly dawned
cannot fail to shine outwardly.
Our philosophy forgets it just as little.
Yet, there is a great difference
in doing right from such different sources.
Doing right from self-interested cleverness,
or from self-regard arising as a result of
a categorical imperative, yields cold, dead fruit,
lacking blessing for both agent and the recipient.
Now, as ever, the former hates the law,
and would much prefer that it not exist.
Therefore, happiness with himself and his act never arises.
The latter cannot animate and bring to life
that which fundamentally has no life.
Only when right action arises from clear insight
does it occur with love and pleasure.
Only then does the act, self-sufficient
and requiring nothing else, reward itself.
