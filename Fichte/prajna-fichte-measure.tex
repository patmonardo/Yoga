Twelfth Lecture
Monday, May 7, 1804
Honored Guests:

In the last discussion period,
it was obvious from those who attended
and allowed themselves to converse about things
not only that you have followed me well
even in the most recent deep investigations,
but that, as is ever so much more important,
a comprehensive vision of the inner spirit
and outer method of the science
we here pursue has grown in you.
Consequently, I assume that this is
even truer with the rest of you who did not speak;
and I [will] abstract from everything
that does not arise for me on this path,
with no misgivings about carrying the investigation
forward in the strength and depth with which we have begun.

A brief review in four parts:

1. production of an insight,
which may have many genetic aspects in its content,
but which at its root can only be factical,
since otherwise we would not have been able to go higher.
If there really is to be a “through,”
then (as a condition of its possibility)
we must presuppose an inner life,
independent in itself from the “through,”
and resting on itself.

2. We then made this insight,
produced within ourselves,
into an object, in order to analyze it
and consider it in its form.
There we proposed initially (in the whole of the second part)
that we saw into our concept of an actual “through,”
which appears freely created;
or rather, since everything depends on this,
[we proposed] that this concept was energetic and living,
that the inward life of this concept was
the principle of the energetic insight into a life beyond,
which grasped us, and which was intuited
in this insight as self-sufficient.
Thus, it was the principle of intuition
and of life in the intuition.
This latter need not arise except in intuition,
and its characterization as life in and for itself
is not intrinsically valid.
Instead, it can be fully explained
from the mere form of intuition
as projecting something self-sufficient
in the form of external existence.
In case another perspective is also possible,
and since it begins with the energy of reflection
and makes it a principle,
this way of looking at the insight can conveniently
be labelled the idealistic perspective,
in the terminology already provisionally adopted and explained.

3. But this other view, posited as the basis for the insight,
also proved to be possible, and was realized as well.
The presupposed life in itself should be
entirely and unconditionally in-itself;
and it is intuited as such.

Therefore, all being and life originates with it,
and apart from it there can be nothing.
The reported subjective condition of
this perspective and insight was this:
that one not stubbornly hold on to
the principle of idealism,
the energy of reflection,
but rather yield patiently
to this opposite insight.
The realistic perspective.

With this a warning!
not as if I detected traces of this misunderstanding
in someone speaking about the matter,
but rather because falling into this error is very easy,
as nearly all the philosophical public has done
in regard to the published science of knowing.
Do not think of “idealism” and “realism”
here as artificial philosophical systems
which the science of knowing wants to oppose:
having arrived in the circle of science,
we have nothing more to do with the criticism of systems.
Instead, it is the natural idealism and realism
that arise without any conscious effort
on our part in common knowing,
at least in its derived expressions and appearances:
and notwithstanding [the fact] that
both can certainly be understood in this depth
and so on the basis of their principles,
they nevertheless arise only in philosophy
and especially in the science of knowing.
It is still the latter's intention
to derive them as wholly natural disjunctions
and partialities of common knowing,
arising from themselves.

4. Both these perspectives were more closely specified
in their inner nature and character.
Thus, just as at the start,
we elevated ourselves above both
(we are not enclosed within them,
since we can move from one to the other),
[and we moved] from their facticity to the genesis of both,
out of their relative and mutual principles.
Hence, the insight which in this fourth part
we lived and were, was their genesis,
just as they were the genesis
for the previously created [terms],
in which both came together.
Thus, according to the basic law of our science,
we are constantly rising to a higher genesis
until we finally lose ourselves completely in it.

We characterized them in this way:
through its mere being,
the idealistic mode of thinking locates itself
in the standpoint of reflection,
makes this standpoint absolute by itself,
and its further development is nothing more than
the genesis of that which it already was
without any genesis other than its own absolute origin.
In its root, therefore, it is factical,
not just in relation to something outside itself
(as is, e.g., Kant's highest principle)
but rather in relation to itself.
It just posits itself unconditionally,
and everything else follows of itself;
and it frees itself from any further
accounting for its absolute positing.
The realistic mode of thinking proceeds no differently.
Abstracting entirely from [the facticity of] its thinking,
it presupposes the bare content of its thought
as solely valid and unconditionally true,
and completely consistently it denies any other
truth which is not contained in this content,
or, as would actually be the case here,
which contradicts it.
But this residing in the content is
itself an absolutely given fact,
which makes itself absolute
without wanting to give any further account of itself,
just like idealism.
Therefore, both are at their root factical;
and even ignoring that they are presented one-sidedly,
each annulling the other,
in this facticity each bears in itself
the mark of its insufficiency as
a highest principle for the science of knowing.

Let me describe it again with this formula:
at this highest point of contradiction
between the two terms absolutely demanding unification,
we find these: 0 and C,
or the form and the content,
or the forms of outer and inner existence,
or [as] in the previous lectures, essence and existence.
We appear to have obtained the absolute disjunction;
its unification promises to bring with it absolute oneness
and so to resolve our task fundamentally.

Today we will present considerations
regarding this solution
which still remain preliminary
(preliminary because we must advance
even further just to get to the point)
in order to prepare ourselves soundly
for the highest oneness.

First, it must be clear that the problem
cannot be resolved simply by
combining, rearranging, etc.,
what we know so far.
In relation to our next aim,
everything up to now is only
preparation and strengthening our spirit
for the highest insight;
and if the preceding should have
some further significance beyond this,
this significance can arise only
by deduction from the highest principle.
Now we must bring up something entirely new,
according to the insight adduced just above,
it is certain that something has remained
to some extent empirical and concrete for us.
We must investigate this and master it genetically.
The rule, therefore, is to investigate this facticity.
We have demonstrated the factical principles
of the perspectives in which we recently spent ourselves,
one after the other
(and which therefore undoubtedly contain
the highest [form of being] that we,
the scientists of knowing,
have ourselves so far attained).
One of the two must be developed.
Which shall it be?

If one grasps hold of it first,
the principle of idealism is
admitted to be absolutely incontrovertible.
Realism attacks this immediately
as idealistic stubbornness and a false maxim,
which it repudiates.
Thus, [realism] denies the principle
and cannot reason with idealism in any way.
On the other hand, idealism in turn makes
even the beginnings of realism impossible;
it ignores realism completely,
and hence can not have anything against it,
since, for idealism, [realism] does not exist.
Now realism obviously takes itself to be superior
just because of its denial of idealism's principle
and by its origin from this denial;
thus in realism at least
a negative relationship to idealism remains,
whereas even the possibility of such relatedness
is extirpated in idealism.
We must therefore attend to realism,
temporarily abstracting in every way from idealism;
since, as said before, we cannot let
realism be absolutely valid,
but rather want to correct it,
and since we cannot combat it
from the perspective of idealism,
we must fight against it on its own grounds:
catch it in self-contradiction.
Through this very contradiction,
which indeed brings a disjunction with it,
realism's empirical principle would become genetic,
and in this genesis, perhaps it will become
the principle of a higher realism and idealism [united] into one.
We have solved the first task,
finding which factical principle to develop.
This is where we grasp realism in its strength.
Its crucial point was life's in-itself and within-itself.
For now, we hold to this feature alone,
and in the mean time we could let go of life.
From this “in-itself ” realism infers
the negation of everything outside it.

But how does it bring about this very in-itself?
Let's reconstruct the process,
thinking the in-itself energetically.
I assert, and I invite you to consider this yourselves,
and see it immediately as true, I assert:
the in-itself has meaning only to the extent
that it [is] not what has been constructed,
and completely denies everything constructed,
all construction, and all constructability.
Consider well: when you say,
“thus it is in itself, unconditionally in itself,”
then you are saying [that] it exists thus
entirely independently from my asserting and thinking,
and from all asserting, thinking, and intuiting,
and from whatever other things outside
the in-itself may have a name.
This, you say, is how the in-itself must explain itself,
if it wishes to explain itself.
No other explanation yields the in-itself.
Result: the in-itself is to be described purely
as what negates thinking.
[This is] the first surprising observation:
here for the first time realism
[the perspective which, during the last lecture,
we made evident only factically
on the basis of its consequences, ]
is understood genetically.
Previously, that is, this insight arose and grasped us,
that if this life in itself is posited,
nothing apart from it can exist.
That is how it was;
we saw into it this way
and could not do otherwise.
Now we see that realism,
or we ourselves standing in its perspective,
act like the in itself,
which negates everything outside itself.
[We see] that realism therefore to some extent
(at least in its effects) is itself the in-itself,
and collapses into it;
and for this implicit reason,
in the appearance of our insight,
which grasped us in the previous lecture,
it annuls everything outside itself.
Therefore, we have comprehended genetically
something about realism which previously was only factical.

With this out of the way on one side,
let's reflect more closely about our own insight,
evoked before, and its principle.
I call on you to think the in-itself
and its meaning exactly and energetically,
whereupon you would then see into, etc.
You admit that without this exact thinking
you would not have seen it;
perhaps you even admit that for your whole life
you have thought the in-itself, faintly to be sure,
and yet this insight has not been produced in you.
(It can be shown that things have gone this way
for all philosophy without exception:
since if this insight opened itself really vividly for anyone,
then the discovery of the science of knowing
would not have taken so long!)
Thus, your insight into the negation of thinking in itself
presupposes positive thought,
and the proposition is as follows:
“In thought, thinking annuls itself
in the face of the in-itself.”

Now, to add even more consequences,
with which I only wish to make you acquainted:
the negation of thinking over against the in-itself
is not thought in free reflection,
as the in-itself ought to be thought by us,
rather it is immediately evident.
This is what we called intuition, and without doubt,
since the absolute in itself is found here,
this is the absolute intuition.
What the absolute intuition projects
would therefore be negation, absolute pure nothing,
obviously in opposition to the absolute in itself.
And thus idealism, which posits an absolute intuition of life, is
refuted at its root by a still deeper founding of realism.
It may well come up again as an appearance;
but, taken as absolute in the way it gave itself
out to be before, it is merely illusion.
Hence, we do not get past the previously mentioned
fundamental negation of ourselves over against the absolute.

The negation is intuited;
the in-itself is thought.
I ask how and in what way it is thought;
and I explain this implicitly obscure question
by the answer itself.
That is, we constructed this in-itself,
assembling it from parts, just as, for example,
at the beginning of our enterprise we constructed
oneness in the background as not being
empirically manifest identity,
and not multiplicity either,
but rather as the union of these two.
I ought not believe, instead we posit it directly,
together with its meaning,
in pure simplicity as the genuine construction:
thinking's negation is directly evident to us,
it grasps us as proceeding from it in its simplicity.
We therefore (this is very important)
didn't actually construct it,
instead it constructed itself
by means of itself.

Intuition, the absolute springing forth of light and insight,
was bound directly together with this construction.
We, however, would certainly not have produced this,
since it obviously produces itself and draws us forward with it.
Thus, the absolute's absolute self-construction
and the original light are completely and entirely one and inseparable,
and light arises from this self-construction
just as this self-construction comes from absolute light.
Hence, nothing at all remains here of a pregiven us:
and this is the higher realistic perspective.
But now we still hold on to the requirement, as with right we can,
that we should be able to think the in-itself,
and think it energetically,
thus that the living self-construction of
the in-itself within the light must have yielded to us,
and that once again this energy must be
the first condition of everything,
which results in an idealism,
which lies even higher.

But on this subject there are once again two things to consider:
first, we are also aware of this thinking, of this energy,
and apparently our claim that it exists
(without which general existence it could surely not be a principle)
is grounded solely on this awareness.
This, however, presupposes the light.
But since the light itself, at least in this its objective form,
does not exist in itself apart from the absolute
(as indeed it cannot, since nothing exists apart from the absolute)
but rather has its source in the in-itself,
we cannot appeal to it:
something which itself bears witness against us,
if we examine it more closely.

If one retains this higher presupposition of
the light in all possible deliverances of self-consciousness
(as the source of all idealistic assertions),
then the constant spirit of idealism in its highest form
and the fundamental error which contradicts and destroys it
fundamentally would be that it remains fixed on facticity,
at the objectifying light from which one can never
begin factically, but only intellectually.

So then (which is surely the same point from another angle),
one must reflect in opposition
to the idealistic objection:
You are not thinking the in-itself,
constructing it originally,
you are not thinking it out;
indeed how could you!
Nor is it known to you through something else,
which is not itself the in-itself,
rather it is merely known by you:
thus your knowing in and of itself sets it down;
or, as the matter may be more accurately put,
it sets itself down in your knowing
and as your knowing.
You have been doing this your whole life
without your will
and without the least effort
and in various forms,
as often as you expressed the judgment:
so and so exists.
And indeed philosophy has made war with you
and bound you in its circle,
not because of this procedure itself
but because of the thoughtlessness of it.
You will not give any credit to
your freedom and energy regarding the event itself.
Only now that you are aware of
this action and its significance
does your energy add anything;
likewise again with the declaration
of that which gives itself to you without any effort:
intuition.
Therefore, before we listen to you at all,
we must inquire more exactly
how far the testimony of intuition is valid.
This, additionally, by way of conclusion.
Whether it appears in its oneness
as the concept of some philosophical system,
either killed or never having lived,
as it was for us before
our realization of its significance,
or in a particular determination
as the “is” of a particular thing,
the faded in-itself is always [an object of] intuition
and is therefore dead.
For us it exists in the concept,
and is therefore living.
Hence for us there is nothing in intuition,
since everything is in the concept.
This is the most decisive distinguishing ground
between the science of knowing
and every other possible standpoint for knowing.
It grasps the in-itself conceptually:
every other mode of thought does not
conceive it, but only intuits it,
and in that way kills it to some extent.
The science of knowing grasps
each of these modes of thought
from its own perspective
as negations of the in-itself.
Not as absolute negations,
but as privative ones.
The things which, as we ascend,
we find to be not absolutely valid
(for instance, one-sided realism and idealism),
or which might be found to be so,
the science of knowing will take up again
on its descent as similar possible negations
of the absolute insight.

Thirteenth Lecture
Wednesday, May 9, 1804
Honored Guests:

Again today and even beyond I will climb on freely.
I say “freely” for you, because
I cannot provide the foundation for the distinctions
that will emerge here before using them,
rather I must first acquaint you with them in use,
even though a firm rule of ascent may well stand
at the basis of what I am doing.
If one just grasps my lecture precisely in other respects,
there is no danger of confusion despite the former circumstance,
because instead of the initially given crucial points
L (life) and C (concept)
we have the two perspectives:
realism (= genesis of life) and
idealism (= genesis of the concept).
This should be known from Friday's clear presentation
and its review the day before yesterday.
(Not generally, [but rather] in relation to the point,
as it was grasped there, one had to adhere to life in itself,
which was required to animate a “through.”
We will not move very far away from this now,
and it will easily be possible to reproduce from it
everything that needs to be presented,
or to trace the latter back to it.)
In a word, these two perspectives are our present guide,
until we find their principle of oneness
and can dispense with them directly.
Here, if anywhere, we need the capacity
to hold tight to what is presented firmly and immovably,
and to separate from everything that may
very well be rationally bound to it.
Otherwise, one will leap ahead, anticipate the inquiry,
and will not grasp the genetic process
linking what is taken up first and its higher terms,
which is what really matters;
instead the two will flow into one another factically.
“One will anticipate,” I said;
but it is not actually “one,”
not the self to whom this happens,
rather it is speculative reason,
running along automatically.
(Then let me add this remark in passing:
speculation, once aroused and brought into play,
as I partly know it has truly been brought into play in you,
is as active and vital as the empirical association of ideas ever can be,
because it is surrounded by a freer, lighter atmosphere:
and once one has entered this world,
one must be just as watchful against
leaps of speculation as previously against
the stubbornness of empiricism.
I want especially to warn those for whom the objects
of the present investigation appear very easily about this danger;
I advise you to make them a little harder for yourself,
since this appearance of ease may well arouse the suspicion
that the subject can be grasped more easily by speculative fantasy
than by pure, ever serene, reason.)

To work.

The in-itself reveals itself immediately
as entirely independent from knowledge or thought of itself,
and therefore as wholly denying the latter
in its own essential effects, in case one assigns such to it.
We did not construct the in-itself
in this immediately true and clear concept,
instead, as was immediately evident to us,
it constructed itself just as it was
in the constructive process, as denying thinking;
immediate insight, the absolute light, was
immediately united with this concept
and was evident in the same way.
Thus, the absolute in-itself revealed itself
as the source of the light;
hence the light was revealed
as in no way primordial:
which is now the first point
and which obviously bears in itself
the stamp of a higher realism.

Another idealism now tried to raise itself
against this realism, proceeding from this basis:
since we saw into the in-itself as negating vision,
we must ourselves have reflected energetically on it.
Thus, although we cannot deny that it constructed itself,
and the light as well, all of this was nevertheless
qualified by own vigorous reflection,
which therefore was the highest term of all.
As basing itself on absolute reflection,
this is obviously idealism;
and, since it does not depend,
as did the previous idealism,
on reflection about something conditioned
(actually carrying out a “through”)
as a means to realize the condition
[but instead rests on reflection
on the unconditioned in-itself]
it is a higher idealism.

We quickly struck down this idealism
with the following observation.
(“You then,” we address it in personified form,
“You think the in-itself: that is your principle.
But on what basis do you know this?
You cannot answer otherwise,
or bring up a different answer than this:
'I just see it, am immediately aware of it,'
and to be sure you do see it,
unconditionally objective and intuiting.”
(The last point is important and I will analyze it more closely.
In realism too we simply had an insight
into the self-construction of the in-itself;
but we had an insight, we saw into something living in itself,
and this living thing swept insight along with itself,
the very same relationship which we have already frequently found
in every manifestness presented genetically.
Now, however, ignoring [the fact] that
a purely objectifying intuition seems
to hover above the origin,
still this intuition is drawn at once toward,
and along with, the genesis.
In this insight, therefore,
a unification of the forms
of outer and inner existence,
of facticity and genetic development
seems to be hinted at.
Things stand completely differently with the seeing of
its thought to which idealism appeals.
That is, in that case we will certainly
not wish to report that we witness thinking as thinking
(as producing the in-itself)
in the act of producing, 
as we certainly actually and in fact witness
the in-itself producing its construction;
rather intuition accommodates itself immediately only to
a thinking which is essentially opaque
and can be presented only factically.
Thus, it remains completely ambiguous
whether thinking originates from this intuition,
the intuition comes from the thinking,
or whether both might be only appearances
of a deeper hidden oneness which grounds them.

In case it is necessary to make this clearer.
Could you ever think really clearly and energetically,
which is what we are discussing here
(because faded thoughts and dreaming are
completely to be ignored),
without being aware of it;
and conversely could you possibly be
aware of such thinking without assuming
that you really and in fact were thinking?
Would the least doubt remain for you about the truth
of this testimony of your consciousness?
I think not.
It is therefore admittedly clear,
and immediately proved by the facts
that you cannot distinguish genuine thinking
from consciousness of it, and vice versa;
and that in this facticity, thinking posits its intuition,
and the intuition posits the absolute
truth and validity of its testimony.
We do not quarrel with you about that.
But you cannot provide the genetic middle term
for these two disjunctive terms.
Hence, you remain stuck in a facticity.
But, on the other hand, the genesis
which has arisen in the opposite,
realistic perspective opposes you;
that is to say, in that perspective we know nothing about
one term of your synthesis, your so called thinking.
But we recognize that to which you appeal
for verification of the latter.
Although [we do not do so] immediately,
we still do in its principle.
I say “not immediately”;
just as little do we recognize
there a simple consciousness,
expressing a fact absolutely,
as yours does according to our closer analysis.
But I also said “in its principle”:
in any event your consciousness presupposes light
and is only one of its determinations.
But light has been realized as itself originating from the in-itself
and its absolute self-construction;
however, if it originates from the in-itself,
then this latter cannot likewise originate from it, as you wish.
In your report that you are thinking actually and in fact
because you are conscious of yourself doing so,
you must posit your consciousness as absolute,
but the very source of this consciousness,
pure light, is not looked at factically,
which would bring us to the same level you occupy;
rather (which is more significant) it is
seen into genetically as itself not absolute.
And so this new idealism has been in part
determined further;
it does not, as first appears, even posit
as absolute a reflection which, according to it,
belongs simply to thinking, instead,
it posits the immediate intuition of this reflection as absolute
and is therefore different in kind from the first:
in part it is refuted as in truth valid,
although as an appearance it is not yet derived.
Assuming that it has become entirely clear to you,
hold onto this and let go of what is deeper.

By the way: in passing and while doing something else,
I have touched here on the very important distinction
between a merely factical regarding,
like our thinking of the in-itself,
and genetic insight,
like that into the in-itself's self-construction.
By means of the immediate testimony of our consciousness,
we cannot witness our thinking, as thinking,
literally as production;
we see it only so long as it exists,
or should exist, and it already is, or should be, while we see it;
on the other hand, we see the in-itself as
existing and as self-constructing,
simultaneously and reciprocally.
This point will have to come up again naturally
as the higher point of disjunction
for a still higher oneness,
and it will be very significant.
Meanwhile, let it be impressed [on you],
and, as an explanation,
let the following historical commentary be added
which may have whatever worth it can for admirers.
At the same time it may serve as an external test
of whether one has understood me.

Reinhold (or, as Reinhold claims, Bardili) wishes
to make thinking qua thinking the principle of being.
Therefore, on the most charitable interpretation,
his system would be situated within
the idealism we have just described,
and one must assume that by this he means
the thinking of the in itself
which we carried out the day before yesterday.
Now, first of all, he is very far removed from
explaining that this in-itself negates seeing,
as we have shown;
but then, which is worse,
in regard to the in-itself's real existence
(with which he generally has no dealings
and which, to be sure, he would be able to prove
only factically from the existence of individual things)
he does not appeal to consciousness
(a fact of which I reminded him, but which escaped him)
because he sees very well that doing so
would lead him to an idealism,
and he seems to have developed
an unconquerable horror of any idealism.

Hence, in the first place, his principle stands entirely in the air,
and he works to build a realism on absolutely nothing;
and he could only be driven to this by despair, following the rule:
“since it wouldn't work with anything I tried so far,
then it must work with the one thing still left in my field of vision.
”Second (and I have made this observation especially to make this point),
since, according to an absolute law of reason,
thinking does not let itself be seen into
as producing itself as thinking,
then naturally Reinhold cannot realize it this way either,
nor can he genetically deduce the least thing from it.
Hence he can only say, like Spinoza:
since everything that is lies in it,
and since now so-and-so exists,
it too must lie in it.
Because he was educated in the Kantian school,
and later by the science of knowing,
he may not now do this.
Hence he labors to deduce;
but since, if one only has a clear concept,
this appears to be completely impossible,
total darkness and obscurity arise in his system,
so that nobody grasps what he really wants.
If one considers this system from
the point of view of the science of knowing,
and indeed from the very point from
which we have just seen it,
then the obscurity of its principles is clear.)
Let's get back to business and draw the conclusion,
because in passing we have once again gained
a very clear insight into the true essence
of the science of knowing, that is,
of the principle which we are still obliged to present.

The refuted idealism makes immediate consciousness
into the absolute, the primordial source, the protector of truth;
and indeed absolute consciousness reveals itself in it
as the oneness of all other possible consciousness,
as reflection's self-consciousness.
In the first place, then, this stands firm
as one of our basic foundations.
Wherever we say “I am conscious of that,”
our testimony bears the basic formal character
of an absolute intuition,
which we have just described,
and makes a claim for the intrinsic validity of its contents.
This consciousness is now realized as self-consciousness
and reflection in its root:
all possible disjunctions and modes of consciousness
must be deduced from self-consciousness;
and we would therewith already
have achieved a comprehensive study.

It is clear that this consciousness is
completely one in itself
and capable of no inner disjunction;
because the thinking that arose in it was
that of the in-itself
which, as in-itself, is
entirely one and self-same;
thus it too was one,
and the consciousness of it was
only this one consciousness;
and therefore it was also one.
The self, or I, which arises here is
consequently the absolute I
[pure eternally self-identical, and unchanging]
but not the absolute as will soon be more precisely evident.
A specific disjunctive principle will
also have to be identified if:

a. in the course of thinking “the one in-itself ”
it should appear in a multi-faceted perspective,
even though in the background it is to remain
perpetually the same single in-itself, or categorical is, and
b. as a result there also arises a manifold view of thinking,
or reflection, and hence of the reflecting subject, or I
(all of which, just like the in-itself, are also
to remain the same one in the background).
It could well be, and indeed will turn out, that,
if we remain trapped in the absolute I
and never raise ourselves beyond it,
we may never discover this disjunctive principle
in proper genetic fashion,
and it will have to be disclosed factically.

(An historical note in passing:
even where things go the best for the science of knowing,
it has been taken for this idealistic system we have just described,
which presupposes what we have exactly characterized as
the absolute I to be the absolute
and derives everything else from it;
and no author known to me, friend or foe,
has risen to a higher conception of it.
That most have remained at an even lower level
than this conception goes without saying.
If a higher understanding is to arise
for anyone besides the originator of this science,
it could only be among the present listeners,
but it would not be written;
because what can be understood in writing
stands under the previous rule;
or it will be hit upon by you.
This remark has the consequence that nobody will get
a report about the essence of the science of knowing
from anyone other than the present one from the originator.
It will be immediately evident even from just
the clear and decisive letters of
what has been published on the topic
how false this interpretation is.)

The intrinsic validity of this idealism is refuted;
nevertheless, it may preserve its existence as an appearance,
and indeed as the foundation of all appearance,
which we have to expect:
it has been refuted on the grounds that it is factical
and that a higher development points to its origin.
One calls a fact “factical,”
and since here we are speaking of consciousness,
this fact would be a “fact of consciousness”;
or, to put it more strongly:
according to this idealistic system,
consciousness itself is a fact,
and since consciousness is for it the absolute,
the absolute would be a fact.
Now, from the first moment of its arising
the science of knowing has declared that the
primary error of all previous systems has been
that they began with something factical
and posited the absolute in this.
It, on the other hand, lays as its ground
and has given evidence for, an enactment,
which in these lectures I have called by the Greek term genesis,
since these are often more easily understood rightly than German terms.
Therefore from its first arising,
the science of knowing has gone beyond the idealism we have described.
It has shown this in another equally unambiguous way:
particularly regarding its basic point, the I.
It has never admitted that this I
[as found and perceived]
is its principle.
“As found” it is never the pure I,
but rather the individual personal being of each one,
and whoever claims to have found it as pure finds himself
in a psychological illusion of the kind
with which we have been charged by those
ignorant of this science's true principle.
So then, the science of knowing has always testified
that it recognizes the I as pure only as produced,
and that, as a science, it never places the I
at the pinnacle of its deductions,
because the productive process will always stand
higher than what is produced.
This production of the I,
and with it the whole of consciousness,
is now our task.

The idealism which has been rejected as intrinsically valid is
the same as absolute immediate consciousness.
Therefore, so that we now express
as forcefully as possible what it comes down to,
the science of knowing denies the validity of
immediate consciousness's testimony absolutely
as such and for this exact reason:
that it is this, and it proves this denial.
Solely in this way does the science of knowing
bring reason in itself to peace and oneness.
Only pure reason, which is to be grasped merely by intellect,
remains as solely valid.
So that no one is confused even for a moment
by a fancy which might easily arise here,
I immediately add a hint which must be discussed
further in what follows.

Namely, someone may say:
“But how can I grasp something in intellect without
being conscious in this intellectualizing?”
I answer: “Of course you cannot.”
But the ground of truth as truth does not rest in consciousness,
but only and entirely in truth itself.
You must always separate consciousness from truth,
as in no way making a difference to it.
Consciousness remains only an outer appearance of truth,
from which you can never escape
and whose grounds are to be given to you.
But if you believe that the grounds why truth is truth
are found in this consciousness, you lapse into illusion;
and every time something seems true to you
because you are conscious of it,
you become at the root idle illusion and error.
Here now it is obvious:

1. how in fact the science of knowing keeps its promise
and, as a doctrine of truth and reason,
expunges all facticity from itself.
The primordial fact and source of
everything factical is consciousness.
It can verify nothing, in light of the science of knowing's proof
that whenever truth is at issue
one should turn it aside and abstract from it.
To the extent that this science
in its second part is a phenomenology,
a doctrine of appearance and illusion,
which is possible only out of the first
and on the ground and basis of the latter,
to that extent it surely deduces both as existing,
but simply as they indeed exist, as factical.

2. It has become completely clear why nothing external
could be brought up against the science of knowing,
but that one always must begin with it to gain entrance into it.
The beginning point for a fight against it is
either grasped in intellect or not.
If it is grasped in intellect,
then it is either grasped intellectually immediately
[and that is the principle of the science]
or it is grasped mediately
and these must be either deductions of
the fundamental phenomenon 
or phenomena derived from it.

One can come to the latter only through the former.
In this case, therefore, one would in every circumstance
be at one with the science of knowing,
be the science of knowing itself,
and in no case be in conflict with it.
If it were not grasped intellectually
but should nevertheless be true,
then one must appeal for verification
to one's immediate consciousness,
since there is no third way to get to the absolute itself
or even to an appearance of it.
But making this appeal one is straightaway turned aside
with the instruction that precisely because you are
immediately aware of it and appeal to that fact,
it must be false.
Of course thoughtlessness and drivel have
created a fancy title for themselves,
that of “skepticism,” and they believe that nothing is
so high that it cannot be forced under this rubric.
It must stay away from the science of knowing.
In pure reason doubt can no longer arise;
the former bears and holds itself
[and everyone who enters this region]
firmly and undisturbedly.
But if skepticism wishes to doubt
the implicit validity of consciousness
[and this is approximately what it
wants in some of its representatives]
and it does this provisionally in this or that corner,
although without having been able to bring about
a properly basic general doubt;
if it wants this, then with its general doubt
it has arrived too late for the science of knowing,
since the latter does not just doubt
this implicit validity in a provisional way,
but rather asserts and proves the invalidity of
what the general doubt only puts in question.
Just the possessor of this science
(who surveys all disjunctions in consciousness,
disjunctions which, if one assumes the validity
of consciousness in itself, become contradictions)
could present a skepticism which totally negate
everything assumed so far;
a skepticism to which those who have been playing
with all kinds of skeptical doubts as a pastime
might blanch and cry out: “Now the joke goes too far!”
Perhaps in this way one might even contribute
to arousing the presently stagnating philosophical interests.

Fourteenth Lecture
Thursday, May 10, 1804
Honored Guests:

[Yesterday, an idealism, which made absolute consciousness
(in its actuality that is) into its principle was
presented, characterized, and refuted;
“in its actuality” I say,
since today we will uncover still
a different [idealism] in a place
where we do not expect it,
one which makes the same thing its principle,
only merely in its possibility.
Now this absolute consciousness was self-consciousness
in the energy (of reflective thinking, as it later turned out).
On this point I will add another remark
relating to the outer history of the science of knowing,
which naturally is not intended to parade my conflict
with this unphilosophical age in front of you,
but rather only to provide hints to those
who are following this science in my published writings
and who want to rediscover it in the form employed there,
telling them where they should direct their attention.]

(Let me add just this,
it is clear that just as
the form of outer existence as such perishes,
so its opposite as such perishes as well;
therefore, realism, or more accurately objectivism
perishes along with that idealism which,
because of language's ambiguity,
we might better call subjectivism.
Reality remains, as inner being;
as we must express ourselves just in order to talk;
but in no way does it remain a term of any relation,
since a second term for the relation,
and indeed all relations in general,
have been given up.
Therefore, it is not “objective,”
since this word has meaning only
over against subjectivity,
which has no meaning from our standpoint.
Only one recent philosophical writer
(I mean Schelling)
has had a suspicion about this truth,
with his so-called system of identity;
not, certainly, that he had seen into
the absolute negation of subject and object,
but that with his system he aimed
at a synthesis post factum;
and with this operation he believed
he had gone beyond the range
of the science of knowing.
Here is how things stand about that:
he had perceived this synthesizing
in the science of knowing,
which carried it out,
and he believed himself to be something more
when he said what it did.
This is the first unlucky blow that befell him:
saying, which always stems from subjectivity
and by its nature presents a dead object, is
not more but less excellent than doing,
which stands between both in
the midpoint of inwardly living being.
Moreover, he does not prove this claim,
but lets the science of knowing do it for him,
which again seems odd:
that a system which admittedly contains
the grounds for proving our own system's
basic principle should be placed below it.
Now he begins and asserts:
reason is the absolute indifference
between subject and object.
But here must first also be added
that it cannot be an absolute point of indifference
without also being an absolute point of differentiation.
That it is neither one nor the other absolutely
but both relatively;
and that therefore, however one may begin it,
no spark of absoluteness may be brought into this reason.
So then he says: reason exists;
in this way he externalizes reason from the start
and sets himself apart from it;
thus one must congratulate him
that with his definition he has not hit the right reason.
This objectification of reason is
completely the wrong path.
The business of philosophy is not
to talk around reason from the outside,
but really and in all seriousness
to conduct rational existence.
Nevertheless, this author is
the hero of all passionate,
and therefore empty and confused heads;
and especially of those who do not disavow
defects like those reproved above,
to which, when possible,
they come even more extremely
because they think either that the inferences are good,
although the principles are false,
but that the whole is still excellent,
admittedly overlooking that all
the individual parts are good for nothing;
or finally that although it is
neither true, nor good, nor beautiful,
it still remains very interesting.
For my own person, I have said all this
only in the interest of history
and to elucidate my own views,
but in no way to weaken anyone's respect
for their hero or to lead them to myself.
Because if anyone wishes to be condemned to error,
I have nothing against it.)

Further, and to the point;
[here is the] chief result:
Consciousness has been rejected in its intrinsic validity,
despite the fact that we have admitted we cannot escape it.
We absolutely, even here in the science of knowing.
Therefore,
1. if we have once seen into this fact,
although factically we could never negate consciousness,
we will not really believe it when judging truth;
instead, when judging, we will abstract from it;
indeed, on the condition that we want to get to truth,
we must do this, but not unconditionally,
since it is not necessary that we see into the truth.
Here for the first time,
we ourselves have become entwined mediately
in science and the circle of its manifestness,
and we became the topic without any effort on our part
in a new way, because we were speaking of consciousness
and we ourselves occur empirically as consciousness,
which might serve very well for
the genetic deduction of the I, which we seek.
Further, we should cultivate here a maxim for ourselves,
a rule of judging which can be appropriated only through freedom;
and this maxim should become
the absolute principle in and for us:
if never of truth itself,
then of this truth's factical appearance.
In one respect, this is generally meaningful
and may lead to a new idealism,
in a region where it alone can have value
as the principle of appearance;
and in another respect it confirms
the opinion expressed previously
when we were characterizing and refuting
the more deeply placed idealism and realism:
that, as grounded on conflicting maxims,
both could only be reunited
by means of a higher maxim.
And so notice this also at the same time:
that the present maxim is quite different
from that of the previous realism
(which allowed truth alone to be unconditionally valid)
in that the present one has a condition:
“if truth is to be valid, then it must, etc.”;
at the same time certainly acknowledging that
truth need not necessarily be valid.
Finally, freedom shows itself here
in its most original form,
in respect to its actual operation
as we have always described it,
not as affirmative, creating truth,
but rather merely as negative, averting illusion.
All of which, although very significant,
are only expositions of this insight:
if consciousness in itself has
no validity and relation to truth,
then we must abstract from
all effects of this consciousness in
the investigations which lie before us,
and whose task is to deliver truth
and the absolute purely to the light of day.

2. From what, then, do we actually need to abstract,
and what is its unavoidable effect?
This is evident from that salient point and nerve,
for whose sake consciousness was rejected as insufficient.
But, in virtue of our last investigation, the nerve was this,
that it projected something factically,
that is in its highest potency;
and in our case [it projected] the energy,
which then would become thinking,
whose genetic connection with it
it could in no way give;
thus a thing that it projected
purely and through an absolute gap.
Grasp this character exactly,
just as it has been given, and
to that end remember what was said last time:
e.g., you would not presume
that you could actually think
without being conscious of it,
and vice versa;
nor that you could be conscious of your thinking,
without in fact actually thinking,
or that this consciousness was deceiving you;
but if you were asked to provide
an explicable and explanatory ground
for the connection of these two terms,
you would never be able to provide such a ground.
Thus, changing places with you at the site
from which you conducted your proof:
your consciousness of thinking should
contain a thinking process,
actual, true and really present,
without you being able to give an accounting of it;
therefore, this consciousness projects
a true reality outward, discontinuously:
an absolute inconceivability and inexplicability.

This discontinuous projection is evidently
the same one that we have previously called,
and presently call, the form of outer existence,
which shows itself in every categorical is.
For what this means, as a projection,
concerning which no further account can be given
and which thus is discontinuous,
is the same as what we called “death at the root.”
The gap, the rupture of intellectual activity in it,
is just death's lair.
Now we should not admit the validity of this projection,
or form of outer existence,
although we can never free ourselves from it factically;
and we should know that it means nothing;
we should know, wherever it arises,
that it is indeed only the result
and effect of mere consciousness
(ignoring that this consciousness
remains hidden in its roots)
and therefore not let ourselves
be led astray by it.
This is the sense of our discovered maxim,
which is to be ours from now on in every case
where we need it.
This very is is the original appearance:
which is closely related to,
and may well be the same thing as,
the I which we presented previously
as the original appearance.

3. Thus it is decreed against the highest idealism,
and this maxim imposes the highest realism yet on us.
Before we go further under its leadership, though,
it may be advisable to test it against the law
which it itself has brought forward,
thus to draw it directly before its own seat of judgment
in order to discover whether it itself is indeed pure realism.
It proceeds from the in-itself
and proposes this as the absolute.
But what is this in-itself as such
and in its own self?
You are invited here to
a very deep reflection and abstraction.
Although the foregoing brings to an end
the thinking of this in-itself,
reflected first by consciousness,
although, we have likewise already had to
admit before that we did not construct this in-itself
but that it already is found in advance
as completely constructed and finished
and comprehensible in itself,
and thus as constructed by itself,
so that we in any case have nothing to do with this;
we may still investigate this original construction
more closely in terms of its content.

I have said that I invite you to
a very deep reflection and abstraction.
What this abstraction is
(an abstraction which may be described
in words as well as possible)
will barely be made clear from what follows;
but this will not harm anything,
and it is certain in any case that it is already clear,
and I impose on myself the task of grasping the highest in words,
and on you the task of understanding it in a pure form.
Thus, once again, as already previously,
the topic is the in-itself,
and we are called, here as before, to
a consideration of its inner meaning
and to its re-construction.
We will not complete again what has been done before,
an act which, holding us in a circle,
would not advance us from the spot.
But, if we wish [to achieve] something else,
how is this to be distinguished from the preceding?
Thus, above we presupposed the in-itself
and considered its meaning while we supplemented it with life,
or a primal fantasy, dissolved ourself in the latter,
and had our root in it.
Of course this life is not supposed to be our life,
but rather the very life and self-construction of the in-itself:
it was then an inner determination
(one which arose immediately in this context)
of the original life itself,
which still remained dominant in this case.
So it was previously.
Now, however, we elevate ourselves for the first time to
the in-itself that is presupposed in this procedure,
[knowing it] as presupposed and unconditionally immediate,
independent of this living reconstruction,
determinate, and comprehensible:
and without this original significance the reconstruction,
as a reconstruction and clarification,
would have no basis and no guide.
For this reason, I said previously that
the originally completed construction,
the enduring content, should be demonstrated.
Therefore, things must proceed in this work
so that what is absolutely presupposed remains presupposed,
so that as a result the living quality
which we bring with us will mean nothing at all
[neither as our vivacity, nor as vivacity in general]
likewise the very validity of primal fantasy,
although it cannot be withheld factically,
yet is denied as real, in which denial
the true essence of reason may well consist.
(Or, more succinctly, if only one will understand it;
what has been made perceptible resides
in this latter construction,
and the meaning of the in-itself
grasped purely intellectually
should be found there.)

So much in the way of preliminary formal description
of this new reconstruction of the in-itself.
Now to the solution.
However one may wish to take up the in-itself,
it is still always qualified by
the negation of something opposed to it,
thereby as in-itself it is itself something relative,
the oneness of a duality, and vice versa.
Certainly, it is genuinely at once
a synthetic and analytic principle,
as we have all along looked for:
but still it is no true self-sufficient oneness;
since the oneness lets itself be grasped only through duality:
although admittedly duality also lets itself be completely
grasped and explained through oneness.
In a word, the in-itself, grasped more profoundly,
is no in-itself, no absolute, because it is not a true oneness,
and even our realism has not pushed through to the absolute.
Viewed still more rigorously,
in the oneness there is in the background
a projection of in-itself and not-in-itself,
which posit one another reciprocally
for explanation and comprehensibility,
and which negate one another in reality;
and in return, the oneness is a projection of both terms.
Further, this projection happens completely immediately,
through a gap, without being able to provide
the requisite accounting of itself.
Because how an in-itself and a not-in-itself
follow from oneness as simple, pure oneness
cannot be explained.
Of course, it can be done if the oneness is already assumed to be
the oneness of the in-itself and not-in-itself;
but then the inconceivability
and inexplicability is in this determinateness of oneness,
and it itself is only a projection through an irrational gap.
This determinateness would have no warrant
other than immediate awareness;
and actually, if we will think back to
how we have arrived at everything so far,
it has no other ground.
“Think an in-itself” it began,
and this thinking, or consciousness was possible.
And this possibility has shaped our entire investigation to date;
thus we have supported ourselves on consciousness,
if not quite on its actuality,
then certainly on its possibility,
and in this quality we have had it for our principle.
Hence, our highest realism,
the highest standpoint of our own speculation,
is itself revealed here as an idealism,
which so far has just remained hidden in its roots;
it is fundamentally factical
and a discontinuous projection,
does not stand up to its own criteria,
and, according to the rules it itself established,
it is to be given up.

4. Why should it be given up?
What was the true source of the error
which we discovered in it?
Being in-itself [was discovered] as
a negation and a relational term.
Hence we must unconditionally let that go,
if it, or our entire system, is to survive.
But something is still left over for us.
I affirm this and instruct you to find it with me:
being and existence and resting,
taken as absolute, remains,
and of course I add:
“being and resting on itself,”
but I already clearly knew that
the latter would be a mere supplement
for clarification and illustration,
but would mean nothing at all
in and for itself
and would add no supplement to the
completeness and self-sufficiency of
being's inner essence.
If I wish to look back to the previous,
already discarded expression,
“being in-itself” means a being
which indeed needs no other being for its existence.
Precisely through this not-needing
it becomes intrinsically more and more
real than it was before;
and not-needing this not-needing does not also
belong to its absolute not-needing,
and so too with the not-needing of
this not-needing of not-needing,
so that this supplement in its
endless repeatability remains always the same
and always meaning nothing in relation to
the essence taken seriously and inwardly.
Thus, I see into [the fact] that
(generally, in its core,
thus indeed as the point of oneness,
which was previously tested and discarded)
the entire relation and comparison
with the not-in-itself
(from which the form of the in-itself as such arises)
is completely null in comparison to the essence.
It is without meaning or effect.
Because I see into this,
and thereby handle the addition
in almost as negative a way
as does the essence itself,
then I must, as an insight,
participate in the essence
in some manner still to be developed.

Now, to be sure, if I pay attention to myself,
I can always become aware that I objectify
and project this pure being:
but I already certainly know
that this means nothing,
alters nothing about being,
and adds nothing to it.
To be sure, in another shape
this projection is the in-itself's supplement,
whose nothingness has already been realized:
therefore I will never be deceived by it.
In brief, the entire outer existential form
has perished in this shape,
since it is the latter in the highest [element]
in which it occurs, the in-itself;
we have only the inner essence left with which to deal,
in order to work it through:
but we truly work it through
if we see into it as the genesis
for its appearance in the outer existential form;
and nothing else can lead us to that except
not allowing ourselves to be deceived by this form.
If it does deceive us, then we just are it,
dissolved and lost in it,
and we will never arrive at its origin.

I wished at least to attach this last, fourth, point here,
in order not to end the lecture with death and destruction,
as it does on first appearance.
Tomorrow its further development.

Fifteenth Lecture
Friday, May 11, 1804
N.B. Which Contains the Basic Proposition
Honored Guests:

My task for today is this:
first of all to work out fully and completely
the main point discovered last time;
then to present a general review of
the new material added this week,
and thus, as it were, to balance the accounts,
since with this lecture we conclude the week's work
and a discussion period intervenes

On the first matter.
This much as a reminder in advance.
To begin with, the point I must present is
the clearest and simultaneously the most hidden of all,
in the place where there is no clarity.
Not much can be said about it,
rather it must be conceived at one stroke;
even less can anything be said about it
or words used to assist comprehension,
since objectivity, the first basic twist of all language,
has already long been abandoned in our maxim,
and is to be annulled here within absolute insight.
At this point, therefore, I can rely only
on the clarity and rapidity of spirit
which you have achieved in the previous investigations.
So then, on a particular occasion I divided
the science of knowing into two main parts;
one, that it is a doctrine of reason and truth,
and second, that it is a doctrine of appearance and illusion,
but one that is indeed true and is grounded in truth.
The first part consists of a single insight and is begun
and completed with the single point which I will now present.
To work!
After the problem of absolute relation,
which appeared in the original in-itself,
which itself pointed to a not-in-itself,
nothing remained for us except
the pure, bare being by which, following the maxim,
our objectivizing intuition must be rejected as inadequate.
What then is this pure being in its abstraction from relatedness?
Could we make it even clearer to ourselves and reconstruct it?
I say yes: the very abstraction imposed on us helps.
Being is entirely of itself, in itself, and through itself;
this self is not to be taken as an antithesis,
but grasped with the requisite abstraction purely inwardly,
as it very well can be grasped,
and as I for example am most fervently conscious of grasping it.
Therefore, to express ourselves scholastically,
it has been constructed as a being in pure act,
so that both being and living, and living and being
completely interpenetrate, dissolve into one another,
and are the same, and this self-same inwardness is
the one completely unified being,
which was the first point.

This sole being and life can not exist,
or be looked for, outside itself,
and nothing at all can exist outside it.
Briefly and in a word:
duality or multiplicity does not occur at all
or under any conditions, only oneness;
because by its own agency
being itself carries self-enclosed oneness with itself,
and its essence consists in this.
Being [understood by language as a noun]
cannot literally be actively
except immediately in living;
but it [“being” understood as a noun] is only verbally;
because completely noun-like being is objectivity,
which in no way suffices:
and it is only by surrendering
this substantiality and objectivity,
not merely in pretense
but in the fact and truth of insight,
that one arrives at reason.

On the other side,
what lives immediately is the “esse,”
since only the “to be” lives,
and then it, as an indivisible oneness,
which cannot exist outside itself
and cannot go out of itself into duality,
is something to which the things
we have demonstrated immediately apply.

But we live immediately in the act of living itself,
therefore we are the one undivided being itself,
in itself, of itself, through itself,
which can never go outside itself to duality.

“We,” I say, to be sure, we are immediately conscious that,
insofar as we speak of it, we again objectify
this “We” itself with its inward life
but we already know that this objectification means
as little in this case as in any other.
And surely we know that we are not talking about this We-in-itself,
separated by an irrational discontinuity from the other We
which ought to be conscious;
rather [we are talking] purely
about the one We-in-itself,
living purely in itself,
which we conceive merely through
our own energetic negation of the conceiving
which obtrudes on us empirically here.
This We, in immediate living itself;
this We, not qualified or characterizable
by anything that might occur to someone,
but rather characterizable purely
by immediately actual life itself.

This was the surprising insight
to which I wished to elevate you,
in which reason and truth emerge purely.
If anyone should need it, I will point
to it briefly from yet another aspect.
If being is occupied with its own absolute living,
and can never emerge out from it,
then it is a self-enclosed I,
and can be nothing else besides this;
and likewise a self-enclosed I is being:
which “I” we could now call “We”
in anticipation of a division in it.
Hence, we in no way depend here on
an empirical perception of our life,
which would need to be completely rejected
as a modification of consciousness;
rather we are depending on the genetic insight into life and the I,
which emerges from the construction of the one being, and vice versa.
We already know, and abstract completely from, the fact,
that this very insight as such, together with its reversal,
is irrelevant and vanishes;
we will need to look back to it again only in deducing phenomena.

As it has been presented now,
this intrinsically cannot be made clearer by anything else;
since it itself is the original source
and ground of all other clarity.
Still, the subjective eye can get clearer,
and become more fit for this clarity,
through deeper explanation of the immediately surrounding terms;
therefore I add in this regard another consideration
which lies on the system's path anyway.
Yesterday, and again today too at the beginning of our meditation,
although we constructed being according to its inner essence,
if we only remember, we placed being objectively before ourselves,
despite the fact that, though only as a result of following our maxim,
we grant no validity to this objectivity:—factically,
even though surely not intelligibly or in reason,
being remains separated from itself.
But, just as in this reflective process we are
grasped by the insight that being itself is an absolute I, or we;
thus the first remaining disjunction,
between being and the we, is completely annulled, even in facticity,
and the first version of the form of existence is also factically negated.
Previously [we knew that] at the very least
we emerged factically out of ourselves toward being,
and [in this process] it could very well happen that
being would not come out of itself,
especially if we did not wish to be being.

We did not accept this being as valid,
merely on the basis of a maxim
that had its proof via derived terms,
and thus might well need a new proof here.
As we become being itself in the insight we have produced,
we can, as a result of this insight, no longer come out
of ourselves toward being, since we are it;
and really we absolutely cannot come out of ourselves at all,
because being cannot come out of itself.
Here the preceding maxim has received its proof,
its law, and its immediate realization in the insight:
because this insight in fact no longer objectifies being.
Now to be sure, I say, no other objectifying consciousness arises
together with this insight, because in order for that to occur,
a self-reflecting would be required to stand in between,
however the possibility does arise of an objectifying consciousness:
[namely,] our own.
Now, as regards the content of this new objectification,
it is already clear that it does not bring with itself
any disjunction in our subject matter,
as does the first, between real being and absolute non-being,
rather this content brings only the mere repetition
and repeated supposition of one and the same I, or We,
which is entirely self-enclosed, which encompasses all reality in itself,
and which is therefore entirely unalterable;
therefore, it does not contradict the original law
of not going out of oneself in essence.

But, as the first stage of our descent into phenomenology,
we will have to explore whence this
empty repetition and doubling may arise;
today it is merely a matter of establishing the insight
which is expressive of pure reason, that being or the absolute is
a self-enclosed I, in its unalterability.

Now on to the second part of the general review.
As must arise in any presentation of the science of knowing,
we have proceeded factically, inwardly completing
something useful for our purpose,
and paying attention to how we have done it,
compelled always, as is evident, by
an unconscious law of reason working in us.
On this path, which I will not repeat now,
we had elevated ourselves to a pure “through,”
as the essence of the concept;
and had understood that the latter's realization
presupposed a self-subsisting being.

Having presented this insight as a fact
and reflected further about its principle,
it turned out either that one could posit
the energy of thinking a “through” needing
completion as the absolute
and thus as the source of intuition
and of life in itself within intuition,
which was an idealism;
or that, considering that life ought to exist in itself,
one could take the latter as the principle,
with the result that everything else perishes;
this latter [was] a realism.

Both were supported by maxims.
The former on this: the fact of reflection is
to be taken as valid, and nothing else;
the latter on this:
the content of the evident proposition is
to be taken as valid, and nothing else;
and, for that very reason, both are at bottom factical,
since indeed even the contents of what is manifest,
which alone should be valid for realism, is only a fact.

Along with the necessity that arose from this
to ascend higher and to master the facts genetically,
we turned our attention to what promised to be most significant here,
[namely] to the in-itself, bound to the realistic principle,
to life in itself:
and this further deliberation was the first step we made this week.
It turned out that the in-itself manifests as an absolute
negation of the validity of all seeing directed toward itself:
that it constructs itself in immediate manifestness,
and with its own self-construction even gives
off immediate manifestness or light:
yielding a higher realism which deduces
insight and the light themselves,
items that the first realism was content to ignore.
A new idealism attempts to establish itself
against this new realism.

That is, we had to take command of ourselves
and struggle energetically to contemplate
the in-itself in its significance.
So, [we] believed we realized that this in-itself first appeared
as a result of this reflection as
simultaneously constructing itself with
immediate manifestness in the light;
and that consequently this energy of ours would be
the basic principle and first link in the whole matter.

Realism, or we ourselves, since we are nothing else than this realism,
fights very boldly against this as follows:
“If you really actually think ...”
and to what will you appeal for confirmation of this assertion:
you can adduce nothing more than that you are aware of yourself,
but you cannot derive thinking genetically in its reality and truthfulness,
as you should, from your consciousness in which you report it;
but, by contrast, we can derive the very consciousness to which you appeal,
and which you make your principle, genetically,
since this can surely be only a modification of insight and light,
but light proceeds directly out of the in-itself,
manifestness in unmediated manifestness.
The higher maxim presented in this reasoning would just be this:
to give no credence to the assertions
of simple, immediate consciousness,
even if one cannot factically free oneself from them,
but rather to abstract from them.

What is this consciousness's effect,
for the sake of which it is discarded;
and therefore what is that
which must always be removed from the truth?
Answer:
the absolute projection of an object
whose origin is inexplicable,
so that between the projective act
and the projected object
everything is dark and bare;
as I think I can express very accurately,
if a little scholastically,
a proiectio per hiatum irrationale
(projection through an irrational gap).

Let me again draw your attention to this point,
both for now and for all your future
studies and opinions in philosophy:
if my current presentation of
the science of knowing has been clearer
than all my previous versions of the same science,
and can maintain itself in this clarity,
and if clear understanding of the system is
to make a new advance by these means,
the ground for this must lie simply
in the impartial establishment of
the maxim that immediate consciousness is
in no way sufficient and that hence
it does not suffice in its basic
law of projection per hiatum.
Of course the essence of this truth has ruled
in every possible presentation of
the science of knowing from the very first hint
which I gave in a “Review of Aenesidemus”
in the Allgemeinen Literatur Zeitung;
because this maxim is identical with
the principle of absolute genesis.
If nothing that has not been realized
genetically is permitted,
then projection per hiatum will not be permitted,
since its essence consists precisely in non-genesis.
If one has not made himself explicitly aware
that this non-genesis,
which is to be restrained in thinking,
remains a factical element of the consciousness
which is unavoidable on every path
of all our investigations
and of the science of knowing itself,
then one torments and exhausts himself
trying to eliminate this illusion,
as if that were possible.
And the sole remaining way of breaking through to
truth is to divide the illusion,
and intellectually to destroy each part one at a time,
while during this procedure one actually defers the illusion to
another piece at which annihilation will arrive later,
when the first piece could once again serve
as the bearer of the illusion.
This was the science of knowing's previous path,
and it is clear that it too leads to the goal,
although with greater difficulty.
However, if one knows the origin of non-genesis in advance,
and that it always comes to nothing,
although it is unavoidable, then one no longer fights against it,
but rather allows it to work peacefully:
one simply ignores it and abstains from its results.
It is possible in this way alone to gain
access to insight immediately as we have done decisively,
and not just by inference from the nonexistence of the two halves.

Let me continue the recapitulation:
the realism which presented
the recently analyzed maxim was
itself questionable and was brought before
the judgment seat of its own maxims.
There, on closer consideration, the in-itself
(inasmuch as it is assumed to be something original
and independent of all living construction
and to possess this same guiding meaning)
turned out to be incomprehensible
without a not-in-itself.
Therefore, in the understanding
it turned out to be no in-itself
(something comprehensible by itself alone) at all,
and instead it [turned out to be] understandable
only through its correlative term.
Therefore, the unity of understanding,
which reason presupposes here,
cannot merely be a simple self-determined oneness;
instead it must be a unity-in-relation,
meaningless without two terms
which arise within it in two different connections:
in part as positing one another
and in part as negating one another,
thus the well-known “through” and
the five-foldness recognized in it.
If one must also now concede that
once oneness has been admitted,
the terms incontestably posit themselves genetically,
still oneness itself is not thereby explained genetically;
hence it is present simply by means of
a proiectio per hiatum irrationalem,
which this system, [presenting itself] as realism,
has made against its own principles.

With this disclosed, absolutely everything
in the in-itself which pointed to relations
was to be abandoned,
and so nothing else remained behind
except simple, pure being
as absolute, self-enclosed oneness,
which can only arise in itself,
and in its own immediate arising or life:
which therefore always arises
from a place where an arising,
a living, simply occurs, and
it does not arise except in such an arising,
and therefore occurs as absolute I,
as today's disclosure could be put briefly.
Generally, one can think this simplest
of all insights in indefinitely many forms,
if it has once become clear.
Its spirit is that being exists
immediately only in being, or life,
and that it exists only
as a whole, undivided oneness.
