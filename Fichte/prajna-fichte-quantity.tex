Eighth Lecture
Monday, April 30, 1804
Honorable Guests:

I believe we have arrived at a focal point in our lectures,
a point which more effectively facilitates
a clear insight and overview than does any other,
and which therefore will permit us greater brevity in what follows.
Therefore, let us not economize on time now,
and from here on out we will make ourselves more secure.
Today we will do this with the contents of the last lecture.

We have seen actually and in fact,
and not just provisionally,
that an absolute, self-grounded insight negates
an equally absolutely created division
(that is, one not grounded in things)
as invalid, and that this insight posits
in the background a self-subsistence,
which cannot itself be described more precisely.
Let this be today's first observation:
at this point the main thing is that
all of us assembled here together have
really and actually seen into this
just as it has been presented,
and that we will never again forget this
self-insight or allow it to fade;
but rather we will take root in it
and flow together with it into one.

Thus, what has been said is not just my report
or that of any other philosopher,
but rather it exists unconditionally,
and it remains always true,
before anyone actually sees into it,
and even if no one ever does.
We, in our own persons, have penetrated to the core
and have viewed the truth with our own eyes.
Likewise, as has been evident from the first,
what has been said is in no way proposed
as a hypothetical proposition,
which is shown to be true in itself
only by way of its usefulness in explaining phenomena,
as is the case in the Kantian, and every other philosophy.
Instead it is immediately true independently of
all phenomena and their explicability.
(A good reason for making this more precise!)
Therefore, what genuinely follows from this,
if only it is itself completely enough determined,
is also as unconditionally true as it [the original insight] is.
And everything which contradicts it,
or the least of its results, is unconditionally false
and should be abandoned as false and deceptive.
This categorical decisiveness between truth and error
is the condition for our, and every, science;
and it is presupposed.
It is far removed from that skeptical paralysis
which in our days parades as “wisdom,”
doubts what is unconditionally manifest,
and wants to make the latter clearer
and more manifest by the most derivative means.

And as particularly concerns the explanation
of phenomena from a principle that is manifest,
it is obvious right away that if the principle is sound,
and likewise the inferences,
then things will go well with the explanation;
we need only note that,
since the principle first grants us a true insight
into the phenomenon's essence as such,
it may well happen that in this proof process
many things do not even have the honor
of being genuine, orderly phenomena,
but rather dissolve into deceptions and phantasms,
although all ages have held them to be phenomena,
or, God forbid, even to be self-subsisting realities.
Therefore, it may happen that in this regard science,
far from acquiring some law or orientation
from the factical apprehension of appearances,
on the contrary rather legislates for them.
This situation can also be expressed as follows.
Only what can be derived from the principle
counts as a phenomenon;
what cannot be derived from it is an error
simply because of its non-derivation,
although it may perhaps incidentally also be immediate,
if one wants to boast of this direct proof.

This is the second observation;
already as a result of this just repeated insight,
a new world of light has opened for us,
which transcends our entire actual knowledge,
and a world of error,
in which nearly every mortal
without exception finds himself,
has perished,
especially if we appropriate
what follows and is recorded below.
If we take up this result directly here,
it will be invigorating for the attention
and throw a very beneficial light
on what comes next.

1. By annihilating the formal concept,
which is the condition for its own
real appearance and vivacity,
the light, as the one true self-sufficiency,
posits a self-sufficient being,
which is not further determinable and
which, as a result of
the insufficiency of the concept
that attempts to grasp it,
is inconceivable.
The light is simply one,
the concept that disappears in the light is one
(the division of what is one in-itself),
and being is one;
it can never be an issue of
anything other than these three.
The one existence arises in
the intuition of what is independent,
and in the concept's negation
(and it will turn out that whatever genuinely concerns
true existence will also rest in this).
If, as is customary, you want to call
the absolutely independent One,
the self consuming being, God,
then [you could say that] all
genuine existence is the intuition of God.
But at the same time note well
[and already a world of errors will be extinguished]
that this being, despite the fact that
the light posits it as absolutely independent
(because the light loses itself in its life),
is actually not so, just because it bears within itself
the predicate “is,” “persistence,” and therefore death.
Instead, what is truly absolutely
independent is just the light;
and thus divinity must be posited
in the living light and not in dead being.
And not, I certainly hope, in us, as
the science of knowing has
often been misunderstood to say;
for however one may try to understand that,
it is senseless.
This is the difficulty with every philosophy
that wants to avoid dualism
and is instead really serious about
the quest for oneness:
either we must perish, or God must.
We will not, and God ought not!
The first brave thinker
who saw the light about this
must have understood full well that
if the negation is to be carried out,
we must undergo it ourselves:
Spinoza was that thinker.
It is clear and undeniable in his system
that every separate existence vanishes
as [something] independently valid and self-subsistent.
But then he kills even this, his absolute or God.
Substance = being without life
[because he forgets his very own act of insight]
the life in which the science of knowing
as a transcendental philosophy makes its entrance.
(“Atheist or not atheist?”
Only those can accuse the science of knowing of atheism
[I am not concerned here with real events,
because regarding all those the science of knowing is not at issue,
since in fact no one knew anything about it]
who want a dead God, inwardly dead at the root,
notwithstanding that after this it is dressed up
with apparent life, temporal existence, will,
and even sometimes with blind caprice,
whereby neither its life nor ours becomes comprehensible;
and nothing is gained except that one more
number is added to the crowd of finite beings,
of whom there are more than enough in the apparent world:
one more that is just as constrained and finite as themselves
and that is in no way different from them in kind.
[I mention] this in passing to state clearly and in a timely way
a significant basic quality of the science of knowing.)

One term is being, the other [the negated concept] is
without doubt subjective thought, or consciousness.
Therefore we now have one of the two basic disjunctions,
that into B and T (being and thinking),
we have grasped this in its oneness, as we should,
and as proceeding completely and simply
from its oneness, (L = light);
and thereby, so that I can add this, too, parenthetically,
we would simultaneously have the schema for the negation
of the I in the pure light and even have it intuitively.
For if, as everyone could easily agree, one posits
that the principle of the negated concept is just the I
(since I indeed appear as freely constructing and sketching out
the concept in response to an invitation),
then its destruction in the face of what is valid in itself is
simultaneously my destruction in the same moment,
since I as its principle no longer exist.
My being grasped and torn apart by the manifestness
which I do not make, but which creates itself, is
the phenomenal image of my being negated and extinguished
in the pure light.

2.  This, I say, is a result of the light itself
and its inward living expression:
things must remain here as a consequence of
this insight and in case they simply follow it,
and we will never get beyond it.
But I claim that, if only we reflect correctly,
we are already beyond it:
we have certainly considered the light
and objectified it: the light therefore
(whomever has forgotten this circumstance
during the previous explanation should recall it now)
has a twofold expression and existence,
partly its inner expression and existence
(conditioned by the negation of the concept,
conditioning and positing absolute being),
partly an external and objective expression and existence,
in and for our insight.

As concerns the latter, that at first we speak of it alone;
we surely remember that we did not possess it,
and everything that lies within it, immediately;
instead we raised ourselves to it from
the beginning of our investigation,
initially by abstracting from the whole variety
of objective knowings [and rising] to
the absolute, self-presenting insight
that genuine knowing must always remain self-same in this variation;
and then [we continued] by means of
deeper, genetic examination of this insight itself.
So far this has been our procedure;
the new and unknown spiritual world
in which we pursue our path has arisen
by this procedure alone,
and without it we would be speaking of nothing.
It now further appears to us that
we could very properly have neglected this procedure,
just as we have undoubtedly neglected it every day
of our lives before we came to the science of knowing.
Taking up this appearance now
[while not making any further inquiry
into its validity or invalidity]
[we find that] it contains the following:
the light's external existence in an insight directed to it,
as the one absolute, eternally self-same
in its fundamental division into being and thinking;
[it] is conditioned by a series of abstractions and reflections
that we have conducted freely,
in short by the procedure that we state as
the free, artificially created science of knowing;
this external existence arises only in this way and for it,
and otherwise not at all.
This the first point here.

But, on the contrary, we assert,
as concerns the inner existence and
expression of the light
that if the light exists unconditionally
[and in particular whether we have insight into it or not,
and this is the very insight which depends on appearing freedom]
it is in and for itself the very same one,
eternally self-identical, and thoroughly necessary,
if only the light exists.
Therefore we assert a meaningful consequence,
something I bid you to mark well,
that there are two different modes
of light's life and existence:
the one mediately and externally in the concept,
the other simply immediate through itself,
even if no one realizes it.
Strictly speaking in actual fact
no one ever does realize it,
but instead this inner life of the light is
completely inconceivable.
This is the second point.

Light is originally divided into being and thinking.
That the light lives unconditionally therefore means
that it splits itself completely originally,
and also inexplicably, into being and concept,
which persists, even though it is negated qua concept.
To be sure, insight can follow this very split,
just as it is now on our side following reconstructively
the split into concept qua concept and being qua being;
but at the same time insight must leave the inner
split standing as impenetrable to it.
This yields, in addition to the previously
discovered and well-conceived form of inconceivability,
a material content of light that remains ever inconceivable.

(I have just expressed myself on a major point
in the science of knowing more clearly than
I have previously succeeded in doing.
We would accomplish a great deal if this
became clear to us right now on the spot.
That the light lives absolutely through itself must mean:
it splits itself absolutely into B (being) and T (thinking).
But “absolutely through itself” also means
“independent of any insight [into itself]
and absolutely negating the possibility of insight.”
Nevertheless for the last several lectures
we have seen, and had insight into,
the fact that light splits itself into B and T:
consequently this split as such
no longer resides in the light, as we had thought,
but in the insight into the light.
What then still remains?
The inward life of the light itself in pure identity,
from itself, out of itself, through itself
without any split;
a life which exists only in immediate living
and has itself and nothing else.
“It lives”;
and thus it will live and appear
and otherwise no path leads to it.
“Good, but can you not provide me
with a description of it?”
Very good, and I have given it to you;
it is precisely what cannot be realized,
what remains behind after
the completely fulfilled insight
which penetrates to the root,
and therefore what should exist through itself.
“How then do you arrive at these predicates
of what cannot be realized;
is not to be constructed from disjunctively related terms
as being is from thinking and vice versa
predicates such as that it is
“what remains behind after the insight,”
“that which ought to exist through itself,”
qualities which are the content or the reality
you have claimed to deduce fundamentally?
Manifestly only by negation of the insight:
hence all these predicates,
leading with the most powerful, absolute substance,
are only negative criteria, in themselves null and void.
“Then your system begins with negation and death?”
By no means;
rather it pursues death
all the way to its last resort
in order to arrive at life.
This lies in the light
that is one with reality,
and reality opens up in it.
And the whole of reality as such
according to its form is
nothing more than the graveyard of the concept,
which tries to find itself in the light.)

It is obvious that our entire enterprise
has achieved a new standpoint
and that we have penetrated more deeply into its core.
The light, which up to now has been understood only
in its form as self-creating manifestness,
and hence assigned only a merely formal being,
has transformed itself into one living being
without any disjunctive terms.
What we have so far assumed to be
the original light has now changed itself
into mere insight and representation of the light;
and we have not merely negated the concept
that has been recognized as a concept,
but even light and being as well.
Previously only the mere being
of the concept was to have been negated;
but how were we supposed to have arrived
at this being even though it is empty?
It was to be negated by something
that itself was nothing.
How could that be possible?
Now we have an absolute reality
in the light itself,
out of which perhaps the being that appears,
as well as its nonexistence in the face of the absolute,
might be made comprehensible.

I now explicitly mention in addition
what direct experience teaches in any case,
that this reality in the original light,
as it has been described,
is unconditionally and completely one and self same,
and that no insight is allowed into how it might arrive
inwardly at a division and at multiplicity.
Observe: the division into being and thinking
(as well as what might, following previously given hints, depend on this)
resides in the concept
which perishes in the presence of reality
and so has nothing at all to do with reality and the light.
Now, according to the testimony of appearance in life
to which our system has provisionally
granted phenomenological truth,
another disjunction ought to arise
which stands either higher,
or at least on the same level
with being and thinking,
since it ranges across both of the latter;
and this is taken for a disjunction in reality.
Since this last contradicts our previous
insight it is thus certainly false;
hence this new grounds for disjunction must
lie in a determination of the concept
that has not yet been recognized
or is not yet sufficiently explored.
The concept, as a concept, must itself be conceivable,
and so no new inconceivability can appear here.
But if this determination of
the concept is grasped conceptually,
then everything that it contains can be
derived conceptually from it.
Whatever range of differences may come forward
in appearing reality now and for all time,
yet it is once and for all clear
a priori that they are B—T + C + L;
one-and-the-same,
remaining eternally self-identical,
and only different in the concept.
Therefore, it is clear that,
since everything true must begin with it
and that falsity and illusion must be turned away,
reality [with which alone true philosophy can be concerned],
not only is generally completely deduced
and made comprehensible,
but also divided and analyzed a priori
into all its possible parts.
“Into its parts,” I say,
excluding from this L (= the light).
For in fact this is not a part,
but the one true essence.
It is hereby likewise clear how far
the deduction and reconstruction of
true knowing goes in the science of knowing:
insight can have insight into itself,
the concept can conceive itself;
as far as one reaches, the other reaches.
The concept finds its limits;
conceives itself as limited,
and its completed self-conceiving is
the conceiving of this limit.
The limit, which no one will transgress,
even without any request or command from us,
it recognizes exactly;
and beyond it lies the one, pure living light;
insight points therefore beyond itself
to life, or experience,
but not to that miserable assembly of
empty and null appearances in which
the honor of existing has no part:
but rather to that experience
which alone contains something new:
to a divine life.

Ninth Lecture
Wednesday, May 2, 1804
Honored Guests:

In the next three lectures I am about to enter
into a deeper investigation than has been made so far.
This investigation will, as it happens,
set out to secure a stable focus
and, leading from it, a permanent guide for our science,
even before we possess this guide.
So, in order not to get confused, a lot depends
on our holding on to what we have laid down provisionally;
therefore:

1. Formally, in relation to the material
which we are investigating
and to the manner in which we take it,
we are already located beyond the prolegomenon
and actually inside the science of knowing;
because (the previous lecture began by recalling this)
we have already actually created insights in ourselves,
which have transposed us into an entirely new world
belonging to the science of knowing
and raised above all factical manifestness,
in whose realm the prolegomenon always remains.
We have passed unnoticed out of
the prolegomenon and into science;
and indeed the transition started as follows:
we had to elucidate the procedure of
the science of knowing by examples,
and, since I found that the state of
the audience made it possible,
we made use of the actual thing
as the original example.
Let us now drop this as a mere example
and take things up earnestly and for real;
thus we are inside the science.
Just as this has so far happened tacitly,
let us now proceed conscientiously and explicitly.

2. Here is how things stood in the hour before last:
I—L—B.a (a = our insight into the matter).
I (Image), positing something imaged in it,
= B (being) and vice versa;
united in the oneness of the light (L).
Thus, on the one side,
the connection of I—L—B,
the essential element of all light without exception:
on the other side, the modifications
without which it does not exist.
This effectively indicates the way in general,
but nothing is still specially known thereby.
It is only the prolegomenon to our investigation.

Additionally, this gives a good hint
concerning an important point
which is not to be handled
without difficulty as to its form.
Knowing should divide itself entirely at one stroke
according to two distinct principles of division:
B–T / oneness, and x, y, z / oneness.
Here we see that the light,
in itself eternally one and self-identical,
does not divide itself in itself,
but rather divides itself,
in its insight and as being seen,
into this multiplicity, whatever x, y, z may be;
the [same] light, which, in itself and in its eternal
identity independent from insight into it
(at least as we have posited this more deeply),
divides itself into being and thinking.
Therefore, if the light does not even exist
except in being the object of insight,
this again divided;
likewise the light does not exist in itself
without dividing itself into being and thinking,
so this disjunction is absolutely one and indivisible
according to both distinguishing grounds.
At present things must remain here,
and this proposition, together with all
further qualifications which it may yet receive,
true in itself and remaining true,
will never be permitted to fall.
(Just by virtue of the fact that
one has fixed termini in
the conduct of the investigation,
one is able to follow the investigation's
most divergent turns without confusion,
and to orient oneself in it
as long as the point at which
everything ties together remains;
while otherwise one would very quickly
be led into confusion.)

Now, in regard to the concept
[which lies neither in the light,
as what is imaged for the concept,
nor in the insight, as the image itself,
but rather between these two]
we realize that formally in itself
this concept is a mere through-one-another,
without any external consequences,
[without antecedent and without consequent,]
which two, and all their shifting relations,
arise only out of the living exhibition of this concept.
This insight, which, if I am not mistaken,
has been presented with the highest clarity,
is presupposed and here only recalled for you.
If I wished to add something here
to sharpen this insight,
then it could only be this:
since the concept, as absolute relation of
the imaged to the image and vice versa,
is only this relationship,
it makes no difference to it that
the thing imaged should be self-sufficient light
and that the image should be this image.
Something imaged and the image,
simply as such, are sufficient.
Further, the imaged thing and the image are
also of little concern to the concept's inner essence,
the latter presupposed as absolutely self-sufficient;
instead this inner essence is
evidently a mere through-one-another.
That this through-one-another, as simply existing,
manifests in the image and the thing imaged
has shown itself empirically.
But who then authorizes us to say,
on the one hand either that this
through-one-another must manifest itself or exist,
or, on the other hand in case the former should be true,
that it must construct itself directly
in the image and the thing imaged
rather than, say, construct itself
for another and under other conditions
in an endless variety of ways?
Through this consideration
we lose the subordinate terms,
and their distinguishing grounds,
for a system of genetic knowing.
On the other hand, in case someone is
willing to grant us this,
who would then authorize us
to assume that the thing imaged
could only be the light,
and that therefore necessarily
the image of the light,
which arises in the concept
as its imaged object and by means of it,
must thereby bring in the other distinguishing ground?
Thereby we also lose the second half in any system
which does not rest content with factical manifestness,
and rejects everything that is not
seen genetically as necessary.

Of course, this result comes out the only way it could,
as soon as we think seriously.
If we posit the concept, the absolute through-one-another,
as an independent self-subsisting being,
then everything external to it disappears
and no possibility of escaping it is to be found;
just as things happened previously
with the light when we likewise posited it.
That is obvious.
Any independent being annuls any other being external to it.
Whenever you might wish to posit a being of this sort
it will always similarly have this result,
which resides in its form.

This observation provides exactly the right task
for our further procedure;
and I wish that we could come to know
this procedure in its unity right now in advance,
so that we would not go astray
among the various forms and changes
which it may assume as we go along,
would easily recognize the same pathway
in every possible circumstance
only with this or that modification,
and would know which modification
it was and from whence it comes.
The genetic relation whose interruption
has come to light must be completed.
This cannot simply be done
by inserting new terms
and thereby filling the gap,
for where would we get them?
We are scarcely capable of adding something in
thought where nothing exists.
Therefore, the genetic relation which is currently absent
must be found in the terms already available;
we have not yet considered them correctly, completely genetically,
but so far still only considered them in part factically.
“In the terms already available,” I say;
thus, if the only important thing were
to arrive at our goal by this path,
it would not matter with which available term we began.
If we worked through only one of these
to its implicit, creative life,
then the flood of light
which simultaneously overcomes and connects
everything would of necessity dawn in us.
But beyond this we also have
the task of following the shortest way;
and so it is quite natural for us to hold on
to what has shown itself to us as the most immediate,
those terms in which we alternatingly have placed the absolute,
and in regard to which we now find ourselves in doubt
as to which is the true absolute,
namely light and concept.

If we work through both
so that each shows itself
as the principle of the other,
then it is clear that

a. in each we have grasped mediately
the distinguishing ground which is immediately
present in the other,
and that
b. beginning from both, we, in our scientific procedure,
have obtained a yet higher common principle
of distinction and oneness for both on essential grounds.

Therefore, both of these lose their absolute character
and retain only relative validity.
Thus, our knowledge of the emerging
science of knowing transcends
them as something absolutely presupposed;
and according to its external form
this is a synthesis post factum.
But since this transcendence is
itself genetic in its inner essence
(and it is not simply as Kant,
and indeed we ourselves speaking preliminarily, have said,
that “there must surely be some yet higher oneness,”
but rather this oneness in its inner essence is
actually constructed)
it is a genetic synthesis.
But again, the science of knowing,
which is genetic in its principles
and which permeates the higher oneness,
is permeated by it,
and is therefore itself identical with it,
steps down into multiplicity and is
simultaneously analytic and synthetic,
[truly, livingly genetic.]
Our task is discovering this
oneness of L [light] and C [concept],
and discovering it in this briefly
but precisely prescribed way;
this discovery is the common point to which the
whole of our next stage refers.
This procedure's modifications
and various turnings are grounded in
the necessity now of properly permeating C
by genetically permeating L,
and then again the other way.
Thus, [it grounds itself] on
constant shifts in standpoint
and being tossed from one to the other.
I will not conceal the fact that
this procedure is not without difficulty
and that it demands a particularly
high degree of attention;
instead I announce this explicitly.
But I am overwhelmingly convinced
that whoever has actually seen into
what has been presented so far,
and holds fast to the present schema
and the just asserted common point for our investigation,
orienting himself in terms of these from time to time,
will not be led astray.
On the other hand,
this is the only truly difficult part of our science.
The other part,
deducing the mediate and secondary disjunctions,
is a brief and easy affair for those
who have properly achieved the first,
no matter how monstrous and mad
it may appear to those who know nothing of the first.
This second part namely,
as is evident from the foregoing
and, which I mention here only redundantly,
has the task of deducing all possible modifications
of apparent reality.
The individual who has so far remained trapped
in factical manifestness wonders at this
because it is the only difficulty
which is accessible and apparent to him.
But until it has its own openly declared principle,
this deduction (of the manifold of apparent reality) 
is nothing more than a clever discovery,
which has recourse to the reader's genius and sense of truth,
but can never justify itself before rigorous reason,
if it does not have, and declare, its own principle.
Now to discover and clarify this principle may 
certainly be the right work:
for one who possesses it,
the application will thus surely be simple,
and (since the most complete clarity and distinctness is
to be found here)
it will be even simpler than the
application of principles in other cases.
Indeed, one could, if necessary,
simply rest satisfied to have shown
this application through a few examples.
Since I would gladly dispose of this once and for all,
let me take it down to specific cases:
the deduction of time and space
with which the Kantian philosophy exhausts itself
and in which a certain group of Kantians remain
imprisoned for life as if in genuine wisdom,
or of the material world in its various levels of organization,
or of the world of the understanding in universal concepts,
or of the realm of reason in moral or religious ideas,
or even the world of minds,
presents no difficulty and is certainly
not the masterpiece of philosophy.
Because all these things,
together with whatever one might wish to add to them,
are actually and in fact nonexistent;
instead, in case you have only just understood their nonexistence,
each one is the very easily grasped appearance
of the truly existing One.
To be sure, up until now,
some have freely believed in the existence of bodies
(truthfully, in the nothing which is presented as nothing)
and, at the most, in the existence of souls
(truthfully, in ghosts),
and perhaps they have even conducted
deep researches into the relation of body and soul
or into the soul's immortality.
Let me add that not for one moment do I support
skepticism about the latter or wish to wound faith.
The science of knowing legislates
nothing about the immortality of the soul,
since in its terms there is
neither soul, nor dying, nor mortality,
and hence there is likewise no immortality;
instead there is only life,
and this is eternal in itself.
Whatever exists, is in life,
and is as eternal as life is.
Thus, the science of knowing holds with Jesus that
“whoever believes in me shall never die,”
but it is given him to have life in himself.
But I say, picking up the thread,
whoever has believed something like this
and is used to philosophical questions of this kind
demands that a science, which says the things ours does,
address this point with him and free him from error,
if only by an induction on what he has so far taken as reality.
This is what Kant for instance, did;
but it did not help at all.
Nothing could help that did not address the problem at its roots.
The science of knowing does even better than they wish,
according to rigorous methods and in the shortest possible way.
It doe s not cut off errors individually,
since it is evident that in this work
as soon as error is removed on one side
it springs up on the other;
rather it insists on cutting off
the single root for all the various branches.
For now, the science of knowing asks for patience
and that one not sympathize with
the individual appearances of disease,
which [our science] has no wish to heal:
if only the inner man is first healed,
then these individual appearances
will take care of themselves.

What must be built up in us today is
this declaration of our proper standpoint,
and to consider the unity of our following proceedings,
its coherence as a first part,
with a later piece that can be seen as the second part;
and in relation with this you can view
everything previous as a condition of
clear insight into today's material.
Nothing is thereby gained for material insight
into our investigation's object;
there is indeed a very important point in this insight
which we found last time while doing something else,
dropped today as not relevant to our purpose,
and which we will investigate again tomorrow
for the purpose we have announced clearly today.
As concerns the form, however,
a general perspective and orientation has been achieved
which will guard us from any future confusions.
The schema serves as provisionally valid
and it will endure only those revisions
grounded in our growing insight and no capricious ones.

In conclusion, in order to point out to those
who have attended my previous course
where they find themselves in the process
and thereby to put them in a position
to view the science of knowing with
the complexity that repeatedly pursuing it allows:
what I am now calling concept was named
in the first series “inner essence of knowing,”
what is called here “light”
was there called “knowing's formal being,”
the former [was called] simply the intelligible,
the latter intuition.
For it is clear that the inner essence of knowing
can only be manifested in the concept,
and indeed in the original concept;
likewise, [it is clear that] that this concept,
as implicitly insight, must posit insight, or light.
Therefore, it is clear that the task here expressed as
“finding the oneness of C and L” is
the same one expressed there by the sentence:
“The essence of knowing [is] not without its being, and vice versa,
nor intellectual knowing without intuition and vice versa,
which are to be understood so that the disjunction
that lies within them must become one
in the oneness of the insight.”
Recall that we have concerned ourselves with this
insight for a long time,
and that it has returned under various guises
and in various relations, but always according to synthetic rules.
Certainly it could not happen any differently in this case,
and it was this something, surrendering itself even then,
which I meant when I spoke previously about the manifold shifts
and modifications of the one selfsame process.
The difference from before
[and it seems to me also an advantage of the present path]
is this:
that already from the start,
even before we plunge
into the labyrinth of appearance,
we can recognize our various future observations
in their spiritual oneness.
It is to be hoped
(and this hope does not really concern
my own knowledge and procedure in lecturing
but rather the capacity of this audience
to follow the presentation)
that an ordering principle for these various shifts
will soon be available,
by means of which the process will be further facilitated.
And so it will not be difficult for this part of the audience
to recognize in what is now expressed in a particular way
what was said in a different way before and vice versa.
In being liberated from my two different literal presentations
they may free themselves from any literal presentation,
which would mean nothing and would be better not existing
if it were possible to hold a lecture without one.
And in freeing themselves,
[they] build realization for themselves in their own spirit,
free from any formulas and with independent control in every direction.

Let me add the following while we still have time,
although it is not essential and has relevance
only to the smallest circle of those gathered here:
except for the fact that one possesses
as one's own genuine property only that which one possesses
independently of the form in which one received it,
one can intentionally present it afresh
and share it only under such conditions.
Only what is received living in the moment,
or not far removed from it, strikes living minds;
not those forms which have been deadened by
being passed from hand to hand or by a long interval.
If I had needed to hold these lectures
on the science of knowing immediately
after the previous series to the same audience,
who had all known the science for a long time
no need for further instruction in it,
and who simply wanted to prepare themselves
further for their own oral presentation of philosophy,
yet I still believe that I would have been required to
take almost as diverging a path as I have taken this time,
and I would have had to advise these future
teachers of philosophy about the utility of this divergence
in just about the same way I have advised you,
for whom it is relevant.

Tenth Lecture
Thursday, May 3, 1804
Honored Guests:

Our next task is now clearly determined:
to see into L (light)
as the genetic principle of C (concept)
and vice versa, and thus to find
the oneness and disjunction of the two.

(Let me add yet another parenthetical remark.
Who among you, prior to studying the science of knowing,
has known L or C, not in general and with confusion
[since any sort of philosophy
distinguishes an immediate presentation
of an actual object and a concept,
which is usually an abstract one]
but true L and C in the purity and simplicity
with which they have been presented here?
Our task concerns itself with doing this;
and, with the resolution of this task,
the science of knowing is completed
in its essentials.
Accordingly, the science answers
a question that it itself must pose,
and dissolves a doubt that it has first raised.
It should not seem strange to anyone that
there is no bridge to it from the usual point of view
and that one must first learn everything
about it from within it.)
Something has happened for the resolution
of this task on Monday,
which we will now briefly review
to confirm our grasp of it.

As factical manifestness makes clear,
light plainly arises in a dual relation:
in part as inwardly living
[and through this inner life of its own it must
divide itself into concept and being]
and in part in an external insight,
which is freely created
and which objectifies this light
along with its inner life.
Let us take up the first.
What makes this inward life inward?
Obviously that it is not external.
But it becomes external in [being seen by] the insight.
Thus, what follows immediately and is synonymous:
it is an [inward] life because in this regard it is
outside any insight, is inaccessible to it, and negates it.
Therefore light's absolute, inner life is posited;
it exists only in living itself and not otherwise;
therefore it can be encountered only immediately
in living and nowhere else.
I said that the genuinely, truly real in knowing rests here.
But we ourselves have just now spoken of this inner life
and therefore in someway conceived it.
Yes, but how? As absolutely inaccessible to insight;
so we have conceived and determined it only negatively.
It is not conceivable in any other way.
The concept of reality, of the inner material content of knowing, etc.,
which we have introduced is only the negation of insight
and arises only from it;
and this should not just be honestly admitted,
rather, a philosophy which truly understands its own advantage
should carefully enjoin this idea.
In truth it is no negation,
but rather the highest affirmation,
which indeed is once again a concept;
but in truth we in no way conceive it,
but rather we have it and are it.

Let this be completed and determined by us right now,
and in this act it will have its application uninterruptedly.
And don't let the truly crucial
point of the matter escape:
[there are] two ways
for the light to live absolutely:
internally and externally;
externally in the insight,
internally, therefore, absolutely not in the insight,
and not for it, but instead turning it away.
By this means, our system is protected against
the greatest offense with which
one can charge a philosophical system,
and without exception nearly always justly:
namely vacuity.
Reality, as genuine true reality, has been deduced.
No one will confuse this reality with being (objectivity);
the latter is subsistence-for-self and dependence-on-self
which is closed in on itself and therefore dead.
The former exists only in living,
and living exists only in it;
it can do nothing else than live.
Therefore, because our system has taken life
itself as its root,
it is secured against death,
which in the end grasps every [other]
system without exception somewhere in its root.
Finally, we have seen and enjoined that,
since light and life too are absolutely one,
this reality
[and the insight,
through the negation of which
it becomes reality,]
can altogether be only one
and eternally self-same.
Thereby our system has won enduring oneness
and has secured itself from the charge that
there may still be some duality in its root.

Insight, I assert, is completely negated in living light.
But then we see, and see into, the disjunction into C and B.
Therefore, this disjunction, which we previously ascribed
to the inner light itself,
should not be ascribed to this,
but instead to the insight that takes its place,
or to the original concept of light.
The concept reaches higher,
the true light withdraws itself.
The absolute negation of the concept may well
remain a nothing for the science of knowing,
which has its essence in concepts,
and only become an affirmation in living.
With this, two further comments which belong
to philosophy's art and method.

1. Here we retract an error in which we have so far hovered.
How did we arrive at this error,
or at the proposition that
we now take back as erroneous?
Let us recall the process.
Driven by a mechanically applied law of reason
(therefore, factically),
we realized immediately that it [the absolute]
cannot reside in A (the oneness of B and T)
nor in the disjunction point,
but rather in the oneness of both;
this was the first step.
Then, as the second step,
we raised ourselves to the apprehension
of the general law for this event,
which naturally we could apprehend only as follows:
in an immediately self-presenting insight,
a disjunction is negated as intrinsically valid
and a oneness, which cannot be any more exactly described,
is absolutely posited.
What then did we finally do?
In fact we did nothing new,
apart from the fact that we relinquished
the specificity of the disjunctive terms “A” and “.”
and likewise the specificity of their unity,
and posited disjunction,
and also self-sufficient oneness,
generally and unconditionally,
in which case the possibility of the procedure
could arouse wonder and give rise to a question.
Besides this, I say, we simply grasped
the rules of the event historically,
always led by the event,
and, if that were removed from us,
[we] lacked all support for our assertion.
Therefore, although this, our second insight,
appears to possess a certain genetic character
in the first mentioned part,
in the second part it is something merely factical;
and so what we advanced yesterday as the
ground for the uninterrupted connection
between the disjunctive terms is
confirmed here at a genuinely central point:
that our whole insight might not yet be purely genetic,
but is instead still partially factical.
To get back to the point;
this insight, arising in concrete cases
and led to its general rule in the second step,
we now named pure, absolute light,
simply in this respect,
that in terms of content it arises immediately,
without any premises or conditions.
But in its form it remains factical and is dependent
on the prior completion in concrete cases.
We might have inferred from the following
that it cannot possibly have its application here:
although dividing the concept
into “A” and “.” was given up as inadequate,
there yet remained a new disjunction
in what was taken as absolute,
since it was simultaneously negating and positing,
the former through its formal being,
and the latter through its essence.
But no disjunction can be absolute and merely factical,
rather, as surely as it is a disjunction,
it must become genetic,
since disjunction is genetic in its root.
(Remarks of the kind just made bring
no progress in the subject matter,
but they elevate the freedom of
self-possession and reproduction for everyone
and facilitate the comprehension of what follows.)
Result: since our initial supposition
grounded itself partially on factical insight,
we must give it up.

Further, how then have we arrived at
this insight of giving something up,
as well as at the higher [term]
for which we give it up?
If you recall, we did so by means
(i) of the distinction between two ways
the light exists and lives: inwardly and outwardly,
a distinction admittedly given only with factical manifestness;
(ii) by genetic insight into this distinction and by the question
how something absolutely inward might arise as inward;
and (iii) by elevating into a genetic perspective
something that had been thought previously
only in faded, factical terms.
Moreover, I admitted, as is indeed evident
and as everyone will remember,
that this entire disjunction between inner and outer
arises only in a factical point of view.
Here, too, this observation:
that in our present investigation as well,
the facticity, which is erased on one side,
pops up again on the other,
and that we will not be entirely and purely relieved of it
until the present task is completed.
(For returning listeners this as well:
the distinction drawn here
between the light's inner and outer life is
the same as the distinction
between immanent and emanent forms of existence
that was so important and meaningful in the previous series.)

Second remark: C and L are both only concepts:
the first is purely disjunction in general,
a disjunction which can give no further account of itself;
from our current point of view, it has simply two terms which are
not further distinguishable.
L, on the other hand, is not a disjunction in general,
but is rather the specific disjunction into being and concept.
The latter, as the principle of disjunction in general,
consequently has enduring inner content,
as does B [Being], the principle of oneness.
Therefore, the terms of this second disjunction are not
just two terms in general, they also have an internal difference.
From our standpoint, therefore, L is still by no means negated,
nor can it be from that perspective.
If nevertheless it must be negated,
as it evidently must be a priori,
since otherwise it could not come to the zero state
in which no further disjunction truly remains,
then completely different means will have to be employed
from the ones now available to us.

Now to characterize (in relation to our task)
the point I have just repeated,
a point which fits our system in every aspect,
of which I said last time that
it already belonged to our process,
and which we need to apply in solving
our next and primary task:
C and L are to be reduced to oneness,
just as they were before this point of ours;
and this will have to be done so that C is
so rigorously mastered that we see into it
as the genetic principle of light, and vice versa.
With which of the terms to begin is left either to our caprice or
our philosophical skill,
which is unable to give any account of its maxims before applying them.
In the previously discussed point,
L was taken up as a starting point,
as things stood then;
if so, the concept proceeds genetically from this L,
since L has transformed itself into the concept.
Or said more exactly:
our own observing
—which was not then visible, which we
lived, and into which we merged—
divided itself beyond the then regnant L,
and in this division negated the L (light) into 0/C;
thus creating both out of itself.
Now note well that this shift in viewpoint is
in no way merely a change of the word and the sign,
but rather that it truly is a real change;
because what stood here previously,
whether called L or C, light or concept,
was to be the absolute (which is a real predicate)
and should divide itself into C and B
(which is also a real predicate).
Both predicates combine to form
a synthetic sentence that determines the absolute.
In its essence
[entirely apart from the expressions and signs
in which one realizes and presents this essence]
this sentence is contradicted
by the really opposite sentence:
the principle of the disjunction
into being and thinking is not absolute
but something subordinated
(however one may more exactly name and signify
this subordinate something);
in the absolute the two are not distinguished.
From this, another correction must follow first and foremost,
not so much regarding our viewpoint as our manner of expression.
There were supposed to be two different distinguishing grounds,
which of course are to be reunited again,
but which would have been held sufficiently
far apart from one another by two basic principles
that were as distinct as until now
light in itself and its representative concept have been.
Now all disjunction collapses into one and the same concept,
and hence this latter could very easily provide
the one eternally self-same disjunctive factor,
which does not appear in the original appearance,
but rather appears as doubled in the secondary appearance,
in the appearance as appearance.

But let me go back.
As things stood previously,
the spirit of our task was to realize
L as the genetic principle of C and vice versa.
We have tried by beginning with L:
the attempt had its narrowly circumscribed result
and the matter does not stand as it did before
but as the schema instructs.
The spirit of the task remains the same
through all shifts in perspective,
just because it is spirit:
L through C and vice versa.
Our true L is now = 0,
and it is clear that
this cannot be approached more closely:
it negates all insight.
Therefore, this first path is
already completed on the initial attempt.
Nothing more remains besides
taking up C and testing whether
through it we can further determine
[not 0, since this remains purely
unchangeable and indeterminable]
but rather it as the truly highest term
that we now are and live.
Thus, a new classificatory division,
the determination on the basis of C,
becomes the second principal part
of our present work.

Let me now give some preliminary hints
and thereby prepare you for tomorrow's lecture,
setting out a rough outline for you.

The concept's inward and completely immutable essence has
already been acknowledged in an earlier lecture as a “through.”
Although in its content this insight is in no way factical
but is instead a purely intellectual object,
still it has an factical support:
the construction of the image and the thing imaged,
and the indifference of the inference between them.
However, we would be permitted to use this basic quality of the concept,
if only, in this application, we succeeded in negating its factical origin.
If one embraces a “through” just a little energetically,
it can readily be seen that the same principle is a disjunction.
Except the same question must always be repeated which
already arose previously on the same occasion:
how should a dead “through,” defined as we have defined it,
come to life
(despite all the capability with
which it is prepared to meet life,
especially by means of the “throughness”,
or the transition that it makes from one to another,
if only it is brought into play)
because it has no basis in itself
for coming to actualization?
How would it be if the internal life
of the absolute light (= 0) were its life,
and therefore the “through” was itself
first of all deducible from the light by this syllogism:

i. If there is to be an expression
[an outward existence of the immanent lifas such]
then this is possible only with an absolutely existent “through.”
ii. But there must be such an expression.
iii. Hence, the absolute “through” (the original concept, or reason)
exists absolutely as everyone can easily see for himself.

Further, how would it be if just this living “through”
(living to be sure by an alien life, but still living)
as the oneness of the “through,”
divided itself into thinking and being,
in itself, and in the origin of its life?
This division, as a division of the enduring “through” as such,
would be comprehensive for the same reason,
and inseparable from it and its life.
How would it be if it did not remain trapped in this,
its essence as “through,” but rather might itself
be objectifying and deducing the latter;
it would hence certainly have to be able
to do just as we ourselves have done
[the objectification and deduction
can themselves come about according to the law of the “through,”
since fundamentally and at base it is nothing but a “through”:
how would it be if in this objectification and deduction
it split itself again in the second way?
[Further, how would it be,
since a “through” clearly can exist
only by means of a “through”,
that is, its own being as a “through” can be only mediate—
and the first mode of being
would scarcely be possible
without at least a little of the second,
if the first division could not be
without some of the second, and vice versa?
Since this “through” is our own inner essence,
and a “through” dissolves completely into another “through,”
then absolutely everything based in these terms
must be completely conceivable and deducible.]
In all of these “how would it be if ____” clauses,
I have consistently regarded “0” as life;
but it is not merely this,
rather it is something indivisibly joined with life,
a thing we grasp by the purely negative concept of reality.
If it is indivisible from life,
if life lives in a “through,”
then it lives as absolute reality,
but since it is a “through,”
it lives it in a “through”
and as a “through.”
Now, consider what follows
if the one absolute reality,
which can only be lived immediately,
occurs in the form of an absolute “through.”
I should think this:
that it cannot be grasped anywhere,
unless an antecedent arises
for what has been grasped,
through which it is to be;
and, since it is grasped only as a “through,”
it must also have a consequence
that is to follow from it.
This must follow unavoidably
by absolutely every act
of apprehending reality.
In short, the infinite divisibility in absolute continuity;
in a word, what the science of knowing calls quantifiability
as the inseparable form of reality's appearance
arises as the basic phenomenon of all knowing.

In this last brief paragraph of my talk
I have pulled together the entire content
of the science of knowing.
Whoever has grasped it
and who can see it as necessary
[the premises and conditions for this
manifestness have already been completely laid out]
such a one can learn nothing more here,
and he can only clarify analytically
what has already been seen into.
Whoever has not yet seen into it has
at least been well prepared for what is to follow.
For the one as for the other,
we will move forward tomorrow.

Eleventh Lecture
Friday, May 4, 1804
Honored Guests:

Yesterday, I succeeded in presenting the essence
and entire contents of the science of knowing
in a few brief strokes.
Losing time at the right place means gaining it;
therefore, against my initial purpose,
I will apply today's hour to presenting
further observations about this brief sketch.
The more certain we are in advance about the form,
the easier the actual working out
of the contents within this form will be.

C = “through,” in which resides disjunction.
“If only this “through” could be brought to life,” I said;
it has nearly all the natural tendencies of life,
nevertheless in itself it is only death.
It would be useful to reflect further about this expression,
since the “through” can be more clearly understood in it,
than it has yet been understood:
this “through,” which according to the preceding
represents the central point in our entire investigation.
Indeed, it is immediately clear
what it means to say:
“a 'through' actually arises,”
“a 'through' has taken place,”
or “there is an existing 'through.'”
I further believe it will be clear
to anyone who considers the possibility of this existence that,
considered formally, something else belongs to it
besides the pure “through.”
In the “through” we find only
the bare formal duality of the terms;
if this is to find completion,
then it needs a transition from one to the other,
thus it needs a living oneness for the duality.
From this it is clear that life as life cannot lie in the “through,”
although the form which life assumes here,
as a transition from one to the other,
does lie in the “through”:
so life generally comes entirely from itself
and cannot be derived from death.

Result: the existence of a “through”
presupposes an original life,
grounded not in the “through”
but entirely in itself.

We see into this at once:
but what is contained in this insight?
Evidently the insight formed in positing
the “through's” existence
(and the question about the possibility of this existence)
brings life with it,
that is in the image and concept.
Therefore, in this insight life is grasped
in the form of a “through,” only mediately.
The explanation of the “through” is itself a “through.”
The first of these posits its terms at one stroke;
and, in the resulting insight,
it is itself posited as positing them in one stroke
by the explanatory “through” (horizontally arranged):
a
——
a × b
So too, in the same way, with regard to
the inner meaning, sense, or content,
the first “through” does not posit
its terms at one stroke;
rather, life should be the condition
and the “through's” existence should be
what is conditioned;
thus it [life] should be in the concept as a concept
[in truth and in itself] the antecedent,
and the latter should be the consequent.
[This is the] perpendicular arrangement.
Both obviously [exist] only in connection with
each other, and [are] only distinguishable in this context.

The concept remains the focus of everything.
(To reconstruct: here in a certain respect to preconstruct.)
[The concept] constructs a living “through,”
and, to be sure, does so hypothetically.
Should this latter be,
then the existence of living follows from it.
It is immediately clear that a hypothetical “should” is
not grounded on any existing thing,
but is rather purely in the concept,
and collapses if the concept collapses;
and that, therefore, the concept announces
itself in this “should” as pure,
as existing in itself,
and as creator and sustainer
from itself, of itself, and through itself.
The “should” is just the immediate
expression of its independence;
but if its inner form and essence are independent,
then so are its contents as well.
Hence, the existence of a “through”
announces itself here as completely absolute and a priori,
in no way grounded again on another real existence which precedes it.
Therefore, the concept is here the antecedent
and absolute prius to the hypothetical positing
of the “through's” existence:
the latter is only the concept's expression,
something which depends on it
and through which it, as concept, preserves itself
as an absolutely inward “through.”
Which was the first [point].

In this, its vivacity,
the concept changes itself into an insight,
which, unconditioned, produces itself
insight into a life in and of itself,
which must necessarily be presupposed.
Ascending, I can therefore say:
the absolute concept is the principle of the insight,
or intuition, into life in itself,
that is to say into [life] in intuition.
It seems quite possible, namely in a shallow and faded way,
to think the existence of a “through” without any insight
arising into the life that it absolutely presupposes.
For the latter to happen, this existence must
be conceived with full energy and vivacity:
Now, I say, (as is clear right away):
in this pale imitation the “through” is
not really thought as it must be thought:
that is, as a genetic principle.
For if it were thought in that way,
then what ought to be manifest would be manifest.
So the true focus, the genuine ideal prius is
no longer just the concept,
but rather the inward life
whose outcome (posterius) is the concept.
And the “should” is not, as I said before, the highest
exponent of the independence of reason,
rather the appearance of inner energy is this highest exponent.
(If I ask you to think energetically,
I am really demanding that you be fundamentally rational!)
The hypothetical “should” is again an exponent of this exponent,
and the concept is not, as I said initially,
the principle of intuition,
but rather the inner immediate life of reason,
merely existing and never appearing,
which appears as energy
(energy which obviously is again
the expression of a “through” immanent in itself).
This inner life, I say, is the principle
of concept and intuition at once
and in the same stroke:
thus it is the absolute principle of everything.
This, I say, would be the idealistic argument.

Once we have proceeded in this way,
let us climb higher in order to get
to know the real spirit and root
of this mode of argument.
Without further ado,
it is obvious that we could have expressed our entire procedure thus:
there may be the intuition of an original and absolute life,
but how and from what does it come to be?
Just construct this being as I have,
or grasp it in its becoming;
now this has happened actually and in fact,
and the inner life of reason as a living “through”
has been deduced as the genetic principle for this being.
Thus, the basic character of the ideal perspective is
that it originates from the presupposition of
a being which is only hypothetical
and therefore based wholly on itself;
and it is very natural that it finds just this same being,
which it presupposes as absolute,
to be absolute again in its genetic deduction,
since it certainly does not begin there
in order to negate itself,
but to produce itself genetically.
Thus, the maxim of the form of outward existence is
the principle and characteristic spirit
of the idealistic perspective.
By its means, reason, which we already know very well
as a living “through,” becomes the absolute;
it becomes this in the genetic process
because it already exists absolutely
as the constant presupposition.
Absolute reason, as absolute, is therefore a “through” =
the form of outward existence.
Prior and absolute being shows itself to be
inwardly static, motionless, and dead
at just this [point of going] “through,”
where it always remains;
the inward life of reason,
which we have already established,
shows itself in this being's hypothetical quality.
One need only add now what is implicitly clear:
that this idealistic perspective does not arise
purely in the genetic process,
since it assumes a being as given,
and that therefore it is not
the science of knowing's true standpoint.
This is also clear for another reason:
in the idealistic perspective, reason exists,
or lives, as absolute reason.
But it lives only as absolute
(in the image of this “as”);
hence it does not live absolutely;
its life or its absoluteness is itself mediated by
a higher “through,”
so that in this standpoint it is only derivative.
So much by  way of a sharp, penetrating
critique of the idealistic perspective.
This is especially important,
because beginners are easily tempted to remain
one-sidedly trapped in this point of view,
since it is the perspective in which
their speculative power first develops.

Now let us turn things around
and grasp them from the other side.
If the “through” is to come to existence,
then an absolute life, grounded in itself,
is likewise presupposed.
Therefore, this life is the true absolute
and all being originates in it.
With this, intuition itself is obviously negated,
though not indeed as empirically given,
since if we simply try to remain in an energy state
and to consider nothing further,
then we will always find that we still grasp
[this state] in intuition.
Let me mention in passing that
this is idealism's stubbornness:
not to let one go further,
once one has finally arrived at it.
Faced with this, since in any case
idealism is something absolute,
it does not allow itself to be explained away
by any machinations of reasoning,
but rather yields only to the arrival
of what is primordially absolute.
Among other things, this idealistic stubbornness too has been
attributed to the fantasy of the science of knowing
which circulates among the German public,
disregarding the fact that of course
people cannot speak clearly about this charge
because they do not know the genuine science;
e.g., Reinhold did this for his whole career.
It is like this:
the non-philosopher or half-philosopher forgets himself,
or the absolute intuition, either because he never knew it
or, if he knew it, he periodically forgets it again.
The one-sided idealist who knows it and holds it fast
does not let it develop, because he knows nothing else.
To return:
by recognizing the absolute immanent life
we negate intuition as something that is genetically explicable
and [that plays a part] in a system of purely genetic knowledge.
Because if the immanent life is self-enclosed,
and all reality whatsoever is encompassed in it;
then not only can one not see how to achieve
an objectifying and expressive intuition of it,
one can even see that such an intuition could never arise;
and, because of its facticity, just this last insight
cannot itself be comprehended anew,
but simply directly carried out;
it is the absolute, self-originating insight.
So, however stubbornly one might hold on
to his immediate consciousness of this intuition,
it does not help things;
no one challenges this intuition in its facticity.
What has been asserted and demonstrated is just this:
not merely that [this consciousness] is inconceivable,
but even that it can be conceived to be impossible.
Thus, the truth of what it asserts is denied,
but in no way its bare appearance.
Let me note in passing that the place
for denying ourselves at the root is here,
just in the intuition of the absolute,
which of course might very well be our root,
and which up to now has played that role.
Whoever perishes here will not expect
any restoration at some relative, finite, and limited place.
But we do not achieve this annulment
by an absence of thought and energy,
as happens in other cases.
Instead, [we do so] through the highest thinking,
the thought of the absolute immanent life,
and through devotion to the maxims of reason,
of genesis or of the absolute “through,”
which denies its applicability here
and thus denies itself through itself.
Everything has dissolved into the one = 0.

Reasoning which is conducted and characterized
in these terms is realistic.
There is no progression or multiplicity
in it except for pure oneness.
[Let me relocate you in the context.
The two highest disjunctive terms stand
here absolutely opposed to one another,
life's inner and outer life,
the forms of immanent and emanent existence as well,
separated by an impassable gulf
and by truly realized contradiction.
If one wishes to think of them as united,
then they are united exactly
by this gulf and this contradiction.]

As we did before with the idealistic perspective,
let us now discover the inner spirit and character
of the realistic perspective just laid out.
Obviously, this whole perspective takes
its departure from the maxim:
do not reflect on the factical self-givenness
of our thinking and insight,
or on how this occurs in mind,
rather reckon only the content of this insight as valid.
Thus, in other words:
do not pay attention to the external form
of thought's existence in ourselves,
but only the inner form of that thought.
We posit an absolute truth,
which manifests itself as the content of thinking,
and it alone can be true.
As before, it happens for us as we presupposed and wished;
since the inner content alone should be valid,
so in fact it alone really matters,
and it negates what it does not contain.
Made genetic by us, it was just so.
So much in general.

Now allow us to delineate realism's presupposition
of an inner absolute truth more precisely.
I believe that there is no way to give a better description,
such as is indeed needed, than this:
this implicit truth appears as an image,
living, completely determined, and immutable,
which holds and bears itself in this immutability.
Now this implicit truth reveals itself in absolute life,
and it is immediately evident that it can only
reveal itself in the latter.
Because life is just as truth is:
the self-grounded, held and sustained by itself.
Truth is, therefore, in and through itself only life's image,
and likewise only an image of life gives truth,
just as we have described it.
By means of the truth, as grounded by itself,
only the image is added.
So we stand at about the
same point as before,
between life itself and the image of life;
as regards this, we saw that they are
fully identical in terms of content,
which alone matters in realism,
and are different only in form,
which realism leaves to one side.

Living
—————————
Concept Image Being

Now it is noteworthy that the image
(which holds and sustains itself)
should exist only in the truth as truth,
when the former seems, according to its character,
to be exactly the same as thought
and the latter the same as being;
and that therefore in realism
(and working from it, if we simply compel it
to clarify its fundamental assumption)
we are led to a perspective,
which is so similar to idealism
that it could even be the same.

Without venturing further here into this last hint,
which meanwhile could be tossed in
to direct our attention to what follows,
I will simply conclude today's lecture
and pull it together into a whole
by means of the following observation.
Both idealism and realism grounded themselves on assumptions.
Both of these assumptions
[that is in their facticity and in the circumstance
that one actually arises here and the other there]
grounded themselves on an inner maxim of the thinking subject.
Hence, both rest on an empirical root.
This is less remarkable in the case of idealism,
which asserts facticity, than of realism,
which, in its effects and contents,
denies and contradicts what it itself fundamentally is.
As we have seen, both are equally possible,
and, if only one grants their premise,
they are equally consistent in their development.
Each contradicts the other in the same way:
absolute idealism denies the possibility of realism,
and realism denies the possibility of
being's conceivability and derivability.
It is clear that this conflict,
as a conflict in maxims, can be alleviated
only by setting out a law of maxims,
and that therefore we need to search out such a law.

We can get a rough idea in advance about
how a settlement to the conflict will work out.
All the expressions of the science of knowing so far
show a predilection for the realistic perspective.
The justifiability of this preference follows
from this, among other things:
that idealism renders impossible
even the being of its opposite,
and thus it is decidedly one-sided.
On the other hand, realism at least leaves
the being of its opposite undisputed.
It only makes it into an inconceivable being,
and thereby brings into the light of day
its inadequacy as the principle for a science
in which everything must be conceived genetically.
Perhaps a simple misunderstanding underlies the proof,
given earlier in the name of realism,
that an expressive intuition of
absolute life can in no wise arise.
In that case, what is proven
and needs to be asserted is only that
such an intuition,
as valid for itself and self-supporting,
can never arise.
This assertion very conveniently
leaves room for an interpolation:
this intuition might well arise,
and must arise under certain conditions,
simply as a phenomenon not grounded in itself.
Insight into this interpolation could thus provide
the standpoint for the science of knowing
and the true unification of idealism and realism;
so that the very intuition, purely as such,
which we previously called “our selves at root,”
would be the first appearance
and the ground of all other appearances;
and because this would not be any error
but instead genuine truth,
it and all its modifications,
which must also be intuited as necessary,
would be valid as appearances.
On the other side, however, seeming and error
enter where appearance is taken for being itself.
This seeming and error arise necessarily from [truth's] absence,
and hence can themselves be derived
as necessary in their basis and form,
from the assumption that this absence itself is necessary.
Some have either discovered or thought they discovered,
I know not which, that in measuring the brow
they could measure people's mental capacity
on the basis of their skulls.
The science of knowing could easily claim
to possess a similar measure of inner mental capacity,
if only it could be applied.
In every case the rule is this:
tell me exactly what you do not know
and do not understand,
and I will list with total precision
all errors and illusions in which you believe,
and it will prove correct.
