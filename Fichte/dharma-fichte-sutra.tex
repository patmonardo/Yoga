# REFLECTION

IV.1
jati-antara-parinama prakrti-apurat

The fundamental principle presented now:

Being is entirely a self-enclosed singularity of
immediately living being that can never get outside itself.

    This principle contains and completes
    what one could present as the first part
    of the science of knowing:
    the pure theory of truth or reason.
    We proceed now to the second part;
    in order to deduce from the first part,
    as necessary and true appearances,
    everything which up to now we have
    let go as merely empirical
    and not intrinsically valid.

IV.2
nimittam aprayojakam prakrtinam varana-bhedas tu tata ksetrikavat

IV.3
nirmana-cittani-asmita-matra

IV.4
pravrtti-bhede prayojakam cittam ekam anekesam

IV.5
tatra dhyana-jam anasayam

IV.6
karma-asukla-akrsnam yogina trividham itaresam

IV.7
tatas tad-vipaka-anugunanam eva-abhivyakti vasananam

IV.8
jati-desa-kala vyavahitanam apyanantaryam smrti-samskarayo eka-rupatvat

IV.9
tasam anaditvam ca-asiso nityatvat

IV.10
hetu-phala-asraya-alambana samgehitatvad esam abhave tad-abhava

# APPEARANCE

IV.11
atita-anagatam svarupato ‘styadhva-bhedad dharmanam

IV.12
te vyaktasuksma guna-atmana

IV.13
parinama-ekatvad vastu-tattvam

IV.14
vastu-samye citta-bhedat tayor vibhakta pantha

IV.15
na caika-citta-tantram vastu tad apramanakam tada kim syat

IV.16
tad-uparaga-apeksitvat-cittasya vastu jnata-ajnatam

IV.17
sada jnata citta-vrttaya tat-prabho purusasya-aparinamitvat

IV.18
na tat sva-abhasam drsyatvat

IV.19
eka-samaye ca-ubhaya-anavadharanam

IV.20
citta-antara-drsye buddhi-buddher atiprasanga smrti-sankara ca

# ACTUALITY

IV.21
citer apratisamkramayas tad-akara-apattau svabuddhi-samvedanam

Whether one names the absolute “being” or “light,”
it has already been completely familiar for several weeks.
Since attaining this familiarity, we are working on
deriving not, as is obvious, the thing itself,
but rather its appearance.
The request for this derivation can mean nothing else than
that something still undiscovered remains in the absolute itself,
through which it coheres with its appearance.

We know from the foregoing
(which, to be sure has been discovered only factically,
but which nonetheless would have its application
in a purely genetic derivation)
that the principle of appearance is
a principle of disjunction
within the aforementioned undivided oneness
and at the same time, obviously, within appearance.
However, as regards the absolute disjunction,
I urge you to recall an analysis conducted right at
the beginning of this lecture series,
in which the following became evident.
If disjunction were to be found
directly within absolute oneness,
as is unavoidably required by
the final form of the science of knowing,
and is what we intend here,
it must not be grasped as a simple disjunction,
but rather as the disjunction of
two different disjunctive foundations.
[It must be] not just a division,
but rather the self-intersecting division
of a presupposed division that again presupposes itself.
Or, using the expression with which we have designated it
in our most recent mention of it,
[it is] no simple “from,”
but rather a “from” in a “from,”
or a “from” of a “from.”
The most difficult part of the philosophical art is
avoiding confusion about this intersection,
and distinguishing that which is endlessly similar
and is distinguishable only through the subtlest mental distinguishing.
I remind you of this so that you will not become mistrustful
if, in what follows, we enter regions
in which you no longer understand the method
and [in which] it should even seem miraculous.
Afterwards we will give an account of it;
but beforehand we actually cannot.

IV.22
drastr-drsya-uparaktam cittam sarva-artham

This much as a general introduction for the week:
Now back to the point at which we stood at the end of the last hour.
Absolute self–genesis posited and given a principle,
obviously within knowing,
which is thus a “principle-providing” [occurrence].
Thus within this knowing there follows
the absolute, positive negation of genesis =
completed and enduring being;
and indeed, because this entire investigation
concerns light’s pure immanence,
our investigation has long ago brought in
a presumed being external to knowing,
knowing’s [own] completed and enduring being.

Now We saw into this connection during the last hour
and see into it again here;
as is evident, we see one of the two terms
in and through the insight into this connection,
determined as such.
Thus, this latter is itself mediated,
and just we (= the insight we have now achieved) are
therefore the unconditionally immediate [term].

Two remarks about this.
1. I have just recalled again,
that here the inner being and persevering is knowing’s being,
the very thing already recognized as absolute genesis
and [the thing] we have also already validated in the last hour
as a priori rational knowledge of an absolute principle.
At this point, we must hold on to
the fact that it is knowing’s being,
even in case we should let go of
the addition as a shortcut in speaking;
because otherwise we will fall back into
where we were before, far removed from further progress.
Therefore, our entire chain of reasoning
must always be present to us, now more than ever.

2. It is said that every philosophical system
remains stuck somewhere in dead being and enduring.
If now a system derived this being itself in its inner essence,
as ours has done, by positing a higher principle for absolute genesis,
whereby it then necessarily becomes the positive negation of genesis,
and therefore [becomes] being;
if too this being is not the being of an object and so doubly dead,
but rather the being of knowing, and so of inner life;
then such a system seems to have already
accomplished something unheard-of.
However, we are required to see clearly here that
by this we have not yet achieved anything,
since even this saturated being of living is
again something mediate and is derived from that
which alone now remains for us,
the insight into the connection.

Now let us apply this directly for our true purpose,
which we have long recognized.
Knowing’s derived being will now yield
ordinary, non-transcendental, knowing.
Through our present insight into
the genesis of the former’s principle
(that is, the principle of the recently derived knowing),
and through reflecting on this insight
we elevate ourselves to genuine transcendental knowing
or the science of knowing;
and [we have done so] not merely factically,
with our factical selves,
so that we are the factical root.
We have already been this since
the time when we dissolved into pure light;
instead [we have done so] objectively and intelligibly,
so that we, achieving insight factically,
at the same time penetrate the law of this insight.
Henceforth we have to work in the higher region
that has now been opened.
Only here will the principle of
appearance and disjunction
that we seek show itself to us;
which then should only be applied
to existing (= ordinary) actual knowing.

IV.23
tad asamkhyeya-vasanabhi citram api para-artham samhatya-karitvat

1. Now also this addition:
since the beginning of what we
have provisionally called the second half,
a hypothetical “should” has been evident
as simply creating a connection,
and as linking a conditioning and conditioned term
(which are both produced absolutely from it)
to a knowing, which must be originally present
independently of the “should”
and its entire operation,
if one just understands it correctly.
This could be called the first
sub-section of the second part.
Since we concerned ourselves with
an absolute presupposition about
the essence of knowing as an absolute “from,”
we wished to know nothing more about
this entire hypothetical “should”
and its power of uniting and joining,
as a merely apparent knowing.
We said at this point that so far we
(the we that still has not been grasped so far)
have concurred about the premise’s arbitrary positing,
pointed out by energetic reflection,
and only the connection has made itself manifest
without our assistance.
Now the premise too presents itself without our assistance;
therefore in the premise too we coincide
with the absolutely self-active light.
Let’s hold on to this.
We have been doing this for a long time
in our discussions about the “from,”
until I thought you were sufficiently prepared for
the higher flight that we began in the last hour.
You may take this as the second sub-section of the second part.
In the last hour, the bare connection presented itself again,
and (as we might suspect, but will more exactly show and demonstrate)
with it the hypothetical “should,”
from which we had already hoped to be free.
This should surprise us.
If this “should” has reappeared with the same significance
in which it was previously struck down,
then we have not advanced
and are just sailing on
in the seas of speculation without a compass.
Through the hint given recently as to
the difference between ordinary knowing
(based on the principle of knowing’s being)
and transcendental [knowing]
(based in the genetic insight into this same principle),
it is probable that [the “should”] does
not arise in the same way.
Instead, the “should” that we have dropped
operates in ordinary knowing
with tacitly assumed premises;
in contrast, the “should” arising now
operates in transcendental knowing,
which grounds its premises genetically
as emanating from a “should.”
Therefore, in the preceding lecture,
we have begun a third sub-section of the second part,
and the two extreme sections come together
in the middle (with the premise) once again
distinguished by a duality in the premise.
By this means, the two outermost parts
(= transcendental and actually existent knowing)
would be the two different distinguishing grounds,
proceeding from the middle ground of the premise,
which both unites and separates them.
This is the compass that I would share with you
for the journey we have already begun.

2. I said that a hypothetical “should” appears
again in our completed insight.
To begin with, this is obvious.
“Given [that there is] a principle
for self-genesis, then it follows that ____ .”
Previously, we have found the two terms factically,
and to that extent separately;
but in the last hour, we have united them genetically,
according to our basic rules and maxims.
Because we comprehend them mediately
in the insight into their connection,
then, given this insight,
we no longer need to assume them factically.
They reside a priori in the insight,
and we can drop the empirical construction,
until perhaps it arises again in some deduction.

3. Now let us grasp this hypothetical character at its core.
According to the maxims and rules that we arbitrarily adopted
at the beginning of our entire science, and hence arbitrarily,
we appear to ourselves
(so it has been, and so it now is openly admitted)
as genetically uniting both terms.
Here is the inner root of hypotheticalness
(now fully abstracted from the hypothetical
character of the subordinate terms),
precisely the “should’s” admitted
inner production, containment, and support of itself
as identical with the free We, the science of knowing.
This very inward hypotheticalness could be
what first shows itself and breaks through
in the subordinate terms’ hypothetical character.
Hence, it is only a matter of
negating this inner hypotheticalness,
so that thereby the quality of being categorical
will be manifest in it,
and thereby we will justify our insight
in its truth, necessity, and absolute priority.
In that case, the inference,
made here only provisionally,
first achieves categorical validity;
namely, the inference that both terms
(knowing’s self-genesis and its being)
occur only mediately in a genetic insight
into the oneness of both,
and in no wise immediately.

(A remark belonging to method:
One must not allow oneself to be distracted
[as to] how I legitimately assume this.
This is more necessary than ever,
since here the method itself becomes absolutely creative;
further, nothing besides a remark like this
can be adduced in explanation of what happens here.
Our insight has come about through
the application of the basic maxims of our science,
to apply the principle of genesis
thoroughly and without exception.
This has been required,
and it should prove itself.
Likewise, the maxims of the science of knowing itself,
and with it all of science, have been required,
and they should prove themselves.
Science itself should justify
and prove itself before it truly begins.
Thereby, the science of knowing would be liberated
from freedom, arbitrariness, and accident;
as it must be, or else one could never come to it.)

4. Without digression, we conduct the required proof,
according to a law that has already been applied, thus.
We could produce this insight,
and we actually did so; we are knowing;
thus this insight is possible in knowing
and is actual in our current knowing.
Just a few remarks about this proof.

IV.24
visesa-darsina atma-bhava-bhavana-vinivrtti

IV.25
tada viveka-nimnam kaivalya-prag-bharam cittam

IV.26
tad-chidresu pratyaya-antarani samskarebhya

IV.27
hanam esam klesavad uktam

IV.28
prasamkhyane api-akusidasya sarvatha viveka-khyater dharma-megha samadhi

IV.29
tata klesa-karma-nivrtti

IV.30
tada sarvavarana-malapetasya jnaanasyanantyaj jnaeyam alpam

IV.31
tata-krta-arthanam parinama-krama-samaptir gunanam

IV.32
ksana-pratiyogi parinama-aparanta-nirgrahya krama

IV.33
purusa-artha-sunyanam gunanam pratiprasava kaivalyam
svarupa-pratistha va citi-sakti iti

the principle of appearance is
a principle of disjunction
within the aforementioned undivided oneness
and at the same time, obviously, within appearance.
However, as regards the absolute disjunction,
if disjunction were to be found
directly within absolute oneness,
as is unavoidably required by
the final form of the science of knowing,
and is what we intend here,
it must not be grasped as a simple disjunction,
but rather as the disjunction of
two different disjunctive foundations.
[It must be] not just a division,
but rather the self-intersecting division
of a presupposed division that again presupposes itself.
Or, using the expression with which we have designated it
in our most recent mention of it,
[it is] no simple “from,”
but rather a “from” in a “from,”
or a “from” of a “from.”

This much as a general introduction for the week:
Absolute self–genesis posited
and given a principle,
obviously within knowing,
which is thus a “principle-providing” [occurrence].
Thus within this knowing there follows
the absolute, positive negation of genesis =
completed and enduring being;
and indeed, because this entire investigation
concerns light’s pure immanence,
our investigation has long ago brought in
a presumed being external to knowing,
knowing’s [own] completed and enduring being.

Now We saw into this connection during the last hour
and see into it again here;
as is evident, we see one of the two terms
in and through the insight into this connection,
determined as such.
Thus, this latter is itself mediated,
and just we (= the insight we have now achieved) are
therefore the unconditionally immediate [term].
