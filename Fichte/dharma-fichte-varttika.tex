A. REFLECTION

IV.1
jati-antara-parinama prakrti-apurat

Sixteenth Lecture
Tuesday, May 15, 1804
Honorable Guests:

The fundamental principle presented now:
“Being is entirely a self-enclosed singularity
of immediately living being
that can never get outside itself”
is in part immediately clear in itself,
and in part it has been shown in the discussion
that it is clear to this assembly in particular.
Hence, we do not need to linger with it any further.

I said that it contains and completes
what one could present as the first part
of the science of knowing:
the pure theory of truth or reason.
We proceed now to the second part;
in order to deduce from the first part,
as necessary and true appearances,
everything which up to now we have
let go as merely empirical
and not intrinsically valid.
In advance of this undertaking,
I must remind you of only one thing.
Resolving this task in absolute oneness
of principle is not without difficulties;
especially since, according to a remark about method
I made at the end last time
(on the occasion of a general review),
this task is entirely new
and has not even arisen in the earlier
presentations of the science of knowing.
For that reason, it happens that this resolution
cannot remain without some complications.

However, in order to be completely clear with you about this point,
I will employ here the method that I have generally used before
of giving you an initial factical acquaintance with the terms
which come next so as to prepare you adequately
for their subsequent combination and connection.
This preparation is the next purpose of today's lecture.

IV.2
nimittam aprayojakam prakrtinam varana-bheda tu tata ksetrikavat

1. In the insight we had produced into inner being,
we began, after fully abstracting from that objectification,
about which we already know that it intrinsically lacks validity,
from this being's construction,
to which we expressly challenged ourselves.
(You see that I revert to this,
partly as it has always been done up to now,
and partly because thereby some kind of
idealistic outlook enters in again.
I refrain from giving an account of this here.)
Now mark this well, since this point can bring you great clarity,
I will not myself reason here as all previous idealisms have reasoned:
“consequently being depends on its being constructed,
and this is its principle”;
because this claim could have truth and meaning,
but only in relation to being's factical existence
in the form of external, objectifying existence,
which existence then [is] absolutely presupposed
and so, according to our basic maxim,
the projection through a gap would not be abstracted from.
This factical existence in general is to be put into question
in its basic principle and deduced for the first time;
but, trusting in the truth of the insight's content
and so in our principle,
we will thus conclude entirely realistically:
if being cannot ever get outside of itself
and nothing can be apart from it,
then it must be being itself
which thus constructs itself,
to the extent that this construction is to occur.
Or, as is completely synonymous:
We certainly are the agents who carry out this construction,
but we do it insofar as we are being itself,
as has been seen, and we coincide with it;
but by no means as a “we” which
is free and independent from being,
as could possibly seem to be the case,
and as it actually appears to be,
if we give ourselves over to appearance.
In short, if being is constructed,
as in fact it seems to us to be,
then it is constructed entirely through itself.
The basis for this construction,
as is immediately apparent and understandable to us,
can not be located external to being,
but only within itself as being, entirely and absolutely;
and indeed absolutely and necessarily
apart from all contingency.
At this point notice

IV.3
nirmana-cittani-asmita-matra

2. Let us go back.
a) I have said “If being is constructed,”
expressing myself hypothetically,
thereby perhaps reserving a future division
of the statement into a true part and a false part.
So if someone were to insist that it was actually constructed,
you might wonder how on the present standpoint
such a one might conduct the proof?
I know of no other way than by means of his consciousness.
However, we have already given up on [accepting]
such a proof as valid by itself;
but here for the first time [the question of]
to what extent and in what sense
consciousness and its statements
can be [accepted as] valid is to be decided.
In particular, we must decide on the extent
to which consciousness suffices
in the highest things it asserts,
of which the fact that being is constructed
can serve here as an example.
Therefore, we should not reach ahead
anticipating the results of the investigation,
which we will make possible only by means of
the hypothetical assertion.

b) To be sure, it has already become immediately clear
in our earlier investigation that mere being is
of itself, from itself, through itself immediately an esse,
that it therefore constructs itself,
and that it is only in this self-construction.
This comprised the whole content of our insight.
But the self-construction that we talk about here,
which we present only hypothetically as
a declaration of pure consciousness,
and which we append to being in itself
only mediately as an inference,
this is, as I ask you to grasp immediately,
something entirely different, merely idealistic, imaginal.
This is the only way I can describe it in words.
In contrast, the first one alone would be real,
clearly having the predicate “real”
only by contrast with the other,
and thus negating the absoluteness of
the previous insight with this predicate,
which is understandable only relationally
and through its opposite.
Now the task is just to find to what extent,
I say not so much being's ideal or real self-construction,
as rather the analytic/synthetic principle,
which grounds it, is to be [accepted as] valid.
This question of validity can only be resolved
by deducing the principle genetically.
Therefore, we take care not to anticipate,
and we grant the entire distinction
only hypothetical validity.

IV.4
pravrtti-bhede prayojakam cittam ekam anekesam

2. Let us go back.
If being is constructed ideally, as we assume,
then this happens directly as
a result of its own immanent essence.
Be sure not to overlook [the fact] that we have thereby
actually won something new and great.
That is, the ideal is posited in this absolute insight
organically and absolutely in essential being itself,
completely, without any real hiatus in essence,
and so without any disjunction in essence.
This insight is also genetic,
positing an absolute origin as unconditionally necessary
on the condition that it be the ground and be assumed.
Now this insight brings along an absolute that,
but by no means a how; we cannot see how
the absolute essence ideally constructs itself,
nor can the inner ground of this construction be constructed further.
We must not be put off by this,
since only thereby does
this insight secure itself as the absolute,
beyond which there is no other,
and this construction as the absolute one,
beyond which no other can be placed.
To be sure, it must come down to such
an absolute insight and construction,
and it is clear that, only at this point,
with an insight and construction proceeding
directly from essence, could we arrive there.
The gap, which as a result of
the absolute insight is in essence nothing at all,
exists only in respect of the We;
and, indeed, in case the essence of consciousness,
properly so-called, is to consist just in this,
no longer in the absolute and pure genesis
but rather in the genesis of the genesis, as it appears here,
then, if this We (or this regeneration of the absolute genesis)
were to be deduced, it would be in consciousness
that it [the gap] could well remain.
[We can then very easily see] that here, quite probably,
we have untied in passing the genuine knot at its root;
and that the new difficulty,
which has not concealed itself,
has fallen further down,
where it can let itself
be easily resolved by closer consideration
of the basic point already discovered.
In the meantime, since the point has
not yet been put as clearly as possible,
we will continue our exposition without staying here longer.

IV.5
tatra dhyana-jam anasayam

3. We have now grasped this.
Following our consistent method,
let us make this insight itself genetic.
Under what condition did it arise?
Evidently this, that an ideal
self-construction of being be assumed,
at least hypothetically.
“It is assumed,” obviously means and
can be explained ipso facto like this:
it is projected absolutely
into the form of outer existence,
in a provisional way,
without any ground or principle
for this projection,
thus through an irrational gap.
Now, a major portion of our task is
to demonstrate the genetic principle
for this irrational gap,
which so far we have presented only factically,
whose validity we have denied,
but without our being able to dispense with it.

(Observe. A philosophical lecture can frequently count on
the unnoticed assistance of the understanding,
without always providing the distinguishing grounds
for the distinctions that it makes;
the fact usually explains its true meaning
through itself and its results.
However, in such cases one always counts on
a happy accident that is just as likely not to occur.
It is always more exact not to leave
any distinguishing grounds unexplained;
and especially we should not allow ourselves
to be led astray by the fact that often, and on many subjects,
the explanation makes more obscure what was clearer
with the unnoticed assistance of the understanding,
because it should not be so, and satisfying ourselves
with understanding's unnoticed aide is not
the genuinely philosophical disposition.
In the hour before last, we had looked at the case of
a distinction between two ways of thinking about the in-itself,
whose distinguishing ground I specifically stated,
although the distinction might have been clear enough
simply as a matter of fact.
The present case is similar.
The principle for the irrational gap as such,
for the absolute absence of principle, as such,
should be demonstrated.
Obviously not insofar as it is an absence of principle,
because then it would negate and destroy itself,
a very different thing from it being provided with a principle.
So, in what respect is it to be and in what respect not?
Let us now make the meaning clear.
Being's ideal self-construction is projected through an absolute gap,
and is thereby made into an absolutely factical and external existence.
Now this existence, as absolute existence, can have no higher
principle at all in the sphere of existence,
and in this sense is precisely lacking a principle.
Its “principle” in this unprincipledness is just the projection itself.
Hence too it is not claimed, and cannot be claimed,
that being in itself constructs itself ideally,
but rather only that it is projected as constructing itself so;
this is important and breaks the doubt
aroused by the first remark at 1. above.
Therefore, nothing remains
[once this factical being has been annulled as absolute
by the demonstration that the projection is its principle]
except the projection itself, and this as an act,
as everyone is requested to become aware.
To say that a principle must be provided for it means therefore
that a principle must be provided for it solely as an act in general,
and as this act that in itself posits something unprincipled.)

What could this principle be?
The absolute insight, which forces itself on us,
that the ideal self-construction must
itself be grounded in absolute essence,
is conditioned by the presupposition of
this ideal self-construction without any ground,
and thus by this projection
we ourselves have made to complete
the science of knowing.
And so, the principle has been found in
what is conditioned by it,
and the newer, higher insight
that is thereby created can be
encompassed in the following sentence:
If the absolute insight is to arise, that, etc.,
then such an ideal self-construction
must be posited entirely factically.
The explanation through immediate insight is
conditioned by the absolutely factical
presupposition of what is to be explained.

4. Now do not forget that everything here
remains only hypothetical.
If it should be seen into, then must ____, etc.
Should the consequent be posited
absolutely and categorically?
Undoubtedly, if the antecedent is,
and without doubt not, if the antecedent is not;
because the latter has no principle except the former.
But if the first should be posited absolutely,
then it is not apparent as absolute,
because it has been posited as absolute hypothetically.
As I add now only to arouse attention,
in this hypothetical “should” as our highest point so far,
everything comes together whose derivation is now our task:
the ideal construction of being as a self-construction,
as well as the projection through a gap.
Just so, it is clear that this hypothetical quality
of the “should” must remain as it has been presented.
However, it is equally clear that
something categorical must arise too,
since otherwise our science would be baseless
and without principle through its whole range
as well as in its starting point.
However, this categorical quality
must now just manifest itself hypothetically
in the “should” qua “should,”
so that henceforth the chief principle
of the process of appearance
(and, if it were believed, of what appears)
should consist in this:
that the absolutely categorical “should”
appear as hypothetical in relation to the insight,
the true and the certain.
That is, [it should appear]
as able to be or not to be,
as able to be thus or be otherwise.

IV.6
karma-asukla-akrsnam yogina trividham itaresam

5. In order to prepare the way for this point,
to the extent time allows,
I urge you to reflect maturely with me
on the essence of the “should.”
Obviously, an inner self-construction is
expressed in the “should”:
an inner, absolute, pure, qualitative
self-making and resting-on-itself.
One can assist the intuition of this truth,
which in any case also makes itself.
It is, I say, an “inner self-construction,”
completely as such:
nothing else supports the hypothetical “should,”
except its inner postulation entirely by itself
and without any other ground;
because if it had some other ground,
it would no longer be a hypothetical “should,”
but rather a categorical “must.”
“Inner postulation entirely by itself”
I have said; hence a creation from nothing,
producing itself entirely as such.
A “resting-on-itself” I have said, because
(letting myself take it up in a sensory form,
which harms nothing here)
it falls back into nothing
without this continuing pursuit of
inward, living postulation
and creation from nothing.
Hence it is the self-creator of its own being
and the self-support of its duration.

This, as we have described it,
is simply then the “should,”
and, according to the presupposition,
it is grasped intuitively by all of you in this way.
Therefore, with all its initially apparent hypothetical character,
just for that very reason there is something
categorical and absolute here,
the absolute determinateness of its essence.
Before we now show further what follows from this,
let me add today two further comments in conclusion.

a. The “should” bears every criterion of
the intrinsic being intuited in the basic principle:
[it is] an inner, living from itself, through itself, in itself,
creating and bearing itself, pure I, and so forth;
and [it is] certainly organized and
coherent internally, entirely as such.
As regards the latter, in case it needs further explanation
after the clarity with which it must have
already presented itself previously in intuition:
we then always objectified the fundamental principle's
“inner being” factically,
although this objectivity was not valid.
We also have previously objectified the “should.”
Finally, however, we have been lost in it factically,
in its inner description and insight,
and now for the first time
we free ourselves from it,
and it from us, in reflecting about it,
a process which, according to our previous method,
can be explained as a projection through a gap
taking as its principle the “should” itself.
Accordingly, this “should”
(purely and simply in its oneness,
and without any supplement)
can easily be being's immediate ideal self-construction,
that is in no way to be further reconstructed,
but rather that provides the subject matter
directly in the construction itself.
On the other hand, being's previous,
hypothetically posited,
construction from the “should” has
finally found a principle in this “should,”
just as inner being finally has too
for its projection through a gap,
which we had proposed accordingly.
[This principle] in itself is
construction and subject matter,
ideality and reality,
and it cannot be one without the other.
This duality may reside in our objectifying
consideration of the science of knowing,
which therefore abandons its claim to
intrinsic validity.

b. This “should” has constantly, but without notice,
played the principal role in all our previous investigations.
“Should it come to this or that,
to a realization of the through, etc., then must ...”;
our insights have always gone along in this fashion.
Therefore, no wonder that after
letting go of everything else,
what remains for us is only the thing
that is truly first in all these cases.

IV.7
tatas tad-vipaka-anugunanam eva-abhivyakti vasananam

I have stated in the last hour how
the part of our science on which we are now working
might be different from the part completed first,
and what our following lectures intend:
namely to introduce for the first time
the materials for resolving our second task,
and to make you familiar with them.
At the same time, I admitted that the next lectures
might not be without difficulty and confusion.
It is easier to take in and to grasp
that which rests in reason completely and simply as oneness,
as did the earlier fundamental principle,
since only abstraction is needed for this task.
“Easier,” I say, than to trace what
in itself and originally is never a oneness
back to a oneness in order to produce
a completely new and unheard of concept in oneself,
for which other arts are undoubtedly required.
Now, we first lay out multiplicity in an order
in which is most convenient to us for insight.
These terms can first be correctly
ordered and understood on the basis of their principle,
which itself is first to be discovered from them.
At this point in the course of the external lectures,
there is an unavoidable circle that can be annulled
only by its own completion.
It is possible, and indeed expected, however,
for one to grasp the process
[that, to be sure, has its proper order]
and terms, and to give them what clarity
they can have under the circumstances.
I have said that a new, heretofore entirely unknown,
principle must be presented;
and also simultaneously I would add this remark: that
(thinking of the previous division
of the science of knowing into two parts)
we are concerned not just with presenting the second part,
but also with uniting the latter with the first part.

The course of the previous sessions was this:
we constructed the pure being, which we had grasped,
as an entirely self-enclosed singularity.
In this way, I assumed, we could become
immediately conscious of ourselves,
and, as required, we were actually conscious of ourselves.
This therefore was a completely simple,
factically objectifying projection of
an act that we ascribe to ourselves
as likewise independently existing entities;
and in this manner we could have been tempted
to deduce being itself from this act of
construction in a one-sidedly idealistic fashion.
However, we wisely refrained from doing this,
well understanding that by this
we would return whence we had first arisen
and consequently would not have advanced.
But we proceeded in this manner,
and necessarily had to do so,
if we wanted to come to something
more than the one being,
for example to the latter's way of appearing.
“As concerns the truth in itself of this construction,
this can appeal to nothing else besides
the bare assertion of consciousness.”
We cannot now discard this statement unconditionally,
as just previously we unconditionally rejected it,
thanks to our present, entirely altered, aim;
because previously we sought pure being in itself,
and it has been shown that consciousness is
entirely insufficient for this purpose.
Now we no longer seek this pure being in itself,
since we already have it
and so our search for it is over.
Instead, we want to grasp it in its primordial appearance;
and so consciousness, and here in particular the construction,
could be the first term of this appearance for us to grasp.
We cannot allow this term to be unconditionally valid
any more than before;
since the extent to which and the conditions under which
it is valid are exactly the issue.
Therefore, we must present this claim hypothetically,
without prejudging future inquiry:
if, and to the extent that, a construction of this kind is actual,
that is, takes part in being and not merely seems to be,
but has being actually appearing in it, then ____ .
Through this then, we are asked to point out
in immediate manifestness the condition for
the real and true being of such a construction,
in case and to the extent that being could come to it.
This condition has now been found and has become
evident without any difficulty:
If this construction, which appears to us,
is actually and in fact connected with true being in reason,
but in no way connected with factical existence in consciousness,
which counts for nothing until it is better grounded
(this detail is not to be overlooked)
if the construction which appears to us is in this sense,
then it is not in any way based in the vain “I”
of consciousness that emptily objectifies being;
rather it is grounded in being itself.
For being is one, and where it is, it is whole; in being qua being;
therefore entirely and absolutely necessary.

(On the condition that you do not allow yourselves to be distracted,
let me add here an additional remark that can spread much illumination.
Posit pure immanent being as the absolute, substance, God,
as indeed it really is, and posit appearance,
that is grasped here in its highest point
as the absolute's internal genetic construction,
as the revelation and manifestation of God,
then the latter is understood as absolutely essential
and grounded in the essence of the absolute itself.
I assert that this insight into absolute inward necessity
is a distinguishing mark of the science of knowing
as against all other systems.
I cannot emphasize it enough,
because the absolute absence of insight
strives against it with all its might,
since freedom is always the last thing
[this darkness] will surrender.
If it cannot save [freedom] for itself,
then at least it tries to secure it in God.
In everyone without exception,
an absolute contingency exists
next to absolute substance.
Here something is seen from the beginning
as absolutely necessary in reason and in itself,
which afterwards will appear
not in reason and not in itself
but as contingent in another connection
that still is to be worked out.
Only on this condition can the science of knowing
hope to deduce the phenomenon in a genuine and grounded manner,
and not merely as a pretense;
because a genuine derivation must have a reliable principle.
Otherwise, as has often actually happened,
one deduces from the intrinsically contingent
something else which is also contingent,
and obtains other contingent things from these,
which themselves stand firm only on condition
of the reliability of the previous thing,
whose reliability likewise depends on the first.
As if a good, proper, and reliable
standpoint could arise when one had two terms,
neither of which could stand by itself,
each relying reciprocally on the other.)

This remark as well:
it is evident that in our present investigation
it still seems as if, as I freely admitted at the outset,
this investigation is still searching
for its principle but has not yet got it,
something I have described as erroneous,
since its first term [the construction of inward being]
still remains hypothetical in connection to that
about which alone we are inquiring, true being in reason.
So, the thing which can first be ascertained under this condition,
being's necessary self-construction,
can itself not be otherwise than hypothetical to the same degree.
Therefore, from here on you should direct your attention
to the question whether and when
a self-sustaining principle emerges.

If there is a construction of being,
then it is grounded absolutely in being itself;
we grasped this directly and reflected further on the
insight and its inner, law-governed form.
Then it was immediately clear that
we began with the presupposition of inner being's construction,
which we incorrectly attributed to the “I” of consciousness,
but we have already learned better than this
and let go of the attribution.
But this much remains indubitable,
that being's construction is projected as an absolute fact.
Have we now brought this implicitly simple, factical projection
into connection with other terms by the use we have made of it?
Evidently so; we saw that if such a construction exists,
then it must be grounded in being.
Now, we have undertaken this entire speculative venture freely;
the resulting insight
(which might very well not have been engendered)
is conditioned by our procedure
(which we might very well have omitted),
and therefore it is in no way a firm standpoint.
All the same, in order to achieve such a firm standpoint
we applied a procedure that, to the extent that it needed to,
proved its legitimacy by its bare possibility.
We said: assume that the insight, engendered by us,
is to arise, then you will see that under these conditions
the projection of factical being, previously only possible,
becomes necessary.

In this way, we would first of all have made
good progress beyond everything achieved so far
toward a proof that, although to be sure
we do not feel firm ground beneath us,
we might be on a good path.
The absolute projection through a gap,
and thereby the form of outer existence,
that could not be understood conceptually
in all our previous investigations,
is explained as necessary
under the assumption that
a higher term (the insight) should be,
an assumption that itself was previously hypothetical.
Thus, the hypothetical status departs from the lower term,
though only by transferring itself to the higher;
but at least with this it is simplified
and its proper location is revealed to us,
where we can hope to grasp it at the root.

IV.8
jati-desa-kala vyavahitanam api-anantaryam smrti-samskarayo eka-rupatvat

After what I have said about the necessity
of a self-supporting principle for this investigation as well,
we will next eliminate this hypothetical status completely;
and here the most secure means is to look it straight in the eye.
It is entirely compressed into the hypothetical “should”;
this is sufficient by itself for our next purpose;
therefore, we let go of the site
where this “should” appears, insight, etc.
Quite apart from our current procedure,
it could be obvious from the entire previous investigation
that one now needs to keep this “should”
as one of the deepest foundation points of all appearance,
as I will observe in passing.
All our preceding investigations and engendered
insights have started with the hypothetical “should”
and have proceeded from it as a terminus a quo:
“If there is really to be a “through,” then there must ____”;
“Should the achieved insight arise, then there must ____”: idealism;
“should this life be life in itself, then there must ____”: realism;
all the way up to the highest relation:
“Should an in-itself be comprehensible,
then a not-in-itself must be thought” and so on.

This “should” loses itself entirely
only in the insight into pure being
and into the way in which we evaporate into it,
so that an absolute categorical character enters,
without any hypothetical presupposition.
As soon as we reflect again on this insight,
the process which yielded the historical origin
of our second part and our entire present investigation,
it reinstates itself with a “should,”
thus as something contingent,
seeking the basic condition for this contingent quality,
a necessary self-construction of being.
Now, so far in the ascent we have clung to
the content of the generated insight without reflecting
on the hypothetical form in which, as a whole, it appeared.
This was entirely correct because we wanted to arrive
at the original content of the truth as such.
(Here in passing the question that some have asked me concerning
the true grounds for our first part's preference for realism and
for the maxim that ruled there always to orient ourselves realistically
answers itself decisively and fundamentally.)
But now as we descend we have to hold on
to just this neglected “should,”
which indeed provides the enduring
inner soul of all the idealisms,
which consistently excluded themselves during the ascent,
and which were struck down by an opposed higher term
only in respect to their content;
but still persist in their form, as we see.
Now this form cannot be disturbed
directly by the original content,
since everything that the latter can attain
has already been achieved in the ascent.
Rather, it must be explained
and justified inwardly on its own terms.
It must refute its own ungrounded claims,
to the extent that they are ungrounded;
roughly just the way we refuted
the content's highest idealism,
that at first presented itself as realism,
by means of the law that it itself presented,
and revealed it as idealism.

In a word, and in order to lead you
even deeper into the systematic connection between
a. the term that earlier was the highest in appearance,
the distinguishing and the unification of
the in-itself and not-in-itself
in the whole five-fold synthesis, and
b. absolute inner being,
as the absolutely realistic element,
the “should” enters here as a new middle term,
in which the self-differentiating
and likewise synthetic relation
of the two indicated relational terms must show itself.
To find this is the proper content of our task:
to find it as a firm principle is its form.

First, however, the connection to inward being.
The form of being is categoricalness.
Therefore, something categorical must be found
in the “should” itself,
however hypothetical it might appear.
In order to uncover this,
I have demanded that the inner essence
of a hypothetical “should” be carefully considered
(following our consistent method of raising into clarity
something that was at first dimly projected).
We have already done this last time;
because of the subject's importance,
I will repeat the entire operation today.

If you say forcefully and deliberately:
“should so and so be,” then it is clear that
thereby an inner assumption is expressed,
without any foundation, simply of itself and from itself,
thus an inwardly pure creation,
and to be sure standing there completely pure entirely as such,
because the “should,” if it is taken only as purely hypothetical
[as is required here and without achieving which
the required insight will not arise]
expresses complete external groundlessness,
simple internal self-grounding, and nothing else.
Further, (in this way, I tried to grasp the same thing
and make it clear from the other side):
the absolute assumption is expressed in the should,
an assumption that is unconditionally allowed to drop,
just as it is unconditionally presupposed.
Should it [and with it probably the entire
“If ... should, then ____ must”
that depends on it] not drop away,
(with which dropping away all knowledge and insight
probably drop away as well)
then it must hold and sustain itself.
As surely as we have now seen into this,
just that surely has the should been illuminated
for us as an absolute that holds and sustains itself
out of itself, of itself, and through itself as such,
on the condition that it exists.
This, I say, is a “should”;
and were it not precisely so,
then it would not be a “should”;
therefore we have a categorical insight
into the unchangeable, unalterable nature of the “should,”
an insight in which we can completely abstract
from the outward existence of such a “should.”
“Can abstract,” I say since, with adequate deliberation,
I refrain from drawing a conclusion here
which easily presents itself,
but which is not yet sufficiently ripe, given the context.
To the extent that our task simply consisted
in discovering something categorical in the “should,”
it has been fulfilled by what has happened.

In explaining the should, I have not warned you
about the illusion that it is we who assume there
what is hypothetical and who hold and sustain it;
since the rule is to leave this “we”
of mere consciousness entirely out of action until it is deduced,
and being able to do so is the art without which
no entry into the domain of the science of knowing is possible.
If, in the meantime, this “I” has forced itself on anyone,
then let it immediately remove itself at this point.
Namely, whether or not you have created and carried the assumption,
it is still always completely clear that you have a “should”
only on this condition of self-creation and carrying forward.
Therefore, even if you are the creator,
the “should ” always contains the rule and law
of proceeding in that manner,
otherwise it is not a “should,”
and we have not wished to say any more than that here;
abstracting completely from the question that you raised
and that we will work out in another place.

And now, in conclusion, a very sharp distinction,
that will become decisive in what follows,
and that cannot be made clear too soon.
The strong similarity between inner being
as something self-enclosed and self-sufficient
in-itself, of-itself, and through-itself,
and the should as just the same
has already been pointed out earlier.
There is nonetheless a distinction
between the two that I have named
and made dimly recognizable in the stated formula:
“the should” is something in-itself, etc., as such.
I urge you now to clarify this distinction
for yourselves along with me.
Being was constructed as something absolute in itself, etc.
I ask: should there now exist,
or is their actually in our insight,
if it is of the right kind,
another persisting being or substantive,
besides this absolute, self-constructing esse?
Not at all.
Instead both merge into each other
and into the pure self-enclosed singularity,
and the doubled repetition is entirely
superfluous, insufficient, and neglected.
This is not at all the case with the “should,”
if you will look into it quite acutely.
The latter stands out as a fixed, substantial
middle point and bearer of
absolute self-production and continuation.
The latter is not just immediate,
as was the case with being,
but rather only mediate through
presupposing and positing a “should”;
in brief on the assumption that
the “should” itself again should be,
and thus should be seen through its own doubling.
Here there is not, as there was before,
an immediate rational insight,
but rather only a mediate one,
conditioned again by a higher
projection through a gap,
precisely of the “should”;
just in the way we have actually proceeded.
We have wished to indicate this relation
by the added phrase “as such,”
itself in objective, factical oneness of essence.

To what further things this new discovery
might lead must emerge on its own.
Before hand, this much arises in regard to method:
that, just as a projection through a gap
(the projection of being's construction)
is deduced as necessary from the fact
that a particular insight “should be”,
another projection, just that of the “should” itself,
presents itself [on the one hand] as
a condition for this insight
and on the other side again as conditioned by it.
We now need to venture further into this;
that therefore our present investigation,
just like the previous one,
advances upward only in this
precisely delineated circle,
because it is still looking
for the latter's principle.

IV.9
tasam anaditvam ca-asisa nityatvat

Eighteenth Lecture
Thursday, May 17, 1804
Honored Guests:

[Here is] what has been presented so far:
we presuppose a construction of being.
On the principle that nothing can be except being,
the construction is seen as arising necessarily from being,
of course, with the same certainty it has generally;
supposing therefore “If it should be ..., then ____ must be.”
But the “should” is something in itself, of itself and from itself "as such.”
This, and in particular the “as” most recently added,
is now firmly fixed for you as another new middle point
and bearer for the self-producing and self-sustaining “should.”

Today I add another basic observation
concerning the true inner spirit of
the reasoning processes presented so far,
and [we] will then work on
our remaining task from another angle.

1. As concerns the first,
our higher insight from the standpoint of
the hypothetical “should” took the following form:
should an insight into this or that occur
(in this case in particular the insight that
the ideal self-construction is to be grounded in being itself),
then ____ must.
“Since you now,” I would say,
“actually provide the content of this insight,
which, according to your account hasn't yet occurred
but whose condition you are seeking,
you already without doubt have it
in sight and in your concept;
you are constructing it really and in fact”;
(as is the case here with being's ideal self-construction).

This remark permeates all consciousness
and can be illustrated in every case.
I cannot reflect how and according to what law anything
(e.g., a body in space, space, a line, etc.)
is conceived or constructed,
unless I have already grasped it apart from all reflection
and according to a universal law.
In the present case, the law is constructed in
one of the most general cases,
which contains others within itself.
“Therefore,” I continue, “you seek either
that which you already have,
or you seek the same vision and the same concept
(the same in regard to contents)
only in another qualitative form.
That the latter is the case becomes clear
through a more exact consideration of the proposition laid down.
The content of your vision, which, as the content of
mere seeing, is separated and existing for itself,
should be brought into connection with something else in seeing,
both as its condition and as conditioned by it,
initially through what you call insight.
Thus, in order to state the true result of
your desire definitely and exactly:
just to arrive at your demand,
a seeing that is already completely determinate in and of itself
(and that you must presuppose as determinate)
should be further qualitatively determined
in this persistent, objective determinateness as seeing,
since the objective determinateness remains.
Therefore, to put it briefly,
you demand a new genesis in the seeing
that has already been presupposed as existing
and as remaining the same objectively.

A new inner genesis of seeing,
as formal seeing itself,
without any alteration in the [seen] content;
(what we have already called objectivity).
Now, the material of this formal genesis,
its result, is itself again a genesis:
the constant content should be brought
into a genetic relation with another term,
that creates, and is again created by, it;
thus the entire familiar “through,”
or the relation in its synthetic five-foldness,
should come in.
As things stand, it can well be that
this external material genesis
with and out of the content,
which is nonetheless not changed in its inner nature,
is itself grounded in mere seeing's formal genesis,
and resides not so much in the subject matter as in the altered eye,
through which the entire present multiplicity is traced
back to the oneness of the same principle,
of the formal further determination.
This formal further determination,
or new genesis, is called for through a “should,”
which has itself been recognized as a genesis
in its inner nature unconditionally as such.
And so this genesis could have its ground
in the “should” itself as the relation and
five-fold synthesis within the formal genesis,
so that the “should” is the basic principle for everything,
as we have previously already taken it to be.
In brief, the spirit of our whole reasoning process,
conducted since the beginning of the second part,
is the demand for an inner genesis
in the seeing presupposed for genesis itself.
This process adds nothing to the seeing in its true meaning,
and so it must be inoperative in relation to this meaning,
just as we have always wished.
Likewise, this very inner formal genesis,
as wholly concerning only the way of viewing,
may be the principle of absolute idealism = of appearance;
and we ourselves have entered into a new and higher idealism
through the principle presupposed in
our entire reasoning process:
that being is constructed ideally,
separately from its real self-construction.

That just this insight,
now characteristically distinguished
from the presupposed original seeing,
presents itself alone as certain,
compared to which the original seeing is
to be only hypothetical in relation to its content
(it is clear on immediate reflection that the matter stands so,
and our certainty appears finished and closed);
this circumstance probably lies in the partiality of idealism itself,
which here gives testimony for itself, knowing nothing else.
Now we have to investigate this claim for the first time.

2. A recognized basic rule:
nothing can be accomplished in any way against an idealism
except from the standpoint of realism.
Therefore, as soon as our reasoning
has been traced back to its spiritual oneness
and understood to be idealism,
we cannot stand by it any longer
without being driven around in circles.
We must turn instead to the corresponding realism
and consider this more deeply in its origins.

a. As we remember, we entered this realism
after the last consideration of the in-itself,
and of the insight that, in our knowing,
this in-itself is relation and multiplicity;
therefore, that it is not absolute oneness,
thinkable without any composition or division,
but is rather, as we said:
a oneness of understanding.
We discarded this knowing entirely
and yet knowing still remained,
which thus was absolute inner oneness,
without any combination or separation:
oneness in itself.
We also refrain from saying for
example that we have produced it in this oneness;
since we truly would not have wished that
something should remain behind
after abstracting from everything,
or [have wished] to encompass what remains with our will,
had we willed or been able to will this,
so that it would indeed have been left over for us:
instead it was just unconditionally left over:
oneness of itself.
Everything depends on this last point;
it is what has been overlooked in every system
and what becomes clear only to the deepest deliberation.
What we are naming the We, that is our freedom,
which is derived here for the first time
from the previously mentioned,
new formal genesis of the absolutely
presupposed seeing = reconstruction,
can only abstract from its own
creation of the act of reconstruction,
but it cannot creatively construct primordial reason;
although after complete abstraction
primordial reason enters without delay.
So then anyone who [in inseparable awareness of
the simultaneity of his completed abstraction
and the arrival of primordial reason,
and in the equally inseparable awareness
that he is the one freely abstracting]
immediately transfers his own
freedom to reason's emergence,
such a person deceives himself
and remains trapped in an idealism.
This final illusion is negated here
in immediate manifestness by means of deep reflection.
After abstraction from the highest oneness of understanding,
a knowing remains, just because it remains,
without any possible assistance from us,
pure light or pure reason in itself.

b. This pure reason is equally immediately
inner being and completely one with it.
Previously we called what remained
after all abstraction “inner being”;
here we have called it “pure light,” or “reason.”
But whatever we may wish to name it,
it is what remains unconditionally
by itself after all abstraction,
an entirely indivisible singularity;
and I would very much like to know
whether any disjunction can be made
in the presented concept,
and whether the insight that it is
a completely self-contained singularity
does not clearly show that,
whatever variation in the words used to name it,
one and the same nature could be meant.

c. Previously as well as now,
we have named it a real self-construction
in itself, of itself, and through itself,
and we could not describe it differently.
Now, abstracting completely from
the facticity of this description,
which to be sure can only be a reconstruction,
and through which we happen into the first named idealism,
[let us] reflect [instead] on its inner truth,
and [with this I ask for your complete attention]
on the surprising result that I intend to bring out.
I ask: does it not now depend entirely
on the pure thing itself
remaining after all abstraction
that it exists entirely from itself
whether you call it being or reason?
For example, is it arbitrarily posited
as existing on its own?
How could it be?
For this would be a genuine contradiction,
since in that case it would not be from itself
but would exist through an arbitrary act of positing.
If it is posited as something left over
after abstracting from everything outside itself,
then it is necessarily posited as of itself.
For if it were not of itself,
then it would be of another,
so that in its absolute positing
[in the original creation of its being]
it would not be possible to abstract from this other.
(That because of babble and thoughtlessness this other might not
be considered could be factically true and still should be explained;
it is not true in the one absolute, self-consuming oneness.)
Once again, it is posited absolutely,
creatively, as something of itself;
it is evident that this of itself is
actually manifested and is not just thought up;
so it is posited as existing absolutely
and remaining behind after abstraction from everything.
Hence it is clear that light, or reason,
or absolute being, which are all the same,
cannot posit itself as such without constructing itself, and vice versa:
that both coincide in their essence and are entirely one.
Notice here:

1. the insight that being must construct itself
unconditionally has arisen here
through the mere consideration
of its inner nature entirely immediately
and without any factical presupposition,
an insight that, according to idealism's pretensions,
should only be producible mediately
from the factical presupposition
that a constructive act is present.
By this means idealism is
first of all fully refuted,
insofar as it grounds itself
in the necessity of a presupposition
for a particular insight,
although merely a possible one,
since the insight has actually
been produced without the presupposition.
Idealism must therefore look
around for higher support,
if we are still to come to it.
Further, the proposition alluded
to in passing has therefore come up,
that this same insight is possible
in two different ways:
mediately, from presuppositions,
and completely immediately.
How would it be,
if the entire distinction that we have sought
between philosophical and common knowledge,
between the standpoint of the science of knowing
and that of ordinary knowing
(and in case within the latter
there should be degrees of mediatedness,
the distinction between the various standpoints
of this common knowing)
were to lie in just this distinction
between these differing ways.
Philosophical systems are always closest for us:
the presupposition that idealism wants
as the principle of mediate insight, is factical.
How would it be if, for example,
the proof of absolute being
from the factical existence of finite entities,
which is conducted in nearly every system,
and according to them in ordinary consciousness as well,
were just this idealistic path of mediate insight,
with which one remains satisfied,
for the lack of the immediate path.
In itself this is correct
and is applicable in its place
within the gradual process of cultivation,
in rising up to the highest;
but it generally fails the test against criticisms
that strive ahead to the highest!

2. The distinction between being's
real and ideal self-construction
that we made earlier,
and on which idealism built,
is now completely annulled.
Being, or reason, and light are one;
and this one cannot posit itself, or be,
without constructing itself;
this is therefore grounded in its nature,
and is entirely unitary, as is its nature.
Therefore, if we are to return
later to such a distinction,
then it must first be derived.

3. We saw that, in reason per se,
its self-positing and its self-construction
as “from itself,” etc., coalesce entirely into one.
And as certainly as we saw into this,
in this insight we were the oneness of reason itself.
Now a duality still remains here,
not however as in the oneness of understanding,
whose parts are to be integrated
[since parts within the oneness are
rather completely denied and negated here;
and the oneness does not understand itself through parts
but rather posits itself unconditionally and absolutely]
but rather as a means for achieving oneness.
Therefore, it may perhaps turn out that
a reconstruction is already present here,
one that would be posited backwards
toward the idealistic side by an absolute “should,”
and which we could not avoid merely factically,
even though its intrinsic validity is not admitted;
therefore that we stand at the precise place
from which our task could be completed.
How things may stand with this
I reserve for further investigation.

Now I add a supplementary remark,
with which I did not previously wish
to interrupt the course of the inquiry.
As the opportunity has arisen,
I have tested those recent philosophical systems,
which have made the greatest impression
in regards to their principles,
in order thereby to bring greater clarity
to the science of knowing;
thus Reinhold's system and thus Schelling's system.
Next to these, and perhaps even more than they,
Jacobi's system recommends itself,
because with great philosophical talent
it tries to jettison philosophy itself,
and thus it flatters the prevailing spiritual
indolence and denial toward philosophy.
The scene for testing this system's principles was just above.
It proceeds from the following principles:

a. We can only reconstruct what originally exists.

We ourselves have precisely presented
and precisely defined this claim,
which for Jacobi is almost a postulate:
the seeing, determined primordially in its content,
is formally genetic in relation to an unaltered content,
and therefore it is the insight into a connection;
and we ascribe this genesis to ourselves,
a genesis that is only reconstruction
in relation to the truly original content
and that is truly original
construction and creation from nothing
in relation to the terms added factically.
Regarding the last point,
the absolute creation of everything factical from the I,
he very clearly took this over from us;
and it is very plausible that he granted being to the factical,
to what is sensible outside the one rational being,
and thereby left us only reconstruction.

b. Philosophy should reveal and discover being in and of itself.

Correct, and exactly our purpose.
Through the persistent assertion of these two principles
this author has earned the age's great thanks,
and has favorably distinguished himself from
all of the philosophers who just reconstruct impartially,
or even just fool around with nature and reason.

c. Therefore, we cannot philosophize, and there can be no philosophy.

This latter claim, just as I have stated it,
is his true opinion, and must be his true opinion,
if he is to have any opinion at all.
For he contributes nothing by his usual addition:
philosophy as a whole.
Because, if there is no philosophy as a whole,
then there is no philosophy at all,
but rather only edifying remarks for every day of the year.
I grant him everything as it is presented,
only taking it more seriously than its original proponent does.
We, the we who can only reconstruct, cannot do philosophy:
equally there is no philosophy individually and personally;
instead philosophy must just be,
but this is possible only to the extent
that we perish, along with all reconstruction,
and pure reason emerges pure and alone;
since this latter in its purity is philosophy itself.
From the perspective of “we” or “I”
there is no philosophy;
there is one only [once one has gone] beyond the I.
Therefore, the question about the possibility of philosophy
depends on whether the I can perish
and reason can come purely to manifestation.
This author could demonstrate that
this must indeed be possible from his own words.
Because when he says:
we can only reconstruct, he achieves ipso facto
in that very moment something more than reconstruction,
and has at least drawn himself happily out
of the “We” of which we have spoken.
For if he could [do] this,
then for his whole lifetime he would enact,
but without speaking about it,
just as by his previous statement
he enacted elevating himself to
reconstructing the reconstruction.
Of, if we will free him from this,
he [can] tell us how he came to the universal statement
by which he prescribed an absolute law for his “We,”
and thereby pre-constructed the “We's” essence for them,
and did not merely reconstruct it.
In which case he would have to resign himself
to express himself like this:
“I and everyone I know,
as many as I can remember to the present
could only reconstruct;
whether perhaps tomorrow something else will happen,
we will have to see.”
Finally he will have to tell us whether
he understands this concept of “reconstruction”
without presupposing something original,
independent prior to all construction.
As surely as he understands himself,
he must become aware of such
a thing beyond all reconstruction.
Grasping this original something and reconstruction
as following from it as an absolutely essential law of the “We,”
just as we have articulated it here is
the task of a philosophical system,
which we have presented entirely
according to its sense,
but have only partially solved.

IV.10
hetu-phala-asraya-alambana sangrhitatvad esam abhave tad-abhava

Nineteenth Lecture
Friday, May 18, 1804
Honored Guests:

Since today we finish the week,
I do not wish to let you go without
having equipped you with some definite result.
This resolution compels me for now to pass by
certain middle terms that still remain
for deeper consideration between
that with which I ended yesterday
and that which I will attach to it today,
in order to reserve them for the descent.

1. As an introduction to our essential business for today,
[here is] a clarifying remark that
should direct your subsequent attention
and that at the same time also briefly and concisely
repeats the first major part of yesterday's lecture!
I say: in all derivative knowing, or in appearance,
a pure absolute contradiction exists
between enactment and saying:
propositio facto contraria.
(Let me add here by the way,
as I thought previously on an appropriate occasion,
a thoroughgoing skepticism must base itself on just this
and give voice to this ineradicable
contradiction in mere consciousness.
The very simple refutation of all systems
that do not elevate themselves to pure reason,
their dismissal and the presentation of their insufficiency
even though their originator is not thereby improved,
is based on just the fact that one points out
the contradiction between what they assert in their principles
and what they actually do [in asserting them]:
as has been done with every system that we have tested so far,
and yesterday with Jacobi's as well.)

In the first half of yesterday's talk,
this contradiction showed up
in what we had identified so far as
the highest principle of appearance,
that is in the “should,”
immediately after we had conceived it
in its firm and completely determinate nature
as something from itself, etc., as such;
namely, a particular insight
(which in our case was this:
that being constructs itself)
is posited through the “should”
as not present but rather merely as possible,
and as possible only under a certain condition
that is still sought.
If we are even to arrive at the consideration of
its conditioned possibility,
this [condition] finally must be presupposed as
a seeing that is fixed in its content
and to that extent unchangeable,.
Hence, [two things] stand in
complete contradiction in this “should,”
its enactment [its true inner effect,
to presuppose a seeing that is unalterable in its contents]
and its saying [a different action on its part,
according to which the insight is supposed
to be not actual but only possible
under a condition yet to be added.]
I add only for the sake of recapitulation
that the true external nature of this “should” is
found as the demand for a further inner
and merely formal determination
of a presupposed seeing that is
unalterable in its content,
through which further determination
this presupposed seeing comes into
a genetic connection with another term
that is created purely by this further determination.
And I immediately formulate the following conclusion:
absolute reason is distinguished from this relative knowing
by the fact that, in the case of absolute reason,
what exists, or what it does, is expressly said in it;
and that it does what is expressed in it
in absolute qualitative sameness.

2. In the second part of yesterday's investigation,
we tried to represent pure reason in ourselves.
I noted at the end of this presentation that,
because of the duality that to be sure was annulled intellectually
but that remained factically inextinguishable in you,
it became evident that pure reason could not
display itself immediately in you
and could rather only be reconstructed.
The same qualitative determination of
a presupposed seeing that is unalterable in its contents,
a determination pointed out within the “should,”
we also called reconstruction;
therefore, the contradiction between saying and doing
just discovered in all derivative knowing is
contained within reconstruction itself,
a fact that can itself be made clear immediately.
To be sure, reconstruction explicitly puts
itself forward as reconstruction
and therefore in its own concept
quite properly posits the point of origin,
and in that there is no contradiction.
But since it leaves the content unchanged
and can actually create nothing new
without completely negating the relation
between itself and the absolute,
its construction is therefore groundless
and the fact itself contradicts the postulate
of the absolute necessity in
the pure, positive in-itself.

I should now immediately climb past this
contradiction [we have] discovered and relieve it
(that is, past the groundlessness of
the concept of a reconstruction).
However, in accordance with my initially stated resolve,
I am retaining it [in order] to annul it mediately in the descent;
and so [let's turn] directly to yesterday's reasoning
to indicate the location of absolute reconstruction,
and to remove this circumstance.

We brought the already established absolute insight
to life in this way:

a. it arose for us after
we abstracted completely from all relations,
and it remained behind as a oneness,
not just because we wished it,
but simply by itself.
Pure light, or reason.

b. Previously we named it inner being,
here light or reason;
but it is clear that no distinction whatsoever occurs
in the one singularity that remains behind by itself as one,
and that consequently both designations are only
two different names for the one
that is grasped as completely
indivisible and inseparable.

c. We saw into this “one,”
and still see into it now as
something from-itself,
etc. = [as] self constructing.

I asked: should not then this from-itself reside completely
in its nature as absolute truth?
And I discussed it even further in the following consideration:
out of its self-positing as this,
self-construction follows, and vice versa;
because if it is posited as this,
as remaining after abstraction from everything else,
then it is posited as remaining and persisting because of itself;
since if it were not because-of-itself,
it would be because-of-another,
from which it would not then be possible to abstract
in its true original creation,
or which could not be absent for this creation.
Conversely, if it is a true, actual, energetic from-itself,
then it is not from another;
since then it would not be truly from itself.
Therefore, it is necessary to posit it, as it has been posited.
But let us take a keener look at this reasoning itself
and the procedure within it.
(And I remind you that this is the most difficult
and significant thing that has so far come before us.)
First of all, without exception in our whole argument process
and in the entire conduct of our lectures up to now,
the absolute has been treated as what is left over
after abstraction from everything manifold;
and if equally we have expressed specifically enough
the absolute from-itself and pure oneness in-itself,
then with these words which we have added as clarifications,
we have surely again made use of this same relation;
as more certain proof that even we ourselves,
the scientists of knowing and what we actually did and pursued,
found ourselves in the previously uncovered contradiction
between saying (of the from-itself)
and doing (explaining by means of the not from-itself).
Thus the first premise of our proof here reads:
“If it is posited as this,
as left over after abstraction
from everything else ...”;
which is a sure demonstration of reconstruction.
Second, in the center of our entire proof
we have absolutely presupposed both genesis
and the absolute validity of the Law of Principles.
The center of the proof was,
“If it is not from another, then it is from itself;
and if it is not from itself, then it is from another.”
If someone now were to say to us:
“Quite right: one of the two
from itself or from another;
and in case one, then not the other,
if of course I grant you the use of your “from” at all.
But if I say instead:
in brief it is, and that's all,
who will then ask about a “from”?”
To be sure we can answer such a one as follows:
“You are reflecting; so in addition to this “is”
you also have consciousness;
you therefore have not one but two,
that you can never make into one
and an irrational gap lies between them;
you are in the familiar death of reason”;
so the loophole always remains open to him
that is taken by every non-philosopher:
“I must just stay in this 'from'
and it is impossible to escape it”;
so everything finally comes down to this
that we justify ourselves in the use of the “from.”
Therefore this would be our next task,
to justify the “from” in general as such,
entirely abstracting from its application.
So far, as I ask that you recall,
it has not arisen in any other way
than in factical necessity.

This justification will disclose itself
if only we rigorously pursue
the analysis of the preceding argument.
In the first half are to be found
the remarkable words that without doubt
became immediately evident and clear to you:
“if it were not because-of-itself,
it would be because-of-another
from which it would not then be possible
to abstract in the true primordial act of creating,
or which could not be absent for this creation”;
and yesterday I also added:
“even for truly primordial creation,”
since through thoughtlessness and foolishness
one could easily forget the other through
which alone the first can be.
What then is understood by this primordial creation
which likewise in total tranquility
provides the center of the proof?
Evidently that our thinking, or the light,
if it should be of the right kind,
must accompany the genuine real creation of things
and originate along with it:
hence if the one were to be through another,
it would have to take the “through another”
up into itself and express it;
contrariwise, a thinking that omitted this “through”
would be mere thinking and not absolute,
and would set down a true creation
only factically as bare, dead existence.
This was the first point.

Now it seems here as if the real creation, as real,
could exist on its own and go its own way; and some assert it.
The basis for this illusion has in fact been grasped here.
That is, it rests in the possibility of viewing
primordial creation too in a pale and factical way,
as a result of which it seems to be capable of
existing independent of, and separated from, its appearing.
But we have already seen earlier that light and inward being
(by no means the external existence created by faded thought)
are entirely one and the same;
or, in case we had not yet realized this,
then this is the place to prove it immediately;
because if absolutely unchanging and unchangeable
self identical light must accompany creation,
then there is no light without creation
and creation is likewise inseparable from light:
since it is only because of the light and in the light.
Creation = “from,” “through,” etc.,
so absolute light is itself an absolute “from.”
This was the second point.

Now we, the scientists of knowing, have tacitly
presupposed this as the inner principle of
the possibility for the entire subordinate proof procedure,
which we are now dropping,
and indeed, this is the important thing,
we have done it without any design or plan
before the deed and immediately through the deed itself.
But I claim that the bare possibility
of this presupposition shows its truth and correctness.
Let me prove this first indirectly.
We ourselves in our doing and pursuing are knowing,
thinking, light, or whatever we wish to call it.
If knowing were now absolutely limited,
to the faded thought of an existence separated from thinking,
then we could never have been able to get out of it
to this presupposition of an absolute creation.
Since we have really posited it,
and the light as absolutely one with it,
since we ourselves are immediate light,
we have quite certainly validated the truth
of our claim in immediate being, in action,
since we enacted in that place the very thing we said,
and said what we enacted;
and the one could not be without the other.
Results:

1. The contradiction we have noted up to now in
what we ourselves are and pursue,
between doing and saying = the real and the ideal,
is now annulled, as it alone can be,
ipso facto in us ourselves,
and since this is the criterion of pure reason,
we are ipso facto pure reason.

2. Light has a primordial conception of its own nature
that ipso facto preserves itself in immediate
visible completion of itself.
(Note well: here we are holding simply to the
immediately evident content of our sentences.
It is obvious that questions can
still be raised about their form.
These questions will raise themselves,
and the basic principles for deriving relation
from the absolute may well lie just in answering them.)

3. On the very grounds given, let us leave our
factical conception of the nature of the light,
which may well give rise to the entire we
whose origin we are seeking,
and let us hold simply to the content.
In light, absolute genesis.
Obviously, the light, as light, is qualitative oneness
(which in fact enters as just plain seeing
that cannot further be seen),
which permeates the entire
inner genesis as bare pure genesis
(I am relying here on your [powers of] penetration;
since language can in no way bring us to our goal).
I can now construct this for you further as follows:
this oneness permeates the duality in the “from a-b”;
which duality exists only in the absolute “from”;
but not at all outside it in some independence
and [in an] independent differentiation of terms
so that [the duality's terms] may be reversed
with complete indifference.
These all are constructions in sensory terms,
through which I anticipate myself.
The ground of their possibility must lie in
and be derived from me myself,
insofar as I am the factical concept.
In the strictest sense, nothing matters more than this:
light is the qualitative oneness that penetrates the “from.”
This was the first point.

Now likewise, following our concept,
this “from” and (just for that reason and consequently)
the light's permeation of it,
and also therefore the entire
qualitative oneness of the light,
that indeed can only be thought in relation to
a “from” and its duality in order to annul it;
all this, I say, has its ground in the light itself,
no longer as qualitative but rather as an inscrutible oneness.
Therefore, there exists between the light in itself and
the entire preceding relationship
a new, only entirely one-sided “from”;
and this latter denotes the absolute effect of the light;
to the contrary, the entire first relationship simply shows
the appearance of this effect,
of the qualitative immediately self-effecting light.
This was the second point.

As a genesis, every “from” posits light;
just as previously light posited genesis:
and indeed, since the absolute “from” of
the pure, inaccessible principle rests here,
it posits absolute light,
without the genesis ever becoming visible,
and [posits] itself only
in this absolutely factical light
and from this factical light.
If you have seen into this,
then reflect now on yourself.
We have seen into the “from” in just this way,
and by means of it have seen into the 0
whose inaccessibility we have previously admitted,
and have seen into it as unconditionally existing,
objective and so as having to exist,
if appearance is to arise.
This is the fact.
How have we explained it?
Thus: there is an absolute, immediate “from,”
which as such must appear in a seeing,
itself moreover invisible.

Hence we ourselves,
with the whole content
of our immediate seeing,
are the primordial appearance
of the inaccessible light
in its primordial effect,
and a-b is mere appearance of appearance.
And so the primordial facticity,
the original objectification of reason,
as existing and genetic, is thereby clarified
from the original law of light,
and our task has been completed
in its highest principle.
I have no reservations in letting you go
for the week with these provisions.
Monday [21 May], a discussion.

B. APPEARANCE

IV.11
atita-anagatam svarupato 'styadhva-bhedad dharmanam

Twentieth Lecture
Wednesday, May 23, 1804
Honored Guests:

Being is an unconditionally
self-enclosed, living oneness.
Being and light are one.
Since in the light's existence
(= in ordinary consciousness)
a manifold is encountered
(we have initially expressed our problem empirically
and we must continue to speak this way until it has been solved),
a ground for this manifold must let itself appear
in the light itself as absolute oneness
and in its manifestation,
a ground that will explain this entire manifold
as it occurs empirically.
“In the light and its manifestation”
I have said:
therefore we must first of all derive
the appearance of the light from the light,
[and] the manifold will arise in the former.
This is roughly the main content of
what has been achieved so far
and of what remains to be done.
This is to be noted especially:
[the task is] to present appearance in general and as such.
(Obviously, as soon as appearance has been explained
and the principle of the manifold has been explained
from it a priori and in principle,
all appeal to empirical experience falls away,
and what was previously held factically
will be conceived genetically.)

At present we have already pushed ourselves
quite near our highest principle.
If transcendental insight has been opened for us,
then having in mind the most recent link in the chain is
sufficient for understanding the lectures;
if the earlier links by means of which we ascended
to the later ones are not equally present, nothing is lost;
we will rediscover everything anew on the descent.
I must bring you back to this last link by repeating
the previous lecture, at the same time I will also expand and add.
It has already been proven earlier that the absolute,
simply as absolute, must be from itself,
whatever else it may be
(earlier it was “being,” “light,” and “reason,”
none of which mattered to this argument and did not belong to it);
and this proof further coheres with the postulate
that inner being could not be constructed from
outside but must rather construct itself,
with which postulate we opened the entire
so-called “second half” of our investigation.
(In this fashion, everything achieved so far toward
bringing out the second part, and with it the whole,
could also again be reproduced.)
In the last hour, this completed proof was
itself investigated in its central nerve
and points of manifestness,
and what turned out to be its foundation was
the simple presupposition that genuine true seeing,
or light, must have accompanied the actual Creation;
and since being and seeing had already been grasped earlier
as being the same, that genuine true light must itself be
an immanent creation, or an absolute “from”.
This, I say, turned up as a mere presupposition,
grounding the process of our proof regarding
the essence of the absolute, but itself based on nothing.
Still, a little reflection shows us that
this presupposition proves its correctness
simply by its mere possibility and facticity;
because we ourselves were the knowing,
insofar as we conducted the proof
and made the fundamental assumption
concerning the essence of knowing:
that it is a “from”;
and, note well, we certainly could be,
and indeed are satisfied,
that knowing cannot be both
something in and for itself
apart from any view into itself
and also a “from,”
but rather that it can be both
only within [such a] view.
By the actuality of this view within ourselves,
we have proven directly and factically
that in this respect it is so.
It is, and it is this;
because it quite certainly is,
and quite certainly is this,
and we ourselves, the scientists of knowing,
are it as such.
This is an immediate demonstration of the essence of knowing,
conducted through the fact itself and its possibility.
At this point let yourself take in even more fully
what was established last time,
although only in passing:
we did not make this presupposition
because we wished to,
or with any sort of freedom;
and if only this free element,
which is to be summoned in response to
some particular reflection,
is to be called We,
then we actually did not make it;
rather it made itself directly through itself.
All our preceding investigations have started from
the fact that we were requested to think energetically
about something we were aware of internally and
also were able to ignore;
so both took place only in consciousness;
this provided our premises, and, to be sure,
this energetically considered object was always
accompanied by the explicit supplement of a “should”:
“If this should be so, then ____ .”
From our thinking this premise energetically,
manifestness grips us without any assistance on our part,
and carries us away, attaching to the controversial premise
that conditions it and is conditioned by it.
Therefore, the knowing, which we pursued in this way,
instantiated the basic characteristic mentioned before
of being merely reconstructive
and was, in this reconstruction,
a secondary and merely apparent knowing,
transferring knowing's implicitly unconditional content
into a conditioned relationship.
All systems without exception remain fixed in this knowing;
their premises therefore are hypothetical for them only
(but not absolutely in reason,
by which even they themselves are driven,
although to be sure without their knowing it)
and the relationship alone is evident,
which however gives no final or fixed manifestness,
since the relationship itself depends on
the reality of its terms.
They supplement the lack of this strength
only by an arbitrary reliance on the premises
and by averting their eyes from their difficulties;
without this reliance they could release themselves
to absolute skepticism at any moment.

IV.12
te vyakta-suksma guna-atmana

Up to this point [it has been] this way.
Now, absolute manifestness has extended itself to the premise,
to the absolute presupposition itself;
and thereby it has annulled both all freedom
and every “We” that was presupposed as a premise
in relation to the secondary manifestness of the context.
Hence, we are transposed into a
completely different region of knowing,
not simply as something purely self-grounding,
but rather immediately, and ipso facto from itself.
But as for what relates to the premise as a premise,
undoubtedly in this quality a consequence is posited through it,
and these two in turn posit a relation;
therefore, in this quality,
it serves admirably to
explain secondary knowing;
and, since it is absolute,
to explain the latter from the absolute,
which is exactly our task.
As a premise, it is undoubtedly
the principle of appearance we have been seeking.
But since appearance is not itself
the purely self-contained absolute,
as becomes evident in the premise:
the simple fact that it requires
a consequence and a context
shows it to be insufficient by itself;
there must therefore be a higher notion of knowing.
This remark can cast a great deal of light
over what we still have left to achieve,
so I want to analyze it further.

Now we let go of the point that we presuppose it,
or more accurately, that it posits itself as a presupposition,
and thus let go of the proposition's form
for reasons having to do with method,
and simply hold on to the content of the proposition:
“the light is an absolute “from”,”
analyzing what we actually mean by this.

All along, and obviously in this proposition,
light is posited first and foremost as
an essential, qualitative, and material oneness,
not further conceivable, but instead only to be carried out at once,
just the way we carry it out in all of our knowing,
from which we cannot escape.
I want to be understood on this point that is easy in itself,
that requires simple, strong attention.
Question:
what then is knowing?
If you know, then you just know.
You cannot know knowing again
in its qualitative absoluteness;
since if you did know it,
and even now were knowing it,
then for you the absolute would not stay
in the knowing that you knew about,
but rather in the knowing by which you knew it;
and it would go on this way for you
even if you repeated the procedure a thousand times.
It remains forever the same, that in absolute knowing you
recapitulate knowing as essential qualitative oneness.
Initially this insight needs only to be carried out;
reflecting on the law of its completion still remains before us.
This light is now absolutely presupposed as a “from”
without prejudice to its qualitative oneness;
since [if it did not preserve this oneness] the light would
not be “from,” and so [it is presupposed] as permeating this “from.”
Notice first what is new and important here:
it is presupposed to be like this, unconditionally.
Thus it has presupposed itself in a particular act;
and this presupposition is now proven by its facticity and possibility,
and further by the possibility of a deeper determination of knowing,
which opposes the simple existential form
in its attachment to the mere dead “is.”

In our insight, it in no way follows from our insight
into the essential light as such
(which we once again should have been grasping
energetically and freely),
by means of which manifestness descended into the connection
between light as such and the “from.”
And we again lapsed into secondary and merely apparent knowing
with which there must, to be sure, eventually be an end,
an end we have sought so avidly from the beginning.
“It does not follow,”
I said, since, as we have seen, there is generally
no such insight into the light in itself,
hence, I said: the light posits itself
as a “from” in a particular and absolute act or genesis;
an act that cannot be mastered immediately in this genesis,
as the genesis of genesis,
because otherwise the genesis would
not be an absolute genesis.
(What this latter means, and does not mean,
since here as well is yet another disjunction,
will show itself it what follows.)
I say: according to the preceding observation,
it thus posits itself absolutely;
the act is a self-contained, self-sufficient act;
it is posited by us merely in our inferential chain,
which we now entirely let go of,
as the mere means by which we have
ascended to our present insight,
until we find it again on the descent.
This is the first, and significant, point.

That the light in its changeless qualitative oneness
is a “from” therefore means:
it is a qualitatively changeless
permeation of the “from.”
In the previous hour,
we made the following application of this point:
disjunction is found everywhere in the “from”;
absolutely out of, and from, the “from”;
by no means presupposing terms
that were primordially different
independently of the “from,”
instead [they were] produced absolutely as terms,
absolutely distinguished as such
only through the “from”
and otherwise through nothing at all.
The one, eternally qualitatively self-identical light,
by virtue of its identity with the “from,”
must, in this qualitative oneness,
spread itself over these terms,
whatever their distinction from each other.

IV.13
parinama-ekatvad vastu-tattvam

Let me now apply and animate this insight right away,
and thereby make it unforgettable for you.
A “from” is posited immediately through the light:
L
a — b

Hence, if light exists,
then necessarily there is also a “from.”
Now if the light is identical with the “from,”
then, as surely as it itself exists,
it spreads itself in unchanged qualitative oneness
across every “from,” and comprehends every “from.”
And if one again posits within the first “from”
another one that is deducible and conceivable
on the basis of the original synthesis of
light and the “from”

a — b
|
|
a — b  a . b

it is completely clear that
the same original light, qualitatively unchanged,
by means of its identity with the original “from,”
must at the same stroke accompany
all subordinate divisions of
the original “from” into further “from's.”
And it is also clear that whatever possesses
the principle of this secondary splitting
of the original “from” accompanies
this progression of the light
as entirely necessary and at one stroke,
and can reconstruct it purely a priori
and without any empirical presuppositions;
which indeed is the second and subordinate task
of the science of knowing,
since we now are pursuing the much higher task
of presenting the principle of this principle.
This “from,” in pure, absolute, immediate oneness
and without any disjunction,
as the pure self-positing of the original light,
is the light's first and absolute creation;
the ground and original source even of the "is,"
and of everything that exists;
and the disjunction within this “from,”
in which true living perishes
and is reduced to the mere intuition of a dead being,
is the second re-creation in intuition,
that is, in the already divided original light.
And thus the science of knowing
justifiably presents itself
as the complete resolution of
the puzzle of the world
and of consciousness.

IV.14
vastu-samye citta-bhedat tayo vibhakta pantha

This, I asserted, was the next application
that I made last hour of the proposition,
“The light is a 'from,'” attending to
the disjunction in the “from.”
But it is even more important to look at
the essential and qualitative oneness of this “from”
and at the words that were said previously
about the original creation.
Recall them.
In its pure qualitative oneness,
“from” is genesis:
that the light is identical with it
and permeates it in this its essence, means:
in this its second power (namely its appearance),
it is itself genesis;
genesis and seeing converge together
completely and unconditionally.
The words are easy to understand;
it is not so easy to give them the deep meaning
intended here in living insight;
and it is nearly true that the only way
I am able to guide you forward is with an example.
The subject matter that I wish to
present to your intuition appears
in every transition from lassitude to energy,
and for our purposes,
the example cited just above will serve best:
the one in which we had tacitly presupposed
absolute knowing to be a “from”;
when interrogated about the justification
of this presupposition,
we recalled that indeed we knew ouselves
in this presupposition and were the knowing.

IV.15
na ca-eka-citta-tantram vastu tad apramanakam tada kim syat

I ask:
does not this new awareness,
that was not yet there prior to our presupposing,
seem as if it were a popping up, a new production?
Now at this point you are
certainly able to abstract purely,
as is my present demand:
that this is consciousness,
is a consciousness of knowing,
and what is more of knowing as a “from.”
What remains for you after this [act of] abstraction?
Evidently just a knowing/seeing/light,
exactly absolute, qualitative,
as it has already been described,
and [it was] this therefore because
you abstracted completely from all content,
which you could do according to the presupposition;
consequently, [it], as itself light, conducted
the proof of legitimacy empirically;
further, [it is] a consciousness of absolute genesis.
Now (note this addition, the proof becomes more rigorous
and the insight purer through it)
you can more suitably posit this genesis or freedom
in the act of abstraction from all content
of the presented consciousness,
which is thus required of you.
As things stand, it is immediately clear to you
that this pure light, as it has been described,
could not arise without abstracting from all content,
nor can the latter appear without arriving at pure light;
that therefore the appearances of both terms are indivisible,
and permeate one another;
and that hence pure light appears as permeating genesis,
or as producing itself.
By means of this proof more is
nearly proved than should be proved,
and future research is anticipated,
as I note in preparation;
the light's positing of the “from,”
and the fact that it posits itself as a “from,”
has already become immediately visible.
What we have to be concerned with next here
can be shown with a little preparation in two examples.
Because you were instructed to reflect energetically
and a new consciousness emerged for you,
this new consciousness is not to exist
as something new without the energy;
this consciousness and the energy should
open up together indivisibly.
Now you certainly posit genesis here
partly in yourself,
in the energy of your reflection,
and partly in the essence of reason itself,
since the manifestness is to emerge
without any further action on your part;
but this entire distinction ought to have no validity in itself,
and it should be abstracted out, and so,
(leaving undecided whether genesis's true principle
lies in me or in reason itself)
there always remains an absolute, self-producing knowing
that does not possibly occur without the genesis.

IV.16
tad-uparaga-apeksitvat-cittasya vastu jnata-ajnatam

Now this means, as was said before,
that light permeates the “from”
in the qualitative oneness of its (the “from's”) essence:
the presented intuitions of this penetration
were only explanatory means.
But, independent of all facticity,
we have seen a priori
that if light is to be,
such a permeation is necessary.

This is the one side of the previous proof
for the content of the sentence:
the light = “from,”
that we have repeated and enriched today.
There is still another,
and of this more tomorrow,
equipped with today's new discoveries!

In conclusion another comment about
the whole of the science of knowing,
one that I share with you not so much for your own guidance,
since I hope you do not need it,
but rather as a weapon of defense against the ignorant.
Already earlier, and again today in passing,
the proof of knowing's essential criteria is conducted
on the basis of our capacity to see it thus.
The nerve of the proof is clear:
we ourselves are knowing;
since we can know only in this way,
and presently actually know thus,
then knowing is constituted so.
It is equally clear that the failure
to discover this principle of proof,
or not paying attention to it once it has been found,
grounds itself on the truly foolish maxim
of searching for knowing outside of knowing.
Concerning this, nothing more needs to be said.
I would only bring this to your attention:
the proof simply does not succeed for anyone
who is really not able to make clear and intuitable
what can only be made so by his own capacities;
through his incompetence he is
barred from the subject itself
and from any judgment about this world
that is entirely concealed from him.
It is the same for those who could but will not, that is,
who will not submit themselves to preliminary conditions
of sharp thinking and strong attention;
because everyone who can, will do the thing itself;
and everyone who will, can do it.
This is true when the science of knowing
does not yet stand at the apex.
One should not therefore wonder
how that which has in itself
the highest clarity and manifestness,
cannot in any way be made clear and true
for very many people;
one can rather himself lay out
the grounds for this impossibility,
if they will just come to understand the premise
that there might be something they do not now know;
and that they are not able to know directly
and without much preparation and strong discipline,
as things are with them now.

IV.17
sada jnata citta-vrttaya tat-prabho purusasya-aparinamitvat

Twenty-first Lecture
Thursday, May 24, 1804
Honored Guests:

(We will make use at once of what we have already understood,
and take a shortcut without further repetition
and closer definition of subordinate terms.
You know that such a thing is possible
in the science of knowing, and why.
That is, [it is possible] because
the subordinate terms will recur
in their full developmental clarity during the descent;
and the ascent is undertaken not for the subject matter itself,
but for clearing our vision
and opening it to the absolute
by abstracting from all relations.)

I connect this with what has gone before:
the light has been presupposed as an absolute “from.”
Then we immediately proved the legitimacy of
this presupposition by means of
its bare possibility and facticity,
because we ourselves were light and knowing.
Based on this last key step in the proof,
the presupposition is true and legitimate in the “We”;
not, of course, in the previous “We” that freely posited premises,
(since in this case knowing posits itself,
as was clearly explained yesterday).
Instead, [it is true and legitimate] in the We
that merges into the light,
and is identical with it.
Moreover, it truly is just as it factically occurs,
but it occurs as a presupposition.
Hence, taken strictly
(as we have not so far taken it, and for good reasons)
it has been truly and factically proven that
the light can presuppose itself as a “from,”
and that in us it actually does so.
In us, to the extent that we have merged,
and disappear identically, into the light itself =
[we] are the science of knowing.
Unnoticed, this presupposition has made itself,
and we will build on that.
But [the We] on the occasion of which it made itself,
has in that sense not even made itself,
instead we, who are freely abstracting and reflecting, have made it.
Consequently, by this “We” one may well mean that
light makes itself into a “from” only
in the science of knowing, as a higher, absolute knowing;
and so we provisionally indicate a distinguishing ground
(for which we have been searching)
between lower, ordinary, empirical knowing
and higher, scientific, genetic knowing.

We said, “It is presupposed.”
However, all presuppositions
bring along a hypothetical “should”;
and let themselves be expressed through it.
In fact, we have not argued differently than this
in the two previous sessions
when analyzing the contents of this “from”:
“Is there light” = “if light is to be”
and “Is there an absolute 'from'” =
“If there should be an absolute 'from,'
then must ____ ,” etc.
However, we have not only presupposed
the absolute oneness of the light hypothetically;
instead we have also realized it unconditionally.
To be sure, we have done so only in its qualitative character
(which as you will remember, was itself a result of the “from”)
as was the case with the absolute origin in knowing
as permeating the “from” in its qualitative oneness,
as we discovered yesterday.
Hence, both are a result of the hypotheticalness,
so that only pure, bare oneness,
henceforth presented as inconceivable
and understood as categorical,
remains left over.
I wished to undertake the delineation of
this very boundary in passing,
and it is commended to you.

Now back.
1. Our reasoning has proceeded in
the hypothetical form of a “should”;
and this to be sure unconditionally as itself knowing,
and as primordial knowing,
since knowing itself has posited this “from,”
then transcended this posited and objectified “from,”
which we analyze from below and derive from it.

(This as well about method.
Obviously we are once again reflecting about
what we were and did in the previous presupposition and analysis,
in the same way we have proceeded in our entire ascent;
and I could have proclaimed our activities in just this form.
Purely because we have left the realm of arbitrary freedom behind
and have arrived with our own effort in the realm of organic law,
I preferred to compel you to the present reflection
through the reminder that indeed everything
grounds itself simply on the presupposition,
rather than appealing to your freedom.)

2. In its innermost essence,
a “should” is itself genesis and demands a genesis.
This is easily understood;
you ask, If such and such should be,
then is it or isn't it?
The “should” tells you nothing about this.
What then does it say?
It sees a principle;
therefore, it explains categorically
that being can be admitted only on
the condition of the principle.
Thus, only genetic being,
or being's genesis, can be admitted.
Thus, it is the absolute postulation of genesis;
and since everyone whose
transcendental sense has been awakened
will allow no genesis to be
valid in and for itself
without such a postulation,
even immediately absolute genesis,
and the genesis of objective genesis only mediately,
according to a law that we have yet to exhibit.
Or this as reinforcement:
I have said [the “should”] is
the postulation of genesis.
Now it is immediately clear that
the “should” is a postulation,
and that a postulation is a genesis,
at least an ideal one;
otherwise, it is, as such,
completely incomprehensible and
accordingly the addition “of genesis”
would not be worth while in any way.
So it is evident that, in our hypothetical “should,”
being's genesis is demanded, which, as a genesis of being,
the hypothetical “should” is content not to be able to provide.
Instead, it waits for it from a principle outside itself.
The demand, as itself a genesis
(ideal, as we have called it, in order to name it
only provisionally with this partially clear term),
however, lies in the “should,”
and the “should” is it.
Thus, there can be a disjunction
within the absolute genesis itself,
through which it would be real and ideal.
This entire disjunction,
discovering the basis of which
may well be our most important task,
can now follow from genesis,
or the “should.”
Through resolving this disjunction,
those words, which we have used so far only provisionally and
according to a dim instinct in the hope of an eventual clarification,
will themselves become clear.
This is merely a hint at a part of our system
that necessarily must remain obscure here.

However, the following is
completely clear in what has been said:
in virtue of yesterday's demonstration,
genesis = the “from” in its qualitative oneness.
We ourselves, or knowing and light as such,
which are entirely the same as us
on the level of our present speculations,
are this “from” immediately,
in that which we pursue and live.
So there is no further need for the “from”
that is posited and presupposed through
some specific act of ours or of the light,
nor for anything that we have derived from it in our analysis.
Therefore, we let it all go as just a means of ascent,
until it shows up again on our descent.
I said, “in that which we pursue and live”;
and this very pursuing and living,
as pursuing and living,
follow directly from [our] dissolution into genesis.

3.  By virtue of the hypothetical “should,”
the we, or knowing, is absolutely genetic
in relation to itself.
Because we ourselves were knowing,
we pursued it in the following way:
“should knowing be
(that is, should we ourselves be,
since we ourselves are knowing),
then ____ must” and so forth.
Thus, [it is the] genesis of nothing else,
but rather of itself,
of the simultaneously productive [one].
Thus, with this it is absolute genesis,
which carries in itself the already sufficiently
seen character of being or light:
that it is completely self-enclosed
and can never go outside itself.

4. This absolute self-enclosure
of genesis in its fundamental point
(in which it should be a genesis of genesis)
does not prevent two points of origin
or two knowings from appearing impermanently.
We ourselves conduct one when we say,
“If knowing (or we ourselves) should be ____”;
and the other one should be,
if its principle is fulfilled.

I regard insight into this distinction of two aspects of knowing,
a distinction that is still only factical, as simple.
Yet, it is so important that I cannot
very well leave it to mere luck,
and so a bit more by way of elucidation.
We ourselves are the absolute light,
the absolute light is us,
and this is genesis itself.
Nothing can depart from this;
therefore, a distinction cannot be admitted
within the subject matter itself,
without contradicting our first fundamental insight.
Hence, the disjunction that remains is not
a disjunction between two fundamentally distinct terms,
instead it is a disjunction within one,
which remains one throughout all disjunctions.
Something of this sort has already presented itself to us earlier.
Stated popularly: it is not a disjunction of two things,
but rather just different aspects of one and the same thing.

IV.18
na tat sva-abhasam drsyatvat

5. Letting this disjunction stand provisionally,
just as it has appeared to us factically,
with the intention of working further
on the basis of it, the question arises,
which of the two aspects is to be
considered provisionally as absolute
in order to explain the other from it?
[It is] obviously the first term,
in what we ourselves live and pursue,
given the insight aroused in us yesterday,
that seeing and light reside always
only in immediate seeing itself and
never in the seeing that is seen.
By no means could it be in the objectified is,
that waits [to receive] its being from a principle
and is therefore truly dead within.
This choice can be shown to be completely necessary
through another circumstance as well.
For if we wish to work further,
then we wish to pursue and live knowing further as well.
Therefore, in fact and absolutely, we must remain in life
and cannot abstract from it, as is evident.
To do so would exactly be not to live and search further,
but instead to remain here,
which would contradict our intention
not to stand still and instead to go further.

(In passing, this aspect is the one
that we have always called idealistic.
Thus, our science, standing between
idealistic and realistic principles,
would at last become idealistic,
and indeed, as we have seen,
be forced to do so by necessity,
and contrary to its persistent
preference for realism.
We will not promise that the matter
will come to rest with this principle,
as it now stands and as it will
at once be explained more clearly.
We can promise more confidently that
we will never again use objectivity as a principle.
From this it will follow that,
if the idealistic principle too
should prove inadequate,
we will need to find a third, higher principle
that unites the two.)

6. Just as is demanded in this principle of absolute idealism,
the inner self-genesis is presupposed
as a living inward oneness
(what this is, on that point you understand me)
as oneness, thus as light, qualitatively absolute,
only as something to be enacted
and by no means as something to be understood.
This latter [oneness is to be presupposed] as genesis
(as was made obvious yesterday in each transition
from dull to energetic thinking),
as disappearing into the arising
of an absolute “from”
that in turn merges into it,
so that seeing and this arising are entirely inseparable;
that is, as genesis of self, or I,
so that accordingly what emerges in immediate light
may be an “I” as a result of which,
light and this very We, or I,
would merge purely into one another.
The principle demands this:
inner self-genesis is to be presupposed
as intrinsically living oneness,
then also knowing's objective aspect is
allowed to stand and to be united
with the previous [aspect],
as it can be united in knowing alone.
From the genetic principle,
it would then follow that a principle must be assumed for
the absolute, inner, and living self-genesis,
and that this latter [event must occur]
in a higher knowing that united both, as is obvious.
This latter knowing is then the highest,
and the two subordinate terms are
merely what is mediated through it.

IV.19
eka-samaye ca-ubhaya-anavadharanam

That in the higher knowing
a principle is presupposed for absolute self-genesis
means that inwardly and materially
this higher knowing is non-self-genesis.
Yet it does not not exist,
rather it exists actually and in fact;
thus [it is] positive non-self-genesis;
and yet it is immanent and is itself an I,
because this is its imperishable character, as absolute.
What else is negated besides genesis?
Nothing, and to be sure this is negated positively;
but the positive negation of genesis is an enduring being.
Thus, knowing's absolute, objective,
and presupposed being becomes evident
in this higher knowing, hence directly genetic,
as it has previously appeared merely factically.
Once again, in order to review the terms of the proof:
as a result of positing a principle
for self-genesis within knowing itself,
[we derive] the explanation of genesis as not absolute;
consequently, [we infer] its positive negation within knowing;
and consequently [we also infer] the positing within knowing
of knowing's absolute being.

If you have just grasped this rigorously,
then I can add something else in clarification.
This knowing, that only ought to exist, is of course
a self-genesis of knowing, its self-projection beyond itself,
as we, who are standing over it reconstructing
the process and its laws, very well understand.
However, the question still always remains as
to just how we arrive at this insight
and so apparently get outside of knowing.
Yet, in contrast to an absolute self-genesis,
which is itself annulled as absolute
by the addition of a principle,
the immanent knowing that never can get outside itself
for just that reason can never appear as self-genesis
but only as the negation of all genesis.
Here, therefore, there is a necessary gap
in continuity of genesis,
and a projection per hiatum,
but here presumably not an irrational one.
Rather, it is [a projection] which separates
reason in its pure oneness from all appearance,
and annuls the reality of appearance in comparison with it.

IV.20
citta-antara-drsye buddhi-buddher atiprasanga smrti-sankara ca

“Reason,” I say, in order to clarify this for us;
in this case we were only concerned with deriving
the form of pure being and persistence.
In our case, this persistence is now,
and certainly always and eternally, genesis.
This existing knowing, which to that extent
is not genetic as regards its external form,
is enclosed in itself in unchangeable oneness,
and so indeed [is] also genesis,
just as it seems to be above.
Thereby the absolute, inward awareness declares itself,
without any external perceiving, knowing, or intuiting,
all of which fall out in self-genesis,
[an awareness] of an original principle
and an original principled thing
in a one-sided, and certainly not reciprocal, order;
or pure reason, a priori,
independent of all genesis,
and negating it as something absolute.

Let us go further:
in what we most recently lived and pursued,
we ourselves have not become pure reason itself,
nor dissolved into it, instead we have merely
deduced it from an insight into it.
However, this was possible only to the extent
that we presupposed self-construction as absolute, as we did;
because what should follow, follows only
on the condition that it [that is, self-construction] is
annulled as absolute in itself, from itself, and through itself;
and this was the center {Nerv} of our proof.
Since it was mentioned previously that the higher knowing,
which projects reason, is also at bottom self-genesis
and consequently does not just appear to be,
we can very appropriately call this self-genesis
the reconstruction of the non-appearing original genesis,
thus the clarification of the terms of the original genesis,
hence [we can call it] the understanding.
Accordingly, it follows for us that
there is no insight into the essence of reason
without presupposing understanding as absolute;
conversely, [there is] no insight into
the essence of understanding except by
means of its absolute negation through reason.
However, the highest, in which we remain,
is the insight into both,
and this necessarily posits both,
although [it posits] the one in order to negate it.
From this standpoint, we are the understanding of reason,
and the reason of understanding, and thus both in oneness.
Now the disjunction stands forth in its clearest definition.
Just one more principle and the matter will be completely explained.
[We will talk] about this, next Monday.

This besides; I regard what I have just
presented to you as not at all easy.
However, that lies in the subject matter,
and we have to go through it sometime,
if we want to see solid ground.
I can promise you a bit more illumination on this
from an insight into the principle we are still seeking,
but then the difficulty will lie in the principle itself.

One cannot speak properly in front of others about
speculation in these heights freely and without preparation,
since one has enough work speaking of it in formal, prepared lectures.
For this reason, and in order to escape our mutual impulse
nevertheless to handle this matter freely,
[we will now have] a special discussion period.

C. ACTUALITY

IV.21
citer apratisankramaya tad-akara-apattau svabuddhi-samvedanam

Twenty-second Lecture
Monday, May 28, 1804
Honored Guests:

Although I can justifiably report that
our speculations now already hover at
a height not reached previously,
and have introduced insights
that fundamentally change the view
of all being and knowing
(and I hope that all of you who have historical knowledge
of philosophy's condition up to now will agree),
nevertheless, all of this is still only preparation
for really resolving speculation's task.
We intend to complete this resolution during the current week.
Hence, your entire attention is claimed again anew.
Whoever has completely understood everything so far,
and seen into [it] to the level of
eternally ineradicable, forever immovable conviction,
but who has not yet seen into,
and achieved conviction about
what is to be presented now,
such a person has achieved at least
this protection from all false philosophy:
he can set each of them straight fundamentally.
He also possesses some significant truths,
disconnected and separated from one another;
but he has not yet become able to construct
within himself the system of truths
as a whole and out of a single piece.
I now intend to impart this capacity to you,
and after that the main purpose of these lectures
on the pure science of knowing will have been achieved.

Whether one names the absolute “being” or “light,”
it has already been completely familiar for several weeks.
Since attaining this familiarity, we are working on
deriving not, as is obvious, the thing itself,
but rather its appearance.
The request for this derivation can mean nothing else than
that something still undiscovered remains in the absolute itself,
through which it coheres with its appearance.

We know from the foregoing
(which, to be sure has been discovered only factically,
but which nonetheless would have its application
in a purely genetic derivation)
that the principle of appearance is
a principle of disjunction
within the aforementioned undivided oneness
and at the same time, obviously, within appearance.
However, as regards the absolute disjunction,
I urge you to recall an analysis conducted right at
the beginning of this lecture series,
in which the following became evident.
If disjunction were to be found
directly within absolute oneness,
as is unavoidably required by
the final form of the science of knowing,
and is what we intend here,
it must not be grasped as a simple disjunction,
but rather as the disjunction of
two different disjunctive foundations.
[It must be] not just a division,
but rather the self-intersecting division
of a presupposed division
that again presupposes itself.
Or, using the expression with which we have designated it
in our most recent mention of it,
[it is] no simple “from,”
but rather a “from” in a “from,”
or a “from” of a “from.”
The most difficult part of the philosophical art is
avoiding confusion about this intersection,
and distinguishing that which is endlessly similar
and is distinguishable only through
the subtlest mental distinguishing.
I remind you of this so that you will not become mistrustful
if, in what follows, we enter regions
in which you no longer understand the method
and [in which] it should even seem miraculous.
Afterwards we will give an account of it;
but beforehand we actually cannot.

IV.22
drastr-drsya-uparaktam cittam sarva-artham

This much as a general introduction for the week:
Now back to the point at which we stood at the end of the last hour.
Absolute self-genesis posited and given a principle,
obviously within knowing,
which is thus a “principle-providing” [occurrence].
Thus within this knowing there follows
the absolute, positive negation of genesis =
completed and enduring being;
and indeed, because this entire investigation
concerns light's pure immanence,
our investigation has long ago brought in
a presumed being external to knowing,
knowing's [own] completed and enduring being.

Now We saw into this connection during the last hour
and see into it again here;
as is evident, we see one of the two terms
in and through the insight into this connection,
determined as such.
Thus, this latter is itself mediated,
and just we (= the insight we have now achieved) are
therefore the unconditionally immediate [term].

Two remarks about this.
1. I have just recalled again,
that here the inner being and persevering is knowing's being,
the very thing already recognized as absolute genesis
and [the thing] we have also already validated in the last hour
as a priori rational knowledge of an absolute principle.
At this point, we must hold on to
the fact that it is knowing's being,
even in case we should let go of
the addition as a shortcut in speaking;
because otherwise we will fall back into
where we were before, far removed from further progress.
Therefore, our entire chain of reasoning
must always be present to us, now more than ever.

2. It is said that every philosophical system
remains stuck somewhere in dead being and enduring.
If now a system derived this being itself in its inner essence,
as ours has done, by positing
a higher principle for absolute genesis,
whereby it then necessarily becomes
the positive negation of genesis,
and therefore [becomes] being;
if too this being is not the being of an object and so doubly dead,
but rather the being of knowing, and so of inner life;
then such a system seems to have already
accomplished something unheard-of.
However, we are required to see clearly here that
by this we have not yet achieved anything,
since even this saturated being of living is
again something mediate
and is derived from that
which alone now remains for us,
the insight into the connection.
Now let us apply this directly for our true purpose,
which we have long recognized.
Knowing's derived being will now yield
ordinary, non-transcendental, knowing.
Through our present insight into
the genesis of the former's principle
(that is, the principle of the recently derived knowing),
and through reflecting on this insight
we elevate ourselves to genuine transcendental knowing
or the science of knowing;
and [we have done so] not merely factically,
with our factical selves,
so that we are the factical root.
We have already been this since
the time when we dissolved into pure light;
instead [we have done so] objectively and intelligibly,
so that we, achieving insight factically,
at the same time penetrate the law of this insight.
Henceforth we have to work in the higher region
that has now been opened.
Only here will the principle of
appearance and disjunction
that we seek show itself to us;
which then should only be applied to
existing (= ordinary) actual knowing.

IV.23
tad asamkhyeya-vasanabhi citram api para-artham samhatya-karitvat

1. Now also this addition:
since the beginning of what we
have provisionally called the second half,
a hypothetical “should” has been evident
as simply creating a connection,
and as linking a conditioning and conditioned term
(which are both produced absolutely from it)
to a knowing, which must be originally present
independently of the “should”
and its entire operation,
if one just understands it correctly.
This could be called the first
sub-section of the second part.
Since we concerned ourselves with
an absolute presupposition about
the essence of knowing as an absolute “from,”
we wished to know nothing more about
this entire hypothetical “should”
and its power of uniting and joining,
as a merely apparent knowing.
We said at this point that so far we
(the we that still has not been grasped so far)
have concurred about the premise's arbitrary positing,
pointed out by energetic reflection,
and only the connection has made itself manifest
without our assistance.
Now the premise too presents itself without our assistance;
therefore in the premise too we coincide
with the absolutely self-active light.
Let's hold on to this.
We have been doing this for a long time
in our discussions about the “from,”
until I thought you were sufficiently prepared for
the higher flight that we began in the last hour.
You may take this as the second sub-section of the second part.
In the last hour, the bare connection presented itself again,
and (as we might suspect, but will more exactly show and demonstrate)
with it the hypothetical “should,”
from which we had already hoped to be free.
This should surprise us.
If this “should” has reappeared with the same significance
in which it was previously struck down,
then we have not advanced
and are just sailing on
in the seas of speculation without a compass.
Through the hint given recently as to
the difference between ordinary knowing
(based on the principle of knowing's being)
and transcendental [knowing]
(based in the genetic insight into this same principle),
it is probable that [the “should”] does
not arise in the same way.
Instead, the “should” that we have dropped
operates in ordinary knowing
with tacitly assumed premises;
in contrast, the “should” arising now
operates in transcendental knowing,
which grounds its premises genetically
as emanating from a “should.”
Therefore, in the preceding lecture,
we have begun a third sub-section of the second part,
and the two extreme sections come together
in the middle (with the premise) once again
distinguished by a duality in the premise.
By this means, the two outermost parts
(= transcendental and actually existent knowing)
would be the two different distinguishing grounds,
proceeding from the middle ground of the premise,
which both unites and separates them.
This is the compass that I would share with you
for the journey we have already begun.

2. I said that a hypothetical “should” appears
again in our completed insight.
To begin with, this is obvious.
“Given [that there is] a principle
for self-genesis, then it follows that ____ .”
Previously, we have found the two terms factically,
and to that extent separately;
but in the last hour, we have united them genetically,
according to our basic rules and maxims.
Because we comprehend them mediately
in the insight into their connection,
then, given this insight,
we no longer need to assume them factically.
They reside a priori in the insight,
and we can drop the empirical construction,
until perhaps it arises again in some deduction.

3. Now let us grasp this hypothetical character at its core.
According to the maxims and rules that we arbitrarily adopted
at the beginning of our entire science, and hence arbitrarily,
we appear to ourselves
(so it has been, and so it now is openly admitted)
as genetically uniting both terms.
Here is the inner root of hypotheticalness
(now fully abstracted from the hypothetical
character of the subordinate terms),
precisely the “should's” admitted
inner production, containment, and support of itself
as identical with the free We, the science of knowing.
This very inward hypotheticalness could be
what first shows itself and breaks through
in the subordinate terms' hypothetical character.
Hence, it is only a matter of
negating this inner hypotheticalness,
so that thereby the quality of being categorical
will be manifest in it,
and thereby we will justify our insight
in its truth, necessity, and absolute priority.
In that case, the inference,
made here only provisionally,
first achieves categorical validity;
namely, the inference that both terms
(knowing's self-genesis and its being)
occur only mediately in a genetic insight
into the oneness of both,
and in no wise immediately.

(A remark belonging to method:
One must not allow oneself to be distracted
[as to] how I legitimately assume this.
This is more necessary than ever,
since here the method itself becomes absolutely creative;
further, nothing besides a remark like this
can be adduced in explanation of what happens here.
Our insight has come about through
the application of the basic maxims of our science,
to apply the principle of genesis
thoroughly and without exception.
This has been required,
and it should prove itself.
Likewise, the maxims of the science of knowing itself,
and with it all of science, have been required,
and they should prove themselves.
Science itself should justify
and prove itself before it truly begins.
Thereby, the science of knowing would be liberated
from freedom, arbitrariness, and accident;
as it must be, or else one could never come to it.)

4. Without digression, we conduct the required proof,
according to a law that has already been applied, thus.
We could produce this insight,
and we actually did so; we are knowing;
thus this insight is possible in knowing
and is actual in our current knowing.
Just a few remarks about this proof.

IV.24
visesa-darsina atma-bhava-bhavana-vinivrtti

a. The genesis first accomplished by us is
an absolute, self-enclosed, genesis;
by no means is it a genesis of a genesis,
because it negates itself inwardly within knowing,
as we have shown in the last hour.
To us, however, who are contemplating further
and constructing the process in its laws,
it manifested as genesis.
In immediate knowing, however,
it was merely a persisting intuition,
as external in its result:
which explicitly was non-genesis = being.

b. The proof of absolute genesis was conducted purely
through its possibility and facticity,
and thus [is] itself only immediately factical.
In this case, therefore, facticity and genesis entirely coincide.
Knowing's immediate facticity is absolute genesis;
and the absolute genesis is (exists as a mere fact)
without any possible further ground.
To be sure, it must happen so,
if we are ever actually to arrive at the ground.

c. This is a cogent example of how much
in the science of knowing depends on one
always having the whole context present,
since the distinctions can be drawn
provisionally only through this context.
Knowing, as genesis, is proved factically in this way.
What then happened several sessions ago when
we proved knowing as a “from” factically in the same way?
Is the “from” something other than genesis,
and have we not conducted the identical proof?
Yet the present, factically demonstrated, genesis is
something entirely different from the one proved earlier.
You could grasp this distinction only by
noticing that in this case it is a question of
the genesis of absolute knowing in its fundamental construction,
whereas previously [it was a question of] the
genesis of its absolute self-genesis.
In that lecture, to be sure, I had to let this
criterion go and abstract from it,
looking simply at the core element of the new proof
(since otherwise we would never arrive at this new proof)
and relying on the solid insight already engendered in you.
However, you can add this criterion now,
and can use it to rebuild and reinforce the insight,
in case it begins to waver.
Later of course I will add inner distinguishing criteria,
e.g., for both these points of origin,
by which they can be distinguished in themselves,
independent of their relation.
However, they are not even possible, or comprehensible,
before the distinction in the thread of relatedness
has been completed factically,
because these inner distinctions are nothing but
the genetic law of the factical difference,
which arises only in the fact.
For just this reason, the science of knowing is
not a lesson to be learned by heart,
but rather an art.
Presenting it, too, is not without art.

IV.25
tada viveka-nimnam kaivalya-prag-bharam cittam

d. I also want to make you aware of the following.
Even in the recently conducted proof,
within whose content facticity and genesis should merge
purely into each other,
there still remains in the minor premise of the syllogism
the same term that arose above factically
and still has not been genetically mastered.
“We know, or are knowing.”
[This is] of course immediately clear and intelligible,
but its principle is by no means clear.
Investigations remain to be made here,
and herein lies perhaps the most important
part of our remaining solution.

IV.26
tad-chidresu pratyaya-antarani samskarebhya

Let me announce the process that will follow.
For good reasons lying in my art,
I will not proceed at once with this point,
but instead I will wait until it arises of itself.
On the contrary, I will add this also:
in the insight we have completed,
an insight has now indeed arisen for us
that is objective, immediately compelling for us,
as well as intrinsically determinate and clear.
It is this, if we abstract from the subordinate terms
as hypothetical and as themselves just the externalization of
the inner hypotheticalness of our performance
(of course, we need to abstract from this hypotheticalness as well),
then we intend to turn from the form of our insight
to its content, in order to explain the form from it,
as we have frequently done and as is actually a realistic move.

To repeat briefly:
the content of the insight we have recently achieved
must be clearly present to you.
Providing absolute self-genesis with a principle
yields absolute non-genesis = being.
As we just recently undertook to do,
today we have abstracted completely
from the two subordinate terms.
And, looking only at our insight itself
and the manner of its production, we have justified,
as was possible only in a factical manner,
the absolute application of both
the maxim of self-genesis and this procedure.
This is the essential content of the little, easily remembered,
bit that we have achieved for our topic.

IV.27
hanam esam klesavad uktam

Further, at many turning points we have made
extremely penetrating remarks about method,
which I urge you to keep in mind,
because only with them will you find
your way through the maze which we confront.

IV.28
prasankhyane api-akusidasya sarvatha viveka-khyate dharma-megha samadhi

Twenty-third Lecture
Wednesday, May 30, 1804
Honored Guests:

Providing a principle for absolute self-genesis,
as we have described the light, creates non-genesis,
or being (of knowing, it goes without saying).
We have seen this, have reflected about
the method for producing the insight,
and have justified it factically.
It is possible and actual within knowing,
because it is possible and actual within us,
and we are knowing.
However, no genetic derivation of
this last point has occurred just yet.

We can justify this procedure even more deeply
and from another point of view.
An objective, and absolutely compelling insight
has actually arisen for us;
this process hereby has also shown itself
to be coherent with the absolute, self-producing light.
Consequently, the task which we reported
at the end of the last hour as coming next
(to investigate this objective insight
in respect to its true content)
will simultaneously be our first attempt at justifying more deeply
what has up to now been presented only factically,
and perhaps even to make the point genetic.
And so to the content of the self-presenting objective insight!

1. Evidently, the absolute relation of both subordinate terms.
However, these are still hypothetical;
but without them, there is no connection;
it itself is the result of hypothetical terms,
and so itself is hypothetical;
a fact from which we should abstract.
What is still left?
Plainly nothing other than the insight's inner certainty;
and because even the insight as such depends on the terms,
[it is] nothing more than a purely inner certainty.

The first claim on you is to grasp
this certainty sharply and altogether purely.
It is not certainty about anything in particular,
as the relation of the subordinate terms was in our case,
because we have abstracted from that.
Rather, it is certainty pure and as such,
with complete abstraction from everything.

At first, accordingly, it is immediately clear
that certainty would have to be thought completely purely.
Moreover, the ground of its material whatness lies in the “what”
(that it is, that it is what it is, that it is certain);
but the ground for certainty cannot in any way be located there,
because this does not belong to the "what."
Therefore, certainty lies simply in itself,
unconditionally and purely as such;
and it is unconditionally of and through itself.
It must be thought in this way,
or else certainty is not conceived as certainty.

In passing, being is not a reality that is
derived from the sum of all possible realities
(from the possible determinations of a “what”);
rather, it is completely closed into itself,
and outwardly, in its properties, is
first the condition and support of every “what.”

Kant was the first to validate this proposition,
misunderstood almost entirely by the old philosophy.
As regards the first half of the sentence:
being is something living
from itself, out of itself, and through itself,
absolutely self-enclosed and never coming out of itself,
as we grasped clearly at the end of what we called the first part.
Kant merely added the second part of the sentence,
“condition and support of the 'what'”
factically, without ever deriving it.
We will append it genetically.
In brief, the task of deducing appearance
(as we have so far labeled our second part)
is entirely the same as completely proving the
stated sentence from being {esse}.
Now so far we have inferred,
and proved to this extent {in tantum},
that light and being are completely identical,
because we are, and are light,
all of which is surrounded by facticity.
Further, in our view the light carries
another qualitative character, the “from,”
and lastly absolute genesis as well.
At the highest point of our speculation,
we said that the light is absolute, qualitative oneness,
which cannot be penetrated further,
briefly, thus, an occult quality.
Now for the first time we have arrived
at a property of light through which it shows itself
immediately as one with the being previously seen into:
certainty, pure and for itself, as such.

2. How would you proceed, if I asked you
to describe this certainty more closely, to make it clear?
Not otherwise, I believe, than by conceiving it
as an unshakable continuance and resting
in the same unchangeable oneness;
in the same, I said, and therefore
in the very same “what,” or quality.
Accordingly, you could not describe pure certainty otherwise
than as pure unchangeability;
and [you could not describe] unchangeability otherwise
than as the persisting oneness of the “what,” or of quality.

3. I inquire further, is the main thing for you
in describing pure, bare certainty
that the “what” be something particular,
or does your description not rather explicitly contain
absolute indifference toward all more exact determination of the “what”?
Only the latter, you will say, and without doubt
you will admit that the “what” remains one;
but by no means what else it may be.
Therefore, in this case merely the pure form of the “what,”
or quality in general, is employed in the description,
and it is the required description of pure certainty
only on condition of this formal purity.

4. With this, the concept of the “what,” or quality,
is completely explained and derived for the first time.
This is a concept that has so far always remained in the dark,
as much in its content as in its factical genesis.
Quality is the absolute negation of
changeability and multiplicity, purely as such;
the concept is thereby closed without
any possibility of further supplements.
I add here as a supplement that through this negation,
the negated term (changeability) is posited at once,
purely as such, and without further determination.
(Quantifiability through quality, and vice versa.)
“Genetically derived,” I have said.
As such, certainty cannot be described otherwise
than through absolute quality.
“If it should be described, then ____ must” and so on.
Therefore, we have taken a very important step
for our primordial derivation of the “what” = appearance.
Now everything depends on how
the description (reconstruction) of certainty arises.

5. We argue thus:
In this way we have seen into and described certainty.
However, is our description of certainty itself then
certain, true, and legitimate?
As we have always done with similar questions,
let us pay attention to our manner of proceeding.
We have constructed a general “what”
and posited it as unchangeable.
The essence of certainty appeared therein.
I ask, if we repeat this procedure infinitely many times,
as we seem able to do, could we ever do it any differently?
The “what's” construction is entirely unchangeable,
and in all these infinitely many repetitions,
it is possible in only the one way we have described:
through negating changeability.
Therefore, we view ourselves in the very same way
we have described certainty, as persisting unchangeably
in the construction's single same “what”;
we are what we say, and we say what we are.

6. Certainty is grounded completely and absolutely in itself.
However, according to its description,
certainty is persistence in the same “what.”
Thus, in the description, the ground of the “what's” oneness is
to be posited completely inwardly within certainty itself.
The oneness of the “what” lies in this,
that certainty is,
and in no other external ground.

7. “Certainty is grounded in itself” also means
that it is absolute, immanent, self-enclosed,
and can never go outside itself;
in itself it is I;
in just the very same way that
the proof of being's form was previously conducted.
Therefore, it is clear that the externalized and objectivized
certainty we presented previously is not
the absolute one, according to its form,
although in its content and essence it may very well be.
Hence, it is clear that in searching for the absolute
we must abstract from the latter,
and search simply in what manifests
itself as immanence, as I (or We).

In this We, we have now surely found certainty,
as the necessity of resting
in the procedure's qualitative oneness;
and there the matter now rests
(this enumeration demands all our attention).
First and foremost, absolute certainty is,
in and from itself,
the same as “I” or “We,”
us or its own self
(all of which mean the same thing),
completely inaccessible,
purely self-enclosed,
and hidden.
For if it
(or “We,” which means the same thing)
were accessible,
then it would have to be outside itself,
which is contradictory.
As a result of the preceding insight,
we must abstract from just this:
that we are actually speaking about it now,
and thus are manifesting it;
and this is the appearance
(which is mere appearance,
because it contradicts the truth)
whose possibility we must deduce
from the system of appearance.

In a way that arises from a ground
that will show itself very soon but
that is not yet clear,
certainty now expresses itself immanently in itself
(that is, in us)
and thus in every expression,
as perception of a particular,
completely unchanging process.
This manifestation shows up here,
but only as an absolute fact,
and so an obscurity still remains.
(Process is living as living;
the process's unchanging qualitative oneness is
life's immanence and self-groundedness,
expressed immediately in life itself.)
Let us press ahead toward clarity,
to the extent that we can do so here.
Immediately living and immanent being-a-principle is light,
and is intuition with an inner necessity.
I say, “being-a-principle,”
therefore just projecting and intuiting.
I say the projected term is “immediately living and immanent”
simply in intuiting, from intuiting, and out of intuiting;
and the projecting is just light's life as “principle-providing”.
I say “with an inner necessity,”
and thus that this necessity must
completely express itself,
because it is just “principle-providing”
as being principle-providing.

It is absolute, immanent projection,
and so it is a projection of nothing else than itself,
wholly and completely as it inwardly is.
Note, as it is, it is projecting,
first of itself inwardly and qualitatively,
but not at all understood objectively.
Thus, way of inner, living “being-a-principle,”
as thinking and intuiting unconditionally at a single stroke;
but, in fact and truthfully, it is the latter as a result of the former.
Thus in this inner qualitative self-projecting,
it necessarily projects itself, objectively
(et in virtute eius, minime per actum specialem);
not yet to be sure as an objectively present I,
but rather as it inwardly is:
first of all as living, one, and grounded in itself formally;
but this is process in pure qualitative oneness.
This is the intuition of inner certainty
and oneness of process that concerned us previously.
This oneness expresses itself with necessity,
because it is the result of
the absolute, living principle-providing.
We called this process “describing certainty”
and sought a ground for it.
It has been found.
I ask to wit, does such a description of certainty
happen in itself and actually?
But how could it; it is nothing else than
the necessary expression and result of certainty's life
as pure “providing-itself-a-principle”
completely derived and explained by us.
Yet this life is unconditionally necessary in being or certainty.
Further, it projects itself as it inwardly is;
but it is not merely living, instead it is its own life,
and, as such, it is self-projection.
The life derived in this way as a constructive process is
therefore the construction of itself in the projection
(and therefore likewise in the certainty taken objectively),
which we found as a first term at the beginning of our investigation,
when we were unfamiliar with the higher terms.
In the living description, certainty is
more primordial in us than it is
objectively in itself without any description.
It is the latter only as
a result of the construction,
which is also projective.

Now let us more clearly and definitely grasp
the three primary modifications of the primordial light,
which we have disclosed today.

IV.29
tata klesa-karma-nivrtti

Certainty, or light, is an immediately living principle,
and thus the pure absolute oneness of just the light,
which cannot be described further in any way,
but rather can only be carried out.
If we wished to describe it, then we would have to
describe it as a qualitative oneness,
which in this case does not serve us.
It is always and eternally immediate I.
Therefore, since we said the foregoing,
we already contradicted ourselves in what we say.
When we said, “it is a living principle”
we began to describe it, but [to do so] primordially.
“Principle-providing” is already its effect,
but its primordial effect in us, since we are it.
It (or we, which is the same thing) describes itself in this way.
Principle-providing, if only you think it precisely, is
projecting, immanent self-projecting.
To be sure, since it lies immediately and directly in living itself,
it [consists in] making oneself into projection and intuition,
not through a gap and objectively,
but inwardly and essentially through transubstantiation.
Notice, because this is situated in light's very life,
all light whatsoever is immediately self-creating,
and so it is like this:
thus, it is absolutely intuiting.
Even the science of knowing, in all its living activity,
cannot avoid this fate.
We have not avoided it either, despite the fact that,
according to a law that we have not yet explained,
it invades the principle and becomes a self-making in it,
because otherwise [the principle] remains just a being.
This entire insight into
the primordially real principle-providing is
a matter for the science of knowing;
which is the first factor.

IV.30
tada sarva-avarana-mala-apetasya jnanasya-anantya jneyam alpam

Living knowing intuits itself
unconditionally as it is inwardly
just because it really projects itself.
However, above all it is unconditionally from itself;
it must therefore intuit itself as being thus,
and here specifically as not existing outside intuition.
Here, therefore, the absolute gap arises,
and the projection through a gap,
as a pure, rational expression
of the true relationship of things:
the notion of intuition,
or the concept,
in its separation from essence,
not as the essence itself,
but rather merely as its image,
and the negation of the latter beforehand.

Hence, it is a “principle-providing,”
and it must intuit itself objectively and through a gap.
At this point it is clear that this “principle-providing,”
its process, must appear within the intrinsically immanent view as
from itself, out of itself, and through itself;
as by no means grounded in it, but rather as happening
because it projects through a gap.
However, the scientist of knowing,
who understands the view itself in its arising,
knows very well all the same that this entire independence is
not intrinsically true as self-producing,
but instead is only the appearance of a higher,
absolutely unintuitable, principle-providing.
Therefore [he understands] that all the flexibility
lying in the procedure's appearance is not grounded in truth,
so that it can only be grasped with difficulty,
just as the qualitative oneness of intuition
by no means comes to an end with it.

IV.31
tata-krta-arthanam parinama-krama-samapti gunanam

So then, it is an absolutely immanent
providing-itself-a-principle and indeed,
as will now be explained more exactly,
in absolute fact,
without any other intervening light,
or seeing: as intuition.
This very intuition must itself be intuited,
or projected through a gap:
through which arises the intuition of
a primordially complete and persisting knowing,
what we previously called the being of knowing.
The first described principle-providing in intuition
relates itself to this “being of knowing” as reconstruction.
So, from certainty we have derived both subordinate terms
(which previously just stood there hypothetically)
out of a deeper insight into the essence of their relation.

We have not sought to conceal
where the difficulty now still remains.
That is, [we need] to ground the possibility,
and justify the truth and validity
of the science of knowing's presupposition that
living certainty is genuinely “principle-providing."
I say carefully “principle-providing,”
not “providing-itself-a-principle."
If we prove the first, then the second follows
from absolute immanence and self-enclosure,
as is completely obvious from itself.
More about this tomorrow.

IV.32
ksana-pratiyogi parinama-aparanta-nirgrahya krama

Twenty-fourth Lecture
Thursday, May 31, 1804
Honored Guests:

We have posited the primordial light
as “principle-providing”
living immediately in itself,
and from this, we have derived
three necessarily arising fundamental
determinations of the light.
At first, it was enough for us
to understand this proposition
and the inferences following from it,
which demanded energetic thinking and
inwardly living imagination above all.
However, I believe that I have succeeded in being comprehensible.
Finished with this business, we raised the question,
what authorized us, as scientists of knowing,
to make this assumption?
We held ourselves back from answering this, until today.
To begin with, consider the sense
in which we then asked the question.
We know that the science of knowing is I,
that light is completely I.
Further, someone could attempt to conduct a proof here
in the same way we did it previously;
the science of knowing as I, and therefore light, can and does;
hence the light can and does.
Nevertheless, this mode of proof must fall away
and receive its higher premises.
When this occurs, the place is revealed where,
at the very same time,
the I that can act and the light that can act fall away,
along with all arbitrariness,
the appearance of which our presupposition
certainly still carries.

So much generally.
Now I request that you undertake
the following reflection with me:

1. If arbitrariness is to vanish,
an immediate, factical necessity of
immediate self-projection must show up
in the science of knowing.
“Immediate necessity,” I say;
the science of knowing must really do this,
or better the necessity must take place itself,
without the science's help.
I say “necessity” and “factical”:
it must be immediately intuited as necessary,
but without additional higher grounds.
The remark, which we make from time to time and made yesterday,
that we cannot escape from knowing's
projecting and objectifying,
does not suffice for our purpose.
It has not been shown under what conditions and
in which context we cannot escape.
Thus, if we could not escape only in attentive thinking,
we nevertheless seem to escape in weak thinking.
We should explain just how it stands
with these two contradictory appearances.

However, as I will show you briefly
and without further derivation,
the following suffices and leads to our goal.
I cannot predicate anything of the light,
unless I just project and objectify it
[the light] as the subject of the predicate.
“Non nisi formaliter obiecti sunt predicata.”
If one merely ponders this proposition attentively,
it is immediately clear, without any further possible grounds;
and if it is to belong here, then it must be just so,
according to our previous remark.
It is more important for us to understand its contents properly.
“I predicate of the light”
(or, what is synonymous, the light predicates of itself)
means that it projects itself through a gap,
by means of the enduring intuition
sufficiently characterized yesterday.
“I cannot do so without projecting it” generally means:
without projecting it in
the primordial, real, inner, and essential projection,
which first makes it intuition,
as again was sufficiently described yesterday.

Let us look back at our task.
We put forward the very claim about which
we admitted that it was not made in any factical knowing whatsoever
(which is always limited to already completed intuitions)
but rather is made only by the science of knowing:
light's absolutely primordial act of
making itself an intuition.
This claim should be demonstrated
in an immediately self-producing insight and manifestness.
This is the case with the insight produced by us.
Therefore, absolute necessity, etc., is comprehended.

But how is it comprehended?
Not unconditionally, but under conditions.
Our insight brings it into connection with something else.
If [something] should be predicated
(that is, be intuition), then ____ must.
However, should the antecedent be?

In that case neither is the consequent:
the first can be seen into only under a condition
—thus it is conditioned intellectually;
if the consequent happens to occur—
that [would be] real conditionedness.
So everything is just about what we wished;
but not the unconditionedness and oneness for which we strove.

(
a. A logical clarification:
Predicate = minor premise;
absolute objectification of the logical subject = major premise.
Both posit themselves unconditionally reciprocally;
and so something else underlies the inference
much more deeply than the major premise,
with discovery of which we began.
Ignoring this, all philosophical systems without exception
fail to arrive at an absolute major premise.
Therefore, if they do not wish their thinking
to stand still somewhere arbitrarily,
they must sink into a rootless skepticism.

b. With the repeated appearance of the “should,”
of hypotheticalness, and of relatedness,
no one now fears that we would be driven back again
to the old point that we have already abandoned.
Because obviously the currently related terms are
higher than were the previous ones.
Previously, [we understood] self-construction to be
the procedure that arose yesterday
in the already established intuition,
and, [we] took the being of knowing
to be the highest, the established intuition.
Now, though, the established intuition itself is not
displaced by a higher one in relation to
pure and real self-projection,
rather it is simply discovered in it.
)

2. Since I, a scientist of knowing,
see into this connection
as absolutely necessary and unchangeable,
I myself project and objectify knowing
as just this relation,
as a “oneness through itself,”
determined without any possible assistance
from some external factor;
[a oneness] which I simultaneously permeate
and construct in its inner essence and content.
What then is the content?
Above all, something arbitrary
and dependent simply on freedom and the fact,
therefore something unconditionally necessary,
which, facticity, if it happens to be called into life,
grasps and determines without further ado.
Both in relation to one another,
so that the one is indeed the completely
proper principle of its being,
but cannot be this without
in the same undivided stroke
having its principle in the other.
Likewise, the other is not actually a principle
unless the one posits itself.

We can best name the former “law,”,
a principle that, in order to
provide a principle factically,
presupposes yet another
absolutely self-producing principle.
[We can name] the latter a pure, primordial “fact,”
which is only possible according to a law.

3. This is what knowing is, absolutely and unalterably,
without any exception, and it is understood as such.
Now, in the insight just produced and completed,
I myself, the scientist of knowing, am a knowing.
Indeed, as it seems to me,
I am a free and factical knowing,
since I could well have omitted the preceding reflection,
and moreover [I am a knowing] predicated of knowing,
describing its entire essence.
According to my own statement,
facticity always and everywhere takes up
the law of primordial projection,
therefore in actual fact it must have
taken me up, albeit invisibly.
Formal projection in general rests in the law,
and this is evident enough factically;
I add here only that it exists as a result of the law.
The fact also rests in the law
that it [facticity] is projected just as it inwardly is,
or, (as we could say more accurately
in order to be safe from all misunderstanding)
nota bene, as it must be projected according to the law.
However, at the heights of our speculation,
we know no other law at all except
the law of lawfulness itself,
that it is projected according to law.
This is just how it expressed itself to us earlier in factical terms;
therefore, we have nothing else to do regarding this material point
than to add that this projection occurs according to the absolute law.
In that way, by applying its own material assertion to its own form,
we have genetically derived the very insight from which we began today,
and which previously was produced factically.
This is the first important result.
a. No doubt, we will get to say more about such an application of
what a proposition asserts to the proposition itself.
It is remarkable.
It seems to be just the determination of
the major premise by the minor premise
that we mentioned previously;
and therefore truth seems to begin running back into itself,
which in all other systems would have neither beginning nor end,
if they were consistent.
b. On every occasion, we have posited as
the absolute of its kind that which posits itself;
and so, in the investigation just completed,
we have [posited] a law of law, or lawfulness itself.
That this is the absolute law cannot be doubted;
nor that the act we carry out in accord with it is
the highest, immediately law-conformable
act of the science of knowing.
(What the stated limitations mean
will become evident at the appropriate time.)

4. Now we proceed to another investigation
that is of the highest importance, interesting
(if I can succeed in making myself understandable to you),
and even agreeable.

Without doubt the absolute law,
according to which we projected knowing
in its essence in the insight
that we completed today and then analyzed,
has absolute real causality on the inwardness of the act
(I do not speak of its outward form, which appears free).
So that the law and the act permeate one another inwardly
(and indeed the act with the oneness of
all its distinguishable determinations)
without any gap between them.
The projection is in part formal (objectifying),
and in part material (expressing the essence of knowing).
It is by no means the latter without the former,
but instead is both at a single stroke,
because it is both through an absolutely effective law.
Hence, the material expression must
simultaneously express the projection's form.
Or, to say it exactly:
disregarding the prior proof of the opposite,
knowing in projection, at least formally,
cannot simply be just what it is in itself or
according to the law, without any projection.
What could it be that projection alters in it?
Of course, you do not forget that we are speaking here only about
projection's inner material form,
abstracting from its outer form,
on the basis of which we merely argued and which we now drop.
The inner essence of projection is living principle-providing;
this [latter] must remain in [the former]
completely and absolutely as such, and can never be destroyed.
What is this?
Answer: only absolute description, as description,
which stepped between the two terms in a wonderful way.
Among other things, [it is] quite evident in the relation as such.
For what is the relation except
describing one on the basis of the other?
The latter must remain immanently in the former
and can never be destroyed,
precisely as intrinsically living “principle-providing.”
Thus, it must allow itself to be
perpetually renewed as such, although the content,
determined by the absolute law, remains the same.
On this basis, one can now explain the appearance of
energetic reflection, and the infinite repetition of
the content that qualitatively remains absolutely one,
content which, to our great astonishment,
has not yet wished to leave us.
I said that the law should determine the inner content.
While we examine it, if we now think of description as
an absolutely living “principle-providing” in itself,
then it is clear at once that it must appear just through the law
as the re-construction of an original pre-construction,
and so, in a word, as an image.
Or, [it must appear] as the oft-mentioned mere statement,
only the enunciation and expression,
of that which to be sure should in itself be just thus.
In a word, [description is] the whole simply ideal element,
as which we must consider our seeing to be,
if we transfer ourselves into the standpoint of reflection,
of “principle-providing.”

Therefore, to apply this at once,
this is how it stands with knowing.
The entire form of objectivity,
or the form of existence,
has in itself no relation to truth.
Knowing itself, however,
and everything which should arise in it,
splits itself absolutely into a duality,
whose one term is to be the primordial,
and whose other term is to be
the reconstruction of the primordial,
completely without any diversity of content,
and so again absolutely one;
differing only in the given form,
which obviously indicates
a reciprocal relation to one another.
(It is really like this in every possible consciousness,
if you wish to test the proposition there.
Object, representation)

However, let us carry this reasoning further.
At the beginning of our investigation,
because the looked-for absolute oneness created us,
we stood (as we later discovered) under the law
without knowing either it or our act as such.
Only when we reflected on this act were
we able to apply the content of our discovered insight
to it as the middle term of our inference,
and arrive at an insight into the previously concealed law.
Without doubt, we constructed or described
the law itself in this insight,
and could catch ourselves in the act.

Therefore, we truly stand under the law
just in case no law arises in knowing;
and we are beyond the law,
constructing it itself,
if it does arise in consciousness.
Thus, our entire inference grounds itself on a mere fact,
without law, which therefore cannot be justified;
and the inference itself only speaks of a law
without being or having one.
Hence, this reasoning too, however much luster it shows,
unravels into nothing.

Applying this to the preceding:
the ostensible primordial construction,
which is supposed to justify the reconstruction
(that admittedly presents itself as such),
is itself really just a reconstruction,
but one that does not present itself as such.
However, the entire appearance disappears on closer inspection.

We must be glad that it now drops away as well,
and that this standpoint was not the highest.
Because yet another disjunction lies in it,
whose genetic principle is not yet fundamentally clear,
and at which we cannot remain.
Of course, it is not the outward
disjunction into subject and object,
which fell away by means of the full annulment
of the persistent form of projection and objectivity;
rather [it is] the inner living difference between both:
two forms of life.

Just how this difficulty is resolved
and where our path will go on from there
can be gathered very easily.
We described and constructed the absolute law;
that is the difficulty.
It must be evident that we cannot construct it;
rather it constructs itself on us and in us.
In short, it is the law itself,
which posits us, and itself in us.
On this subject, tomorrow.

IV.33
purusa-artha-sunyanam gunanam pratiprasava kaivalyam
svarupa-pratistha va citi-sakti iti

Twenty-fifth Lecture
Friday, June 1, 1804
Honored Guests:

We understood that, as a condition,
if knowing is to predicate something of itself,
then it must in general simply project itself.
Looking simply at the form of this insight,
it contains a free, arbitrary fact and an absolute law,
of which this fact, becoming actual,
should immediately avail itself.
This is the first point.

Now in this “understanding”, we, as scientists of knowing,
are also knowing, indeed a free factical knowing,
and so in the same circumstance of which we have spoken.
Therefore, we ourselves, along with this fact of ours,
come under the law to project knowing in general,
and to project it as it is inwardly, or according to law.
Thus, knowing has disclosed itself in our insight
as objective and unchangeable oneness,
and with the absolute manifestness that
it behaves in just the way we have said.
We add only the new point that this too
exists in this way because of the invisible law.

This point was argued further as follows.
This projection occurs
(at least according to the material,
or to the content of knowing stated in it)
according to an absolute law,
which cannot not be a law
and cannot not have causality.
Hence, it is an absolutely immanent projection
and can never escape being so;
and, as we very illuminatingly add,
the light in the projection can not be just
the same as it is inwardly or
(according to the law)
apart from all projection.
What does this mean?
It must permanently bear the mark of
the intrinsic living principle-providing;
and [it must] appear in its form as
the product of that sort of “principle-providing.”
Therefore, [it must appear] repeatedly to infinity
as a reconstruction related to
the primordial projection through the law.
This establishes the fundamental disjunction in knowing.
At this point, we raised this objection:
is not the law itself reconstructed by us?
Obviously; therefore, we never really arrive
at a primordial construction and law,
instead, viewing the matter aright,
[we] merely have two reconstructions,
one of which presents itself as what it is,
and one of which lies.
However, [we] can uncover this illusion.
Thus, we still find ourselves in
the realm of arbitrariness,
not yet having entered the necessary.
Now to solve this.

1. In the first place, why is it, really,
that we will not trust this reconstruction of the law?
Because it seems arbitrary.
If it showed itself as necessary,
then it would show itself also as
something conforming to law
and as itself the immediate
inner expression and causality of the law.
We received an immediately factical law,
pervading the facts: posit the law.
Now, merely because it is projected,
the projected law can always be
the result of a reconstruction;
but at least the inner construction of
this objective law is not a reconstruction,
but rather the primordial construction itself.

2. Can this proof of a law's reconstruction be carried out?
I say: as it seems to me, it can be done easily in the following way.
The first, primordially permanent projection
intrinsically bears the character of an image,
of a reconstruction, etc.
However, the image as such refers to a content;
and reconstruction as such refers to an original.
Hence, the task of understanding this concept of intuition
completely and lawfully contains another [task],
to posit this prior one.
Now I ask, how then is the image an image,
the reconstruction a reconstruction?
Because they presuppose a higher law,
and exist because of this law, we have said;
and therefore [they] point in the image, as an image,
to the law, which is already contained there,
at least virtually and in its results.
We, scientists of knowing, stand presently
just in the image as an image;
thus, it is the law implicitly and virtually
present in us ourselves that constructs, or posits, itself ideally.
So, what we undertook to prove yesterday is completely proven,
“The law itself posits itself in us ourselves.”
Image as image is the crucial premise {nervus probandi}.

Notice also:

1. In order to conduct the proof we have just completed,
we first had really and intrinsically to presuppose
the law as the image's primordial ground,
without giving any account of how we arrive
at the concept or its projection.
Now we have completely explained how we arrived at it;
but the variation of this concept's form
has not yet been explained.
I content myself here with merely pointing out
this as yet obscure section historically,
and adding that its clarification lies
in the reply to the question of
the science of knowing's possibility
as a science of knowing,
which will now be addressed continually,
but achieved only at the end.

2. We have now changed the following
in yesterday's sketch of knowing.
Knowing stands neither in
the reconstruction as such (representation),
nor in the original (the thing-in-itself),
but rather wholly in a standpoint between the two.
It stands in the image of the reconstruction, as an image,
in which image the positing of a law arises
immediately through an inner law.
This permeation of the image's essence is
the primordial, absolute, unchangeable oneness.
As wholly internal, in projection,
it just divides itself projectively
into a permanent objective image
and a permanent objective law.

I wish to be completely understood on this.
The light lives what it is
in its very self,
it inhabits its living.
Now it is image:
as image, I have added,
that is living, self-enclosed imaging.
You have to attend very closely to the latter;
because otherwise, it is not yet clear
how you could have come to the former,
to the image as a closed oneness
that you have undoubtedly objectified.
It is an imaging, formally immanent;
it images, or projects itself as
just what it inwardly is, as an image.
It is intelligibly immanent and self-enclosed.
Yet, the image posits a law,
therefore it projects a law;
and it projects both as
standing completely in
the one-sided, determinate relation
in which we have thought them.

Note further here: according to yesterday's disjunction,
both primordial image and copy should be qualitatively one,
because otherwise they could not cohere.
Now, both cohere inwardly and essentially
as image, positing a law, etc.,
and qualitative oneness can not come
into it in any way.
Qualitative oneness is the absolute negation of variation;
therefore, it can come into play only
where variability can be posited.
However, image, as image, is intrinsically invariant.
It is essential oneness,
the law of the image is likewise oneness,
and they posit one another entirely
only through their inner essence,
without any further supplement.
This total removal of the material, qualitative oneness,
which so far has not left us,
is a new guarantee that we have climbed higher.

Since we end the week today, I will not begin any new investigation;
instead, I will estimate what we have left to do in the coming week,
during which I intend to complete this entire presentation,
should that prove possible.

Initially, it is readily clear that,
if we only establish ourselves correctly in
what has just been presented,
no possibility can be foreseen as to
how we should escape it and go further.
Here we are immediately absolute knowing.
This is an “image-making process”
positing itself as an image,
and positing a law of the image-making process
as an explanation of the image.
With this everything has been unfolded, and
is completely explained and comprehensible in itself.
The terms come together to form a synthetic cycle,
into which nothing else can enter.

The particular point at which we have closed off
further progress is very clearly evident above.
[We did so] through the total negation of
the concept of qualitative oneness.
By negating this concept,
quantifiability was established
at the same time that we opened
a more comfortable way to descend
into life and its multiplicity,
as we know from the preceding.
Now, this qualitative oneness has been
negated through positing a oneness of image
and its law that repels all variability,
a oneness essential in itself
but still merely hypothetical.
Since this oneness has become manifest absolutely,
it must be applicable here,
so that we cannot possibly prevent ourselves
blending this quality in, without further ado.
As to this, I wished particularly
that you had noticed for yourselves,
that we had to descend again to this quality
only through a term of the necessary relation,
one which is produced lawfully.
(The occult quality is entirely cut off.)

Noteworthy too is the fact that
the concept of the science of knowing
as a particular knowing has now disappeared completely.
What we have derived is the one pure knowing
in its absolute disjunction,
which is explained from its essence as oneness.
We now are this one pure knowing;
we are now the science of knowing as well
and [we] hope to become it again,
so the science of knowing is absolute knowing itself,
and we are the latter too only to the extent that
the science of knowing is it.

The last consideration leads us to
the path on which we can go further;
the science of knowing must emerge again
as a particular knowing.
Of course, we know very well that
we have not always been the one insight
that we now are and live,
but that we have ascended to it
by means of all our previous considerations.
We must retrace this,
our transformation into absolute knowing
not empirically and artificially, as we have before;
instead, we must explain it.
In short, we must once more see in its genesis
what we are at the conclusion of today's lecture.
This insight into the genesis,
not of absolute knowing in itself,
because this knows no origin;
but rather of absolute knowing's
actual existence and appearance in us;
this would be the science of knowing in specie,
in so far as it is a particular knowing whose
nonexistence is as possible as its existence.

Now, it may happen that
ordinary knowing is
the primordial condition for
the genetic possibility of
absolute knowing's existence,
or of the science of knowing.
Hence, [it may happen] that
its determinations can be explained
simply from the presupposition that
the science of knowing ought to arise,
and the sum of our entire system resolves itself
into the following rational inference.
“If absolute knowing is to appear, then ____ must, etc.”
Now, knowing is determined like this,
so consequently this should clearly happen.

I have said that all determinations must be
explicable and understandable simply on
the basis of the presupposition:
“It should unconditionally____ , etc.”;
and I ask you to take this in its full strength.
By way of explanation, I add the following:
we have seen that knowing in itself is
unconditionally one, without any material quality or quantity.
How then does this knowing descend in itself
to qualitative multiplicity and difference,
and to the entire infinity of quantity and its forms
(time, space, etc.) in which we encounter it?
We have to prove the following:
simply because absolute knowing's being
can be produced only genetically,
and because it can be so only under
just the types of conditions
that we find originally in living,
therefore life coheres indivisibly
with the science of knowing,
and with that which it produces.
Everyone must confess that,
apart from the extent to which
he elevates himself to absolute knowing,
his entire life would be nothing,
would lack worth and meaning,
and would truly not even exist.

In brief:
absolute affirmation of the genesis of
the existence of absolute knowing
(according to the preceding,
no term in this description is superfluous)
unites both ends of knowing,
the ordinary and also the absolute and transcendental,
and clarifies them reciprocally.
We must put ourselves into this point,
as the genuine standpoint of the science of knowing in specie.
With this, I believe that I have shared with you
a very important conception of our entire procedure
up to now and into the future.

Hence, the entire outcome of our doctrine is this.
Simple existence, whatever name it may have,
from the very lowest up to the highest
(the existence of absolute knowing),
does not have its ground in itself,
but instead in an absolute purpose.
And this purpose is that absolute knowing should be.
Everything is posited and determined through this purpose;
and it achieves and exhibits its true destination
only in the attainment of this purpose.
Value exists only in knowing, indeed in absolute knowing;
all else is without value.
I have deliberately said “in absolute knowing,”
and by no means “in the science of knowing in specie,”
because the latter is only a means,
and has only instrumental value,
by no means intrinsic value.
Whoever has arrived no longer worries about the ladder.

This result, that only true knowing or wisdom has value,
is very shocking in our age, which counts only on external workings,
and undoubtedly it appears to us as a great innovation.
It is remarkable that this doctrine,
which is an innovation to our age,
is really the primordial one, as is almost always the
case with all our characteristic works and ways.
I will prove this, not to support by age and authority
what can prove itself, but rather to give you
a parenthetical opportunity for comparison.
In Christianity, (which may in its essence
be much older than we assume,
and concerning which I have frequently said that,
in its roots and especially in its charter [the Gospels],
which I hold to be its purest expression,
it [Christianity] completely agrees with realized philosophy)
the final purpose, especially in the record of it,
which I hold as the purest,
is that people come to eternal life,
to having this life and its joy and blessedness
in themselves and out of themselves.

In what then does eternal life consist?
“This is eternal life,” it says,
“that they know [recognize] you.
and [him] whom you have sent”
(for us, this means the primordial law
and its eternal image);
merely know [recognize].
Yet, indeed, this recognition not
only leads to life, instead it is life.
Thus, from that time on,
through all the centuries and
in all forms of Christianity,
and consistently with this principle,
faith in the teaching that
true knowledge of the super-sensible is
the main and essential thing,
has been insisted upon.
Only in the last half a century,
after the almost total decline of
true scholarship and deep thinking,
have people changed Christianity into
a doctrine and ethics of prudence.

This doctrine has not forgotten to teach
that in genuine, truly living knowledge
right conduct arises on its own,
and that, even if he wishes to do so,
the one in whom the light has inwardly dawned
cannot fail to shine outwardly.
Our philosophy forgets it just as little.
Yet, there is a great difference
in doing right from such different sources.
Doing right from self-interested cleverness,
or from self-regard arising as a result of
a categorical imperative, yields cold, dead fruit,
lacking blessing for both agent and the recipient.
Now, as ever, the former hates the law,
and would much prefer that it not exist.
Therefore, happiness with himself and his act never arises.
The latter cannot animate and bring to life
that which fundamentally has no life.
Only when right action arises from clear insight
does it occur with love and pleasure.
Only then does the act, self-sufficient
and requiring nothing else, reward itself.
