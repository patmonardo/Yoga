Twentieth Lecture
Wednesday, May 23, 1804
Honored Guests:

Being is an unconditionally
self-enclosed, living oneness.
Being and light are one.
Since in the light's existence
(= in ordinary consciousness)
a manifold is encountered
(we have initially expressed our problem empirically
and we must continue to speak this way until it has been solved),
a ground for this manifold must let itself appear
in the light itself as absolute oneness
and in its manifestation,
a ground that will explain this entire manifold
as it occurs empirically.
“In the light and its manifestation”
I have said:
therefore we must first of all derive
the appearance of the light from the light,
[and] the manifold will arise in the former.
This is roughly the main content of
what has been achieved so far
and of what remains to be done.
This is to be noted especially:
[the task is] to present appearance in general and as such.
(Obviously, as soon as appearance has been explained
and the principle of the manifold has been explained
from it a priori and in principle,
all appeal to empirical experience falls away,
and what was previously held factically
will be conceived genetically.)

At present we have already pushed ourselves
quite near our highest principle.
If transcendental insight has been opened for us,
then having in mind the most recent link in the chain is
sufficient for understanding the lectures;
if the earlier links by means of which we ascended
to the later ones are not equally present, nothing is lost;
we will rediscover everything anew on the descent.
I must bring you back to this last link by repeating
the previous lecture, at the same time I will also expand and add.
It has already been proven earlier that the absolute,
simply as absolute, must be from itself,
whatever else it may be
(earlier it was “being,” “light,” and “reason,”
none of which mattered to this argument and did not belong to it);
and this proof further coheres with the postulate
that inner being could not be constructed from
outside but must rather construct itself,
with which postulate we opened the entire
so-called “second half” of our investigation.
(In this fashion, everything achieved so far toward
bringing out the second part, and with it the whole,
could also again be reproduced.)
In the last hour, this completed proof was
itself investigated in its central nerve
and points of manifestness,
and what turned out to be its foundation was
the simple presupposition that genuine true seeing,
or light, must have accompanied the actual Creation;
and since being and seeing had already been grasped earlier
as being the same, that genuine true light must itself be
an immanent creation, or an absolute “from”.
This, I say, turned up as a mere presupposition,
grounding the process of our proof regarding
the essence of the absolute, but itself based on nothing.
Still, a little reflection shows us that
this presupposition proves its correctness
simply by its mere possibility and facticity;
because we ourselves were the knowing,
insofar as we conducted the proof
and made the fundamental assumption
concerning the essence of knowing:
that it is a “from”;
and, note well, we certainly could be,
and indeed are satisfied,
that knowing cannot be both
something in and for itself
apart from any view into itself
and also a “from,”
but rather that it can be both
only within [such a] view.
By the actuality of this view within ourselves,
we have proven directly and factically
that in this respect it is so.
It is, and it is this;
because it quite certainly is,
and quite certainly is this,
and we ourselves, the scientists of knowing,
are it as such.
This is an immediate demonstration of the essence of knowing,
conducted through the fact itself and its possibility.
At this point let yourself take in even more fully
what was established last time,
although only in passing:
we did not make this presupposition
because we wished to,
or with any sort of freedom;
and if only this free element,
which is to be summoned in response to
some particular reflection,
is to be called We,
then we actually did not make it;
rather it made itself directly through itself.
All our preceding investigations have started from
the fact that we were requested to think energetically
about something we were aware of internally and
also were able to ignore;
so both took place only in consciousness;
this provided our premises, and, to be sure,
this energetically considered object was always
accompanied by the explicit supplement of a “should”:
“If this should be so, then ____ .”
From our thinking this premise energetically,
manifestness grips us without any assistance on our part,
and carries us away, attaching to the controversial premise
that conditions it and is conditioned by it.
Therefore, the knowing, which we pursued in this way,
instantiated the basic characteristic mentioned before
of being merely reconstructive
and was, in this reconstruction,
a secondary and merely apparent knowing,
transferring knowing's implicitly unconditional content
into a conditioned relationship.
All systems without exception remain fixed in this knowing;
their premises therefore are hypothetical for them only
(but not absolutely in reason,
by which even they themselves are driven,
although to be sure without their knowing it)
and the relationship alone is evident,
which however gives no final or fixed manifestness,
since the relationship itself depends on
the reality of its terms.
They supplement the lack of this strength
only by an arbitrary reliance on the premises
and by averting their eyes from their difficulties;
without this reliance they could release themselves
to absolute skepticism at any moment.

Up to this point [it has been] this way.
Now, absolute manifestness has extended itself to the premise,
to the absolute presupposition itself;
and thereby it has annulled both all freedom
and every “We” that was presupposed as a premise
in relation to the secondary manifestness of the context.
Hence, we are transposed into a
completely different region of knowing,
not simply as something purely self-grounding,
but rather immediately, and ipso facto from itself.
But as for what relates to the premise as a premise,
undoubtedly in this quality a consequence is posited through it,
and these two in turn posit a relation;
therefore, in this quality,
it serves admirably to
explain secondary knowing;
and, since it is absolute,
to explain the latter from the absolute,
which is exactly our task.
As a premise, it is undoubtedly
the principle of appearance we have been seeking.
But since appearance is not itself
the purely self-contained absolute,
as becomes evident in the premise:
the simple fact that it requires
a consequence and a context
shows it to be insufficient by itself;
there must therefore be a higher notion of knowing.
This remark can cast a great deal of light
over what we still have left to achieve,
so I want to analyze it further.

Now we let go of the point that we presuppose it,
or more accurately, that it posits itself as a presupposition,
and thus let go of the proposition's form
for reasons having to do with method,
and simply hold on to the content of the proposition:
“the light is an absolute “from”,”
analyzing what we actually mean by this.

All along, and obviously in this proposition,
light is posited first and foremost as
an essential, qualitative, and material oneness,
not further conceivable, but instead only to be carried out at once,
just the way we carry it out in all of our knowing,
from which we cannot escape.
I want to be understood on this point that is easy in itself,
that requires simple, strong attention.
Question:
what then is knowing?
If you know, then you just know.
You cannot know knowing again
in its qualitative absoluteness;
since if you did know it,
and even now were knowing it,
then for you the absolute would not stay
in the knowing that you knew about,
but rather in the knowing by which you knew it;
and it would go on this way for you
even if you repeated the procedure a thousand times.
It remains forever the same, that in absolute knowing you
recapitulate knowing as essential qualitative oneness.
Initially this insight needs only to be carried out;
reflecting on the law of its completion still remains before us.
This light is now absolutely presupposed as a “from”
without prejudice to its qualitative oneness;
since [if it did not preserve this oneness] the light would
not be “from,” and so [it is presupposed] as permeating this “from.”
Notice first what is new and important here:
it is presupposed to be like this, unconditionally.
Thus it has presupposed itself in a particular act;
and this presupposition is now proven by its facticity and possibility,
and further by the possibility of a deeper determination of knowing,
which opposes the simple existential form
in its attachment to the mere dead “is.”

In our insight, it in no way follows from our insight
into the essential light as such
(which we once again should have been grasping
energetically and freely),
by means of which manifestness descended into the connection
between light as such and the “from.”
And we again lapsed into secondary and merely apparent knowing
with which there must, to be sure, eventually be an end,
an end we have sought so avidly from the beginning.
“It does not follow,”
I said, since, as we have seen, there is generally
no such insight into the light in itself,
hence, I said: the light posits itself
as a “from” in a particular and absolute act or genesis;
an act that cannot be mastered immediately in this genesis,
as the genesis of genesis,
because otherwise the genesis would
not be an absolute genesis.
(What this latter means, and does not mean,
since here as well is yet another disjunction,
will show itself it what follows.)
I say: according to the preceding observation,
it thus posits itself absolutely;
the act is a self-contained, self-sufficient act;
it is posited by us merely in our inferential chain,
which we now entirely let go of,
as the mere means by which we have
ascended to our present insight,
until we find it again on the descent.
This is the first, and significant, point.

That the light in its changeless qualitative oneness
is a “from” therefore means:
it is a qualitatively changeless
permeation of the “from.”
In the previous hour,
we made the following application of this point:
disjunction is found everywhere in the “from”;
absolutely out of, and from, the “from”;
by no means presupposing terms
that were primordially different
independently of the “from,”
instead [they were] produced absolutely as terms,
absolutely distinguished as such
only through the “from”
and otherwise through nothing at all.
The one, eternally qualitatively self-identical light,
by virtue of its identity with the “from,”
must, in this qualitative oneness,
spread itself over these terms,
whatever their distinction from each other.

Let me now apply and animate this insight right away,
and thereby make it unforgettable for you.
A “from” is posited immediately through the light:
L
a — b

Hence, if light exists,
then necessarily there is also a “from.”
Now if the light is identical with the “from,”
then, as surely as it itself exists,
it spreads itself in unchanged qualitative oneness
across every “from,” and comprehends every “from.”
And if one again posits within the first “from”
another one that is deducible and conceivable
on the basis of the original synthesis of
light and the “from”

a — b
|
|
a — b  a . b

it is completely clear that
the same original light, qualitatively unchanged,
by means of its identity with the original “from,”
must at the same stroke accompany
all subordinate divisions of
the original “from” into further “from's.”
And it is also clear that whatever possesses
the principle of this secondary splitting
of the original “from” accompanies
this progression of the light
as entirely necessary and at one stroke,
and can reconstruct it purely a priori
and without any empirical presuppositions;
which indeed is the second and subordinate task
of the science of knowing,
since we now are pursuing the much higher task
of presenting the principle of this principle.
This “from,” in pure, absolute, immediate oneness
and without any disjunction,
as the pure self-positing of the original light,
is the light's first and absolute creation;
the ground and original source even of the "is,"
and of everything that exists;
and the disjunction within this “from,”
in which true living perishes
and is reduced to the mere intuition of a dead being,
is the second re-creation in intuition,
that is, in the already divided original light.
And thus the science of knowing
justifiably presents itself
as the complete resolution of
the puzzle of the world
and of consciousness.

This, I asserted, was the next application
that I made last hour of the proposition,
“The light is a 'from,'” attending to
the disjunction in the “from.”
But it is even more important to look at
the essential and qualitative oneness of this “from”
and at the words that were said previously
about the original creation.
Recall them.
In its pure qualitative oneness,
“from” is genesis:
that the light is identical with it
and permeates it in this its essence, means:
in this its second power (namely its appearance),
it is itself genesis;
genesis and seeing converge together
completely and unconditionally.
The words are easy to understand;
it is not so easy to give them the deep meaning
intended here in living insight;
and it is nearly true that the only way
I am able to guide you forward is with an example.
The subject matter that I wish to
present to your intuition appears
in every transition from lassitude to energy,
and for our purposes,
the example cited just above will serve best:
the one in which we had tacitly presupposed
absolute knowing to be a “from”;
when interrogated about the justification
of this presupposition,
we recalled that indeed we knew ouselves
in this presupposition and were the knowing.
I ask:
does not this new awareness,
that was not yet there prior to our presupposing,
seem as if it were a popping up, a new production?
Now at this point you are
certainly able to abstract purely,
as is my present demand:
that this is consciousness,
is a consciousness of knowing,
and what is more of knowing as a “from.”
What remains for you after this [act of] abstraction?
Evidently just a knowing/seeing/light,
exactly absolute, qualitative,
as it has already been described,
and [it was] this therefore because
you abstracted completely from all content,
which you could do according to the presupposition;
consequently, [it], as itself light, conducted
the proof of legitimacy empirically;
further, [it is] a consciousness of absolute genesis.
Now (note this addition, the proof becomes more rigorous
and the insight purer through it)
you can more suitably posit this genesis or freedom
in the act of abstraction from all content
of the presented consciousness,
which is thus required of you.
As things stand, it is immediately clear to you
that this pure light, as it has been described,
could not arise without abstracting from all content,
nor can the latter appear without arriving at pure light;
that therefore the appearances of both terms are indivisible,
and permeate one another;
and that hence pure light appears as permeating genesis,
or as producing itself.
By means of this proof more is
nearly proved than should be proved,
and future research is anticipated,
as I note in preparation;
the light's positing of the “from,”
and the fact that it posits itself as a “from,”
has already become immediately visible.
What we have to be concerned with next here
can be shown with a little preparation in two examples.
Because you were instructed to reflect energetically
and a new consciousness emerged for you,
this new consciousness is not to exist
as something new without the energy;
this consciousness and the energy should
open up together indivisibly.
Now you certainly posit genesis here
partly in yourself,
in the energy of your reflection,
and partly in the essence of reason itself,
since the manifestness is to emerge
without any further action on your part;
but this entire distinction ought to have no validity in itself,
and it should be abstracted out, and so,
(leaving undecided whether genesis's true principle
lies in me or in reason itself)
there always remains an absolute, self-producing knowing
that does not possibly occur without the genesis.

Now this means, as was said before,
that light permeates the “from”
in the qualitative oneness of its (the “from's”) essence:
the presented intuitions of this penetration
were only explanatory means.
But, independent of all facticity,
we have seen a priori
that if light is to be,
such a permeation is necessary.

This is the one side of the previous proof
for the content of the sentence:
the light = “from,”
that we have repeated and enriched today.
There is still another,
and of this more tomorrow,
equipped with today's new discoveries!

In conclusion another comment about
the whole of the science of knowing,
one that I share with you not so much for your own guidance,
since I hope you do not need it,
but rather as a weapon of defense against the ignorant.
Already earlier, and again today in passing,
the proof of knowing's essential criteria is conducted
on the basis of our capacity to see it thus.
The nerve of the proof is clear:
we ourselves are knowing;
since we can know only in this way,
and presently actually know thus,
then knowing is constituted so.
It is equally clear that the failure
to discover this principle of proof,
or not paying attention to it once it has been found,
grounds itself on the truly foolish maxim
of searching for knowing outside of knowing.
Concerning this, nothing more needs to be said.
I would only bring this to your attention:
the proof simply does not succeed for anyone
who is really not able to make clear and intuitable
what can only be made so by his own capacities;
through his incompetence he is
barred from the subject itself
and from any judgment about this world
that is entirely concealed from him.
It is the same for those who could but will not, that is,
who will not submit themselves to preliminary conditions
of sharp thinking and strong attention;
because everyone who can, will do the thing itself;
and everyone who will, can do it.
This is true when the science of knowing
does not yet stand at the apex.
One should not therefore wonder
how that which has in itself
the highest clarity and manifestness,
cannot in any way be made clear and true
for very many people;
one can rather himself lay out
the grounds for this impossibility,
if they will just come to understand the premise
that there might be something they do not now know;
and that they are not able to know directly
and without much preparation and strong discipline,
as things are with them now.

Twenty-first Lecture
Thursday, May 24, 1804
Honored Guests:

(We will make use at once of what we have already understood,
and take a shortcut without further repetition
and closer definition of subordinate terms.
You know that such a thing is possible
in the science of knowing, and why.
That is, [it is possible] because
the subordinate terms will recur
in their full developmental clarity during the descent;
and the ascent is undertaken not for the subject matter itself,
but for clearing our vision
and opening it to the absolute
by abstracting from all relations.)

I connect this with what has gone before:
the light has been presupposed as an absolute “from.”
Then we immediately proved the legitimacy of
this presupposition by means of
its bare possibility and facticity,
because we ourselves were light and knowing.
Based on this last key step in the proof,
the presupposition is true and legitimate in the “We”;
not, of course, in the previous “We” that freely posited premises,
(since in this case knowing posits itself,
as was clearly explained yesterday).
Instead, [it is true and legitimate] in the We
that merges into the light,
and is identical with it.
Moreover, it truly is just as it factically occurs,
but it occurs as a presupposition.
Hence, taken strictly
(as we have not so far taken it, and for good reasons)
it has been truly and factically proven that
the light can presuppose itself as a “from,”
and that in us it actually does so.
In us, to the extent that we have merged,
and disappear identically, into the light itself =
[we] are the science of knowing.
Unnoticed, this presupposition has made itself,
and we will build on that.
But [the We] on the occasion of which it made itself,
has in that sense not even made itself,
instead we, who are freely abstracting and reflecting, have made it.
Consequently, by this “We” one may well mean that
light makes itself into a “from” only
in the science of knowing, as a higher, absolute knowing;
and so we provisionally indicate a distinguishing ground
(for which we have been searching)
between lower, ordinary, empirical knowing
and higher, scientific, genetic knowing.

We said, “It is presupposed.”
However, all presuppositions
bring along a hypothetical “should”;
and let themselves be expressed through it.
In fact, we have not argued differently than this
in the two previous sessions
when analyzing the contents of this “from”:
“Is there light” = “if light is to be”
and “Is there an absolute 'from'” =
“If there should be an absolute 'from,'
then must ____ ,” etc.
However, we have not only presupposed
the absolute oneness of the light hypothetically;
instead we have also realized it unconditionally.
To be sure, we have done so only in its qualitative character
(which as you will remember, was itself a result of the “from”)
as was the case with the absolute origin in knowing
as permeating the “from” in its qualitative oneness,
as we discovered yesterday.
Hence, both are a result of the hypotheticalness,
so that only pure, bare oneness,
henceforth presented as inconceivable
and understood as categorical,
remains left over.
I wished to undertake the delineation of
this very boundary in passing,
and it is commended to you.

Now back.
1. Our reasoning has proceeded in
the hypothetical form of a “should”;
and this to be sure unconditionally as itself knowing,
and as primordial knowing,
since knowing itself has posited this “from,”
then transcended this posited and objectified “from,”
which we analyze from below and derive from it.

(This as well about method.
Obviously we are once again reflecting about
what we were and did in the previous presupposition and analysis,
in the same way we have proceeded in our entire ascent;
and I could have proclaimed our activities in just this form.
Purely because we have left the realm of arbitrary freedom behind
and have arrived with our own effort in the realm of organic law,
I preferred to compel you to the present reflection
through the reminder that indeed everything
grounds itself simply on the presupposition,
rather than appealing to your freedom.)

2. In its innermost essence,
a “should” is itself genesis and demands a genesis.
This is easily understood;
you ask, If such and such should be,
then is it or isn't it?
The “should” tells you nothing about this.
What then does it say?
It sees a principle;
therefore, it explains categorically
that being can be admitted only on
the condition of the principle.
Thus, only genetic being,
or being's genesis, can be admitted.
Thus, it is the absolute postulation of genesis;
and since everyone whose
transcendental sense has been awakened
will allow no genesis to be
valid in and for itself
without such a postulation,
even immediately absolute genesis,
and the genesis of objective genesis only mediately,
according to a law that we have yet to exhibit.
Or this as reinforcement:
I have said [the “should”] is
the postulation of genesis.
Now it is immediately clear that
the “should” is a postulation,
and that a postulation is a genesis,
at least an ideal one;
otherwise, it is, as such,
completely incomprehensible and
accordingly the addition “of genesis”
would not be worth while in any way.
So it is evident that, in our hypothetical “should,”
being's genesis is demanded, which, as a genesis of being,
the hypothetical “should” is content not to be able to provide.
Instead, it waits for it from a principle outside itself.
The demand, as itself a genesis
(ideal, as we have called it, in order to name it
only provisionally with this partially clear term),
however, lies in the “should,”
and the “should” is it.
Thus, there can be a disjunction
within the absolute genesis itself,
through which it would be real and ideal.
This entire disjunction,
discovering the basis of which
may well be our most important task,
can now follow from genesis,
or the “should.”
Through resolving this disjunction,
those words, which we have used so far only provisionally and
according to a dim instinct in the hope of an eventual clarification,
will themselves become clear.
This is merely a hint at a part of our system
that necessarily must remain obscure here.

However, the following is
completely clear in what has been said:
in virtue of yesterday's demonstration,
genesis = the “from” in its qualitative oneness.
We ourselves, or knowing and light as such,
which are entirely the same as us
on the level of our present speculations,
are this “from” immediately,
in that which we pursue and live.
So there is no further need for the “from”
that is posited and presupposed through
some specific act of ours or of the light,
nor for anything that we have derived from it in our analysis.
Therefore, we let it all go as just a means of ascent,
until it shows up again on our descent.
I said, “in that which we pursue and live”;
and this very pursuing and living,
as pursuing and living,
follow directly from [our] dissolution into genesis.

3.  By virtue of the hypothetical “should,”
the we, or knowing, is absolutely genetic
in relation to itself.
Because we ourselves were knowing,
we pursued it in the following way:
“should knowing be
(that is, should we ourselves be,
since we ourselves are knowing),
then ____ must” and so forth.
Thus, [it is the] genesis of nothing else,
but rather of itself,
of the simultaneously productive [one].
Thus, with this it is absolute genesis,
which carries in itself the already sufficiently
seen character of being or light:
that it is completely self-enclosed
and can never go outside itself.

4. This absolute self-enclosure
of genesis in its fundamental point
(in which it should be a genesis of genesis)
does not prevent two points of origin
or two knowings from appearing impermanently.
We ourselves conduct one when we say,
“If knowing (or we ourselves) should be ____”;
and the other one should be,
if its principle is fulfilled.

I regard insight into this distinction of two aspects of knowing,
a distinction that is still only factical, as simple.
Yet, it is so important that I cannot
very well leave it to mere luck,
and so a bit more by way of elucidation.
We ourselves are the absolute light,
the absolute light is us,
and this is genesis itself.
Nothing can depart from this;
therefore, a distinction cannot be admitted
within the subject matter itself,
without contradicting our first fundamental insight.
Hence, the disjunction that remains is not
a disjunction between two fundamentally distinct terms,
instead it is a disjunction within one,
which remains one throughout all disjunctions.
Something of this sort has already presented itself to us earlier.
Stated popularly: it is not a disjunction of two things,
but rather just different aspects of one and the same thing.

5. Letting this disjunction stand provisionally,
just as it has appeared to us factically,
with the intention of working further
on the basis of it, the question arises,
which of the two aspects is to be
considered provisionally as absolute
in order to explain the other from it?
[It is] obviously the first term,
in what we ourselves live and pursue,
given the insight aroused in us yesterday,
that seeing and light reside always
only in immediate seeing itself and
never in the seeing that is seen.
By no means could it be in the objectified is,
that waits [to receive] its being from a principle
and is therefore truly dead within.
This choice can be shown to be completely necessary
through another circumstance as well.
For if we wish to work further,
then we wish to pursue and live knowing further as well.
Therefore, in fact and absolutely, we must remain in life
and cannot abstract from it, as is evident.
To do so would exactly be not to live and search further,
but instead to remain here,
which would contradict our intention
not to stand still and instead to go further.

(In passing, this aspect is the one
that we have always called idealistic.
Thus, our science, standing between
idealistic and realistic principles,
would at last become idealistic,
and indeed, as we have seen,
be forced to do so by necessity,
and contrary to its persistent
preference for realism.
We will not promise that the matter
will come to rest with this principle,
as it now stands and as it will
at once be explained more clearly.
We can promise more confidently that
we will never again use objectivity as a principle.
From this it will follow that,
if the idealistic principle too
should prove inadequate,
we will need to find a third, higher principle
that unites the two.)

6. Just as is demanded in this principle of absolute idealism,
the inner self-genesis is presupposed
as a living inward oneness
(what this is, on that point you understand me)
as oneness, thus as light, qualitatively absolute,
only as something to be enacted
and by no means as something to be understood.
This latter [oneness is to be presupposed] as genesis
(as was made obvious yesterday in each transition
from dull to energetic thinking),
as disappearing into the arising
of an absolute “from”
that in turn merges into it,
so that seeing and this arising are entirely inseparable;
that is, as genesis of self, or I,
so that accordingly what emerges in immediate light
may be an “I” as a result of which,
light and this very We, or I,
would merge purely into one another.
The principle demands this:
inner self-genesis is to be presupposed
as intrinsically living oneness,
then also knowing's objective aspect is
allowed to stand and to be united
with the previous [aspect],
as it can be united in knowing alone.
From the genetic principle,
it would then follow that a principle must be assumed for
the absolute, inner, and living self-genesis,
and that this latter [event must occur]
in a higher knowing that united both, as is obvious.
This latter knowing is then the highest,
and the two subordinate terms are
merely what is mediated through it.

That in the higher knowing
a principle is presupposed for absolute self-genesis
means that inwardly and materially
this higher knowing is non-self-genesis.
Yet it does not not exist,
rather it exists actually and in fact;
thus [it is] positive non-self-genesis;
and yet it is immanent and is itself an I,
because this is its imperishable character, as absolute.
What else is negated besides genesis?
Nothing, and to be sure this is negated positively;
but the positive negation of genesis is an enduring being.
Thus, knowing's absolute, objective,
and presupposed being becomes evident
in this higher knowing, hence directly genetic,
as it has previously appeared merely factically.
Once again, in order to review the terms of the proof:
as a result of positing a principle
for self-genesis within knowing itself,
[we derive] the explanation of genesis as not absolute;
consequently, [we infer] its positive negation within knowing;
and consequently [we also infer] the positing within knowing
of knowing's absolute being.

If you have just grasped this rigorously,
then I can add something else in clarification.
This knowing, that only ought to exist, is of course
a self-genesis of knowing, its self-projection beyond itself,
as we, who are standing over it reconstructing
the process and its laws, very well understand.
However, the question still always remains as
to just how we arrive at this insight
and so apparently get outside of knowing.
Yet, in contrast to an absolute self-genesis,
which is itself annulled as absolute
by the addition of a principle,
the immanent knowing that never can get outside itself
for just that reason can never appear as self-genesis
but only as the negation of all genesis.
Here, therefore, there is a necessary gap
in continuity of genesis,
and a projection per hiatum,
but here presumably not an irrational one.
Rather, it is [a projection] which separates
reason in its pure oneness from all appearance,
and annuls the reality of appearance in comparison with it.

“Reason,” I say, in order to clarify this for us;
in this case we were only concerned with deriving
the form of pure being and persistence.
In our case, this persistence is now,
and certainly always and eternally, genesis.
This existing knowing, which to that extent
is not genetic as regards its external form,
is enclosed in itself in unchangeable oneness,
and so indeed [is] also genesis,
just as it seems to be above.
Thereby the absolute, inward awareness declares itself,
without any external perceiving, knowing, or intuiting,
all of which fall out in self-genesis,
[an awareness] of an original principle
and an original principled thing
in a one-sided, and certainly not reciprocal, order;
or pure reason, a priori,
independent of all genesis,
and negating it as something absolute.

Let us go further:
in what we most recently lived and pursued,
we ourselves have not become pure reason itself,
nor dissolved into it, instead we have merely
deduced it from an insight into it.
However, this was possible only to the extent
that we presupposed self-construction as absolute, as we did;
because what should follow, follows only
on the condition that it [that is, self-construction] is
annulled as absolute in itself, from itself, and through itself;
and this was the center {Nerv} of our proof.
Since it was mentioned previously that the higher knowing,
which projects reason, is also at bottom self-genesis
and consequently does not just appear to be,
we can very appropriately call this self-genesis
the reconstruction of the non-appearing original genesis,
thus the clarification of the terms of the original genesis,
hence [we can call it] the understanding.
Accordingly, it follows for us that
there is no insight into the essence of reason
without presupposing understanding as absolute;
conversely, [there is] no insight into
the essence of understanding except by
means of its absolute negation through reason.
However, the highest, in which we remain,
is the insight into both,
and this necessarily posits both,
although [it posits] the one in order to negate it.
From this standpoint, we are the understanding of reason,
and the reason of understanding, and thus both in oneness.
Now the disjunction stands forth in its clearest definition.
Just one more principle and the matter will be completely explained.
[We will talk] about this, next Monday.

This besides; I regard what I have just
presented to you as not at all easy.
However, that lies in the subject matter,
and we have to go through it sometime,
if we want to see solid ground.
I can promise you a bit more illumination on this
from an insight into the principle we are still seeking,
but then the difficulty will lie in the principle itself.

One cannot speak properly in front of others about
speculation in these heights freely and without preparation,
since one has enough work speaking of it in formal, prepared lectures.
For this reason, and in order to escape our mutual impulse
nevertheless to handle this matter freely,
[we will now have] a special discussion period.
