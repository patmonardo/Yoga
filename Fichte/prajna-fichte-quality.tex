Fifth Lecture
Monday, April 23, 1804
Honored Guests:

It might be appropriate to cancel the lecture this Wednesday,
because of the general day of prayer.
I would have taken up preliminary matters again today
had I not also seen the necessity, because of this,
for sparing you the strongest nourishment.

Indeed, I have already adduced and shared with you
everything which is conducive for understanding these lectures
and which helps one enter their standpoint except for two things:
first, what really cannot be imparted,
namely the knack for grasping them;
and second, some observations which tend
not to be received well and which
I had hoped to be able to omit this time.

As far as concerns the first item,
the knack for grasping these lectures is
the knack of full, complete attention.
This should be acquired and exercised
before one enters on the study of the science of knowing.
For this reason, in the written prospectus for these lecture
(available at the place of subscription)
I have established as the sole, but serious precondition
for understanding this science the requirement
that participants should have experience with
fundamental scientific inquiry.
Not, of course, for the sake of
the specific information so gained,
since none of that is presupposed
or even accepted here without qualification;
rather, I did so because this study also
awakens and exercises full, complete attention.
Collaterally, one gains a knowledge of scientific terminology,
which we are using freely here.
Full, complete attention, I have said,
which throws itself into the present object
with all its spiritual power,
puts itself there and is completely absorbed in it,
so that no other thought or fancy can occur;
since there is no room for anything strange
in a spirit totally absorbed in its object:
full, complete attention as distinct
from that partial attention
which hears with half an ear
and thinks with half its thinking power,
interrupted and criss-crossed by
all kinds of fugitive thoughts and fantasies,
which eventually succeed in totally overwhelming
the mind so that the person gradually falls
into a dreamy fog with eyes wide open.
And if he should chance to come back to awareness,
he will wonder where he is and what he has heard.
This full, complete attention of which I speak,
and which only those who possess can recognize, has no degrees.
It is distinguished from that scattered attention,
which is capable of many degrees, not merely quantitatively;
but is totally different and even logically contradictory to it.
It fills the spirit completely,
while incomplete attention does not.

For understanding these lectures everything depends
on one's possessing this kind of attention;
everything which makes understanding difficult
or impossible follows solely from its absence;
if one is freed from this lack,
then all these things are ripped out by the roots.
So, for example, if this lack is removed,
the phenomenon of believing one cannot intuit
the particular theorems presented in the lecture
because one is too quickly confused will fall away.
As I like to repeat frequently so no one will lose heart,
in the nature of our science the same thing is
constantly repeated in the most various terms
and for the most diverse purposes,
so that an insight which is
missed on one occasion can be produced
or made good on another occasion.
But strictly speaking it should be, and is actually,
demanded of everyone that they see into each theorem
when it is initially presented.
So those for whom things don't happen as we expect
have not used these lectures as they should be used;
and if things don't flow smoothly,
they have only themselves to blame.
To give the most decisive proof that
what I demand is possible under conditions
of complete attentiveness;
any distinction between faster or slower mental capacity
has no place in the science of knowing,
and the presentation of this science aims
neither at good nor slow minds
but at minds as such,
if only they can pay attention.
For this is our procedure as it has gone up to now
and as it will remain:
first we are required to construct
a specific concept internally.
This is not difficult:
anyone just paying attention to the
description can do it;
and we construct it in front of him.
Next, hold together what has been constructed;
and then, without any assistance from us,
an insight will spring up by itself,
like a lightning flash.
The slowness or speed of one's mind has
nothing more to do in this final event,
because the mind in general has no role in it.
For we do not create the truth,
and things would be badly arranged
if we had to do so;
rather, truth creates itself by its own power,
and it does so wherever the conditions
of its creation are present,
in the same way and at the same rate.
And in case the ensuing manifestness did not arise
for someone who had really carried out
the construction which we postulated,
this would only mean that he did not sustain
the construction in all clarity and power,
but instead that it faded
because some distraction intervened.
That is, he did not place his total attention
on the present operation.

Or, if this lack is removed,
another equally common phenomenon
would be destroyed at its root:
namely, that an illusion which
we have already revealed as an illusion,
nevertheless can return and deceive us again,
as if it had truth and meaning,
or at least confuse us and make us uncertain
about insights that we had otherwise already achieved.
For example, if you have really seen that
in intuiting the one, eternally self-same knowing,
all differentiation into subjective and objective
completely disappears as arising only in what is changeable,
then how can you ever again allow yourself
to be deceived by the illusion
(which, to be sure, as an illusion, can always recur)
that you yourself are the very thing
that objectively posits this one knowing
(that you therefore are the subject with it as your object)?
Because you indeed have seen once and for all
that this disjunction is always and everywhere the same
illusion and never the truth,
no matter in what form or in what place it might be manifest.
If you have seen this, then you have attained this insight
and dissolved into it.
How then could you possibly cease being what you are, unless,
because you have not really entirely become it
but only entered half way,
you never threw yourself into,
and rooted yourself in,
this insight which now remains
wavering and deceptive for you.
In this case the old illusion returns
at the first opportunity.
But note well the sequence:
the insight does not leave you
because the illusion steps in,
rather the illusion enters
because the insight has left you!
So much as regards the talent of
total attentiveness as the sure and unerring
means of correctly grasping the science of knowing.
Second, I want to mention a few more things
that block apprehension of the science of knowing,
because they do not allow proper attention to arise.
I take these things up collectively in their oneness,
as is my custom (as will likely be the
custom of anyone who becomes familiar with the science).
They arise together from a lack in one's love of science,
which is either a simple lack:
a weak, powerless, and distracted love;
or a secret hatred of knowledge because of
some other love already present in the mind.

Let us first take up the last:
the other love which leads
to a secret hatred of knowledge is
the same one from which hatred
arises against every good,
namely, a perverted self-love
for the empirically arisen self
instead of for the self which is immersed
in the good, the true and the beautiful.
This love is either that of self-valuation,
which therefore becomes pride,
or that of self-enjoyment,
which therefore becomes spiritual lasciviousness.

The first of these is unwilling to admit
that anything could occur in the domain of knowing
that it had not itself discovered,
and long been aware of.
Whether it explicitly says so or not, to such a one,
the science of knowing's claim to absolute novelty
seems to be a statement of contempt for itself.
It would very much like to humiliate
this arrogance on the part of this science;
for this is how things must seem to it.
Therefore, instead of giving itself freely
and with complete attention from the start,
it focuses on whether it can possibly catch
this science in some failing;
and ambivalence of purpose distracts it
so that it misses the right idea,
does not enter into the true subject matter,
but rather finds just what it was looking for
in the confused concepts
which it obtains of the enterprise,
weaknesses in its “science of knowing.

The other mode of thought,
love for the empirical self's self-enjoyment,
loves the free play of its mental capacities
(which it partially becomes)
with the objects of knowing
(which it in the same manner partially becomes).
I think it can best be characterized in the following way:
it calls making something up “thinking”
and it names the invention of a truth
for oneself in one's own body
“thinking for oneself.
A science which brings all
thinking without exception
under the most stringent rules
and annuls all freedom of spirit
in the one, eternal, self-sustaining truth
can hardly please such a disposition,
and it must also incite this later
mode of thought to the same
secret polemic against itself,
producing the very consequences we have just described.
Moreover [just to take this opportunity
to make the point decisively]
I do not warn each of you against
this secret inner polemic for my own sake,
but rather for yours,
because one cannot achieve correct attention,
let alone understanding,
while doing it.
If one will only first understand
and master the science of knowing
and then feel a desire to argue against it,
I will have nothing more to say against this.

Or again it could be a ruling passion
for the merely empirical
and for the absolute impossibility of
feeling and enjoying one's spirit in any way
except as trained memory.
These personified memories are not capable of such secret hate;
but they necessarily become very ill-tempered in this setting.
They want what they call results:
namely what can be observed and
can be reproduced in similar circumstances;
“[that is] a sufficient statement and one that says something.
Now when they think they have grasped something of this kind,
the next lecture arrives further qualified,
differently arranged, symbols and expressions change,
so that not much remains from the hard-won treasure.
“What eccentricity! Why couldn't the man
just say what he meant from the start?”
For people like that the most extreme confusion and contradiction
must arise from what has the purest oneness and strongest coherence,
simply because it is the true inner coherence
and not the merely external schematic coherence,
which is all they really want.

Originally, I first mentioned cold, weak love of science
(which is not exactly hate) as an obstacle to attention.
Namely, whoever seeks, desires, or wishes in science
for something besides science itself does not love it
as it ought to be loved and will never enjoy
its complete love and favor in return.
Even the most beautiful of all purposes
(that of moral improvement)
is too lowly in this case;
what should I say of other, obviously inferior ones!
Love of the absolute (or God) is the rational spirit's true element,
in which alone it finds peace and blessedness;
but science is the absolute's sweet expression;
and, like the absolute, this can be loved only for its own sake.
It is self-evident that there is no room for anything
common or ignoble in a soul given over to this love,
and that its purification and healing are intrinsic to it.

This love, like every absolute,
recognizes only the one who has it.
To those who are not yet possessed by it,
it can give only the negative advice to remove
all false loves and subordinate purposes,
and to allow nothing of that kind to arise in them
so that the right will spontaneously manifest itself
without any assistance from them.
This much should be remembered
once and for all on this subject.

Now to the topic set for today.
When I presented it, I already suspected that
my last talk might seem too rigorous and
deep for a fourth lecture,
and it was made so in part to help me
discover what mode of presentation
I would need to follow with this new audience.
Now I will repeat it in a suitable form:

1. First, a remark that is valid
for all previous and subsequent lectures,
and that will be very useful in order
to reproduce and review them.
Our procedure is almost always this:

a. we perform something, undoubtedly led in this process
by a rule of reason which operates immediately in us.
What in this case we really are in our highest peak,
and that in which we culminate, is still only facticity.

b. we then search out and reveal the law
which guided us mechanically in the initial action.
Hence, we see mediately into what we previously
had seen into immediately,
on the basis of its principle
and the ground of its being as it is;
and we penetrate it in the origin
of its determinateness.
In this way we will ascend
from factical terms to genetic ones.
These genetic elements can themselves
become factical in another perspective,
in which case we would be compelled again
in connection with this new facticity
to ascend genetically,
until we arrive at the absolute source,
the source of the science of knowing.
This is now noted and can be clarified
in reference to the consequences:
x is nothing but the developmental link to y,
and y in turn to z.

Now, whoever either has not comprehended z from the start,
or has lost and forgotten this understanding in the process,
for that person neither x nor y exists and
the entire lecture has become an oration about nothing,
through no fault of the lecturer.
This, I say, has been and will continue to be
our procedure for some time.
It was so in the last lecture.
Whoever may have recognized this process
(it was obvious for everyone to see,
and the earlier distinction between
factical and genetic manifestness
should have led right to it)
could have reproduced the entire lecture
and made it intrinsically clear by simply asking:
Was any such factical term presented,
and which one was it?
Which could it be after the earlier ones?
Did the presentation succeed in
presenting the genetic term after the factical one?
Assume that I may have completely forgotten
this second step or perhaps never heard of it.
Then I will have to discover it for myself
just as it was discovered in the talk,
because the rule of reason is unitary, and
all reason which simply collects itself is self-same.

So what was the factical term?
It was not in A and not in the point,
but unconditionally in both.
We have now grasped this,
it has made itself evident,
and so it is.
Analyze it however you wish:
it contains A, the point, and,
in the background, a union of both.
With the first two terms denied as
the true point of oneness,
the other one is thereby posited,
and in this fashion you will not
arrive at any other term.
It is so, factically.
But now I ask on another level:
How have we brought it about that
this insight has arisen for us?
We did not reflect further on the content,
which we completely abandoned;
but focused instead on the procedure,
asking about the origin.
In this way, as I indicated earlier,
the initial, materially constituted,
immediacy becomes mediately visible:
once such an origination is posited,
this factical insight is posited,
but solely by means of
our establishment of the origin.

How did we do it?
Apparently we made a division in something which
on the other hand ought to be a oneness.
I say division and disjunction in a general way,
because one can ignore the fact that the terms separated
are “A” and the point, when it is a question of the act qua act.
This division shows itself to be invalid in an immediate insight.
We did not produce this insight because we wished to,
instead it produced itself absolutely
(not from any ground or premise)
in an absolutely self-generating
and self-presenting manifestness,
or pure light.
The distinction, in the sense that it should be valid by itself,
would therefore be annulled by the [one's] manifestness.
On the contrary, the same manifestness posits
a self-same, intrinsically valid oneness
which is incapable of any inner disjunction.
The principle of division equals the principle of construction,
and thus of the concept as well.
[Now, consider] this principle in its absoluteness
(and by that I mean, the principle as dividing
the wholly and intrinsically one,
which is seen into as one,
[working] wholly and absolutely by itself
without other ground and doing so rather
in contradiction to the truth)
this principle is negated in its absoluteness,
in its intrinsic validity.
It is seen unconditionally as negated,
and therefore it is negated
in and through the absolute light.
Thus, in this annulling of the absolute concept
in relation to intrinsic being,
this being is inconceivable.
Without this relation it is not even inconceivable
but rather is only absolute self-sufficiency.
But further even this predicate “is”
derives from manifestness.
Hence the sole remaining ground and midpoint is
the pure light, and so on.

This was by far the greatest part
of the earlier talk's content.
That this all has intrinsic clarity
and incomparable manifestness is
obvious to anyone who sees it at all.
I am convinced that it could not be presented
with greater order, distinctness, clarity,
and precision than in this case.
Whomever doesn't see it now must be lacking
in the undivided attention required here.

The part still to be added, which I will repeat now,
is another developmental analysis of the insight we have achieved:
I said that we saw into [the fact] that the light was the sole midpoint.
In this reflexion, the process we initially unfolded itself becomes factical.
Now, since in this case we have produced nothing,
and [since] rather the insight as insight has produced itself,
we cannot ask as we did before how we did this,
but we can rise to greater clarity.
It is clear; if only we see that it is the light,
then we are not immediately consumed in this light,
instead we have the light present
through its agent or representative,
that is, through an insight into the light,
into its originality or absolute quality.
We must disregard the fact that
we cannot now ask without contradiction
how the light itself is produced;
since it is recognized as the principle of absolute creation,
the question would deny the insight again.
We can certainly still ask how the insight into light
(which we called not light itself
but rather its agent and representative)
has been produced; that is a different question.
Therefore we only need to pay
attention to how the production of this insight has taken place.
1. We have put ourselves in the condition;
2. how could we do this?
Both are true:
not the light and not even the insight into light,
but the insight into the insight into the light
stands between both.
The emanence and the immanence:
these are matters with which we must concern ourselves.

Regarding the entire distinction
between the immanence and emanence
of the production of an insight into light,
one must not forget that the same thing
extends to the insight and to the light itself.
As before, the objective light qua objective
neither is, nor can become, the one true light.
Instead pure light enters insight under this aspect.
But here we have won this:
that the highest object is no longer
substance for itself, but light.
Substance is only the form of light as self-sufficient.
On the other hand, insight (subjectivity),
actually the inner expression and life of light,
disengages itself from the negation of
the concept, and of division.
Can you penetrate into the true midpoint
more deeply in any other way?
Into the entirely unique concept
that is nevertheless required here?
1. It is clear that its being is not grasped
except in immediate doing.
2. It can be made clear here that immediate doing is
a dissolving into immanence
(the initially uncovered making of his being,
as this sort of making).
First of all, doing deposes being
and being deposes doing
or stated otherwise:
here is the fundamental reversal;
and this must be understood:
doing replaces actual being;
being beyond all being
(not actual or material being)
deposes doing.
Now it is also very clear that
to posit it without any actuality
(just in barely hypothetical form and in a “should”)
and thereby to deduce and materialize doing itself,
thus also to intellectualize and idealize,
being negates itself in the other
as a result of its own doing.
Thus we once again come back to the previous point,
and we find the previous principle again in this self-negation.
Perhaps this is just the concept of being, dead in-itself:
clearly there is a division in it
between being (what endures) and doing:
and indeed as a division this is intrinsic
to constructing non-separation or oneness.
Thus this negation would be true in a certain respect.
It is right since primordially the division
into being and doing is nothing at all.

Sixth Lecture
Thursday, April 26, 1804
Honorable Guests

In today's and tomorrow's talks I will continue
with the further development of
what has already been presented.
In doing so I aim at an end
useful to both sorts of listeners.
That is, since, like all philosophy,
the science of knowing has the task of
tracing all multiplicity back to absolute oneness
(and, correlatively, to deduce all multiplicity from oneness),
it is clear that it itself stands
neither in oneness nor in multiplicity,
but rather stays persistently between both.
It never descends into absolute multiplicity,
which must after all exist and indeed does exist
(as mere empirical givenness),
but rather it maintains the perspective from above,
from the standpoint of its origin.
Therefore, in the science of knowing we will be
very busy with multiplicity and disjunctions.
Now, these disjunctions, or differences and distinctions,
which the science of knowing has to make
are new and previously unrecognized.
Therefore, in the usual modes of representation
and speech from which we begin,
these differences collapse unnoticed into oneness,
and when we are required to draw them,
they seem very minute.
It is hairsplitting, as the literary rabble has put it;
and it is necessary that it be so,
since if a science that is to trace
everything that is multiple, that is,
everything in which a distinction can be constructed,
back to oneness allows any distinction
that the science could possibly make to remain hidden,
then it has failed in its purpose.
Therefore one of the main problems for
the science of knowing consists in just this:
making its very precise distinctions visible and distinct;
so that when this problem is finally solved,
these distinctions will be fixed and established
in the mind of those studying it,
so that they will never again confuse them.
I think that both difficulties
will be significantly reduced
if I lay out for you in advance
(so far as this is possible)
the general schema and basic rule
in terms of which these divisions will come about
although in an empty and purely formal way.
And, so that this schema can be
correctly understood and noticed,
I will deduce it in its unity
and from its principles,
to the extent that this can be done
with what we know so far.

To begin I mention in general the following:

1. Since, according to the nature of our science,
we must stand neither in oneness nor in multiplicity
but instead between the two, it is clear
[and I focus on this because I believe
I have detected several of you making this error]
that no oneness at all
that appears to us as a simple oneness,
or that will appear to us as such in what follows,
can be the true oneness.
Rather, the true and proper oneness
can only be the principle simultaneously
of both the apparent oneness
and the apparent multiplicity.
And it cannot be this as something external,
such that it merely projects oneness
and the principle of multiplicity,
throwing off an objective appearance;
rather it must be so inwardly and organically,
so that it cannot be a principle of oneness
without at the same moment being
a principle of disjunction, and vice versa;
and it must be comprehended as such.
Oneness consists in just this absolute, inwardly living,
active and powerful, and utterly irrepressible essence.
To put it simply, oneness cannot in any way consist
in what we see or conceive as the science of knowing,
because that would be something objective;
rather it consists in what we are, and pursue, and live.
Let this be introduced once and for all
to characterize the oneness which we seek
and to eliminate all the errors about this central point
which, if they continued, would necessarily
be very confusing in what follows.
And be warned, not only so that you don't
content yourself merely with that sort of oneness,
taken just relatively and one-sidedly,
as if it was the absolute,
but also so that if I in this lecture,
or any other philosopher,
remain content with such a oneness,
you will know and state strongly that
this philosopher has stopped half way
and has not made things clear.

2. In consequence:
Since the true oneness is
the principle simultaneously of
the (apparent) oneness and of disjunction,
and not one without the other,
it therefore makes no difference
whether we regard what we will present
provisionally as our highest principle
at each juncture in the progress of our lectures
as a principle of oneness or of disjunction.
Both perspectives are one-sided,
merely our necessary point of view,
but not true in themselves.
Implicitly the principle is neither one nor the other;
rather it is both as an organic oneness
and is itself their organic oneness.

Therefore, so that I can say it even more clearly:
first of all only principles can enter the circle of our science.
Whatever is not in any possible respect a principle,
but is instead only a principled result and phenomenal,
falls to the empirical level,
which, of course, we understand
on the basis of its principle,
but which we never scientifically construct,
as this cannot ever be done.
Then, every principle that enters our science
(and indeed every principle qua principle) is
simultaneously a principle of oneness and multiplicity,
and it is truly understood only insofar
as it is conceived in those terms.
Our own scientific life and activity,
therefore, to the extent that it is a process of
penetrating and merging with the principle,
never enters into that oneness
which is opposed to multiplicity
nor into multiplicity;
rather it maintains itself undisturbed between both,
just like the principle.
Finally each principle in which we stand
(and we never stand anywhere but in a principle)
yields an absolutely self-differentiating oneness:
x {a = (a) — y} z
The only question is whether this oneness is the highest.
If not, and there are several such a's
(a 1, a 2, a 3, ...)
then not just in the former case
but in the latter as well,
“(a)” is still in this regard
a principle of disjunction for unities,
which to be sure would be unities in relation to
x y z
but in connection to one another,
they would by no means be so.
For these a's we need a new a,
until we have uncovered the highest oneness,
which would be the absolute disjunction,
just as we have described it in relation
to the absolute oneness.
This gives us the first general model
for the procedure of the science of knowing.
One comment here: the interchangeability of
direction from a to x, y, z
and vice versa is evident,
and this greatly aids their linkage.

3. Now the same point, from another side and deeper.
As regards the explanation we have been
pursuing up to now in this hour
(not about the principle of disjunction,
since strictly speaking there is no such thing,
but rather our view of the one implicit principle
as a principle of disjunction,
a one-sided view that we undoubtedly must start with
since the science of knowing finds us completely trapped
in this one-sidedness and starts from there),
we find ourselves trapped in the familiar,
frequently cited inexpressibility:
that the oneness is to separate itself
at one stroke into being and thinking
and into x, y, z, both equally immediately.
In expressing this verbally and in diagrams,
we were compelled to make one of the two the immediate term,
though our inner insight contradicted this,
negating the intrinsic adequacy of
the construction of our mode of expression.
Expressing this curious relation in
its logical form will help us to speak precisely:
in this actual disjunction there are
two distinguishing grounds
neither of which can occur without the other.
Therefore, expressing the matter just in the way
we have done is probably an empirically discovered turn of phrase,
since we found it on the occasion of explaining Kant's philosophy
and in adding a disjunction,
which has been demonstrated
neither by Kant nor yet by us,
not only between being and thinking,
but between sensible and supersensible being and thinking.
And the claim that both distinguishing grounds are
absolutely inseparable would therefore be grounded simply on this:
If what is evident in empirical self-observation is to be explained,
then we must assume that the distinguishing grounds are inseparable.
This “must” grounds itself directly
on a law of reason which operates in us mechanically
and without our awareness [of it].
Thus at bottom we had only an empirical basis
on top of which we postulated a supersensible one;
that is, we began a synthesis post factum.
This cannot be blamed on the science of knowing as
long as it is the science of knowing;
it is not permitted simply to report this
inseparability of the distinguishing grounds,
instead it must grasp this ground conceptually
in its principle and from its principle as necessary.
It must therefore see into it genetically and mediately.
“It grasps this ground conceptually”
means it sees the distinguishing grounds
(and by no means just the actual
factically evident distinctions;
whoever remains with these has simply not
finished the climb we have just completed)
as themselves disjunctive terms of a higher oneness,
in which they are one and inseparable
as they are when enacted, so that, as we have said,
it remains one and the same stroke.
But they are separable and conceptually distinguishable,
as we may provisionally think in order thereby
to have something to think.
“Separable” so that, for example, the ground for
distinguishing being and thinking can appear as
a further determination and modification of
the ground for distinguishing sensible and supersensible,
and so that likewise from another point of view
the latter can appear vice versa as
a further determination and modification of the former.
As has been said, when it is enacted,
this disjunction in the oneness of the mere concept
concresces into a factical oneness which is not further distinguishable,
and in this concrete union, every eye that remains factical is
entirely closed to the higher world of the conceptual beyond.

(And now a number of additional remarks.
I ask that you not allow yourself to be
distracted while I state them):

1. I have now specified the boundary point
between absolutely all factical insight
and truly philosophical and genetic insight
entirely and exactly,
and I have opened up the sources of
the entirely new world in concepts
which appear only in the science of knowing.
The creation and essence of this new world is found just here,
in the negation of the primordial disjunctive act as immediate
and in the insight into this primordial act's principle
materially, that it is thus, and formally, that it is at all.

2. I have here explained the essence of
the final scientific form of the science of knowing
from a single point more precisely
than I have been able to do previously.
The main point of this scientific form
resides in seeing into the oneness
of the distinguishing ground:
of being and thinking as one
and of sensible and supersensible
(as I will say in the meantime) as one.
Whoever has understood this
[as it is to be understood up to now,
namely as an empty form]
and holds on to it firmly,
can scarcely make any further errors
in the subsequent actual employment of this form.

3. In order to assist both your memories and repeated reproduction:
in the last hour I said that the path of our lectures was,
and would for a long time continue to be,
that we first present something
in factical manifestness
and then would ascend to
a genetic insight into this object
on the basis of its principles.
This is exactly what we have done
in the just-completed explanation.
Already since the second lecture,
we had developed the inseparability of
both recognized distinguishing grounds
historically out of Kant's own statement,
and we admitted the factical correctness
of this statement.
Now we raise ourselves
[to be sure not to a genetic insight
into the principle of this inseparability,
since we do not yet actually know
this inseparability itself, nor its terms,
but have only assumed all this temporarily
and for the time being]
but to the genetic insight,
which must be the form of this principle,
if such inseparability and
such a genetic principle are to exist).

Now back to our project.
It is also not at all our intention to see directly
into this inseparability and its principle,
since these do not allow themselves
to be “seen into” directly.
And in fact:

4. to take the process further
by our beginning we have already jumped past this principle,
which was discussed here in its form simply
for the intelligibility of what really concerns us,
in order to derive it deductively;
and indeed we have already uncovered
good preliminaries for this derivation.
Namely, you recall that we have already presented
a point of oneness and difference,
which covers the oneness of these distinguishing grounds:
the one between A and the point,
and, in connection with the deeper distinguishing grounds
which are materially different from oneness and difference,
we have said that this might be only a profounder view
of this same higher principle,
disregarding the fact that we could not
yet prove this contention.

[Let me] repeat a third time this oneness
which has already been constructed twice before our eyes.
To that end I recall only that
an absolute dividing principle was evident there;
not A and the mentioned disjunctive point,
since these are principled results of absolute division,
which disappear when one looks to the principle;
but rather the living absolute separation within us.
I stress again what I said before about this essential point,
which is designed to tear our eyes away from facticity
and to lead them into the world of the pure concept,
if I can succeed in making it clearer.
I hope nobody assumes here that the act of thinking
the distinction between A and . is actually grounded
in an original distinction in these things,
independent of our thinking.
Or, in case someone is led to this conclusion
by the previous factical ascent with which we had to begin,
he will recover from this idea if he considers
that in A and . he thinks only the oneness
which, according to him, should be unconditionally one
containing no distinction within itself;
that he himself thus makes clear that
the distinction is not based in the object itself,
since he could not think the object except
by virtue of this distinction;
that he thus expressly makes his own thinking as thinking
into the distinguishing principle.
But the validity and result of this product of thought
expressly surrenders and dies
in relation to the thing itself.
With it as the root, its products A and . are also
doubtlessly uprooted and destroyed as intrinsically valid.
Thus away with all words and signs!
Nothing remains except our living thinking and insight,
which can't be shown on a blackboard
nor be represented in any way
but can only be surrendered to nature.

We intuit, I say, that it rests neither in A nor in .
but rather in the absolute oneness of both;
we intuit it unconditionally without sources or premises.
Absolute insight therefore presents itself here.
Pure intuition, pure light,
from nothing out of nothing, going nowhere.
To be sure bringing oneness with it,
but in no way based on it.

Here everything depends on this:
that each person correctly identify
with this insight, in this pure light;
if each one does, then nothing will happen
to extinguish this light again
and to separate it from yourself.
Each will see that the light exists only
insofar as it intuits vitally in him,
even intuits what has been established.
The light exists only in living
self-presentation as absolute insight,
and whomever it does not thus grasp,
hold, and fix in the place where we now stand,
that one never arrives at the living light,
no matter what apparent substitute for it he may have.

5. Consideration of the light in its inner quality,
and what follows from it,
to which we will proceed after this step,
is entirely different from this
surrender and disappearance into the living light.
This consideration as such will inwardly
objectify and kill the light,
as we will soon see more precisely.
But first we said:
only the light remains as eternal and absolute;
and this [light], through its own inner immediate essence,
sets down what is self-subsistent,
and this latter loses its previously admitted
immediacy to the light, whose product it is.
But there is no life or expression of this light
except through negating the concept, and hence through positing it.
As we said, no expression or life of the light could arise
unless we first unconditionally posit and see a life as
a necessary determination of the light's being,
without which no being is ever reached, except,
that is, in the light itself;
its essence in itself and its being,
which can only be a living being.
Thereby, however, what matters to us,
since we add life to light, is
that we have nevertheless divided the two,
have therefore, as I said, actually killed
the light's inner liveliness
by our act of distinction;
that is, by the concept.
Now, to be sure, we contradict ourselves,
ipso facto denying that life can be distinguished from light,
the very thing we have just accomplished.
This is a contradiction which may well be essential and necessary,
since it may implicitly in itself be the negation of
the concept to which, according to the foregoing,
it must someday certainly come.
(What I'm saying now is added parenthetically for future use.
It is easy to remember;
since it connects with our reflections on
the objectifying consideration of the light,
and allows itself to be reproduced from it
for anyone who has paid even a little attention
to our proceedings, in case he has otherwise completely forgotten.)

To review: in this consideration of the light,
light shows itself through its mere positedness,
absolutely and without anything further,
as the ground for a self-subsisting being
and at the same time for the concept;
and, to be sure, for the concept in a twofold sense:
in part as negated, precisely in its intrinsic validity;
and in part as posited, posited as absolute
but not realized (though still actual);
that is, as appearance and as the light's vitality,
but in no way an appearance that conditions its inner essence.
By the concept's being posited,
A and . are also posited;
to be sure as appearance
and certainly not as primordial appearance,
but rather as conditioning appearance
and the inner life of the primordial appearance = b,
thus appearance's appearance.
[In its inner life,
appearing should occur again as
the unity of the above mentioned distinguishing grounds,
its life comes from the livingness of the concept,
this in turn stems from the light's livingness,
thus an appearance of appearance of appearance.
Everything is brought together again when enacted.
This would be the schema of
an established, rule-governed descent,
in no way like [the one outlined] yesterday,
one equally possible on all sides
and therefore very exposed to error.]

Seventh Lecture
Friday, April 27, 1804
Honorable Guests:

[Our] purpose [here is] to give a brief account of
the rules according to which the disjunction
we will have to make proceeds.

1. [It is a disjunction into] principles,
with each being equally a principle
of unity and of disjunction.

2. [It provides a deduction from this to a general]
schema of the total empirical domain
according to the form of its genetic principle.
This will be an entirely new explanation,
because I observed to my pleasure that
some of you had seen that there is
something else even more deeply hidden,
despite the fact that you could not
assist yourselves [to find it],
which, to be sure, was not even required.

3. With the remark that our investigation has
already gone beyond this principle to a higher one
and that it has already begun to deduce this principle itself,
[I repeat] this achieved insight.
Neither in A nor .;
for us the oneness beyond is nothing in itself,
although it is posited as in itself;
rather it exists only through the light
and in the light, and (is) its projection;
light itself; contemplation of light.
Now back to our former topic.

[There is] one further step which opens up
a whole new side of our investigation;
as I said: we have already previously begun to
derive the principle of oneness and disjunction of
materially different principles of division,
only without recognizing that this was what we were doing.

So (dropping what we have done so far until I take it up again)
recall with me and consider the following:
when we observe the light, the light is objectified,
alienated from us and killed as something primordial.
We have explained what is attributed materially to
the light in this observation of the light,
and we have connected this explanation to
the schema under consideration.
Now we will explain this observing itself
in its inner form, that is,
no longer asking what it contains and leads to,
but rather how it itself inwardly occurs,
while also rising to its principle
and viewing it genetically to some extent.
It is immediately clear that:
the light is in us
(that is, in what we ourselves are and do in observing it)
not immediately but rather through a representative or proxy,
which objectifies it as such, and so kills it.
So then, where does the highest oneness and
the true principle now rest?
No longer, as above, in the light itself,
since we, as living, dissolve in the light.
Neither [is it] in the representative and image
of the light which is to be identified now:
because it is clear that a representative
without the representation of what is represented
or an image without the imaging of what it images,
is nothing.
In short, an image as such,
according to its nature,
has no intrinsic self-sufficiency,
but rather points toward some external, primordial source.
Here, therefore, we have not only, as above,
factical manifestness, as with A and .
instead [we even have] conceptual manifestness:
oneness only with disjunction, and vice versa.
“Even conceptual manifestness,” I say:
something imaged (like the light, in this case)
is not thinkable without an image,
nor likewise an image, qua image,
without something imaged.
Notice this important fact,
which will take one deep into the subject matter,
if it is properly grasped here.
In this case you carry out an act of thinking,
which has essence, spirit, and meaning
and is fully and completely self-identical
and unchangeable in relation to this essence.
I cannot share this with you directly,
nor can you share it with me;
but we can construct it,
either from the concept of something imaged
which then posits an image,
or from the image which then posits something imaged.
I ask: apart from the arrangement of the terms,
which is irrelevant here,
have we then thought two different things
in the two concepts thus fulfilled,
or have we not rather thought exactly the same thing in both,
an issue that genuinely touches the inner content of thinking?
The listener must be able to elevate himself
to the required abstract level
out of the irrelevance of the arrangement
to the essential matter of the content,
of spirit, and of meaning, and
then the insight which is intended will
immediately manifest itself to him.
Should this indeed be the case,
then an absolute oneness of content is
manifest here that remains unaltered as oneness
but that splits itself only in
the vital fulfillment of
thinking into an inessential disjunction,
which neither spoils the content
in any way nor is grounded in it.
Either [there is] an objective disjunction
into something imaged and its image,
or, if you prefer, [there is]
a subjective-objective disjunction
into a conception of something imaged
on the basis of the directly posited image,
and a conception of the image on the basis of
the directly posited imaged something.
I do advise you to prefer the latter,
since in that case you have
the disjunction first hand.
And thus in this case our principle
in the genetically-oriented view of the light
would be the concealed oneness,
which cannot be described further,
but which is lived only immediately
in this act of seeing, and which,
as the primordial concept's content,
presents itself as absolute oneness,
and as absolute disjunction,
in its living fulfillment.
Now, the something imaged
in the concept's content
should be the light:
therefore our principle (we, ourselves)
rests no longer either in the light
or in the light's representative,
but rather in the oneness in and between the two,
a oneness realized in our act of thinking.
Therefore, I have called the concept situated here
the primordial concept.
[I call it this] because what up to now
we have taken as the source
of the absolutely self-sufficient,
and which therefore appeared as the original
and was original for us,
actually first arises in the way it appears,
in its objectivity,
from this concept
as one of its disjunctive terms.
Therefore, this concept is more original
than the light itself;
hence, so far as we have yet gone,
it is in this sense what is truly original.
Thus we have given a deeper genetic explanation of the hint,
given only as a fact in the lecture just eight days ago,
concerning the representatives of the original light,
although, to be sure, we have done so
for the particular end we intend here.

Thereby you will see that the concept is
determined further and grasped more deeply
than it has been heretofore.
Until now it was a dividing principle
which, as self-sufficient, expired in the light,
and which preserved only a bare
factical existence as an appearance,
qualifying the appearance of the original light.
Further, it had no contents and acquired no contents
except that which pure light added to it
in immediate intuition through a higher synthetic unity.
Now, however, the concept has its own implicit content,
which is self-subsistent, totally unchangeable and undeniable;
and the principle of division
(which arises again in this case,
and as before is negated as intrinsically valid)
is no longer essential to it,
but instead only conditions its life,
its appearance.
The concept's content, I say, is self-subsistent;
thus it is exactly the same substantial being
which was previously projected out of intuition
and which manifests here in the concept
as prior to all intuition
and as the principle of
the objective and objectifying intuition itself.
Previously, the concept qualified
both life and the appearance of light,
and these conversely qualified the concept's being.
Therefore, it was a reciprocal influence,
and every [act of] thinking the two terms
was qualified externally.
Now the same single concept
grounds its appearance
through its own essential being;
therefore, in this concept
the image and what it images
are posited absolutely,
things which are constructed
organically only through one another.
And, hence, its appearance announces,
and is the exponent of, its inner being,
as an organic unity of the through-one-another,
which must be presupposed.
Its being for itself,
permanent and unchanging,
and as an inner organization of
the through-one-another
(essential, but in no way externally constructed)
are completely one:
therefore, in this case,
absolute oneness is grounded
and explained through itself.

We will achieve a great deal if here and now
we see fundamentally into what I meant
by the inner organic oneness
of the primordial concept,
which I mentioned just now;
since this oneness is the very thing
we will need as we continue.
In this regard I ask:
Does the image, as image,
completely and unconditionally
posit something imaged?
And if you answer “Yes,”
does not the something imaged
likewise posit such an image?
Now without further ado I admit
that both can be seen (by you)
as posited immediately by the other,
but only if you posit one of the two as prior.
But I ask you for once to abstract from your own insight.
This is possible in the way
that I will preconstruct for you now,
and in ordinary life it happens
constantly when it shouldn't.
Further, one could not ever enter
the science of knowing without this abstraction.
That is, I ask about the truth in itself,
which we recognize as being and remaining true
even if no one saw it, and we ask:
Is it not true in itself that
the image entails something imaged
and vice versa?
And, in this case,
what exactly is true in itself?
Just reduce what remains as
a pure truth to the briefest expression.
Perhaps that a posits b and b, a?
Do we want to divide the true into two parts
and then link these parts
by the empty expletive “and,”
a word which we scarcely understand
and which is the least understandable word in all language,
a word which is unexplicated by any previous philosophy?
It is indeed the synthesis post factum.
How could we, since beyond this it is certainly clear
that the determination of the terms derives
solely from their place in the sequence:
for example, that image is the consequent
because something imaged is the antecedent,
and vice versa.
Further, if one enters more deeply into the
meaning and sense of both terms,
it is clear that their meaning simply changes itself
into the expression “antecedent” and “consequent,”
while something imaged is really antecedent and so forth:
thus all this dissolves into appearance.
So then, what common element remains behind
as the condition for the whole exchange?
Obviously only the through-one-another that
initially holds together every inference
however it might have been grasped,
and which, as through-one-another,
leaves the consequence relation exactly
as free in general as it has appeared to be.

[Let me say this in a preliminary way
(there is not time today to explain the
deeper view which is possible here and
which I will go into in the next hour).
The focal point = the concept of
a pure enduring through-one-another
in living appearance:

A | E /I B-A A-B \R a-B B-a

a—act and consequence either ideally, or really.
I say either/or, it always remains act,
the concept proceeds from it alive,
but not finished and complete;
whomever wants this must do it.

On the other side, the concept projects
the one eternally self-same light as intuition,
from which follows (and this is its absolute essence):
what stands beneath it are parts of its externalization
and thus are further modifications not of the light
but of its appearance in the living concept.

1. Only through life to the concept
and only through the concept to life
and the appearance of the light in itself:
but its first modification, never as pure
but in one or another variant.

2. Creation of the science of knowing
in its possible modifications.]
