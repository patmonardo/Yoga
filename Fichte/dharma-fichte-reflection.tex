Sixteenth Lecture
Tuesday, May 15, 1804
Honorable Guests:

The fundamental principle presented now:
“Being is entirely a self-enclosed singularity
of immediately living being
that can never get outside itself”
is in part immediately clear in itself,
and in part it has been shown in the discussion
that it is clear to this assembly in particular.
Hence, we do not need to linger with it any further.

I said that it contains and completes
what one could present as the first part
of the science of knowing:
the pure theory of truth or reason.
We proceed now to the second part;
in order to deduce from the first part,
as necessary and true appearances,
everything which up to now we have
let go as merely empirical
and not intrinsically valid.
In advance of this undertaking,
I must remind you of only one thing.
Resolving this task in absolute oneness
of principle is not without difficulties;
especially since, according to a remark about method
I made at the end last time
(on the occasion of a general review),
this task is entirely new
and has not even arisen in the earlier
presentations of the science of knowing.
For that reason, it happens that this resolution
cannot remain without some complications.

However, in order to be completely clear with you about this point,
I will employ here the method that I have generally used before
of giving you an initial factical acquaintance with the terms
which come next so as to prepare you adequately
for their subsequent combination and connection.
This preparation is the next purpose of today's lecture.

1. In the insight we had produced into inner being,
we began, after fully abstracting from that objectification,
about which we already know that it intrinsically lacks validity,
from this being's construction,
to which we expressly challenged ourselves.
(You see that I revert to this,
partly as it has always been done up to now,
and partly because thereby some kind of
idealistic outlook enters in again.
I refrain from giving an account of this here.)
Now mark this well, since this point can bring you great clarity,
I will not myself reason here as all previous idealisms have reasoned:
“consequently being depends on its being constructed,
and this is its principle”;
because this claim could have truth and meaning,
but only in relation to being's factical existence
in the form of external, objectifying existence,
which existence then [is] absolutely presupposed
and so, according to our basic maxim,
the projection through a gap would not be abstracted from.
This factical existence in general is to be put into question
in its basic principle and deduced for the first time;
but, trusting in the truth of the insight's content
and so in our principle,
we will thus conclude entirely realistically:
if being cannot ever get outside of itself
and nothing can be apart from it,
then it must be being itself
which thus constructs itself,
to the extent that this construction is to occur.
Or, as is completely synonymous:
We certainly are the agents who carry out this construction,
but we do it insofar as we are being itself,
as has been seen, and we coincide with it;
but by no means as a “we” which
is free and independent from being,
as could possibly seem to be the case,
and as it actually appears to be,
if we give ourselves over to appearance.
In short, if being is constructed,
as in fact it seems to us to be,
then it is constructed entirely through itself.
The basis for this construction,
as is immediately apparent and understandable to us,
can not be located external to being,
but only within itself as being, entirely and absolutely;
and indeed absolutely and necessarily
apart from all contingency.
At this point notice

a) I have said “If being is constructed,”
expressing myself hypothetically,
thereby perhaps reserving a future division
of the statement into a true part and a false part.
So if someone were to insist that it was actually constructed,
you might wonder how on the present standpoint
such a one might conduct the proof?
I know of no other way than by means of his consciousness.
However, we have already given up on [accepting]
such a proof as valid by itself;
but here for the first time [the question of]
to what extent and in what sense
consciousness and its statements
can be [accepted as] valid is to be decided.
In particular, we must decide on the extent
to which consciousness suffices
in the highest things it asserts,
of which the fact that being is constructed
can serve here as an example.
Therefore, we should not reach ahead
anticipating the results of the investigation,
which we will make possible only by means of
the hypothetical assertion.

b) To be sure, it has already become immediately clear
in our earlier investigation that mere being is
of itself, from itself, through itself immediately an esse,
that it therefore constructs itself,
and that it is only in this self-construction.
This comprised the whole content of our insight.
But the self-construction that we talk about here,
which we present only hypothetically as
a declaration of pure consciousness,
and which we append to being in itself
only mediately as an inference,
this is, as I ask you to grasp immediately,
something entirely different, merely idealistic, imaginal.
This is the only way I can describe it in words.
In contrast, the first one alone would be real,
clearly having the predicate “real”
only by contrast with the other,
and thus negating the absoluteness of
the previous insight with this predicate,
which is understandable only relationally
and through its opposite.
Now the task is just to find to what extent,
I say not so much being's ideal or real self-construction,
as rather the analytic/synthetic principle,
which grounds it, is to be [accepted as] valid.
This question of validity can only be resolved
by deducing the principle genetically.
Therefore, we take care not to anticipate,
and we grant the entire distinction
only hypothetical validity.

2. Let us go back.
If being is constructed ideally, as we assume,
then this happens directly as
a result of its own immanent essence.
Be sure not to overlook [the fact] that we have thereby
actually won something new and great.
That is, the ideal is posited in this absolute insight
organically and absolutely in essential being itself,
completely, without any real hiatus in essence,
and so without any disjunction in essence.
This insight is also genetic,
positing an absolute origin as unconditionally necessary
on the condition that it be the ground and be assumed.
Now this insight brings along an absolute that,
but by no means a how; we cannot see how
the absolute essence ideally constructs itself,
nor can the inner ground of this construction be constructed further.
We must not be put off by this,
since only thereby does
this insight secure itself as the absolute,
beyond which there is no other,
and this construction as the absolute one,
beyond which no other can be placed.
To be sure, it must come down to such
an absolute insight and construction,
and it is clear that, only at this point,
with an insight and construction proceeding
directly from essence, could we arrive there.
The gap, which as a result of
the absolute insight is in essence nothing at all,
exists only in respect of the We;
and, indeed, in case the essence of consciousness,
properly so-called, is to consist just in this,
no longer in the absolute and pure genesis
but rather in the genesis of the genesis, as it appears here,
then, if this We (or this regeneration of the absolute genesis)
were to be deduced, it would be in consciousness
that it [the gap] could well remain.
[We can then very easily see] that here, quite probably,
we have untied in passing the genuine knot at its root;
and that the new difficulty,
which has not concealed itself,
has fallen further down,
where it can let itself
be easily resolved by closer consideration
of the basic point already discovered.
In the meantime, since the point has
not yet been put as clearly as possible,
we will continue our exposition without staying here longer.

3. We have now grasped this.
Following our consistent method,
let us make this insight itself genetic.
Under what condition did it arise?
Evidently this, that an ideal
self-construction of being be assumed,
at least hypothetically.
“It is assumed,” obviously means and
can be explained ipso facto like this:
it is projected absolutely
into the form of outer existence,
in a provisional way,
without any ground or principle
for this projection,
thus through an irrational gap.
Now, a major portion of our task is
to demonstrate the genetic principle
for this irrational gap,
which so far we have presented only factically,
whose validity we have denied,
but without our being able to dispense with it.

(Observe. A philosophical lecture can frequently count on
the unnoticed assistance of the understanding,
without always providing the distinguishing grounds
for the distinctions that it makes;
the fact usually explains its true meaning
through itself and its results.
However, in such cases one always counts on
a happy accident that is just as likely not to occur.
It is always more exact not to leave
any distinguishing grounds unexplained;
and especially we should not allow ourselves
to be led astray by the fact that often, and on many subjects,
the explanation makes more obscure what was clearer
with the unnoticed assistance of the understanding,
because it should not be so, and satisfying ourselves
with understanding's unnoticed aide is not
the genuinely philosophical disposition.
In the hour before last, we had looked at the case of
a distinction between two ways of thinking about the in-itself,
whose distinguishing ground I specifically stated,
although the distinction might have been clear enough
simply as a matter of fact.
The present case is similar.
The principle for the irrational gap as such,
for the absolute absence of principle, as such,
should be demonstrated.
Obviously not insofar as it is an absence of principle,
because then it would negate and destroy itself,
a very different thing from it being provided with a principle.
So, in what respect is it to be and in what respect not?
Let us now make the meaning clear.
Being's ideal self-construction is projected through an absolute gap,
and is thereby made into an absolutely factical and external existence.
Now this existence, as absolute existence, can have no higher
principle at all in the sphere of existence,
and in this sense is precisely lacking a principle.
Its “principle” in this unprincipledness is just the projection itself.
Hence too it is not claimed, and cannot be claimed,
that being in itself constructs itself ideally,
but rather only that it is projected as constructing itself so;
this is important and breaks the doubt
aroused by the first remark at 1. above.
Therefore, nothing remains
[once this factical being has been annulled as absolute
by the demonstration that the projection is its principle]
except the projection itself, and this as an act,
as everyone is requested to become aware.
To say that a principle must be provided for it means therefore
that a principle must be provided for it solely as an act in general,
and as this act that in itself posits something unprincipled.)

What could this principle be?
The absolute insight, which forces itself on us,
that the ideal self-construction must
itself be grounded in absolute essence,
is conditioned by the presupposition of
this ideal self-construction without any ground,
and thus by this projection
we ourselves have made to complete
the science of knowing.
And so, the principle has been found in
what is conditioned by it,
and the newer, higher insight
that is thereby created can be
encompassed in the following sentence:
If the absolute insight is to arise, that, etc.,
then such an ideal self-construction
must be posited entirely factically.
The explanation through immediate insight is
conditioned by the absolutely factical
presupposition of what is to be explained.

4. Now do not forget that everything here
remains only hypothetical.
If it should be seen into, then must ____, etc.
Should the consequent be posited
absolutely and categorically?
Undoubtedly, if the antecedent is,
and without doubt not, if the antecedent is not;
because the latter has no principle except the former.
But if the first should be posited absolutely,
then it is not apparent as absolute,
because it has been posited as absolute hypothetically.
As I add now only to arouse attention,
in this hypothetical “should” as our highest point so far,
everything comes together whose derivation is now our task:
the ideal construction of being as a self-construction,
as well as the projection through a gap.
Just so, it is clear that this hypothetical quality
of the “should” must remain as it has been presented.
However, it is equally clear that
something categorical must arise too,
since otherwise our science would be baseless
and without principle through its whole range
as well as in its starting point.
However, this categorical quality
must now just manifest itself hypothetically
in the “should” qua “should,”
so that henceforth the chief principle
of the process of appearance
(and, if it were believed, of what appears)
should consist in this:
that the absolutely categorical “should”
appear as hypothetical in relation to the insight,
the true and the certain.
That is, [it should appear]
as able to be or not to be,
as able to be thus or be otherwise.

5. In order to prepare the way for this point,
to the extent time allows,
I urge you to reflect maturely with me
on the essence of the “should.”
Obviously, an inner self-construction is
expressed in the “should”:
an inner, absolute, pure, qualitative
self-making and resting-on-itself.
One can assist the intuition of this truth,
which in any case also makes itself.
It is, I say, an “inner self-construction,”
completely as such:
nothing else supports the hypothetical “should,”
except its inner postulation entirely by itself
and without any other ground;
because if it had some other ground,
it would no longer be a hypothetical “should,”
but rather a categorical “must.”
“Inner postulation entirely by itself”
I have said; hence a creation from nothing,
producing itself entirely as such.
A “resting-on-itself” I have said, because
(letting myself take it up in a sensory form,
which harms nothing here)
it falls back into nothing
without this continuing pursuit of
inward, living postulation
and creation from nothing.
Hence it is the self-creator of its own being
and the self-support of its duration.

This, as we have described it,
is simply then the “should,”
and, according to the presupposition,
it is grasped intuitively by all of you in this way.
Therefore, with all its initially apparent hypothetical character,
just for that very reason there is something
categorical and absolute here,
the absolute determinateness of its essence.
Before we now show further what follows from this,
let me add today two further comments in conclusion.

a. The “should” bears every criterion of
the intrinsic being intuited in the basic principle:
[it is] an inner, living from itself, through itself, in itself,
creating and bearing itself, pure I, and so forth;
and [it is] certainly organized and
coherent internally, entirely as such.
As regards the latter, in case it needs further explanation
after the clarity with which it must have
already presented itself previously in intuition:
we then always objectified the fundamental principle's
“inner being” factically,
although this objectivity was not valid.
We also have previously objectified the “should.”
Finally, however, we have been lost in it factically,
in its inner description and insight,
and now for the first time
we free ourselves from it,
and it from us, in reflecting about it,
a process which, according to our previous method,
can be explained as a projection through a gap
taking as its principle the “should” itself.
Accordingly, this “should”
(purely and simply in its oneness,
and without any supplement)
can easily be being's immediate ideal self-construction,
that is in no way to be further reconstructed,
but rather that provides the subject matter
directly in the construction itself.
On the other hand, being's previous,
hypothetically posited,
construction from the “should” has
finally found a principle in this “should,”
just as inner being finally has too
for its projection through a gap,
which we had proposed accordingly.
[This principle] in itself is
construction and subject matter,
ideality and reality,
and it cannot be one without the other.
This duality may reside in our objectifying
consideration of the science of knowing,
which therefore abandons its claim to
intrinsic validity.

b. This “should” has constantly, but without notice,
played the principal role in all our previous investigations.
“Should it come to this or that,
to a realization of the through, etc., then must ...”;
our insights have always gone along in this fashion.
Therefore, no wonder that after
letting go of everything else,
what remains for us is only the thing
that is truly first in all these cases.

Seventeenth Lecture
Wednesday, May 16, 1804
Honored Guests:

I have stated in the last hour how
the part of our science on which we are now working
might be different from the part completed first,
and what our following lectures intend:
namely to introduce for the first time
the materials for resolving our second task,
and to make you familiar with them.
At the same time, I admitted that the next lectures
might not be without difficulty and confusion.
It is easier to take in and to grasp
that which rests in reason completely and simply as oneness,
as did the earlier fundamental principle,
since only abstraction is needed for this task.
“Easier,” I say, than to trace what
in itself and originally is never a oneness
back to a oneness in order to produce
a completely new and unheard of concept in oneself,
for which other arts are undoubtedly required.
Now, we first lay out multiplicity in an order
in which is most convenient to us for insight.
These terms can first be correctly
ordered and understood on the basis of their principle,
which itself is first to be discovered from them.
At this point in the course of the external lectures,
there is an unavoidable circle that can be annulled
only by its own completion.
It is possible, and indeed expected, however,
for one to grasp the process
[that, to be sure, has its proper order]
and terms, and to give them what clarity
they can have under the circumstances.
I have said that a new, heretofore entirely unknown,
principle must be presented;
and also simultaneously I would add this remark: that
(thinking of the previous division
of the science of knowing into two parts)
we are concerned not just with presenting the second part,
but also with uniting the latter with the first part.

The course of the previous sessions was this:
we constructed the pure being, which we had grasped,
as an entirely self-enclosed singularity.
In this way, I assumed, we could become
immediately conscious of ourselves,
and, as required, we were actually conscious of ourselves.
This therefore was a completely simple,
factically objectifying projection of
an act that we ascribe to ourselves
as likewise independently existing entities;
and in this manner we could have been tempted
to deduce being itself from this act of
construction in a one-sidedly idealistic fashion.
However, we wisely refrained from doing this,
well understanding that by this
we would return whence we had first arisen
and consequently would not have advanced.
But we proceeded in this manner,
and necessarily had to do so,
if we wanted to come to something
more than the one being,
for example to the latter's way of appearing.
“As concerns the truth in itself of this construction,
this can appeal to nothing else besides
the bare assertion of consciousness.”
We cannot now discard this statement unconditionally,
as just previously we unconditionally rejected it,
thanks to our present, entirely altered, aim;
because previously we sought pure being in itself,
and it has been shown that consciousness is
entirely insufficient for this purpose.
Now we no longer seek this pure being in itself,
since we already have it
and so our search for it is over.
Instead, we want to grasp it in its primordial appearance;
and so consciousness, and here in particular the construction,
could be the first term of this appearance for us to grasp.
We cannot allow this term to be unconditionally valid
any more than before;
since the extent to which and the conditions under which
it is valid are exactly the issue.
Therefore, we must present this claim hypothetically,
without prejudging future inquiry:
if, and to the extent that, a construction of this kind is actual,
that is, takes part in being and not merely seems to be,
but has being actually appearing in it, then ____ .
Through this then, we are asked to point out
in immediate manifestness the condition for
the real and true being of such a construction,
in case and to the extent that being could come to it.
This condition has now been found and has become
evident without any difficulty:
If this construction, which appears to us,
is actually and in fact connected with true being in reason,
but in no way connected with factical existence in consciousness,
which counts for nothing until it is better grounded
(this detail is not to be overlooked)
if the construction which appears to us is in this sense,
then it is not in any way based in the vain “I”
of consciousness that emptily objectifies being;
rather it is grounded in being itself.
For being is one, and where it is, it is whole; in being qua being;
therefore entirely and absolutely necessary.

(On the condition that you do not allow yourselves to be distracted,
let me add here an additional remark that can spread much illumination.
Posit pure immanent being as the absolute, substance, God,
as indeed it really is, and posit appearance,
that is grasped here in its highest point
as the absolute's internal genetic construction,
as the revelation and manifestation of God,
then the latter is understood as absolutely essential
and grounded in the essence of the absolute itself.
I assert that this insight into absolute inward necessity
is a distinguishing mark of the science of knowing
as against all other systems.
I cannot emphasize it enough,
because the absolute absence of insight
strives against it with all its might,
since freedom is always the last thing
[this darkness] will surrender.
If it cannot save [freedom] for itself,
then at least it tries to secure it in God.
In everyone without exception,
an absolute contingency exists
next to absolute substance.
Here something is seen from the beginning
as absolutely necessary in reason and in itself,
which afterwards will appear
not in reason and not in itself
but as contingent in another connection
that still is to be worked out.
Only on this condition can the science of knowing
hope to deduce the phenomenon in a genuine and grounded manner,
and not merely as a pretense;
because a genuine derivation must have a reliable principle.
Otherwise, as has often actually happened,
one deduces from the intrinsically contingent
something else which is also contingent,
and obtains other contingent things from these,
which themselves stand firm only on condition
of the reliability of the previous thing,
whose reliability likewise depends on the first.
As if a good, proper, and reliable
standpoint could arise when one had two terms,
neither of which could stand by itself,
each relying reciprocally on the other.)

This remark as well:
it is evident that in our present investigation
it still seems as if, as I freely admitted at the outset,
this investigation is still searching
for its principle but has not yet got it,
something I have described as erroneous,
since its first term [the construction of inward being]
still remains hypothetical in connection to that
about which alone we are inquiring, true being in reason.
So, the thing which can first be ascertained under this condition,
being's necessary self-construction,
can itself not be otherwise than hypothetical to the same degree.
Therefore, from here on you should direct your attention
to the question whether and when
a self-sustaining principle emerges.

If there is a construction of being,
then it is grounded absolutely in being itself;
we grasped this directly and reflected further on the
insight and its inner, law-governed form.
Then it was immediately clear that
we began with the presupposition of inner being's construction,
which we incorrectly attributed to the “I” of consciousness,
but we have already learned better than this
and let go of the attribution.
But this much remains indubitable,
that being's construction is projected as an absolute fact.
Have we now brought this implicitly simple, factical projection
into connection with other terms by the use we have made of it?
Evidently so; we saw that if such a construction exists,
then it must be grounded in being.
Now, we have undertaken this entire speculative venture freely;
the resulting insight
(which might very well not have been engendered)
is conditioned by our procedure
(which we might very well have omitted),
and therefore it is in no way a firm standpoint.
All the same, in order to achieve such a firm standpoint
we applied a procedure that, to the extent that it needed to,
proved its legitimacy by its bare possibility.
We said: assume that the insight, engendered by us,
is to arise, then you will see that under these conditions
the projection of factical being, previously only possible,
becomes necessary.

In this way, we would first of all have made
good progress beyond everything achieved so far
toward a proof that, although to be sure
we do not feel firm ground beneath us,
we might be on a good path.
The absolute projection through a gap,
and thereby the form of outer existence,
that could not be understood conceptually
in all our previous investigations,
is explained as necessary
under the assumption that
a higher term (the insight) should be,
an assumption that itself was previously hypothetical.
Thus, the hypothetical status departs from the lower term,
though only by transferring itself to the higher;
but at least with this it is simplified
and its proper location is revealed to us,
where we can hope to grasp it at the root.

After what I have said about the necessity
of a self-supporting principle for this investigation as well,
we will next eliminate this hypothetical status completely;
and here the most secure means is to look it straight in the eye.
It is entirely compressed into the hypothetical “should”;
this is sufficient by itself for our next purpose;
therefore, we let go of the site
where this “should” appears, insight, etc.
Quite apart from our current procedure,
it could be obvious from the entire previous investigation
that one now needs to keep this “should”
as one of the deepest foundation points of all appearance,
as I will observe in passing.
All our preceding investigations and engendered
insights have started with the hypothetical “should”
and have proceeded from it as a terminus a quo:
“If there is really to be a “through,” then there must ____”;
“Should the achieved insight arise, then there must ____”: idealism;
“should this life be life in itself, then there must ____”: realism;
all the way up to the highest relation:
“Should an in-itself be comprehensible,
then a not-in-itself must be thought” and so on.

This “should” loses itself entirely
only in the insight into pure being
and into the way in which we evaporate into it,
so that an absolute categorical character enters,
without any hypothetical presupposition.
As soon as we reflect again on this insight,
the process which yielded the historical origin
of our second part and our entire present investigation,
it reinstates itself with a “should,”
thus as something contingent,
seeking the basic condition for this contingent quality,
a necessary self-construction of being.
Now, so far in the ascent we have clung to
the content of the generated insight without reflecting
on the hypothetical form in which, as a whole, it appeared.
This was entirely correct because we wanted to arrive
at the original content of the truth as such.
(Here in passing the question that some have asked me concerning
the true grounds for our first part's preference for realism and
for the maxim that ruled there always to orient ourselves realistically
answers itself decisively and fundamentally.)
But now as we descend we have to hold on
to just this neglected “should,”
which indeed provides the enduring
inner soul of all the idealisms,
which consistently excluded themselves during the ascent,
and which were struck down by an opposed higher term
only in respect to their content;
but still persist in their form, as we see.
Now this form cannot be disturbed
directly by the original content,
since everything that the latter can attain
has already been achieved in the ascent.
Rather, it must be explained
and justified inwardly on its own terms.
It must refute its own ungrounded claims,
to the extent that they are ungrounded;
roughly just the way we refuted
the content's highest idealism,
that at first presented itself as realism,
by means of the law that it itself presented,
and revealed it as idealism.

In a word, and in order to lead you
even deeper into the systematic connection between
a. the term that earlier was the highest in appearance,
the distinguishing and the unification of
the in-itself and not-in-itself
in the whole five-fold synthesis, and
b. absolute inner being,
as the absolutely realistic element,
the “should” enters here as a new middle term,
in which the self-differentiating
and likewise synthetic relation
of the two indicated relational terms must show itself.
To find this is the proper content of our task:
to find it as a firm principle is its form.

First, however, the connection to inward being.
The form of being is categoricalness.
Therefore, something categorical must be found
in the “should” itself,
however hypothetical it might appear.
In order to uncover this,
I have demanded that the inner essence
of a hypothetical “should” be carefully considered
(following our consistent method of raising into clarity
something that was at first dimly projected).
We have already done this last time;
because of the subject's importance,
I will repeat the entire operation today.

If you say forcefully and deliberately:
“should so and so be,” then it is clear that
thereby an inner assumption is expressed,
without any foundation, simply of itself and from itself,
thus an inwardly pure creation,
and to be sure standing there completely pure entirely as such,
because the “should,” if it is taken only as purely hypothetical
[as is required here and without achieving which
the required insight will not arise]
expresses complete external groundlessness,
simple internal self-grounding, and nothing else.
Further, (in this way, I tried to grasp the same thing
and make it clear from the other side):
the absolute assumption is expressed in the should,
an assumption that is unconditionally allowed to drop,
just as it is unconditionally presupposed.
Should it [and with it probably the entire
“If ... should, then ____ must”
that depends on it] not drop away,
(with which dropping away all knowledge and insight
probably drop away as well)
then it must hold and sustain itself.
As surely as we have now seen into this,
just that surely has the should been illuminated
for us as an absolute that holds and sustains itself
out of itself, of itself, and through itself as such,
on the condition that it exists.
This, I say, is a “should”;
and were it not precisely so,
then it would not be a “should”;
therefore we have a categorical insight
into the unchangeable, unalterable nature of the “should,”
an insight in which we can completely abstract
from the outward existence of such a “should.”
“Can abstract,” I say since, with adequate deliberation,
I refrain from drawing a conclusion here
which easily presents itself,
but which is not yet sufficiently ripe, given the context.
To the extent that our task simply consisted
in discovering something categorical in the “should,”
it has been fulfilled by what has happened.

In explaining the should, I have not warned you
about the illusion that it is we who assume there
what is hypothetical and who hold and sustain it;
since the rule is to leave this “we”
of mere consciousness entirely out of action until it is deduced,
and being able to do so is the art without which
no entry into the domain of the science of knowing is possible.
If, in the meantime, this “I” has forced itself on anyone,
then let it immediately remove itself at this point.
Namely, whether or not you have created and carried the assumption,
it is still always completely clear that you have a “should”
only on this condition of self-creation and carrying forward.
Therefore, even if you are the creator,
the “should ” always contains the rule and law
of proceeding in that manner,
otherwise it is not a “should,”
and we have not wished to say any more than that here;
abstracting completely from the question that you raised
and that we will work out in another place.

And now, in conclusion, a very sharp distinction,
that will become decisive in what follows,
and that cannot be made clear too soon.
The strong similarity between inner being
as something self-enclosed and self-sufficient
in-itself, of-itself, and through-itself,
and the should as just the same
has already been pointed out earlier.
There is nonetheless a distinction
between the two that I have named
and made dimly recognizable in the stated formula:
“the should” is something in-itself, etc., as such.
I urge you now to clarify this distinction
for yourselves along with me.
Being was constructed as something absolute in itself, etc.
I ask: should there now exist,
or is their actually in our insight,
if it is of the right kind,
another persisting being or substantive,
besides this absolute, self-constructing esse?
Not at all.
Instead both merge into each other
and into the pure self-enclosed singularity,
and the doubled repetition is entirely
superfluous, insufficient, and neglected.
This is not at all the case with the “should,”
if you will look into it quite acutely.
The latter stands out as a fixed, substantial
middle point and bearer of
absolute self-production and continuation.
The latter is not just immediate,
as was the case with being,
but rather only mediate through
presupposing and positing a “should”;
in brief on the assumption that
the “should” itself again should be,
and thus should be seen through its own doubling.
Here there is not, as there was before,
an immediate rational insight,
but rather only a mediate one,
conditioned again by a higher
projection through a gap,
precisely of the “should”;
just in the way we have actually proceeded.
We have wished to indicate this relation
by the added phrase “as such,”
itself in objective, factical oneness of essence.

To what further things this new discovery
might lead must emerge on its own.
Before hand, this much arises in regard to method:
that, just as a projection through a gap
(the projection of being's construction)
is deduced as necessary from the fact
that a particular insight “should be”,
another projection, just that of the “should” itself,
presents itself [on the one hand] as
a condition for this insight
and on the other side again as conditioned by it.
We now need to venture further into this;
that therefore our present investigation,
just like the previous one,
advances upward only in this
precisely delineated circle,
because it is still looking
for the latter's principle.

Eighteenth Lecture
Thursday, May 17, 1804
Honored Guests:

[Here is] what has been presented so far:
we presuppose a construction of being.
On the principle that nothing can be except being,
the construction is seen as arising necessarily from being,
of course, with the same certainty it has generally;
supposing therefore “If it should be ..., then ____ must be.”
But the “should” is something in itself, of itself and from itself "as such.”
This, and in particular the “as” most recently added,
is now firmly fixed for you as another new middle point
and bearer for the self-producing and self-sustaining “should.”

Today I add another basic observation
concerning the true inner spirit of
the reasoning processes presented so far,
and [we] will then work on
our remaining task from another angle.

1. As concerns the first,
our higher insight from the standpoint of
the hypothetical “should” took the following form:
should an insight into this or that occur
(in this case in particular the insight that
the ideal self-construction is to be grounded in being itself),
then ____ must.
“Since you now,” I would say,
“actually provide the content of this insight,
which, according to your account hasn't yet occurred
but whose condition you are seeking,
you already without doubt have it
in sight and in your concept;
you are constructing it really and in fact”;
(as is the case here with being's ideal self-construction).

This remark permeates all consciousness
and can be illustrated in every case.
I cannot reflect how and according to what law anything
(e.g., a body in space, space, a line, etc.)
is conceived or constructed,
unless I have already grasped it apart from all reflection
and according to a universal law.
In the present case, the law is constructed in
one of the most general cases,
which contains others within itself.
“Therefore,” I continue, “you seek either
that which you already have,
or you seek the same vision and the same concept
(the same in regard to contents)
only in another qualitative form.
That the latter is the case becomes clear
through a more exact consideration of the proposition laid down.
The content of your vision, which, as the content of
mere seeing, is separated and existing for itself,
should be brought into connection with something else in seeing,
both as its condition and as conditioned by it,
initially through what you call insight.
Thus, in order to state the true result of
your desire definitely and exactly:
just to arrive at your demand,
a seeing that is already completely determinate in and of itself
(and that you must presuppose as determinate)
should be further qualitatively determined
in this persistent, objective determinateness as seeing,
since the objective determinateness remains.
Therefore, to put it briefly,
you demand a new genesis in the seeing
that has already been presupposed as existing
and as remaining the same objectively.

A new inner genesis of seeing,
as formal seeing itself,
without any alteration in the [seen] content;
(what we have already called objectivity).
Now, the material of this formal genesis,
its result, is itself again a genesis:
the constant content should be brought
into a genetic relation with another term,
that creates, and is again created by, it;
thus the entire familiar “through,”
or the relation in its synthetic five-foldness,
should come in.
As things stand, it can well be that
this external material genesis
with and out of the content,
which is nonetheless not changed in its inner nature,
is itself grounded in mere seeing's formal genesis,
and resides not so much in the subject matter as in the altered eye,
through which the entire present multiplicity is traced
back to the oneness of the same principle,
of the formal further determination.
This formal further determination,
or new genesis, is called for through a “should,”
which has itself been recognized as a genesis
in its inner nature unconditionally as such.
And so this genesis could have its ground
in the “should” itself as the relation and
five-fold synthesis within the formal genesis,
so that the “should” is the basic principle for everything,
as we have previously already taken it to be.
In brief, the spirit of our whole reasoning process,
conducted since the beginning of the second part,
is the demand for an inner genesis
in the seeing presupposed for genesis itself.
This process adds nothing to the seeing in its true meaning,
and so it must be inoperative in relation to this meaning,
just as we have always wished.
Likewise, this very inner formal genesis,
as wholly concerning only the way of viewing,
may be the principle of absolute idealism = of appearance;
and we ourselves have entered into a new and higher idealism
through the principle presupposed in
our entire reasoning process:
that being is constructed ideally,
separately from its real self-construction.

That just this insight,
now characteristically distinguished
from the presupposed original seeing,
presents itself alone as certain,
compared to which the original seeing is
to be only hypothetical in relation to its content
(it is clear on immediate reflection that the matter stands so,
and our certainty appears finished and closed);
this circumstance probably lies in the partiality of idealism itself,
which here gives testimony for itself, knowing nothing else.
Now we have to investigate this claim for the first time.

2. A recognized basic rule:
nothing can be accomplished in any way against an idealism
except from the standpoint of realism.
Therefore, as soon as our reasoning
has been traced back to its spiritual oneness
and understood to be idealism,
we cannot stand by it any longer
without being driven around in circles.
We must turn instead to the corresponding realism
and consider this more deeply in its origins.

a. As we remember, we entered this realism
after the last consideration of the in-itself,
and of the insight that, in our knowing,
this in-itself is relation and multiplicity;
therefore, that it is not absolute oneness,
thinkable without any composition or division,
but is rather, as we said:
a oneness of understanding.
We discarded this knowing entirely
and yet knowing still remained,
which thus was absolute inner oneness,
without any combination or separation:
oneness in itself.
We also refrain from saying for
example that we have produced it in this oneness;
since we truly would not have wished that
something should remain behind
after abstracting from everything,
or [have wished] to encompass what remains with our will,
had we willed or been able to will this,
so that it would indeed have been left over for us:
instead it was just unconditionally left over:
oneness of itself.
Everything depends on this last point;
it is what has been overlooked in every system
and what becomes clear only to the deepest deliberation.
What we are naming the We, that is our freedom,
which is derived here for the first time
from the previously mentioned,
new formal genesis of the absolutely
presupposed seeing = reconstruction,
can only abstract from its own
creation of the act of reconstruction,
but it cannot creatively construct primordial reason;
although after complete abstraction
primordial reason enters without delay.
So then anyone who [in inseparable awareness of
the simultaneity of his completed abstraction
and the arrival of primordial reason,
and in the equally inseparable awareness
that he is the one freely abstracting]
immediately transfers his own
freedom to reason's emergence,
such a person deceives himself
and remains trapped in an idealism.
This final illusion is negated here
in immediate manifestness by means of deep reflection.
After abstraction from the highest oneness of understanding,
a knowing remains, just because it remains,
without any possible assistance from us,
pure light or pure reason in itself.

b. This pure reason is equally immediately
inner being and completely one with it.
Previously we called what remained
after all abstraction “inner being”;
here we have called it “pure light,” or “reason.”
But whatever we may wish to name it,
it is what remains unconditionally
by itself after all abstraction,
an entirely indivisible singularity;
and I would very much like to know
whether any disjunction can be made
in the presented concept,
and whether the insight that it is
a completely self-contained singularity
does not clearly show that,
whatever variation in the words used to name it,
one and the same nature could be meant.

c. Previously as well as now,
we have named it a real self-construction
in itself, of itself, and through itself,
and we could not describe it differently.
Now, abstracting completely from
the facticity of this description,
which to be sure can only be a reconstruction,
and through which we happen into the first named idealism,
[let us] reflect [instead] on its inner truth,
and [with this I ask for your complete attention]
on the surprising result that I intend to bring out.
I ask: does it not now depend entirely
on the pure thing itself
remaining after all abstraction
that it exists entirely from itself
whether you call it being or reason?
For example, is it arbitrarily posited
as existing on its own?
How could it be?
For this would be a genuine contradiction,
since in that case it would not be from itself
but would exist through an arbitrary act of positing.
If it is posited as something left over
after abstracting from everything outside itself,
then it is necessarily posited as of itself.
For if it were not of itself,
then it would be of another,
so that in its absolute positing
[in the original creation of its being]
it would not be possible to abstract from this other.
(That because of babble and thoughtlessness this other might not
be considered could be factically true and still should be explained;
it is not true in the one absolute, self-consuming oneness.)
Once again, it is posited absolutely,
creatively, as something of itself;
it is evident that this of itself is
actually manifested and is not just thought up;
so it is posited as existing absolutely
and remaining behind after abstraction from everything.
Hence it is clear that light, or reason,
or absolute being, which are all the same,
cannot posit itself as such without constructing itself, and vice versa:
that both coincide in their essence and are entirely one.
Notice here:

1. the insight that being must construct itself
unconditionally has arisen here
through the mere consideration
of its inner nature entirely immediately
and without any factical presupposition,
an insight that, according to idealism's pretensions,
should only be producible mediately
from the factical presupposition
that a constructive act is present.
By this means idealism is
first of all fully refuted,
insofar as it grounds itself
in the necessity of a presupposition
for a particular insight,
although merely a possible one,
since the insight has actually
been produced without the presupposition.
Idealism must therefore look
around for higher support,
if we are still to come to it.
Further, the proposition alluded
to in passing has therefore come up,
that this same insight is possible
in two different ways:
mediately, from presuppositions,
and completely immediately.
How would it be,
if the entire distinction that we have sought
between philosophical and common knowledge,
between the standpoint of the science of knowing
and that of ordinary knowing
(and in case within the latter
there should be degrees of mediatedness,
the distinction between the various standpoints
of this common knowing)
were to lie in just this distinction
between these differing ways.
Philosophical systems are always closest for us:
the presupposition that idealism wants
as the principle of mediate insight, is factical.
How would it be if, for example,
the proof of absolute being
from the factical existence of finite entities,
which is conducted in nearly every system,
and according to them in ordinary consciousness as well,
were just this idealistic path of mediate insight,
with which one remains satisfied,
for the lack of the immediate path.
In itself this is correct
and is applicable in its place
within the gradual process of cultivation,
in rising up to the highest;
but it generally fails the test against criticisms
that strive ahead to the highest!

2. The distinction between being's
real and ideal self-construction
that we made earlier,
and on which idealism built,
is now completely annulled.
Being, or reason, and light are one;
and this one cannot posit itself, or be,
without constructing itself;
this is therefore grounded in its nature,
and is entirely unitary, as is its nature.
Therefore, if we are to return
later to such a distinction,
then it must first be derived.

3. We saw that, in reason per se,
its self-positing and its self-construction
as “from itself,” etc., coalesce entirely into one.
And as certainly as we saw into this,
in this insight we were the oneness of reason itself.
Now a duality still remains here,
not however as in the oneness of understanding,
whose parts are to be integrated
[since parts within the oneness are
rather completely denied and negated here;
and the oneness does not understand itself through parts
but rather posits itself unconditionally and absolutely]
but rather as a means for achieving oneness.
Therefore, it may perhaps turn out that
a reconstruction is already present here,
one that would be posited backwards
toward the idealistic side by an absolute “should,”
and which we could not avoid merely factically,
even though its intrinsic validity is not admitted;
therefore that we stand at the precise place
from which our task could be completed.
How things may stand with this
I reserve for further investigation.

Now I add a supplementary remark,
with which I did not previously wish
to interrupt the course of the inquiry.
As the opportunity has arisen,
I have tested those recent philosophical systems,
which have made the greatest impression
in regards to their principles,
in order thereby to bring greater clarity
to the science of knowing;
thus Reinhold's system and thus Schelling's system.
Next to these, and perhaps even more than they,
Jacobi's system recommends itself,
because with great philosophical talent
it tries to jettison philosophy itself,
and thus it flatters the prevailing spiritual
indolence and denial toward philosophy.
The scene for testing this system's principles was just above.
It proceeds from the following principles:

a. We can only reconstruct what originally exists.

We ourselves have precisely presented
and precisely defined this claim,
which for Jacobi is almost a postulate:
the seeing, determined primordially in its content,
is formally genetic in relation to an unaltered content,
and therefore it is the insight into a connection;
and we ascribe this genesis to ourselves,
a genesis that is only reconstruction
in relation to the truly original content
and that is truly original
construction and creation from nothing
in relation to the terms added factically.
Regarding the last point,
the absolute creation of everything factical from the I,
he very clearly took this over from us;
and it is very plausible that he granted being to the factical,
to what is sensible outside the one rational being,
and thereby left us only reconstruction.

b. Philosophy should reveal and discover being in and of itself.

Correct, and exactly our purpose.
Through the persistent assertion of these two principles
this author has earned the age's great thanks,
and has favorably distinguished himself from
all of the philosophers who just reconstruct impartially,
or even just fool around with nature and reason.

c. Therefore, we cannot philosophize, and there can be no philosophy.

This latter claim, just as I have stated it,
is his true opinion, and must be his true opinion,
if he is to have any opinion at all.
For he contributes nothing by his usual addition:
philosophy as a whole.
Because, if there is no philosophy as a whole,
then there is no philosophy at all,
but rather only edifying remarks for every day of the year.
I grant him everything as it is presented,
only taking it more seriously than its original proponent does.
We, the we who can only reconstruct, cannot do philosophy:
equally there is no philosophy individually and personally;
instead philosophy must just be,
but this is possible only to the extent
that we perish, along with all reconstruction,
and pure reason emerges pure and alone;
since this latter in its purity is philosophy itself.
From the perspective of “we” or “I”
there is no philosophy;
there is one only [once one has gone] beyond the I.
Therefore, the question about the possibility of philosophy
depends on whether the I can perish
and reason can come purely to manifestation.
This author could demonstrate that
this must indeed be possible from his own words.
Because when he says:
we can only reconstruct, he achieves ipso facto
in that very moment something more than reconstruction,
and has at least drawn himself happily out
of the “We” of which we have spoken.
For if he could [do] this,
then for his whole lifetime he would enact,
but without speaking about it,
just as by his previous statement
he enacted elevating himself to
reconstructing the reconstruction.
Of, if we will free him from this,
he [can] tell us how he came to the universal statement
by which he prescribed an absolute law for his “We,”
and thereby pre-constructed the “We's” essence for them,
and did not merely reconstruct it.
In which case he would have to resign himself
to express himself like this:
“I and everyone I know,
as many as I can remember to the present
could only reconstruct;
whether perhaps tomorrow something else will happen,
we will have to see.”
Finally he will have to tell us whether
he understands this concept of “reconstruction”
without presupposing something original,
independent prior to all construction.
As surely as he understands himself,
he must become aware of such
a thing beyond all reconstruction.
Grasping this original something and reconstruction
as following from it as an absolutely essential law of the “We,”
just as we have articulated it here is
the task of a philosophical system,
which we have presented entirely
according to its sense,
but have only partially solved.

Nineteenth Lecture
Friday, May 18, 1804
Honored Guests:

Since today we finish the week,
I do not wish to let you go without
having equipped you with some definite result.
This resolution compels me for now to pass by
certain middle terms that still remain
for deeper consideration between
that with which I ended yesterday
and that which I will attach to it today,
in order to reserve them for the descent.

1. As an introduction to our essential business for today,
[here is] a clarifying remark that
should direct your subsequent attention
and that at the same time also briefly and concisely
repeats the first major part of yesterday's lecture!
I say: in all derivative knowing, or in appearance,
a pure absolute contradiction exists
between enactment and saying:
propositio facto contraria.
(Let me add here by the way,
as I thought previously on an appropriate occasion,
a thoroughgoing skepticism must base itself on just this
and give voice to this ineradicable
contradiction in mere consciousness.
The very simple refutation of all systems
that do not elevate themselves to pure reason,
their dismissal and the presentation of their insufficiency
even though their originator is not thereby improved,
is based on just the fact that one points out
the contradiction between what they assert in their principles
and what they actually do [in asserting them]:
as has been done with every system that we have tested so far,
and yesterday with Jacobi's as well.)

In the first half of yesterday's talk,
this contradiction showed up
in what we had identified so far as
the highest principle of appearance,
that is in the “should,”
immediately after we had conceived it
in its firm and completely determinate nature
as something from itself, etc., as such;
namely, a particular insight
(which in our case was this:
that being constructs itself)
is posited through the “should”
as not present but rather merely as possible,
and as possible only under a certain condition
that is still sought.
If we are even to arrive at the consideration of
its conditioned possibility,
this [condition] finally must be presupposed as
a seeing that is fixed in its content
and to that extent unchangeable,.
Hence, [two things] stand in
complete contradiction in this “should,”
its enactment [its true inner effect,
to presuppose a seeing that is unalterable in its contents]
and its saying [a different action on its part,
according to which the insight is supposed
to be not actual but only possible
under a condition yet to be added.]
I add only for the sake of recapitulation
that the true external nature of this “should” is
found as the demand for a further inner
and merely formal determination
of a presupposed seeing that is
unalterable in its content,
through which further determination
this presupposed seeing comes into
a genetic connection with another term
that is created purely by this further determination.
And I immediately formulate the following conclusion:
absolute reason is distinguished from this relative knowing
by the fact that, in the case of absolute reason,
what exists, or what it does, is expressly said in it;
and that it does what is expressed in it
in absolute qualitative sameness.

2. In the second part of yesterday's investigation,
we tried to represent pure reason in ourselves.
I noted at the end of this presentation that,
because of the duality that to be sure was annulled intellectually
but that remained factically inextinguishable in you,
it became evident that pure reason could not
display itself immediately in you
and could rather only be reconstructed.
The same qualitative determination of
a presupposed seeing that is unalterable in its contents,
a determination pointed out within the “should,”
we also called reconstruction;
therefore, the contradiction between saying and doing
just discovered in all derivative knowing is
contained within reconstruction itself,
a fact that can itself be made clear immediately.
To be sure, reconstruction explicitly puts
itself forward as reconstruction
and therefore in its own concept
quite properly posits the point of origin,
and in that there is no contradiction.
But since it leaves the content unchanged
and can actually create nothing new
without completely negating the relation
between itself and the absolute,
its construction is therefore groundless
and the fact itself contradicts the postulate
of the absolute necessity in
the pure, positive in-itself.

I should now immediately climb past this
contradiction [we have] discovered and relieve it
(that is, past the groundlessness of
the concept of a reconstruction).
However, in accordance with my initially stated resolve,
I am retaining it [in order] to annul it mediately in the descent;
and so [let's turn] directly to yesterday's reasoning
to indicate the location of absolute reconstruction,
and to remove this circumstance.

We brought the already established absolute insight
to life in this way:

a. it arose for us after
we abstracted completely from all relations,
and it remained behind as a oneness,
not just because we wished it,
but simply by itself.
Pure light, or reason.

b. Previously we named it inner being,
here light or reason;
but it is clear that no distinction whatsoever occurs
in the one singularity that remains behind by itself as one,
and that consequently both designations are only
two different names for the one
that is grasped as completely
indivisible and inseparable.

c. We saw into this “one,”
and still see into it now as
something from-itself,
etc. = [as] self constructing.

I asked: should not then this from-itself reside completely
in its nature as absolute truth?
And I discussed it even further in the following consideration:
out of its self-positing as this,
self-construction follows, and vice versa;
because if it is posited as this,
as remaining after abstraction from everything else,
then it is posited as remaining and persisting because of itself;
since if it were not because-of-itself,
it would be because-of-another,
from which it would not then be possible to abstract
in its true original creation,
or which could not be absent for this creation.
Conversely, if it is a true, actual, energetic from-itself,
then it is not from another;
since then it would not be truly from itself.
Therefore, it is necessary to posit it, as it has been posited.
But let us take a keener look at this reasoning itself
and the procedure within it.
(And I remind you that this is the most difficult
and significant thing that has so far come before us.)
First of all, without exception in our whole argument process
and in the entire conduct of our lectures up to now,
the absolute has been treated as what is left over
after abstraction from everything manifold;
and if equally we have expressed specifically enough
the absolute from-itself and pure oneness in-itself,
then with these words which we have added as clarifications,
we have surely again made use of this same relation;
as more certain proof that even we ourselves,
the scientists of knowing and what we actually did and pursued,
found ourselves in the previously uncovered contradiction
between saying (of the from-itself)
and doing (explaining by means of the not from-itself).
Thus the first premise of our proof here reads:
“If it is posited as this,
as left over after abstraction
from everything else ...”;
which is a sure demonstration of reconstruction.
Second, in the center of our entire proof
we have absolutely presupposed both genesis
and the absolute validity of the Law of Principles.
The center of the proof was,
“If it is not from another, then it is from itself;
and if it is not from itself, then it is from another.”
If someone now were to say to us:
“Quite right: one of the two
from itself or from another;
and in case one, then not the other,
if of course I grant you the use of your “from” at all.
But if I say instead:
in brief it is, and that's all,
who will then ask about a “from”?”
To be sure we can answer such a one as follows:
“You are reflecting; so in addition to this “is”
you also have consciousness;
you therefore have not one but two,
that you can never make into one
and an irrational gap lies between them;
you are in the familiar death of reason”;
so the loophole always remains open to him
that is taken by every non-philosopher:
“I must just stay in this 'from'
and it is impossible to escape it”;
so everything finally comes down to this
that we justify ourselves in the use of the “from.”
Therefore this would be our next task,
to justify the “from” in general as such,
entirely abstracting from its application.
So far, as I ask that you recall,
it has not arisen in any other way
than in factical necessity.

This justification will disclose itself
if only we rigorously pursue
the analysis of the preceding argument.
In the first half are to be found
the remarkable words that without doubt
became immediately evident and clear to you:
“if it were not because-of-itself,
it would be because-of-another
from which it would not then be possible
to abstract in the true primordial act of creating,
or which could not be absent for this creation”;
and yesterday I also added:
“even for truly primordial creation,”
since through thoughtlessness and foolishness
one could easily forget the other through
which alone the first can be.
What then is understood by this primordial creation
which likewise in total tranquility
provides the center of the proof?
Evidently that our thinking, or the light,
if it should be of the right kind,
must accompany the genuine real creation of things
and originate along with it:
hence if the one were to be through another,
it would have to take the “through another”
up into itself and express it;
contrariwise, a thinking that omitted this “through”
would be mere thinking and not absolute,
and would set down a true creation
only factically as bare, dead existence.
This was the first point.

Now it seems here as if the real creation, as real,
could exist on its own and go its own way; and some assert it.
The basis for this illusion has in fact been grasped here.
That is, it rests in the possibility of viewing
primordial creation too in a pale and factical way,
as a result of which it seems to be capable of
existing independent of, and separated from, its appearing.
But we have already seen earlier that light and inward being
(by no means the external existence created by faded thought)
are entirely one and the same;
or, in case we had not yet realized this,
then this is the place to prove it immediately;
because if absolutely unchanging and unchangeable
self identical light must accompany creation,
then there is no light without creation
and creation is likewise inseparable from light:
since it is only because of the light and in the light.
Creation = “from,” “through,” etc.,
so absolute light is itself an absolute “from.”
This was the second point.

Now we, the scientists of knowing, have tacitly
presupposed this as the inner principle of
the possibility for the entire subordinate proof procedure,
which we are now dropping,
and indeed, this is the important thing,
we have done it without any design or plan
before the deed and immediately through the deed itself.
But I claim that the bare possibility
of this presupposition shows its truth and correctness.
Let me prove this first indirectly.
We ourselves in our doing and pursuing are knowing,
thinking, light, or whatever we wish to call it.
If knowing were now absolutely limited,
to the faded thought of an existence separated from thinking,
then we could never have been able to get out of it
to this presupposition of an absolute creation.
Since we have really posited it,
and the light as absolutely one with it,
since we ourselves are immediate light,
we have quite certainly validated the truth
of our claim in immediate being, in action,
since we enacted in that place the very thing we said,
and said what we enacted;
and the one could not be without the other.
Results:

1. The contradiction we have noted up to now in
what we ourselves are and pursue,
between doing and saying = the real and the ideal,
is now annulled, as it alone can be,
ipso facto in us ourselves,
and since this is the criterion of pure reason,
we are ipso facto pure reason.

2. Light has a primordial conception of its own nature
that ipso facto preserves itself in immediate
visible completion of itself.
(Note well: here we are holding simply to the
immediately evident content of our sentences.
It is obvious that questions can
still be raised about their form.
These questions will raise themselves,
and the basic principles for deriving relation
from the absolute may well lie just in answering them.)

3. On the very grounds given, let us leave our
factical conception of the nature of the light,
which may well give rise to the entire we
whose origin we are seeking,
and let us hold simply to the content.
In light, absolute genesis.
Obviously, the light, as light, is qualitative oneness
(which in fact enters as just plain seeing
that cannot further be seen),
which permeates the entire
inner genesis as bare pure genesis
(I am relying here on your [powers of] penetration;
since language can in no way bring us to our goal).
I can now construct this for you further as follows:
this oneness permeates the duality in the “from a-b”;
which duality exists only in the absolute “from”;
but not at all outside it in some independence
and [in an] independent differentiation of terms
so that [the duality's terms] may be reversed
with complete indifference.
These all are constructions in sensory terms,
through which I anticipate myself.
The ground of their possibility must lie in
and be derived from me myself,
insofar as I am the factical concept.
In the strictest sense, nothing matters more than this:
light is the qualitative oneness that penetrates the “from.”
This was the first point.

Now likewise, following our concept,
this “from” and (just for that reason and consequently)
the light's permeation of it,
and also therefore the entire
qualitative oneness of the light,
that indeed can only be thought in relation to
a “from” and its duality in order to annul it;
all this, I say, has its ground in the light itself,
no longer as qualitative but rather as an inscrutible oneness.
Therefore, there exists between the light in itself and
the entire preceding relationship
a new, only entirely one-sided “from”;
and this latter denotes the absolute effect of the light;
to the contrary, the entire first relationship simply shows
the appearance of this effect,
of the qualitative immediately self-effecting light.
This was the second point.

As a genesis, every “from” posits light;
just as previously light posited genesis:
and indeed, since the absolute “from” of
the pure, inaccessible principle rests here,
it posits absolute light,
without the genesis ever becoming visible,
and [posits] itself only
in this absolutely factical light
and from this factical light.
If you have seen into this,
then reflect now on yourself.
We have seen into the “from” in just this way,
and by means of it have seen into the 0
whose inaccessibility we have previously admitted,
and have seen into it as unconditionally existing,
objective and so as having to exist,
if appearance is to arise.
This is the fact.
How have we explained it?
Thus: there is an absolute, immediate “from,”
which as such must appear in a seeing,
itself moreover invisible.

Hence we ourselves,
with the whole content
of our immediate seeing,
are the primordial appearance
of the inaccessible light
in its primordial effect,
and a-b is mere appearance of appearance.
And so the primordial facticity,
the original objectification of reason,
as existing and genetic, is thereby clarified
from the original law of light,
and our task has been completed
in its highest principle.
I have no reservations in letting you go
for the week with these provisions.
Monday [21 May], a discussion.
