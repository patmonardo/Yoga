There is a truth which alone is true
and everything apart from this is unconditionally false.
Further this truth can actually be found
and be immediately evident as unconditionally true.
Philosophy should present the truth.
But what is truth?
And what do we actually search for when we search for it?
Let's just consider what we will not allow to count as truth;
namely when things can be this way or equally well the other;
for example the multiplicity and variability of opinions.
Thus, truth is absolute oneness and invariableness of opinion.
The essence of philosophy is to trace all multiplicity back to absolute oneness.
All multiplicity absolutely without exception.
Absolute oneness is what is true and in itself unchangeable,
its opposite contained purely within itself.
The task is to reciprocally conceive
multiplicity through oneness and oneness through multiplicity.
As a principle, Oneness (= A) illuminates such multiplicity
and conversely multiplicity in its ontological ground
can be grasped only by proceeding from A.

The difference in all philosophical systems
can only reside in their principle of oneness.
The one true self-contained in-itself.
The task of philosophy is to present the absolute.
If one principle alone is right and true,
then only one philosophy is true.
In all philosophies prior to Kant,
the absolute was located in being.
The absolute should be the in-itself.
Since there is only one absolute,
a philosophy which has not made this one true absolute its own
doesnt have the absolute at all but only something relative,
a product of an unperceived disjunction,
which for this reason must have an opposing term.
Such a philosophy leads all multiplicity back,
not to absolute oneness as the task requires,
but only to a subordinate, relative oneness,
and thus it is refuted.

Absolutely all being posits a thinking or consciousness of itself.
For all that one names it (absolute, I) and separates it from being,
if one objectifies it and separates it from oneself,
then it is the same old thing in itself.
Therefore mere being is one half of a whole
together with the thought of it
and is therefore one term of
an original and more general disjunction.
Thus absolute oneness can no more reside in being
than in its correlative consciousness.
It can be as little posited in the thing
as in the representation of the thing.
Rather it resides in the principle of
the absolute oneness and indivisibilty of both.
which is equally the principle of their disjunction.
This is the principle of pure knowing, knowing in itself,
thus completely objectless knowing;
because otherwise it would not be knowing in itself
but would require objectivity for its being.
It is distinct from consciousness
which posits a being and is only a half.
Absolutely all being presupposes a thought or consciousness
having that being as its object.
Consequently being is a term of a disjunction,
the other half of which is thinking.
For this reason oneness is not to be found
in the one or the other
but in the connection of both.
Oneness thus is the same as pure knowing in and for itself
and therefore it is knowledge of nothing.
Oneness is found in truth and certainty,
which is not certainty about any particular thing,
since in that case the disjunction of being and thinking
is already posited in it.

Transcendental Philosophy

All transcendental philosophy posits the absolute
neither in being nor in consciousness but in the union of both.
Truth and certainty in and for itself equals the absolute.
For this kind of philosophy the difference between being and thinking,
as valid in itself, totally disappears in the insight that
no being is possible without thinking, and vice versa.

Critical Philosophy

The system of the transcendental ideas

What we have to do with here is not a logical dialectic
that abstracts from every content of cognition
and merely discovers false illusion in the form of syllogisms,
but rather a transcendental dialectic that, fully a priori,
is supposed to contain both the origin of certain cognitions from pure reason
and inferred concepts whose objects cannot be given empirically at all,
and so lies wholly outside the faculty of the pure understanding.
We have gathered from the natural relation that the transcendental use of our cognition
must have in its inferences as well as in its judgments that
there will be only three species of dialectical inferences
relating to the three species of inference
by which reason can arrive at cognition from principles
and that in everything the concern of reason is to ascend from the conditioned synthesis
to which the understanding always remains bound
toward the unconditioned which the understanding can never reach.

Now what is universal in every relation that our representations can have is
1) the relation to the subject,
2) the relation to objects,
and indeed  either as appearances or as objects of thinking in general.
If we combine the subdivision with the above division,
then all the relation of representations of which we can make
either a concept or an idea of are three sorts:
1) the relation to the subject,
2) the manifold of the object in appearance, and
3) to all things in general.

Now all pure concepts have to do generally with the synthetic unity of representations,
but the concepts of pure reason (transcendental ideas) have to do with the
unconditioned synthetic unity of all conditions in general.
Consequently all transcendental ideas will be brought under three classes,
of which the first contains the absolute (unconditioned) unity of the thinking subject,
the second the absolute unity of the series of conditions of appearance,
the third the absolute unity of the condition of all objects of thought in general.

The thinking subject is the object of psychology,
the sum total of all appearances (the world) is the object of cosmology,
and the thing that contains the supreme condition of
the possibility of everything that can be thought
(the being of all beings) is the object of theology.
Thus pure reason provides the ideas for
a transcendental doctrine of the soul,
transcendental science of the world and
finally also a transcendental cognition of God.
Even so much as the mere sketch of the sciences is not prescribed by the understanding.
even if it's combined with the highest logical use of reason,
with all the inferences through which we can think of progressing from
an object of the understanding (appearances) to all other objects
even to the most distant members of the empirical synthesis
rather such a project is exhausted exclusively
a pure and genuine product or problem of pure reason

What modi of pure rational concepts stand
under these three titles of transcendental ideas
will be finally displayed in the following.
They run along the thread of the categories.
For pure reason is never related directly to objects but instead to
concepts of them given by the understanding.
Likewise it can be made clear only in the complete execution how reason,
exclusively through the synthetic use of the same function
it employs in the categorical syllogism,
must necessarily come to the concept of the absolute unity of the thinking subject.
How the logical procedure and hypothetical syllogisms leads to the
ideas of the absolutely unconditioned in a series of given conditions and
finally how the mere form of the disjunctive syllogism necessarily carries with it
the highest rational concept of a being of all beings;
a thought which at first glance appears extremely paradoxical.

Three kinds of antinomy of pure reason

The following important comment suggests itself here:
namely that there are three kinds of antinomy of pure reason
which, however, all coincide in this,
that they force reason to give up the otherwise very natural presupposition
that holds objects of the senses to be things in themselves
and rather to count them as appearances and ascribe to them
an intelligible substratum (something supersensible)
the concept of which is only an idea and permits no genuine cognition.
Without such an antinomy reason would never be able to decide
on the assumption of a principle that so narrows the field of its
speculation and on sacrifices in which so many otherwise
shining hopes must entirely disappear for even now when in
compensation for these losses on all the greater employment opens up for
Anna practical respect it seems unable to depart from those hopes to free itself
from its old dependency without pain.

That there are three kinds of antinomies is grounded
in the fact that there are three cognitive faculties --
understanding, the power of judgment, and reason --
each of which (as a higher cognitive faculty) must have its a priori principles
for then reason insofar as that judges concerning these principles
and so unremittingly demands with the cards all of them the unconditioned
for the given conditions which however can never be found if one considers
the sensible as belonging to the things of themselves
rather than describing to it as mere appearance something super sensible
the intelligible substratum of nature outside as soon within us
as a thing in itself there is then an antinomy of reason with regard
to the theoretical use of the understanding extending to the unconditioned
for the faculty of cognition and then to the movie of reason with regards
to the aesthetic use of the power of judgment for the feeling of pleasure
and displeasure and antinomy with regard to the practical use of reason
which is intrinsically self legislated for the extent that all these faculties
have their higher principles a priori and in accordance with an inescapable
requirement of supposed to be able to judge and determine their object
unconditionally in accordance with these principles

With regard to two antinomies of those higher cognitive faculties,
that of the theoretical and that of the practical employment,
we've already shown elsewhere their {unavoidability}
if judgments of this kind do not look back to a supersensible substratum
of the given objects, as appearances, but also their {resolvability}
as soon as the latter happens.
Now as far as the antinomy in the use of the power of judgment
and the resolution of it given here are concerned,
there is no other means for avoiding it then
{either} to deny that the
aesthetic judgment of taste is grounded on any principal a priori,
so that all claim to the necessity of universal universal assent is a
groundless, empty delusion, and a judgment of taste deserves to be held to be
correct only insofar as it {happens} that many people agree about it,
and even this, strictly speaking, not because once one suspects an a priori
principle behind their consensus, but rather (as in the taste of the palete)
because the subjects are contingently organized in the same way;
{or} one must assume that the judgment of taste is really a concealed
judgment of reason about the perfection that is revealed anything in a relation
of the manifold in it to the end so that is called aesthetic only on account
of the confusion that attaches to this reflection of ours, although at
bottom it is teleological -- in which case one could declare the resolution
of the antinomy by means of transcendental ideas to be unnecessary
and void, and thus unite those laws of taste of the objects of the senses
not as mere appearances but also as things in themselves.
But how little the one as well as the other subterfuge succeeds
has been shown in several places in the exposition of the
judgments of taste.

But if it is conceded that our deduction is at least on the right track,
even if it has not been made clear enough in every detail,
then three ideas are revealed:
first, that of the supersensible in general,
without further determination, as the substratum of nature;
second, the very same thing as the principle of the subjective
purposiveness of nature for our faculty of cognition;
third, the very same thing, as the principle of the ends of freedom
and principle of the correspondence of freedom with those ends in the moral sphere.

{Architectonic of pure reason}

By an architectonic I understand the art of systems.
Since systematic unity is that which first makes ordinary cognition into science, i.e.,
makes a system out of a mere aggregate of it,
architectonic is the doctrine of that which is
scientific in our cognition in general,
and therefore necessarily belongs to the doctrine of method.

Under the government of reason our cognitions cannot at all
constitute a rhapsody but must constitute a system,
in which alone they can support and advance its essential ends.
I understand by a system, however,
the unity of the manifold cognitions under one idea.
This is the rational concept of the form of a whole,
insofar as through this the domain of the manifold
as well as the position of the parts with respect to
each other is determined a priori.
The scientific rational concept thus contains
the end and the form of the whole that is congruent with it.
The unity of the end, to which all parts are related
and in the idea of which they are also related to each other,
allows the absence of any part to be noticed
in our knowledge of the rest,
and there can be no contingent addition
or undetermined magnitude of perfection
that does not kant have its boundaries determined a priori.
The whole is therefore articulated (articulatio)
and not heaped together (coacervatio);
it can, to be sure, grow internally (per intus susceptionem)
but not externally (per appositionem),
like an animal body, whose growth does not add a limb
but rather makes each limb stronger and fitter
for its end without any alteration of proportion.

For its execution the idea needs a schema, i.e.,
an essential manifoldness and order of the parts
determined a priori from the principle of the end.
A schema that is not outlined in accordance with an idea,
i.e., from the chief end of reason, but empirically,
in accordance with aims occurring contingently
(whose number one cannot know in advance),
yields technical unity,
but that which arises only in consequence of an idea
(where reason provides the ends a priori
and does not await them empirically)
grounds architectonic unity.
What we call science,
whose schema contains the outline (monogramma)
and the division of the whole into members
in conformity with the idea,
i.e., a priori, cannot arise technically,
from the similarity of the manifold
or the contingent use of cognition in concreto
for all sorts of arbitrary external ends,
but arises architectonically,
for the sake of its affinity and its derivation
from a single supreme and inner end,
which first makes possible the whole;
such a science must be distinguished from all others
with certainty and in accordance with principles.

Nobody attempts to establish a science
without grounding it on an idea.
But in its elaboration the schema,
indeed even the definition of the science
which is given right at the outset,
seldom corresponds to the idea;
for this lies in reason like a seed,
all of whose parts still lie very involuted
and are hardly recognizable even under microscopic observation.
For this reason sciences, since they have all been thought out from
the viewpoint of a certain general interest, must not be explained and
determined in accordance with the description given by their founder,
but rather in accordance with the idea, grounded in reason itself, of the
natural unity of the parts that have been brought together.
For the founder and even his most recent successors often fumble around with
an idea that they have not even made distinct to themselves
and that therefore cannot determine the special content,
the articulation (systematic unity) and boundaries of the science.

It is too bad that it is first possible for us to glimpse
the idea in a clearer light and to outline a whole architectonically,
in accordance with the ends of reason,
only after we have long collected relevant cognitions haphazardly
like building materials and worked through them technically
with only a hint from an idea lying hidden within us.
The systems seem to have been formed, like maggots, by a generatio aequivoca
from the mere confluence of aggregated concepts,
garbled at first but complete in time,
although they all had their schema, as the original seed,
in the mere self-development of reason,
and on that account are not
merely each articulated for themselves in accordance with an idea
but are rather all in turn purposively united with each other
as members of a whole in a system of human cognition,
and allow an architectonic to all human knowledge,
which at the present time, since so much material has already been collected
or can be taken from the ruins of collapsed older edifices,
would not merely be possible but would not even be very difficult.
We shall content ourselves here with the completion of our task, namely,
merely outlining the architectonic of all cognition from pure reason,
and begin only at the point where the general root of our cognitive power
 divides and branches out into two stems, one of which is reason.
By "reason" I here understand, however, the entire higher faculty of cognition,
and I therefore contrast the rational to the empirical.

If I abstract from all content of cognition, objectively considered,
then all cognition, considered subjectively, is either historical or rational.
Historical cognition is cognitio ex datis,
a rational cognition, however, cognitio ex principiis.
However a cognition may have been given originally,
it is still historical for him who possesses it if he cognizes it
only to the degree and extent that it has been given to him from elsewhere,
whether it has been given to him through immediate experience
or told to him or even given to him through instruction (general cognitions).
Hence he who has properly learned a system of philosophy,
e.g., the Wolffian system,
although he has in his head all of the principles, explanations, and proofs
together with the division of the entire theoretical edifice,
and can count everything off on his fingers,
still has nothing other than a complete historical cognition
of the Wolffian philosophy;
he knows and judges only as much as has been given to him.
If you dispute one of his definitions,
he has no idea where to get another one.
He has formed himself according to an alien reason,
but the faculty of imitation is not that of generation,
i.e., the cognition did not arise from reason for him,
and although objectively it was certainly a rational cognition,
subjectively it is still merely historical.
He has grasped and preserved well, i.e., he has learned,
and is a plaster cast of a living human being.
Rational cognitions that are objectively so (i.e.,
could have arisen originally only out of
the reason of human beings themselves),
may also bear this name subjectively only if
they have been drawn out of the universal sources of reason,
from which critique, indeed even the rejection of what has been learned,
can also arise, i.e., from principles.

Now all rational cognition is either cognition from concepts
or cognition from the construction of concepts;
the former is called philosophical, the latter mathematical.
I have already dealt with the inner difference between
the two in the first chapter.
A cognition can accordingly be objectively philosophical
and yet subjectively historical,
as is the case with most students
and with all of those who never see beyond their school
and remain students their whole lives.
But it is strange that mathematical cognition, however one has learned it,
can still count subjectively as rational cognition,
and that the difference present in the case of
philosophical cognition is not present in this case.
The cause of this is that the sources of cognition on which alone
the teacher can draw lie nowhere other than in the
essential and genuine principles of reason,
and consequently cannot be derived from anywhere else by the student,
nor disputed in any way, precisely because
reason is here used in concreto though nevertheless a priori,
founded, that is, in pure and therefore error-free intuition,
and excludes all deception and error.
Among all rational sciences (a priori), therefore,
only mathematics can be learned, never philosophy (except historically);
rather, as far as reason is concerned, we can at best only learn to philosophize.

Now the system of all philosophical cognition is philosophy.
One must take this objectively if one understands by it the archetype for the assessment
of all attempts to philosophize, which should serve to assess" each subjective philosophy,
the structure of which is often so manifold and variable.
In this way philosophy is a mere idea of a possible science,
which is nowhere given in concreto,
but which one seeks to approach in various ways until the only footpath,
much overgrown by sensibility, is discovered,
and the hitherto unsuccessful ectype,
so far as it has been granted to humans,
is made equal to the archetype.
Until then one cannot learn any philosophy;
for where is it, who has possession of it,
and by what can it be recognized?
One can only learn to philosophize,
i.e., to exercise the talent of reason in prosecuting its general
principles in certain experiments that come to hand,
but always with the reservation of the right of reason
to investigate the sources of these
principles themselves and to confirm or reject them.

Until now, however, the concept of philosophy has been only a scholastic concept,
namely that of a system of cognition that is sought
only as a science without having as its end anything more
than the systematic unity of this knowledge,
thus the logical perfection of cognition.
But there is also a cosmopolitan concept (conceptus cosmicus)
that has always grounded this term,
especially when it is, as it were, personified
and represented as an archetype in the ideal of the philosopher.
From this point of view philosophy is
the science of the relation of all  cognition
to the essential ends of human reason (teleologia rationis humanae),
and the philosopher is not an artist of reason
but the legislator of human reason.
It would be very boastful to call oneself a philosopher in this sense
and to pretend to have equaled the archetype, which lies only in the idea.

The mathematician, the naturalist, the logician are only artists of reason,
however eminent the former may be in rational cognitions and
however much progress the latter may have made in philosophical cognition.
There is still a teacher in the ideal, who controls all of these
and uses them as tools to advance the essential ends of human reason.
Him alone we must call the philosopher;
however, since he himself is still found nowhere,
although the idea of his legislation is found in every human reason,
we will confine ourselves to the latter
and determine more precisely what philosophy,
in accordance with this cosmopolitan concept,
prescribes for systematic unity from the standpoint of ends.
Essential ends are on this account not yet the highest,
of which (in the complete systematic unity of reason)
there can be only a single one.
Hence they are either the final end, or subalternate ends,
which necessarily belong to the former as means.
The former is nothing other than the entire vocation of human beings,
and the philosophy of it is called moral philosophy.
On account of the preeminence which moral philosophy had
over all other applications of reason, the ancients understood
by the name of "philosopher" first and foremost the moralist,
and even the outer appearance of self-control through reason still suffices today
for calling someone a philosopher after a certain analogy,
in spite of his limited knowledge.

Now the legislation of human reason (philosophy) has two objects,
nature and freedom,
and thus contains the natural law as well as the moral law,
initially in two separate systems
but ultimately in a single philosophical system.
The philosophy of nature pertains to everything that is;
that of morals only to that which should be.

All philosophy, however, is either cognition from pure reason
or rational cognition from empirical principles.
The former is called pure philosophy, the latter empirical.

Now the philosophy of pure reason is either propaedeutic (preparation),
which investigates the faculty of reason in regard to all pure a priori cognition,
and is called {critique},
or, second, the system of pure reason (science),
the whole (true as well as apparent) philosophical cognition
from pure reason in systematic interconnection,
and is called {metaphysics};
this name can also be given to all of pure philosophy including the critique,
in order to comprehend the investigation of everything that can ever be cognized a priori
as well as the presentation of that which constitutes
a system of pure philosophical cognitions of this kind,
but in distinction from all empirical as well as mathematical use of reason.

Metaphysics is divided into the metaphysics of the speculative
and the practical use of pure reason, and is therefore either
metaphysics of nature or metaphysics of morals.
The former contains all rational principles from mere concepts
(hence with the exclusion of mathematics)
for the theoretical cognition of all things;
the latter, the principles which determine action and omission a priori
and make them necessary.
Now morality is the only lawfulness of actions which can be
derived entirely a priori from principles.
Hence the metaphysics of morals is really the pure morality,
which is not grounded on any anthropology (no empirical condition).
The metaphysics of speculative reason
is that which has customarily been called metaphysics
in the narrower sense;
but insofar as the pure doctrine of morals nevertheless
belongs to the special stem of human
and indeed philosophical cognition from pure reason,
we will retain this term for it,
although we set it aside here as not {now} pertaining to our end.

It is of the utmost importance to {isolate} cognitions
that differ from one another in their species and origin,
and carefully to avoid mixing them together with others
with which they are usually connected in their use.
What chemists do in analyzing materials,
what mathematicians do in their pure theory of magnitude,
the philosopher is even more obliged to do,
so that he can securely determine the proper value and influence
of the advantage that a special kind of cognition has
over the aimless use of the understanding.
Hence human reason has never been able to dispense with a metaphysics
as long as it has thought, or rather reflected,
though it has never been able to present it in a manner
sufficiently purified of everything foreign to it.
The idea of such a science is just as old as speculative human reason;
and what reason does not speculate, whether in a scholastic or a popular manner?
One must nevertheless admit that the distinction of the two elements in our cognition,
one of which is in our power completely a priori
but the other of which can be derived only from experience a posteriori,
has remained very indistinct, even among professional thinkers,
and hence the determination of the bounds of a special kind of cognition,
and thus the genuine idea of a science with which human reason
has so long and so intensively occupied itself, has never been accomplished.
When it was said that metaphysics is
the science of the first principles of human cognition,
an entirely special kind of cognition was not thereby marked off,
but only a rank in regard to generality, through which, therefore,
it could not be clearly differentiated from empirical cognition;
for even among empirical principles some are more general
and therefore higher than others,
and in the series of such a subordination
(where one does not differentiate that which can be cognized completely a priori
from that which can be cognized only a posteriori),
where is one to make the cut that distinguishes
the first part and highest members
from the last part and the subordinate members?
What would one say if chronology could designate the epochs of the world only by
dividing them into the first centuries and the rest that follow them?
One would ask, Do the fifth century, the tenth century, and so on
also belong among the first ones?;
likewise I ask, Does the concept of that
which is extended belong to metaphysics?
You answer, Yes! But what about that of body?
Yes! And that of fluid body? You are stumped,
for if  it goes on this way,
then everything will belong to metaphysics.
From this one sees that the mere degree of subordination
(the particular under the universal)
cannot determine any boundaries for a science,
but rather, in our case, only the
complete heterogeneity and difference of origin can.
But what obscured the fundamental idea of metaphysics
from yet another side was that, as a priori cognition,
it shows a certain homogeneity with mathematics, to which,
as far as a priori origin is concerned, it is no doubt related;
but the comparison between the kind of
cognition from concepts in the former
with the manner of judging a priori
through the mere construction of concepts in the latter
requires a difference between philosophical and mathematical cognition -
thus a decided heterogeneity is revealed, which was always felt, as it were,
but was never able to be brought to distinct criteria.
Thus it has been the case until now that since philosophers themselves
erred  in the development of the idea of their science,
its elaboration could have no determinate end and no secure guideline,
and philosophers, with such arbitrarily designed projects,
ignorant of the path they had to take,
and always disputing among themselves about the discoveries
that each would like to have made on his own,
have brought their science into contempt
first among others and finally even among themselves.

Thus all pure a priori cognition,
by means of the special faculty of cognition
in which alone it can have its seat,
constitutes a special unity,
and metaphysics is that philosophy which is
to present that cognition in this systematic unity.
Its speculative part, to which this name has been especially appropriated,
namely that which we call metaphysics of nature
and which considers everything insofar as it is
(not that which ought to be)
on the basis of a priori concepts,
is divided in the following way.

Metaphysics in this narrower sense consists of
transcendental philosophy and the physiology of pure reason.
The former considers only the understanding and reason itself
in a system of all concepts and principles
that are related to objects in general,
without assuming objects that would be given (Ontologia);
the latter considers nature, i.e.,
the sum total of given objects
(whether they are given by the senses or,
if one will, by another kind of intuition),
and is therefore physiology
(though only rationalis).
Now, however, the use of reason in this rational consideration of nature
is either physical or hyperphysical,
or, better, either immanent or transcendent.
The former pertains to nature so far as
its cognition can be applied in experience (in concreto),
the latter to that connection of the objects of experience
which surpasses all experience.
Hence this transcendent physiology has either an inner
connection to its object or an outer one,
both of which, however, go beyond possible experience;
the former is the physiology of nature in its entirety, i.e.,
the {transcendental cognition of the world},
the latter that of the connection of nature in its entirety to a being beyond nature, i.e.,
the transcendental {cognition of God}.

Immanent physiology, on the contrary, considers nature as the sum
total of all objects of the senses,
thus considers it as it is given to us,
but only in accordance with a priori conditions,
under which it can be given to us in general.
There are, however, only two sorts of objects for this.
1. Those of outer sense, thus the sum total of these, corporeal nature.
2. The object of inner sense, the soul,
and, in accordance with the fundamental concepts of this in general, thinking nature.
The metaphysics of corporeal nature is called physics,
but, since it is to contain only the principles
of its a priori cognition, rational physics.
The metaphysics of thinking nature is called psychology,
and because of the cause that has just been adduced
only the rational cognition of this ishere meant.

Accordingly, the entire system of metaphysics consists of four main parts.
I. Ontology. 2. Rational Physiology. 3. Rational Cosmology. 4. Rational Theology.
The second part, namely the doctrine of nature of pure reason,
contains two divisions, physica rationalis and psychologia rationalis.

The original idea of a philosophy of pure reason itself prescribes this division;
it is therefore architectonic, in conformity with its essential ends,
and not merely technical, in accordance with contingently perceived affinities
and, as it were, established by good luck;
and for that very reason it is unchangeable and legislative.
However, there are several points here which could arouse reservations
and weaken the conviction of its lawfulness.

First, how can I expect an a priori cognition and thus a metaphysics
of objects that are given to our senses, thus given a posteriori?
And how is it possible to cognize the nature of things
in accordance with a priori principles
and to arrive at a rational physiology? The answer is:
We take from experience nothing more than
what is necessary to give ourselves an object,
partly of outer and partly of inner sense.
The former is accomplished through the mere concept of matter
(impenetrable lifeless extension),
the latter through the concept of a thinking being
(in the empirically inner representation "I think").
Otherwise, we must in the entire metaphysics of these objects
abstain entirely from any empirical principles
that might add any sort of experience beyond the concept
in order to judge something about these objects.

Second: Once one gives up the hope of achieving anything useful a priori,
where does that leave empirical psychology,
which has always asserted its place in metaphysics,
and from which one has expected such great enlightenment in our own times?
I answer: It comes in where the proper (empirical) doctrine of nature must be put,
namely on the side of applied philosophy,
for which pure philosophy contains the a priori principles,
which must therefore be combined but never confused with the former.
Empirical psychology must thus be entirely banned from metaphysics,
and is already excluded by the idea of it.
Nevertheless, in accord with the customary scholastic usage
one must still concede it a little place
(although only as an episode) in metaphysics,
and indeed from economic motives,
since it is not yet rich enough to comprise a subject on its own
and yet it is too important for one to expel it entirely
or attach it somewhere else where it may well have
even less affinity than in metaphysics.
It is thus merely a long-accepted foreigner,
to whom one grants refuge for a while until it can establish
its own domicile in a complete anthropology
(the pendant to the empirical doctrine of nature).

This is, therefore, the general idea of metaphysics, which,
since we initially expected more from it than could appropriately be demanded
and long amused ourselves with pleasant expectations,
in the end fell into general contempt when we found ourselves deceived in our hopes.
From the whole course of our critique we will have been sufficiently convinced
that even though metaphysics cannot be the foundation of religion,
yet it must always remain its bulwark, and that human reason,
which is already dialectical on account of the tendency of its nature,
could never dispense with such a science, which reins it in
and, by means of a scientific and fully illuminating self-knowledge,
prevents the devastations that a lawless speculative reason
would otherwise inevitably perpetrate in both morality and religion.
We can therefore be sure that however obstinate or disdainful they may be
who know how to judge a science not in accord with its nature,
but only from its contingent effects,
we will always return to metaphysics as to a beloved
from whom we have been estranged,
since reason, because essential ends are at issue here,
must work without respite either for sound insight
or for the destruction of good insights that are already to hand.

Thus the metaphysics of nature as well as morals,
but above all the preparatory (propaedeutic) critique of reason
that dares to fly with its own wings, alone constitute
that which we can call philosophy in general sense.
This relates everything to wisdom, but through the path of science,
the only one which, once cleared, is never overgrown, and never leads to error.
Mathematics, natural science, even the empirical knowledge of humankind,
have a high value as means, for the most part to contingent
but yet ultimately to necessary and essential ends of humanity,
but only through the mediation of a rational cognition from mere concepts,
which, call it what one will, is really nothing but metaphysics.

Just for this reason metaphysics is also
the culmination of all culture of human reason,
which is indispensable even if one sets aside its
influence as a science for certain determinate ends.
For it considers reason according to its elements and highest maxims,
which must ground even the possibility of some sciences
and the use of all of them.
That as mere speculation it serves more to prevent errors
than to amplify cognition does no damage to its value,
but rather gives it all the more dignity and authority
through its office as censor,
which secures the general order and unity,
indeed the well-being of the scientific community,
and prevents its cheerful and fruitful efforts
from straying from the chief end, that of the general happiness.

\subsection{Science of Knowing}

Kant conceived the absolute as the indivisible union of being and thinking.
But he did not conceive it in its pure self-sufficiency in and for itself
but rather only as a common basic determination or accident of its three primoidial modes.
As a result for him there are three absolutes and the true unitary absolute
fades to their common property.
In the Critique of Pure Reason, his absolute (x) is the sensible world
In the Critique of Practical Reason, the I comes to light as
something in itself through the inherent categorical concept
and thus second absolute (z) is the moral world
Its essence consists in discovering the root in which the
sensible and supersensible worlds come together and then
providing the actual conceptual derivation of both worlds
from a single principle.
A is admitted. It divides itself into B and T
and simultaneously into x, y, z.

Both divisions are equally absolute;
one is not possible without the other.

Therefore the divisions cannot be seen into immediately;
instead they can be seen into only mediately
in the higher insight of their oneness.

Kant understood A as the link between B and T
but he did not grasp it in its absolute autonomy.
Instead he made it the basic common property
and accident of three absolutes.
In this the science of knowing distinguishes itself from him.
This science must hold that knowing (or certainty),
as soon as we have characterized it as absolute,
must be a purely self sustaining substance.
Examing sequentially the memory of inner experience
reveals the object and its representation with all their modifications

First let me clarify a point raised at the end of the last lecture
which might occassion misunderstanding.
Absolute A is divided into B and T and into x,y,z all at one stroke;
as into x,y,z so into B and T as into B and T so into x,y,z.
But how have I expressed myself just now?
Once commencing with x and the other time commencing with B.
This is just a perspective bias in my speaking.
I certainly know and even expressly assert that implicitly
beyond the possibilities of my mode of expression
and my discursive construction, both are totally identical,
completely comprehended in one self-contained stroke
Therefore I am constructing what cannot be constructed
with full awareness that it cannot be constructed.

Let me continue now to characterize the science of knowing
on the basis of the indicators found in comparing it with
Kantian transcendentalism, among other things.
I said that Kant very well understood
A as the link between B and T,
but that he did not grasp it in its absolute autonomy.
Instead, he made it the basic common property and accident of three absolutes.
In this the science of knowing distinguishes itself from him.
Therefore this science must hold that knowing (or certainty),
as soon as we have characterized it as A,
must actually be a purely self-sustaining substance;
that we can realize it as such for ourselves;
it is just in this realization that
the genuine realization of the science consists.

To begin with, we can demonstrate immediately that
knowing can actually appear as something standing on its own.
I ask you to look sequentially at your own inner experience:
if you remember it accurately, you will find the object
and its representation, with all their modifications.
But now I ask further:
do you not know in all these modifications;
and is not your knowing, as knowing,
the same self-identical knowing in all variations of the object.
As surely as you say "Yes" to this inquiry,
(which you would certainly do
if you have carried out the given task),
so surely will knowledge manifest
and present itself to you as (=A)
whatever the variation of its object
(and hence in total abstraction from all objectivity)
remaining as (=A);
and thus as a substantive,
as staying the same as itself through all changes in its object;
and thus as oneness, qualitatively changeless in itself.
This was the first point.

Thus it presents itself to you with
impressive absolutely irrefutable manifestness.
You understand it so certainly that you say
It is absolutely thus I cannot conceive it differently.
And if you were asked for reasons you could refuse the request
and still not give up this contention.
It is manifest to you as absolutely certain.
During all possible variations in the object,
you have said knowing always remain self-identical.
Have you then run through and exhausted all possible changes in object,
testing in each case whether there knowing remains the same?
I do not think so, because how could you have done it?
Therefore this knowing manifests itself independently of such experiments
and completely a priori as self-sustaining and self-identical
independent from all subjectivity and objectivity.

Now, note what actually belongs to this substantial knowing, so conceived,
and do so with the deepest sincerity of self-consciousness,
so that the erroneous view of the science of knowing,
which was criticized at the end of our first lecture,
namely that it locates the Absolute in the knowing
which stands over against its object, doesn't arise here.
It is true that in our experiment we have started with
this consciousness or presentation of an object,
T.B-T.B and so on.
In this part of the experiment B made the T different in every new moment,
because the T was altogether nothing else than the T for this B and disappears with it.
Now when we raised ourselves to the second part by asking
"Is not knowing one and the same throughout?"
and finding it to be so, we raised ourselves
above all differences of T as well as of B.
Therefore we could express ourselves much more accurately and precisely:
knowing, which for this reason is not subjective,
is absolutely unalterable and self identical
not just independently from all variability of the object,
but also independently from all variability of the subject
without which the object doesn't exist.
The changeable is nothing further, neither the object nor the subject,
but just the mere pure changeable and nothing else.
Now this changeable and its continuing changeability,
which is itself unchangeable,
divides itself into subject and object
in the purely unchangeable,
in which division of subject and object falls away as does change,
opposes itself to the changeable.

Here has been disclosed a splendid example of an insight
that comes from exhaustive, continuous searching,
which cannot be derived from experience,
but which rather is absolutely a priori.
And so past experiences obligate me to entreat everyone here
to whom this insight has been evident
(which I think is the case with all of you given the simplicity of the task)
to keep this very example in mind. to hold on to it,
and, if the old empiricist demon shows up,
to attack it and send it away promptly,
until we succeed in completely exterminating it.
I would very much like to be spared the eternal struggle about
whether in general there is manifestness or something a priori,
for both are the same.
Individuals come to the conclusion that this is the case
only by producing it somehow or other in themselves.
This has happened here today and I simply asked that you not forget it.

Result: knowing, in the mentioned meaning as A, has actually appeared to us
as self-sufficient, as independent of all changeability,
and as selfsame and self-contained oneness,
as was presupposed by previous historical reports of the science of knowing.
We therefore already seem to have realized the principle of
the science of knowing in ourselves  and to have penetrated into it.

The second advance in today's lecture is this:
we only seem to have done this.
But this is an empty seeming.
We see merely that it is so,
but we do not see into what it authentically is as this qualitative oneness.
Precisely because we see into only such a that,
we are trapped in a disjunction
and thus in two absolutes, changeability and unchangingness,
to which we might possibly append a third, the undiscoverable root of both,
and thus end up in the same shape as Kant's philosophy.
The ground of this duality, insurmountable in this way, is as follows:
the that must seem self creating just as our recent insight seem to be.
But this appearance is possible only under the condition
that a point of origin appears which seems (as opposed to
the self-creation) to be produced by us
just as the first part of our previously conducted experiment
actually in fact appear to be.
In a word we grasp both changeability and unchangingness
thus equally are inwardly torn into two or three immediate terms.
How then is this to be?
Obviously, is it is clear without any further steps
both that one of the two would have to be grasped mediately
and that this term which is grasped immediately cannot be unchangingness,
which as the absolute can only be realized absolutely
rather must be changeability.
The unchangeable would have to be intuited
not only in its being, which we have already done,
but instead would have to be penetrated in its essence,
its one absolute quality.
It the unchangeable would have to be worked through
in such a way that changeability would be seen
as necessarily proceeding from it
and as mediately graspable only by means of it.

Briefly, clearly, and to fix the point easily in memory:
the insight that knowing is a self-sustaining qualitative oneness
(an insight which is purely provisional
and belongs to a theory of the science of knowing)
leads to the question,
"What is knowing in this qualitative oneness?"
The true nature of the science of knowing resides in answering this question.
In order to analyze this even further, it is clear that for this purpose
one was inwardly construct this essence of knowing.
Or, as in this case is exactly the same thing,
this essence must construct itself.
In this constructive act, it is without any doubt
and is what it is as existing;
and, as existing, it is what it is.
Therefore it is clear that the science of knowing
and the knowing that presents itself as its essential oneness
are entirely one and the same;
that the science of knowing and primordial essential knowing
merge reciprocally into one another and permeate each other;
that in themselves they are not different;
and that the difference which we will still make here
is only a verbal difference of the exact kind mentioned
at the start of this lecture.
The primordial essential knowing is constructive,
thus intrinsically genetic;
this would be the original knowing or certainty in itself.
Manifestness in itself is therefore genetic.

And with this we have specified the deepest characteristic difference
demarcating the science of knowing from all other philosophies,
particularly from the most similar, the Kantian.
All philosophy should terminate in knowing in and for itself.
Knowing, or manifestness in and for itself, is actively genetic.
The highest appearance of knowing,
which no longer expresses its inner essence
but instead just its external existence, is factical;
and since it is still the appearance of knowing, factical manifestness.
All factical manifestness, even if it is the absolute,
remains something objective, alien, self-constructing
but not constructed of knowing,
and therefore something inwardly unexplained,
which an exhausted speculation, skeptical of its own power, calls inexplicable.
Kantian speculation ends at its highest point with factical manifestness:
the insight that at the basis of both the sensible and supersensible world
there must be a principle of connection, thus a thoroughly genetic principle
which creates and determines both worlds absolutely.
This insight which is completely right in itself, could occur to Kant
only as a result of his reason's absolute
but unconsciously operative law;
that it [that is his reason] come to a stop only with absolute oneness,
recognize only this as absolutely substantial,
and derive everything changeable from this one.
This basic law of oneness remains only factical for him,
and therefore its object is unexamined,
because he allows it to work on him only mechanically;
but he does not bring this action itself
and its law into his awareness anew.
If he did so, pure light would dawn on him
and he would come to the science of knowing.
Kant's factical manifestness is not even the highest kind,
because he lets its object emerge from two related terms
and does not grasp it as we have grasped the highest factical object,
namely as pure knowing;
instead he grasps it with the qualification
that it is the link between the sensible and supersensible worlds.
That is, he does not grasp it inwardly and in itself as oneness, but as duality;
his highest principle is a synthesis post factum.
Namely this means a case when by self observation
one discovers in one's consciousness two terms of a disjunction,
and compelled by reason sees that they must intrinsically
be one disregarding the fact that one cannot say how given this oneness
they can likewise become two.
Briefly this is exactly the same procedure by which in our first lecture
we rose from the discovery of the duality of being and thinking
to A as their required necessary connection in order initially
to construct for ourselves the transcendentalism
common to the science of knowing and Kantianism.
but the matter was not to rest
Additionally there should be a synthesis a priori
which is equally an analysis
since it's simultaneously provides the basis for both oneness and duality
Kant's highest manifestness, I said, is factical and not even the highest factical kind.
The highest factical manifest hass has been presented today
the insight into knowings absolute self-sufficiency
without any determination by anything outside itself anything changeable
This is contrary to the Kantian absolute which is determined by
the transition between the sensible and supersensible.
Since now this presently factical element in science is itself
to become actively genetic and developing, then,
change in general will be grounded in it
as a genesis pure and simple.
But by no means will any particular change be so grounded.
It seems that absolute facticity could be discovered only by those
who have raised themselves above all facticity,
as I have actually discovered it
and consistently made use of it
only after discovering the true inner principle of the science of knowing,
and as I am using it now to lead the audience
from that point forward in the genetic process.

Kant's manifestness is factical
we ourselves are presently also standing in facticity
and I add that everywhere in the scientific world
there is no other kind of manifestness except the factical
namely in the first principles
except in the science of knowing
As far as philosophy is concerned
after conducting the demonstration with Kant
we can safely omit test on other
after philosophy mathematics makes claims to manifestness
indeed and some of its Representatives it takes on airs
by elevating itself above philosophy
an error which can be excused
in the light of today's philosophical eclecticism
no extracting here completely from the fact that
things are not so wonderful for mathematics
not even in regard to how it can and should be
the science must confess that its principles
and it nothing more than factical manifest
s regardless of the fact that they will become actively genetic as
for let the arithmetician qua resma Titian simply tell me
how he is able to elicit a solid and permanent number one
or let the geometer explain what fixes and holds his space for him
while he draws his continuous lines through it and Wednesdays
and ever so many other ingredients which he needs for the possibility
of his derivations be given it any other way than through factical intuition
of course this is not in any way constitute an objection
to mathematics as mathematics it can and should be nothing else
it is certainly not our business to obscure the boundaries of the sciences
but one should simply recognize and this size like all other should know
that is neither the first nor self-sufficient
Verizon that the principles of its possibility lion another higher science.

Now, since in the actual sciences generally
no other principles are available
than those that are factically manifest
and since by contrast the science of knowing
intends to introduce entirely genetic manifestness
and then that to deduce the factor call from Ed
it is clear that essentially it's Spirit life
the science of knowing is wholly different from
all previous scientific employments of reason.
It is clear that it is not known to anyone who has not studied it directly
and that nothing can take the place of such study
it is equally clear that there is no perspective or premise
which has appeared in previous life or science from which can be seen as true,
attacked or refuted cuz we ever this perspective or premise is
and however certain it might be still
is none the less surely only fact Khalid manifest
and this science accepts nothing of this kind unconditionally
But does so only under conditions which it determines
in its genetic analysis
but whoever wants to argue against the science of knowing
using such a perspective as its principle wants unconditional agreement
which is already once and for all ruled out in advance
Therefore he is arguing from a premise that has not been accepted
and makes himself ridiculous
the science of knowing can only be judged internally
it could be attacked and refuted only internally
by pointing out an internal contradiction an inner inconsistency or insufficiency
Therefore such activity must be preceded by study and comprehension
and must begin with that.
until now to be sure the opposite order has been tried;
first judging and refuting and after that, God willing, understanding.
as a result nothing has ensued except that the blows
have completely missed the science of knowing,
which has remained hidden from view like an invisible spirit
and they have struck instead the chimeras
which these men have created with their own hands.
Following this, they have gone so far astray with these fantasies
and have spread confusion so extensively that today can be expected
that they would at least understand that they are confused!

Knowing is unchangeable, self-same, self-sustaining,
beyond all change and beyond the subjectivity-objectivity
which is inseparable from change.
The science of knowing must still actually construct
this inner qualitatively unchanging being,
and as soon as it does, it will simultaneously create change as well.

On introducing the schema A / x y z . B T,
I said the science of knowing stood in the point.
But the question arises why not in A? The strict answer is that
it doesnt belong in either of these but rather the oneness of both.
By itself A is objective and therefore inwardly dead.
It should become actively genetic.
The point on the other hand, is merely genetic.
Mere genesis is nothing at all but this is not mere genesis
but the determinate genesis that is required by the absolute qualitative A;
it is a point of oneness.

Since reconstruction is conceiving
and since this very conceiving abandons its own instrinsic validity,
this is precisely conceiving the inconceivable as inconceivable.

this deeper connection must indeed itself be a result
and a lower result of the higher one just described.

Secondary knowing or consciousness with its whole lawful play
by means of fixed change and the manifold, of sensible and supersensible,
of space and time, comes to be, in principle,
through this recently noted and demonstrated
division taken merely as division and nothing else.
Everything we attribute to the subject as originating from it, arises from this.
Because it is clear that the disjunction must be just as absolute as the oneness;
otherwise we would be stuck in oneness and never get outside it to changeability.

As scientists of knowing we never escape the principle of division
but we certainly escape it intellectually with regard to what is valid in itself
in which very regard the principle of division surrenders and negates itself.

Manifestly in our construction by means of the principle of division,
it stands in the place not of
that which is to be valid to the extent it is constructed
but of that which is intrinsically valid.
Thus it stands wholy autonomously between the two principles of oneness and separation.
simultaneously annulling both and positing both.
What is the absolute oneness of the science of knowing?
Neither A nor the point but instead the inner organic oneness of both.
This description is the original and absolutely authentic one.
What are its constituents?
The organic oneness of both is a construction or a concept,
indeed the single absolute concept, abstracted from nothing existing,
since even its own existence and hence the separate existence of
everything conceptual is denied.
Further the construction as such is denied by the manifestness
of what exists autonomously;
thus even the inconceivable, as the inconceivable and nothing more,
is posited by this manifestness.
posited through the negation of the absolute concept,
which must be posited just so it can be annulled.

The necessary unification and indivisibility of the concept and the inconceivable
is clearly seen into and the result may be expressed thus:
if the absolutely inconceivable is to be manifest as solely self-sustaining,
then the concept must be annulled but to be annulled it must be posited
because the inconceivable becomes manifest only with the negation of the concept.
Supplement: hence inconceivable equals unchangeable, concept equals change.
Therefore along with the foregoing it is evident that
if the unchangeable is to appear, there must be change.

Now to be sure inconceivability is only the negation of the concept,
an expression of its annulment. Therefore it is something
which originates from both the concept and knowing themselves;
it is a quality transferred by means of absolute manifestness.
Noting this and therefore abstracting from this quality
nothing remains for oneness except absoluteness or pure self-sufficiency in itself.

The following consideration makes this particularly important and relevant.
What is pure self-sufficient knowing in itself?
The science of knowing has to answer this question or
as we put it more precisely it has to construct
the presupposed inner quality of knowing's inner quality.
We are undertaking this construction here
negating the concept by means of manifestness
and thus the self creation of inconceivability
is this living construction of knowing in
This inconceivability itself originates in the concept
and in pure immediate manifestness;
Likewise the whole quality of the absolute
as well as the fact that a quality can be applied to it at all
originates in the concept.
The absolute is not intrinsically inconceivable since this makes no sense.
It is inconceivable only when the concept itself tries for it,
and this inconceivability is its only property.
Having recognized this inconceivability as an alien quality introduced by knowing,
I said before that only pure self-sufficiency or substantiality
remains in the absolute; and it's quite true that at best
this self sufficiency does not originate in the concept
since it enters only with the latters annulling.
But it is clear that this quality enters only within the immediate manifestness, within intuition,
and this is only the representative and correlate of pure light.
This latter is it genetic principle by which
first of all according to our hypothesis
all manifestness opens up into genetic manifestness,
since pure light manifests itself implicitly as genesis.
secondly the previously presented relationship of concept to being
and vice versa is further determined as follows .
If there is to be an expression and realization of the absolute light
then the concept must be posited so that it can be negated by
the immediate light.
Since the expression of pure light consists just in this negation
but the result of this expression is being in itself.
This result is inconceivable precisely because pure light is simultaneously
destruction of the concept
Thus pure light has prevailed as the one focus and
the sole principle of both being and the concept.

From the preceding it follows that this inconceivable,
as the bearer of all reality in knowing, which we grasp in its principle,
is absolute only as inconceivable, and cannot be thought in any other way.
No other additional hidden qualities are attributable to it.
Just as little can any quality be added to light,
beyond the previously mentioned characteristic,
namely that it annuls the concept and remains absolute being.
If we made such additions we would,
as Kant has been criticized for doing,
run up against something unexplained and perhaps inexplicable.
As support for this contention notice that we have understood it as inconceivable
purely in its form and nothing more.
we have no right to assert anything before we have seen into it
So if we posit some other hidden quality, we have either invented it but
better since pure invention from nothing is completely impossible
we have manufactured it by trying to supply a principle for some facticity.
This happened with Kant when he first factically discovered
the distinction between the sensible and the supersensible worlds
and then added to his absolute the additional inexplicable quality
of linking the two worlds,
a move which pushed us back from genetic manifestness
into merely practical manifestness,
completely contravening the inner spirit of the science of knowing.
Therefore it is important to note that whatever yet to be determined
characteristics the reality appearing in our knowing may carry in itself,
besides the common basic property of inconceivable,
such characteristics by no means require any new absolute grounding principle
besides the one principle of pure light
since this would multiply the number of absolutes.
Rather the multiplicity and change of these various traits
is to be deduced purely from the interaction of the light with itself
and its multifarious relations to concepts and to inconceivable being.

The focus of everything is pure light.
To truly come to this requires that the concept be posited and annulled
and that an intrinsicly inconceivable being be posited.
If it is granted tha light should exist,
then in this judgment everything else mentioned is possible as well.
We have now seen this; it is true; and it remains true forever;
and it expresses the basic principle of all knowing.
We can so designate it for ourselves.

Now, however, I want to completely ignore the content of this insight
and reflect on its form, on our actual situation of insight.
The process was that we freely constructed the concepts and premises
with which we began, that we held them up to each other freely,
and that in holding them together we were gripped by the conviction
that they belonged together absolutely and formed an indivisible oneness.
Thus we created at least the conditions for the self manifesting insight,
and so we likewised appeared to ourselves unconditionally.

But let us not go to work with too much haste;
rather let us consider things a little more deeply.
Did we create what we created because we wished to do so,
and therefore as the result of some earlier knowledge,
which we would have created because we wished to create it
as the result of an even earlier knowledge,
and so on to infinity, so that we might never arrive at a first creation?
Somewhere, if the concept is created it must absolutely and thoroughly create itself
without anything antecedent and without any necessity of a we;
because this we as has been as has been shown always and everywhere
requires some previous knowing and cannot achieve immediate knowing
therefore we cannot create the conditions they must emerge spontaneously.
Reason must create itself independent from any volition or freedom or self
but this proposition disclosed through analysis contradicts the first
which is given reflexively and so immediately
Which one is true and on which should we rely
before trying to answer let us return again to the matter of
the insight which has become conscious versus or in regard to its principal
 in order to grasp it's meaning and truworth clearly
We realized that if light is to be, then the concept must be posited and negated
therefore the light itself is not immediately present in this inside
in the insight does not dissolve into light and coincide with
Instead it is only an insight in relation to the light
an insight which objectifies it grasping it by its inner quality
Thus whatever the principal and true bearer of this insight might be
whether we ourselves as it seems to be poop yourself creating reason
as it also seems to be
the light is not immediately present in this bear
instead it is present merely mediately
in a representative and lightness of itself
First of all that this light occurs merely mediately plies
not just in the science of knowing but in any possible consciousness
that has deposit a concept so it can annul it
And the science of knowing rest on a completely different point
then the one on which many may have assumed it to rest after the last lecture
because undoubtedly knowing was understood to simply

And now to answer the question:
both are clear, therefore both are equally true;
manifestness rests neither on one or the other,
but entirely between both.
We arrive here and this is the first important and significant result,
at the principle of division, not as before a division between two terms,
which in that case are to be instrinsically distinct like A and the point,
but instead a division of something which
remains always inwardly self-same through all division.
In a word, we have to do with conscructing and creating
the very same primal concepts which appear one time
as immanent, in the unconditionally evident final being, the I;
and appear the same time as emmanent, in reason absolute and in itself,
which nevertheless is completely objectified.
Thus, it is a division pure and in itself,
without any division or alteration in the object.

Further, manifestness oscillates betwen these two perspectives.
If it is to be really constructed, then it must be constructed in this way.
Thus it must be constructed as oscillating from a to b and again from b to a,
and as completely creating both;
thus as oscillating between the twofold oscillation,
which was the first point, and which gave rise to a
three or fivefold synthesis.

What is the common element in all these determinations?
The very same representative of the light, seen in its familiar inner quality.
Here it stays. Therefore everything is the same one common consciousness of light.
This consciousness, which is held in common and therefore can only be thought
by means of the science of knowing, can be regarded or represented in
the three or fivefold  modifications only from another standpoint,
which this science alone oversees.

System of Science

Reason

\subsubsection{Being}

{General division}

Being is determined, first, as against another in general; secondly, it is
internally self-determining; thirdly, as this preliminary division is cast off,
it is the abstract indeterminateness and immediacy in which it must be the
beginning.
According to the first determination, being partitions itself off from
essence, for further on in its development it proves to be in its totality only
one sphere of the concept, and to this sphere as moment it opposes another
sphere.
According to the second, it is the sphere within which fall the determi-
nations and the entire movement of its reflection. In this, being will posit
itself in three determinations:
I. as determinateness; as such, quality;
II. as sublated determinateness; magnitude, quantity;
III. as qualitatively determined quantity; measure.
This division, as was generally remarked of such divisions in the
Introduction, is here a preliminary statement; its determinations must
first arise from the movement of being itself, and receive their definitions
and justification by virtue of it. As regards the divergence of this division
from the usual listing of the categories, namely quantity, quality, relation
and modality – for Kant, incidentally, these are supposed to be only classi-
fications of his categories, but are in fact themselves categories, only more
abstract ones – about this, there is nothing to remark here, since the entire
listing will diverge from the usual ordering and meaning of the categories
at every point.
This only can perhaps be remarked, that the determination of quantity
is ordinarily listed ahead of quality and as a rule this is done for no given
reason. It has already been shown that the beginning is made with being
as such, and hence with qualitative being. It is clear from a comparison
of quality with quantity that the former is by nature first. For quantity is
quality which has already become negative; magnitude is the determinate-
ness which, no longer one with being but already distinguished from it, is
the sublated quality that has become indifferent. It includes the alterabil-
ity of being without altering the fact itself, namely being, of which it is
the determination; qualitative determinateness is on the contrary one with
its being, it neither transcends it nor stays within it but is its immediate
estrictedness. Hence quality, as the determinateness which is immediate,
is the first and it is with it that the beginning is to be made.
Measure is a relation, not relation in general but specifically of quality
and quantity to each other; the categories dealt with by Kant under relation
will come up elsewhere in their proper place. Measure, if one so wishes,
can be considered also a modality; but since with Kant modality is no
longer supposed to make up a determination of content, but only concerns
the reference of the content to thought, to the subjective, the result is a
totally heterogeneous reference that does not belong here.
The third determination of being falls within the section Quality inas-
much as being, as abstract immediacy, reduces itself to one single determi-
nateness as against its other determinacies inside its sphere.

\subsubsection{Essence}

The truth of being is essence.
Being is the immediate. Since the goal of knowledge is the truth, what
being is in and for itself, knowledge does not stop at the immediate and
its determinations, but penetrates beyond it on the presupposition that
behind this being there still is something other than being itself, and that
this background constitutes the truth of being. This cognition is a mediated
knowledge, for it is not to be found with and in essence immediately, but
starts off from an other, from being, and has a prior way to make, the way
that leads over and beyond being or that rather penetrates into it. Only
inasmuch as knowledge recollects itself into itself out of immediate being,
does it find essence through this mediation. – The German language has
kept “essence” (Wesen) in the past participle (gewesen) of the verb “to be”
(sein), for essence is past – but timelessly past – being.
When this movement is represented as a pathway of knowledge, this
beginning with being and the subsequent advance which sublates being
and arrives at essence as a mediated term appears to be an activity of
cognition external to being and indifferent to its nature.
But this course is the movement of being itself. That it is being’s nature
to recollect itself, and that it becomes essence by virtue of this interiorizing,
this has been displayed in being itself.
If, therefore, the absolute was at first determined as being, now it is
determined as essence. Cognition cannot in general stop at the manifold of
existence; but neither can it stop at being, pure being; immediately one is
forced to the reflection that this pure being, this negation of everything finite,
presupposes a recollection and a movement which has distilled immediate
existence into pure being. Being thus comes to be determined as essence, as
a being in which everything determined and finite is negated. So it is simple
unity, void of determination, from which the determinate has been removed
in an external manner; to this unity the determinate was itself something
external and, after this removal, it still remains opposite to it; for it has not
been sublated in itself but relatively, only with reference to this unity. – We
already noted above 1 that if pure essence is defined as the sum total of all
realities, these realities are equally subject to the nature of determinateness
and abstractive reflection and their sum total is reduced to empty simplicity.
Thus defined, essence is only a product, an artifact. External reflection,
which is abstraction, only lifts the determinacies of being out of what is left
over as essence and only deposits them, as it were, somewhere else, letting
them exist as before. In this way, however, essence is neither in itself nor for
itself; it is by virtue of another, through external abstractive reflection; and it
is for another, namely for abstraction and in general for the existent which
still remains opposite to it. In its determination, therefore, it is a dead and
empty absence of determinateness.
As it has come to be here, however, essence is what it is, not through
a negativity foreign to it, but through one which is its own – the infinite
movement of being. It is being-in-and-for-itself – absolute in-itselfness; since
it is indifferent to every determinateness of being, otherness and reference
to other have been sublated. But neither is it only this in-itselfness; as
merely being-in-itself, it would be only the abstraction of pure essence;
but it is being-for-itself just as essentially; it is itself this negativity, the
self-sublation of otherness and of determinateness.
Essence, as the complete turning back of being into itself, is thus at first
the indeterminate essence; the determinacies of being are sublated in it; it
holds them in itself but without their being posited in it. Absolute essence
in this simple unity with itself has no existence. But it must pass over into
existence, for it is being-in-and-for-itself; that is to say, it differentiates the
determinations which it holds in itself, and, since it is the repelling of itself
from itself or indifference towards itself, negative self-reference, it thereby
posits itself over against itself and is infinite being-for-itself only in so far
as in thus differentiating itself from itself it is in unity with itself. – This
determining is thus of another nature than the determining in the sphere of
being, and the determinations of essence have another character than the
determinations of being. Essence is absolute unity of being-in-itself and
being-for-itself; consequently, its determining remains inside this unity;
it is neither a becoming nor a passing over, just as the determinations
themselves are neither an other as other nor references to some other; they
are self-subsisting but, as such, at the same time conjoined in the unity
of essence. – Since essence is at first simple negativity, in order to give
itself existence and then being-for-itself, it must now posit in its sphere the
determinateness which it contains in principle only in itself.
Essence is in the whole what quality was in the sphere of being: absolute
indifference with respect to limit. Quantity is instead this indifference in
immediate determination, limit being in it an immediate external determi-
nateness; quantity passes over into quantum; the external limit is necessary
to it and exists in it. In essence, by contrast, the determinateness does not
exist; it is posited only by the essence itself, not free but only with reference
to the unity of the essence. – The negativity of essence is reflection, and the
determinations are reflected – posited by the essence itself in which they
remain as sublated.

Essence stands between being and concept; it makes up their middle, its
movement constituting the transition of being into the concept. Essence
is being-in-and-for-itself, but it is this in the determination of being-in-
itself; for its general determination is that it emerges from being or that
it is the first negation of being. Its movement consists in positing negation
or determination in being, thereby giving itself existence and becoming as
infinite being-for-itself what it is in itself. It thus gives itself its existence
which is equal to its being-in-itself and becomes concept. For the concept
is the absolute as it is absolutely, or in and for itself, in its existence. But
the existence which essence gives to itself is not yet existence as it is in and
for itself but as essence gives it to itself or as posited, and hence still distinct
from the existence of the concept.
First, essence shines within itself or is reflection; second, it appears; third, it
reveals itself. In the course of its movement, it posits itself in the following
determinations:
I. As simple essence existing in itself, remaining in itself in its determina-
tions;
II. As emerging into existence, or according to its concrete existence and
appearance;
III. As essence which is one with its appearance, as actuality.

\subsubsection{Concept}

The concept, as considered so far, has demonstrated itself to be the unity of
being and essence. Essence is the first negation of being, which has thereby
become reflective shine; the concept is the second negation, or the negation
of this negation, and is therefore being which has been restored once more,
but as in itself the infinite mediation and negation of being. – In the con-
cept, therefore, being and essence no longer have determination as being and
essence, nor are they only in such a unity in which each would reflectively
shine in the other. Consequently, the concept does not differentiate itself
into these determinations. The concept is the truth of the substantial rela-
tion in which being and essence attain their perfect self-subsistence and
determination each through the other. The truth of substantiality proved
to be the substantial identity, an identity that equally is, and only is, posited-
ness. Positedness is determinate existence and differentiation; in the concept,
therefore, being-in-and-for-itself has attained a true existence adequate to
it, for that positedness is itself being-in-and-for-itself. This positedness
constitutes the difference of the concept in the concept itself; and because
the concept is immediately being-in-and-for-itself, its differences are them-
selves the whole concept – universal in their determinateness and identical in
their negation.

This is now the concept itself of the concept, but at first only the concept
of the concept or also itself only concept. Since the concept is being-in-and-
for-itself by being a positedness, or is absolute substance, and substance
manifests the necessity of distinct substances as an identity, this identity must
itself posit what it is. The moments of the movement of the substantial
relation through which the concept came to be and the reality thereby
exhibited are only in the transition to the concept; that reality is not yet the
concept’s own determination, one that has emerged out of it; it fell in the
sphere of necessity whereas the reality of the concept can only be its free
determination, a determinate existence in which the concept is identical
with itself and whose moments are themselves concepts posited through the
concept itself.

At first, therefore, the concept is only implicitly the truth; because it is
only something inner, it is equally only something outer. It is at first simply
an immediate and in this shape its moments have the form of immediate,
fixed determinations. It appears as the determinate concept, as the sphere of
mere understanding. – Because this form of immediacy is an existence still
inadequate to the nature of the concept, for the concept is free and only
refers to itself, it is an external form in which the concept does not exist
in-and-for-itself, but can only count as something posited or subjective. –
The shape of the immediate concept constitutes the standpoint that makes
of the concept a subjective thinking, a reflection external to the subject
matter. This stage constitutes, therefore, subjectivity, or the formal concept.
Its externality is manifested in the fixed being of its determinations that
makes them come up each by itself, isolated and qualitative, and each only
externally referred to the other. But the identity of the concept, which is
precisely their inner or subjective essence, sets them in dialectical movement,
and through this movement their singleness is sublated and with it also
the separation of the concept from the subject matter, and what emerges
as their truth is the totality which is the objective concept.

Second, in its objectivity the concept is the fact itself as it exists in-and-for-itself.
The formal concept makes itself into the fact by virtue of the
necessary determination of its form, and it thereby sheds the relation
of subjectivity and externality that it had to that matter. Or, conversely,
objectivity is the real concept that has emerged from its inwardness and has
passed over into existence. – In this identity with the fact, the concept
thus has an existence which is its own and free. But this existence is still a
freedom which is immediate and not yet negative. Being at one with the
subject matter, the concept is submerged into it; its differences are objective
determinations of existence in which it is itself again the inner. As the soul
of objective existence, the concept must give itself the form of subjectivity
that it immediately had as formal concept; and so, in the form of the free
concept which in objectivity it still lacked, it steps forth over against that
objectivity and, over against it, it makes therein the identity with it, which
as objective concept it has in and for itself, into an identity that is also posited.

In this consummation in which the concept has the form of freedom
even in its objectivity, the adequate concept is the idea. Reason, which is
the sphere of the idea, is the self-unveiled truth in which the concept
attains the realization absolutely adequate to it, and is free inasmuch as in
this real world, in its objectivity, it recognizes its subjectivity, and in this
subjectivity recognizes that objective world.

{Subjectivity}

The concept is, to start with, formal, the concept in its beginning or as
the immediate concept. – In this immediate unity, its difference or its
positedness is, first, itself initially simple and only a reflective shine, so that
the moments of the difference are immediately the totality of the concept
and only the concept as such.

But, second, because it is absolute negativity, the concept divides and
posits itself as the negative or the other of itself; yet, because it is still
immediate concept, this positing or this differentiation is characterized by
the reciprocal indifference of its moments, each of which comes to be on
its own; in this division the unity of the concept is still only an external
connection. Thus, as the connection of its moments posited as self-subsisting
and indifferent, the concept is judgment.

Third, although the judgment contains the unity of the concept that
has been lost in its self-subsisting moments, this unity is not posited. It
will become posited by virtue of the dialectical movement of the judgment
which, through this movement, becomes syllogistic inference, 30 and this is
the fully posited concept, for in the inference the moments of the concept
as self-subsisting extremes and their mediating unity are both equally posited.

But since this unity itself, as unifying middle, and the moments, as self-
subsisting extremes, stand at first immediately opposite one another, this
contradictory relation that occurs in the formal inference sublates itself, and
the completeness of the concept passes over into the unity of totality; the
subjectivity of the concept into its objectivity.

{Objectivity}

First, then, objectivity is in its immediacy. Its moments, on account of
the totality of all moments, stand in self-subsistent indifference as objects
each outside the other, and as so related they possess the subjective unity of
the concept only as inner or as outer. This is mechanism.

But, second, inasmuch as in mechanism that unity reveals itself to be
the immanent law of the objects, their relation becomes one of non-
indifference, each specifically different according to law; a connection in
which the objects’ determinate self-subsistence is sublated. This is chemism.

Third, this essential unity of the objects is thereby posited as distinct
from their self-subsistence. It is the subjective concept, but posited as
referring in and for itself to the objectivity, as purpose. This is teleology.

\subsection{Nature}
\subsection{Spirit}
{document}
